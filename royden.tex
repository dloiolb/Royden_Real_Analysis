\documentclass[a4paper,10pt]{book}
\usepackage[utf8]{inputenc}
\usepackage[T1]{fontenc}
\usepackage[]{mdframed}
\usepackage{lipsum}
\usepackage{xcolor}
\usepackage{amsfonts} 
\usepackage{times}
\usepackage[shortlabels]{enumitem}
\usepackage{amsmath}
\usepackage{centernot}
\usepackage{amsthm,amssymb}
\usepackage{verbatim}
\usepackage{multicol}
\usepackage{titletoc}
\usepackage{graphicx}
\usepackage{minitoc}

%\newtheorem*{theorem*}{}
%\newtheorem{theorem}{Theorem}

\usepackage{amsthm}
%\usepackage{thmtools}

\newtheorem{theorem}{Theorem}[section] % the main one
%\newtheorem{lemma}[theorem]{Lemma}

\theoremstyle{plain} % just in case the style had changed
\newcommand{\thistheoremname}{}

\begin{comment}
\newtheorem{genericthm}[theorem]{\thistheoremname}
\newenvironment{namedthm}[1]
	{\renewcommand{\thistheoremname}{#1}%
	\begin{genericthm}}
	{\end{genericthm}}
\end{comment}

% https://tex.stackexchange.com/questions/12913/customizing-theorem-name

\newtheorem*{genericthm*}{\thistheoremname}
\newenvironment{namedthm*}[1]
	{\renewcommand{\thistheoremname}{#1}%
	\begin{genericthm*}}
	{\end{genericthm*}}


%\newmdtheoremenv{theo}{Theorem}

% https://tex.stackexchange.com/questions/43536/writing-steps-in-an-equation


\usepackage[T1]{fontenc}
\usepackage{geometry}
\usepackage{titlesec}
\usepackage{titletoc}
\usepackage{blindtext}  % drop in actual document

\titleformat{name=\chapter}[display]
{\normalfont\huge\bfseries}
{\chaptertitlename\ \thechapter}
{20pt}
{\Huge}
[\normalsize\normalfont\vspace*{1pc}%
\hbox{\large\bfseries\contentsname}\vspace{6pt}\titlerule\vspace{3pt}
\startcontents
\printcontents{l}{1}{\setcounter{tocdepth}{2}}\vspace{1pt}
\titlerule\vspace{1pc}]

\titleformat{name=\chapter,numberless}[display]
{\normalfont\huge\bfseries}
{}
{20pt}
{\Huge}

\usepackage{url}
\usepackage{hyperref}
\usepackage{geometry}
\geometry{a4paper}
\usepackage[english]{babel}
\title{Real Analysis Royden - Fourth Edition\\
	\large Notes + Solved Exercises :) \\
	\large \href{https://latex-programming.fandom.com/wiki/List_of_LaTeX_symbols}{Latex Symbols}
}
\author{J.B.}
\date{\small May 2024}

\begin{document}

\pagenumbering{roman}
\maketitle
\tableofcontents

\clearpage\pagenumbering{arabic}
%\pagenumbering{arabic}

\setcounter{chapter}{0}
\chapter*{I LEBESGUE INTEGRATION FOR FUNCTIONS OF A SINGLE REAL VARIABLE}
\addcontentsline{toc}{chapter}{I LEBESGUE INTEGRATION FOR FUNCTIONS OF A SINGLE REAL VARIABLE}
\setcounter{chapter}{0}

\setcounter{chapter}{0}
\chapter*{Preliminaries on Sets, Mappings, and Relations}
\addcontentsline{toc}{chapter}{Preliminaries on Sets, Mappings, and Relations}
\setcounter{chapter}{0}

\begin{flushleft}

\begin{namedthm*}{Definition}
A relation $R$ on a set $X$ is called an \textbf{equivalence relation} provided:
\begin{enumerate}[label=(\roman*),align=left]
	\item $xRx$ for all $x \in X$ (reflexive),
	\item $xRy$ implies $yRx$ for all $x,y \in X$ (symmetric),
	\item $xRy$ and $yRz$ imply $xRz$ for all $x,y,z \in X$ (transitive).
\end{enumerate}
\end{namedthm*}
\bigskip

\begin{namedthm*}{Partial Ordering on a set $X$}
A relation $R$ on a nonempty set $X$ is called a \textbf{partial ordering} provided:
\begin{enumerate}[label=(\roman*),align=left]
	\item $xRx$ for all $x \in X$ (reflexive),
	\item $xRy$ and $yRx$ imply $x=y$ for all $x,y \in X$ (antisymmetric),
	\item $xRy$ and $yRz$ imply $xRz$ for all $x,y,z \in X$ (transitive).
\end{enumerate}
A subset $E$ of $X$ is \textbf{totally ordered} provided either $xRy$ or $yRx$ for all $x,y \in E$.
A member $x$ of $X$ is said to be an \textbf{upper bound} for a subset $E$ of $X$ provided that 
\[ yRx \text{ for all } y \in E.\]
A member $x$ of $X$ is said to be \textbf{maximal} provided that 
\[ xRy \text{ implies that } y =x \text{ for } y \in X.\]
\end{namedthm*}
\medskip

\begin{namedthm*}{Strict Partial Ordering on a set $X$}
A relation $R$ on a nonempty set $X$ is called a \textbf{strict partial ordering} provided:
\begin{enumerate}[label=(\roman*),align=left]
	\item not $xRx$ for all $x \in X$ (irreflexive),
	\item $xRy$ implies not $yRx$ for all $x,y \in X$ (asymmetric),
	\item $xRy$ and $yRz$ imply $xRz$ for all $x,y,z \in X$ (transitive).
\end{enumerate}
A subset $E$ of $X$ is \textbf{strictly totally ordered} provided either $xRy$ or $yRx$ if $x\neq y$ for all $x,y \in E$.\par
\end{namedthm*}


\begin{namedthm*}{Zorn's Lemma}
	Let $X$ be a partially ordered set for which every totally ordered subset has an upper bound. Then $X$ has a maximal member.
\end{namedthm*}

\begin{namedthm*}{Every vector space has a basis}
\end{namedthm*}
\begin{proof}
Let V be any vector space, and let L be the collection of all linearly independent subsets of V. 
L is nonempty as the singleton sets are linearly independent. 
Define a partial order on L in the form $C \subseteq C'$ for $C,C' \in L$.
For any chain (a totally ordered subset of a partially ordered set) $\mathcal{C}$ of $L$, where $\mathcal{C}$ consists of the sets $C_1 \subseteq C_2 \subseteq \cdots$, we can construct a linearly independent set $C' = \bigcup_{C \in \mathcal{C}} C$ that is an upper bound of $\mathcal{C}$.
By Zorn's Lemma, L has a maximal element, say M.
This collection $M$ is a basis for $V$. To show this, suppose by contradiction that there exists a vector $v \in V$ s.t. $v \notin \text{ Span}\{M\}$.
Then $v \cup M$ is linearly independent and $M \subseteq v \cup M$, a contradiction to the fact that $M$ is maximal.
\end{proof}

\end{flushleft}

% Chapter 1
\chapter{The Real Numbers: Sets, Sequences, and Functions}

% 1.1
\section{The Field, Positivity, and Completeness Axioms}

\begin{flushleft}

\textbf{The field axioms}\par
Consider $a,b,c \in \mathbb{R}$:
\begin{enumerate}
	\item Closure of Addition: $a+b \in \mathbb{R}$.
	\item Associativity of Addition: $(a+b)+c = a+(b+c)$.
	\item Additive Identity: $0+a=a+0=a$.
	\item Additive Inverse: $(-a)+a=a+(-a)=0$.
	\item Commutativity of Addition: $a+b=b+a$.
	\item Closure of Multiplication: $ab \in \mathbb{R}$.
	\item Associativity of Multiplication: $(ab)c = a(bc)$.
	\item Distributive Property: $ a(b+c)=ab+ac$.
	\item Commutativity of Multiplication: $ab=ba$.
	\item Multiplicative Identity: $1a=a1=a$.
	\item No Zero Divisors: $ab=0 \implies a=0 \text{ or } b=0$.
	\item Multiplicative Inverse: $a^{-1}a=aa^{-1}=1$.
	\item Nontriviality: $1 \neq 0$.
\end{enumerate}
\medskip

\textbf{The positivity axioms}\par
The set of \textbf{positive numbers}, $\mathcal{P}$, has the following two properties:
\begin{itemize}
    \item [P1] If $a$ and $b$ are positive, then $ab$ and $a+b$ are both positive.
    \item [P2] For a real number $a$, exactly one of the three is true: $a$ is positive, $-a$ is positive, $a=0$.	
\end{itemize}

We call a nonempty set $I$ of real numbers an \textbf{interval} provided for any two points in $I$, all the points that lie between these two points also lie in $I$.
That is, $\forall x,y \in I, \lambda x + (1-\lambda)y \in I \text{ for } \lambda \in [0,1]$.
\medskip

\textbf{The completeness axiom}\par
A nonempty set $E$ of real numbers is said to be \textbf{bounded above} provided there is a real number $b$ such that $x \le b$ for all $x\in E$: the number $b$ is called an \textbf{upper bound} for $E$.
We can similarly define a set being \textbf{bounded below} and having a \textbf{lower bound}. A set that is bounded above need not have a largest member.
\begin{namedthm*}{The Completeness Axiom}
Let $E$ be a nonempty set of real numbers that is bounded above. The among the set of upper bounds for $E$ there is a smallest, or least, upper bound.
(This least upper bound is called the \textbf{supremum} of $E$. Also, it can be shown that any nonempty set $E$ that is bounded below has a greatest lower bound, called the \textbf{infimum} of $E$).	
\end{namedthm*}

\medskip
\textbf{The extended real numbers}\par
The extended real numbers: $\mathbb{R} \cup \{-\infty,\infty\}$\par
Every set of real numbers has a supremum and infimum that belongs to the extended real numbers.

\end{flushleft}

\begin{center}
	\textbf{PROBLEMS}
\end{center}
\begin{enumerate}
	\setcounter{enumi}{0}
	\item For $a\neq 0$ and $a\neq 0$, show that $(ab)^{-1} = a^{-1}b^{-1}$.\par

		\begin{align*}
			(ab)(ab)^{-1} & = 1 && \tag*{by multiplicative inverse}\\
			a(b(ab)^{-1}) & = 1 && \tag*{by associativity of multiplication} \\
			a^{-1}a(b(ab)^{-1}) & = a^{-1}1 \\
			1(b(ab)^{-1}) & = a^{-1}1 && \tag*{by multiplicative inverse} \\
			b(ab)^{-1} & = a^{-1} && \tag*{by multiplicative identity} \\
			b^{-1}b(ab)^{-1} & = b^{-1}a^{-1} \\
			1(ab)^{-1} & = b^{-1}a^{-1} && \tag*{by multiplicative inverse} \\
			(ab)^{-1} & = b^{-1}a^{-1} && \tag*{by multiplicative identity} \\
			(ab)^{-1} & = a^{-1}b^{-1} && \tag*{by commutativity of multiplication} \\
		\end{align*}

	\item Verify the following:
	\begin{enumerate}[label=(\roman*),align=left]
        \item For each real number $a\neq 0$, $a^2>0$. In particular, $1>0$ since $1 \neq 0$ and $1=1^2$.\par
        By positivity axiom P2, since $a\neq 0$, either $a$ is positive or $-a$ is positive.\par
		In the case $a$ is positive, $a^2$ is positive by positivity axiom P1.\par
		In the case $-a$ is positive, $(-a)(-a)$ is positive by P1.
		\begin{align*}
			(-a)(-a) & = (-a)(-a) + a(0) && \tag*{by additive identity}\\
			& = (-a)(-a) + a(-a+a) && \tag*{by additive inverse}\\
			& = (-a)(-a) + a(-a) + a(a) && \tag*{by distributive property} \\
			& = (-a + a)(-a) + a^2 && \tag*{by distributive property} \\
			& = 0(-a) + a^2 &&\tag*{by additive inverse} \\
			& = a^2 &&\tag*{by additive identity}
		\end{align*}
		Therefore $a^2$ is positive by equality.
        \item For each positive number $a$, its multiplicative inverse  $a^{-1}$ also is positive.\par
        The multiplication of two positive numbers is positive by positivity axiom P1.\par
		The multiplication of two non-positive numbers is positive: by reformulating the previous result from (i), we can see $0 < (-a)(-b) = ab$ for $a,b \neq 0$. \par
		The multiplication of a positive number and a non-positive number is not positive. 
		To see this, suppose $a$ is positive and $b$ is not positive, but $ab$ is positive. By P1 and P2, $a(-b)$ is also positive.
		By P1, $ab + a(-b)$ is positive. However,
		\[
		ab + a(-b) = a(b-b) = a(0) = 0.
		\]
		This is a contradiction to P2. Therefore $ab$ is not positive.
		\par
		The result from (i) shows that $1$ is positive. By multiplicative inverse, $aa^{-1} = 1 > 0$. Therefore $a^{-1}$ must be positive because $a$ is positive.
        \item If $a>b$, then \[ ac >bc \text{ if } c>0 \text{ and } ac < bc \text{ if } c<0. \]
        Proof that $a(-1)=-a$ for a real number $a$:
		\[a+(-1)a = 1a+(-1)a = (1+-1)a = 0a = 0.\]		
		$a>b$ implies that $a-b$ is positive.\par
		If $c$ is positive, then $(a-b)c = ac-bc$ is positive, and $ac>bc$.\par
		If $c$ is not positive, then $(a-b)c = ac-bc$ is not positive, and $-(ac-bc) = bc-ac$ is positive, so $bc>ac$.\par
    \end{enumerate}
	\item For a nonempty set of real numbers $E$, show that $\inf E = \sup E$ iff $E$ consists of a single point.\par
	$(\implies)$ Suppose $\inf E = \sup E$.\par
	Then $\inf E \le x \le \sup E$ for all $x\in E$. But this implies $x = \inf E = \sup E$ for all $x\in E$, so $E$ consists of the single point $x$.\par
	$(\impliedby)$ Suppose $E={x}$ is a singleton set.\par
	Clearly $x$ is an upper bound and a lower bound for $E$, as $x\le x$. 
	By completeness of the reals, there exists $\sup E$ and $\inf E$ s.t. $x \le \inf E \le x \le \sup E \le x$, as $\inf E$ is the greatest lower bound, and $\sup E$ is the least upper bound.
	Therefore $\inf E = \sup E$.
	\item Let $a$ and $b$ be real numbers.
	\begin{enumerate}[label=(\roman*),align=left]
        \item Show that if $ab = 0$, then $a=0$ or $b=0$.\par
        Contrapositive: Let $a\neq0$ and $b\neq0$. In 2.(ii), it was shown that the multiplication of two nonzero numbers is either positive or not positive. Therefore $ab\neq 0$.
        \item Verify that $a^2 -b^2 = (a-b)(a+b)$ and conclude from part (i) that if $a^2 = b^2$, then $a=b$ or $a=-b$.\par
        \begin{align*}
			(a-b)(a+b) & = (a-b)(a) + (a-b)(b) && \tag*{by distributive property}\\
			& = (a)(a)+(-b)(a) + (a)(b)+(-b)(b) && \tag*{by distributive property}\\
			& = (a)(a)+(-b+b)(a) +(-b)(b) && \tag*{by distributive property}\\
			& = (a)(a) +(-b)(b) && \tag*{by additive inverse}\\
			& = a^2 - b^2 
		\end{align*}
		Suppose $a^2=b^2$. Then $(a-b)(a+b)=a^2-b^2=0$, and by (i), $(a-b)=0 \implies a=b$ or $(a+b)=0 \implies a=-b$.
        \item Let $c$ be a positive real number. Define $E = \{ x \in \mathbb{R} \ |\  x^2 < c\}$. Verify that $E$ is nonempty and bounded above.
		Define $x_0 = \sup E$. Show that $x_0^2 = c$. Use part (ii) to show that there is a unique $x>0$ for which $x^2=c$. It is denoted by $\sqrt{c}$.\par
		We can consider $0\in \mathbb{R}$. $0^2=0<c$, so $0\in E$ and $E$ is nonempty. Also, $c+1$ is a real number and an upper bound for $E$; thus by the completeness axiom, $E$ has a supremum, say $x_0$.
		We can see that for any upper bound $b$ of $E$, $x \le x_0 \le b$ for all $x \in E$. Then $x^2 \le x_0^2 \le b^2$ implies $x_0^2=c$, else $x_0$ is not the supremum. \par
		Suppose there exists $x_1,x_2 > 0$ such that $x_1^2 = c$ and $x_2^2 = c$. This implies $x_1^2 = x_2^2$, and by part (ii), $x_1 = x_2$ or $x_1 = -x_2$. Because $x_1,x_2$ are positive, $x_1 = x_2$.
    \end{enumerate}
	\item Let $a,b,c$ be real numbers s.t. $a\neq 0$ and consider the quadratic equation \[ ax^2+bx+c=0, x \in \mathbb{R}.\]
	\begin{enumerate}[label=(\roman*),align=left]
        \item Suppose $b^2 - 4ac >0$. Use the Field Axioms and the preceding problem to complete the square and thereby show that this equation has exactly two solutions given by
		\[  x = \dfrac{-b + \sqrt{b^2-4ac}}{2a} \ \text{and}\  x = \dfrac{-b - \sqrt{b^2-4ac}}{2a}. \]
		\begin{align*}
			ax^2+bx+c & = 0 \\
			4a(ax^2+bx+c) & = 4a(0) \\
			4a^2x^2+4abx+4ac & = 0 && \tag*{by distributive property}\\
			4a^2x^2+4abx+4ac+b^2-b^2 & = 0 && \tag*{by additive inverse}\\
			4a^2x^2+4abx+b^2 & = b^2-4ac \\
			(2ax+b)^2 & = b^2-4ac 
		\end{align*}
		By 4(iii), because $b^2 - 4ac >0$, there is a unique $y>0$ for which $y^2 = b^2-4ac$. It is denoted by $y = \sqrt{b^2-4ac}$.\par
		By 4(ii), $(2ax+b)^2 = b^2-4ac = y^2$ implies $(2ax+b) = \sqrt{b^2-4ac} = y$ or $(2ax+b) = -\sqrt{b^2-4ac} = -y.$\par
		\begin{align*}
			2ax+b & = \pm \sqrt{b^2-4ac} \\
			2ax & = -b \pm \sqrt{b^2-4ac} \\
			x & = \dfrac{-b \pm \sqrt{b^2-4ac}}{2a}.
		\end{align*}
	\end{enumerate}
	\item Use the Completeness Axiom to show that every nonempty set of real numbers that is bounded below has an infimum and that
	\[\inf E =-\sup \{-x \ |\ x \in E\}.\]
	Let $E$ be a set that is bounded below; that is, there exists $l\in \mathbb{R}$ such that $l \le x$ for all $x\in E$.
	Then $-l \ge -x$ for all $x \in E$, and $-l$ is an upper bound for $-E=\{-x \ | \ x\in E\}$. 
	Therefore the set $-E$ is bounded above, and by the completeness axiom, there exists a least upper bound $c= \sup (-E)$.
	Then for any upper bound $u$ of $-E$, $u \ge c \ge -x$ for all $x \in E.$
	Then $-u$ is a lower bound of $E$, and $-u \le c \le x$ for all $x \in E$, and $c$ is the greatest lower bound and thus infimum of $E$.
	\item For real numbers $a$ and $b$, verify the following:
	\begin{enumerate}[label=(\roman*),align=left]
		\item $|ab| = |a||b|.$\par
		We have 
		\[ 
		|ab| =
		\begin{cases} 
			ab & \text{ if } ab \ge 0, \\
			-(ab) & \text{ if } ab < 0.
		\end{cases}
		\]
		The case where either $a$ or $b$ are zero is trivial.
		In problem 2(ii), it was shown that $ab>0$ if $a,b$ are the same sign, and $ab<0$ if $a,b$ are opposite signs.\par
		Case $a,b>0$: Then $ab>0$ so $|ab| = ab$, and $|a| = a$ and $|b|=b$ so $|a||b| = ab$.\par
		Case $a,b<0$: Then $ab>0$ so $|ab| = ab$, and $|a| = -a$ and $|b|=-b$ so $|a||b| = (-a)(-b)=ab$.\par
		Case $a<0,b>0$: Then $ab<0$ so $|ab| = -(ab) = (-1)ab$, and $|a| = -a = (-1)a$ and $|b|=b$ so $|a||b| = (-1)ab$.
		\item $|a+b| \le |a|+|b|.$\par
		The case where both $a,b=0$ is trivial.\par
		Case $a,b>0$: Then $a+b > 0 $, so $|a+b| = a+b$ and $|a|+|b| = a+ b$.\par
		Case $a>0,b=0$: Then $a+b = a+0=a > 0 $, so $|a+b| = a$ and $|a|+|b| = a+ 0 = a$.\par
		Case $a<0,b=0$: Then $a+b = a +0=a<0 $, so $|a+b| = -a$ and $|a|+|b| = -a +0 = -a$.\par
		Case $a,b<0$: Then $a+b < 0 $, so $|a+b| = -(a+b) = -a-b$ and $|a|+|b| = -a- b$.\par
		That is, equality holds except for the case where $a,b$ are nonzero opposite signs:\par
		Case $a>0,b<0$: $|a+b| \in \{a+b, -(a+b)\}$.\par
		$b<0<-b \implies a+b<a<a-b$, and $-a<0<a \implies -(a+b)=-a-b<-b<a-b$.\par
		$|a|+|b| = a- b$, so $|a+b| < |a|+|b|$.
		\item For $\epsilon >0,$
		\[ |x-a| < \epsilon \text{  iff  } a - \epsilon < x < a + \epsilon.\]
		We have
		\[ 
		|x-a| =
		\begin{cases} 
			x-a & \text{ if } x-a \ge 0, \\
			-(x-a) & \text{ if } x-a < 0.
		\end{cases}
		\]
		$(\implies)$ Suppose $|x-a| < \epsilon$.\par
		Then $x-a < \epsilon$ and $a-x < \epsilon$.\par
		Then $x< a+\epsilon$ and $a-\epsilon<x$.\par
		$(\impliedby)$ Suppose $a - \epsilon < x < a + \epsilon$.\par
		Then
		\begin{align*}
		a - \epsilon-a &< x-a < a + \epsilon-a \\
		- \epsilon &< x-a < \epsilon
		\end{align*}
		So $x-a < \epsilon$ and $- \epsilon < x-a \implies -(x-a)< \epsilon$, so $|x-a| < \epsilon$.
	\end{enumerate}
\end{enumerate}

% 1.2
\section{The Natural and Rational Numbers}
\begin{flushleft}

\begin{namedthm*}{Definition}
	A set $E$ of real numbers is said to be \textbf{inductive} provided it contains 1 and if the number $x$ belongs to $E$, the number $x+1$ also belongs to $E$.
\end{namedthm*}

The set of \textbf{natural numbers}, denoted by $\mathbb{N}$, is defined to be the intersection of all inductive subsets of $\mathbb{R}$. 

\begin{namedthm*}{Theorem 1}
	Every nonempty set of natural numbers has a smallest member.
\end{namedthm*}
\begin{proof}
	Let $E$ be a nonempty set of natural numbers. Since the set $\{x\in \mathbb{R}\ |\ x \ge 1 \}$ is an inductive set, by definition of intersection, $\mathbb{N} \subseteq \{x\in \mathbb{R}\ |\ x \ge 1 \}$, and the natural numbers are bounded below by $1$.
	Therefore $E$ is bounded below by $1$. By the Completeness Axiom, $E$ has an infimum; let $c=\inf E$.
	Since $c+1$ is not a lower bound for $E$, there is an $m \in E$ for which $m < c+1$.
	We claim that $m$ is the smallest member of $E$. Otherwise, there is an $n\in E$ for which $n<m$. Since $n \in E$, $c\le n$. Thus $c \le n < m < c+1$ and therefore $m-n<1$.
	Therefore the natural number $m$ belongs to the interval $(n,n+1)$. However, an induction argument shows that $(n, n+1) \cap \mathbb{N} = \emptyset$ (Problem 8). 
	This is a contradiction to $m \in E$. Therefore $m$ is the smallest member of $E$.
\end{proof}

\begin{namedthm*}{Archimedean Property}
	For each pair of positive real numbers $a$ and $b$, there is a natural number $n$ for which $na>b$. This can be reformulated: for each positive real number $\epsilon$, there is a natural number $n$ for which $\dfrac{1}{n} < \epsilon$.
\end{namedthm*}

The set of \textbf{integers}, denoted $\mathbb{Z}$, is defined to be the set of numbers consisting of the natural numbers, their negatives, and zero.
\par
\medskip
Consider the number $2$. From problem 4(iii), there is a unique $x>0$ for which $x^2=2$. It is denoted by $\sqrt{2}$. This number is not rational.
Suppose that $x$ is rational: then there exist $p,q \in \mathbb{Z}$ such that $(\dfrac{p}{q})^2=2$. Then $p^2=2q^2$. 
By the unique prime factorizations of $p$ and $q$, $p^2$ is divisible by $2^{2k}$ for some $k \in \mathbb{Z}_{\ge 0}$, while $2q^2$ is divisible by $2 \cdot 2^{2j} = 2^{2j+1}$ for some $j \in \mathbb{Z}_{\ge 0}$.
$2^{2k} \neq 2^{2j+1}$ for any combinations of $k,j$ so $p^2=2q^2$ is not possible, and $\sqrt{2}$ is not rational.

\begin{namedthm*}{Definition}
A set $E$ of real numbers is said to be \textbf{dense} in $\mathbb{R}$ provided that between any two real numbers there lies a member of $E$.	
\end{namedthm*}

\begin{namedthm*}{Theorem 2}
The rational numbers are dense in $\mathbb{R}$.	
\end{namedthm*}
\begin{proof}
Let $a,b \in \mathbb{R}$ with $a<b$.\par
Case $a>0$:\par
By the Archimedean Property, for $(b-a)>0$, there exists $q \in \mathbb{N}$ for which $\dfrac{1}{q} < b-a$. \par
Again by the Archimedean Property, for $b,\dfrac{1}{q}>0$, there exists $n \in \mathbb{N}$ for which $n(\dfrac{1}{q})>b$.\par
Therefore the set $S=\{n \in \mathbb{N} \ |\ \dfrac{n}{q} \ge b \}$ is nonempty. Because $S$ is a set of natural numbers, by Theorem 1, $S$ has a smallest member $p$.
Noticing $\dfrac{1}{q} < b-a < b$, we see that $1 \notin S$ and $p>1$. Therefore $p-1$ is a natural number (Problem 9).
Because $p$ is the smallest member of $S$, $p-1 \notin S$ and $\dfrac{(p-1)}{q} < b$.
Also, 
\[
a = b-(b-a) < \dfrac{p}{q} - (\dfrac{1}{q}) = \dfrac{(p-1)}{q}.
\]
Therefore the rational number $\dfrac{(p-1)}{q}$ lies between $a$ and $b$.\par
Case $a<0$:\par
By the Archimedean Property, for $1,-a>0$, there exists $n \in \mathbb{N}$ for which $n(1) > -a$, which implies $n+a>0$, and $b>a$ implies $n+b>n+a>0$.
Then we can use the first case to show that there exists a rational number $r$ such that $n+a<r<n+b$. Therefore the rational number $r-n$ lies between $a$ and $b$.
\end{proof}

\end{flushleft}

\begin{center}
	\textbf{PROBLEMS}
\end{center}
\begin{enumerate}
	\setcounter{enumi}{7}
	\item Use an induction argument to show that for each natural number $n$, the interval $(n, n+1)$ fails to contain any natural number.\par
	For $n \in \mathbb{N}$, let $P(n)$ be the assertion that $(n, n+1) \cap \mathbb{N} = \emptyset$.\par
	$P(1)$: $(1, 2)=\{x\ |\ 1<x<2\}$. Suppose there exists a natural number $q \in (1, 2)$. Then $q>1$ and by problem 9, $q-1$ is a natural number. 
	However, $1<q<2 \implies 0<q-1<1$, which is a contradiction to the fact that the natural numbers are bounded below by $1$ (Theorem 1). Therefore there are no natural numbers in $(1, 2)$.\par
	Suppose $P(k)$ is true for some natural number $k$.\par
	$P(k+1)$: Suppose there exists a natural number $p \in (k+1, (k+1)+1)$; that is, $k+1<p<k+2$.\par
	Case $p=1$: then $k+1<1<k+2$. but $k \in \mathbb{N}$ so $k+1 >1$. Thus $p=1$ is not possible.\par
	Case $p>1$: then by problem 9, $p-1 \in \mathbb{N}$, so $k+1<p<k+2 \implies k<p-1<k+1$. This is a contradiction to $P(k)$, the assumption that there are no natural numbers between $(k,k+1)$.
	Therefore $P(k+1)$ is true.
	\item Use an induction argument to show that if $n>1$ is a natural number, then $n-1$ also is a natural number. The use another induction argument to show that if $m$ and $n$ are natural numbers with $n>m$, then $n-m$ is a natural number.\par
	For $n \in \mathbb{N}$, let $P(n)$ be the assertion that $n=1$ or $n-1 \in \mathbb{N}$.\par
	$P(1)$: $1=1$, true.\par
	Suppose $P(k)$ is true for some $k \in \mathbb{N}$.\par
	$P(k+1)$: $(k+1)-1 = k \in \mathbb{N}$, true.\par
	\medskip
	For $n \in \mathbb{N}$, let $Q(n)$ be the assertion that for all $m \in \mathbb{N}$ such that $n>m$, then $n-m \in \mathbb{N}$.\par
	$Q(1)$: true trivially, because there are no natural numbers less than $1$.\par
	Suppose $Q(k)$ is true for some $k \in \mathbb{N}$; that is, for all $m \in \mathbb{N}$ such that $k>m$, then $k-m \in \mathbb{N}.$\par
	$Q(k+1)$: For all the $m$ from $Q(k)$, we have $(k+1)>k>m$.\par
	We want to show that $(k+1)-m \in \mathbb{N}$.\par
	This is clearly true for $m=1$ because $(k+1)-1 = k \in \mathbb{N}$. 
	Otherwise, $m>1$, so by $P(m)$, $m-1 \in \mathbb{N}$ and $(k+1)-m = k -(m-1)$.
	$Q(k)$ is true tells us that because $(m-1) \in \mathbb{N}$ and $k>m>m-1$, then $k-(m-1) \in \mathbb{N}$.
	Therefore $Q(k+1)$ is true.
	\item Show that for any real number $r$, there is exactly one integer in the interval $[r,r+1)$.\par
	This is trivial if $r \in \mathbb{Z}$.\par
	Consider the smallest integer $p$ less than $[r,r+1)$.
	Then $p<r<r+1$ (and $r<p+1$, because $r=p+1 \implies r \in \mathbb{Z}$ and $r>p+1 \implies p$ is not the smallest integer less than $[r,r+1)$), therefore $r<p+1<r+1$. Because the integers are inductive, $p+1 \in \mathbb{Z}$.\par
	To show that there is not more than one integer between $[r,r+1)$:
	let $q$ be a natural number such that $r \le q < r+1$. Then $q-1 < r \le q$ and $q< r+1 \le q+1$. 
	From problem 8, we see that there are no integers between $(q-1,q)$ and $(q,q+1)$, 
	so there is only one integer in $(q-1,q)\cup q \cup (q,q+1) \supseteq [r,r+1)$.
	\item Show that any nonempty set of integers that is bounded above has a largest member.\par
	Let $E$ be a nonempty set of integers that is bounded above. By the completeness axiom, there exists $c = \sup E$. 
	That is, $x\le c$ for all $x \in E$. Then $c-1 < z \le c$ for some $z \in E$ because $c-1$ is not an upper bound of $E$.
	Suppose $c$ is not in $E$. Then $c-1 < z < c$. 
	This implies that $c-1 < z < w \le c$ for some $w \in E$ because $z$ is not an upper bound of E.
	But then there exists two integers in the interval $(c-1,c]$, which is a contradiction to problem 10.
	Therefore $c$ is an element of $E$, and it is the largest member.
	\item Show that the irrational numbers are dense in $\mathbb{R}$.\par
	Choose any two real numbers $a,b$ and any irrational number $z$. Then $\dfrac{a}{z},\dfrac{b}{z}$ are real numbers. 
	By density of the rationals in $\mathbb{R}$, there exists a rational $r$ such that $\dfrac{a}{z}<r<\dfrac{b}{z}$. This implies $a<rz<b$, where $rz$ is an irrational number.\par
	Proof that $rz$ is irrational:\par
	Let $r = \dfrac{p}{q}$ and suppose that $rz$ is rational; then $rz = \dfrac{m}{n}$.
	\begin{align*}
		\dfrac{p}{q}z &= \dfrac{m}{n}\\
		z &=\dfrac{m}{n} \dfrac{q}{p}\\
		z &= \dfrac{mq}{np}
	\end{align*}
	Then $z$ is rational, a contradiction.
	\item Show that each real number is the supremum of a set of rational numbers and also the supremum of a set of irrational numbers.\par
	Choose any real number $a$. Let $S=\{ r \in \mathbb{Q}\ |\ r\le a\}$.
	Then $a$ is an upper bound for this set. To show that $a$ is the supremum, suppose by contradiction that it is not.
	Then there exists $c \in \mathbb{R}$ such that $r \le c < a$. 
	However, the rational numbers are dense in $\mathbb{R}$, so there exists a rational between $c$ and $a$, a contradiction to the assumption that $c$ is an upper bound to $S$. 
	\par
	The same argument can easily be shown for the irrational numbers.
	\item Show that if $r>0$, then, for each natural number $n$, $(1+r)^n \ge 1+n \cdot r$.\par
	Let $r>0$.\par
	For $n \in \mathbb{N}$, let $P(n)$ be the assertion that $(1+r)^n \ge 1+n \cdot r$.\par
	$P(1)$: $(1+r)^1 = 1+1 \cdot r$, true.\par
	Suppose $P(k)$ is true for some $k \in \mathbb{N}$. Then $(1+r)^k \ge 1+k \cdot r$. \par
	$P(k+1)$:\par
	$(1+r)^{k+1} = (1+r)^k(1+r) \ge (1+kr)(1+r) = 1+ kr + r +kr^2 > 1+ kr + r = 1+(k+1) \cdot r$.
	\item Use induction arguments to prove that for every natural number $n$,
	\begin{enumerate}[label=(\roman*),align=left]
        \item \[ \sum_{j=1}^n j^2 = \dfrac{n(n+1)(2n+1)}{6}, \]
        $P(1)$: $\sum_{j=1}^1 j^2 = 1 = \dfrac{1(1+1)(2+1)}{6}$.\par
		Suppose $P(k)$ is true for $k \in \mathbb{N}$.\par
		$P(k+1)$: 
		\begin{align*}
			\sum_{j=1}^{k+1} j^2 &= \sum_{j=1}^{k} j^2 + (k+1)^2 \\
			&= \dfrac{k(k+1)(2k+1)}{6} + (k+1)^2\\
			& = \dfrac{k(2k^2+k+2k+1)}{6} + \dfrac{6(k^2+2k+1)}{6} \\
			& = \dfrac{(2k^3+k^2+2k^2+k)+(6k^2+12k+6)}{6} \\
			&= \dfrac{2k^3+9k^2+13k+6}{6} \\
			&= \dfrac{(k+1)(2k^2+7k+6)}{6} \\
			&= \dfrac{(k+1)(k+2)(2(k+1)+1)}{6}.
		\end{align*}
        \item \[ 1^3 + 2^3 + \cdots + n^3 = (1+2+\cdots +n)^2, \]
        $P(1)$: $a^3 = 1 = (1)^3$.\par
		Suppose $P(k)$ is true for $k \in \mathbb{N}$.\par
		$P(k+1)$: 
		\begin{align*}
			1^3 + 2^3 + \cdots + (k+1)^3 &= 1^3 + 2^3 + \cdots + k^3 + (k+1)^3 \\
			&= (1+2+\cdots +k)^2 + (k+1)^3 \\
			&= \biggl (\dfrac{k(k+1)}{2} \biggr)^2 + (k+1)^3 \\
			&= \dfrac{k^2(k+1)^2}{4} + \dfrac{4(k+1)^3}{4} \\
			&= \dfrac{k^2(k+1)^2+(4k+4)(k+1)^2}{4} \\
			&= \dfrac{(k^2+4k+4)(k+1)^2}{4} \\
			&= \dfrac{(k+2)^2(k+1)^2}{2^2} \\
			&= \biggl (\dfrac{(k+2)(k+1)}{2} \biggr )^2 \\
			&= \biggl (\dfrac{((k+1)+1)(k+1)}{2} \biggr )^2 \\
			&= \biggl (1+2+ \cdots + (k+1) \biggr )^2.
		\end{align*}
        \item \[ 1+r+\cdots +r^n = \dfrac{1-r^{n+1}}{1-r} \text{ if } r \neq 1.\]
        $P(1)$: $1+r^1 = \dfrac{(1+r)(1-r)}{1-r} = \dfrac{1-r^2}{1-r}$.\par
		Suppose $P(k)$ is true for $k \in \mathbb{N}$.\par
		$P(k+1)$: 
		\begin{align*}
			1+r+\cdots +r^{k+1} &= 1+r+\cdots + r^k +r^{k+1} \\
			&= \dfrac{1-r^{k+1}}{1-r} +r^{k+1} \\
			&= \dfrac{1-r^{k+1}}{1-r} +\dfrac{(1-r)r^{k+1}}{1-r} \\
			&= \dfrac{1-r^{k+1} + r^{k+1} - r^{(k+1)+1} }{1-r} \\
			&= \dfrac{1 - r^{(k+1)+1} }{1-r}.
		\end{align*}
    \end{enumerate}
\end{enumerate}

% 1.3
\section{Countable and Uncountable Sets}

\begin{flushleft}

Two sets $A$ and $B$ are \textbf{equipotent} provided there exists a bijection between them.\par
A set $E$ is \textbf{countable} if it is equipotent to a set of natural numbers.\par
For a countably infinite set $X$, we say that $\{x_n \ |\ n \in \mathbb{N} \}$ is an \textbf{enumeration} of $X$ provided
\[
	X = \{x_n \ |\ n \in \mathbb{N} \} \ \text{ and }\ x_n \neq x_m \ \text{ if }\ n \neq m. 
\]
\par
\medskip
\begin{namedthm*}{Theorem 3}
	A subset of a countable set is countable. In particular, every set of natural numbers is countable.
\end{namedthm*}
\begin{namedthm*}{Corollary 4}
	The following sets are countably infinite:
	\begin{enumerate}[label=(\roman*),align=left]
		\item For each natural numbers $n$, the Cartesian product $\mathbb{N}^n = \mathbb{N} \times \cdots \times \mathbb{N}$.
		\item The set of natural numbers $\mathbb{Q}$.
	\end{enumerate}
\end{namedthm*}
\par
\medskip
The rationals are countable: 
$\mathbb{Q} = \{0,\dfrac{1}{1},-\dfrac{1}{1},\dfrac{1}{2},-\dfrac{1}{2},\dfrac{2}{1},-\dfrac{2}{1}, \dfrac{3}{1}, -\dfrac{3}{1},\dfrac{1}{3},-\dfrac{1}{3},\dfrac{1}{4},-\dfrac{1}{4},\dfrac{2}{3},-\dfrac{2}{3},\cdots \}$.
\par
\medskip
\begin{namedthm*}{Corollary 6}
The union of a countable collection of countable sets is countable.
\end{namedthm*}
An interval of real numbers is called degenerate if it is empty or contains a single member.
\begin{namedthm*}{Theorem 7}
A nondegenerate interval of real numbers is uncountable.	
\end{namedthm*}
\begin{proof}
Let $I$ be a nondegenerate interval of real numbers. Clearly $I$ is not finite. Suppose $I$ is countably infinite.
Let $\{x_n \ |\ n \in \mathbb{N} \}$ be an enumeration of $I$. 
For each $n \in \mathbb{N}$, choose a nondegenerate compact subinterval $[a_n,b_n] \subseteq I$ such that $x_n \notin [a_n,b_n]$. 
Let the set of such intervals $\{[a_n,b_n]\}_{n=1}^\infty$ be descending: $[a_{n+1},b_{n+1}] \subseteq [a_n,b_n]$ (That is, $a_n \le a_{n+1}<b_{n+1}\le b_n$.)
Now, the nonempty set $E = \{a_n \ |\ n \in \mathbb{N} \}$ is bounded above by $b_1$.
Then the Completeness Axiom implies that $E$ has a supremum, say $x^* = \sup E$. 
Then for each $n$, $a_n \le x^* \le b_n$ because $x^*$ is the supremum of $E$ and each $b_n$ is an upper bound for $E$.
Therefore $x^*$ belongs to $[a_n,b_n]$ for each $n$.
But then $x^*$ is an element of $I$ and thus has an index $n_0 \in \mathbb{N}$ such that $x^* = x_{n_0}$. But $x^* \in [a_{n_0},b_{n_0}]$, a contradiction.
Therefore $I$ is not countable.
\end{proof}

\end{flushleft}

\begin{center}
	\textbf{PROBLEMS}
\end{center}
\begin{enumerate}
	\setcounter{enumi}{15}
	\item Show that the set $\mathbb{Z}$ of integers is countable.\par
	There exists a bijection $\phi: \mathbb{Z} \to \mathbb{N}$ with
	\[ 
		\phi(x) =
		\begin{cases} 
			2x & \text{ if } x > 0, \\
			-2x+1 & \text{ if } x \le 0.
		\end{cases}
	\]
	\begin{align*}
		\mathbb{Z} &= \{0,1,-1,2,-2,3,-3,4,-4, \cdots\} \\
		\mathbb{N} &= \{1,2,3,4,5,6,7,8,9, \cdots\}
	\end{align*}
	\item Show that a set $A$ is countable iff there is an injective mapping of $A$ to $\mathbb{N}$.\par
	$(\implies)$ Suppose $A$ is countable.\par
	Then either $A$ is equipotent to $\mathbb{N}$, or there is an $n \in \mathbb{N}$ such that $A$ is equipotent to $\{1,2, \cdots, n \}$.
	In the case $A$ is countably infinite, we have a bijection with $\mathbb{N}$ and thus an injection. In the case $A$ is finite, we have an injection with a subset of $\mathbb{N}$, and thus an injection with $\mathbb{N}$
	(injection: $f(a)=f(b) \implies a=b$ for $a,b \in A$).
	\par
	$(\impliedby)$ Suppose there is an injective mapping of $A$ to $\mathbb{N}$.\par
	Then there is a bijection from $A$ to some subset $B$ of $\mathbb{N}$.
	By Theorem 3, every subset of natural numbers is countable, and because $A$ is equipotent to this countable set $B$, then $A$ is countable.
	\item Use an induction argument to complete the proof of part (i) of Corollary 4.\par
	(Not an induction argument)\par
	Consider the function $f:\mathbb{N}^2 \to \mathbb{N}$, where $f(m,n) = 2^m3^n$. 
	By the Fundamental Theorem of Arithmetic, $2^m3^n = 2^{m'}3^{n'} \implies m=m',n=n'$.
	Then clearly $f$ is an injection. By problem 17, we see that $\mathbb{N}^2$ is countable.
	\par
	For any $k\in \mathbb{N}$ we can construct a function $f:\mathbb{N}^k \to \mathbb{N}$, where we have $n$ primes such that $f(m_1,m_2, \cdots, m_k) = p_1^{m_1}p_2^{m_2} \cdots p_k^{m_k}$.
	By the fundamental theorem of arithmetic, this is an injection and thus $\mathbb{N}^k$ is countable.
	\item Prove Corollary 6 in the case of a finite family of countable sets.\par
	Let $\{S_n\}_{n=1}^k$ be a finite family of countable sets.
	Then each set $S_n$ is countable, and we can enumerate as follows: $S_n = \{s_{nm} \ | \ m \in \mathbb{N} \}$.
	Then because there is only a finite number of countable sets, we can construct a function $f: \bigcup_{n=1}^k S_n \to \mathbb{N}$ seeing that 
	\[
	\bigcup_{n=1}^k S_n = \{s_{11},s_{21},s_{31},\cdots, s_{k1}, s_{12}, s_{22},s_{32},\cdots,s_{k2},s_{13}, \cdots \}.
	\]
	\item Let both $f:A \to B$ and $g:B \to C$ be injective and surjective. Show that the composition $g \circ f:A \to B$ and the inverse $f^{-1}:B \to A$ are also injective and surjective.\par
	$g \circ f$:\par
	By surjectivity of $g$, for all $c \in C$, there exists a $b \in B$ such that $g(b)=c$.
	Then by surjectivity of $f$, there exists an $a \in A$ such that $f(a)=b$.\par
	Therefore for any $c \in C$:
	\begin{align*}
		c & = g(b) && \text{ for some $b \in B$}\\
		& = g(f(a))&& \text{ for some $a \in A$}\\
		& =g \circ f (a)
	\end{align*}
	Therefore $g \circ f$ is surjective.\par
	By injectivity of $g$, $g(b)=g(b') \implies b = b'$.\par
	By injectivity of $f$, $f(a)=f(a') \implies a = a'$.
	\begin{align*}
		g \circ f (a) & = g \circ f (a')\\
		g(f(a)) & = g(f(a')) \\
		f(a) & = f(a')&& \text{ by injectivity of $g$}\\
		a & = a'&& \text{ by injectivity of $f$}
	\end{align*}
	Therefore $g \circ f$ is injective.
	\par
	$f^{-1}$:\par
	Because $f$ is a function from $A$ to $B$, $f(a) \subseteq B$ is defined for all $a \in A$.
	That is, for all $a \in A$, there exists a $b \in B$ such that $f^{-1}(b) = a$.
	Thus $f^{-1}$ is surjective.\par
	Because $f$ is a function, for each $a \in A$, $f(a)=b$ and $f(a)=b'$ imply $b=b'$. That is, $f^{-1}(b)=f^{-1}(b') \implies b=b'$.
	Thus $f^{-1}$ is injective.   
	\item Use an induction argument to establish the pigeonhole principle.\par
	For $n \in \mathbb{N}$, let $P(n)$ be the assertion that for any $m \in \mathbb{N}$, the set $\{1,2, \cdots, n\}$ is not equipotent to the set $\{1,2, \cdots, n+m\}$.\par
	$P(1)$: We have the sets $A=\{1\}$ and $B=\{1,2, \cdots, 1+m\}$, for $m \in \mathbb{N}$.
	In the case $m=1$, $B=\{1,1+1\}=\{1,2\}$, and clearly $A$ is not equipotent to $B$. Clearly $A$ is also not equipotent to $B$ for any other natural number $m>1$.\par
	Suppose $P(k)$ is true for some $k \in \mathbb{N}$. Then $\{1,2, \cdots, k\}$ is not equipotent to the set $\{1,2, \cdots, k+m\}$, for any $m \in \mathbb{N}$.\par
	$P(k+1)$: Then clearly $\{1,2, \cdots, k+1\}$ is not equipotent to the set $\{1,2, \cdots, (k+1), \cdots, (k+1)+m\}$, for any $m \in \mathbb{N}$.
	\item Show that $2^{\mathbb{N}}$, the collection of all sets of natural numbers, is uncountable.\par
	(Cantor's Theorem: for a set $A$, any function $f:A\to \mathcal{P}(A)$ is not surjective.)\par
	Let $f:\mathbb{N}\to \mathcal{P}(\mathbb{N})$ be any map. Let $E = \{n \in \mathbb{N}\ | \ n \notin f(n) \}$. 
	Then $E$ is a subset of the naturals that is not in the image of $f$, so $f$ is not surjective. 
	Therefore there is no bijection between $\mathbb{N}$ and  $\mathcal{P}(\mathbb{N})$.
	\item Show that the Cartesian product of a finite collection of countable sets is countable. Use the preceding theorem to show that $\mathbb{N}^{\mathbb{N}}$, the collection of all mappings of $\mathbb{N}$ into $\mathbb{N}$, is not countable.\par
	In problem 18, we showed that for any $k \in \mathbb{N}$, the set $\mathbb{N}^k = \mathbb{N} \times \mathbb{N} \times \cdots \times \mathbb{N}$ is countable. 
	It is then trivial to see that the Cartesian product of any finite collection of countable sets $S_1 \times S_2 \times \cdots \times S_k$ is countable.\par
	Notation:
	\[
		0=\emptyset, 1= \{0\}, 2=\{0,1\}, 3 = \{0,1,2\}, \cdots
	\]
	We can let $2^{\mathbb{N}}= \{0,1\}^{\mathbb{N}}$ be the set of functions $f:\mathbb{N} \to \{0,1\}$.\par
	Then, for any subset $A \subseteq \mathbb{N}$, there exists a function $f \in \{0,1\}^{\mathbb{N}}$ such that 
	\[
		f(x) =
	\begin{cases}
		1 & \text{if } x \in A,\\
		0 & \text{if } x \notin A,
	\end{cases}	
	\]
	and we have a bijection between the elements of $\{0,1\}^{\mathbb{N}}$ and the subsets of $\mathbb{N}$ ("Two sets that are equipotent are, from a set-theoretic point of view, indistinguishable").
	Therefore $2^{\mathbb{N}}= \{0,1\}^{\mathbb{N}}$ can be used to represent the collection of subsets of $\mathbb{N}$.\par
	Now, because the set of functions $f:\mathbb{N} \to \{0,1\}$ is uncountable, then clearly the set of functions $f:\mathbb{N} \to \mathbb{N} \supseteq \{0,1\}$ is uncountable (including zero in the naturals for notation convenience).
	\item Show that a nondegenerate interval of real numbers fails to be finite.\par
	Theorem 7 tells us that a nondegenerate interval of real numbers is uncountable, and thus, finite.	
	\item Show that any two nondegenerate intervals of real numbers are equipotent.\par
	We can prove this by showing that any interval is equipotent to the interval $(0,1)$.\par
	For any bounded interval $(a,b),(a,b],[a,b),[a,b]$, there exists a bijection to $(0,1),(0,1],[0,1),[0,1]$ respectively,
	of the form $f(x) = \dfrac{1}{b-a}(x-a)$.\par
	\item Is the set $\mathbb{R} \times \mathbb{R}$ equipotent to $\mathbb{R}$?\par
	yes (Schr\"oder-Bernstein theorem	)
\end{enumerate}

% 1.4
\section{Open Sets, Closed Sets, and Borel Sets of Real Numbers}

\begin{namedthm*}{Proposition 9}
	Every nonempty open set is the union of a countable, disjoint collection of open intervals.	
\end{namedthm*}

\begin{namedthm*}{The Heine-Borel Theorem}
Let $F$ be a closed and bounded set of real numbers. Then every open cover of $F$ has a finite subcover. 	
\end{namedthm*}
\begin{proof}
	Let $F$ be the closed, bounded interval $[a,b]$. Let $\mathcal{F}$ be an open cover of $[a,b]$. 
	Define $E$ to be the set of numbers $x \in [a.b]$ with the property that the interval $[a,x]$ can be covered by a finite number of the sets of $\mathcal{F}$.
	Since $a\in [a,b] \subseteq \mathcal{F}$ implies that $a$ is in one of the sets $\mathcal{O}' \subseteq \mathcal{F}$ by definition of union, $\mathcal{O}'$ is a finite subcover of $[a,a]=\{a\}$, and thus $a \in E$ and $E$ is nonempty.
	Since $E \subseteq [a,b] = \{x\ |\ a \le x \le b\}$, $E$ is bounded above by $b$, so by the completeness of $\mathbb{R}$, $E$ has a supremum $c = \sup E$.
	Because $c \le b$, clearly $c$ belongs to $[a,b]$, and this implies that there is an $\mathcal{O} \subseteq \mathcal{F}$ that contains $c$.
	Since $\mathcal{O}$ is open, there is an $\epsilon >0$ such that that the interval $(c- \epsilon, c+ \epsilon) \subseteq \mathcal{O}$.
	Now $c-\epsilon$ is not an upper bound for $E$, and so there must be an $x \in E$ with $c-\epsilon < x$. Because $x \in E$, there exists a finite collection $\{ \mathcal{O}_1, \cdots, \mathcal{O}_k \}$ of sets in $\mathcal{F}$ that covers $[a,x]$.
	Then clearly the finite collection $\{ \mathcal{O}_1, \cdots, \mathcal{O}_k, \mathcal{O} \}$ covers the interval $[a,c+ \epsilon)$.
	Therefore $c=b$, otherwise there exists a number $c +\tfrac{1}{2}\epsilon$ that has a finite subcover and $c < c +\tfrac{1}{2}\epsilon$ implies that $c$ is not an upper bound for $E$.
	Thus $[a,b] \in E$ and $[a,b]$ can be covered by a finite number of sets of $\mathcal{F}$.
\end{proof}

\begin{namedthm*}{The Heine-Borel Theorem $(\impliedby)$}
	Let $F$ be a real set such that every open cover of $F$ has a finite subcover. Then $F$ is closed and bounded.
\end{namedthm*}
\begin{proof}
	Let $K$ be a compact subset of a metric space $X$. Proving that $X \setminus K$ is open will show that $K$ is closed.
	Consider any $p \in X \setminus K$. For a $k \in K$, let $O_k$ and $I_k$ be neighborhoods of $p$ and $k$ respectively, with radius less than $\tfrac{1}{2} d(p,q)$.
	Because $K$ is compact, there are finitely many points $k_1, \cdots, k_n$ in $K$ such that $K \subseteq I_{k_1} \cup \cdots \cup I_{k_n}$.
	Let $O = O_{k_1} \cap \cdots \cap O_{k_n}$ so that $O$ is an open neighborhood of $p$ that does not intersect $K$.
	Then $O \subseteq X \setminus K$ and $X\setminus K$ is open. Therefore $K$ is closed. 
\end{proof}

\begin{namedthm*}{The Nested Set Theorem}
Let $\{F_n\}_{n=1}^\infty$ be a descending countable collection of nonempty closed sets of real numbers for which $F_1$ is bounded.
Then
\[
    \bigcap_{n=1}^\infty F_n \neq \emptyset.
\]
\end{namedthm*}
\begin{proof} 
By contradiction, suppose that $\bigcap_{n=1}^\infty F_n = \emptyset$. 
Then $\bigcup_{n=1}^\infty F_n^c = (\bigcap_{n=1}^\infty F_n)^c  = \emptyset^c = \mathbb{R}$, and we have an open cover of $\mathbb{R}$ and thus an open cover of $F_1 \subseteq \mathbb{R}$. 
By the Heine-Borel Theorem, there exists an $N \in \mathbb {N}$ such that $F_1 \subseteq \bigcup_{n=1}^N F_n^c$.  
Because $\{F_n\}$ is descending, $F_n \supseteq F_{n+1}$ for any $n \ge 1$. 
This implies $F_{n}^c \subseteq F_{n+1}^c$, and thus $F_1 \subseteq \bigcup_{n=1}^N F_n^c = F_N^c = \mathbb{R}\setminus F_N$.
This is a contradiction to the assumption that $F_N$ is a nonempty subset of $F_1$.
\end{proof}

\begin{center}
	\textbf{PROBLEMS}
\end{center}
\begin{enumerate}
	\setcounter{enumi}{26}
	\item Is the set of rational numbers open or closed?\par
	The set of rationals is neither open nor closed.
	The rationals is not open because the irrationals are dense in the rationals; that is, between any two rationals there is an irrational.
	The rationals is not closed because it does not contain all its limit points; a sequence of rationals can be constructed that converges to an irrational.
	(Thus we see that the irrationals is neither open nor closed as well.)
	\item What are the sets of real numbers that are both open and closed?\par
	It is clear that $\mathbb{R}$ is open, and $\emptyset$ is open (vacuously).
	Then because the complement of an open set is closed, $\mathbb{R}$ and $\emptyset$ are both closed as well.
	\item Find two sets $A$ and $B$ such that $A \cap B = \emptyset$ and $\overline A \cap \overline B \neq \emptyset.$\par
	Let $A= (4,5)$ and $B = (5,20)$. Then $(4,5) \cap (5,20) = \emptyset$ and $\overline A= [4,5]$ and $\overline B = [5,20]$ so $[4,5] \cap [5,20]= \{5\} \neq \emptyset$.\par
	Let $A= \mathbb{Q}$ and $B = \mathbb{Q}^c$. Then $\mathbb{Q} \cap \mathbb{Q}^c = \emptyset$ and $\overline A= \mathbb{R}$ and $\overline B = \mathbb{R}$ so $\mathbb{R} \cap \mathbb{R}= \mathbb{R} \neq \emptyset$.\par
	\item A point $x$ is called an \textbf{accumulation point} of a set $E$ provided it is a point of closure of $E \setminus \{ x\}.$
	\begin{enumerate}[label=(\roman*),align=left]
        \item Show that the set $E'$ of accumulation points of $E$ is a closed set.\par
        Then for $x \in E'$, every open interval that contains $x$ also contains a point in $E \setminus \{x\}$.\par
		Suppose $E'$ is not closed. 
		Then there exists an element $y \notin E'$ such that every open interval that contains $y$ also contains a point $x \in E'$.
		Then every open interval that contains $x$ contains a point $z \in E \setminus \{x\}$... 
		It can be shown that $y \in E'$ and so $E'$ contains all its points of closure and is thus closed.
        \item Show that $\overline E = E \cup E'.$\par
        $E$ includes all the isolated points not included in $E'$. 
    \end{enumerate}
	\item A point $x$ is called an \textbf{ isolated point} of a set $E$ provided there is an $r>0$ for which $(x-r,x+r)\cap E = \{x\}.$ Show that if a set $E$ consists of isolated points, then it is countable.\par
	Each singleton set $\{x\}$ can be enumerated.
	\item A point $x$ is called an \textbf{interior point} of a set $E$ if there is an $r>0$ such that the open interval $(x-r,x+r)$ is contained in $E$. The set of interior points of $E$ is called the \textbf{interior} of $E$ denoted by int $E$. Show that
	\begin{enumerate}[label=(\roman*),align=left]
        \item $E$ is open iff $E = \text{ int } E$.\par
        $(\implies)$ Suppose $E$ is open.\par
		Then clearly every point of $E$ is an interior point.
		\par
		$(\impliedby)$ Suppose $E = \text{ int } E$.\par
		Then every point has an open neighborhood contained in $E$, so $E$ is open.
        \item $E$ is dense iff $ \text{ int } (\mathbb{R} \setminus E)= \emptyset$.
    \end{enumerate}
	\item Show that the nested set theorem is false if $F_1$ is unbounded.\par
	The nested set theorem works because the compactness of $F_1$ allows us to reach a contradiction to the fact that the intersection is empty (see the proof above).\par
	Consider
	\[
	\bigcap_{n=1}^\infty [n, \infty) = \emptyset.
	\]
	This intersection is empty because for any $x$, there exists an $n \in \mathbb{N}$ such that $x < n$ and thus $x \notin [n,\infty)$.
	\item Show that the assertion of the Heine-Borel Theorem is equivalent to the Completeness Axiom for the real numbers. Show that the assertion of the Nested Set Theorem is equivalent to the Completeness Axiom for the real numbers.\par
	The Heine-Borel Theorem States that Closed and bounded sets are compact; that is, every open cover of a closed and bounded set has a finite subcover.
	If a set $E$ is bounded, then for any open cover $E \subseteq \mathcal{F}$ there exists a finite open subcover $\mathcal{O} \subseteq \mathcal{F}$. 
	We can consider the intersection of all such $\mathcal{O}$ so that $E \subseteq \bigcap_{\mathcal{O} \subseteq \mathcal{F}} \mathcal{O} \subseteq \mathcal{O}$, and this intersection is the supremum. 
	\par
	Clearly the descending sets from the nested set theorem are closed and bounded, so the Heine-Borel Theorem discussed above can be used to imply the Completeness Axiom.
	\item Show that the collection of Borel sets is the smallest $\sigma$-algebra that contains the closed sets.\par
	The Borel sets is defined to be the smallest $\sigma$-algebra that contains all the open sets of real numbers.
	Any sigma-algebra that contains the closed sets contains the open sets by the complement property of a sigma-algebra, so the Borel sets is the smallest sigma-algebra that contains the closed sets as well. 
	\item Show that the collection of Borel sets is the smallest $\sigma$-algebra that contains the intervals of the form $[a,b)$, where $a<b.$\par
	Any interval $[a,b)$ can be written in the form
	\[
	[a,b) = \bigcup_{n=1}^\infty [a,b-\tfrac{1}{n}]	
	\] 
	\item Show that each open set is an $F_{\sigma}$ set.\par
	Any open set $(a,b)$ can be written in the form
	\[
		(a,b) = \bigcup_{n=1}^\infty [a+\tfrac{1}{n},b-\tfrac{1}{n}].	
	\] 
\end{enumerate}

% 1.5
\section{Sequences of Real Numbers}

\begin{namedthm*}{Proposition 14}
Let the sequence of real numbers $\{a_n\}$ converge to the real number $a$. 
Then the limit is unique, the sequence is bounded, and, for a real number $c$, 
\[
\text{if } a_n \le c \text{ for all } n, \text{ then } a\le c.	
\]	
\end{namedthm*}
\begin{proof}
	Suppose there exist $a$ and $b$ such that $\{a_n\}\to a$ and $\{a_n\}\to b$.
	Then For any $\epsilon >0$, there exists the index $N = \max \{N_a,N_b\}$ such that for all $n \ge N \ge N_a,N_b$, then $|a-a_n| < \epsilon$ and $|b-a_n| < \epsilon$.
	By the triangle inequality, $|a-b| \le |a-a_n| + |a_n-b| < \epsilon + \epsilon = 2 \epsilon = \epsilon ' $.
	Therefore $a=b$, and the limit is unique. \par
	Consider $\epsilon =1$. Then there exists an index $N \in \mathbb{N}$ such that for all $n \ge N$, $|a_n-a| < 1$.
	Also, $|a_n|-|a| \le |a_n-a| <1\implies |a_n| < |a| +1$.
	Let $M = \max \{|a_1|, |a_2|, \cdots, |a_N|, |a|+1 \}$. The maximum exists because this is a finite set of real numbers (totally ordered).
	Considering any $n \in \mathbb{N}$, if $n \ge N$, then $|a_n-a| <1\implies |a_n| < |a| +1 \le M$, and if $n<N$, then $|a_n| \le \max \{|a_1|, |a_2|, \cdots, |a_N|, |a|+1 \} =M$, so $M$ is a bound for this sequence.
	\par
	Suppose that for all $n$, $a_n \le c$ but $a>c$. 
	Then $a_n \le c < a$ for all $n$, and $0 \le c-a_n <a-a_n$.
	Choosing $\epsilon = c-a_n$, there exists an index such that $|a-a_n| < c-a_n$. But this is a contradiction. 
	Therefore $a \le c$.
\end{proof}

\begin{flushleft}

\begin{namedthm*}{Theorem 15}[the Monotone Convergence Criterion for Real Sequences]
	A monotone sequence of real numbers converges iff it is bounded.
\end{namedthm*}
\begin{proof}
	$(\implies)$ Suppose a monotone sequence converges.\par
	By the above proposition, it is bounded.\par
	$(\impliedby)$ Suppose a monotone sequence $\{a_n\}$ is bounded.\par
	By the Completeness Axiom, there exists a supremum say $a$ such that $a_n \le a$ for all $n$.
	Consider any $\epsilon >0$. Now, $a-\epsilon$ is not an upper bound, and because the sequence is increasing, there exists an index $N$ for which $a_n \ge a_N > a-\epsilon$ for all $n \ge N$.
	Then $\epsilon > a-a_n$ and the sequence converges to $a$. The proof is the same for a decreasing sequence. 
\end{proof}

\begin{namedthm*}{Theorem 16}[The Bolzano-Weierstrass Theorem]
Every bounded sequence of real numbers has a convergent subsequence.	
\end{namedthm*}
\begin{proof}
	Let $a_n$ be a bounded sequence of real numbers. Choose $M>0$ s.t. $|a_n| \le M$ for all $n$. 
	Define $E_n = \overline{\{a_j \ |\ j \ge n\}}$. Then we also have $E_n \subseteq [-M,M]$ and $E_n$ is closed since it is the closure of a set.
	Therefore $\{E_n\}$ is a descending sequence of nonempty closed bounded subsets of real numbers. 
	The Nested Set Theorem tells us that $\bigcap_{n=1}^\infty E_n \neq \emptyset$, so there exists $a \in \bigcap_{n=1}^\infty E_n$.
	For each natural number $k$, $a$ is a point of closure of $\{a_j \ |\ j \ge k\}$.
	Thus for infinitely many indices $j \ge n$, $a_j$ belongs to $(a-\tfrac{1}{k},a+\tfrac{1}{k})$.
	By induction, choose a strictly increasing subsequence of natural numbers $n_k$ such that $|a-a_{n_k}|< \tfrac{1}{k}$ for all $k$.
	From the Archimedean Property of the reals, the subsequence $\{ a_{n_k} \}$ converges to $a$.
\end{proof}

\begin{namedthm*}{Proposition 19}
Let $\{a_n \}$ and $\{b_n \}$ be sequences of real numbers.
\begin{enumerate}[label=(\roman*),align=left]
	\item $\lim \sup \{a_n \}=\ell \in \mathbb{R}$ iff for each $\epsilon >0$, there are infinitely many indices $n$ for which $a_n > l-\epsilon $ and only finitely many indices $n$ for which $a_n < l-\epsilon $.\par
	$(\implies)$ Suppose $\lim \sup \{a_n \}=\ell \in \mathbb{R}$.\par
	Then by problem 38, $\ell$ is a cluster point of the sequence. This means that for all $\epsilon > 0$, there exists a subsequence $\{a_{n_k} \}$ such that $ \ell - a_{n_k} < \epsilon$ for all $n_k$ greater than some index, and thus $ \ell - \epsilon < a_{n_k} $ for infinitely many indices $n_k$.\par
	Suppose by contradiction that for $\epsilon >0$, there are infinitely many indices $n$ for which $a_n < l-\epsilon $.
	That is, no matter how large the epsilon we choose, there exists a subsequence $\{a_{n_k} \}$ such that $\epsilon < l-a_{n_k} $ for all $n_k$ after a certain index.
	This implies that $\{a_n\}$ is not bounded, so by Proposition 14, the sequence does not converge to a real number.
	This is a contradiction to $\ell \in \mathbb{R}$.
	\par
	$(\impliedby)$ Suppose for $\epsilon >0$, there are infinitely many indices $n$ for which $a_n > l-\epsilon $ and only finitely many indices $n$ for which $a_n < l-\epsilon $.\par
	Then choosing specific indices $n_k$, there exists a subsequence $\{a_{n_k} \}$ such that $\ell - a_{n_k} <\epsilon $ for all $n_k$, and this implies the subsequence converges to $\ell$.
	If we suppose that $\ell \neq \lim \sup \{a_n \}$, then there exists some $\delta >0$ such that $\ell > \ell - \delta = \lim \sup \{a_n \}$.\par
	Now, $\ell - \delta = \lim \sup \{a_n \} = \lim_{n \to \infty} \sup \{ a_k\ |\ k \ge n\}$.
	That means for any $n$, $a_k \le \ell - \delta$ for $k \ge n$.
	However, this is a contradiction to the fact that there are only finitely many such indices $k$ for which this is true.
	Therefore $\ell = \lim \sup \{a_n \}$.
	\item $\lim \sup \{a_n \}=\infty$ iff $\{a_n \}$ is not bounded above.\par
	$(\implies)$ Suppose $\lim \sup \{a_n \}=\infty$.\par
	This implies that $\infty = \lim \sup \{a_n \}$ is a cluster point and there exists a subsequence that converges to infinity.
	Therefore $\{a_n \}$ is not bounded above.\par
	$(\impliedby)$ Suppose $\{a_n \}$ is not bounded above.\par
	By Proposition 4, $\{a_n \}$ does not converge to a real number.
	Also,$\{a_n \}$ is not bounded above implies that for any real number $c$, there exists an index such that $a_n > c$.
	Then the only upper bound of this sequence is $\infty$ and thus $\lim \sup \{a_n \}=\infty$.
	\item 
	\[
		\lim \sup \{a_n \}= -\lim \inf \{-a_n \}. 	
	\]
	Definitions of limsup and liminf:\par
	$\lim \sup \{a_n \} = \lim_{n \to \infty} [\sup \{ a_k\ |\ k \ge n\}]$
	$\implies$ for any $n \in \mathbb{N}$, $\sup \{ a_k\ |\ k \ge n\} \ge a_k$ for $k \ge n$.\par
	$\lim \inf \{a_n \} = \lim_{n \to \infty} [\inf \{ a_k\ |\ k \ge n\}]$.
	$\implies$ for any $n \in \mathbb{N}$, $\inf \{ a_k\ |\ k \ge n\} \le a_k$ for $k \ge n$.\par
	Now we have\par
	$\lim \inf \{-a_n \} = \lim_{n \to \infty} [\inf \{ -a_k\ |\ k \ge n\}]$.\par
	$\implies$ for any $n \in \mathbb{N}$, $\inf \{ -a_k\ |\ k \ge n\} \le -a_k$ for $k \ge n$.\par
	$\implies$ for any $n \in \mathbb{N}$, $-\inf \{ -a_k\ |\ k \ge n\} \ge a_k$ for $k \ge n$, the definition of limsup.\par
	\item A sequence of real numbers $\{ a_n\}$ converges to an extended real number $a$ iff 
	\[
		\lim \inf \{a_n \}= \lim \sup \{a_n \} = a.
	\]
	$(\implies)$ Suppose a sequence of real numbers $\{ a_n\}$ converges to an extended real number $a$.\par
	Clearly $\lim \inf \{a_n \} \le a \le \lim \sup \{a_n \} $.\par
	If $\lim \inf \{a_n \} < a < \sup \{a_n \} $, then we reach a contradiction to the infimum and supremum respectively.\par
	Therefore $\lim \inf \{a_n \} = a = \lim \sup \{a_n \} $.
	\par
	$(\impliedby)$ Suppose $\lim \inf \{a_n \}= \lim \sup \{a_n \} = a$.\par
	Then for any $n \in \mathbb{N}$, $\inf \{ a_k\ |\ k \ge n\} \le a_k \le \sup \{ a_k\ |\ k \ge n\}$ for $k \ge n$, which implies
	\[a= \lim \inf \{a_n \} = \lim_{n \to \infty} \inf \{ a_k\ |\ k \ge n\} \le \lim_{n \to \infty} a_k \le \lim_{n \to \infty} \sup \{ a_k\ |\ k \ge n\}= \lim \sup \{a_n \} = a\]
	Clearly $\{ a_n\}$ converges to $a$.
	\item If $a_n \le b_n$ for all $n$, then
	\[
		\lim \sup \{a_n \} \le \lim \sup \{b_n \}.	
	\]
	For any $n \in \mathbb{N}$, $a_k \le \sup \{ a_k\ |\ k \ge n\}$ and $b_k \le \sup \{ b_k\ |\ k \ge n\} $ for all $k \ge n$.\par
	If we suppose $\lim \sup \{a_n \} > \lim \sup \{b_n \}$, then there exists a natural number $n$ such that $\sup \{ a_k\ |\ k \ge n\} > \sup \{ b_k\ |\ k \ge n\} \ge b_k \ge a_k$ for all $k \ge n$.
	However, by problem 38, we see that $\lim \sup \{a_n \}$ is a cluster point of $\{a_n \}$, and we reach a contradiction. (or contradiction to def of supremum?)
\end{enumerate}	
\end{namedthm*}

\begin{namedthm*}{Proposition 20}
	Let $\{a_n \}$ be a sequence of real numbers.
	\begin{enumerate}[label=(\roman*),align=left]
		\item The series $\textstyle \sum_{k=1}^\infty a_k$ is summable iff for each $\epsilon >0$, there is an index $N$ for which
		\[
			\biggl | \sum_{k=n}^{n+m} a_k \biggr | < \epsilon \text{ for } n \ge N \text{ and any natural number } m.	
		\]
		$(\implies)$ Suppose the series $\textstyle \sum_{k=1}^\infty a_k$ is summable.\par
		That is, there exists an $s$ such that $\{\textstyle \sum_{k=1}^n a_k\}$ converges to $s$.
		Convergent sequences are Cauchy, so for any $\epsilon>0$, there exists and index $N$ such that for all $n+m\ge n-1\ge N$,
		\begin{align*}
			\biggl| \sum_{k=1}^{n+m} a_k - \sum_{k=1}^{n-1} a_k\biggr| &< \epsilon\\
			\biggl|\sum_{k=1}^{n-1} a_k + \sum_{k=n}^{n+m} a_k - \sum_{k=1}^{n-1} a_k\biggr| &< \epsilon\\
			\biggl|\sum_{k=n}^{n+m} a_k\biggr| &< \epsilon.
		\end{align*}
		\par
		$(\impliedby)$ Suppose that for each $\epsilon >0$, there is an index $N$ for which
		\[
			\biggl | \sum_{k=n}^{n+m} a_k \biggr | < \epsilon \text{ for } n \ge N \text{ and any natural number } m.	
		\]
		Then
		\begin{align*}
			\biggl | \sum_{k=n}^{n+m} a_k +\sum_{k=1}^{n-1} a_k - \sum_{k=1}^{n-1} a_k \biggr | &< \epsilon\\
			\biggl | \sum_{k=1}^{n+m} a_k - \sum_{k=1}^{n-1} a_k\biggr | &< \epsilon
		\end{align*}
		Without loss of generality, we can suppose that $n-1 \ge N$, and because $m$ is a natural number, $n+m>n-1\ge N$.
		Clearly this describes a Cauchy Sequence, and because the real numbers is complete, this sequence converges and thus the series is summable.
		\item If the series $\sum_{k=1}^\infty |a_k|$ is summable, then $\sum_{k=1}^\infty a_k$ is also summable.\par
		By subadditivity of absolute value, we can show that for each $\epsilon >0$, there is an index $N$ for which
		\[
			\biggl | \sum_{k=n}^{n+m} a_k \biggr |\le \biggl | \sum_{k=n}^{n+m} |a_k| \biggr | < \epsilon \text{ for } n \ge N \text{ and any natural number } m.	
		\]
		\item If each term $a_k$ is nonnegative, then the series $\sum_{k=1}^\infty a_k$ is summable iff the sequence of partial sums is bounded.\par
		Let $\{a_k\}$ be a sequence of nonnegative numbers.\par
		$(\implies)$ Suppose the series $\sum_{k=1}^\infty a_k$ is summable.\par
		Then the sequence of partial sums converges to a real number.
		By Proposition 14, the sequence of partial sums is  bounded.
		\par
		$(\impliedby)$ Suppose the sequence of partial sums is bounded.\par
		Because each $a_k$ is positive, the sequence of partial sums is positive monotonic:
		\[
			\sum_{k=1}^n a_k < \sum_{k=1}^n a_k + a_{n+1} = \sum_{k=1}^{n+1} a_k.
		\]
		Therefore by Theorem 15, the sequence of partial sums converges; that is, the series is summable.
	\end{enumerate}	
\end{namedthm*}


\end{flushleft}

\begin{center}
	\textbf{PROBLEMS}
\end{center}
\begin{enumerate}
	\setcounter{enumi}{37}
	\item We call an extended real number a \textbf{cluster point} of a sequence $\{ a_n\}$ if a subsequence converges to this extended real number. Show that $\lim \inf \{a_n\}$ is the smallest cluster point of $\{a_n\}$ and $\lim \sup \{a_n\}$ is the largest cluster point of $\{a_n\}$.\par
	Let $s = \lim \sup \{a_n\} = \lim_{n \to \infty} \sup \{ a_k\ |\ k \ge n\}$.
	Suppose there exists a subsequence $\{ a_{n_k} \}$ that converges to an extended real number $a$.
	Fix $\epsilon >0$. Then there exists an index $M$ such that $|a-a_{n_m}| < \epsilon$ when $n_m \ge M$, and $a_{n_m} \le \sup \{ a_k\ |\ k \ge M\}$.\par
	Then $\lim_{M \to \infty} a_{n_m} \le \lim_{M \to \infty} \sup \{ a_k\ |\ k \ge M\} \implies a \le s$.\par
	Therefore $\lim \sup \{a_n\}$ is the largest cluster point of $\{a_n\}$.
	($\lim \sup \{a_n\}$ is itself a cluster point else we reach a contradiction to the supremum.)
	The same method can be used to prove $\lim \inf \{a_n\}$.
	\item Prove proposition 19.\par
	See above for proof.
	\item Show that a sequence $\{a_n\}$ is convergent to an extended real number iff there is exactly one extended real number that is a cluster point of the sequence.\par
	$(\implies)$ Suppose $\{a_n\}$ is convergent to an extended real number $a$.\par
	By Proposition 19(iv), we have $\lim \inf \{a_n \}= \lim \sup \{a_n \} = a$, so clearly any cluster point is equal to $a$.
	\par
	$(\impliedby)$ Suppose there is exactly one extended real number $a$ that is a cluster point of $\{a_n\}$.\par
	Then there exists a subsequence that converges to $a$.
	Suppose that $\{a_n\}$ does not converge to $a$.
	Then there exists an $\epsilon > 0$ such that there are infinitely many indices $n$ for which $a-a_n > \epsilon$.
	Collect these indices to construct a subsequence $\{a_{n_k}\}$.
	In the case that $\{a_{n_k}\}$ is bounded, there exists another subsequence of $\{a_{n_k}\}$ that converges to a real number $b \neq a$. 
	But this is also a subsequence of the original sequence $\{a_n\}$, which implies $\{a_n\}$ has two cluster points $a$ and $b$, a contradiction.
	In the case that $\{a_{n_k}\}$ is unbounded, then for any real number $c$, there exists an index $n$ such that $|a_n| >c$.
	Then we can construct a subsequence that converges to $+\infty \neq a$ or $-\infty \neq a$, which is again a contradiction to the fact that $\{a_n\}$ has only one cluster point.
	\item Show that $\lim \inf a_n \le \lim \sup a_n$.\par
	For any natural number $n$, we have $\inf \{ a_k\ |\ k \ge n\} \le a_k \le \sup \{ a_k\ |\ k \ge n\}$ for all $k \ge n$.
	Taking the limit with respect to n clearly proves the statement.
	\item Prove that if, for all $n$, $a_n \ge 0$ and $b_n \ge 0 $, then \[ \lim \sup [a_n \cdot b_n] \le (\lim \sup a_n) \cdot (\lim \sup b_n),\] provided the product on the right is not of the form $0 \cdot \infty.$\par
	For any natural number $n$, we can see that 
	\[
	\{a_k \cdot b_k \ |\ k\ge n\} \subseteq \{a_i \cdot b_j \ |\ i,j\ge n\}.	
	\]
	Then this clearly implies
	\begin{align*}
	\sup\{a_k \cdot b_k \ |\ k\ge n\} &\le \sup\{a_i \cdot b_j \ |\ i,j\ge n\}\\
	&=\sup\{a_i\ |\ i\ge n\}\cdot\sup\{b_j\ |\ j\ge n\}.	
	\end{align*}
	Taking the limit on both sides proves the inequality.
	\item Show that every real sequence has a monotone subsequence. Use this to provide another proof of the Bolzano-Weierstrass Theorem.\par
	Let $\{a_n\}$ be any sequence of real numbers.
	Supposing that there exist no monotone subsequences of $\{a_n\}$, then there are only finitely many indices $n$ for which $a_n \le a_{n+1}$, and only finitely many indices $n$ for which $a_n \ge a_{n+1}$.
	Clearly we see a contradiction so there must exist a monotone subsequence.
	\par
	Now, in the case that $\{a_n\}$ is bounded, then the monotone subsequence $\{a_{n_k}\}$ is also bounded.
	By Theorem 15, $\{a_{n_k}\}$ converges.
	Thus $\{a_n\}$ has a convergent subsequence. 
	\item Let $p$ be a natural number greater than 1, and $x$ a real number $0 \le x \le 1.$ Show that there is a sequence $\{a_n\}$ of integers with $0 \le a_n < p$ for each $n$ such that \[ x = \sum_{n=1}^\infty\dfrac{a_n}{p^n} \] 
	and that this sequence is unique except when $x$ is of the form $q/p^n$, $0<q<p^n$, in which case there are exactly two such sequences. Show that, conversely, if $\{a_n\}$ is any sequence of integers with $0\le a_n < p$, the series \[ x = \sum_{n=1}^\infty\dfrac{a_n}{p^n} \] 
	converges to a real number $x$ with $0 \le x \le 1$. If $p = 10$, this sequence is called the \textit{decimal} expansion of $x$. For $p=2$ it is called the \textit{binary} expansion; and for $p=3$, the \textit{ternary} expansion.\par

	For each $m \in \mathbb{N}$, we can construct a partial sum:
	\[
		\sum_{n=1}^{m} \frac{a_n}{p^n} = \sum_{n=1}^{m-1} \frac{a_n}{p^n} + \frac{a_m}{p^m}	
	\]
	We choose each $a_m$ in the following way:\par
	(The $\sum_{n=1}^{m-1} \frac{a_n}{p^n}$ is a fixed value found from the previous iteration, so for each step, we are simply choosing the best $a_m$).
	\par
	Case $\sum_{n=1}^{m-1} \frac{a_n}{p^n} + \frac{a_m}{p^m} = x$ for some $a_m \in \{0,1,\cdots,p\}$:
	Then set $a_k=0$ for all $k \ge m$, and the equality is clear.\par
	Else: Choose $a_m \in \{0,1,\cdots,p\}$ such that:
	\[
		\sum_{n=1}^{m-1} \frac{a_n}{p^n} + \frac{a_m}{p^m} < x <  \sum_{n=1}^{m-1} \frac{a_n}{p^n} + \frac{a_m+1}{p^m}.	
	\]
	\par
	In this way we can construct a monotone sequence (of partial sums) that is bounded above by $x$:
	\[
	\sum_{n=1}^k \frac{a_n}{p^n} \le \sum_{n=1}^{k+1} \frac{a_n}{p^n} \le x \text{ for all }k \in \mathbb{N}.
	\]
	By showing that $x$ is the supremum, we can apply Theorem 15 to show that this sequence of partial sums converges to its supremum:
	\[
	\lim_{k\to\infty}\sum_{n=1}^k\frac{a_n}{p^n} = \sum_{n=1}^\infty\frac{a_n}{p^n} = x.
	\]
	Suppose that $x$ is not the supremum. 
	Then there exists an $\epsilon > 0$ such that $\sum_{n=1}^k \frac{a_n}{p^n} \le x-\epsilon < x$ for all $k$.
	Now, by the Archimedean Property, there exists a natural number $m$ such that $\frac{1}{m} < \epsilon$; therefore $0 < \epsilon-\frac{1}{m}$.
	Now, because $p>1$, there exists a natural number $l$ such that $m<p^l$, so $0<\frac{1}{p^l}<\frac{1}{m}$ and thus $x-(\epsilon-\frac{1}{p^l})<x-(\epsilon-\frac{1}{m})<x$.\par
	Then for all natural numbers $k$,
	\[
		\sum_{n=1}^k \frac{a_n}{p^n} \le x-\epsilon< x-\epsilon+\frac{1}{p^l}<x-\epsilon+\frac{1}{m}< x.
	\]
	However, there exists the natural number $l$ such that
	\begin{align*}
		\sum_{n=1}^l \frac{a_n}{p^n} = \sum_{n=1}^{l-1} \frac{a_n}{p^n} + \frac{a_l}{p^l}&\le x-\epsilon < x-\epsilon+\frac{1}{p^l}< x\\
		\sum_{n=1}^{l-1} \frac{a_n}{p^n} + \frac{a_l+1}{p^l}& \le x-\epsilon +\frac{1}{p^l}<x.
	\end{align*}
	This is a contradiction to our choice of $a_l$ so that 
	\[
		\sum_{n=1}^{l-1} \frac{a_n}{p^n} + \frac{a_l}{p^l} < x <  \sum_{n=1}^{l-1} \frac{a_n}{p^n} + \frac{a_l+1}{p^l}.	
	\]
	Therefore $x$ is the supremum, and the series $\sum_{n=1}^\infty\dfrac{a_n}{p^n}$ is summable to $x$.\par
	In the case that $x$ is of the form $q/p^n$, the obvious solution would be to set $a_n =q$ (assuming $q$ is an integer), and all other $a_k=0$.
	The second solution would be to use the method described above.\par
	For the converse,  $0\le a_n \le p-1$ implies that 
	\[
		\sum_{n=1}^\infty\dfrac{a_n}{p^n} \le \sum_{n=1}^\infty\dfrac{p-1}{p^n}=(p-1)\sum_{n=1}^\infty\dfrac{1}{p^n}
	\]
	Showing that $(p-1)\sum_{n=1}^\infty\dfrac{1}{p^n} <1$ implies that $\sum_{n=1}^k\dfrac{a_n}{p^n}$ is a bounded, monotone sequence of partial sums, and therefore it converges to a number in $[0,1]$.\par
	\par
	Ex: $x=.547$; decimal expansion: \[x=\frac{5}{10^1}+\frac{4}{10^2}+\frac{7}{10^3}+\frac{0}{10^4}+\frac{0}{10^5}+\cdots=.5+.04+.007+0+0+\cdots\]
	\item Prove Proposition 20.\par
	See above.
	\item Show that the assertion of the Bolzano-Weierstrass Theorem is equivalent to the Completeness Axiom for the real numbers. Show that the assertion of the Monotone Convergence Theorem is equivalent to the Completeness Axiom for the real numbers.\par
	The Bolzano-Weierstrass Theorem asserts that every bounded sequence of real numbers has a convergent subsequence.\par
	The Completeness Axiom asserts that every nonempty set of real numbers that is bounded above has a supremum.\par
	The Monotone Convergence Theorem asserts that a monotone sequence of real numbers converges iff it is bounded.
\end{enumerate}

% 1.6
\section{Continuous Real-Valued Functions of a Real Variable}
\begin{flushleft}

\begin{namedthm*}{Proposition 21}
	A real-valued function $f$ defined on a set $E$ of real numbers is continuous at the point $x_*\in E$ iff 
	whenever a sequence $\{x_n\}$ in $E$ converges to $x_*$, its image sequence $\{f(x_n)\}$ converges to $f(x_*)$.
\end{namedthm*}
\begin{proof}
	Let $f$ be a real-valued function defined on a set $E$.\par
		$(\implies)$ Suppose that $f$ is continuous at the point $x_*\in E$.\par
		Then for all $\epsilon>0$, there exists a $\delta>0$ such that 
		\[
			\text{if } x'\in E\text{ and }|x-x'|<\delta,\text{ then }|f(x)-f(x')|<\epsilon.
		\]
		Suppose that a sequence $\{x_n\}$ in $E$ converges to $x_*$. 
		Then for any $\delta>0$, there exists an index $N$ such that when $n\ge N$, $|x_*-x_n|<\delta$.
		Then by continuity of $f$, $|f(x_*)-f(x_n)|<\epsilon$, and thus the image sequence converges.\par
		$(\impliedby)$ Suppose that whenever a sequence $\{x_n\}$ in $E$ converges to $x_*$, its image sequence $\{f(x_n)\}$ converges to $f(x_*)$.\par
		That is, for any $\delta>0$, there exists an index $N$ such that $|x_*-x_n|<\delta$ whenever $n\ge N$, 
		and this implies that for any $\epsilon>0$, there exists an index $M$ such that $|f(x_*)-f(x_n)|<\epsilon$ whenever $n\ge M$.
		Thus continuity is clear.
\end{proof}

\begin{namedthm*}{Proposition 22}
	Let $f$ be a real-valued function defined on a set $E$ of real numbers.
	Then $f$ is continuous on $E$ iff for each open set $\mathcal{O}$,
	\[
	f^{-1}(\mathcal{O})=E \cap \mathcal{U}\text{ where }\mathcal{U}\text{ is an open set.}	
	\]
\end{namedthm*}

\begin{namedthm*}{The Extreme Value Theorem}
	A continuous real-valued function on a nonempty, closed, bounded set of real numbers takes a minimum and a maximum value.	
\end{namedthm*}
\begin{proof}
	Let $f$ be a continuous real-valued function on a nonempty, closed, bounded set $E$ of real numbers.
	Suppose by contradiction that $f$ is not bounded.
	Then for any $n \in \mathbb{N}$, there exists $x_n \in E$ such that $f(x_n) > n$.
	With this we can construct a sequence $\{x_n\}$ in $E$. Because $E$ is bounded, $\{x_n\}$ is bounded, and so by the Bolzano Weierstrass Theorem, there exists a convergent subsequence $\{x_{n_k}\}$.
	Because $f$ is continuous, $\{x_{n_k}\}$ is convergent implies $\{f(x_{n_k})\}$ is convergent.
	However, for each element in the image sequence, $f(x_{n_k})>n_k$, and $\{f(x_{n_k})\}$ is unbounded, thus it cannot converge, and we reach a contradiction.\par
	Because $f$ is bounded, then it has a supremum $s$ such that $f(x) \le s$ for all $x \in E$.
	Suppose that $f$ does not have a maximum. Then there is no $x \in E$ such that $f(x)=s$.
	Then $f(x) < s \implies f(x) \in (-\infty,s)$ for all $x \in E$.
	(We can use the fact that $(-\infty,s)$ is open $\implies f^{-1}(-\infty,s)$ is open):
	Then we reach a contradiction because $E$ is closed. The same proof can be used for the minimum.
\end{proof}

\begin{namedthm*}{The Intermediate Value Theorem}
	Let $f$ be a continuous real-valued function on the closed, bounded interval $[a,b]$ for which $f(a)<c<f(b)$.
	The there is a point $x_0$ in $(a,b)$ at which $f(x_0)=c$.
\end{namedthm*}

\begin{namedthm*}{Theorem 23}
	A continuous real-valued function on a closed, bounded set of real numbers is uniformly continuous.
\end{namedthm*}
\begin{proof}
	Let $E$ be a closed, bounded set of real numbers, and let $f$ be a continuous real-valued function on $E$.\par
	Fix some $\epsilon>0$.\par
	By the continuity of $f$, for all $x\in E$, there exists $\delta_x>0$ such that for $y\in E$ satisfying $|x-y|< 2\delta_x$, then $|f(x)-f(y)|< \frac{\epsilon}{2}$.
	Then we can construct an open cover of $E$ consisting of the open balls $\mathbb{B}(x,\delta_x)$ for all $x \in E$.\par
	Because $E$ is compact, there exists a finite subcover $\{\mathbb{B}(x_1,\delta_1),\cdots,\mathbb{B}(x_n,\delta_1)\}$.\par
	Let $\delta_* = \min\{\delta_1,\cdots,\delta_n\}$.\par
	Consider $x,y\in E$ such that $|x-y|<\delta_*$.\par
	Because $y \in E \subseteq \{\mathbb{B}(x_1,\delta_1),\cdots,\mathbb{B}(x_n,\delta_1)\}$, there exists an index $j\in \{1,\cdots,n\}$ such that $y \in \mathbb{B}(x_j,\delta_j)$; therefore
	\[
		|x_j-y|<\delta_j <2\delta_j.
	\]
	By continuity of $f$, $|f(x_j)-f(y)|<\frac{\epsilon}{2}$. (A)\par
	By the triangle inequality,
	\[
		|x-x_j|\le |x-y| + |y-x_j| <\delta_* + \delta_j \le 2\delta_j.
	\]
	By continuity of $f$, $|f(x)-f(x_j)|<\frac{\epsilon}{2}$. (B)\par
	By the triangle inequality using (A) and (B):
	\[
		|f(x)-f(y)| \le |f(x)-f(x_j)| +|f(x_j)-f(y)| < \frac{\epsilon}{2}+\frac{\epsilon}{2} = \epsilon.
	\]
\end{proof}

\end{flushleft}
\begin{center}
	\textbf{PROBLEMS}
\end{center}
\begin{enumerate}
	\setcounter{enumi}{46}
	\item Let $E$ be a closed set of real numbers and $f$ a real-valued function that is defined and continuous on $E$. Show that there is a function $g$ defined and continuous on all of $\mathbb{R}$ such that $f(x) = g(x)$ for each $x \in E$. (Hint: Take $g$ to be linear on each of the intervals of which $\mathbb{R} \setminus E$ is composed.)\par
	Because $E$ is closed, then $\mathbb{R} \setminus E$ is open. 
	In the case that $E = \mathbb{R}$, then  $\mathbb{R} \setminus E = \emptyset$ and the conclusion is trivial. 
	Else $\mathbb{R} \setminus E$ is nonempty. By proposition 9, $\mathbb{R} \setminus E$ is the union of a countable, disjoint collection of open intervals.	
	\\In the case that $(-\infty,a)$ [or $(a,\infty)$] is in $\mathbb{R} \setminus E$, then $a \in E$ and $f(a)$ is defined.
	Simply let $g(x)=f(a)$ be the constant function on $(-\infty,a)$ [or $(a,\infty)$].
	\\In the case that $(a,b) \in \mathbb{R} \setminus E$, then $a,b\in E$ and $f(a)$,$f(b)$ are defined.
	Let 
	\[
		g(x)=\frac{f(b)-f(a)}{b-a}(x-a)+f(a)\text{ on } (a,b).
	\]
	Also let $g(x)=f(x)$ whenever $x\in E$. Then we see that $g$ is continuous.
	\item Define the real-valued function $f$ on $\mathbb{R}$ by setting 
	\[ 
	f(x) =
	\begin{cases} 
		x & \text{if x irrational}\\
		p \sin \dfrac{1}{q} & \text{if } x = \dfrac{p}{q} \text{ in lowest terms.} \\
	\end{cases}
	\]
	At what points is $f$ continuous?\par
	See Thomae's Function for something similar.\par
	$f$ should be discontinuous at each rational number and continuous at each irrational number.
	\item Let $f$ and $g$ be continuous real-valued functions with a common domain $E$.
	\begin{enumerate}[label=(\roman*),align=left]
        \item Show that the sum, $f+g$, and product, $fg$, are also continuous functions.\par
        Suppose $\{x_n\}\in E$ converges to $x\in E$.
        Then $\{f(x_n)\}$ converges to $f(x)$ and $\{g(x_n)\}$ converges to $g(x)$ by continuity of $f,g$.
		\par
		That is, for any $\epsilon>0$, there exists a $0<\delta\le \delta_f,\delta_g$ such that
		$|f(x_n)-f(x)|<\frac{\epsilon}{2}$ and $|g(x_n)-g(x)|<\frac{\epsilon}{2}$ whenever $|x_n-x|<\delta$.
		By the triangle inequality,
		\begin{align*}
			|(f+g)(x_n)-(f+g)(x))| &= |(f(x_n)+g(x_n))-(f(x)+g(x))| \\
			&\le |f(x_n)-f(x)|+|g(x_n)+g(x)| \\
			&< \frac{\epsilon}{2} + \frac{\epsilon}{2} = \epsilon.
		\end{align*}
        \par
		Fix any $\epsilon>0$. By continuity of $f,g$, there exists a $0<\delta\le \delta_f,\delta_g$ such that
		$|f(y)-f(x)|<\frac{\epsilon}{2|g(x)|}$ and $|g(y)-g(x)|<\frac{\epsilon}{2|f(y)|}$ whenever $|y-x|<\delta$.
		\begin{align*}
			|fg(y)-fg(x)| &= |f(y)g(y)-f(x)g(x)|\\
			&=|f(y)g(y)-f(y)g(x)+f(y)g(x)-f(x)g(x)|\\
			&=|f(y)(g(y)-g(x))+g(x)(f(y)-f(x))|\\
			&\le|f(y)(g(y)-g(x))|+|g(x)(f(y)-f(x))|\\
			&=|f(y)||g(y)-g(x)|+|g(x)||f(y)-f(x)|\\
			&<|f(y)|\frac{\epsilon}{2|f(y)|}+|g(x)|\frac{\epsilon}{2|g(x)|}\\
			&=\epsilon
		\end{align*}
        (This one is a bit janky).
        \item If $h$ is a continuous function with image contained in $E$, show that the composition $f \circ h$ is continuous.\par
        Suppose $\{x_n\}\in E$ converges to $x\in E$.
		Then $\{h(x_n)\}\in E$ converges to $h(x)\in E$ by continuity of $h$.
		Then $\{f\circ h(x_n)\}=\{f(h(x_n))\}$ converges to $f\circ h(x) = f(h(x))$ by continuity of $f$.
		Therefore the composition is continuous.
        \item Let max$\{f,g\}$ be the function defined by max$\{f,g\}(x)$ = max$\{f(x),g(x)\}$, for $x \in E$. Show that max$\{f,g\}$ is continuous.\par
        Fix $\epsilon>0$. By continuity of $f,g$, there exists a $0<\delta\le\delta_f,\delta_g$ s.t. whenever $|x-y|<\delta$, then $|f(x)-f(y)|<\frac{\epsilon}{2}$ and $|g(x)-g(y)|<\frac{\epsilon}{2}$.
		\par
		We can write
		\[
			\max\{f(x),g(x)\} = \frac{f(x)+g(x)}{2}+\frac{|f(x)-g(x)|}{2}.
		\]
		This is by the identity:
		\begin{align*}
		\max(x,y)+\min(x,y)&=x+y\\
		\max(x,y)-\min(x,y)&=|x-y|\\
		\max(x,y)&=\frac{1}{2}(x+y+|x-y|)\\
		\min(x,y)&=\frac{1}{2}(x+y-|x-y|)
		\end{align*}
		Now, $|\max\{f(x),g(x)\}-\max\{f(y),g(y)\}|$ is equal to
		\begin{align*}
			\biggl | \frac{f(x)+g(x)}{2}+\frac{|f(x)-g(x)|}{2} - (\frac{f(y)+g(y)}{2}+\frac{|f(y)-g(y)|}{2}) \biggr | \\
			= \biggl | \frac{f(x)-f(y)+g(x)-g(y)+|f(x)-g(x)|-|f(y)-g(y)|}{2} \biggr |\\
			\le \frac{|f(x)-f(y)|+|g(x)-g(y)|+ |\ |f(x)-g(x)|-|f(y)-g(y)|\ |}{2} \\
			\le \frac{|f(x)-f(y)|+|g(x)-g(y)|+ |f(x)-g(x)-f(y)+g(y)|}{2} \\
			\le \frac{|f(x)-f(y)|+|g(x)-g(y)|+ |f(x)-f(y)|+|g(y)+g(x)|}{2} \\
			< \frac{\frac{\epsilon}{2}+\frac{\epsilon}{2}+ \frac{\epsilon}{2}+\frac{\epsilon}{2}}{2} \\
			=\epsilon.
		\end{align*}
        \item Show that $|f|$ is continuous.\par
        For any $\epsilon>0$, there exists a delta such that whenever $|x-y|<\delta$, by the reverse triangle inequality:
        \begin{align*}
			|\ |f(x)|-|f(y)|\ |\le|f(x)-f(y)| <\epsilon.
		\end{align*}
    \end{enumerate}
	\item Show that a Lipschitz function is uniformly continuous but there are uniformly continuous functions that are not Lipschitz.\par
	Lipschitz: there exists $L\ge0$ s.t. for all $x,x'$:
	\[
		|f(x)-f(x')|\le L|x-x'|
	\]
	Fixing any $\epsilon>0$, whenever $|x-x'|\le\delta$, we have
	\[
		|f(x)-f(x')|\le L|x-x'|<L\delta,
	\]
	so we can set $\delta=\dfrac{\epsilon}{L}$.
	The $\delta$ is the same for any values of $x$, so $f$ is uniformly continuous.\par
	The function $\sqrt{x}$ is uniformly continuous but not Lipschitz.
	\item A continuous function $\phi$ on $[a,b]$ is called \textbf{piecewise linear} provided there is a partition $a=x_0<x_1< \cdots <x_n = b$ of $[a,b]$ for which $\phi$ is linear on each interval $[x_i, x_{i+1}]$. Let $f$ be a continuous function on $[a,b]$ and $\epsilon$ a positive number. 
	Show that there is a piecewise linear function $\phi$ on $[a,b]$ with $|f(x)-\phi (x)| < \epsilon$ for all $x \in [a,b]$.\\
	Start with $f(x_0)$, and choose $x_1$ so that $f(x_1)=f(x_0)\pm \epsilon$.\\
	Define $\phi(x) = \frac{f(x_1)-f(x_0)}{x_1-x_0}(x-x_0)+f(x_0)$ on $[x_0,x_1]$.\\
	Repeat this process to choose each interval:\\
	Start with $f(x_i)$, and choose $x_{i+1}$ so that $f(x_{i+1})=f(x_i)\pm \epsilon$.\\
	Define $\phi(x) = \frac{f(x_{i+1})-f(x_i)}{x_{i+1}-x_i}(x-x_i)+f(x_i)$ on $[x_i,x_{i+1}]$.\\
	Then we see that $f$ and $\phi$ are always within $\epsilon$ of each other, and $\phi$ is continuous and piecewise linear.
	\item Show that a nonempty set $E$ of real numbers is closed and bounded if and only if every continuous real-valued function on $E$ takes a maximum value.\\
	Let $E$ be a nonempty set of real numbers.\\
	$(\implies)$ Suppose $E$ is closed and bounded.\\
	By the Extreme Value Theorem, every continuous real-valued function on $E$ takes a maximum (and minimum) value.\\
	$(\impliedby)$ Suppose every continuous real-valued function on $E$ takes a maximum value.\\
	Suppose that $E$ is not closed.
	The continuous real-valued function $f(x)=\frac{1}{x}$ on the open set $E=(0,1)$ does not take on a maximum value. Contradiction.\\
	Suppose $E$ is not bounded. 
	The continuous real-valued function $f(x)=x^2$ on the unbounded set $E=[0,\infty)$ does not take on a maximum value. Contradiction.\\
	Therefore $E$ must be closed and bounded.
	(Not the right way to do this...)
	\item Show that a set $E$ of real numbers is closed and bounded iff every open cover of $E$ has a finite subcover.\par
	Let $E$ be a set of real numbers.\par
	$(\implies)$ Suppose $E$ is closed and bounded.\par
	By the Heine-Borel Theorem, every open cover of $E$ has a finite subcover.
	\par
	$(\impliedby)$ Suppose every open cover of $E$ has a finite subcover.\par
	See the proof in 1.4 after the Heine-Borel Theorem.
	\item Show that a nonempty set $E$ of real numbers is an interval iff every continuous real-valued function on $E$ has an interval as its image.\\
	Let $E$ be a nonempty set of real numbers.\\
	$(\implies)$ Suppose $E$ is an interval.\\
	Then for any two points $x,y\in E$, the set $[x,y]$ is in $E$.
	Let $f$ be a continuous real-valued function on $E$.
	Now, we have $f(x),f(y)$ in the image of $f$.
	Suppose, without loss of generality, that $f(x)<f(y)$.
	By the Intermediate Value Theorem, for any $c$ such that $f(x)<c<f(y)$, there exists $x_0\in(x,y)\subseteq E$ such that $f(x_0)=c$.
	That is, for any two points in the image of $f$, every point between them is also in the image of $f$.
	Therefore the image of $f$ is an interval.\\
	$(\impliedby)$ Suppose every continuous real-valued function on $E$ has an interval as its image.\\
	Suppose $E$ is not an interval.
	Then there exist two points $x,y\in E$ such that $x<a<y$ but $a \notin E$.
	\\Let $f$ be a continuous real-valued function on $E$, and without loss of generality, let $f$ be monotonically increasing.
	Because $x,y\in E$, then $f(x)$,$f(y)$ are defined, so $[f(x),f(y)]$ is in the image of $f$.
	\\Define two disjoint collections of subsets of E: $I_{<a}=\{I\subseteq E\ |\ x<a \ \forall x \in I\}$ and $I_{>a}=\{I\subseteq E\ |\ x>a \ \forall x \in I\}$, so that $I_{<a}\cap I_{>a}=\emptyset$.
	These collections are nonempty because $\{x\}\in I_{<a}$ and $\{y\}\in I_{>a}$.
	Consider $\bigcup I_{<a}\subseteq E$, the union of all elements of $I_{<a}$, and $\bigcup I_{>a}\subseteq E$, the union of all elements of $I_{>a}$.
	By monotonicity of $f$, $f(\bigcup I_{<a})<f(\bigcup I_{>a})$, so $[f(x),f(y)]\not\subseteq f(\bigcup I_{<a})\bigcup f(\bigcup I_{>a})=f(E)$, a contradiction.
	\item Show that a monotone function on an open interval is continuous iff its image is an interval.\\
	Let $f$ be a monotone function on an open interval $E=(a,b)$.\\
	$(\implies)$ Suppose $f$ is continuous.\\
	Then by Problem 54, $E$ is an interval implies that the continuous real-valued function $f$ has an interval as its image.\\
	$(\impliedby)$ Suppose the image of $f$ is an interval.\\
	Let $x_0$ be a point in the open interval $E$, so that $f(x_0)$ is defined. 
	For any sequence $\{x_n\}$ in $E\cap(x_0,\infty)$ that converges to $x_0$, then $\{f(x_n)\}$ converges to $f(x_0^+)$.\\
	Similarly, for any sequence $\{x_n\}$ in $E\cap(-\infty,x_0)$ that converges to $x_0$, then $\{f(x_n)\}$ converges to$f(x_0^-)$.\\
	Then $f(x_0^-)=f(x_0)= f(x_0^+)$ by monotonicity.
	\\(messy)
	\item Let $f$ be a real-valued function defined on $\mathbb{R}$. Show that the set of points at which $f$ is continuous is a $G_\delta$ set.\\
	A $G_\delta$ set is a set that is a countable intersection of open sets.\\
	$f$ is continuous at a point $x$ if for any open set in the image containing $f(x)$, the inverse image is an open set containing $x$.
	\item Let $\{ f_n\}$ be a sequence of continuous functions defined on $\mathbb{R}$. Show that the set of points $x$ at which the sequence $\{f_n(x)\}$ converges to a real number is the intersection of a countable collection of $F_\sigma$ sets.\\
	An $F_\sigma$ set is a set that is a countable union of closed sets.
	\item Let $f$ be a continuous real-valued function on $\mathbb{R}$. Show that the inverse image with respect to $f$ of an open set is open, of a closed set is closed, and of a Borel set is Borel.\\
	The inverse image of an open set is open (See prop 22).\\
	Suppose that the inverse image of a closed set is not closed.
	That is, let $B$ be a closed set of real numbers and let $f^{-1}(B)=\{x\in\mathbb{R}\ |\ f(x)\in B\}$ not be closed. 
	Then there exists a sequence $x_n\in f^{-1}(B)$ that converges to $x \notin f^{-1}(B)$.
	However, by continuity of $f$, $f(x_n)\in B$ converges to $f(x)\notin B$.
	This implies that $B$ does not contain all its limit points, and thus $B$ is not closed, a contradiction.
	Therefore $f^{-1}(B)$ must be closed.
	\item A sequence $\{f_n\}$ of real-valued functions defined on a set $E$ is said to converge uniformly on $E$ to a function $f$ iff given $\epsilon >0$, there is an $N$ such that for all $x \in E$ and all $n \ge N$, we have $|f_n(x) - f(x)| < \epsilon$. Let $\{f_n\}$ be a sequence of continuous functions defined on a set $E$. Prove that if $\{f_n\}$ converges uniformly to $f$ on $E$, then $f$ is continuous on $E$.\\
	We want to show that uniform convergence preserves continuity.\\
	Fix $\epsilon>0$.\\
	By uniform convergence of $\{f_n\}$, there exists an index $N$ such that $|f(x)-f_n(x)|<\frac{\epsilon}{3}$ for all $x\in E$ and all $n\ge N$.\\
	By continuity of each $f_n$, for all $x\in E$, there exists a $\delta>0$ such that $|f_n(x)-f_n(y)|<\frac{\epsilon}{3}$ whenever $|x-y|<\delta$.\\
	Therefore we have:
	\begin{align*}
		|f(x)-f(y)|&\le|f(x)-f_n(x)|+|f_n(x)-f_n(y)|+|f_n(y)-f(y)|\\
		&<\frac{\epsilon}{3}+\frac{\epsilon}{3}+\frac{\epsilon}{3}\\
		&=\epsilon.
	\end{align*}
\end{enumerate}

% Chapter 2
\chapter{Lebesgue Measure}

% 2.1
\section{Introduction}
In this chapter we construct a collection of sets called \textbf{Lebesgue measurable sets}, and a set function of this collection called \textbf{Lebesgue measure}, denoted by $m$. 
(A \textit{set function} is a function that associates an extended real number to each set in a collection of sets.)
The collection of Lebesgue measurable sets is a $\sigma$-algebra which contains all open sets and all closed sets. The set function $m$ possesses the following three properties:
\begin{namedthm*}{The measure of an interval is its length}
Each nonempty interval $I$ is Lebesgue measurable and 
\[
m(I) = \ell(I).
\]
\end{namedthm*}
\begin{namedthm*}{Measure is translation invariant}
If $E$ is Lebesgue measurable and $y$ is any number then the translate of $E$ by $y$, $E+y = \{x+y \ |\ x \in E\}$, also is Lebesgue measurable and
\[
m(E+y) = m(E).
\]
\end{namedthm*}
\begin{namedthm*}{Measure is countably additive over countable disjoint unions of sets}
If $\{E_k\}_{k=1}^\infty$ is a countable disjoint collection of Lebesgue measurable sets, then
\[
m(\bigcup_{k=1}^\infty E_k) = \sum_{k=1}^\infty m(E_k).
\]
\end{namedthm*}
It is not possible to construct a set function that possesses the above three properties and is defined for all sets of real numbers (See Vitali sets).
We first construct a set function called \textbf{outer measure}, denoted by $m^*$, such that: 
\begin{enumerate}[label=(\roman*),align=left]
	\item the outer measure of an interval is its length:
	\[
		m^*(I)=\ell(I).
	\]
	\item outer measure is translation invariant:
	\[
		m^*(A+y)=m^*(A).
	\]
	\item outer measure is countably subadditive:
	\[
		m(\bigcup_{k=1}^\infty E_k) \le \sum_{k=1}^\infty m(E_k).
	\]
\end{enumerate}
Outer measure is defined for all sets of real numbers.
However, outer measure fails to be countably additive: there exists $A,B$ disjoint s.t. $m^*(A\cup B)<m^*(A)+m^*(B)$.\\
Then the Lebesgue measure $m$ is the restriction of $m^*$ to the Lebesgue measurable sets.

\begin{center}
	\textbf{PROBLEMS}
\end{center}
In the first three problems, let $m$ be a set function defined for all sets in a $\sigma$-algebra $\mathcal{A}$ with values in $[0,\infty]$. Assume $m$ is countably additive over countable disjoint collections of sets in $\mathcal{A}$.
\begin{enumerate}
	\setcounter{enumi}{0}
	\item Prove that if $A$ and $B$ are two sets in $\mathcal{A}$ with $A \subseteq B$, then $m(A) \le m(B)$. This property is called \textit{monotonicity}.\par
	$A \subseteq B \implies B = A \cup (B \cap A^c),$ where $A \cap (B \cap A^c) = \emptyset$. The set $(B \cap A^c)$ is measurable because $A^c$ is measurable and countable intersection is measurable, so $m(B) = m(A \cup (B \cap A^c)) = 	m(A) + m(B \cap A^c)$ by countable additivity, and thus $m(B) \ge m(A)$.
	\item Prove that if there is a set $A$ in the collection $\mathcal{A}$ for which $m(A) < \infty$, then $m(\emptyset) = 0$.\par
	We have $A\cap\emptyset = \emptyset$ and $A\cup\emptyset = A$. 
    \begin{align*}
        m(A)&=m(A\cup\emptyset)\\
        m(A)&=m(A)+m(\emptyset)&&\text{ by disjoint additivity}\\
        0&=m(\emptyset).
    \end{align*}
	\item Let $\{E_k\}_{k=1}^\infty$ be a countable collection of sets in $\mathcal{A}$. Prove that $m(\bigcup_{k=1}^\infty E_k) \le \sum_{k=1}^\infty m(E_k).$
    For any two measurable sets $A,B$, we have $A\cup B=(A\setminus B)\cup(B)$.
    By disjoint additivity,
    \[
        m(A\cup B) = m(A\setminus B)+m(B)
    \]
    Now, by problem 1, $(A\setminus B)\subseteq A$ implies that $m(A\setminus B)\le m(A)$.
    Therefore
    \[
        m(A\cup B) \le m(A)+m(B).
    \]
    \item A set function $c$, defined on all subsets of $\mathbb{R}$, is defined as follows.
	Define $c(E)$ to be $\infty$ if $E$ has infinitely many members and $c(E)$ to be equal to the number of elements in $E$ if $E$ is finite; define $c(\emptyset)=0$. Show that $c$ is a countably additive and translation invariant set function. This set function is called the \textbf{counting measure}.\\
    Suppose $E=\{x_1,\cdots,x_n\}$.\\
    Then $m(E)=n$. For any real number $y$, $y+E = \{y+x_1,\cdots,y+x_n\}$, so $m(y+E)=n$.\\
    Suppose $E$ has infinitely many members.\\
    Then $y+E$ has infinitely members as well, so $m(E)=m(y+E)=\infty$.\\
    Let $\{E_k\}_{k=1}^\infty$ be a disjoint collection of sets of real numbers.
    In the case that there exists an $E_k$ with infinitely many members, then the countable additivity is clear.
    \\In the case that all sets $E_k$ are finite, for any two sets $E_i,E_j$:\\
    $E_i = \{x_1,\cdots,x_n\}$\\
    $E_j = \{y_1,\cdots,y_m\}$\\  
    Then $E_i\cup E_j = \{x_1,\cdots,x_n,y_1,\cdots,y_m\}$ and $m(E_i\cup E_j)=n+m=m(E_i) +m(E_j)$.
\end{enumerate}

% 2.2
\section{Lebesgue Outer Measure}
\begin{flushleft}

Let $I$ be a nonempty interval of real numbers. We define its length:
\[
	\ell(I)=
	\begin{cases}
		\infty&\text{if $I$ is unbounded}\\
		b-a&\text{endpoints $a,b$}
	\end{cases}	
\]
For a set $A$ of real numbers, consider the countable collections $\{I_k\}_{k=1}^\infty$ of nonempty open, bounded intervals that cover $A$; that is, collections for which $A\subseteq\bigcup_{k=1}^\infty I_k$.
For each such collection, consider the sum of the lengths of the intervals in the collection. Since the lengths are positive numbers, each sum is uniquely defined independently of the order of the terms.
We define the \textbf{outer measure} of $A$, $m^*(A)$, to be the infimum of all such sums, that is
\[
	m^*(A)=\inf\biggl\{\sum_{k=1}^\infty\ell(I_k)\ \biggl|\ A\subseteq\bigcup_{k=1}^\infty I_k\biggr\}.
\]
It follows immediately from the definition of outer measure that $m^*(\emptyset)=0$.
Moreover, since any cover of a set $B$ is also a cover of any subset of $B$, outer measure is \textbf{monotone} in the sense that
\[
	A\subseteq B\implies m^*(A)\le m^*(B).	
\]
Then because $\emptyset\subseteq A$ for any set $A$, we have $0=m^*(\emptyset)\le m^*(A)$.
\begin{namedthm*}{Example}
	A countable set $C$ has outer measure zero.\\
	Because $C$ is countable, enumerate $C$ such that $C=\{c_k\}_{k=1}^\infty$.
	Fix $\epsilon>0$. 
	For each $k\in\mathbb{N}$, define an open interval $I_k= (c_k-\frac{\epsilon}{2^{k+1}},c_k+\frac{\epsilon}{2^{k+1}})$.
	Then $C\subseteq\bigcup_{k=1}^\infty I_k$.
	Therefore we have, by definition of infimum,
	\[
	0\le m^*(C)\le\sum_{k=1}^\infty\ell(I_k)=\sum_{k=1}^\infty\frac{2\epsilon}{2^{k+1}} = \sum_{k=1}^\infty\frac{\epsilon}{2^k}=\epsilon.
	\]
	This inequality holds for each $\epsilon>0$; thus $m^*(C)=0$.
\end{namedthm*}
\begin{namedthm*}{Lemma}
	$\sum_{k=1}^\infty\frac{1}{2^k}=1$.
\end{namedthm*}
\begin{proof}
	To show that $\sum_{k=1}^n\frac{1}{2^k}=1-\frac{1}{2^n}$ (induction).\\
	Let $P(n)$ be the assertion that $\sum_{k=1}^n\frac{1}{2^k}=1-\frac{1}{2^n}$ for $n\in\mathbb{N}$.\\
	$P(1)$:
	\begin{align*}
		\sum_{k=1}^1\frac{1}{2^k}=\frac{1}{2}=1-\frac{1}{2^1}.
	\end{align*}
	$P(2)$:
	\begin{align*}
		\sum_{k=1}^2\frac{1}{2^k}=\frac{1}{2}+\frac{1}{4}=\frac{3}{4}=1-\frac{1}{2^2}.
	\end{align*}
	Suppose $P(m)$ is true for $m\ge1$; that is, $\sum_{k=1}^m\frac{1}{2^k}=1-\frac{1}{2^m}$.\\
	$P(m+1)$:
	\begin{align*}
		\sum_{k=1}^{m+1}\frac{1}{2^k}&=\sum_{k=1}^m\frac{1}{2^k}+\frac{1}{2^{m+1}}\\
		&=1-\frac{1}{2^m}+\frac{1}{2^{m+1}}\\
		&=1-\frac{2}{2^{m+1}}+\frac{1}{2^{m+1}}\\
		&=1-\frac{1}{2^{m+1}}.
	\end{align*}
	Therefore $P(m)$ is true for all $m\ge1$.
	\[
		\sum_{k=1}^\infty\frac{1}{2^k}=\lim_{n\to\infty}\sum_{k=1}^n\frac{1}{2^k}=\lim_{n\to\infty}(1-\frac{1}{2^n})=1.
	\]
	\\(An alternate proof would be to see that we have a sequence of partial sums that is monotonic with $1$ as the supremum. Then the sequence of partial sums converges to $1$ and the series is summable to $1$.)
\end{proof}

\end{flushleft}
\begin{center}
	\textbf{PROBLEMS}
\end{center}
\begin{enumerate}
	\setcounter{enumi}{4}
	\item By using properties of outer measure, prove that the interval $[0,1]$ is not countable.\\
	Suppose that the interval $[0,1]$ is countable. By an example above, we showed that a countable set has outer measure zero, so $m^*([0,1])=0$.
	Also, the outer measure of an interval is its length. Then $m^*([0,1])=1$, and we reach a contradiction.
	\item Let $A$ be the set of irrational numbers in the interval $[0,1]$. Prove that $m^*(A)=1$.\\
	Let $A=[0,1]\cap\mathbb{Q}^c$.\\
	Then $A\subseteq[0,1]$, so by monotonicity of outer measure, 
	\begin{align*}
		m^*(A)&\le m^*([0,1])\\
		m^*(A)&\le 1.
	\end{align*}
	Also, we have 
	\begin{align*}
		[0,1]&=([0,1]\cap\mathbb{Q}^c)\cup([0,1]\cap\mathbb{Q})\\
		[0,1]&=A\cup([0,1]\cap\mathbb{Q})\\
		[0,1]&\subseteq A\cup([0,1]\cap\mathbb{Q})&&A=B\implies A\subseteq B\text{ and }A\supseteq B\\
		m^*([0,1])&\le m^*(A\cup (m^*([0,1]\cap\mathbb{Q}))&&\text{by monotonicity}\\
		m^*([0,1])&\le m^*(A)+m^*([0,1]\cap\mathbb{Q})&&\text{by countable subadditvity}\\
		m^*([0,1])&\le m^*(A)+0&&\text{countable set has outer measure zero}\\
		1&\le m^*(A).&&\text{outer measure of interval is length}
	\end{align*}
	Then $m^*(A)\le 1$ and $1\le m^*(A)$ imply that $m^*(A)=1$.
	\item A set of real numbers is said to be a $G_\delta$ set provided it is the intersection of a countable collection of open sets.
	Show that for any bounded set $E$, there is a $G_\delta$ set $G$ for which
	\[E\subseteq G \ \text{and}\ m^*(G)=m^*(E).\]
	Suppose $E$ is a bounded set of real numbers.\\
	Then there exists a real number $M$ for which $|x|\le M$ for all $x\in E$; that is, $E\subseteq[-M,M]$.
	By monotonicity of outer measure, $m^*(E)\le m^*([-M,M])=2M<\infty$, and the outer measure of $E$ is finite.\\
	Now, because outer measure is defined as $m^*(E)=\inf\{\sum_{k=1}^\infty\ell(I_k)\ |\ E\subseteq\bigcup_{k=1}^\infty I_k\}$, we have that $m^*(E)$ is the greatest lower bound, so for a natural number $n$, $m^*(E)+\frac{1}{n}$ is not a lower bound.
	That is, there exists a countable sequence of open intervals $\{(I_n)_k\}_{k=1}^\infty$ such that $E\subseteq\bigcup_{k=1}^\infty (I_n)_k$ and 
	\begin{equation}
		m^*(E)\le\sum_{k=1}^\infty\ell((I_n)_k)<m^*(E)+\frac{1}{n}.\tag{1}
	\end{equation}
	Now, for each natural number $n$, we can define the open set
	\begin{equation}
		\mathcal{O}_n:=\bigcup_{k=1}^\infty (I_n)_k.\tag{2}
	\end{equation}
	Also define the countable intersection of open sets; i.e., a $G_\delta$ set:
	\[
		\mathcal{O}:=\bigcap_{n=1}^\infty\mathcal{O}_n.
	\]
	Then because we have $E\subseteq\mathcal{O}_n$ for every $n$, this implies $E\subseteq\bigcap_{n=1}^\infty\mathcal{O}_n=\mathcal{O}$.
	\begin{align*}
		m^*(E)&\le m^*(\mathcal{O})&&\text{by monotonicity of outer measure: }E\subseteq\mathcal{O}\\
		&\le m^*(\mathcal{O}_n)&&\text{by monotonicity of outer measure: }\mathcal{O}=\bigcap_{n=1}^\infty\mathcal{O}_n\subseteq\mathcal{O}_n\\
		&=m^*(\bigcup_{k=1}^\infty (I_n)_k)&&\text{by (2)}\\
		&\le\sum_{k=1}^\infty\ell((I_n)_k)&&\text{by countable subadditivity of outer measure}\\
		&<m^*(E)+\frac{1}{n}.&&\text{by (1)}
	\end{align*}
	Therefore for any natural number $n$,
	\[
		m^*(E)\le m^*(\mathcal{O})<m^*(E)+\frac{1}{n}.	
	\]
	Taking the limit as $n\to\infty$, we get that $m^*(E)= m^*(\mathcal{O})$.\\
	Therefore there exists a $G_\delta$ set $\mathcal{O}$ such that $E\subseteq \mathcal{O}$ and $m^*(E)= m^*(\mathcal{O})$.
	\item Let $B$ be the set of rational numbers in the interval $[0,1]$, and let $\{I_k\}_{k=1}^n$ be a finite collection of open intervals that covers $B$.
	Prove that $\textstyle \sum_{k=1}^n m^*(I_k) \ge 1$.\\
	The rational numbers are dense in the reals; that is, between any two real numbers, there exists a rational number.
	Therefore, the rational numbers are also dense in the real subset $[0,1]$: between any two numbers in $[0,1]$, there exists a rational number. \\
	In the case that $[0,1]\subseteq \bigcup_{k=1}^nI_k$, the inequality is clear by monotonicity and subadditivity: 
	\[
		1=m^*([0,1])\le m^*(\bigcup_{k=1}^nI_k)\le\sum_{k=1}^nm^*(I_k).
	\]
	In the case that $[0,1]\not \subseteq \bigcup_{k=1}^nI_k$, then \[(\bigcup_{k=1}^nI_k)^c\cap[0,1]=(\bigcap_{k=1}^nI_k^c)\cap[0,1]=\bigcap_{k=1}^n(I_k^c\cap[0,1])\neq\emptyset.\]
	\\We want to show that $\bigcap_{k=1}^nI_k^c\cap[0,1]$ is countable so that $m^*(\bigcap_{k=1}^nI_k^c\cap[0,1])=0$.\\
	Because each $I_k^c\cap[0,1]$ is a closed interval (of irrational numbers), the intersection is also a closed interval (nonempty by assumption); that is, $\bigcap_{k=1}^n(I_k^c\cap[0,1])=[a,b]$ for some $a\le b$.
	Suppose by contradiction that $\bigcap_{k=1}^n(I_k^c\cap[0,1])$ is not countable. 
	Then we have that $a<b$.
	However, by density of the rationals, there exists a rational between $[a,b]$, leading to a contradiction.
	\\Therefore $\bigcap_{k=1}^n(I_k^c\cap[0,1])=\{x\}$, where $x\in\mathbb{Q}^c$, and $\bigcap_{k=1}^n(I_k^c\cap[0,1])$ is countable.
	\\Now we can write
	\begin{align*}
		[0,1]&=(\bigcup_{k=1}^nI_k\cap[0,1])\cup(\bigcap_{k=1}^nI_k^c\cap[0,1])\\
		[0,1]&\subseteq(\bigcup_{k=1}^nI_k\cap[0,1])\cup(\bigcap_{k=1}^nI_k^c\cap[0,1])&&A=B\implies A\subseteq B\text{ and }A\supseteq B\\
		m^*([0,1])&\le m^*((\bigcup_{k=1}^nI_k\cap[0,1])\cup(\bigcap_{k=1}^nI_k^c\cap[0,1]))&&\text{by monotonicity}\\
		m^*([0,1])&\le m^*(\bigcup_{k=1}^nI_k\cap[0,1])+m^*(\bigcap_{k=1}^nI_k^c\cap[0,1])&&\text{by countable subadditivity}\\
		m^*([0,1])&\le m^*(\bigcup_{k=1}^nI_k\cap[0,1])+0&&\text{the outer measure of a countable set is zero}\\
		1&\le m^*(\bigcup_{k=1}^nI_k\cap[0,1])\\
		1&\le m^*(\bigcup_{k=1}^nI_k)&&\text{by monotonicity: }\bigcup_{k=1}^nI_k\cap[0,1]\subseteq[0,1]\\
		1&\le\sum_{k=1}^nm^*(I_k).&&\text{by countable subadditivity}
	\end{align*}
	\item Prove that if $m^*(A)=0$, then $m^*(A\cup B) = m^*(B)$.
	\begin{align*}
		m^*(A\cup B)&\le m^*(A)+m^*(B)&&\text{by countable subadditivity}\\
		m^*(A\cup B)&\le m^*(B)&&\text{because }m^*(A)=0.
	\end{align*}
	Also, we have $B\subseteq A\cup B$, so by monotonicity of outer measure,
	\begin{align*}
		m^*(B)&\le m^*(A\cup B).
	\end{align*}
	Then $m^*(A\cup B)\le m^*(B)$ and $m^*(B)\le m^*(A\cup B)$ imply that $m^*(A\cup B) = m^*(B)$.
	\item Let $A$ and $B$ be bounded sets for which there is an $\alpha >0$ such that $|a-b| \ge \alpha$ for all $a \in A, b \in B$.
	Prove that $m^*(A \cup B) = m^*(A)+m^*(B)$.\\
	By countable subadditivity of outer measure, $m^*(A \cup B) \le m^*(A)+m^*(B)$.\\
	We can see that $A$ and $B$ are disjoint: Suppose by contradiction that $A,B$ are not disjoint. Then there exists a real number $x$ such that $x\in A$ and $x\in B$.
	But then $|x-x|=0<\alpha$, a contradiction.\\
	Let $\epsilon$ such that $\alpha/2>\epsilon>0$. By definition of outer measure and infimum, there exists a countable sequence of open intervals $\{I_k\}_{k=1}^\infty$ such that $(A\cup B)\subseteq\bigcup_{k=1}^\infty I_k$ and 
	\begin{equation}
		m^*(A\cup B)\le\sum_{k=1}^\infty\ell(I_k)<m^*(A\cup B)+\epsilon.\tag{1}	
	\end{equation}
	Now, each $I_k$ is such that $A\cap I_k\neq\emptyset$ or $B\cap I_k\neq\emptyset$, but not both.\\
	To see this, suppose by contradiction that there exists an $I_k$ such that $A\cap I_k\neq\emptyset$ and $B\cap I_k\neq\emptyset$.
	Then there exists $a,b\in I_k$ such that $a\in A$ and $b\in B$.
	Without loss of generality, suppose that these are the closest two elements of $A$ and $B$, and suppose $a<b$. Then the interval $(a,b)$ contains no elements of $A$ or $B$, and $m^*(b-a)\ge\alpha>\alpha/2$.
	This is a contradiction to the fact that $\sum_{k=1}^\infty\ell(I_k)$ is within $\alpha/2$ of $m^*(A\cup B)$.\\
	We can then separate $\{I_k\}_{k=1}^\infty$ into two subsequences $\{(I_A)_i\}_{i=1}^\infty$ and $\{(I_B)_j\}_{j=1}^\infty$ such that $A\subseteq\bigcup_{i=1}^\infty (I_A)_i$ and $B\subseteq\bigcup_{j=1}^\infty (I_B)_j$.
	Then because the sum is uniquely defined independently of the order of terms, $\sum_{k=1}^\infty\ell(I_k)=\sum_{i=1}^\infty\ell((I_A)_i)+\sum_{j=1}^\infty\ell((I_B)_j)$.\\
	Therefore we can write
	\begin{align*}
		m^*(A \cup B)&\le m^*(A)+m^*(B)&&\text{by countable subadditivity of outer measure}\\
		&\le m^*(\bigcup_{i=1}^\infty (I_A)_i)+m^*(\bigcup_{j=1}^\infty (I_B)_j)&&\text{by monotonicity of outer measure}\\
		&\le \sum_{i=1}^\infty\ell((I_A)_i)+\sum_{j=1}^\infty\ell((I_B)_j)&&\text{by countable subadditivity of outer measure}\\
		&=\sum_{k=1}^\infty\ell(I_k)&&\text{rearranging the sum}\\
		&<m^*(A\cup B)+\epsilon&&\text{by (1)}
	\end{align*}
	Therefore for any $\epsilon$,
	\[
		m^*(A \cup B)\le m^*(A)+m^*(B)<m^*(A\cup B)+\epsilon,
	\]
	thus $m^*(A \cup B) = m^*(A)+m^*(B)$.
\end{enumerate}



% 2.3
\section{The $\sigma$-Algebra of Lebesgue Measurable Sets}
\begin{flushleft}
	Outer measure is defined for all sets of real numbers, the outer measure of an interval is its length, outer measure is countably subadditive, and outer measure is translation invariant.
	However, outer measure fails to be countably additive or even finitely additive.
	That is, there exists disjoint sets $A,B$ such that 
	\begin{equation}
		m^*(A\cup B)<m^*(A)+m^*(B).\tag{1}	
	\end{equation}
	We identify a $\sigma$-algebra of sets, called the Lebesgue measurable sets, which contains all intervals and all open sets and has the property that the restriction of the set function outer measure to the collection of Lebesgue measurable sets is countably additive.
	\begin{namedthm*}{Definition}
		A set $E$ is said to be \textbf{measurable} provided for any set $A$,
		\[
			m^*(A)=m^*(A\cap E)+m^*(A\cap E^c).	
		\]		
	\end{namedthm*}  
	We see that the strict inequality (1) cannot occur if one of the sets is measurable:\\
	Suppose $A$ is measurable and $B$ is any set disjoint from $A$.
	\begin{align*}
		m^*(A\cup B)&=m^*([A\cup B]\cap A)+m^*([A\cup B]\cap A^c)&&\text{by definition of $A$ measurable}\\
		&=m^*(A)+m^*([A\cap A^c]\cup [B\cap A^c])&&\text{left: absorbtion, right: distributive property}\\
		&=m^*(A)+m^*(\emptyset\cup [B\setminus A])&&\text{complement and def of set difference}\\
		&=m^*(A)+m^*(B).&&\text{identity of union and set difference of disjoint sets}\\
	\end{align*}
	
	Suppose we want to prove that a set $E$ is measurable.\\
	We already have that for any set $A$,
	\begin{align*}
		m^*(A)&=m^*([A\cap E]\cup[A\cap E^c])&&\text{by set properties}\\
		m^*(A)&\le m^*(A\cap E)+m^*(A\cap E^c).&&\text{by subadditivity of outer measure}\\
	\end{align*}
	Therefore to show that $E$ is measurable, it suffices to show the other inequality:
	\begin{equation}
		m^*(A)\ge m^*(A\cap E)+m^*(A\cap E^c).\tag{2}
	\end{equation}
	This inequality holds trivially if $m^*(A)=\infty$. Therefore we need only prove (2) for sets $A$ that have finite outer measure.

	\begin{namedthm*}{Proposition 4}
		Any set of outer measure zero is measurable. In particular, any countable set is measurable.
	\end{namedthm*}
	\begin{proof}
		Let $E$ be such that $m^*(E)=0$. Let $A$ be any set.\\
		\begin{itemize}
			\item $A\cap E\subseteq E$
			\item $A\cap E^c\subseteq A$
		\end{itemize}
		By monotonicity of outer measure,
		\begin{align*}
			m^*(A\cap E)&\le m^*(E)=0\\
			m^*(A\cap E^c)&\le m^*(A)
		\end{align*}
		Therefore
		\begin{align*}
			m^*(A)&\ge m^*(A\cap E^c)+0\\
			m^*(A)&\ge m^*(A\cap E^c)+m^*(A\cap E).
		\end{align*}
	\end{proof}

	Every open set is the disjoint union of a countable collection of open intervals. Every interval is measurable, and the countable union of measurable sets is measurable, so all open sets are measurable.
	By complement, all closed sets are measurable. In the same way, all $G_\delta$ sets and all $F_\sigma$ sets are measurable.
	\par\medskip
	The intersection of all the $\sigma$-algebras of subsets of $\mathbb{R}$ that contain the open sets is a $\sigma$-algebra called the \textbf{Borel $\sigma$-algebra}, members of this collection are called \textbf{Borel sets}.
	That is, the Borel sigma-algebra is the sigma-algebra generated by the open sets.

	\begin{namedthm*}{Lemma 1}
		The set of all subsets of $X$, $\mathcal{P}(X)$ (or $2^X$), is a $\sigma$-algebra of subsets of $X$.
	\end{namedthm*}
	\begin{proof}
		Let $X$ be any set.
		\begin{enumerate}[label=(\roman*),align=left]   
			\item $X\in\mathcal{P}(X)$.
			\item if $A\in\mathcal{P}(X)$, then $A^c=X\setminus A = \{x\in X\ |\ x\notin A\}\in\mathcal{P}(X)$.
			\item if $A_i\in\mathcal{P}(X)$, then $\bigcup_{i=1}^\infty A_i = \{x\in X\ |\ x\in A_i\text{ for some }i\}$.
		\end{enumerate}
	\end{proof}

	\begin{namedthm*}{Lemma 2}
		Given any collection of $\sigma$-algebras $\{\mathcal{F}_\alpha\}_{\alpha\in\mathcal{A}}$ of $X$, the intersection $\bigcap_{\alpha\in\mathcal{A}}\mathcal{F}_\alpha$ is also a $\sigma$-algebra.
	\end{namedthm*}
	\begin{proof}
		Let $X$ be any set.
		\begin{enumerate}[label=(\roman*),align=left]   
			\item $X\in\mathcal{F}_\alpha,\forall\alpha\in\mathcal{A}\implies X\in\bigcap_{\alpha\in\mathcal{A}}\mathcal{F}_\alpha$.
			\item $A\in\bigcap_{\alpha\in\mathcal{A}}\mathcal{F}_\alpha\implies A\in\mathcal{F}_\alpha,\forall\alpha\in\mathcal{A}\implies A^c\in\mathcal{F}_\alpha,\forall\alpha\in\mathcal{A}\implies A^c\in\bigcap_{\alpha\in\mathcal{A}}\mathcal{F}_\alpha$.
			\item $A_i\in\bigcap_{\alpha\in\mathcal{A}}\mathcal{F}_\alpha\implies A_i\in\mathcal{F}_\alpha,\forall\alpha\in\mathcal{A}\implies \bigcup_{i=1}^\infty A\in\mathcal{F}_\alpha,\forall\alpha\in\mathcal{A}\implies \bigcup_{i=1}^\infty A\in\bigcap_{\alpha\in\mathcal{A}}\mathcal{F}_\alpha$.
		\end{enumerate}
	\end{proof}

	\begin{namedthm*}{Theorem}
		Given any collection $\mathcal{C}$ of subsets of $X$, there exists a smallest $\sigma$-algebra containing $\mathcal{C}$.
		(This is called the $\sigma$-algebra generated by $\mathcal{C}$.)
	\end{namedthm*}
	\begin{proof}
		Consider $S=\{\mathcal{F}\ |\ \mathcal{C}\subseteq\mathcal{F},\mathcal{F}\text{ is a }\sigma\text{-algebra of }X\}$.\\
		Now, $S$ is nonempty because $\mathcal{C}\in\mathcal{P}(X)$ and by Lemma 1, $\mathcal{P}(X)$ is a $\sigma$-algebra of $X$; therefore $\mathcal{P}(X)\in S$.\\
		Consider $\bigcap S$, the intersection of all the elements of $S$.\\
		\begin{enumerate}
			\item By Lemma 2, $\bigcap S$ is a $\sigma$-algebra,
			\item $\mathcal{C}\in\mathcal{F},\forall\mathcal{F}\in S\implies\mathcal{C}\in\bigcap S$, so $\bigcap S$ is a $\sigma$-algebra that contains $\mathcal{C}$,
			\item  $\bigcap S\subseteq \mathcal{F}$ for any $\mathcal{F}\in S$ by def of intersection, so $\bigcap S$ is the smallest $\sigma$-algebra containing $\mathcal{C}$.
		\end{enumerate}
	\end{proof}

	\begin{namedthm*}{Proposition 10}
		The translate of a measurable set is measurable.		
	\end{namedthm*}
	\begin{proof}
		Let $E$ be measurable, let $A$ be any set, and let $y$ be any real number.
		\\First we need to see that 
		\begin{align*}
			(A\cap [E+y])-y&=\{x:x\in A,\text{ and }x\in E+y\}-y=\{x:x\in A-y\text{ and }x\in E\}=[A-y]\cap E\\
			(A\cap [E+y]^c)-y&=\{x:x\in A,\text{ and }x\notin E+y\}-y=\{x:x\in A-y\text{ and }x\notin E\}=[A-y]\cap E^c
		\end{align*}
		Now, we have
		\begin{align*}
			m^*(A)&=m^*(A-y)&&\text{outer measure is translation invariant}\\
			&=m^*([A-y]\cap E)+m^*([A-y]\cap E^c)&&\text{because $E$ is measurable}\\
			&=m^*(A\cap [E+y]-y)+m^*(A\cap [E+y]^c-y)&&\text{by above}\\
			&=m^*(A\cap [E+y])+m^*(A\cap [E+y]^c).&&\text{outer measure is translation invariant}
		\end{align*}
		Therefore $E+y$ is measurable.
	\end{proof}

\end{flushleft}
\begin{center}
	\textbf{PROBLEMS}
\end{center}
\begin{enumerate}
	\setcounter{enumi}{10}
	\item Prove that if a $\sigma$-algebra of subsets of $\mathbb{R}$ contains intervals of the form $(a,\infty)$, then it contains all intervals.\\
	Let $\mathcal{M}$ be a $\sigma$-algebra of subsets of $\mathbb{R}$.\\
	Suppose that for any real number $a$, the interval $(a,\infty)\in\mathcal{M}$.\\
	For any real number $b$, because $\mathcal{M}$ is closed under complements,
	\begin{align*}
		(b,\infty)\in\mathcal{M}&\implies(b,\infty)^c=(-\infty,b]\in\mathcal{M}.
	\end{align*}
	For any natural number $n$, because $\mathcal{M}$ is closed under intersections:
	\begin{align*}
		(a-\frac{1}{n},\infty), (-\infty,b]\in\mathcal{M}&\implies(a-\frac{1}{n},\infty)\cap(-\infty,b]=(a-\frac{1}{n},b]\in\mathcal{M},\\
		(a,\infty), (-\infty,b-\frac{1}{n}]\in\mathcal{M}&\implies(a,\infty)\cap(-\infty,b-\frac{1}{n}]=(a,b-\frac{1}{n}]\in\mathcal{M}.
	\end{align*}
	Because $\mathcal{M}$ is closed under countable intersections and countable unions:
	\begin{align*}
		\text{for any $n\in\mathbb{N}$, }(a-\frac{1}{n},b]\in\mathcal{M}\implies\bigcap_{n=1}^\infty(a-\frac{1}{n},b]=[a,b]\in\mathcal{M},\\
		\text{for any $n\in\mathbb{N}$, }(a,b-\frac{1}{n}]\in\mathcal{M}\implies\bigcup_{n=1}^\infty(a,b-\frac{1}{n}]=(a,b)\in\mathcal{M}.
	\end{align*}
	In short, for any real numbers $a,b$, we have
	\begin{align*}
		[a,b]&=\bigcap_{n=1}^\infty(a-\frac{1}{n},b]=\bigcap_{n=1}^\infty(a-\frac{1}{n},\infty)\cap(-\infty,b]=\bigcap_{n=1}^\infty(a-\frac{1}{n},\infty)\cap(b,\infty)\\
		(a,b)&=\bigcup_{n=1}^\infty(a,b-\frac{1}{n}]=\bigcup_{n=1}^\infty(a,\infty)\cap(-\infty,b-\frac{1}{n}]=\bigcup_{n=1}^\infty(a,\infty)\cap(b-\frac{1}{n},\infty)
	\end{align*}
	The construction of intervals of the form $(a,b]$ and $[a,b)$ is similar.
	\item Show that every interval is a Borel set.\\
	Because any interval of the form $(a,\infty)$ is open, $(a,\infty)$ is a Borel set; i.e., it is a member of the Borel sigma-algebra.
	By the previous problem 11, any sigma-algebra that contains intervals of the form $(a,\infty)$ contains all intervals. 
	Therefore the Borel sigma-algebra contains all intervals and thus all intervals are Borel sets.
	\item Show that 
	\begin{enumerate}[label=(\roman*),align=left]                                                                                                         
        \item the translate of an $F_\sigma$ set is also $F_\sigma$,\\
        Let $F$ be an $F_\sigma$ set, that is, $F=\bigcup_{n=1}^\infty F_n$, with $F_n$ closed. \\
		For any real number $y$,
        \begin{align*}
			F+y&=(\bigcup_{n=1}^\infty F_n)+y\\
			&=\{x:x\in F_n\text{ for some }n\}+y\\
			&=\{x:x\in F_n+y\text{ for some }n\}\\
			&=\bigcup_{n=1}^\infty (F_n+y)
		\end{align*}
		The translate of any closed set is closed, so this is still an $F_\sigma$ set.
        \item the translate of a $G_\delta$ set is also $G_\delta$,\\
        Let $\mathcal{O}$ be a $G_\delta$ set, that is, $\mathcal{O}=\bigcap_{n=1}^\infty \mathcal{O}_n$, with $\mathcal{O}_n$ open.\\
        For any real number $y$,
		\begin{align*}
			\mathcal{O}+y&=(\bigcap_{n=1}^\infty \mathcal{O}_n)+y\\
			&=\{x:x\in \mathcal{O}_n\text{ for all }n\}+y\\
			&=\{x:x\in \mathcal{O}_n+y\text{ for all }n\}\\
			&=\bigcap_{n=1}^\infty (\mathcal{O}_n+y)
		\end{align*}
		The translate of any open set is open, so this is still a $G_\delta$ set.
        \item the translate of a set of measure zero also has measure zero.\\
        Let $E$ be a set of measure zero. That is, $m^*(E)=0$.\\
		For any $\epsilon>0$, by definition of infimum, there exists a countable collection of open intervals $\{I_k\}_{k=1}^\infty$ such that  
		\[
		m^*(E)\le\sum_{k=1}^\infty \ell(I_k) <m^*(E)+\epsilon,
		\]
		Thus because the outer measure is zero, 
		\[
			\sum_{k=1}^\infty \ell(I_k) <\epsilon.
		\]
		Now, for any real number $y$,
		\[
			E+y \subseteq (\bigcup_{k=1}^\infty I_k)+y=\bigcup_{k=1}^\infty (I_k+y).
		\]
		By monotonicity of outer measure, 
		\[
			m^*(E+y)\le \sum_{k=1}^\infty\ell(I_k+y)=\sum_{k=1}^\infty\ell(I_k)<\epsilon.
		\]
		Therefore $m^*(E+y)=0$.
    \end{enumerate}
    \item Show that if a set $E$ has positive outer measure, then there is a bounded subset of $E$ that also has positive outer measure.\\
    Suppose $E$ has positive outer measure.\\
	If $E$ is bounded, then clearly $E$ itself is a bounded subset of $E$ with positive outer measure.\\
	If $E$ is unbounded:\\
	First, we can partition the real numbers:
	\[
		\mathbb{R}=\bigcup_{n\in\mathbb{Z}}[n,n+1)	
	\]
	Then we have that 
	\[
		E=E\cap \mathbb{R}=E\cap(\bigcup_{n\in\mathbb{Z}}[n,n+1))=\bigcup_{n\in\mathbb{Z}}E\cap[n,n+1).
	\]
	By countable subadditivity of outer measure,
	\begin{align*}
		0<m^*(E)=m^*(\bigcup_{n\in\mathbb{Z}}E\cap[n,n+1))\le\sum_{n\in\mathbb{Z}}m^*(E\cap[n,n+1))
	\end{align*}
	Then there exists an $n\in \mathbb{Z}$ such that $m^*(E\cap[n,n+1))>0$, else we reach a contradiction.
	Therefore we have $E\cap[n,n+1)\subseteq E$ that is bounded and has positive outer measure.
    \item Show that if $E$ has finite measure and $\epsilon>0$, then $E$ is the disjoint union of a finite number of measurable sets, each of which has measure at most $\epsilon$.\\
    (We are letting $E$ be a measurable set because we are talking about measure specifically, not outer measure.)\\
	If $E$ is countable, then $E$ has measure zero and $E$ itself is the measurable set whose measure is less than any $\epsilon$: $m(E)=0<\epsilon$.
	In fact, if $E$ has measure zero then the conclusion is trivial.\\
	Suppose $E$ has positive measure.\\
	Fix $\epsilon>0$.\\
	In the case that $E$ is not bounded, there exists an $M$ such that 
	\[
		m(E\setminus[-M,M])<\epsilon.\ (\star)
	\]
	$(\star)$: To prove this we can partition $\mathbb{R}$: 
	\[
		\mathbb{R}=\bigcup_{n=0}^\infty\biggl([-(n+1),-n)\cup(n,n+1]\biggr)=\bigcup_{n=0}^\infty I_n.
	\]
	That is, $I_0=[-1,1],I_1=[-2,-1)\cup(1,2],I_2=[-3,-2)\cup(2,3],\cdots$\\
	Therefore $E=E\cap\mathbb{R}=E\cap(\bigcup_{n=0}^\infty I_n)=\bigcup_{n=0}^\infty (E\cap I_n)$.\\
	By countable additivity of measure, and the fact that $E$ has finite measure,
	\[
		m(E)=m(\bigcup_{n=0}^\infty (E\cap I_n))=\sum_{n=0}^\infty m(E\cap I_n)<\infty.
	\]
	Thus we have a sequence of partial sums that converges so there exists an index $M$ such that 
	\[
		\sum_{n=M}^\infty m(E\cap I_n)=\biggl|\sum_{n=0}^\infty m(E\cap I_n)-\sum_{n=0}^{M-1} m(E\cap I_n)\biggr|<\epsilon.
	\]
	We see that $m(E\setminus[-M,M]) = m(\bigcup_{n=M}^\infty (E\cap I_n))=\sum_{n=M}^\infty m(E\cap I_n)<\epsilon$.
	\\
	Therefore $E=(E\cap[-M,M])\cup(E\cap[-M,M]^c))$, a disjoint union, and $m(E\cap[-M,M]^c)<\epsilon$, so we need only worry now about $E\cap[-M,M]$.\\
	Else if $E$ is bounded, then there exists an $M$ such that $E\subseteq[-M,M]$, and $E=E\cap[-M,M]$.\\
    Now, for this $\epsilon$, we can partition the real numbers into a countable collection of disjoint measurable intervals $I_k$ of the form $[x,x+\epsilon)$.\\
	When we choose a natural number $l$ such that $\frac{2M}{\epsilon}<l$, we get $M<-M+l\epsilon$ so that 
	\[
		E\cap[-M,M]\subseteq [-M,M]\subseteq\bigcup_{k=1}^l[-M+(k-1)\epsilon,-M+k\epsilon)=\bigcup_{k=1}^l I_k.
	\]
	Then 
	\[
		E\cap[-M,M]=E \cap (\bigcup_{k=1}^l I_k) = \bigcup_{k=1}^l (E \cap I_k).
	\]
	Thus $E$ is the union of a finite number of disjoint measurable sets, each of which has measure at most $\epsilon$.\\
	(If $E$ is not bounded, $E=(\bigcup_{k=1}^l (E \cap I_k))\cup(E\setminus[-M,M])$, which still satisfies the conclusion.)
\end{enumerate}

% 2.4
\section{Outer and Inner Approximation of Lebesgue Measurable Sets}
\begin{flushleft}
	
	Measurable sets possess the following \textbf{excision property}: If $A$ is a measurable set of finite outer measure that is contained in $B$, then 
	\[
		m^*(B\setminus A)=m^*(B)-m^*(A).	
	\]
	This holds because
	\[
		m^*(B)=m^*(B\cap A)+m^*(B\cap A^c)=m^*(A)+m^*(B\cap A^c).	
	\]

\end{flushleft}
\begin{namedthm*}{Theorem 11}
	Let $E$ be any set of real numbers. Then each of the following four assertions is equivalent to the measurability of $E$.\\
	(Outer Approximation by Open Sets and $G_\delta$ sets)
	\begin{enumerate}[label=(\roman*),align=left]
        \item For each $\epsilon>0$, there is an open set $\mathcal{O}$ containing $E$ for which $m^*(\mathcal{O}\setminus E)<\epsilon$.
        \item There is a $G_\delta$ set $G$ containing $E$ for which $m^*(G\setminus E)=0$. 
    \end{enumerate}
	(Inner Approximation by Closed Sets and $F_\sigma$ sets)
	\begin{enumerate}[label=(\roman*),align=left]
        \setcounter{enumi}{2}
		\item For each $\epsilon>0$, there is a closed set $F$ contained in $E$ for which $m^*(E\setminus F)<\epsilon$.
        \item There is a $F_\sigma$ set $F$ contained in $E$ for which $m^*(E\setminus F)=0$.
    \end{enumerate}
\end{namedthm*}
\begin{proof}
	($E$ is measurable $\implies$ (i)):\\
	Assume $E$ is measurable and fix $\epsilon>0$.\\
	Case: $m^*(E)<\infty$:\\
	By definition of outer measure and infimum, there exists a countable collection of intervals $\{I_k\}_{k=1}^\infty$ such that $E\subseteq\bigcup_{k=1}^\infty I_k$ and
	\[
		m^*(E)\le\sum_{k=1}^\infty\ell(I_k)<m^*(E)+\epsilon.	
	\]
	Defining $\mathcal{O}=\bigcup_{k=1}^\infty I_k$, we see that $\mathcal{O}$ is an open set containing $E$.
	By subadditivity of outer measure,
	\begin{align*}
		m^*(\mathcal{O})=m^*(\bigcup_{k=1}^\infty I_k)\le\sum_{k=1}^\infty\ell(I_k)<m^*(E)+\epsilon,
	\end{align*}
	so that 
	\begin{align*}
		m^*(\mathcal{O})-m^*(E)<\epsilon.
	\end{align*}
	Because $E$ is measurable, has finite outer measure, and is contained in $\mathcal{O}$, we have the excision property:
	\[
		m^*(\mathcal{O}\setminus E)=m^*(\mathcal{O})-m^*(E)<\epsilon.
	\]
	Case: $m^*(E)=\infty$:\\
	Then $E$ may be expressed as the disjoint union of a countable collection $\{E_k\}_{k=1}^\infty$ of measurable sets, each of which has finite outer measure
	(See Problems 14 and 15 for an example of partitioning $\mathbb{R}$).
	Now, for each index $k$, because each $E_k$ is measurable and has finite outer measure, we showed above that there exists an open set $\mathcal{O}_k$ containing $E_k$ for which $m^*(\mathcal{O}_k\setminus E_k)<\epsilon/2^k$.
	The set $\mathcal{O}=\bigcup_{k=1}^\infty \mathcal{O}_k$ is open, it contains $E$ (because $E=\bigcup_{k=1}^\infty E_k \subseteq \bigcup_{k=1}^\infty \mathcal{O}_k =\mathcal{O}$), and we have $E\supseteq E_k \implies E^c\subseteq E_k^c$, so that
	\[
		\mathcal{O}\setminus E=\mathcal{O}\cap E^c = (\bigcup_{k=1}^\infty \mathcal{O}_k)\cap E^c = \bigcup_{k=1}^\infty (\mathcal{O}_k\cap E^c)\subseteq\bigcup_{k=1}^\infty (\mathcal{O}_k\cap E_k^c)=\bigcup_{k=1}^\infty (\mathcal{O}_k\setminus E_k).
	\]
	Therefore by monotonicity and subadditivity of outer measure,
	\[
		m^*(\mathcal{O}\setminus E)	\le m^*(\bigcup_{k=1}^\infty (\mathcal{O}_k\setminus E_k))\le\sum_{k=1}^\infty m^*(\mathcal{O}_k\setminus E_k)<\sum_{k=1}^\infty\epsilon/2^k=\epsilon.
	\]
	Thus property (i) holds for $E$.\\
	\\((i) $\implies$ (ii)):\\
	Now, assume property (i) holds for $E$. 
	Then for each natural number $k$, there exists an open set $\mathcal{O}_k$ that contains $E$ for which $m^*(\mathcal{O}_k\setminus E)<1/k$.
	\\Define $G=\bigcap_{k=1}^\infty\mathcal{O}_k$ so that $E\subseteq\mathcal{O}_k$ for all $k\implies E\subseteq\bigcap_{k=1}^\infty\mathcal{O}_k=G$.
	Then $G$ is a $G_\delta$ set that contains $E$.\\
	Then because for all $k$, $G\subseteq\mathcal{O}_k\implies G\setminus E\subseteq\mathcal{O}_k\setminus E$, by monotonicity of outer measure,
	\[
		m^*(G\setminus E)\subseteq m^*(\mathcal{O}_k\setminus E)<1/k.
	\]
	Thus $m^*(G\setminus E)=0$, and (ii) holds.\\
	\\((ii) $\implies$ $E$ is measurable):\\
	Assume property (ii) holds for $E$.\\
	We can write
	\begin{align*}
		E&=G\cap E\\
		&=\emptyset\cup(G\cap E)\\
		&=(G\cap G^c)\cup(G\cap E)\\
		&=G\cap (G^c\cup E)\\
		&=G\cap (G\cap E^c)^c\\
		&=G\cap (G\setminus E)^c.
	\end{align*}
	Now, $m^*(G\setminus E)=0$, and any set of measure zero is measurable, so $G\setminus E$ is measurable and also $(G\setminus E)^c$ is measurable by complement.
	Also, $G$ is a $G_\delta$ set, and all $G_\delta$ sets are measurable. Finally, the intersection of measurable sets is measurable so $G\cap (G\setminus E)^c$ is measurable.
	Thus $E$ is measurable.
\end{proof}

\begin{center}
	\textbf{PROBLEMS}
\end{center}
\begin{enumerate}
	\setcounter{enumi}{15}
	\item Complete the proof of Theorem 11 by showing that measurability is equivalent to (iii) and also equivalent to (iv).\\
	($E$ is measurable $\implies$ (iii)):\\
	Fix $\epsilon>0$.
	Suppose $E$ is measurable. Then $E^c$ is also measurable.
	Also, stating that $E^c$ is measurable is equivalent to property (i), that is, there exists an open set $\mathcal{O}$ containing $E^c$ such that $m^*(\mathcal{O}\setminus E^c)<\epsilon$.\\
	Then because $\mathcal{O}$ is open, we can define the closed set $F=\mathcal{O}^c$, and we have that 
	\[
		\mathcal{O}\setminus E^c = \mathcal{O}\cap E = F^c\cap E = E\setminus F,
	\]
	and $E^c\subseteq \mathcal{O}\implies E\supseteq \mathcal{O}^c=F$.
	Therefore $F$ is a closed set contained in $E$ for which $m^*(E\setminus F)=m^*(\mathcal{O}\setminus E^c)<\epsilon$, and (iii) holds.\\
	\\((iii) $\implies$ (iv)):\\
	Suppose that property (iii) holds for $E$.
	Then for each natural number $k$, there exists a closed set $F_k$ contained in $E$ for which $m^*(E\setminus F_k)<1/k$.
	\\Then defining $F=\bigcup_{k=1}^\infty F_n$, we have that $F_k\subseteq E,\forall k\implies F=\bigcup_{k=1}^\infty F_k \subseteq E$.
	Then $F$ is an $F_\sigma$ set that is contained in $E$.
	Then $F\supseteq F_k\implies F^c\subseteq F_k^c$ and thus $E\cap F^c\subseteq E\cap F_k^c$ and $E\setminus F\subseteq E\setminus F_k$.
	By monotonicity of outer measure, for all $k$, we have 
	\[
		m^*(E\setminus F)\le m^*(E\setminus F_k)<1/k.	
	\]
	Therefore $m^*(E\setminus F)=0$, and (iv) holds.\\
	\\((iv) $\implies$ $E$ is measurable):\\
	Suppose that property (iv) holds for $E$.\\
	We can write
	\begin{align*}
		E&=E\cap\mathbb{R}\\
		&=[F\cup E]\cap[F\cup F^c]\\
		&=F\cup [E\cap F^c]\\
		&=F\cup [E\setminus F].
	\end{align*}
	Now, $m^*(E\setminus F)=0$ implies $E\setminus F$ is measurable because all sets of measure zero are measurable.
	Also, $F$ is an $F_\sigma$ set, which is measurable. Therefore $F\cup [E\setminus F]$, the intersection of measurable sets, is measurable.
	Thus $E$ is measurable.
	\item Show that a set $E$ is measurable iff for each $\epsilon>0$, there is a closed set $F$ and open set $\mathcal{O}$ for which $F\subseteq E\subseteq \mathcal{O}$ and $m^*(\mathcal{O}\setminus F)<\epsilon$.\\
	Let $E$ be a set, and let $\epsilon>0$.\\
	(This case we assuming $E$ has finite measure to assume excision, maybe proof not complete)\\
	$(\implies)$ Suppose $E$ is measurable.\\
	Then by Theorem 11 (i), (iii), there is an open set $\mathcal{O}$ containing $E$ for which $m^*(\mathcal{O}\setminus E)<\epsilon/2$, and a closed set $F$ contained in $E$ for which $m^*(E\setminus F)<\epsilon/2$.
	That is, $F\subseteq E\subseteq \mathcal{O}$.\\
	By excision, $m^*(E\setminus F)=m^*(E)-m^*(F)$, and we can write
	\begin{align*}
		m^*(E)-m^*(F)&<\epsilon/2\\
		m^*(E)&<m^*(F)+\epsilon/2\\
		-m^*(E)&>-m^*(F)-\epsilon/2
	\end{align*}
	Also by excision, we have  $m^*(\mathcal{O}\setminus E)=m^*(\mathcal{O})-m^*(E)$, and 
	\[
		m^*(\mathcal{O})-m^*(F)-\epsilon/2<m^*(\mathcal{O})-m^*(E)<\epsilon/2
	\]
	Therefore $m^*(\mathcal{O}\setminus F)=m^*(\mathcal{O})-m^*(F)<\epsilon$.\\
	$(\impliedby)$ Suppose there is a closed set $F$ and open set $\mathcal{O}$ for which $F\subseteq E\subseteq \mathcal{O}$ and $m^*(\mathcal{O}\setminus F)<\epsilon$.\\
	By excision and monotonicity of outer measure, we have that 
	\[
		m^*(E\setminus F)=m^*(E)-m^*(F)\le m^*(\mathcal{O})-m^*(F)=m^*(\mathcal{O}\setminus F)<\epsilon.
	\]
	Therefore we have a closed set $F$ contained in $E$ for which $m^*(E\setminus F)<\epsilon$, i.e., proposition (iii), which implies that $E$ is measurable.
	\item Let $E$ have finite outer measure. Show that there is a $G_\delta$ set $G\supseteq E$ with $m(G)=m^*(E)$.
	Show that $E$ is measurable iff there is an $F_\sigma$ set $F \subseteq E$ with $m(F)=m^*(E)$.\\
	Let $E$ be a set with finite outer measure.\\
	Then for each natural number $k$, by definition of infimum, there exists a countable collection of open intervals $\{(I_k)_n\}_{n=1}^\infty$ whose union contains $E$ for which 
	\[
		\sum_{n=1}^\infty \ell((I_k)_n) <m^*(E)+1/k.
	\]
	Now, $\mathcal{O}_k=\bigcup_{n=1}^\infty (I_k)_n$ is an open set, and we can define $G=\bigcap_{k=1}^\infty\mathcal{O}_k$ so that $E\subseteq\mathcal{O}_k$ for all $k\implies E\subseteq\bigcap_{k=1}^\infty\mathcal{O}_k=G$.
	Then $G$ is a $G_\delta$ set that contains $E$. Because $G\subseteq\mathcal{O}_k$ for all $k$, by monotonicity,
	\[
		m^*(G)\le m^*(\mathcal{O}_k)=m^*(\bigcup_{n=1}^\infty (I_k)_n)\le\sum_{n=1}^\infty \ell((I_k)_n) <m^*(E)+1/k.
	\]
	Then we have $m^*(G)<m^*(E)+1/k$ for any natural number $k$, which implies $m^*(G)\le m^*(E)$. \\
	Also, by monotonicity, $E\subseteq G\implies m^*(E)\le m^*(G)$.\\
	Therefore $m^*(G)=m^*(E)$.\\
	\\Let $E$ be a set with finite outer measure.\\
	$(\implies)$ Suppose that $E$ is measurable.\\
	By Theorem 11 (iv), there is an $F_\sigma$ set $F$ contained in $E$ for which $m^*(E\setminus F)=0$. Because $E$ has finite outer measure, then $F$ has finite outer measure by monotonicity of outer measure.
	Then by excision, we have $m^*(E)-m^*(F)=m^*(E\setminus F)=0$, which implies $m^*(E)=m^*(F)$.\\
	$(\impliedby)$ Suppose there is an $F_\sigma$ set $F \subseteq E$ with $m(F)=m^*(E)$.\\
	Then $0=m^*(E)-m^*(F)$. Because $E$ has finite outer measure, then $F$ has finite outer measure by monotonicity of outer measure.
	Therefore by excision we have $0=m^*(E)-m^*(F)=m^*(E\setminus F)$ and Theorem 11 (iv) holds, which implies that $E$ is measurable.
	\item Let $E$ have finite outer measure.
	Show that if $E$ is not measurable, then there is an open set $\mathcal{O}$ containing $E$ that has finite outer measure and for which 
	\[m^*(\mathcal{O}\setminus E)>m^*(\mathcal{O})-m^*(E).\]
	Suppose $E$ is not measurable. 
	However, suppose by contradiction that for all open sets $\mathcal{O}$ containing $E$ that have finite outer measure, we have $m^*(\mathcal{O}\setminus E) \le m^*(\mathcal{O})-m^*(E)$.
	\\Let $\epsilon>0$.
	By definition of outer measure, there exists a countable collection of open intervals $\{I_k\}$ whose union contains $E$ and 
	\[
		\sum_{k=1}^\infty \ell(I_k)<m^*(E)+\epsilon.
	\]
	We can define $\mathcal{O}:=\bigcup_{k=1}^\infty I_k$, which is an open set that contains $E$, and by subadditivity of outer measure, we have that
	\[
		m^*(\mathcal{O})=m^*(\bigcup_{k=1}^\infty I_k)\le\sum_{k=1}^\infty \ell(I_k)<m^*(E)+\epsilon
	\]
	Therefore $m^*(\mathcal{O})-m^*(E)<\epsilon$, and $\mathcal{O}$ has finite outer measure.\\
	By assumption, we have that $m^*(\mathcal{O}\setminus E) \le m^*(\mathcal{O})-m^*(E)<\epsilon$.
	However, this means that we have an open set $\mathcal{O}$ containing $E$ for which $m^*(\mathcal{O}\setminus E) <\epsilon$, Theorem 11 (i), which is equivalent to saying that $E$ is measurable, which is a contradiction.
	\item (Lebesgue). Let $E$ have finite outer measure. Show that $E$ is measurable iff for each open, bounded interval $(a,b)$,
	\[
		b-a=m^*((a,b)\cap E)+m^*((a,b)\setminus E).
	\]
	Let $E$ be a set of finite outer measure.\\
	$(\implies)$ Suppose that $E$ is measurable.\\
	Then for any interval $(a,b)$, we have 
	\[
		b-a=\ell((a,b))=m^*((a,b))=m^*((a,b)\cap E)+m^*((a,b)\cap E^c)=m^*((a,b)\cap E)+m^*((a,b)\setminus E).
	\]
	$(\impliedby)$ Suppose that for each open, bounded interval $(a,b)$, we have $b-a=m^*((a,b)\cap E)+m^*((a,b)\setminus E)$.\\
	Then we have
	\[
		m^*((a,b))=\ell((a,b))=b-a=m^*((a,b)\cap E)+m^*((a,b)\setminus E)=m^*((a,b)\cap E)+m^*((a,b)\cap E^c).
	\]
	(This is only proved for any open interval; measurability of $E$ implies this is true for any set)
	\item Use property (ii) of Theorem 11 as the primitive definition of a measurable set and prove that the union of two measurable sets is measurable. Then do the same for property (iv).\\
	\\(ii) Let $E$ be any set of real numbers. Define $E$ to be measurable if there is a $G_\delta$ set $G$ containing $E$ for which $m^*(G\setminus E)=0$.\\
	Let $A$ and $B$ be two measurable sets under this definition.
	Then there exist $G_\delta$ sets $G_A,G_B$ containing $A,B$ respectively for which $m^*(G_A\setminus A)=0$ and $m^*(G_B\setminus B)=0$.\\
	Now, by definition of $G_\delta$ set:
	\begin{align*}
		G_A &= \bigcap_{k=1}^\infty \mathcal{O}_k,\text{ for }\mathcal{O}_k\text{ open}\\
		G_B &= \bigcap_{n=1}^\infty \mathcal{U}_n,\text{ for }\mathcal{U}_n\text{ open}
	\end{align*}
	Therefore 
	\begin{align*}
		G_A\cup G_B&=(\bigcap_{k=1}^\infty \mathcal{O}_k)\cup(\bigcap_{n=1}^\infty \mathcal{U}_n)\\
		&=\bigcap_{k=1}^\infty (\mathcal{O}_k\cup(\bigcap_{n=1}^\infty \mathcal{U}_n)\\
		&=\bigcap_{k=1}^\infty (\bigcap_{n=1}^\infty(\mathcal{O}_k\cup \mathcal{U}_n))
	\end{align*}
	For each $k,n$ pair, $\mathcal{O}_k\cup \mathcal{U}_n$ is an open set, so $G_A\cup G_B$ is a countable intersection of open sets and thus a $G_\delta$ set.
	Also, $G_A\supseteq A$ and $G_B\supseteq B$ imply that $G_A\cup G_B\supseteq A\cup B$, so $G_A\cup G_B$ is a $G_\delta$ set that contains $A\cup B$.\\
	We can write
	\begin{align*}
		(G_A\cup G_B)\setminus(A\cup B)&=(G_A\cup G_B)\cap(A\cup B)^c\\
		&=(G_A\cup G_B)\cap(A^c\cap B^c)\\
		&=[G_A\cap(A^c\cap B^c)]\cup[G_B\cap(A^c\cap B^c)]\\
		&=[G_A\cap A^c\cap B^c]\cup[G_B\cap B^c\cap A^c]\\
		&\subseteq[G_A\cap A^c]\cup[G_B\cap B^c]\\
		&\subseteq[G_A\setminus A]\cup[G_B\setminus B].\\
	\end{align*}
	By monotonicity of outer measure and subadditivity, 
	\begin{align*}
		m^*((G_A\cup G_B)\setminus(A\cup B))&\le m^*([G_A\setminus A]\cup[G_B\setminus B])\\
		&\le m^*(G_A\setminus A)+m^*(G_B\setminus B)\\
		&=0.
	\end{align*}
	Therefore $A\cup B$ is measurable.\\
	\\(iv) Let $E$ be any set of real numbers. Define $E$ to be measurable if there is an $F_\sigma$ set $F$ contained in $E$ for which $m^*(E\setminus F)=0$.\\
	Let $A$ and $B$ be two measurable sets under this definition.
	Then there exist $F_\sigma$ sets $F_A,F_B$ contained in $A,B$ respectively for which $m^*(A\setminus F_A)=0$ and $m^*(B\setminus F_B)=0$.\\
	Now, by definition of $F_\sigma$ set:
	\begin{align*}
		F_A &= \bigcup_{k=1}^\infty I_k,\text{ for }I_k\text{ closed}\\
		F_B &= \bigcup_{n=1}^\infty J_n,\text{ for }J_n\text{ closed}
	\end{align*}
	Therefore 
	\begin{align*}
		F_A\cup F_B&=(\bigcup_{k=1}^\infty I_k)\cup(\bigcup_{n=1}^\infty J_n),
	\end{align*}
	which is clearly a countable union of closed sets, so $F_A\cup F_B$ is an $F_\sigma$ set.
	Also, $F_A\subseteq A$ and $F_B\subseteq B$ imply that $F_A\cup F_B\subseteq A\cup B$, so $F_A\cup F_B$ is an $F_\sigma$ set that is contained in $A\cup B$.\\
	We can write
	\begin{align*}
		(A\cup B)\setminus(F_A\cup F_B)&=(A\cup B)\cap(F_A\cup F_B)^c\\
		&=(A\cup B)\cap(F_A^c\cap F_B^c)\\
		&=[A\cap(F_A^c\cap F_B^c)]\cup[B\cap(F_A^c\cap F_B^c)]\\
		&=[A\cap F_A^c\cap F_B^c]\cup[B\cap F_B^c\cap F_A^c]\\
		&\subseteq[A\cap F_A^c]\cup[B\cap F_B^c]\\
		&\subseteq[A\setminus F_A]\cup[B\setminus F_B].\\
	\end{align*}
	By monotonicity of outer measure and subadditivity, 
	\begin{align*}
		m^*((A\cup B)\setminus(F_A\cup F_B))&\le m^*([A\setminus F_A]\cup[B\setminus F_B])\\
		&\le m^*(A\setminus F_A)+m^*(B\setminus F_B)\\
		&=0.
	\end{align*}
	Therefore $A\cup B$ is measurable.\\
	\item For any set $A$, define $m^{**}(A)\in[0,\infty]$ by 
	\[
		m^{**}(A)=\inf\{m^*(\mathcal{O})\ |\ \mathcal{O}\supseteq A, \mathcal{O}\text{ open.}\}	
	\]
	How is this set function $m^{**}$ related to outer measure $m^*$?\\
	\\Consider any open set $\mathcal{O}$ such that $A\subseteq\mathcal{O}$.
	By monotonicity of outer measure, $m^*(A)\le m^*(\mathcal{O})$, and therefore $m^*(A)$ is a lower bound to the set $\{m^*(\mathcal{O})\ |\ \mathcal{O}\supseteq A, \mathcal{O}\text{ open.}\}$.
	Because $m^{**}$ is defined as the greatest lower bound, we get
	\[
		m^*(A)\le m^{**}(A).
	\]
	Now, if $m^*(A)=\infty$, then trivially we have
	\[
		m^*(A)\ge m^{**}(A),
	\]
	which implies $m^*(A)= m^{**}(A)$.\\
	Thus we consider the case where $m^*(A)<\infty$.\\
	Then for any $\epsilon>0$, by definition of infimum, there exists a countable collection of open intervals $\{I_n\}_{n=1}^\infty$ whose union contains $A$ for which 
	\[
		\sum_{n=1}^\infty \ell(I_n) <m^*(A)+\epsilon.
	\]
	Now, $\mathcal{O}=\bigcup_{n=1}^\infty I_n$ is an open set that contains $A$, so by definition of $m^{**}$,
	\[
		m^{**}(A)\le m^*(\mathcal{O})=m^*(\bigcup_{n=1}^\infty I_n)\le\sum_{n=1}^\infty \ell(I_n) <m^*(A)+\epsilon.
	\]
	Then $m^{**}(A)<m^*(A)+\epsilon$ implies $m^{**}(A)\le m^*(A)$.\\
	Therefore $m^*(A)= m^{**}(A)$.
	\item For any set $A$, define $m^{***}(A)\in[0,\infty]$ by
	\[
		m^{***}(A)=\sup\{m^*(F)\ |\ F\subseteq A, F\text{ closed.}\}	
	\]
	How is this set function $m^{***}$ related to outer measure $m^*$?\\
	\\Consider any closed set $F$ such that $F\subseteq A$.
	By monotonicity of outer measure, $m^*(F)\le m^*(A)$, and therefore $m^*(A)$ is an upper bound to the set $\{m^*(F)\ |\ F\subseteq A, F\text{ closed.}\}$.
	Because $m^{**}$ is defined as the least upper bound, we get
	\[
		m^{***}(A)\le m^*(A).
	\]
	(In addition, if $A$ is measurable, then $m^{***}(A)= m^*(A)$.)
\end{enumerate}

% 2.5
\section{Countable Additivity, Continuity, and the Borel-Cantelli Lemma}
\begin{flushleft}
	\begin{namedthm*}{Theorem 15}[the Continuity of Measure]
		Lebesgue measure possesses the following continuity properties:
		\begin{enumerate}[label=(\roman*),align=left]
			\item If $\{A_k\}_{k=1}^\infty$ is an ascending collection of measurable sets, then
			\[
				m\biggl(\bigcup_{k=1}^\infty A_k\biggr)=\lim_{k\to\infty}m(A_k).
			\]
			\item If $\{B_k\}_{k=1}^\infty$ is a descending collection of measurable sets and $m(B_1)<\infty$, then
			\[
				m\biggl(\bigcap_{k=1}^\infty B_k\biggr)=\lim_{k\to\infty}m(B_k).
			\]
		\end{enumerate}
	\end{namedthm*}
	\begin{proof}
		Let $\{A_k\}_{k=1}^\infty$ be ascending and measurable.\\
		If there exists an index $k$ such that $m(A_k)>\infty$, then by monotonicity of measure, $m(\bigcup_{k=1}^\infty A_k)=\infty$.
		Also, because this collection is ascending, we have $A_k\subseteq A_n$ whenever $k\le n$; therefore by monotonicity, $\infty=m(A_k)\le m(A_n)$ for all $n$ such that $k\le n$, and thus (i) holds.\\
		Therefore it remains to prove the case that $m(A_k)<\infty$ for all $k$.\\
		Define $A_0=\emptyset$, and define $C_k=A_k\setminus A_{k-1}$. Then $\{C_k\}_{k=1}^\infty$ is disjoint and $\bigcup_{k=1}^\infty C_k=\bigcup_{k=1}^\infty A_k$.
		Now we can write
		\begin{align*}
			m(\bigcup_{k=1}^\infty A_k)&=m(\bigcup_{k=1}^\infty C_k)\\
			&=\sum_{k=1}^\infty m(C_k)&&\text{countable (disjoint) monotonicity}\\
			&=\sum_{k=1}^\infty m(A_k\setminus A_{k-1})\\
			&=\sum_{k=1}^\infty [m(A_k)-m(A_{k-1})]&&\text{by excision: }m(A_{k-1})<\infty\\
			&=\lim_{n\to\infty}\sum_{k=1}^n [m(A_k)-m(A_{k-1})]\\
			&=\lim_{n\to\infty}m(A_n)-m(A_0)&&\text{by telescoping}\\
			&=\lim_{n\to\infty}m(A_n).&&\text{because }A_0=\emptyset
		\end{align*}
		\\Let $\{B_k\}_{k=1}^\infty$ be descending and measurable.\\
		Define $D_k=B_1\setminus B_k=B_1\cap B_k^c$.\\
		Then because $\{B_k\}_{k=1}^\infty$ is descending, 
		\[
			B_k\supseteq B_{k+1}\implies B_1\cap B_k^c\subseteq B_1\cap B_{k+1}^c\implies D_k\subseteq D_{k+1},
		\]
		and $\{D_k\}_{k=1}^\infty$ is ascending.\\
		Now we have
		\[
			\bigcup_{k=1}^\infty D_k=\bigcup_{k=1}^\infty [B_1\cap B_k^c]=B_1\cap[\bigcup_{k=1}^\infty  B_k^c]=(B_1\cap[\bigcap_{k=1}^\infty  B_k]^c=B_1\setminus[\bigcap_{k=1}^\infty  B_k].
		\]
		Then by part (i), we can write
		\begin{align*}
			m(\bigcup_{k=1}^\infty D_k)&=\lim_{k\to\infty}m(D_k)\\
			m(B_1\setminus[\bigcap_{k=1}^\infty  B_k])&=\lim_{k\to\infty}m(B_1\setminus B_k)\\
			m(B_1)-m(\bigcap_{k=1}^\infty  B_k)&=\lim_{k\to\infty}[m(B_1)-m( B_k)]\\
			m(B_1)-m(\bigcap_{k=1}^\infty  B_k)&=m(B_1)-\lim_{k\to\infty}[m( B_k)]\\
			m(\bigcap_{k=1}^\infty  B_k)&=\lim_{k\to\infty}[m( B_k)].
		\end{align*}
	\end{proof}
	\begin{namedthm*}{The Borel-Cantelli Lemma}
		Let $\{E_k\}_{k=1}^\infty$ be a countable collection of measurable sets for which $\sum_{k=1}^\infty m(E_k)<\infty$.
		Then almost all $x\in\mathbb{R}$ belong to at most finitely many of the $E_k$'s.
	\end{namedthm*}
	\begin{proof}
		By countable subadditivity, for each $n$,
		\[
			m(\bigcup_{k=n}^\infty E_k)\le\sum_{k=n}^\infty m(E_k)<\infty.
		\]
		Because $\sum_{k=1}^\infty m(E_k)<\infty$, we have a sequence of partial sums such that for any $\epsilon>0$, there exists an index n for which 
		\[
			\sum_{k=n}^\infty m(E_k)=|\sum_{k=1}^\infty m(E_k)-\sum_{k=1}^{n-1} m(E_k)|<\epsilon.
		\] 
		Therefore there exists an $n$ such that $|\sum_{k=n}^\infty m(E_k)-0|<\epsilon$, and $\lim_{n\to\infty}\sum_{k=n}^\infty m(E_k)=0$.\\
		By continuity of measure (ii),
		\[
			m(\bigcap_{n=1}^\infty[\bigcup_{k=1}^\infty E_k])=\lim_{n\to\infty}m(\bigcup_{k=1}^\infty E_k)\le\lim_{n\to\infty}\sum_{k=n}^\infty m(E_k)=0.
		\]
		Therefore almost all $x\in\mathbb{R}$ fail to belong to $\bigcap_{n=1}^\infty[\bigcup_{k=1}^\infty E_k]$ and therefore belong to at most finitely many $E_k$'s.
	\end{proof}
	Let $\{A_k\}_{k=1}^\infty$ be a countable collection of sets that belong to a $\sigma$-algebra $\mathcal{A}$. Since $\mathcal{A}$ is closed w.r.t. countable unions and intersections, the following two sets belong to $\mathcal{A}$:
	\begin{align*}
		\lim\sup\{A_k\}_{k=1}^\infty&=\bigcap_{n=1}^\infty[\bigcup_{k=1}^\infty A_k]\\
		\lim\inf\{A_k\}_{k=1}^\infty&=\bigcup_{n=1}^\infty[\bigcap_{k=1}^\infty A_k]
	\end{align*}
	The set $\lim\sup\{A_k\}_{k=1}^\infty$ is the set of points that belong to $A_n$ for countably infinitely many indices $n$ while the set $\lim\inf\{A_k\}_{k=1}^\infty$ is the set of points that belong to $A_n$ except for at most finitely many indices $n$.
\end{flushleft}
\begin{center}
	\textbf{PROBLEMS}
\end{center}
\begin{enumerate}
	\setcounter{enumi}{23}
	\item Show that if $E_1$ and $E_2$ are measurable, then
	\[
		m(E_1\cup E_2)+m(E_1\cap E_2) = m(E_1)+m(E_2).	
	\]
	\begin{align*}
		m(E_1\cup E_2)+m(E_1\cap E_2)&=m([E_1\cup E_2]\cap E_1)+m([E_1\cup E_2]\cap E_1^c)+m(E_1\cap E_2)\\
		&=m(E_1)+m([E_1\cap E_1^c]\cup[E_2\cap E_1^c])+m(E_1\cap E_2)\\
		&=m(E_1)+m(\emptyset\cup[E_2\cap E_1^c])+m(E_1\cap E_2)\\
		&=m(E_1)+m(E_2\cap E_1^c)+m(E_1\cap E_2)\\
		&=m(E_1)+m([E_2\cap E_1^c]\cup[E_1\cap E_2])\\
		&=m(E_1)+m([E_2\cup(E_1\cap E_2)]\cap [E_1^c\cup(E_1\cap E_2)])\\
		&=m(E_1)+m(E_2\cap [E_1^c\cup(E_1\cap E_2)])\\
		&=m(E_1)+m(E_2\cap [E_1^c\cup E_1]\cap [E_1^c\cup E_2])\\
		&=m(E_1)+m(E_2\cap \mathbb{R} \cap [E_1^c\cup E_2])\\
		&=m(E_1)+m(E_2\cap [E_1^c\cup E_2])\\
		&=m(E_1)+m(E_2).
	\end{align*}
	\item Show that the assumption that $m(B_1)<\infty$ is necessary in part (ii) of the theorem regarding continuity of measure.\\
	In the proof of (ii), we get to the point 
	\[
		m(B_1)-m(\bigcap_{k=1}^\infty  B_k)=m(B_1)-\lim_{k\to\infty}[m( B_k)].
	\]
	If $m(B_1)=\infty$, then we have
	\[
		\infty-m(\bigcap_{k=1}^\infty  B_k)=\infty-\lim_{k\to\infty}[m( B_k)],
	\]
	and we cannot reach the conclusion we want because $\infty-\infty$ is not defined.
	\item Let $\{E_k\}_{k=1}^\infty$ be a countable disjoint collection of measurable sets. Prove that for any set $A$, 
	\[
		m^*(A\cap\bigcup_{k=1}^\infty E_k)=\sum_{k=1}^\infty m^*(A\cap E_k).	
	\]
	We have by countable subadditivity:
	\begin{align*}
		m^*(A\cap\bigcup_{k=1}^\infty E_k)&=m^*(\bigcup_{k=1}^\infty (A\cap E_k))\le\sum_{k=1}^\infty m^*(A\cap E_k).
	\end{align*}
	Now, for any $n$, we have $A\cap\bigcup_{k=1}^\infty E_k\supseteq A\cap\bigcup_{k=1}^n E_k$, so by monotonicity and Proposition 6,
	\begin{align*}
		m^*(A\cap\bigcup_{k=1}^\infty E_k)&\ge m^*(A\cap\bigcup_{k=1}^n E_k)=\sum_{k=1}^n m^*(A\cap E_k)
	\end{align*}
	The left hand side is independent of $n$, so taking the limit as $n\to\infty$, we get
	\begin{align*}
		m^*(A\cap\bigcup_{k=1}^\infty E_k)&\ge \sum_{k=1}^\infty m^*(A\cap E_k).
	\end{align*}
	\item Let $\mathcal{M}'$ be any $\sigma$-algebra of subsets of $\mathbb{R}$ and $m'$ a set function on $\mathcal{M}'$ which takes values in $[0,\infty]$, is countably additive, and such that $m'(\emptyset)=0$.
	\begin{enumerate}[label=(\roman*),align=left]
		\item Show that $m'$ is finitely additive, monotone, countably monotone, and possesses the excision property.\\
		Countable additivity implies that for any disjoint collection of measurable sets $\{E_k\}_{k=1}^\infty$, we have $m'(\bigcup_{k=1}^\infty E_k)=\sum_{k=1}^\infty m'(E_k)$.\\
		Now, any finite disjoint collection $\{E_k\}_{k=1}^n$ can be extended to the infinite disjoint collection $\{E_k'\}_{k=1}^\infty$, where $E_k'=E_k$ for $k\in\{1,\cdots,n\}$, and $E_k'=\emptyset$ for $k>n$.
		Clearly from this we have finite additivity.\\
		In Problem 1 of this chapter, it was shown that a countably additive set function possesses the monotonicity property.
		Thus $m'$ is monotone. It can clearly be shown that $m'$ is also countably monotone.\\
		To see excision, simply use countable additivity to see that for measurable sets $A,B$ such that $A\subseteq B$, we have
		\[
			m'(B) = m'([B\cap A]\cup[B\cap A^c])=m'(B\cap A)+m'(B\cap A^c)=m'(A)+m'(B\setminus A).
		\]
		\item Show that $m'$ possesses the same continuity properties as Lebesgue measure.\\
		Check Theorem 15 and the Borel-Cantelli Lemma above.
	\end{enumerate}
	\item Show that continuity of measure together with finite additivity of measure implies countable additivity of measure.\\
	Let $\{E_k\}_{k=1}^\infty$ be a disjoint collection of measurable sets. (if any $E_k$ has infinite measure, countable additivity is clear, so we need only consider sets of finite measure for all $E_k$.)\\
	Finite additivity implies that for the disjoint collection of measurable sets $\{E_k\}_{k=1}^n$, we have $m(\bigcup_{k=1}^n E_k)=\sum_{k=1}^n m(E_k)$.\\
	We can define $F_n=\bigcup_{k=1}^n E_k$ so that continuity of measure implies that for the ascending collection $\{F_n\}_{n=1}^\infty$ of measurable sets, we have $m(\bigcup_{n=1}^\infty F_n)=\lim_{n\to\infty}m(F_n)$.\\
	Therefore we can write
	\[
		m(\bigcup_{n=1}^\infty E_n)=m(\bigcup_{n=1}^\infty F_n)=\lim_{n\to\infty}m(F_n)=\lim_{n\to\infty}m(\bigcup_{k=1}^n E_k)=\lim_{n\to\infty}\sum_{k=1}^n m(E_k)=\sum_{k=1}^\infty m(E_k).
	\]
\end{enumerate}

% 2.6
\section{Nonmeasurable Sets}
\begin{flushleft}
	Consider the subgroup under addition $\mathbb{Q}\subseteq\mathbb{R}$.
	Now, $\mathbb{Q}$ is a normal subgroup, and we have the quotient group $\mathbb{R}/\mathbb{Q}$, with the (disjoint) cosets written as $r+\mathbb{Q}$ where $r\in\mathbb{R}$.
	A Vitali set $V\subseteq[0,1]$ is defined to be a set such that for all $r\in\mathbb{R}$, there exists exactly one unique $v\in V$ such that $v-r\in\mathbb{Q}$.
	Every Vitali set is uncountable, and $v-u\notin\mathbb{Q}$ for $u,v\in V$, $u\neq v$.
	\begin{namedthm*}{Theorem}
		A Vitali set is non-measurable.
	\end{namedthm*}
	\begin{proof}
		Suppose by contradiction that a Vitali set V is measurable.\\
		Let $\{q_k\}_{k=1}^\infty$ be an enumeration of the rational numbers in $[-1,1]$:\\
		recall that $\mathbb{Q}$ looks like
		\[
			\mathbb{Q} = \{0,\frac{1}{1},-\frac{1}{1},\frac{1}{2},-\frac{1}{2},\frac{2}{1},-\frac{2}{1}, \frac{3}{1}, -\frac{3}{1},\frac{1}{3},-\frac{1}{3},\frac{1}{4},-\frac{1}{4},\frac{2}{3},-\frac{2}{3},\cdots \},
		\]
		therefore
		\[
			\{q_k\}_{k=1}^\infty= \{0,\frac{1}{1},-\frac{1}{1},\frac{1}{2},-\frac{1}{2},\frac{1}{3},-\frac{1}{3},\frac{1}{4},-\frac{1}{4},\frac{2}{3},-\frac{2}{3},\cdots \}.
		\]
		For each natural number $k$, let $V_k=V+q_k=\{v+q_k:v\in V\}$.\\
		First we will show the following:
		\begin{enumerate}[label=(\roman*),align=left]
			\item $V_i\cap V_j=\emptyset$ for $i\neq j$
			\item $[0,1]\subseteq\bigcup_{k=1}^\infty V_k \subseteq [-1,2]$
		\end{enumerate}
		(i) Suppose by contradiction that $V_i\cap V_j=\emptyset$ for some $i\neq j$.\\
		That is, there exists $x\in V_i,y\in V_j$ such that $x=y$.\\
		Also, there exists $v,u\in V$ such that $x=v+q_i$ and $y=u+q_j$.\\
		By equality, we have $v+q_i=u+q_j$.\\
		In the case that $v=u$, we get $q_i=q_j$, a contradiction.\\
		In the case that $v\neq u$, we can write $v-u=q_j-q_i\in\mathbb{Q}$, a contradiction.\\\medskip
		(ii) For any real $r\in[0,1]$, there exists a $v\in V\subseteq[0,1]$ such that $r-v\in\mathbb{Q}$.\\
		We can see that
		\begin{align*}
			\max(r-v)&=1-0=1,\\
			\min(r-v)&=0-1=-1.
		\end{align*}
		which implies $r-v=q_i\in[-1,1]\cap\mathbb{Q}$ for some $i$, and thus $r=v+q_i\in V_i$.\\
		In short, we can write this as
		\[
			r\in[0,1]\implies r\in V_i\text{ for some }i\implies r\in\bigcup_{k=1}^\infty V_k\implies[0,1]\subseteq\bigcup_{k=1}^\infty V_k. 
		\]
		Now, $V_k=V+q_k,V\subseteq[0,1],q_k\in[-1,1]$, therefore
		\begin{align*}
			\max(v+q_k)&=1+1=2,\\
			\min(v+q_k)&=0-1=-1.
		\end{align*}
		Therefore $V_k\subseteq[-1,2]$ for all $k$, and thus $\bigcup_{k=1}^\infty V_k\subseteq[-1,2]$.\\\medskip
		Then we can write
		\begin{align*}
			m^*([0,1])&\le m^*(\bigcup_{k=1}^\infty V_k) \le m^*([-1,2])&&\text{by monotonicity of outer measure}\\
			1&\le\sum_{k=1}^\infty m^*(V_k)\le 3&&\text{countable additivity (measurability of $V$) }\star\\
			1&\le\sum_{k=1}^\infty m^*(V+q_k)\le 3\\
			1&\le\sum_{k=1}^\infty m^*(V)\le 3&&\text{by translation invariance of outer measure}\\
		\end{align*}
		However, $m^*(V)\ge0$ is a constant, so $\sum_{k=1}^\infty m^*(V)=0$ or $\sum_{k=1}^\infty m^*(V)=\infty$, neither of which is in $[1,3]$, and we reach a contradiction.
	\end{proof}
	For any nonempty set $E$ of real numbers, we define two points in $E$ to be \textbf{rationally equivalent} provided their difference belongs to $\mathbb{Q}$. By a \textbf{choice set} for the rational equivalence relation on $E$ we mean a set $\mathcal{C}_E$ consisting of exactly one member of each equivalence class.
	A choice set $\mathcal{C}_E$ is characterized by the following two properties:
	\begin{enumerate}
		\item the difference of two points in $\mathcal{C}_E$ is not rational;
		\item for each point $x$ in $E$, there is a point $c$ in $\mathcal{C}_E$ for which $x=c+q$, $q\in\mathbb{Q}$.
	\end{enumerate}
	The first property can be reformulated as
	\[
		\text{For any set }\Lambda\subseteq\mathbb{Q},\{\lambda+\mathcal{C}_E\}_{\lambda\in\Lambda}\text{ is disjoint}.
	\]
	We also have that 
	\[
		E\subseteq\bigcup_{\lambda\in\mathbb{Q}}(\lambda+\mathcal{C}_E).
	\]
\end{flushleft}
\begin{center}
	\textbf{PROBLEMS}
\end{center}
\begin{enumerate}
	\setcounter{enumi}{28}
	\item 
	\begin{enumerate}[label=(\roman*),align=left]
		\item Show that rational equivalence defines an equivalence relation on any set.\\
		\\Let $X$ be any set and define $x\sim y$ when $x-y\in\mathbb{Q}$ for $x,y\in X$.
		\begin{enumerate}
			\item $x-x=0\in\mathbb{Q}\iff x\sim x$ for all $x\in X$.
			\item $x\sim y \iff x-y=q \in\mathbb{Q}\iff y-x=-q\in\mathbb{Q}\iff y\sim x$ for all $x,y\in X$.
			\item $x\sim y,y\sim z \iff x-y=q\in\mathbb{Q},y-z=q'\in\mathbb{Q}\iff x-z=x-y+y-z=q+q'\in\mathbb{Q}\iff x\sim z$ for all $x,y,z\in X$.
		\end{enumerate}
		\item Explicitly find a choice set for the rational equivalence relation on $\mathbb{Q}$.\\
		\\(For any nonempty set $E$ of real numbers, we define two points in $E$ to be \textbf{rationally equivalent} provided their difference belongs to $\mathbb{Q}$. By a \textbf{choice set} for the rational equivalence relation on $E$ we mean a set $\mathcal{C}_E$ consisting of exactly one member of each equivalence class.)
		Therefore for the nonempty set $\mathbb{Q}$, we can choose a choice set $\mathcal{C}_{\mathbb{Q}}=\{q\}$ for any $q\in\mathbb{Q}$.
		\item Define two numbers to be irrationally equivalent provided their difference is irrational or zero. Is this an equivalence relation on $\mathbb{R}$? Is this an equivalence relation on $\mathbb{Q}$?\\
		\begin{enumerate}
			\item $x-x=0\in\{\mathbb{Q}^c,0\}\iff x\sim x$ for all $x\in \mathbb{R}$.
			\item $x\sim y \iff x-y=q \in\{\mathbb{Q}^c,0\}\iff y-x=-q\in\{\mathbb{Q}^c,0\}\iff y\sim x$ for all $x,y\in \mathbb{R}$.
			\item $2-\pi\in\{\mathbb{Q}^c,0\},\pi-0\in\{\mathbb{Q}^c,0\}$ but $2-0\notin\{\mathbb{Q}^c,0\}$
		\end{enumerate}
		Not an equivalence relation on $\mathbb{R}$.\\
		\begin{enumerate}
			\item $x-x=0\in\{\mathbb{Q}^c,0\}\iff x\sim x$ for all $x\in \mathbb{Q}$.
			\item $x\sim y \iff x-y=0 \in\{\mathbb{Q}^c,0\}\iff y-x=0\in\{\mathbb{Q}^c,0\}\iff y\sim x$ for all $x,y\in \mathbb{Q}$.
			\item $x\sim y,y\sim z \iff x-y=0\in\{\mathbb{Q}^c,0\},y-z=0\in\{\mathbb{Q}^c,0\}\iff x-z=x-y+y-z=0+0\in\{\mathbb{Q}^c,0\}\iff x\sim z$ for all $x,y,z\in \mathbb{Q}$.
		\end{enumerate}
		An equivalence relation on $\mathbb{Q}$.
	\end{enumerate}
	\item Show that any choice set for the rational equivalence relation on a set of positive outer measure must be uncountably infinite.\\
	\\Let $E$ be a set of positive outer measure.
	Suppose there exists a choice set $\mathcal{C}_E$ for the rational equivalence relation on $E$ such that $\mathcal{C}_E$ is countable. 
	All countable sets have outer measure zero, so $m^*(\mathcal{C}_E)=0$.
	Because we know $E\subseteq\bigcup_{\lambda\in\mathbb{Q}}(\lambda+\mathcal{C}_E)$, by monotonicity, subadditivity, and translation invariance of outer measure,
	\[
		m^*(E)\le m^*(\bigcup_{\lambda\in\mathbb{Q}}(\lambda+\mathcal{C}_E))\le\sum_{\lambda\in\mathbb{Q}}m^*(\lambda+\mathcal{C}_E)=\sum_{\lambda\in\mathbb{Q}}m^*(\mathcal{C}_E)=\sum_{\lambda\in\mathbb{Q}}0=0,
	\]
	and we have a contradiction to the fact that $m^*(E)>0$.
	\item Justify the assertion in the proof of Vitali's Theorem that it suffices to consider the case that $E$ is bounded.\\
	\\(Vitali: Any set of real numbers with positive outer measure contains a subset that fails to be measurable.)
	By Problem 14, we showed that every set of positive outer measure $E$ contains a bounded subset $A\subseteq E$ of positive outer measure. Therefore if there exists a subset $S\subseteq A$ that fails to be measurable, then $S \subseteq A\subseteq E$ is a subset that fails to be measurable.
	\item Does Lemma 16 remain true if $\Lambda$ is allowed to be finite or to be uncountably infinite? Does it remain true if $\Lambda$ is allowed to be unbounded?\\
	\\(Lemma 16: Let $E$ be a bounded measurable set of real numbers. Suppose there is a bounded, countably infinite set of real numbers $\Lambda$ for which the collection of translates of $E$, $\{\lambda+E\}_{\lambda\in\Lambda}$, is disjoint. Then $m(E)=0$.)\\
	Consider the case $\Lambda=\{1,2\}$ is finite, and $E=(0,1)$. Then $\{\lambda+E\}_{\lambda\in\Lambda=\{1,2\}}=\{1+(0,1),2+(0,1)\}=\{(1,2),(2,3)\}$, which is a disjoint collection. However, $m(E)=1\neq0$.\\
	If $\Lambda$ is uncountably infinite and satisfies that the translates are disjoint, then we can choose a countable subset of $\Lambda$ and thus Lemma 16 remains true.\\
	Consider the case $\Lambda=\{1,2,3,\cdots\}$ is unbounded, and $E=(0,1)$. Then the collection of translates of $E$, $\{(1,2),(2,3),(3,4),\cdots\}$ is disjoint but $m(E)=1\neq0$.
	\item Let $E$ be a nonmeasurable set of finite outer measure. Show that there is a $G_\delta$ set $G$ that contains $E$ for which
	\[
		m^*(E)=m^*(G),\text{ while }m^*(G\setminus E)>0.
	\]
	This is a similar construction for the proof from Theorem 11 (i).\\
	Let $E$ be a nonmeasurable set of finite outer measure.\\
	By definition of outer measure and infimum, for any natural number $n$, there exists a countable collection of intervals $\{(I_n)_k\}_{k=1}^\infty$ such that $E\subseteq\bigcup_{k=1}^\infty (I_n)_k$ and
	\[
		m^*(E)\le\sum_{k=1}^\infty\ell((I_n)_k)<m^*(E)+1/n.	
	\]
	Defining $\mathcal{O}_n=\bigcup_{k=1}^\infty (I_n)_k$, we see that $\mathcal{O}_n$ is an open set containing $E$ for each $n$.
	Further define $G=\bigcap_{n=1}^\infty\mathcal{O}_n$ so that $E\subseteq G\subseteq\mathcal{O}_n$ for any $n$ and $G$ is a $G_\delta$ set that contains $E$.\\
	By subadditivity of outer measure,
	\begin{align*}
		m^*(G)\le m^*(\mathcal{O}_n)=m^*(\bigcup_{k=1}^\infty (I_n)_k)\le\sum_{k=1}^\infty\ell(I_k)<m^*(E)+1/n,
	\end{align*}
	so that $m^*(G)<m^*(E)+1/n\implies m^*(G)\le m^*(E)$.
	By subadditivity, $E\subseteq G$ implies we also have $m^*(E)\le m^*(G)$, and so $m^*(E)=m^*(G)$.\\
	Now, we know that the outer measure is nonnegative by monotonicity, so we have the inequality $m^*(G\setminus E)\ge0$.\\
	By Theorem 11 (ii), $m^*(G\setminus E)=0\iff E$ is measurable, so we must have $m^*(G\setminus E)>0$.
\end{enumerate}

% 2.7
\section{The Cantor Set and the Cantor-Lebesgue Function}
\begin{center}
	\textbf{PROBLEMS}
\end{center}
\begin{enumerate}
	\setcounter{enumi}{33}
	\item Show that there is a continuous, strictly increasing function on the interval $[0,1]$ that maps a set of positive measure onto a set of measure zero.\\
	\\The function $\psi:[0,1]\to[0,2]$ defined by $\psi(x)=\varphi(x)+x$ maps the Cantor set $C\subseteq[0,1]$ onto a measurable set of positive measure.
	That is, $m(C)=0$ and $m(\psi(C))>0$.
	We can consider the inverse function $\psi^{-1}:[0,2]\to[0,1]$ restricted to $[0,1]$: $\psi^{-1}|_{[0,1]}:[0,1]\to[0,1]$.
	Now consider the set $C'=C\cap[0,1]$. This set $C'$ is a measurable subset of $C$, a measurable set of measure zero, so by monotonicity of measure, $m(C')=0$.
	Then the function has $\psi^{-1}|_{[0,1]}(\psi(C'))=C'$, where $m(\psi(C'))>0$ and $m(C')=0$, thus mapping the set $\psi(C')$ of positive measure* onto the set $C'$ of measure zero.\\
	(*We know that  $m(\psi(C))>0$, but not shown that $m(\psi(C'))>0$ where $C'\subseteq C$.)
	\item Let $f$ be an increasing function on the open interval $I$. For $x_0\in I$ show that $f$ is continuous at $x_0$ iff there are sequences $\{a_n\}$ and $\{b_n\}$ in $I$ such that for each $n$, $a_n<x_0<b_n$, and $\lim_{n\to\infty}[f(b_n)-f(a_n)]=0$.\\
	\\Let $f$ be an increasing function on the open interval $I$ and let $x_0\in I$.\\
	$(\implies)$ Suppose that $f$ is continuous at $x_0$.\\
	Because $I$ is open, there exists an index $N$ such that for all $n\ge N$, we have that $(x_0-1/n,x_0+1/n)\subseteq I$.
	Then for each $n\ge N$ we can choose $a_n\in(x_0-1/n,x_0)$ and $b_n\in(x_0,x_0+1/n)$, and for $n<N$ let $a_n=a_N$ and $b_n=b_N$, so that $a_n<x_0<b_n$ for all $n$.
	Now,we have
	\begin{align*}
		x_0-1/n<a_n<x_0&\implies x_0-a_n<1/n,\\
		x_0<b_n<x_0+1/n&\implies b_n-x_0<1/n,
	\end{align*}
	therefore $\lim_{n\to\infty} a_n = x_0$ and $\lim_{n\to\infty} b_n = x_0$.
	Because $f$ is continuous and increasing, for all $\epsilon>0$, there exists the number $1/n>0$ such that
	\begin{align*}
	x_0-a_n<1/n&\implies f(x_0)-f(a_n)<\epsilon,\\
	b_n-x_0<1/n&\implies f(b_n)-f(x_0)<\epsilon.
	\end{align*}
	(therefore $\lim_{n\to\infty} f(a_n) = f(x_0)$ and $\lim_{n\to\infty} f(b_n) = f(x_0)$.) We can write
	\[
		[f(b_n)-f(a_n)]=f(x_0)-f(a_n)+f(b_n)-f(x_0) <\epsilon+\epsilon=\epsilon'
	\]
	and so $\lim_{n\to\infty}[f(b_n)-f(a_n)]=0$.\\
	\\$(\impliedby)$ Suppose that there exist sequences $\{a_n\}$,$\{b_n\}$ such that $a_n<x_0<b_n$ and $\lim_{n\to\infty}[f(b_n)-f(a_n)]=0$.\\
	That is, for any $\epsilon>0$, there exists an index $N$ such that $f(b_n)-f(a_n)<\epsilon$ for all $n\ge N$.
	\\Then $f(b_n)<f(a_n)+\epsilon$ and $f(b_n)-\epsilon<f(a_n)$.\\
	Because $f$ is increasing, we have \[f(b_n)-\epsilon<f(a_n)<f(x_0)<f(b_n)<f(a_n)+\epsilon.\]
	Then $f(x_0)-f(a_n)<\epsilon$ and $f(b_n)-f(x_0)<\epsilon$, which implies $\lim_{n\to\infty} f(a_n) = f(x_0)$ and $\lim_{n\to\infty} f(b_n) = f(x_0)$.\\
	By monotonicity of $f$, we also have \[b_n-\epsilon<a_n<x_0<b_n<a_n+\epsilon.\]
	Then $x_0-a_n<\epsilon$ and $b_n-x_0<\epsilon$, which implies $\lim_{n\to\infty} a_n= x_0$ and $\lim_{n\to\infty} b_n = x_0$.\\
	Now, clearly we see that for any $\epsilon>0$, we have $x_0-a_n<\epsilon\iff f(x_0)-f(a_n)<\epsilon$, and $b_n-x_0<\epsilon\iff f(b_n)-f(x_0)<\epsilon$, and continuity at $x_0$ follows.
	\item Let $f$ be a continuous function defined on $E$. Is it true that $f^{-1}(A)$ is always measurable if $A$ is measurable?\\
	\\No, the function $\psi:[0,1]\to[0,2]$ defined by $\psi(x)=\varphi(x)+x$ maps a measurable set $A$, subset of the Cantor set, onto a nonmeasurable set $\psi(A)$. Define $f=\psi^{-1}$ so that $f^{-1}(A)=(\psi^{-1})^{-1}(A)$ is not measurable but $A$ is measurable.
	\item Let the function $f:[a,b]\to\mathbb{R}$ be Lipschitz; that is, there is a constant $c\ge0$ such that for all $u,v\in[a,b]$, $|f(u)-f(v)|\le c|u-v|$.
	Show that $f$ maps a set of measure zero onto a set of measure zero. Show that $f$ maps a $F_\sigma$ set onto an $F_\sigma$ set. Conclude that $f$ maps a measurable set to a measurable set.\\
	\\Let $f$ be a Lipschitz function on the interval $I$. Clearly $f$ is also continuous.\\
	Let $E\subseteq I$ be a set of measure zero; that is, $m^*(E)=m(E)=0$.
	By definition of infimum, for any $\epsilon>0$, there exists a countable collection of open intervals $\{I_k\}_{k=1}^\infty$, $I_k=(a_k,b_k)$, such that $E\subseteq\bigcup_{k=1}^\infty I_k$ and 
	\[
		0\le \sum_{k=1}^\infty \ell(I_k)<0+\frac{\epsilon}{c}.
	\]
	We also have that $E\subseteq\bigcup_{k=1}^\infty I_k\implies f(E)\subseteq f(\bigcup_{k=1}^\infty I_k)=\bigcup_{k=1}^\infty f(I_k)$.
	\\Also, by Chapter 1 Problem 54, Because $I_k$ is an interval, the continuous real-valued function $f$ on $I_k$ has an interval as its image; that is, $f(I_k)$ is an interval.
	Then there exists some $u_k,v_k\in(a,b)$ such that $f(I_k)=(f(u_k),f(v_k))$ and $m(f(I_k))=f(v_k)-f(u_k)$. Then because $f$ is Lipschitz, $|f(v_k)-f(u_k)|\le c|v_k-u_k|$ for all $k$.
	\begin{align*}
		m(f(E))&\le	m(\bigcup_{k=1}^\infty f(I_k))&&\text{ by monotonicity}\\
		&\le \sum_{k=1}^\infty m(f(I_k))&&\text{ by subadditivity}\\
		&=\sum_{k=1}^\infty m(f(v_k)-f(u_k))\\
		&\le \sum_{k=1}^\infty c|v_k-u_k|&&\text{ because $f$ is Lipschitz}\\
		&\le\sum_{k=1}^\infty c|b_k-a_k|&&\text{ because }(u_k,v_k)\subseteq(a,b)\\
		&= \sum_{k=1}^\infty c\ell(I_k)\\
		&<\epsilon.
	\end{align*}
	Therefore $m(f(E))=0$.\\
	\item Let $F$ be the subset of $[0,1]$ constructed in the same manner as the Cantor set except that each of the intervals removed at the $n$th deletion stage has length $\alpha 3^{-n}$ with $0<\alpha<1$.
	Show that $F$ is a closed set, $[0,1]\setminus F$ is dense in $[0,1]$, and $m(F)=1-\alpha$. Such a set $F$ is called a generalized Cantor set.\\
	\\Define $F$ to be constructed in the same manner as the Cantor set, with 
	\[
		F=\bigcap_{k=1}^\infty F_k,	
	\]
	where $\{F_k\}_{k=1}^\infty$ is a descending sequence of closed sets, and each $F_k$ is a disjoint union of $2^k$ closed intervals, each of length $\alpha/3^k$.
	\\It can clearly be seen that $F$ is a closed set because it is an intersection of closed sets.
	\\Now, for any point $x\in[0,1]$, there exists an index $k$ such that $x\notin F_k$; that is, $x\in F_k^c$, which is an open set. Therefore we can construct sequences in $([0,1]\setminus F)\setminus\{x\}$ that converge to $x$.
	\\Each $F_k$ is the disjoint union of $2^k$ closed intervals each of length $\alpha/3^k$, so at each step we remove $2^{k-1}$ open intervals of length $\alpha/3^k$:
	\begin{align*}
		m(F_1)&=1-\alpha/3\\
		m(F_2)&=1-\alpha/3-2\alpha/3^2\\
		m(F_3)&=1-\alpha/3-2\alpha/3^2-2^2\alpha/3^3\\
		\vdots\\
		m(F_n)&=1-\sum_{k=1}^n2^{k-1}\alpha/3^k
	\end{align*}
	Then by the continuity of measure, we have \[m(\bigcap_{k=1}^\infty F_k)=\lim_{n\to\infty}m(F_n)=\lim_{n\to\infty}(1-\sum_{k=1}^n2^{k-1}\alpha/3^k).\]
	We can see that
	\begin{align*}
		\lim_{n\to\infty}\sum_{k=1}^n2^{k-1}\alpha/3^k&=\alpha/3\lim_{n\to\infty}\sum_{k=1}^n(\frac{2}{3})^{k-1}\\
		&=\alpha/3\lim_{n\to\infty}\sum_{k=0}^{n-1}(\frac{2}{3})^{k}\\
		&=\alpha/3\lim_{n\to\infty}\frac{1-(2/3)^n}{1-(2/3)}\\
		&=\alpha/3\frac{1}{1-(2/3)}\\
		&=\alpha/3(\frac{1}{1/3})\\
		&=\alpha.
	\end{align*}
	Therefore $m(F)=m(\bigcap_{k=1}^\infty F_k)=1-\alpha$.
	\item Show that there is an open set of real numbers that, contrary to intuition, has a boundary of positive measure. (Hint: consider the complement of the generalized Cantor set of the preceding problem.)\\
	\\We have $F\cup(F^c\cap[0,1])=[0,1]$, and $m(F)=1-\alpha$, and $m(F^c\cap[0,1])=\alpha$...
	\item A subset $A$ of $\mathbb{R}$ is said to be \textbf{nowhere dense} in $\mathbb{R}$ provided that every open set $\mathcal{O}$ has a non-empty open subset that is disjoint from $A$. Show that the Cantor set is nowhere dense in $\mathbb{R}$.\\
	\\The Cantor set $C\subseteq[0,1$] is defined to be the countable intersection of sets $C_k$, where $C_k$ is the disjoint union of $2^k$ closed intervals of length $1/3^k$ each.
	From Ch1 Proposition 9, we know that every open set is the countable disjoint union of open intervals. 
	Therefore we need only prove Problem 40 for any open interval.
	\\Consider any open interval $(a,b)\in\mathbb{R}$. 
	\\In the case that there exists an index $k$ such that $(a,b)\in C_k^c$, then the proof is done:
	\\Ex: $(a,b)=(3/18,4/18)$. Then for $k=2$, we have 
	\[
		C_2=[0,1/9]\cup[2/9,1/3]\cup[2/3,7/9]\cup[8/9,1],	
	\]
	so that
	\[
		(3/18,4/18)\subseteq C_2^c=(1/9,2/9)\cup(1/3,2/3)\cup(7/9,8/9)=(2/18,4/18)\cup(1/3,2/3)\cup(7/9,8/9).	
	\] 
	In the case that for all indices $k$ we have that $(a,b)\in C_k$, then simply choose an index far enough so that one of the "open middle third" removal generated from $C_k$ is a subset of $(a,b)$.
	\\Ex: $(a,b)=(6/10,7/10)\ni 2/3$ and $2/3\in C$ so $(a,b)\cap C\neq\emptyset$. Then for $k=1$, we have
	\[
		(6/10,2/3)\subseteq(a,b)\text{ and } (6/10,2/3)\notin C_1=[0,1/3]\cup[2/3,1].
	\]
	Ex: $(a,b)=(2/3,20/27$. Then for $k=3$, we have
	\[
		C_3=[0,\frac{1}{27}]\cup[\frac{2}{27},\frac{1}{9}]\cup[\frac{2}{9},\frac{7}{27}]\cup[\frac{8}{27},\frac{1}{3}]\cup[\frac{2}{3},\frac{19}{27}]\cup[\frac{20}{27},\frac{7}{9}]\cup[\frac{8}{9},\frac{25}{27}]\cup[\frac{26}{27},1].
	\]
	so that
	\[
		(19/27,20/27)\subseteq(a,b)\text{ and } (19/27,20/27)\notin C_3.
	\]
	\item Show that a strictly increasing function that is defined on an interval has a continuous inverse.\\
	\\Let $f$ be a strictly increasing function on the interval $I$. Then for $x,y\in I$ such that $x<y$, we have $f(x)<f(y)$.
	\\Then $f$ is injective because 
	\[
		x\neq y\implies x<y\text{ or }x>y\implies f(x)<f(y)\text{ or }f(x)>f(y)\implies f(x)\neq f(y).
	\]
	Therefore the inverse $f^{-1}:im(f)\to I$ exists:
	\[
		f^{-1}(x)\neq f^{-1}(y)\implies f^{-1}(f(x))\neq f^{-1}(f(y))\implies x\neq y,
	\]
	that is, $f^{-1}$ is a function because $x=y\implies f^{-1}(x)= f^{-1}(y)$ for all $x,y\in im(f)$.\\
	Let $x\in I$ such that $a_n,b_n\in I$ with $a_n < x < b_n$ and $\lim_{n\to\infty}a_n=x$, $\lim_{n\to\infty}b_n=x$. 
	Then clearly, $\lim_{n\to\infty}[b_n-a_n]=0$.
	Then because $f$ is strictly increasing, $f(a_n) < f(x) < f(b_n)$.\\
	Now, we have the sequences $f(a_n)$ and $f(b_n)$ in $im(f)$ such that for each $n$, $f(a_n) < f(x) < f(b_n)$, and $\lim_{n\to\infty}[f^{-1}(f(b_n))-f^{-1}(f(a_n))]=\lim_{n\to\infty}[b_n-a_n]=0$. The results from Problem 35 tells us that $f^{-1}$ is continuous at $f(x)$.
	\item Let $f$ be a continuous function and $B$ be a Borel set. Show that $f^{-1}(B)$ is a Borel set. (Hint: the collection of sets $E$ for which $f^{-1}(E)$ is Borel is a $\sigma$-algebra containing the open sets.)\\
	\\Let $S=\{E\ |\ f^{-1}(E)\text{ is Borel}\}$.\\
	To show that $S$ is a $\sigma$-algebra, know that the Borel sets is a $\sigma$-algebra.\\
	Observe that:
	\begin{enumerate}[label=(\roman*),align=left]
		\item $f^{-1}(\emptyset)=\emptyset\implies\emptyset\in S$.
		\item $E\in S\implies f^{-1}(E)\text{ is Borel }\implies f^{-1}(E)^c=f^{-1}(E^c)\text{ is Borel }\implies E^c\in S$.
		\item $E_k\in S\implies f^{-1}(E_k)\text{ is Borel }\implies\bigcup_{k=1}^\infty f^{-1}(E_k)=f^{-1}(\bigcup_{k=1}^\infty E_k)\text{ is Borel }\implies \bigcup_{k=1}^\infty E_k\in S$.
	\end{enumerate}
	Also, any open set $\mathcal{O}$ is in $S$ because $f^{-1}(\mathcal{O})$ is open and thus Borel.
	Thus $S$ is a $\sigma$-algebra containing the open sets; that is, the Borel $\sigma$-algebra is a subset of $S$.
	Therefore for any Borel set $B$, $B\in S$ and thus $f^{-1}(B)$ is Borel.
	\item Use the preceding two problems to show that a continuous strictly increasing function that is defined on an interval maps Borel sets to Borel sets.\\
	\\Let $I$ be an interval and $f:I\to\mathbb{R}$ be a continuous strictly increasing function.
	\\By Problem 41, we showed that $f^{-1}:im(f)\to I$ exists and is continuous.
	\\Let $B\in I$ be any Borel set. 
	By Problem 42, $(f^{-1})^{-1}(B)=f(B)$ is a Borel set.
\end{enumerate}	

% Chapter 3
\chapter{Lebesgue Measurable Functions}

% 3.1
\section{Sums, Products, and Compositions}
\begin{center}
	\textbf{PROBLEMS}
\end{center}
\begin{enumerate}
	\setcounter{enumi}{0}
	\item Suppose $f$ and $g$ are continuous functions on $[a,b]$. Show that if $f=g$ a.e. on $[a,b]$, then, in fact, $f=g$ on $[a,b]$.
    Is a similar assertion true if $[a,b]$ is replaced by a general measurable set $E$?
    \item Let $D$ and $E$ be measurable sets and $f$ a function with domain $D\cup E$. We proved that $f$ is measurable on $D\cup E$ iff its restrictions to $D$ and $E$ are measurable.
    Is the same true if "measurable" is replaced by "continuous"?
    \item Suppose a function $f$ has a measurable domain and is continuous except at a finite number of points.
    Is $f$ necessarily measurable?
    \item Suppose $f$ is a real-valued function on $\mathbb{R}$ such that $f^{-1}(c)$ is measurable for each number $c$. Is $f$ necessarily measurable?
    \item Suppose the function $f$ is defined on a measurable set $E$ and $\{x\in E\ |\ f(x)>c\}$ is a measurable set for each rational number $c$. Is $f$ necessarily a measurable function?
    \item Let $f$ be a function with measurable domain $D$. Show that $f$ is measurable iff the function $g$ defined on $\mathbb{R}$ by $g(x)=f(x)$ for $x\in D$ and $g(x)=0$ for $x\notin D$ is measurable.
    \item Let the function $f$ be defined on a measurable set $E$. Show that $f$ is measurable iff for each borel set $A$, $f^{-1}(A)$ is measurable. (Hint: the collection of sets $A$ that have the property that $f^{-1}(A)$ is measurable is a $\sigma$-algebra.)
    \item (Borel measurability) A function $f$ is said to be \textbf{Borel measurable} provided its domain $E$ is a Borel set and for each $c$, the set $\{x\in E\ |\ f(x)>c\}$ is a Borel set.
    Verify that Proposition 1 and Theorem 6 remain valid if we replace "(Lebesgue) measurable set" by "Borel set".
    Show that:
    \begin{enumerate}[label=(\roman*),align=left]
        \item every Borel measurable function is Lebesgue measurable
        \item if $f$ is Borel measurable and $B$ is a Borel set, then $f^{-1}(B)$ is a Borel set
        \item if $f$ and $g$ are Borel measurable, so is $f\circ g$, 
        \item if $f$ is Borel measurable and $g$ is Lebesgue measurable, then $f\circ g$ is Lebesgue measurable.
    \end{enumerate}
    \item Let $\{f_n\}$ be a sequence of measurable functions defined on a measurable set $E$.
    Define $E_0$ to be the set of points of $x$ in $E$ at which $\{f_n(x)\}$ converges. Is the set $E_0$ measurable?
    \item Suppose $f$ and $g$ are real-valued functions defined on all of $\mathbb{R}$, $f$ is measurable, and $g$ is continuous.
    Is the composition $f\circ g$ necessarily measurable?
    \item Let $f$ be a measurable function and $g$ be a one-to-one function from $\mathbb{R}$ onto $\mathbb{R}$ which has a Lipschitz inverse. Show that the composition $f\circ g$ is measurable. (Hint: examine Problem 37 in Chapter 2.)
\end{enumerate}

% 3.2
\section{Sequential Pointwise Limits and Simple Approximation}
\begin{center}
	\textbf{PROBLEMS}
\end{center}
\begin{enumerate}
	\setcounter{enumi}{11}
    \item Let $f$ be a bounded measurable function on $E$. Show that there are sequences of simple functions on $E$, $\{\varphi_n\}$ and $\{\psi_n\}$, such that $\{\varphi_n\}$ is increasing and $\{\psi_n\}$ is decreasing and each of these sequences converges to $f$ uniformly on $E$.
    \item A real-valued measurable function is said to be \textit{semisimple} provided it takes only a countable number of values. Let $f$ be any measurable function on $E$.
    Show that there is a sequence of semisimple functions $\{f_n\}$ on $E$ that converges to $f$ uniformly on $E$.
    \item Let $f$ be a measurable function on $E$ that is finite a.e. on $E$ and $m(E)<\infty$.
    For each $\epsilon>0$, show that there is a measurable set $F$ contained in $E$ such that $f$ is bounded on $F$ and $m(E\setminus F)<\epsilon$.
    \item Let $f$ be a measurable function on $E$ that if finite a.e. on $E$ and $m(E)<\infty$. Show that for each $\epsilon>0$, there is a measurable set $F$ contained in $E$ and a sequence $\{\varphi_n\}$ of simple functions on $E$ such that $\{\varphi_n\}\to f$ uniformly on $F$ and $m(E\setminus F)<\epsilon$. (Hint: see the preceding problem.)
    \item Let $I$ be a closed, bounded interval and $E$ a measurable subset of $I$. Let $\epsilon>0$.
    Show that there is a step function $h$ on $I$ and a measurable subset $F$ of $I$ for which 
    \[
        h=\chi_E\text{ on }F\text{ and }m(I\setminus F)<\epsilon.    
    \]
    (Hint: use Theorem 12 of Chapter 2.)
    \item Let $I$ be a closed, bounded interval and $\psi$ a simple function defined on $I$. Let $\epsilon>0$.
    Show that there is a step function $h$ on $I$ and a measurable subset $F$ of $I$ for which 
    \[
        h=\psi\text{ on }F\text{ and }m(I\setminus F)<\epsilon.    
    \] 
    (Hint: use the fact that a simple function is a linear combination of characteristic functions and the preceding problem.)
    \item Let $I$ be a closed, bounded interval and $f$ a bounded measurable function defined on $I$. Let $\epsilon>0$.
    Show that there is a step function $h$ on $I$ and a measurable subset $F$ of $I$ for which 
    \[
        |h-f|<\epsilon\text{ on }F\text{ and }m(I\setminus F)<\epsilon.    
    \]
    \item Show that the sum and product of two simple functions are simple as are the max and the min.
    \item Let $A,B$ be any sets. Show that
    \begin{align*}
        \chi_{A\cap B}&=\chi_A\cdot\chi_B\\
        \chi_{A\cup B}&=\chi_A+\chi_B-\chi_A\cdot\chi_B\\
        \chi_{A^c}&=1-\chi_A\\
    \end{align*}
    \item For a sequence $\{f_n\}$ of measurable functions with common domain $E$, show that each of the following functions is measurable:
    \begin{itemize}
        \item $\inf\{f_n\}$
        \item $\sup\{f_n\}$
        \item $\lim\inf\{f_n\}$
        \item $\lim\sup\{f_n\}$
    \end{itemize}
    \item (Dini's Theorem) Let $\{f_n\}$ be an increasing sequence of continuous functions on $[a,b]$ which converges pointwise on $[a,b]$ to the continuous function $f$ on $[a,b]$.
    Show that the convergence is uniform on $[a,b]$. (Hint: let $\epsilon>0$. For each natural number $n$, define $E_n=\{x\in[a,b]\ |\ f(x)-f_n(x)<\epsilon\}$. Show that $\{E_n\}$ is an open cover of $[a,b]$ and use the Heine-Borel Theorem.)
    \item Express a measurable function as the difference of nonnegative measurable functions and thereby prove the general Simple Approximation Theorem based on the special case of a nonnegative measurable function.
    \item Let $I$ be an interval and $f:I\to\mathbb{R}$ be increasing. Show that $f$ is measurable by first showing that, for each natural number $n$, the strictly increasing function $x\mapsto f(x)+x/n$ is measurable, and then taking pointwise limits.
\end{enumerate}

% 3.3
\section{Littlewood's Three Principles, Ergoff's Theorem, and Lusin's Theorem`'}
\begin{center}
	\textbf{PROBLEMS}
\end{center}
\begin{enumerate}
	\setcounter{enumi}{24}
    \item Suppose $f$ is a function that is continuous on a closed set $F$ of real numbers. Show that $f$ has a continuous extension to all of $\mathbb{R}$. This is a special case of the forthcoming Tietze Extension Theorem.
    (Hint: express $\mathbb{R}\setminus F$ as the union of a countable disjoint collection of open intervals and define $f$ to be linear on the closure of each of these intervals.)
    \item For the function $f$ and the set $F$ in the statement of Lusin's Theorem, show that the restriction of $f$ to $F$ is a continuous function.
    Must there be any points at which $f$, considered as a function of $E$, is continuous?
    \item Show that the conclusion of Egoroff's Theorem can fail if we drop the assumption that the domain has finite measure.
    \item Show that Egoroff's Theorem continues to hold if the convergence is pointwise a.e. and $f$ is finite a.e.
    \item Prove the extension of Lusin's Theorem to the case that $E$ has finite measure.
    \item Prove the extension of Lusin's Theorem to the case that $f$ is not necessarily real-valued, but is finite a.e.
    \item Let $\{f_n\}$ be a sequence of measurable functions on $E$ that converges to the real-valued $f$ pointwise on $E$. Show that $E=\bigcup_{k=1}^\infty E_k$, where for each index $k$, $E_k$ is measurable, and $\{f_n\}$ converges uniformly to $f$ on each $E_k$ if $k>1$, and $m(E_1)=0$.  
\end{enumerate}
% Chapter 4
\chapter{Lebesgue Integration}
% 4.1
\section{The Riemann Integral}

\begin{center}
	\textbf{PROBLEMS}
\end{center}
\begin{enumerate}
	\setcounter{enumi}{0}
    \item Show that, in the above Dirichlet function example, $\{f_n\}$ fails to converge to $f$ uniformly on $[0,1]$.
    \item A partition $P'$ of $[a,b]$ is called a refinement of a partition $P$ provided each partition point of $P$ is also a partition point of $P'$.
    For a bounded function $f$ on $[a,b]$, show that under refinement lower Darboux sums increase and upper Darboux sums decrease.
    \item Use the preceding problem to show that for a bounded function on a closed, bounded interval, each lower Darboux sum is no greater than each upper Darboux sum.
    From this conclude that the lower Riemann integral is no greater than the upper Riemann integral.
    \item Suppose the bounded function $f$ on $[a,b]$ is Riemann integrable over $[a,b]$.
    Show that there is a sequence $\{P_n\}$ of partitions of $[a,b]$ for which $\lim_{n\to\infty}[U(f,P_n)-L(f,P_n)]=0$.
    \item Let $f$ be a bounded function on $[a,b]$. Suppose there is a sequence $\{P_n\}$ of partitions of $[a,b]$ for which $\lim_{n\to\infty}[U(f,P_n)-L(f,P_n)]=0$. Show that $f$ is Riemann integrable over $[a,b]$.
    \item Use the preceding problem to show that since a continuous function $f$ on a closed, bounded interval $[a,b]$ is uniformly continuous on $[a,b]$, it is Riemann integrable over $[a,b]$.
    \item Let $f$ be an increasing real-valued function on $[0,1]$. For a natural number $n$, define $P_n$ to be the partition of $[0,1]$ into $n$ subintervals of length $1/n$. 
    Show that $U(f,P_n)-L(f,P_n)\le 1/n[f(1)-f(0)]$. Use Problem 5 to show that $f$ is Riemann integrable over $[0,1]$.
    \item Let $\{f_n\}$ be a sequence of bounded functions that converges uniformly to $f$ on the closed, bounded interval $[a,b]$. 
    If each $f_n$ is Riemann integrable over $[a,b]$, show that $f$ also is Riemann integrable over $[a,b]$. Is it true that
    \[
        \lim_{n\to\infty}\int_a^bf_n=\int_a^bf?  
    \]
\end{enumerate}

% 4.2
\section{The Lebesgue Integral of a Bounded Measurable Function over a Set of Finite Measure}
\begin{flushleft}
    
\begin{namedthm*}{Remark}
    Prior to the proof of the Bounded Convergence Theorem, no use was made of the countable additivity of the Lebesgue measure on the real line.
    Only finite additivity was used, and it was used just once, in the proof of Lemma 1. Bur for the proof of the Bounded Convergence Theorem we used Egoroff's Theorem.
    Egoroff's Theorem needed the continuity of Lebesgue measure, a consequence of countable additivity of Lebesgue measure.    
\end{namedthm*}

\end{flushleft}
\begin{center}
	\textbf{PROBLEMS}
\end{center}
\begin{enumerate}
	\setcounter{enumi}{8}
    \item Let $E$ have measure zero. Show that if $f$ is a bounded function on $E$, then $f$ is measurable and $\int_Ef=0$.
    \item Let $f$ be a bounded measurable function on a set of finite measure $E$. For a measurable subset $A$ of $E$, show that $\int_Af=\int_Ef\cdot\chi_A$.
    \item Does the Bounded Convergence Theorem hold for the Riemann integral?
    \item Let $f$ be a bounded measurable function on a set of finite measure $E$. Assume $g$ is bounded and $f=g$ a.e. on $E$. Show that $\int_Ef=\int_Eg$.
    \item Does the Bounded Convergence Theorem hold if $m(E)<\infty$ but we drop the assumption that the sequence $\{|f_n|\}$ is uniformly bounded on $E$?
    \item Show that Proposition 8 is a special case of the Bounded Convergence Theorem.
    \item Verify the assertions in the last Remark of this section.
    \item Let $f$ be a nonnegative bounded measurable function on a set of finite measure $E$. Assume $\int_Ef=0$. Show that $f=0$ a.e. on $E$.
\end{enumerate}

% 4.3
\section{The Lebesgue Integral of a Measurable Nonnegative Function}

\begin{center}
	\textbf{PROBLEMS}
\end{center}
\begin{enumerate}
	\setcounter{enumi}{16}
    \item Let $E$ be a set of measure zero and define $f\equiv\infty$ on $E$. Show that $\int_Ef=0$.
    \item Show that the integral of a bounded measurable function of finite support is properly defined.
    \item For a number $\alpha$, define $f(x)=x^\alpha$ for $0<x\le1$, and $f(0)=0$. Compute $\int_0^1f$.
    \item Let $\{f_n\}$ be a sequence of nonnegative measurable functions that converges to $f$ pointwise on $E$.
    Let $M\ge0$ be such that $\int_Ef_n\le M$ for all $n$. Show that $\int_Ef\le M$. Verify that this property is equivalent to the statement of Fatou's Lemma.
    \item Let the function $f$ be nonnegative and integrable over $E$ and $\epsilon>0$. Show there is a simple function $\eta$ on $E$ that has finite support, $0\le\eta\le f$ on $E$ and $\int_E|f-\eta|<\epsilon$.
    If $E$ is a closed, bounded interval, show there is a step function $h$ on $E$ that has finite support and $\int_E|f-h|<\epsilon$.
    \item Let $\{f_n\}$ be a sequence of nonnegative measurable functions on $\mathbb{R}$ that converges pointwise on $\mathbb{R}$ to $f$ and $f$ be integrable over $\mathbb{R}$. Show that
    \[
        \text{if }\int_{\mathbb{R}}f=\lim_{n\to\infty}\int_{\mathbb{R}}f_n,\text{ then }\int_Ef=\lim_{n\to\infty}\int_Ef_n\text{ for any measurable set }E.    
    \]
    \item Let $\{a_n\}$ be a sequence of nonnegative real numbers. Define the function $f$ on $E=[1,\infty)$ by setting $f(x)=a_n$ if $n\le x<n+1$. Show that $\int_Ef=\sum_{n=1}^\infty a_n$.
    \item Let $f$ be a nonnegative measurable function on $E$.
    \begin{enumerate}[label=(\roman*),align=left]
        \item Show there is an increasing sequence $\{\varphi_n\}$ of nonnegative simple functions on $E$, each of finite support, which converges pointwise on $E$ to $f$.
        \item Show that $\int_Ef=\sup\{\int_E\varphi\ |\ \varphi\text{ simple, of finite support and }0\le\varphi\le f\text{ on }E\}$.
    \end{enumerate}
    \item Let $\{f_n\}$ be a sequence of nonnegative measurable functions on $E$ that converges pointwise on $E$ to $f$. Suppose $f_n\le f$ on $E$ for each $n$. Show that
    \[
        \lim_{n\to\infty}\int_Ef_n=\int_Ef.
    \]
    \item Show that the Monotone Convergence Theorem may not hold for decreasing sequences of functions.
    \item Prove the following generalization of Fatou's Lemma: If $\{f_n\}$ is a sequence of nonnegative measurable functions on $E$, then 
    \[
    \int_E\lim\inf f_n\le\lim\inf\int_Ef_n.     
    \]
\end{enumerate}
    
% 4.4
\section{The General Lebesgue Integral}
\begin{center}
	\textbf{PROBLEMS}
\end{center}
\begin{enumerate}
	\setcounter{enumi}{27}
    \item Let $f$ be integrable over $E$ and let $C$ be a measurable subset of $E$. Show that $\int_Cf=\int_Ef\cdot\chi_C$.
    \item For a measurable function $f$ on $[1,\infty)$ which is bounded on bounded sets, define $a_n=\int_n^{n+1}f$ for each natural number $n$.
    Is it true that $f$ is integrable over $[1,\infty)$ iff the series $\sum_{n=1}^\infty a_n$ converges?
    Is it true that $f$ is integrable over $[1,\infty)$ iff the series $\sum_{n=1}^\infty a_n$ converges absolutely?
    \item Let $g$ be a nonnegative integrable function over $E$ and suppose $\{f_n\}$ is a sequence of measurable functions on $E$ such that for each $n$, $|f_n|\le g$ a.e. on $E$. Show that
    \[
        \int_E\lim\inf f_n \le \lim\inf\int_E f_n \le \lim\sup\int_E f_n \le \int_E\lim\sup f_n.
    \]
    \item Let $f$ be a measurable function on $E$ which can be expressed as $f=g+h$ on $E$, where $g$ is finite and integrable over $E$ and $h$ is nonnegative on $E$.
    Define $\int_Ef=\int_Eg+\int_Eh$. Show that this is properly defined in the sense that it is independent of the particular choice of finite integrable function $g$ and nonnegative function $h$ whose sum is $f$.
    \item Prove the General Lebesgue Dominated Convergence Theorem by following the proof of the Lebesgue Dominated Convergence Theorem, but replacing the sequences $\{g-f_n\}$ and $\{g+f_n\}$, respectively, by $\{g_n-f_n\}$ and $\{g_n+f_n\}$.
    \item Let $\{f_n\}$ be a sequence of integrable functions on $E$ for which $f_n\to f$ a.e. on $E$ and $f$ is integrable over $E$. Show that $\int_E|f-f_n|\to0$ iff $\lim_{n\to\infty}\int_E|f_n|=\int_E|f|$.
    (Hint: use the General Lebesgue Dominated Convergence Theorem.)
    \item Let $f$ be a nonnegative measurable function on $\mathbb{R}$. Show that 
    \[
        \lim_{n\to\infty}\int_{-n}^nf=\int_{\mathbb{R}}f.
    \]   
    \item Let $f$ be a real-valued function of two variables $(x,y)$ that is defined on the square $Q=\{(x,y)\ |\ 0\le x\le 1,0\le y\le 1\}$ and is a measurable function of $x$ for each fixed value of $y$.
    Suppose for each fixed value of $x$, $\lim_{y\to0}f(x,y)=f(x)$ and that for all $y$, we have $|f(x,y)|\le g(x)$, where $g$ is integrable over $[0,1]$. Show that
    \[
    \lim_{y\to0}\int_0^1f(x,y)dx=\int_0^1f(x)dx.    
    \] 
    Also show that if the function $f(x,y)$ is continuous in $y$ for each $x$, then 
    \[
        h(y)=\int_0^1f(x,y)dx  
    \]
    is a continuous function of $y$.
    \item Let $f$ be a real-valued function of two variables $(x,y)$ that is defined on the square $Q=\{(x,y)\ |\ 0\le x\le 1,0\le y\le 1\}$ and is a measurable function of $x$ for each fixed value of $y$.
    For each $(x,y)\in Q$ let the partial derivative $\partial f/\partial y$ exist. Suppose there is a function $g$ that is integrable over $[0,1]$ and such that 
    \[
        \biggl | \frac{\partial f}{\partial y}(x,y) \biggr | \le g(x)\text{ for all }(x,y)\in Q.
    \]
    Prove that 
    \[
        \frac{d}{dy}\biggl[\int_0^1f(x,y)dx\biggr]=\int_0^1\frac{\partial f}{\partial y}(x,y)dx\text{ for all }y\in [0,1].
    \]
\end{enumerate}

% 4.5
\section{Countable Additivity and Continuity of Integration}
\begin{center}
	\textbf{PROBLEMS}
\end{center}
\begin{enumerate}
	\setcounter{enumi}{36}
    \item Let $f$ be an integrable function on $E$. Show that for each $\epsilon>0$, there is a natural number $N$ for which if $n\ge N$, then $|int_{E_n}f|<\epsilon$ where $E_n=\{x\in E\ |\ |x|\ge n\}$.
    \item For each of the two functions $f$ on $[1,\infty)$ defined below, show that $\lim_{n\to\infty}\int_1^nf$ exists while $f$ is not integrable over $[1,\infty)$. Does this contradict the continuity of integration?
    \begin{enumerate}[label=(\roman*),align=left]
        \item Define $f(x)=\frac{(-1)^n}{n}$, for $n\le x < n+1$.
        \item Define $f(x) = \frac{(\sin x)}{x}$ for $1\le x<\infty$.
    \end{enumerate}
    \item Prove the theorem regarding the continuity of integration.
\end{enumerate}

% 4.6
\section{Uniform Integrability: The Vitali Convergence Theorem}
\begin{center}
	\textbf{PROBLEMS}
\end{center}
\begin{enumerate}
	\setcounter{enumi}{39}
    \item Let $f$ be integrable over $\mathbb{R}$. Show that the function $F$ defined by 
    \[
        F(x) = \int_{-\infty}^xf\text{ for all }x\in\mathbb{R}
    \]
    is properly defined and continuous. Is it necessarily Lipschitz?
    \item Show that Proposition 25 is false if $E=\mathbb{R}$.
    \item Show that Theorem 26 is false without the assumption that the $h_n$'s are nonnegative.
    \item Let the sequences of functions $\{h_n\}$ and $\{g_n\}$ be uniformly integrable over $E$. Show that for any $\alpha$ and $\beta$, the sequence of linear combinations $\{\alpha f_n + \beta g_n\}$ also is uniformly integrable over $E$.
    \item Let $f$ be integrable over $\mathbb{R}$ and let $\epsilon>0$. Establish the following three approximation properties. 
    \begin{enumerate}[label=(\roman*),align=left]
        \item There is a simple function $\eta$ on $\mathbb{R}$ which has finite support and $\int_{\mathbb{R}}|f-\eta|<\epsilon$. (Hint: first verify this if $f$ is nonnegative.)
        \item There is a step function $s$ on $\mathbb{R}$ which vanishes outside a closed, bounded interval and $\int_{\mathbb{R}}|f-s|<\epsilon$. (Hint: apply part (i) and Problem 18 of Chapter 3.)
        \item There is a continuous function $g$ on $\mathbb{R}$ which vanishes outside a bounded set and $\int_{\mathbb{R}}|f-g|<\epsilon$.
    \end{enumerate}
    \item Let $f$ be integrable over $E$. Define $\hat f$ to be the extension of $f$ to all of $\mathbb{R}$ obtained by setting $\hat f \equiv 0$ outside of $E$. 
    Show that $\hat f$ is integrable over $\mathbb{R}$ and 
\end{enumerate}
% Chapter 5
\chapter{Lebesgue Integration: Further Topics}

% 5.1
\section{Uniform Integrability and Tightness: A General Vitali Convergence Theorem}
\begin{center}
	\textbf{PROBLEMS}
\end{center}
\begin{enumerate}
	\setcounter{enumi}{0}
    \item Let $\{f_n\}_{k=1}^n$ be a finite family of functions, each of which is integrable over $E$.
    Show that $\{f_n\}_{k=1}^n$ is uniformly integrable and tight over $E$.
    \item Prove Corollary 2.
    \item Let the sequences of functions $\{h_n\}$ and $\{g_n\}$ be uniformly integrable and tight over $E$.
    Show that for any $\alpha$ and $\beta$, $\{\alpha f_n +\beta g_n\}$ is also uniformly integrable and tight over $E$.
    \item Let $\{f_n\}$ be a sequence of measurable functions on $E$. Show that f$\{f_n\}$ is uniformly integrable and tight over $E$ iff for each $\epsilon>0$, there is a measurable subset $E_0$ of $E$ that has finite measure and a $\delta>0$ such that for each measurable subset $A$ of $E$ and index $n$,
    \[
        \text{if }m(A\cap E_0)<\delta,\text{ then }\int_A|f_n|<\epsilon.  
    \]
    \item Let $\{f_n\}$ be a sequence of measurable functions on $\mathbb{R}$. Show that f$\{f_n\}$ is uniformly integrable and tight over $\mathbb{R}$ iff for each $\epsilon>0$, there are positive numbers $r$ and $\delta$ such that for each open subset $\mathcal{O}$ of $\mathbb{R}$ and index $n$
    \[
        \text{if }m(\mathcal{O}\cap(-r,r))<\delta,\text{ then }\int_{\mathcal{O}}|f_n|<\epsilon.  
    \]
\end{enumerate}

% 5.2
\section{Convergence in Measure}
\begin{center}
	\textbf{PROBLEMS}
\end{center}
\begin{enumerate}
	\setcounter{enumi}{5}
    \item Let $\{f_n\}\to f$ in measure on $E$ and let $g$ be a measurable function on $E$ that is finite a.e. on $E$. 
    Show that $\{f_n\}\to g$ in measure on $E$ iff $f=g$ a.e. on $E$.
    \item Let $E$ have finite measure, let $\{f_n\}\to f$ in measure on $E$ and let $g$ be a measurable function on $E$ that is finite a.e. on $E$.
    Prove that $\{f_n\cdot g\}\to f\cdot g$ in measure, and use this to show that $\{f_n^2\}\to f^2$ in measure.
    Infer from this that if $\{g_n\}\to g$ in measure, then $\{f_n\cdot g_n\}\to f\cdot g$ in measure.
    \item Show that Fatou's Lemma, the Monotone Convergence Theorem, the Lebesgue Dominated Convergence Theorem, and the Vitali Convergence Theorem remain valid if "pointwise convergence a.e." is replaced by "convergence in measure".
    \item Show that Proposition 3 does not necessarily hold for sets $E$ of infinite measure.
    \item Show that linear combinations of sequences that converge in measure on a set of finite measure also converge in measure.
    \item Assume $E$ has finite measure. Let $\{f_n\}$ be a sequence of measurable functions on $E$ and let $f$ be a measurable function on $E$ for which $f$ and each $f_n$ is finite a.e. on $E$.
    Prove that $\{f_n\}\to f$ in measure on $E$ iff every subsequence of $\{f_n\}$ has in turn a further subsequence that converges to $f$ pointwise a.e. on $E$.
    \item Show that a sequence $\{a_j\}$ of real numbers converges to a real number if $|a_{j+1}-a_j|\le\frac{1}{2^j}$ for all $j$ by showing that the sequence $\{a_j\}$ must be Cauchy.
    \item A sequence $\{f_n\}$ of measurable functions on $E$ is said to be \textbf{Cauchy in measure} provided that given $\eta>0$ and $\epsilon>0$, there is an index $N$ such that for all $m,n\ge N$, 
    \[
        m\{x\in E\ |\ |f_n(x)-f_m(x)|\ge\eta\}<\epsilon.  
    \]
    Show that if $\{f_n\}$ is Cauchy in measure, then there is a measurable function $f$ on $E$ to which the sequence $\{f_n\}$ converges in measure.
    (Hint: choose a strictly increasing sequence of natural numbers $\{n_j\}$ such that for each index $j$, if $E_j=\{x\in E\ |\ |f_{n_{j+1}}(x)-f_{n_j}(x)|>\frac{1}{2^j}\}$, then $m(E_j)<\frac{1}{2^j}$. Now use the Borel-Cantelli Lemma and the preceding problem.)
    \item Assume $m(E)<\infty$. For two measurable functions $g$ and $h$ on $E$, Define
    \[
        \rho(g,h)=\int_E\frac{|g-h|}{1+|g-h|}.
    \]
    Show that $\{f_n\}\to f$ in measure on $E$ iff $\lim_{n\to\infty}\rho(f_n,f)=0$.
\end{enumerate}

% 5.3
\section{Characterizations of Riemann and Lebesgue Integrability}
\begin{center}
	\textbf{PROBLEMS}
\end{center}
\begin{enumerate}
	\setcounter{enumi}{14}
    \item Let $f$ and $g$ be bounded functions that are Riemann integrable over $[a,b]$. Show that the product $fg$ is also Riemann integrable over $[a,b]$.
    \item Let $f$ be a bounded function on $[a,b]$ whose set of discontinuities has measure zero. Show that $f$ is measurable. Then show that the same holds without the assumption of boundedness.
    \item Let $f$ be a function on $[0,1]$ that is continuous on $(0,1]$. Show that it is possible for the sequence $\{\int_{[1/n,1]}f\}$ to converge and yet $f$ is not Lebesgue integrable over $[0,1]$. Can this happen if $f$ is nonnegative?
\end{enumerate}
% Chapter 6
\chapter{Differentiation and Integration}

% 6.1
\section{Continuity of Monotone Functions}

% 6.2
\section{Differentiability of Monotone Functions: Lebesgue's Theorem}

% 6.3
\section{Functions of Bounded Variation: Jordan's Theorem}

% 6.4
\section{Absolutely Continuous Functions}

% 6.5
\section{Integrating Derivatives: Differentiating Indefinite Integrals}

% 6.6
\section{Convex Functions}

\begin{center}
	\textbf{PROBLEMS}
\end{center}
\begin{enumerate}
	\setcounter{enumi}{60}
	\item Show that a real-valued function $\varphi$ on $(a,b)$ is convex iff for points $x_1,\cdots,x_n$ in $(a,b)$ and nonnegative numbers $\lambda_1,\cdots,\lambda_n$ such that $\sum_{k=1}^n \lambda_k=1$,
	\[
        \varphi \biggr ( \sum_{k=1}^n \lambda_k x_k \biggl ) \le \sum_{k=1}^n \lambda_k \varphi(x_k).   
    \]
    Use this to directly prove Jensen's Inequality for $f$ a simple function.
    \item Show that a continuous function on $(a,b)$ is convex iff
    \[
        \varphi(\frac{x_1+x_2}{2})\le\frac{\varphi(x_1)+\varphi(x_2)}{2}\text{ for all }x_1,x_2\in (a,b).  
    \]
    \item A function on a general interval $I$ is said to be convex provided it is continuous on $I$ and (38) holds for all $x_1,x_2\in I$.
    Is a convex function on a closed, bounded interval $[a,b]$ necessarily Lipschitz on $[a,b]$?
    \item Let $\varphi$ have a second derivative at each point in $(a,b)$.
    Show that $\varphi$ is convex iff $\varphi''$ is nonnegative.
    \item Suppose $a\ge 0$ and $b\ge 0$.
    Show that the function $\varphi(t)=(a+bt)^p$ is convex on $[0,\infty)$ for $1\le p < \infty$.
    \item For what functions $\varphi$ is Jensen's Inequality always an equality?
    \item State and prove a version of Jensen's Inequality on a general closed, bounded interval $[a,b]$.
    \item Let $f$ be integrable over $[0,1]$. Show that 
    \[
    \exp\biggl [\int_0^1f(x)dx\biggr ] \le \int_0^1 \exp(f(x))dx.   
    \]
    \item Let $\{\alpha_n\}$ be a sequence of nonnegative numbers whose sum is $1$ and $\{\zeta_n\}$ is a sequence of positive numbers. Show that
    \[
    \prod_{n=1}^\infty \zeta_n^{\alpha_n} \le \sum_{n=1}^\infty \alpha_n \zeta_n. 
    \]
    \item Let $g$ be a positive measurable function on $[0,1]$. Show that $\log(\int_0^1g(x)dx) \ge\int_0^1\log(g(x))dx$ whenever each side is defined.
    \item (Nemytskii) Let $\varphi$ be a continuous function on $\mathbb{R}$.
    Show that if there are constants for which (43) holds, then $\varphi\circ f$ is integrable over $[0,1]$ whenever $f$ is.
    Then show that if $\varphi\circ f$ is integrable over $[0,1]$ whenever $f$ is, then there are constants $c_1$ and $c_2$ for which (43) holds.
\end{enumerate}
% Chapter 7
\authoredby{finished}
\chapter{The $L^p$ Spaces: Completeness and Approximation}

% 7.1
%\authoredby{finished}
\section{Normed Linear Spaces}
\begin{flushleft}
	
	Throughout this chapter $E$ denotes a measurable set of real numbers.
	Define $\mathcal{F}$ to be the collection of all measurable extended real-valued functions on $E$ that are finite a.e. on $E$.
	We can say that two functions $f,g\in\mathcal{F}$ are equivalent, denoted by $f\cong g$, provided
	\[
	f(x)=g(x)\text{ for almost all }x\in E.	
	\]
	This is an equivalence relation and induces a partition of $\mathcal{F}$ into a disjoint collection of equivalence classes, denoted by $\mathcal{F}/\cong$, which is a linear space.
	There is a natural family $\{L^p(E)\}_{1\le p\le\infty}$ of subspaces of $\mathcal{F}/\cong$.
	\\
	For $1\le p<\infty$, define $L^p(E)$ to be the collection of equivalence class $[f]$ for which 
	\[
	\int_E|f|^p<\infty.	
	\]
	Then if $f\cong g$, then $\int_E|f|^p=\int_E|g|^p$.
	Showing that $L^p(E)$ is closed under linear combinations will prove that $L^p(E)$ is a linear subspace.
	To do this, let $c=\max\{|a|,|b|\}$ so that
	\[
	|a+b|\le|a|+|b|\le 2c,
	\]
	which implies
	\[
	|a+b|^p\le 2^pc^p\le2^p(|a|^p+|b|^p).	
	\]
	This inequality, together with the linearity and monotonicity of integration tells us that 
	\[
		\int_E|\alpha f+\beta g|^p \le 2^p(|\alpha|^p\int_E|f|^p+|\beta|^p\int_E| g|^p)<\infty.
	\]
	That is, for $[f],[g]\in L^p(E)$, then $\alpha[f]+\beta[g]\in L^p(E)$.\\
	We call a function $f\in\mathcal{F}$ \textbf{essentially bounded} provided there is some $M\ge 0$, called an \textbf{essential upper bound} for $f$, for which
	\[
	|f(x)|\le M\text{ for almost all }x\in E.	
	\]
	Then we can define $L^\infty(E)$ to be the collection of equivalence classes $[f]$ for which $f$ is essentially bounded. 
	Clearly $L^\infty(E)$ is a linear subspace because
	\[
	|\alpha f(x)+\beta g(x)|\le|\alpha| |f(x)|+|\beta|| g(x)|\le |\alpha| M + |\beta|M' = M'' \text{ a.e. on $E$}
	\]
	To state that a function $f$ in $L^p[a,b]$ is continuous means that there is a continuous function that agrees with $f$ a.e. on $[a,b]$.
	There is only one such continuous function and it is often convenient to consider this unique continuous function as the representative of $[f]$.\\
	It is useful to consider real-valued functions that have as their domain linear spaces of functions: such functions are called \textbf{functionals}.

	\begin{namedthm*}{Definition}
		Let $X$ be a linear space.
		A real-valued functional $\|\cdot\|$  on $X$ is called a \textbf{norm} provided for each $f$ and $g$ in $X$ and each real number $\alpha$,\\
		(The Triangle Inequality)
		\[
			\|f+g\|\le\|f\|+\|g\|,
		\]
		(Positive Homogeneity)
		\[
			\|\alpha f\|=|\alpha|\|f\|,	
		\]
		(Nonnegativity)
		\[
			\|f\|\ge 0\text{ and }\|f\|=0 \iff f=0.
		\]
	\end{namedthm*}
	A \textbf{normed linear space} is a linear space together with a norm.
	If $X$ is a linear space normed by $\|\cdot\|$ we say that a function $f$ in $X$ is a \textbf{unit function} provided $\|f\|=1$.
	For any $f\in X,f\neq 0$, the function $\frac{f}{\|f\|}$ is a unit function: it is a scalar multiple of $f$ which we call the \textbf{normalization} of $f$.

	\begin{namedthm*}{Example}[The Normed Linear Space $L^1(E)$]
		For a function $f$ in $L^1(E)$, define
		\[
		\|f\|_1=\int_E|f|.	
		\]
		Then $\|\cdot\|$ is a norm on $L^1(E)$.\\
		For $f,g\in L^1(E)\subseteq \mathcal{F}$, since $f$ and $g$ are finite a.e. on $E$, the triangle inequality for real numbers tells us that
		\[
		|f+g|\le|f|+|g|\text{ a.e. on }E.	
		\]
		Then by the monotonicity and linearity of integration, we have subadditivity:
		\[
			\|f+g\|_1=\int_E|f+g| \le \int_E[|f|+|g|] = \int_E|f| +\int_E|g| = \|f\|_1+\|g\|_1.	
		\]
		By the linearity of integration, clearly we have absolute homogeneity:
		\[
			\|\alpha f\|_1 = \int_E|\alpha f| = \int_E|\alpha| |f|=|\alpha|\int_E| f| = |\alpha|\|f\|_1.	
		\]
		Clearly $\|f\|$ is nonnegative. Finally, if $f\in L^1(E)$ and $\|f\|_1=0$, then $f=0$ a.e. on $E$. 
		Therefore $[f]$ is the zero element of the linear space $L^1(E)\subseteq \mathcal{F}/\cong$, that is $f=0$.
	\end{namedthm*}

	\begin{namedthm*}{Example}[The Normed Linear Space $L^\infty(E)$]
		For a function $f$ in $L^\infty(E)$, define $\|f\|_\infty$ to be the infimum of the essential upper bounds for $f$.
		\[
			\|f\|_\infty = \inf\{M\ :\ |f(x)|\le M\text{ a.e. on }E\}.
		\]
		We call $\|f\|_\infty$ the \textbf{essential supremum} of $f$ and claim that $\|\cdot\|_\infty$ is a norm on $L^\infty(E)$.
		\\Nonnegativity and positive homogeneity are clear. 
		\\To show that the triangle inequality holds, we see that for each natural number $n$, there is a subset $E_n$ of $E$ for which
		\[
		|f|\le \|f\|_\infty + \frac{1}{n}\text{ on }E\setminus E_n\text{ and }m(E_n)=0.	
		\]
		This is true because $\|f\|_\infty$ is the infimum, the greatest lower bound, so $\|f\|_\infty + \frac{1}{n}$ is not a lower bound and thus there exists a real number $M$ in the set of upper bounds a.e. of $f$ for which $\|f\|_\infty\le M < \|f\|_\infty + \frac{1}{n}$ a.e. on $E$, and so $|f|\le M < \|f\|_\infty + \frac{1}{n}$ a.e. on $E$.
		\\Accepting that the union of sets of measure zero is also measure zero, we can let $E_\infty = \bigcup_{n=1}^\infty E_n$, and so 
		\[
		|f|\le \|f\|_\infty \text{ on }E\setminus E_\infty\text{ and }m(E_n\infty)=0.	
		\]
		Thus we have that $|f|\le \|f\|_\infty$ a.e. on $E$; i.e., ess. sup$f$ is the smallest essential upper bound for $f$.
		\\Now, for $f,g\in L^\infty(E)$, 
		\[
			|f+g|\le|f|+|g|\le\|f\|_\infty+\|g\|_\infty\text{ a.e. on }E.
		\]
		Therefore $\|f\|_\infty+\|g\|_\infty$ is an essential bound for $f+g$ and thus the smallest essential upper bound, $\|f+g\|_\infty$, is such that
		\[
		\|f+g\|_\infty \le \|f\|_\infty+\|g\|_\infty.	
		\]
	\end{namedthm*}

	\begin{namedthm*}{Example}[The Normed Linear Spaces $\ell^1$ and $\ell^\infty$]
		For $1\le p<\infty$, define $\ell^p$ to be the collection of real sequences $a=(a_1,a_2,\cdots)$ for which 
		\[
		\sum_{k=1}^\infty|a_k|^p<\infty .	
		\]
		Let $a,b \in \ell^p$, and let $\alpha , \beta$ be real numbers.
		Then we have that $\sum_{k=1}^\infty|a_k|^p<\infty$ and $\sum_{k=1}^\infty|b_k|^p<\infty$.
		Using the inequality $|a+b|^p\le2^p(|a|^p+|b|^p)$, we have
		\begin{align*}
			\sum_{k=1}^\infty|\alpha a_k + \beta b_k|^p &\le \sum_{k=1}^\infty[2^p(|\alpha a_k|^p + |\beta b_k|^p)]\\ 
			&=\sum_{k=1}^\infty2^p|\alpha|^p |a_k|^p+\sum_{k=1}^\infty2^p|\beta|^p| b_k|^p\\
			&=2^p|\alpha|^p\sum_{k=1}^\infty |a_k|^p+2^p|\beta|^p\sum_{k=1}^\infty| b_k|^p\\
			&< 2^p|\alpha|^p\infty+2^p|\beta|^p\infty\\
			&=\infty.
		\end{align*}
		Thus $\ell^p$ is a linear space.\\
		We define $\ell^\infty$ to be the linear space of real bounded sequences: that is, for any $\{a_k\}$ in $\ell^\infty$, there exists a real number $M$ for which $|a_k|\le M$ for all $k$.
		We can define the following norms:
		\\For $\{a_k\}\in\ell^1$:
		\[
			\|\{a_k\}\|_1 = \sum_{k=1}^\infty|a_k|
		\]
		For $\{a_k\}\in\ell^\infty$:
		\[
			\|\{a_k\}\|_\infty = \sup_{1\le k<\infty}|a_k|
		\]
	\end{namedthm*}

	\begin{namedthm*}{Example}[The Normed Linear Space $C\lbrack a,b\rbrack$]
		Let $[a,b]$ be a closed, bounded interval. 
		The the linear space of continuous real-valued functions on $[a,b]$ is denoted by $C[a,b]$. 
		Since a continuous function on a compact set takes on a maximum value (ch1 problem 52), we can define
		\[
		\|f\|_{\max}=\max_{x\in [a,b]}|f(x)|.	
		\]		
	\end{namedthm*}

\end{flushleft}
\begin{center}
	\textbf{PROBLEMS}
\end{center}
\begin{enumerate}
	\setcounter{enumi}{0}
	\item For $f$ in $C[a,b]$, Define
	\[
	\| f \|_1 = \int_a^b |f|.	
	\]
	Show that this is a norm on $C[a,b]$.
	Also show that there is no number $c \ge 0$ for which
	\[
	\| f \|_{\max}	\le c \| f \|_1 \text{ for all $f$ in $C[a,b]$},
	\]
	but there is a $c \ge 0$ for which 
	\[
	\| f \|_1	\le c \| f \|_{\max} \text{ for all $f$ in $C[a,b]$}.
	\]
	\\
	Let $f,g\in C[a,b]$. For each $x\in [a.b]$, we have the inequality $|f(x)+g(x)|\le|f(x)|+|g(x)|$, so by monotonicity and linearity of integration,
	\[
	\|f+g\|_1=\int_a^b|f(x)+g(x)| \le \int_a^b[|f(x)|+|g(x)|] = \int_a^b|f(x)| +\int_a^b|g(x)| = \|f\|_1+\|g\|_1.	
	\]
	Therefore subadditivity holds.\\
	Also, by linearity of integration, we have
	\[
	\|\alpha f\|_1 = \int_a^b|\alpha f| = \int_a^b|\alpha| |f|=|\alpha|\int_a^b| f| = |\alpha|\|f\|_1.	
	\]
	Therefore absolute homogeneity holds.\\
	Finally, by definition of absolute value, $0 \le |f(x)|$ for all $x\in [a,b]$, and by monotonicity of integration,
	\[
	0=\int_a^b 0 \le \int_a^b |f| = \|f\|_1.	
	\] 
	Clearly $\int_a^b |f| = 0$ iff $f\equiv 0$ on $[a,b]$.
	Therefore positive definiteness holds.\\
	Thus $\|\cdot\|_1$ is a norm on $C[a,b]$.\\
	\\Consider the interval $[a,b]=[0,1]$.
	For any $c>0$ we choose, there exists an $n\in \mathbb{N}$ such that $n> c$, with the continuous function $f_n:[0,1]\to \mathbb{R}$ defined as
	\[ 
		f_n(x) =
		\begin{cases} 
			\frac{n-0}{1/n-0}x& \text{ if } x \in [0,\frac{1}{n}]\\
			\frac{0-n}{2/n-1/n}(x-\frac{1}{n})+n & \text{ if } x \in (\frac{1}{n},\frac{2}{n}]\\
			0& \text{ if } x \in (\frac{2}{n},1]
		\end{cases}
		=
		\begin{cases} 
			n^2x& \text{ if } x \in [0,\frac{1}{n}]\\
			-n^2(x-\frac{1}{n})+n & \text{ if } x \in (\frac{1}{n},\frac{2}{n}]\\
			0& \text{ if } x \in (\frac{2}{n},1]
		\end{cases}
	\]
	(This is a triangle-shaped function that reaches its peak $n$ at $x=\frac{1}{n}$.)\\
	Now, for any $n$, we have $\|f_n\|_1 = \int_0^1|f_n|=1$, and $\|f_n\|_{\max} = n$.\\
	Then $\| f_n \|_{\max}=n > c =  c \| f_n\|_1$.\\
	\\
	Finally, we can see that for any $f$ in $C[a,b]$, by monotonicity of the integral, 
	\begin{align*}
	\|f\|_1 &= \int_a^b|f(x)|\\ 
	&\le \int_a^b\max_{x\in [a,b]}|f(x)|\\
	&=\max_{x\in [a,b]}|f(x)| \int_a^b1\\
	&= \max_{x\in [a,b]}|f(x)| \cdot m([a,b]) \\
	&= \|f\|_{\max} \cdot m([a,b]).
	\end{align*}
	Therefore $\|f\|_1\le m([a,b])\|f\|_{\max}$ for all $f\in C[a,b]$.
	\item Let $X$ be the family of all polynomials with real coefficients defined on $\mathbb{R}$.
	Show that this is a linear space. For a polynomial $p$, define $\| p\|$ to be the sum of the absolute values of the coefficients of $p$.
	Is this a norm?\\
	For any two polynomials $p,q\in X$, there exists natural numbers $n,m$ (suppose without loss of generality that $n\le m$) such that
	\begin{align*}
	p(x) &= a_0+a_1x+a_2x^2+\cdots+a_{n-1}x^{n-1}+a_nx^n+\cdots+0x^m	\\
	q(x) &= b_0+b_1x+b_2x^2+\cdots+b_{n-1}x^{n-1}+b_nx^n+\cdots+b_mx^m	
	\end{align*}
	Now, considering any scalars $\alpha,\beta \in \mathbb{R}$, we have
	\begin{align*}
		\alpha p(x) + \beta q(x) &= \alpha (a_0+a_1x+a_2x^2+\cdots+a_{n-1}x^{n-1}+a_nx^n)\\
		&+ \beta (b_0+b_1x+b_2x^2+\cdots+b_{m-1}x^{m-1}+b_mx^m)\\
		&=(\alpha a_0)+(\alpha a_1)x+(\alpha a_2)x^2+\cdots+(\alpha a_{n-1})x^{n-1}+(\alpha a_n)x^n\\
		&+ (\beta b_0)+(\beta b_1)x+(\beta b_2)x^2+\cdots+(\beta b_{n-1})x^{n-1}+(\beta b_n)x^n+\cdots+(\beta b_m)x^m\\
		&=(\alpha a_0+\beta b_0)+(\alpha a_1+\beta b_1)x+\cdots+(\alpha a_n+\beta b_n)x^n+\cdots+(\beta b_m)x^m
	\end{align*}
	This is also a polynomial, as for each $i$, we have $(\alpha a_i+\beta b_i) \in \mathbb{R}$, so $X$ is a linear space.\\
	Now, for any polynomial
	\[
		p(x) = a_0+a_1x+a_2x^2+\cdots+a_nx^n,	
	\]
	we can define $\|p\| = |a_0|+|a_1|+|a_2|+\cdots+|a_n| = \sum_{i=0}^n|a_i|$.\\
	The triangle inequality is clear because 
	\[
		\|p+q\| = \sum_{i=0}^{m}|a_i+b_i|\le\sum_{i=0}^{m}[|a_i|+|b_i|]=\sum_{i=0}^{m}|a_i|+\sum_{i=0}^{m}|b_i|=\|p\|+\|q\|.
	\]
	Absolute homogeneity is clear because
	\[
		\|\alpha p\| = \sum_{i=0}^n|\alpha a_i|= \sum_{i=0}^n|\alpha|| a_i|=|\alpha|\sum_{i=0}^n| a_i|=|\alpha|\|p\|.
	\]
	Finally, positive definiteness is clear because
	\[
		0 \le |a_i| \implies 0\le \sum_{i=0}^n| a_i|=\|p\|,
	\]
	And $\|p\|=0$ if and only if $p(x)= 0+0x+0x^2+\cdots0x^n=0$.
	\item For $f$ in $L^1[a,b]$, define $\|f\| = \smallint_a^b x^2 |f(x)|dx$.
	Show that this is a norm on $L^1[a,b]$.\\
	For $f\in L^1[a,b]$, then $f$ is measurable and finite a.e. on $[a,b]$, and $\int_a^b|f(x)|dx<\infty$.\\
	Let $f,g\in L^1[a,b]$, and let $\alpha$ be a real number.\\
	Because the triangle inequality holds a.e. on $[a,b]$, by monotonicity and linearity of the integral, we have
	\begin{align*}
	\|f+g\|&=\int_a^b x^2 |f(x)+g(x)|dx\\
	&\le\int_a^b x^2 [|f(x)|+|g(x)|]dx\\
	&=\int_a^b [x^2 |f(x)|+x^2|g(x)|]dx\\
	&=\int_a^b x^2 |f(x)|dx+\int_a^b x^2 |g(x)|dx\\
	&= \|f\|+\|g\|.	
	\end{align*}
	Therefore $\|\cdot\|$ is subadditive.\\
	By linearity of the integral, we have
	\[
	\|\alpha f\| = \int_a^bx^2|\alpha f(x)|dx = \int_a^bx^2|\alpha| |f(x)|dx=|\alpha |\int_a^bx^2 |f(x)|dx	=|\alpha |\|f\|.
	\]
	Therefore $\|\cdot\|$ satisfies absolute homogeneity.\\
	We can use the fact that $0\le x^2$ and $0\le |f(x)|$ implies $0\le x^2|f(x)|$.
	By monotonicity of the integral, we have
	\[
	0=\int_a^b0dx \le \int_a^bx^2|f(x)|dx = \|f\|.
	\]
	Clearly $\|f\|=0$ if and only if $f=0$ a.e. on $[a,b]$ because $x^2\cdot 0 = 0$.\\
	Therefore $\|\cdot \|$ satisfies positive definiteness.
	\item For $f$ in $L^\infty[a,b]$, show that 
	\[
	\| f\|_\infty = \min \biggl \{ M \ \biggl |\ m \{x \in [a,b]\ |\ |f(x)| > M \} =0 \biggr \},
	\] 
	\\
	That is, the sup norm is the smallest real number $M$ such that $|f(x)|>M$ only on a set of measure zero.
	In an above example, we showed that $\|f\|_\infty$ is the smallest essential upper bound for $f$.
	That is, $|f|\le\|f\|_\infty$ a.e. on $E$ (That is, the inequality is true for $E\setminus E_0$, where $m(E_0)=0$.)
	\\
	and if, furthermore, $f$ is continuous on $[a,b]$, that
	\[
	\| f \|_{\infty} = \| f \|_{\max}.	
	\]
	If $f$ is continuous, then there are no jump discontinuities ($f$ is continuous at $x_0$ iff $f(x_0^-)=f(x_0)=f(x_0^+)$).
	Then $|f|\le\|f\|_\infty$ everywhere on $E$.
	\item Show that $\ell^\infty$ and $\ell^1$ are normed linear spaces.\\
	$\ell^\infty$:\\
	Let $a,b \in \ell^\infty$, and let $\alpha , \beta$ be real numbers.\\
	Then for some real numbers $M,N$, we have that $|a_k|\le M$ and $|b_k|\le N$ for all $k$.
	\begin{align*}
		\alpha a + \beta b &= \alpha (a_1,a_2,\cdots)+ \beta (b_1,b_2,\cdots)\\		
		&= (\alpha a_1,\alpha a_2,\cdots)+ (\beta b_1,\beta b_2,\cdots)\\
		&= (\alpha a_1+\beta b_1,\alpha a_2+\beta b_2,\cdots)
	\end{align*}
	Then $|\alpha a_k+\beta b_k|\le \alpha M + \beta N$ for all $k$, and $\ell^\infty$ is a linear space.\\
	To show that $\|a\|_\infty = \sup_{1\le k<\infty}|a_k|$ is a norm:
	\[
		\|a+b\|_\infty = \sup_{1\le k<\infty}|a_k+b_k|\le \sup_{1\le i<\infty}|a_i| + \sup_{1\le j<\infty}|b_j| = \|a\|_\infty + \|b\|_\infty,
	\]
	\[
		\|\alpha a\|_\infty = \sup_{1\le k<\infty}|\alpha a_k| = \sup_{1\le k<\infty}|\alpha|| a_k|= |\alpha|\sup_{1\le k<\infty}| a_k|=|\alpha|\|a\|_\infty, 
	\]
	\[
		0 \le \sup_{1\le k<\infty}| a_k| = \|a\|_\infty,\text{ and }\sup_{1\le k<\infty}| a_k|=0\text{ iff }a_k=0\text{ for all }k.	
	\]
	\\
	$\ell^1$:\\
	Let $a,b \in \ell^1$, and let $\alpha , \beta$ be real numbers.\\
	Then we have that $\sum_{k=1}^\infty|a_k|<\infty$ and $\sum_{k=1}^\infty|b_k|<\infty$.\\
	By the triangle inequality for real numbers, we have
	\[
		\sum_{k=1}^\infty |\alpha a_k + \beta b_k| \le\sum_{k=1}^\infty[ |\alpha|| a_k| + | \beta ||b_k|]= |\alpha|\sum_{k=1}^\infty | a_k| + |\beta |\sum_{k=1}^\infty |b_k| <|\alpha|\infty+|\beta|\infty = \infty.
	\]
	Therefore $\ell^1$ is a linear space.\\
	To show that $\|a\|_1 = \sum_{k=1}^\infty|a_k|$ is a norm:
	\[
		\|a+b\|_1 = \sum_{k=1}^\infty|a_k+b_k|\le\sum_{k=1}^\infty[|a_k|+|b_k|]=\sum_{k=1}^\infty|a_k|+\sum_{k=1}^\infty|b_k|<\infty +\infty = \infty,
	\]
	\[
		\|\alpha a\|_1 = \sum_{k=1}^\infty|\alpha a_k| = \sum_{k=1}^\infty|\alpha|| a_k|=|\alpha|\sum_{k=1}^\infty| a_k|=|\alpha|\|a\|_1, 
	\]
	\[
		0 \le | a_k| \implies 0 \le \sum_{k=1}^\infty |a_k| = \|a\|_1,\text{ and }\sum_{k=1}^\infty| a_k|=0\text{ iff }a_k=0\text{ for all }k.	
	\]
\end{enumerate}

% 7.2
\authoredby{inprogress}
\section{The Inequalities of Young, H\"older, and Minkowski}

\begin{center}
	\textbf{PROBLEMS}
\end{center}
\begin{enumerate}
	\setcounter{enumi}{5}
	\item Show that if H\"older's Inequality is true for normalized functions it is true in general.
	\item Verify the assertions in the above two examples regarding the membership of the function $f$ in $L^p(E)$. 
	\item Let $f$ and $g$ belong to $L^2(E)$. From the linearity of integration show that for any number $\lambda$,
	\[
		\lambda^2\int_Ef^2+2\lambda\int_Ef\cdot g+\int_Eg^2=\int_E(\lambda f+g)^2\ge0.	
	\] 
	From this and the quadratic formula directly derive the Cauchy-Schwarz Inequality.
	\item Show that in Young's Inequality there is equality iff $a^p=b^q$.
	\item Show that in H\"older's Inequality there is equality iff there are constants $\alpha,\beta$ not both zero, for which
	\[
		\alpha|f|^p=\beta|g|^q\text{ a.e. on }E.	
	\]
	For a point $x=(x_1,x_2,\cdots,x_n)$ in $\mathbb{R}^n$, define $T_x$ to be the step function on the interval ...
\end{enumerate}

% 7.3
\authoredby{inprogress}
\section{$L^p$ is Complete: The Riesz-Fischer Theorem}
\begin{namedthm*}{Definition}
	A sequence $\{f_n\}$ in a linear space $X$ that is normed by $\|\cdot\|$ is said to \textbf{converge to $f$ in $X$} provided
	\[
		\lim_{n\to\infty}\|f-f_n\|=0.
	\]
	We write 
	\[
		\{f_n\}\to f\text{ in }X\text{ or }\lim_{n\to\infty}f_n=f\text{ in }X
	\]
	to mean that each $f_n$ and $f$ belong to $X$ and $\lim_{n\to\infty}\|f-f_n\|=0$.
\end{namedthm*}
For a sequence $\{f_n\}$ and a function $f$ in $C[a,b]$, $\{f_n\}\to f$ in $C[a,b]$, normed by the maximum norm, iff $\{f_n\}\to f$ uniformly on $[a,b]$.

For a sequence $\{f_n\}$ and a function $f$ in $L^\infty(E)$, $\{f_n\}\to f$ in $L^\infty(E)$ iff $\{f_n\}\to f$ uniformly on the complement of a set of measure zero.

For a sequence $\{f_n\}$ and a function $f$ in $L^p(E)$, $1\le p<\infty$, $\{f_n\}\to f$ in $L^p(E)$ iff $\lim_{n\to\infty}\int_E|f_n-f|^p=0.$
\begin{namedthm*}{Definition}
	A sequence $\{f_n\}$ in a linear space $X$ that is normed by $\|\cdot\|$ is said to be \textbf{Cauchy} in $X$ provided for each $\epsilon>0$, there is a natural number $N$ such that
	\[
		\|f_n-f_m\|<\epsilon\text{ for all }m,n\ge N.
	\]
	A normed linear space $X$ is said to be \textbf{complete} provided every Cauchy sequence in $X$ converges to a function in $X$.
	A complete normed linear space is called a \textbf{Banach space}.
\end{namedthm*}
\begin{namedthm*}{Proposition 4}
	Let $X$ be a normed linear space.
	Then every convergent sequence in $X$ is Cauchy.
	Moreover, a Cauchy sequence in $X$ converges if it has a convergent subsequence.
\end{namedthm*}
\begin{proof}
	Let $\{f_n\}\to f$ in $X$.
	By the triangle inequality of the norm,
	\[
		\|f_n-f_m\|=\|f_n-f+f-f_m\|\le\|f_n-f\|+\|f-f_m\|\text{ for all }m,n.
	\]
	Therefore $\{f_n\}$ is Cauchy.

	Now let $\{f_n\}$ be a Cauchy sequence in $X$ that has a subsequence $\{f_{n_k}\}$ which converges in $X$ to $f$.
	Let $\epsilon>0$.
	Since $\{f_n\}$ is Cauchy, there exists an $N\in\mathbb{N}$ such that $\|f_n-f_m\|<\epsilon/2$ for all $n,m\ge N$.
	Then, since $\{f_{n_k}\}$ converges to $f$ we can choose $k$ such that $n_k>N$ so that $\|f_{n_k}-f\|<\epsilon/2$.
	Therefore we have
	\[
		\|f_n-f\|=\|f_n-f_{n_k}+f_{n_k}-f\|\le\|f_n-f_{n_k}\|+\|f_{n_k}-f\|<\epsilon/2+\epsilon/2\text{ for }n\ge N.
	\]
	Therefore $\{f_n\}\to f$ in $X$.
\end{proof}
\begin{namedthm*}{Definition}
	Let $X$ be a linear space normed by $\|\cdot\|$.
	A sequence $\{f_n\}$ in $X$ is said to be \textbf{rapidly Cauchy} provided there is a convergent series of positive numbers $\sum_{k=1}^\infty\epsilon_k$ for which 
	\[
		\|f_{k+1}-f_k\|\le\epsilon_k^2\text{ for all }k.
	\]
\end{namedthm*}
\begin{namedthm*}{Proposition 5}
	Let $X$ be a normed linear space.
	Then every rapidly Cauchy sequence in $X$ is Cauchy.
	Furthermore, every Cauchy sequence has a rapidly Cauchy subsequence.
\end{namedthm*}
\begin{proof}
	Let $\{f_n\}$ be a rapidly Cauchy sequence in $X$ and $\sum_{k=1}^\infty\epsilon_k$ a convergent series of positive numbers for which 
	\[
		\|f_{k+1}-f_k\|\le\epsilon_k^2\text{ for all }k.
	\]
	We have by telescoping, 
	\[
		f_{n+k}-f_n=\sum_{j=n}^{n+k-1}[f_{j+1}-f_j]\text{ for all }n,k,
	\]
	and therefore by subadditivity of the norm,
	\[
		\|f_{n+k}-f_n\|=\sum_{j=n}^{n+k-1}\|f_{j+1}-f_j\|\le\sum_{j=n}^{n+k-1}\epsilon_j^2\le\sum_{j=n}^\infty\epsilon_j^2\text{ for all }n,k.
	\]
	For $\varepsilon_1=1>0$, there exists $N_1\in\mathbb{N}$ such that $\sum_{j=p}^\infty\epsilon_j<1$ for all $p\ge N_1$.
	This implies $0<\epsilon_j<1$ for each $j\ge p$, which gives us $\epsilon_j^2<\epsilon_j$ so that $\sum_{j=p}^\infty\epsilon_j^2<\sum_{j=p}^\infty\epsilon_j<\infty$, which implies that $\sum_{j=1}^\infty\epsilon_j^2$ is a convergent series of positive numbers.
	
	Fix $\varepsilon_2>0$.
	Then because $\sum_{j=1}^\infty\epsilon_j^2$ is convergent, there exists an index $N_2\in\mathbb{N}$ such that for all $n\ge N_2$, then 
	\[
		\|f_{n+k}-f_n\|\le\sum_{j=n}^\infty\epsilon_j^2=\sum_{j=1}^\infty\epsilon_j^2-\sum_{j=1}^{n-1}\epsilon_j^2<\varepsilon_2\text{ for all }k.
	\]
	Therefore $\{f_n\}$ is Cauchy.

	Now assume that $\{f_n\}$ is a Cauchy sequence in $X$.
	\\\underline{Base case: $P(1)$}:
	\\Because $\{f_n\}$ is Cauchy, there exists an index $N$ such that for $n_1,n_2\ge N$ with $n_1<n_2$, then 
	\[
		\|f_{n_2}-f_{n_1}\|\le\frac{1}{2^3}<\frac{1}{2^1},
	\]
	and for all $n\ge n_{2}$, we have
	\[
		\|f_{n_2}-f_n\|<\frac{1}{2^3}.
	\]
	\underline{Inductive hypothesis: Suppose $P(k)$}: 
	\\suppose that we already have $n_1,\dots,n_k$ such that
	\[
		\|f_{n_{k}}-f_{n_{k-1}}\|<1/2^{k-1},
	\]
	and for all $n\ge n_{k}$, we have
	\[
		\|f_{n_{k}}-f_{n}\|<1/2^{k+1}.\tag{1}
	\]
	\underline{$P(k+1)$}:
	\\Because $\{f_n\}$ is Cauchy, there exists an index $N$ such that for $m,n\ge N$, then 
	\[
		\|f_m-f_n\|<1/2^{k+2}.\tag{2}
	\]
	Then choose $n_{k+1}>\max\{N,n_k\}$ so that by (2) and (1), for all $n\ge n_{k+1}$ we have
	\[
		\|f_{n_{k+1}}-f_{n_{k}}\|\le\|f_{n_{k+1}}-f_n\|+\|f_n-f_{n_{k}}\|\le\frac{1}{2^{k+2}}+\frac{1}{2^{k+1}}=(\frac{1}{2^2}+\frac{1}{2})\cdot\frac{1}{2^{k}}\le\frac{1}{2^{k}},
	\]
	and because $n_{k+1}\ge N$, for all $n\ge n_{k+1}$ we have
	\[
		\|f_{n_{k+1}}-f_{n}\|<1/2^{k+2}.
	\]
	Therefore we have shown that we can inductively choose a strictly increasing sequence $\{n_k\}$ such that 
	\[
		\|f_{n_{k+1}}-f_{n_k}\|\le\frac{1}{2^k}\text{ for all }k.
	\]
	and $\sqrt{2^k}=(2^k)^{1/2}=\sqrt{2}^k$ for each $k$ tells us that $\sum_{k=1}^\infty\frac{1}{\sqrt{2^k}}=\sum_{k=1}^\infty(\frac{1}{\sqrt{2}})^k$, which converges because $\left|\frac{1}{\sqrt{2}}\right|<1$.
	Therefore $\{f_{n_k}\}$ is rapidly Cauchy.
\end{proof}
\begin{namedthm*}{Theorem 6}
	Let $E$ be a measurable set and $1\le p\le\infty$.
	Then every rapidly Cauchy sequence in $L^p(E)$ converges both w.r.t. the $L^p(E)$ norm and pointwise a.e. on $E$ to a function in $L^p(E)$.
\end{namedthm*}

\begin{center}
	\textbf{PROBLEMS}
\end{center}
\begin{enumerate}
	\setcounter{enumi}{22}
	\item Provide an example of a Cauchy sequence of real numbers that is not rapidly Cauchy.
	\item Let $X$ be a normed linear space.
	Assume that $\{f_n\}\to f$ in $X$, $\{g_n\}\to g$ in $X$, and $\alpha$ and $\beta$ are real numbers.
	Show that
	\[
		\{\alpha f_n+\beta g_n\}\to\alpha f+\beta g\text{ in }X.
	\]
	\item Assume that $E$ has finite measure and $1\le p_1< p_2\le\infty$.
	Show that if $\{f_n\}\to f$ in $L^{p_2}(E)$, then $\{f_n\}\to f$ in $L^{p_1}(E)$.
	\item (The $L^p$ Dominated Convergence Theorem) Let $\{f_n\}$ be a sequence of measurable functions that converges pointwise a.e. on $E$ to $f$.
	For $1\le p<\infty$, suppose there is a function $g$ in $L^p(E)$ such that for all $n$, $|f_n|\le g$ a.e. on $E$.
	Prove that $\{f_n\}\to f$ in $L^p(E)$.
	\item For $E$ a measurable set and $1\le p<\infty$, assume $\{f_n\}\to f$ in $L^p(E)$.
	Show that there is a subsequence $\{f_{n_k}\}$ and a function $g\in L^p(E)$ for which $|f_{n_k}|\le g$ a.e. on $E$ for all $k$.
	\item Assume $E$ has finite measure and $1\le p<\infty$.
	Suppose $\{f_n\}$ is a sequence of measurable functions that converges pointwise a.e. on $E$ to $f$.
	For $1\le p<\infty$, show that $\{f_n\}\to f$ in $L^p(E)$ if there is a $\theta>0$ such that $\{f_n\}$ belongs to and is bounded as a subset of $L^{p+\theta}(E)$.
	\item Consider the linear space of polynomials on $[a,b]$ normed by $\|\cdot\|_{\max}$ norm.
	Is this normed linear space a Banach space?
	\item Let $\{f_n\}$ be a sequence in $C[a,b]$ and $\sum_{k=1}^\infty a_k$ a convergent series of positive numbers such that
	\[
		\|f_{k+1}-f_k\|_{\max}\le a_k\text{ for all }k.
	\]
	Prove that
	\[
		|f_{n+k}(x)-f_n(x)|\le\|f_{n+k}-f_n\|_{\max}\le\sum_{j=n}^\infty a_j\text{ for all }k,n\text{ and all }x\in[a,b].
	\]
	Conclude that there is a function $f\in C[a,b]$ such that $\{f_n\}\to f$ uniformly on $[a,b]$.
	\item Use the preceding problem to show that $C[a,b]$, normed by the maximum norm, is a Banach space.\\
	\\Note: See Chapter 16 Problem 1 that $C[a,b]$ normed by the $L^2[a,b]$ norm is not a Banach space.
	In particular, we had a sequence of continuous functions in $[a,b]$ that was Cauchy (w.r.t. the $L^2$ norm ) but converged ($L^2$) to a discontinuous function in $[a,b]$.
	We again consider this sequence of functions $\{f_n\}$ and now show that it is no longer Cauchy (w.r.t. the maximum norm) and thus does not converge.
	Recall the definition of each function:
	\[
        f_n(x):=
        \begin{cases}
            0 &x\le t\\
            n(x-t)&t<x<t+\frac{1}{n}\\
            1 &x\ge t+\frac{1}{n}
        \end{cases}
    \]
	Fix $\epsilon=\frac{1}{2}$.
	For any $N\in\mathbb{N}$, consider $n\ge N$ and $m=3n>n\ge N$. 
	\\Then we have
	\[
        (f_m-f_n)(x)=
        \begin{cases}
            0-0 &x\le t\\
            m(x-t)-n(x-t)&x\in(t,t+\frac{1}{m})\\
            1-n(x-t)&x\in[t+\frac{1}{m},t+\frac{1}{n})\\
            1-1 &x\ge t+\frac{1}{n}
        \end{cases}
	\]
	then we can clearly see that the maximum occurs at $t+\frac{1}{m}$ so that 
	\[
		\|f_m-f_n\|_{\max}=\max_{x\in[a,b]}|(f_m-f_n)(x)|=1-n((t+\frac{1}{m})-t)=1-\frac{n}{3n}=\frac{2}{3}>\frac{1}{2},
	\]
	and the sequence is not Cauchy.\\
	\\Back to the proof of Problem 31:
	\\Consider the linear space $C[a,b]$ with the maximum norm. 
	Suppose that $\{f_n\}$ is a Cauchy sequence of functions in this space.
	By Proposition 5, there exists a rapidly Cauchy subsequence $\{f_{n_k}\}$.
	That is, there is a convergent series of positive numbers $\sum_{k=1}^\infty\epsilon_k$ for which
	\[
		\|f_{n_{k+1}}-f_{n_k}\|_{\max}\le\epsilon_k^2\text{ for all }k.
	\]
	Then $\sum_{k=1}^\infty\epsilon_k$ converges implies that $\sum_{k=1}^\infty\epsilon_k^2$ also converges.
	\\Therefore by the previous Problem 30, there is a function $f\in C[a,b]$ such that $\{f_{n_k}\}\to f$ uniformly on $[a,b]$.
	Then by Proposition 4, $\{f_n\}$ converges because it has a convergent subsequence $\{f_{n_k}\}$. 
	\item Let $\{f_n\}$ be a sequence in $L^\infty(E)$ and $\sum_{k=1}^\infty a_k$ a convergent series of positive numbers such that
	\[
		\|f_{k+1}-f_k\|_{\infty}\le a_k\text{ for all }k.
	\]
	Prove that there is a subset $E_0$ of $E$ which has measure zero and 
	\[
		|f_{n+k}(x)-f_n(x)|\le\|f_{n+k}-f_n\|_{\infty}\le\sum_{j=n}^\infty a_j\text{ for all }k,n\text{ and all }x\in E\setminus E_0.
	\]
	Conclude that there is a function $f\in L^\infty(E)$ such that $\{f_n\}\to f$ uniformly on $E\setminus E_0$.
	\item Use the preceding problem to show that $L^\infty(E)$ is a Banach space.\\
	\\Consider the linear space $L^\infty(E)$ with the supremum norm. 
	Suppose that $\{f_n\}$ is a Cauchy sequence of functions in this space.
	By Proposition 5, there exists a rapidly Cauchy subsequence $\{f_{n_k}\}$.
	That is, there is a convergent series of positive numbers $\sum_{k=1}^\infty\epsilon_k$ for which
	\[
		\|f_{n_{k+1}}-f_{n_k}\|_{\infty}\le\epsilon_k^2\text{ for all }k.
	\]
	Then $\sum_{k=1}^\infty\epsilon_k$ converges implies that $\sum_{k=1}^\infty\epsilon_k^2$ also converges.
	\\Therefore by the previous Problem 32, there is a function $f\in L^\infty(E)$ and a set $E_0\subseteq E$ of measure zero such that $\{f_{n_k}\}\to f$ uniformly on $E\setminus E_0$.
	Then by Proposition 4, $\{f_n\}$ converges because it has a convergent subsequence $\{f_{n_k}\}$. 
	\item Prove that for $1\le p\le\infty$, $\ell^p$ is a Banach space.
	\item Show that the space $c$ of all convergent sequences of real numbers and the space $c_0$ of all sequences that converge to zero are Banach spaces w.r.t. the $\ell^\infty$ norm. 
\end{enumerate}

% 7.4
\authoredby{untouched}
\section{Approximation and Separability}

% Chapter 8
\authoredby{inprogress}
\chapter{The $L^p$ Spaces: Duality and Weak Convergence}

% 8.1
\authoredby{inprogress}
\section{The Riesz Representation for the Dual of $L^p,a\le p\le \infty$}

\textbf{Example}
Let $E$ be a measurable set, $1\le p<\infty$, $q$ the conjugate of $p$, and $g$ belong to $L^q(E)$.
Define the functional $T$ on $L^p(E)$ by
\[
    T(f)=\int_Eg\cdot f\text{ for all }f\in L^p(E).
\]
H\"older's Inequality tells us that for $f\in L^p(E)$, the product $g\cdot f$ is integrable over $E$ so the functional $T$ is properly defined.
By the linearity of integration, $T$ is linear.
Observe that H\"older's inequality is the statement that 
\[
    |T(f)|\le\|g\|_q\cdot\|f\|_p\text{ for all }f\in L^p(E).
\]
\begin{flushleft}

\textbf{Example}
Let $[a,b]$ be a closed, bounded interval and the function $g$ be of bounded variation on $[a,b]$.
Define the functional $T$ on $C[a,b]$ by
\[
    T(f)=\int_a^bf(x)dg(x)\text{ for all }f\in C[a,b],
\]  
where the integral is in the sense of Riemann-Stieltjes.
The functional $T$ is properly defined and linear.
Moreover, it follows immediately from the definition of this integral that 
\[
    |T(f)|\le TV(g)\cdot\|f\|_{\max}\text{ for all }f\in C[a,b],
\]  
where $TV(g)$ is the total variation of $g$ over $[a,b]$.
\end{flushleft}
\begin{namedthm*}{Definition}
    For a normed linear space $X$, a linear functional $T$ on $X$ is said to be \textbf{bounded} provided there is an $M\ge0$ for which 
\[
    |T(f)|\le M\|f\|\text{ for all }f\in X.
\]
The infimum of all such $M$ is called the \textbf{norm} of $T$ and denoted by $\|T\|_*$.
\end{namedthm*}
The inequalities in the first and second example above tell us that the linear functionals are bounded.

Let $T$ be a bounded linear functional on the normed linear space $X$.
It is easy to see that the above equation holds for $M=\|T\|_*$.
Hence, by the linearity of $T$,
\[
    |T(f)-T(h)|\le \|T\|_*\|f-h\|\text{ for all }f,h\in X.
\]
From this we infer the following continuity property of a bounded linear functional $T$:
\[
    \text{if }\{f_n\}\to f\text{ in }X,\text{ then }\{T(f_n)\}\to T(f).\tag{7}
\]
We leave it as an exercise (see chapter 13 problem 11) to show that 
\[
    \|T\|_*=\sup\{T(f)\mid f\in X,\|f\|\le1\},\tag{8}
\]
and use this characterization of $\|\cdot\|_*$ to prove the following proposition.
\begin{namedthm*}{Proposition 1}
    Let $X$ be a normed linear space.
    Then the collection of bounded linear functionals on $X$ is a linear space on which $\|\cdot\|_*$ is a norm.
    This normed linear space is called the \textbf{dual space} of $X$ and denoted by $X^*$.
\end{namedthm*}
\begin{namedthm*}{Proposition 2}
    Let $E$ be a measurable set, $1\le p<\infty$, $q$ the conjugate of $p$, and $g$ belong to $L^q(E)$.
    Define the functional $T$ on $L^p(E)$ by
    \[
        T(f)=\int_Eg\cdot f\text{ for all }f\in L^p(E).
    \]
    Then $T$ is a bounded linear functional on $L^p(E)$ and $\|T\|_*=\|g\|_q$.
\end{namedthm*}
\begin{proof}
    By linearity of integration, $T$ is linear, and from H\"older's inequality, we infer that $T$ is bounded and $\|T\|_*\le\|g\|_q$.
    For $p>1$, according to Chapter 7 Theorem 1, the conjugate function of $g$, $g^*=\|g\|_q^{1-q}\text{sgn}(g)|g|^{q-1}$, belongs to $L^p(E)$, and 
    \[
        T(g^*)=\int_Eg\cdot g^*=\|g\|_q\quad\text{and}\quad\|g^*\|_p=1.
    \] 
    Then by definition of the operator norm as a supremum (8), we have $\|T\|_*\ge T(g^*)=\|g\|_q$.
    Thus from the two inequalities we obtain $\|T\|_*=\|g\|_q$.
    For $p=1$, we argue by contradiction.
    If we have strict inequality, i.e., $\|g\|_\infty>\|T\|_*$, then by definition of supremum norm there is a set $A$ of finite positive measure on which $|g|>\|T\|_*$.
    Define $f:=[1/m(A)][\text{sgn}(g)]\chi_A$. 
    Then $\|f\|_1=1/m(A)\int_E\chi_A=1$ and yet, using the fact that $g\cdot\text{sgn}(g)=|g|$ and monotonicity of integration,
    \[
        T(f)=1/m(A)\int_E|g|\chi_A>1/m(A)\int_E\|T\|_*\chi_A=\|T\|_*,
    \]
    which is a contradiction to the fact that $\|T\|_*$ is the supremum.

\end{proof}
Our goal now is to prove that for $1\le p<\infty$, every bounded linear functional on $L^p(E)$ is given by integration against a function in $L^q(E)$, where $q$ is the conjugate of $p$.

\begin{center}
	\textbf{PROBLEMS}
\end{center}
\begin{enumerate}
	\setcounter{enumi}{0}
    \item Verify (8).
    
    \ \\(See Chapter 13 Problem 11) 
    Define 
    \[
        \begin{split}
        M'&:=\inf\{M\ge0\mid\|T(f)\|\le M\|f\|\text{ for all }f\in X\},\\
        N'&:=\sup\{\|T(f)\|\mid f\in X, \|f\|\le1\}.
        \end{split}
    \]
    We aim to show that they are equal.
    
    First, for $f\neq0$, then $\|\frac{f}{\|f\|}\|\le1$ so that by linearity of $T$,
    \[\|T(\frac{f}{\|f\|})\|\le N'\implies\|Tf\|\le N'\|f\|\implies M'\le N'.\]
    On the other hand, for $f$ such that $\|f\|\le1$, then
    \[\|Tf\|\le M'\|f\|\le M'\implies N'\le M.\]
    Therefore $M'=N'$.
    \\\item Prove Proposition 1.
    \item Let $T$ be a linear functional on a normed linear space $X$. Show that $T$ is bounded iff the continuity property (7) holds.
    \item A functional $T$ on a normed linear space $X$ is said to be Lipschitz provided there is a $c\ge0$ such that
    \[
        |T(g)-T(h)|\le c\|g-h\|\text{ for all }g,h\in X.  
    \]
    The infimum of such $c$'s is called the Lipschitz constant for $T$. Show that a linear functional is bounded iff it is Lipschitz, in which case its Lipschitz constant is $\|T\|_*$.
    \item Let $E$ be a measurable set and $1\le 0<\infty$. Show that the functions in $L^p(E)$ that vanish outside a bounded set are dense in $L^p(E)$. Show that this is false for $L^\infty(\mathbb{R})$.
    \item Establish the Riesz Representation Theorem in the case $p=1$ by first showing, in the notation of the proof of the theorem, that the function $\Phi$ is Lipschitz and therefore it is absolutely continuous. Then follow the $p>1$ proof.
    \item State and prove a Riesz Representation Theorem for the bounded linear functionals on $\ell^p$, $1\le p<\infty$.
    \item Let $c$ be the linear space of real sequences that converge to a real number and $c_0$ the subspace of $c$ comprising sequences that converge to $0$. Norm each of these linear spaces with the $\ell^\infty$ norm. Determine the dual space of $c$ and $c_0$.
    \item Let $[a,b]$ be a closed, bounded interval and $C[a,b]$ be normed by the maximum norm. Let $x_0$ belong to $[a,b]$. Define the linear functional $T$ on $C[a,b]$ by $T(f)=f(x_0)$. Show that $T$ is bounded and is given by Riemann-Stieltjes integration against a function of bounded variation.
    \item Let $f$ belong to $C[a,b]$. Show that there is a function $g$ that is of bounded variation on $[a,b]$ for which 
    \[
        \int_a^bfdg=\|f\|_{\max}\text{ and }TV(g)=1.  
    \]
    \item Let $[a,b]$ be a closed, bounded interval and $C[a,b]$ be normed by the maximum norm. Let $T$ be a bounded linear functional on $C[a,b]$.
    For $x\in[a,b]$, let $g_x$ be the member of $C[a,b]$ that is linear on $[a,x]$ and on $[x,b]$ with $g_x(a)=0,g_x(x)=x-a$ and $g_x(b)=x-a$. Define $\Phi(x)=T(g_x)$ for $x\in[a,b]$. Show that $\Phi$ is Lipschitz on $[a,b]$.
\end{enumerate}

% 8.2
\authoredby{untouched}
\section{Weak Sequential Convergence in $L^p$}
\begin{center}
	\textbf{PROBLEMS}
\end{center}
\begin{enumerate}
	\setcounter{enumi}{11}
    \item f
\end{enumerate}

% 8.3
\section{Weak Sequential Compactness}
\begin{center}
	\textbf{PROBLEMS}
\end{center}
\begin{enumerate}
	\setcounter{enumi}{36}
    \item f
\end{enumerate}

% 8.4
\section{The Minimization of Convex Functionals}
\begin{center}
	\textbf{PROBLEMS}
\end{center}
\begin{enumerate}
	\setcounter{enumi}{36}
    \item f
\end{enumerate}

% Chapter 9
\chapter{Metric Spaces: General Properties}

% 9.1
\section{Examples of Metric Spaces}
\begin{flushleft}

"The object of the present chapter is to study general spaces called metric spaces for which the notion of distance between two points is fundamental."

\begin{namedthm*}{Definition}
    Let $X$ be a nonempty set. A function $\rho:X\times X\to\mathbb{R}$ is called a \textbf{metric} provided for all $x,y,z\in X$,
    \begin{enumerate}[label=(\roman*),align=right]
        \item $\rho(x,y)\ge0$,
        \item $\rho(x,y)=0$ iff $x=y$,
        \item $\rho(x,y)=\rho(y,x)$,
        \item $\rho(x,y)\le\rho(x,z)+\rho(z,y)$.
    \end{enumerate}
    A nonempty set together with a metric on the set is called a \textbf{metric space}, often denoted by $(X,\rho)$.
\end{namedthm*}

An example of a metric space is the set $\mathbb{R}$ of all real numbers with the metric $\rho(x,y) = |x-y|$. 

\end{flushleft}
\begin{center}
	\textbf{PROBLEMS}
\end{center}
\begin{enumerate}
	\setcounter{enumi}{0}
	\item Show that two metrics $\rho$ and $\tau$ on the same set $X$ are equivalent iff there is a $c>0$ such that for all $u,v\in X$,
	\[
        \frac{1}{c}\tau(u,v)\le\rho(u,v)\le c\tau(u,v).    
    \]
    \item Show that the following define equivalent metrics on $\mathbb{R}^n$:
    \begin{align*}
        \rho^*(x,y) &= |x_1-y_1| + \cdots + |x_n-y_n|;\\
        \rho^+(x,y) &= \max\{|x_1-y_1|, \cdots,|x_n-y_n|\}.
    \end{align*}
    \item Find a metric on $\mathbb{R}^n$ that fails to be equivalent to either of those defined in the preceding problem.
    \item For a closed, bounded interval $[a,b]$, consider the set $X=C[a,b]$ of continuous real-valued functions on $[a,b]$.
    Show that the metric induced by the maximum norm and that induced by the $L^1[a,b]$ norm are not equivalent.
    \item \textit{The Nikodym Metric}. Let $E$ be a Lebesgue measurable set of real numbers of finite measure, $X$ the set of measurable subsets of $E$, and $m$ Lebesgue measure.
    For $A,B\in X$, define $\rho(A,B)=m(A\Delta B)$, where $A\Delta B = [A\setminus B]\cup[B\setminus A]$, the symmetric difference of $A$ and $B$.
    Show that this is a pseudometric on $X$.
    Define two measurable sets to be equivalent provided their symmetric difference has measure zero.
    Show that $\rho$ induces a metric on the collection of equivalence classes.
    Finally, show that for $A,B\in X$, 
    \[
    \rho(A,B)=\int_E|\chi_A-\chi_B|,    
    \]
    where $\chi_A$ and $\chi_B$ are the characteristic functions of $A$ and $B$, respectively.
    \item Show that for $a,b,c\ge0$,
    \[
    \text{if }a\le b+c,\text{ then }\frac{a}{1+a}\le\frac{b}{1+b}+\frac{c}{1+c}.
    \]
    \item Let $E$ be a Lebesgue measurable set of real numbers that has finite measure and $X$ the set of Lebesgue measurable real-valued functions on $E$.
    For $f,g\in X$, define
    \[
    \rho(f,g)=\int_E\frac{|f-g|}{1+|f-g|}.    
    \]
    Use the preceding problem to show that this is a pseudometric on $X$. 
    Define two measurable functions to be equivalent provided they are equal a.e. on $E$.
    Show that $\rho$ induces a metric on the collection of equivalence classes.
    \item For $0<p<1$, show that 
    \[
    (a+b)^p\le a^p+b^p\text{ for all }a,b\ge0.    
    \]\
    \item For $E$ a Lebesgue measurable set of real numbers, $0<p<1$, and $g,h$ Lebesgue measurable functions on $E$ that have integrable $p^{th}$ powers, define
    \[
    \rho_p(h,g)=\int_E|g-h|^p.    
    \]  
    Use the preceding problem to show that this is a pseudometric on the collection of Lebesgue measurable functions on $E$ that have integrable $p^{th}$ powers.
    Define two such functions to be equivalent provided they are equal a.e. on $E$. 
    Show that $\rho_p(\cdot,\cdot)$ induces a metric on the collection of equivalence classes.
    \item Let $\{(X_n,\rho_n)\}_{n=1}^\infty$ be a countable collection of metric spaces.
    Use problem 6 to show that $\rho_*$ defines a metric on the Cartesian product $\prod_{n=1}^\infty X_n$, where for points $x = \{x_n\}$ and $y= \{y_n\}$ in $\prod_{n=1}^\infty X_n$,
    \[
    \rho_*(x,y)=\sum_{n=1}^\infty\frac{1}{2^n}\cdot\frac{\rho_n(x_n,y_n)}{1+\rho_n(x_n,y_n)}.    
    \]
    \item Let $(X,\rho)$ be a metric space and $A$ any set for which there is a one-to-one (injective) mapping $f$ of $A$ onto (surjective?) the set $X$ (bijection?).
    Show that there is a unique metric on $A$ for which $f$ is an isometry of metric spaces.
    (This is the sense in which an isometry amounts merely to a relabeling of the points in a space.)
    \item Show that the triangle inequality for Euclidean space $\mathbb{R}^n$ follows from the triangle inequality for $L^2[0,1]$.
\end{enumerate}

% 9.2
\section{Open Sets, Closed Sets, and Convergent Sequences}

% 9.3
\section{Continuous Mappings Between Metric Spaces}

% 9.4
\section{Complete Metric Spaces}

% 9.5
\section{Compact Metric Spaces}

% 9.6
\section{Separable Metric Spaces}


% Chapter 10
\chapter{Metric Spaces: Three Fundamental Theorems}

\section{The Arzel\'a-Ascoli Theorem}
\section{The Baire Category Theorem}
\section{The Banach Contraction Principle}

% Chapter 11
\chapter{Topological Spaces: General Properties}

\section{Open Sets, Closed Sets, Bases, and Subbases}
\section{The Separation Properties}
\section{Countability and Separability}
\section{Continuous Mappings Between Topological Spaces}
\section{Compact Topological Spaces}
\section{Connected Topological Spaces}

% Chapter 12
\chapter{Topological Spaces: Three Fundamental Theorems}

\section{Urysohn's Lemma and the Tietze Extension Theorem}
\section{The Tychonoff Product Theorem}
\section{Thye Stone-Weierstrass Theorem}

% Chapter 13
\chapter{Continuous Linear Operators Between Banach Spaces}

% 13.1
\section{Normed Linear Spaces}
\begin{center}
	\textbf{PROBLEMS}
\end{center}
\begin{enumerate}
	\setcounter{enumi}{0}
    \item Show that a nonempty subset $S$ of a linear space $X$ is a subspace iff $S+S=S$ and $\lambda \cdot S=S$ for each $\lambda\in\mathbb{R},\lambda\neq0$.
    \item If $Y$ and $Z$ are subspaces of the linear space $X$, show that $T+Z$ is also a subspace and $Y+Z=\text{span}[Y\cup Z]$. 
    \item Let $S$ be a subset of a normed linear space $X$.
    \begin{enumerate}[label=(\roman*),align=left]
        \item Show that the intersection of a collection of linear subspaces of $X$ is also a linear subspace of $X$.
        \item Show that span$[S]$ is the intersection of all the linear subspaces of $X$ that contain $S$ and therefore is a linear subspace of $X$.
        \item Show that $\overline{\text{span}}[S]$ is the intersection of all the closed linear subspaces of $X$ that contain $S$ and is therefore a closed linear subspace of $X$. 
    \end{enumerate}
    \item For a normed linear space $X$, show that the function $\|\cdot\|:X\to\mathbb{R}$ is continuous.
    \item For two normed linear spaces $(X,\|\cdot\|_1)$ and $(Y,\|\cdot\|_2)$, define a linear structure on the Cartesian product $X\times Y$ by ...
\end{enumerate}

% 13.2
\section{Linear Operators}

% 13.3
\section{Compactness Lost: Infinite Dimensional Normed Linear Spaces}

% 13.4
\section{The Open Mapping and Closed Graph Theorems}\

% 13.5
\section{The Uniform Boundedness Principle}

% Chapter 14
\authoredby{inprogress}
\chapter{Duality for Normed Linear Spaces}

For a normed linear space $X$, we denoted the normed linear space of continuous linear real-valued functions of $X$ by $X^*$ and called it the \textbf{dual space} of $X$.
We aim to explore properties of the mapping from $X\times X^*$ to $\mathbb{R}$ defined by
\[
    (x,\psi)\mapsto\psi(x)\text{ for all }x\in X,\psi\in X^*.
\]

Hahn-Banach Theorem: extension of certain linear functionals on subspaces of an unnormed linear space to linear functionals on the whole space.
Hahn-Banach implies:
\begin{enumerate}
    \item for a normed linear space $X$, any bounded linear functional on a subspace of $X$ may be extended to a bounded linear functional on all of $X$, without increasing its norm
    \item for a locally convex Topological vector space $X$, any two disjoint closed convex sets of $X$ may be separated by a closed hyperplane
    \item for a reflexive Banach space $X$, any bounded sequence in $X$ has a weakly convergent subsequence.
\end{enumerate}

% 14.1
\authoredby{inprogress}
\section{Linear Functionals, Bounded Linear Functionals, and Weak Topologies}



\begin{center}
	\textbf{PROBLEMS}
\end{center}
\begin{enumerate}
	\setcounter{enumi}{0}
    \item hellos
\end{enumerate}

% 14.2
\authoredby{untouched}
\section{The Hahn-Banach Theorem}

% 14.3
\section{Reflexive Banach Spaces and Weak Sequential Convergence}

% 14.4
\section{Locally Convex Topological Vector Spaces}

% 14.5
\section{The Separation of Convex Sets and Mazur's Theorem}

% 14.6
\section{The Krein-Milman Theorem}

% Chapter 15
\chapter{Compactness Regained: The Weak Topology}

% 15.1
\section{Alaoglu's Extension of Helley's Theorem}
\begin{center}
	\textbf{PROBLEMS}
\end{center}
\begin{enumerate}
	\setcounter{enumi}{0}
    \item For $X$ a normed linear space with closed unit ball $B$, suppose the function $f:B\to[-1,1]$ has the property that whenever $u,v,u+v,\lambda u$ belong to $B$, $f(u+v)=f(u)+f(v)$ and $f(\lambda u)=\lambda f(u)$.
    Show that $f$ is the restriction to $B$ of a linear functional on all of $X$ which belongs to the closed unit ball of $X^*$.
    \item Let $X$ be a normed linear space and $K$ be a bounded convex weak-$*$ closed subset of $X^*$. Show that $K$ possesses an extreme point.
    \item Show that any nonempty weakly open set in an infinite dimensional normed linear spae is unbounded with respect to the norm.
    \item Use the Baire Category Theorem and the preceding problem to show that the weak topology on an infinite dimensional Banach space is not metrizable by a complete metric.
    \item Is every Banach space isomorphic to the dual of a Banach space?
\end{enumerate}

% 15.2
\section{Reflexivity and Weak Compactness: Kakutani's Theorem}
\begin{center}
	\textbf{PROBLEMS}
\end{center}
\begin{enumerate}
	\setcounter{enumi}{5}
    \item Show that every weakly compact subset of a normed linear space is bounded with respect to the norm.
    \item Show that the closed unit ball $B^*$ of the dual $X^*$ of a Banach space $X$ has an extreme point.
    \item Let $\mathcal{T}_1$ and $\mathcal{T}_2$ be two compact, Hausdorff topologies on a set $\mathcal{S}$ for which $\mathcal{T}_1\subseteq\mathcal{T}_2$. Show that $\mathcal{T}_1=\mathcal{T}_2$.
    \item Let $X$ be a normed linear space containing the subspace $Y$. For $A\subseteq Y$, show that the weak topology on $A$ induced by $Y^*$ is the same as the topology $A$ inherits as a subspace of $X$ with its weak topology.
    \item Argue as follows to show that q Banach space $X$ is reflexive iff its dual space $X^*$ is reflexive.
    \begin{enumerate}[label=(\roman*),align=left]
        \item If $X$ is reflexive, show that the weak and weak-$*$ topologies on $B^*$ are the same, and infer from this that $X^*$ is reflexive.
        \item If $X^*$ is reflexive, use part (i) and Proposition 15 of Chapter 14 to show that $X$ is reflexive.
    \end{enumerate}
    \item For $X$ a Banach space, by the preceding problem, if $X$ is reflexive, then so is $X^*$. Conclude that $X$ is not reflexive if there is a closed subspace of $X^*$ that is not reflexive.
    Let $K$ be an infinite compact Hausdorff space and $\{x_n\}$ an enumeration of a countably infinite subset of $K$. Define the operator $T:l^1\to[C(K)]^*$ by
    \[
        [T(\{n_k\})](f)=\sum_{k=1}^\infty \eta_k\cdot f(x_k)\text{ for all }\{\eta_k\}\in l^1\text{ and }f\in C(k).
    \]
    Show that $T$ is an isometry and therefore, since $l^1$ is not reflexive, neither is $T(l^1)$ and therefore neither is $C(K)$. Use a dimension counting argument to show that $C(K)$ is reflexive if $K$ is a finite set.
    \item If $Y$ is a linear subspace of a Banach space $X$, we define the \textit{annihilator} $Y^\perp$ to be the subspace of $X^*$ consisting of those $\psi\in X^*$ for which $\psi=0$ on $Y$.
    If $Y$ is a subspace of $X^*$, we define $Y^0$ to be the subspace of vectors in $X$ for which $\psi(x)=0$ for all $\psi\in Y$.
    \begin{enumerate}[label=(\roman*),align=left]
        \item Show that $Y^\perp$ is a closed linear subspace of $X^*$.
        \item Show that $(Y^\perp)^0=\overline Y$.
        \item If $X$ is reflexive and $Y$ is a subspace of $X^*$, show that $Y^\perp=J(Y^0)$.
    \end{enumerate}
\end{enumerate}

% 15.3
\section{Compactness and Weak Sequential Compactness: The Eberlein-\v Smulian Theorem}
\begin{center}
	\textbf{PROBLEMS}
\end{center}
\begin{enumerate}
	\setcounter{enumi}{12}
    \item In a general topological space that is not metrizable a sequence may converge to more than one point. Show that this cannot occur for the $W$-weak topology on a normed linear space $X$, where $W$ is a subspace of $X^*$ that separates points in $X$.
    \item Show that there is a bounded sequence in $L^\infty[0,1]$ that fails to have a weakly convergent subsequence. Show that the closed unit ball of $C[a,b]$ is not weakly compact.
    \item Let $K$ be a compact metric space with infinitely many points. Show that there is a bounded sequence in $C(K)$ that fails to have a weakly convergent subsequence (see Problem 11), but every bounded sequence of continuous linear functionals on $C(K)$ has a subsequence that converges pointwise to a continuous linear functional on $C(K)$.
\end{enumerate}

% 15.4
\section{Metrizability of Weak Topologies}
\begin{center}
	\textbf{PROBLEMS}
\end{center}
\begin{enumerate}
	\setcounter{enumi}{15}
    \item Show that the dual of an infinite dimensional normed linear space also is infinite dimensional.
    \item Complete the last step of the proof of Theorem 10 by showing that the inequalities (12) imply that the metric $\rho$ induces the $W$-weak topology.
    \item Let $X$ be a Banach space, $W$ a closed subspace of its dual $X^*$, and $\psi_0$ belong to $X^*\setminus W$.
    Show that if either $W$ is finite dimensional or $X$ is reflexive, then there is a vector $x_0$ in $X$ for which $\psi_0(x_0)\neq0$ but $\psi(x_0)=0$ for all $\psi\in W$.
    Exhibit an example of an infinite dimensional closed subspace $W$ of $X^*$ for which this separation property fails.
\end{enumerate}

% Chapter 16
\authoredby{inprogress}
\chapter{Continuous Linear Operators on Hilbert Spaces}

% 16.1
\authoredby{inprogress}
\section{The Inner Product and Orthogonality}

\begin{namedthm*}{Definition}
    Let $H$ be a linear space.
    A function $\langle\cdot,\cdot\rangle:H\times H\to\mathbb{R}$ is called an \textbf{inner product} on $H$ provided for all $x_1,x_2,x,y\in X$ and real numbers $\alpha,\beta$, it satisfies
    \begin{enumerate}[(i)]
        \item linearity in first argument: $\langle \alpha x_1+\beta x_2,y\rangle=\alpha\langle x_1,y\rangle+\beta\langle x_2,y\rangle$.
        \item symmetry: $\langle x,y\rangle=\langle y,x\rangle$
        \item positive definiteness: $\langle x,x\rangle>0$ if $x\neq0$
    \end{enumerate}
    A linear space $H$ together with an inner product on $H$ is called an \textbf{inner product space}.
\end{namedthm*}
Property (i) along with property (ii) reveals that the real inner product in linear in both arguments: this is called \textbf{bilinearity}.

For two vectors $u=(u_1,\dots u_n)$ ad $u=(v_1,\dots v_n)$ in Euclidean space $\mathbb{R}^n$, the Euclidean inner product, $\langle u,v\rangle$, is defined by
\[
    \langle u,v\rangle=\sum_{k=1}^nu_kv_k.
\]
For two sequences $x=\{x_k\}$ ad $y=\{y_k\}$ in $\ell^2$, the $\ell^2$ inner product, $\langle x,y \rangle$, is defined by
\[
    \langle x,y\rangle=\sum_{k=1}^\infty x_ky_k.
\]
For a measurable set $E\subseteq\mathbb{R}$ and two functions $f$ and $g$ in $L^2(E)$, the $L^2$ inner product, $\langle f,g \rangle$, is defined by
\[
    \langle f,g\rangle=\int_Ef\cdot g.
\]
From the Cauchy-Schwarz (H\"older's) inequality, we infer that these inner products are properly defined (finite).
\begin{namedthm*}{The Cauchy-Schwarz Inequality}
    For any two vectors $u,v$ in an inner product space $H$,
    \[
        |\langle u,v\rangle|\le\|u\|\cdot\|v\|.
    \]
\end{namedthm*}
\begin{namedthm*}{Proposition 1}
    For a vector $h$ in an inner product space $H$, define
    \[
        \|h\|=\sqrt{\langle h,h\rangle}.
    \]
    Then $\|\cdot\|$ is a norm on $H$ called that norm induced by the inner product $\langle\cdot,\cdot\rangle$.
\end{namedthm*}
\begin{proof}
    Let $h\in H$.
    \begin{enumerate}[(i)]
        \item By positive definiteness of the inner product, $\langle h,h\rangle>0\implies \sqrt{\langle h,h\rangle}>0$.
        \\If $h=0$ then $\langle h,h\rangle=0\implies\sqrt{\langle h,h\rangle}=0$.
        \item By bilinearity of the inner product, $\|\alpha h\|=\sqrt{\langle \alpha h,\alpha h\rangle}=\sqrt{\alpha^2\langle h,h\rangle}=|\alpha|\sqrt{\langle h,h\rangle} =|\alpha|\|h\|$.
        \item Use the Cauchy-Schwarz inequality to see that 
        \[
            \|u+v\|^2=\langle u+v,u+v\rangle=\langle u,u\rangle+2\langle u,v\rangle+\langle v,v\rangle\le \|u\|^2+2\|u\|\|v\|+\|v\|^2=(\|u\|+\|v\|)^2.
        \]
    \end{enumerate}
\end{proof}
\begin{namedthm*}{The Parallelogram Identity}
    For any two vectors $u,v$ in an inner product space $H$,
    \[
        \|u-v\|^2+\|u+v\|^2=2\|u\|^2+2\|v\|^2.
    \]
\end{namedthm*}
To verify this we can simply add the following two equalities:
\begin{align*}
    \|u-v\|^2&=\|u\|^2-2\langle u,v\rangle+\|v\|^2,\\
    \|u+v\|^2&=\|u\|^2+2\langle u,v\rangle+\|v\|^2.
\end{align*}

\begin{namedthm*}{Definition}
    An inner product space $H$ is called a \textbf{Hilbert space} provided it is a Banach space with respect to the norm induced by the inner product.
\end{namedthm*}
The Riesz-Fischer Theorem tells us that for $E$ a measurable set of real numbers, $L^2(E)$ is a Hilbert space and, as a consequence, so is $\ell^2$.
\begin{namedthm*}{Proposition 2}
    Let $K$ be a nonempty, closed, convex subset of a Hilbert space $H$ and $h_0$ belong to $H\setminus K$.
    Then there is exactly on vector $h_*\in K$ that is closest to $h_0$ in the sense that
    \[
        \|h_0-h_*\|=\text{dist}(h_0,K)=\underset{h\in K}{\inf}\|h_0-h\|.
    \]
\end{namedthm*}
% \begin{proof}
%     d
% \end{proof}

\begin{center}
	\textbf{PROBLEMS}
\end{center}
In the following problems, $H$ is a Hilbert space.
\begin{enumerate}
	\setcounter{enumi}{0}
    \item Let $[a,b]$ be a closed, bounded interval of real numbers. Show that the $L^2[a,b]$ inner product is also an inner product on $C[a,b]$. Is $C[a,b]$ considered as an inner product space with the $L^2[a,b]$ inner product, a Hilbert space?\\
    \\We see that it is true that the $L^2[a,b]$ inner product is also an inner product on $C[a,b]$:
    \begin{enumerate}[(i)]
        \item 
        $
            \langle \alpha f_1+\beta f_2,g\rangle_{L^2}
            =\int_{[a,b]}( \alpha f_1+\beta f_2)
            %=\int_{[a,b]} \alpha f_1\cdot g+\beta f_2\cdot g
            %&=\int_{[a,b]} \alpha f_1\cdot g+\int_{[a,b]}\beta f_2\cdot g\\
            =\alpha\int_{[a,b]} f_1\cdot g+\beta\int_{[a,b]} f_2\cdot g
            =\alpha\langle f_1,g\rangle_{L^2}+\beta\langle f_2,g\rangle_{L^2}
        $
        \item $\langle f,g\rangle_{L^2}=\int_{[a,b]}f\cdot g=\int_{[a,b]}g\cdot f=\langle g,f\rangle_{L^2}$
        \item $\langle f,f\rangle_{L^2}=\int_{[a,b]}f^2\ge0$, with $\langle f,f\rangle_{L^2}=0\iff f=0$
    \end{enumerate}
    However, the inner product space $(C[a,b],\langle\cdot,\cdot\rangle_{L^2})$ is not a Hilbert space because it is not complete (not a Banach space) with respect to the norm defined by $\|\cdot\|_{L^2}:=\sqrt{\langle\cdot,\cdot\rangle_{L^2}}$.\\
    \\To see this, let $t\in(a,b)$ and consider a sequence of continuous functions $\{f_n\}$ defined on $[a,b]$ such that for each $n$,
    \[
        f_n(x):=
        \begin{cases}
            0 &x\le t\\
            n(x-t)&t<x<t+\frac{1}{n}\\
            1 &x\ge t+\frac{1}{n}
        \end{cases}
    \]
    We aim to show that this sequence is Cauchy but does not converge to a continuous function, and therefore the space is not complete.\\
    \\First we prove that this sequence is Cauchy:
    \\Fix $\epsilon>0$.
    \\Consider any natural numbers $m,n>\frac{1}{\epsilon^2}$, with $m\ge n$.
    \[
        (f_m-f_n)(x)=
        \begin{cases}
            0-0 &x\le t\\
            m(x-t)-n(x-t)&x\in(t,t+\frac{1}{m})\\
            1-n(x-t)&x\in[t+\frac{1}{m},t+\frac{1}{n})\\
            1-1 &x\ge t+\frac{1}{n}
        \end{cases}
        \le
        \begin{cases}
            0 &x\le t\\
            1&x\in(t,t+\frac{1}{m})\\
            1&x\in[t+\frac{1}{m},t+\frac{1}{n})\\
            0 &x\ge t+\frac{1}{n}
        \end{cases}
        :=g(x)
    \]
    By monotonicity of integration, we get
    \[
        \int_{[a,b]}|f_n-f_m|^2\le\int_{[a,b]}|g|^2=\int_{(t,t+\frac{1}{n})}1=m((t,t+\frac{1}{n}))=\frac{1}{n}.
    \]
    Therefore we see that
    \[
        \|f_n-f_m\|_{L^2}=\left(\int_{[a,b]}|f_n-f_m|^2\right)^{1/2}\le\left(\frac{1}{n}\right)^{1/2}<\epsilon,
    \]
    which implies that $\{f_n\}$ is Cauchy.\\
    \\Very similarly, we prove that this sequence converges to a discontinuous function:
    \\Define the discontinuous function $f:[a,b]\to\mathbb{R}$ by
    \[
        f(x):=
        \begin{cases}
            0 &x\le t\\
            1 &x> t
        \end{cases}
    \]
    \\Fix $\epsilon>0$.
    \\Consider any natural number $n>\frac{1}{\epsilon^2}$.
    \[
        (f-f_n)(x)=
        \begin{cases}
            0-0 &x\le t\\
            1-n(x-t)&x\in(t,t+\frac{1}{n})\\
            1-1 &x\ge t+\frac{1}{n}
        \end{cases}
        \le
        \begin{cases}
            0 &x\le t\\
            1&x\in(t,t+\frac{1}{n})\\
            0 &x\ge t+\frac{1}{n}
        \end{cases}
        :=g(x)
    \]
    By monotonicity of integration, we get
    \[
        \int_{[a,b]}|f-f_n|^2\le\int_{[a,b]}|g|^2=\int_{(t,t+\frac{1}{n})}1=m((t,t+\frac{1}{n}))=\frac{1}{n}.
    \]
    Therefore we see that
    \[
        \|f-f_n\|_{L^2}=\left(\int_{[a,b]}|f-f_n|^2\right)^{1/2}\le\left(\frac{1}{n}\right)^{1/2}<\epsilon,
    \]
    which implies that $\{f_n\}$ converges to $f$.
    \item Show that the maximum norm on $C[a,b]$ is not induced by an inner product and neither is the usual norm on $\ell^1$.\\
    \\See Problem 7 to see that for a normed linear space $(X,\|\cdot\|)$, the norm is induced by an inner product iff the parallelogram identity holds.
    Therefore it is sufficient to show that the parallelogram identity does not hold for the spaces $(C[a,b],\|\cdot\|_{\max})$ and $(\ell^1,\|\cdot\|_1)$.\\
    \\$(C[a,b],\|\cdot\|_{\max})$\\
    \\Let $f,g\in C[0,1]$ defined by $f(x)=x^2$, $g(x)=2$.
    \\Recall that $\|f\|_{\max}:=\max_{x\in[a,b]}|f(x)|$.
    \\Then $\|f-g\|_{\max}=2$, $\|f+g\|_{\max}=3$, $\|f\|_{\max}=1$, $\|g\|_{\max}=2$, so that 
    \[
        \|f-g\|_{\max}^2+\|f+g\|_{\max}^2=13\neq 6=2\|f\|_{\max}+2\|g\|_{\max}.
    \]
    $(\ell^2,\|\cdot\|_1)$\\
    \\Let $x=(1,0,0,\dots)\in\ell^1$ and $y=(0,1,0,\dots)\in\ell^1$.
    \\Recall that $\|x\|_1:=\sum_{i=1}^\infty|x_i|$.
    \\Then
    % \begin{align*}
    %     \|x-y\|_1=|1-0|+|0-1|=2\\
    %     \|x+y\|_1=|1+0|+|0+1|=2\\
    %     \|x\|_1=1\\
    %     \|y\|_1=1\\
    % \end{align*}
    $\|x-y\|_1=2,\ \|x+y\|_1=2,\ \|x\|_1=1,\ \|y\|_1=1$, so that
    \[
        \|x-y\|_2^2+\|x+y\|_2^2=8\neq 4=2\|x\|_2+2\|y\|_2.
    \]
    \item Let $H_1$ and $H_2$ be Hilbert spaces. Show that the Cartesian product $H_1\times H_2$ is also a Hilbert space with an inner product with respect to which $H_1\times\{0\}=[\{0\}\times H_2]^\perp$.
    \item Show that if $S$ is a subset of an inner product space $H$, then $S^\perp$ is a closed subspace of $H$.
    \item Let $S$ be a subset of $H$. Show that $S=(S^\perp)^\perp$ iff $S$ is a closed subspace of $H$.
    \item (Polarization Identity) Show that for any two vectors $u,v\in H$,
    \[
        \langle u,v \rangle = \frac{1}{4}[\|u+v\|^2-\|u-v\|^2].
    \]
    \\Recall the Parallelogram inequality and instead of adding, simply subtract the first inequality:
    \begin{align*}
        -\|u-v\|^2&=-\|u\|^2+2\langle u,v\rangle-\|v\|^2,\\
        \|u+v\|^2&=\|u\|^2+2\langle u,v\rangle+\|v\|^2,
    \end{align*}
    so that
    \[
        \|u+v\|^2-\|u-v\|^2=4\langle u,v\rangle.
    \]
    \item (Jordan-von Neumann) Let $X$ be a linear space normed by $\|\cdot\|$. Use the polarization identity to show that a norm $\|\cdot\|$ is induced by an inner product iff the parallelogram identity holds.\\
    \\Let $(X,\|\cdot\|)$ be a normed linear space.\\
    \\$(\implies)$ Suppose that the norm is induced by some inner product $\langle \cdot,\cdot\rangle$ on $X$.
    \\Then for any two vectors $u,v\in X$,
    \begin{align*}
        \|u-v\|^2&=\langle u-v,u-v\rangle=\langle u,u\rangle+\langle u,-v\rangle+\langle -v,u\rangle+\langle -v,-v\rangle=\|u\|^2-2\langle u,v\rangle+\|v\|^2,\\
        \|u+v\|^2&=\langle u+v,u+v\rangle=\langle u,u\rangle+\langle u,v\rangle+\langle v,u\rangle+\langle v,v\rangle=\|u\|^2+2\langle u,v\rangle+\|v\|^2,
    \end{align*}
    and adding the two equalities shows that the parallelogram identity holds.\\
    \\$(\impliedby)$ Suppose that the parallelogram identity holds.
    \\Define the function $\langle\cdot,\cdot\rangle:X\times X\to\mathbb{R}$ by
    \[
        \langle u,v \rangle = \frac{1}{4}[\|u+v\|^2-\|u-v\|^2].
    \]
    \\The aim is to show that $\sqrt{\langle v,v\rangle}=\|v\|$ for $v\in X$, and that $\langle\cdot,\cdot\rangle$ is an inner product.
    \\First see that
    \[
        \sqrt{\langle v,v \rangle} = \sqrt{\frac{1}{4}[\|v+v\|^2-\|v-v\|^2]}=\sqrt{\frac{2^2}{4}\|v\|^2}=\|v\|.\tag{1}
    \]
    (i) Bilinearity (additivity):
    % \\For any $x,y,v\in X$, by the parallelogram identity,
    % \begin{align*}
    %     2\|x+v\|^2+2\|y\|^2&=\|(x+v)-y\|^2+\|(x+v)+y\|^2\tag{a}\\
    %     2\|x-v\|^2+2\|y\|^2&=\|(x-v)-y\|^2+\|(x-v)+y\|^2\tag{b}
    % \end{align*}
    % Then, subtracting (b) from (a), we get
    % \begin{align*}
    %     2\|x+v\|^2-2\|x-v\|^2&=\|(x+v)+y\|^2-\|(x-v)+y\|^2+\|(x+v)-y\|^2-\|(x-v)-y\|^2\\
    %     2[\|x+v\|^2-\|x-v\|^2]&=[\|(x+y)+v\|^2-\|(x+y)-v\|^2]+[\|(x-y)+v\|^2-\|(x-y)-v\|^2]\\
    % %4\frac{1}{4}\left[\|(x-y)+v\|^2-\|(x-y)-v\|^2\right]+4\frac{1}{4}\left[\|(x+y)+v\|^2-\|(x+y)-v\|^2\right]&=2\cdot4\frac{1}{4}\left[\|x+v\|^2-\|x-v\|^2\right]\\
    %     8\langle x,v\rangle&=4\langle x+y,v\rangle+4\langle x-y,v\rangle\\
    %     2\langle x,v\rangle&=\langle x+y,v\rangle+\langle x-y,v\rangle
    % \end{align*}
    % Therefore, when we let $x=\frac{u_1+u_2}{2},\ y=\frac{u_1-u_2}{2},$
    % we get
    % \begin{align*}
    %     \langle u_1+u_2,v\rangle
    %     =2\langle \frac{u_1+u_2}{2},v\rangle
    %     =\langle \frac{u_1+u_2}{2}+\frac{u_1-u_2}{2},v\rangle+\langle \frac{u_1+u_2}{2}-\frac{u_1-u_2}{2},v\rangle
    %     =\langle u_1,v\rangle+\langle u_2,v\rangle.
    % \end{align*}
    \\For any $x,y,v\in X$, by the parallelogram identity,
    \begin{align*}
        \|(u_1+v)+u_2\|^2&=2\|u_1+v\|^2+2\|u_2\|^2-\|(u_1+v)-u_2\|^2\\
        \|(u_2+v)+u_1\|^2&=2\|u_2+v\|^2+2\|u_1\|^2-\|(u_2+v)-u_1\|^2
    \end{align*}
    Then
    \begin{align*}
        \|u_1+u_2+v\|^2&=\frac{1}{2}(\|(u_1+v)+u_2\|^2+\|(u_2+v)+u_1\|^2)\\
        &=\|u_1+v\|^2+\|u_2+v\|^2+\|u_1\|^2+\|u_2\|^2-\frac{1}{2}\|(u_1+v)-u_2\|^2-\frac{1}{2}\|(u_2+v)-u_1\|^2
    \end{align*}
    And so this also holds for $v=-v$:
    \begin{align*}
        \|u_1+u_2+(-v)\|^2&=\|u_1-v\|^2+\|u_2-v\|^2+\|u_1\|^2+\|u_2\|^2-\frac{1}{2}\|(u_1-v)-u_2\|^2-\frac{1}{2}\|(u_2-v)-u_1\|^2\\
        &=\|u_1-v\|^2+\|u_2-v\|^2+\|u_1\|^2+\|u_2\|^2-\frac{1}{2}\|(u_2+v)-u_1\|^2-\frac{1}{2}\|(u_1+v)-u_2\|^2
    \end{align*}
    Therefore we can write
    \begin{align*}
        \langle u_1+u_2,v\rangle&=\frac{1}{4}[\|u_1+u_2+v\|^2-\|u_1+u_2-v\|^2]\\
        &=\frac{1}{4}[\|u_1+v\|^2-\|u_1-v\|^2]+\frac{1}{4}[\|u_2+v\|^2-\|u_2-v\|^2]\\
        &=\langle u_1,v\rangle+\langle u_2,v\rangle.
    \end{align*}
    (i) Bilinearity (homogeneity):
    \\Let $S=\{\alpha\in\mathbb{R}\mid\alpha\langle u,v\rangle=\langle \alpha u,v\rangle\}$.
    \\Clearly $1,0\in S$, and $-1\in S$ because
    \[
        -1\langle u,v\rangle=\frac{1}{4}[\|u-v\|^2-\|u+v\|^2]=\frac{1}{4}[\|-u+v\|^2-\|-u-v\|^2]=\langle -1u,v\rangle.
    \]
    \\Suppose $\alpha,\beta\in S$.
    Then 
    \begin{align*}
        (\alpha+\beta)\langle u,v\rangle
        &=\alpha\langle u,v\rangle+\beta\langle u,v\rangle\\
        &=\langle \alpha u,v\rangle+\langle \beta u,v\rangle\\
        &=\langle \alpha u+\beta u,v\rangle&&\text{by bilinearity (additivity)}\\
        &=\langle (\alpha+\beta)u,v\rangle,
    \end{align*}
    so that $(\alpha+\beta)\in S$ and $S$ contains all integers.\\
    \\Suppose $\alpha,\beta\in S$, $\beta\neq0$.
    Then 
    \[
        \alpha\langle u,v\rangle=\langle \alpha u,v\rangle=\langle \frac{\beta}{\beta}\alpha u,v\rangle=\beta\langle \frac{\alpha}{\beta}u,v\rangle\implies \frac{\alpha}{\beta}\langle u,v\rangle=\langle\frac{\alpha}{\beta}u,v\rangle,
    \]
    so that $\frac{\alpha}{\beta}\in S$ and $S$ contains all rational numbers.\\
    \\Fix any $x,y\in X$.
    Consider the functions $f,g:\mathbb{R}\to\mathbb{R}$ defined by $f(\alpha)=\alpha\langle u,v\rangle$ and $g(\alpha)=\langle \alpha u,v\rangle$.
    The function $f$ is linear on a finite dimensional space and thus continuous (Chapter 13 Problem 29), and the function $g$ is a composition of $\alpha\mapsto\alpha x$ and $t\mapsto\langle t,y\rangle$, which are both continuous ($\ast$).
    Then we have that $f,g$ are continuous with $f(\alpha)=g(\alpha)$ for all $\alpha\in\mathbb{Q}$, which implies that $f(\alpha)=g(\alpha)$ for all $\alpha\in\mathbb{R}$.
    \\That is, $\alpha\langle u,v\rangle=\langle \alpha u,v\rangle$ for all scalars $\alpha$.
    \\(ii) Symmetry:
    \[\langle u,v \rangle = \frac{1}{4}[\|u+v\|^2-\|u-v\|^2]=\frac{1}{4}[\|v+u\|^2-\|v-u\|^2]=\langle v,u\rangle.\]
    (iii) Positive Definiteness:
    \\For $v\neq0$: $\langle v,v \rangle = \|v\|^2>0$ by (1) and positive definiteness of norm.
    \\For $v=0$: $\langle v,v \rangle = \|v\|^2=0$ by (1) and positive definiteness of norm.\\
    \\Therefore $\langle\cdot,\cdot\rangle$ is an inner product.\\
    % \\($\ast$) Linear functions are continuous:
    % \\Let $H$ be a hilbert space, let $a\in H$, and let $f:H\to\mathbb{R}$ be a linear function:
    % \[
    %     f(x):=\langle a,x\rangle\text{ for any }x\in X.
    % \]
    % Fix $\varepsilon>0$.
    % \\Then there exists $\delta=\frac{\varepsilon}{\|a\|}>0$.
    % \\For any $x,y\in H$ such that $\|x-y\|<\delta$, we use Cauchy-Schwarz inequality to see that
    % \[
    %     |f(x)-f(y)|=|\langle a,x\rangle-\langle a,y\rangle|=|\langle a,x-y\rangle|\le\|a\|\cdot\|x-y\|<\varepsilon.
    % \]
    \\($\ast$) For a normed linear space $X$ and $y\in X$, the function $x\mapsto\|x+y\|$ for $x\in X$ is continuous.
    \\(Use this as a simplified version of $x\mapsto\langle x,y\rangle=\frac{1}{4}[\|x+y\|^2-\|x-y\|^2]$, which will also be continuous as it is the product and sum of continuous functions.)
    \\Fix $\varepsilon>0$.
    \\Let $\delta=\varepsilon>0$.
    \\For $x_1,x_2\in X$ such that $\|x_1-x_2\|<\delta=\varepsilon$, we use the reverse triangle inequality to see
    \[
        |f(x_1)-f(x_2)|=|\|x_1+y\|-\|x_2+y\||\le\|x_1+y-(x_2+y)\|<\varepsilon.
    \]
    \item Let $V$ be a closed subspace of $H$ and $P$ a projection of $H$ onto $V$. Show that $P$ is the orthogonal projection of $H$ onto $V$ iff (4) holds.
    \item Let $T$ belong to $\mathcal{L}(H)$. Show that $T$ is an isometry iff
    \[
        \langle T(u),T(v)\rangle=\langle u,v\rangle\text{ for all }u,v\in H.
    \]
    \\$(\implies)$ Suppose that $T$ is an isometry.
    \\Then $\|Tu\|=\|u\|$ for any $u\in H$.
    \\Let $u,v\in H$ so that we can derive:
    \begin{align*}
        \langle T(u),T(v)\rangle&=\frac{1}{4}[\|Tu+Tv\|^2-\|Tu-Tv\|^2]&&\text{Polarization identity}\\
        &=\frac{1}{4}[\|T(u+v)\|^2-\|T(u-v)\|^2]&&T\text{ is linear}\\
        &=\frac{1}{4}[\|u+v\|^2-\|u-v\|^2]&&T\text{ is an isometry}\\
        &=\frac{1}{4}[\langle u+v,u+v\rangle-\langle u-v,u-v\rangle]&&\text{norm induced by inner product}\\
        &=\frac{1}{4}[\|u\|^2+2\langle u,v\rangle+\|v\|^2-\|u\|^2+2\langle u,v\rangle-\|v\|^2]&&\text{norm induced by inner product}\\
        &=\langle u,v\rangle.
    \end{align*}
    \\$(\impliedby)$ Suppose that $\langle T(u),T(v)\rangle=\langle u,v\rangle\text{ for all }u,v\in H$.
    \\Therefore for any $u\in H$, we have 
    \[
        \|Tu\|=\sqrt{\langle Tu,Tu\rangle}=\sqrt{\langle u,u\rangle}=\|u\|.
    \]
    \item Let $V$ be a finite dimensional subspace of $H$ and $\varphi_1,\cdots,\varphi_n$ a basis for $V$ consisting of unit vectors, each pair of which is orthogonal. Show that the orthogonal projection $P$ of $H$ onto $V$ is given by 
    \[
        P(h)=\sum_{k=1}^n\langle h,\varphi_k\rangle\varphi_k\text{ for all }h\in V.
    \]
    \item For $h$ a vector in $H$, show that the function $u\mapsto\langle h,u\rangle$ belongs to $H^*$.\\
    \\The aim is to show that the function $\varphi:H\to\mathbb{R}$ defined by $\varphi(u)=\langle h,u\rangle$ is bounded and linear (see Chapter 8, Proposition 1).
    \\To see boundedness (continuity), use the Cauchy-Schwarz Inequality so that for any $u\in H$,
    \[
        \varphi(u)=\langle h,u\rangle\le\|h\|\cdot\|u\|<\infty.
    \]
    (Norms are defined to be \textbf{real-valued}; therefore the norm of any element in a linear space is a real number and thus finite: $\|h\|,\|u\|<\infty$)
    \\To see linearity, simply use bilinearity of the inner product:
    \[
        \varphi(\alpha u+\beta v)=\langle h,\alpha u+\beta v\rangle=\alpha\langle h,u\rangle+\beta\langle h,v\rangle=\alpha\varphi(u)+\beta\varphi(v).
    \]
    \item For any vector $h\in H$, show that there is a bounded linear functional $\psi\in H^*$ for which 
    \[
        \|\psi\|=1\text{ and }\psi(h)=\|h\|.  
    \]
    \\For $h\in H$ $(h\neq0)$, let $\psi:H\to\mathbb{R}$ be defined by 
    \[
        \psi(u)=\langle \frac{h}{\|h\|},u\rangle\text{ for any }u\in H.
    \]
    By the previous Problem 11, we proved that $\psi\in H^*$.
    \\Then 
    \[
        \psi(h)=\langle \frac{h}{\|h\|},h\rangle=\frac{1}{\|h\|}\langle h,h\rangle=\frac{1}{\|h\|}\|h\|^2=\|h\|.
    \]
    It remains to show $\|\psi\|_*=1$.
    \\Recall Chapter 8.1 for the following:
    \\We define the operator norm:
    \[
        \|\psi\|_*:=\inf\{M\mid|\psi(u)|\le M\|u\|\text{ for all }u\in H\},
    \]
    which implies the two: 
    \begin{itemize}
        \item $|\psi(u)|\le \|\psi\|_*\|u\|\text{ for all }u\in H$
        \item $\|\psi\|_*=\sup\{\psi(u)\mid u\in H, \|u\|\le1\}$
    \end{itemize}
    Therefore
    \[
        \|h\|=|\psi(h)|\le \|\psi\|_*\|h\|\implies 1\le \|\psi\|_*\tag{\text{a}}
    \]
    \[
        \|\psi\|_*=\underset{\|u\|\le1}{\underset{u\in H}{\sup}}\langle \frac{h}{\|h\|},u\rangle\le \|\frac{h}{\|h\|}\|\cdot\|u\|\le1\tag{\text{b}}
    \]
    Then (a) and (b) imply $\|\psi\|_*=1$.
    \item Let $V$ be a closed subspace of $H$ and $P$ the orthogonal projection of $H$ onto $V$.
    For any normed linear space $X$ and $T\in\mathcal{L}(V,X)$, show that $T\circ P$ belongs to $\mathcal{L}(H,X)$, and is an extension of $T:V\to X$ for which $\|T\circ P\|=\|T\|$.
    \item Prove the Hyperplane Separation Theorem for $H$, considered as a locally convex topological vector space with respect to the strong topology, by directly using Proposition 2.
    \item Use Proposition 2 to prove the Krein-Milman Lemma in a Hilbert space.
\end{enumerate}

% 16.2
\authoredby{untouched}
\section{The Dual Space and Weak Sequential Convergence}
\begin{center}
	\textbf{PROBLEMS}
\end{center}
In the following problems, $H$ is a Hilbert space.
\begin{enumerate}
	\setcounter{enumi}{15}
    \item Show that neither $\ell^1,\ell^\infty,L^1[a,b]$ nor $L^\infty[a,b]$ is Hilbertable.
    \item Prove Proposition 7.
    \item Let $H$ be an inner product space. Show that since $H$ is a dense subset of a Banach space $X$ whose norm restricts to the norm induced by the inner product on $H$, the inner product on $H$ extends to $X$ and induces the norm on $X$.
    Thus inner product spaces have Hilbert space completions.
\end{enumerate}

% 16.3
\section{Bessel's Inequality and Orthonormal Bases}
\begin{center}
	\textbf{PROBLEMS}
\end{center}
In the following problems, $H$ is a Hilbert space.
\begin{enumerate}
	\setcounter{enumi}{18}
    \item Show that an orthonormal subset of a separable Hilbert space $H$ must be countable.
    \item Let $\{\varphi_k\}$ be an orthonormal sequence in a Hilbert space $H$. Show that $\{\varphi_k\}$ converges weakly to $0$ in $H$.
    \item Let $\{\varphi_k\}$ be an orthonormal basis for the separable Hilbert space $H$. Show that $\{u_n\}\to u$ in $H$ iff $\{u_n\}$ is bounded and, for each $k$, $\lim_{n\to\infty}\langle u_n,\varphi_k\rangle=\langle u,\varphi_k\rangle$.
    \item Show that any two infinite dimensional separable Hilbert spaces are isometrically isomorphic and that any such isomorphism preserves the inner product.
    \item Let $H$ be a Hilbert space and $V$ a closed separable subspace of $H$ for which $\{\varphi_k\}$ is an orthonormal basis.
    Show that the orthogonal projection of $H$ onto $V$, $P$, is given By
    \[
        P(h)=\sum_{k=1}^\infty \langle \varphi_k,h\rangle\varphi_k\text{ for all }h\in H.  
    \] 
    \item (Parseval's Indentities) Let $\{\varphi_k\}$ be an orthonormal basis for a Hilbert space $H$. Verify that
    \[
        \|h\|^2=\sum_{k=1}^\infty\langle \varphi_k,h\rangle^2\text{ for all }h\in H.
    \]
    Also verify that
    \[
        \langle u,v\rangle =  \sum_{k=1}^\infty a_k \cdot b_k\text{ for all }u,v\in H,
    \]
    where, for each natural number $k$, $a_k=\langle u,\varphi_k\rangle$ and $b_k=\langle v,\varphi_k\rangle$.
    \item Verify the assertions in the example of the orthonormal basis for $L^2[0,2\pi]$.
    \item Use Proposition 10 and the Stone-Weierstrass Theorem to show that for each $f\in L^2[-\pi,\pi]$,
    \[
        f(x)=a_0/2+\sum_{k=1}^\infty[a_k\cdot\cos kx+b_k\cdot\sin kx],  
    \]
    where the convergence is in $L^2[-\pi,\pi]$ and each 
    \[
        a_k=\frac{1}{\pi}\int_{-\pi}^\pi f(x)\cos kxdx\text{ and }b_k=\frac{1}{\pi}\int_{-\pi}^\pi f(x)\sin kxdx.
    \]
\end{enumerate}

% 16.4
\section{Adjoints and Symmetry for Linear Operators}
\begin{center}
	\textbf{PROBLEMS}
\end{center}
In the following problems, $H$ is a Hilbert space.
\begin{enumerate}
	\setcounter{enumi}{26}
    \item Verify (12).
    \item Let $T$ and $S$ belong to $\mathcal{L}(H)$ and be symmetric. 
    Show that $T=S$ iff $Q_T=Q_S$.
    \item Show the symmetric operators are a closed subspace of $\mathcal{L}(H)$.
    Also show that if $T$ and $S$ are symmetric, then so is the composition $S\circ T$ iff $T$ commutes with $S$ with respect to composition; that is, $S\circ T=T\circ S$.
    \item (Helliger-Toplitz) Let $H$ be a Hilbert space and the linear operator $T:H\to H$ have the property that $\langle T(u),v\rangle=\langle u,T(v)\rangle$ for all $u,v\in H$.
    Show that $T$ belongs to $\mathcal{L}(H)$.
    \item Exhibit an operator $T\in\mathcal{L}(\mathbb{R}^2)$ for which $\|T\|>\sup_{\|u\|=1}|\langle T(u),u\rangle|$.
    \item Let $S$ and $T$ in $\mathcal{L}(H)$ be symmetric.
    Assume $S\ge T$ and $T\ge S$.
    Prove that $T=S$.
    \item Let $V$ be a closed nontrivial subspace of a Hilbert space $H$ and $P$ the orthogonal projection of $H$ onto $V$.
    Show that $P=P^*$, $P\ge0$, and $\|P\|=1$.
    \item Let $P\in\mathcal{L}(H)$ be a projection.
    Show that $P$ is the orthogonal projection of $H$ onto $P(H)$ iff $P=P^*$.
    \item Let $\{\varphi_k\}$ be an orthonormal basis for a Hilbert space $H$ and for each natural number $n$, define $P_n$ to be the orthogonal projection of $H$ onto the linear span of $\{\varphi_1,\dots,\varphi_n\}$.
    Show that $P_n$ is symmetric and 
    \[
        0\le P_n\le P_{n+1}\le Id\text{ for all }n.
    \]
    Show that $\{P_n\}$ converges pointwise on $H$ to $Id$ but does not converge uniformly on the unit ball.
    \item Show that if $T\in\mathcal{L}(H)$ is invertible, so is $T^*\circ T$ and therefore so is $T$.
    \item (A General Cauchy-Schwarz Inequality) Let $T\in\mathcal{L}(H)$ be symmetric and nonnegative.
    Show that for all $u,v\in H$,
    \[
        |\langle T(u),v\rangle|^2\le\langle T(u),u\rangle\cdot\langle T(v),v\rangle.
    \] 
    \item Use the preceding problem to show that if $S,T\in\mathcal{L}(H)$ are symmetric and $S\ge T$, then for each $u\in H$,
    \[
        \|S(u)-T(u)\|^4=\langle (S-T)(u),(S-T)(u)\rangle^2\le|\langle(S-T)(u),u\rangle||\langle(S-T)^2(u),(S-T)(u)\rangle|
    \]
    and thereby conclude that
    \[
        \|S(u)-T(u)\|^4\le|\langle S(u),u\rangle-\langle T(u),u\rangle|\cdot\|S-T\|^3\cdot\|u\|^2.
    \]
    \item (a Monotone Convergence Theorem for Symmetric Operators) A sequence $\{T_n\}$ of symmetric operators in $\mathcal{L}(H)$ is said to be monotone increasing provided $T_{n+1}\ge T_n$ for each $n$, and said to be bounded above provided there is a symmetric operator $S$ in $\mathcal{L}(H)$ such that $T_n\le S$ for all $n$.
    \begin{enumerate}[(i)]
        \item Use the preceding problem to show that a monotone increasing sequence $\{T_n\}$ of symmetric operators in $\mathcal{L}(H)$ converges pointwise to a symmetric operator in $\mathcal{L}(H)$ iff it is bounded above.
        \item Show that a monotone increasing sequence $\{T_n\}$ of symmetric operators in $\mathcal{L}(H)$ is bounded above iff it is pointwise bounded; that is, for each $h\in H$, the sequence $\{T_n(h)\}$ is bounded.
    \end{enumerate}
    \item Let $S\in\mathcal{L}(H)$ be a symmetric operator for which $0\le S\le Id$.
    Define a sequence $\{T_n\}$ in $\mathcal{L}(H)$ by letting $T_1=1/2(Id-S)$ and if $n$ is a natural number for which $T_n\in\mathcal{L}(H)$ has been defined, defining $T_{n+1}=1/2(Id-S+T_n^2)$.
    \begin{enumerate}[(i)]
        \item Show that for each natural umber $n$, $T_n$ and $T_{n+1}-T_n$ are polynomials in $Id-S$ with nonnegative coefficients.
        \item Show that $\{T_n\}$ is a monotone increasing sequence of symmetric operators that is bounded above by $Id$.
        \item Use the preceding problem to show that $\{T_n\}$ converges pointwise to a symmetric operator $T$ for which $0\le T\le Id$ and $T=1/2(Id-S+T^2)$.
        \item Define $A=(Id-T)$. Show that $A^2=S$.
    \end{enumerate}
    \item (Square Roots of Nonnegative Symmetric Operators) Let $T\in\mathcal{L}(H)$ be a nonnegative symmetric operator.
    A nonnegative symmetric operator $A\in\mathcal{L}(H)$ is called a square root of $T$ provided $A^2=T$.
    Use the inductive construction in the preceding problem to show that $T$ has a square root $A$ which commutes with each operator in $\mathcal{L}(H)$ that commutes with $T$.
    Show that the square root is unique: it is denoted by $\sqrt{T}$.
    Finally, show that $T$ is invertible iff $\sqrt{T}$ is invertible.
    \item An invertible operator $T\in\mathcal{L}(H)$ is said to be \textbf{orthogonal} provided $T^{-1}=T^*$.
    Show that an invertible operator is orthogonal iff it is an isometry.
    \item (Polar Decompositions) Let $T\in\mathcal{L}(H)$ be invertible.
    Show that there is an orthogonal invertible operator $A\in\mathcal{L}(H)$ and a nonnegative symmetric invertible operator $B\in\mathcal{L}(H)$ such that $T=B\circ A$. 
    (Hint: show that $TT^*$ is invertible and symmetric and let $B=\sqrt{T\circ T^*}$.)
\end{enumerate}

% 16.5
\section{Compact Operators}
\begin{center}
	\textbf{PROBLEMS}
\end{center}
\begin{enumerate}
	\setcounter{enumi}{43}
    \item Show that if $H$ is infinite dimensional and $T\in\mathcal{L}(H)$ is invertible, then $T$ is not compact.
    \item Prove Proposition 18.
    \item Let $\mathcal{K}(H)$ denote the set of compact operators in $\mathcal{L}(H)$.
    Show that $\mathcal{K}(H)$ is a linear subspace of $\mathcal{L}(H)$.
    Moreover, show that for $K\in\mathcal{K}(H)$ and $T\in\mathcal{L}(H)$, both $K\circ T$ and $T\circ K$ belong to $\mathcal{K}(H)$.
    \item Show that a linear operator $T:H\to H$ is continuous iff it maps weakly convergent sequences to weakly convergent sequences.
    \item Show that $K\in\mathcal{L}(H)$ is compact iff whenever $\{u_n\}\to u$ in $H$ and $\{v_n\}\to v$ in $H$, then $\langle K(u_n),v_n\rangle\to\langle K(u),v\rangle$.
    \item Let $\{P_n\}$ be a sequence of orthogonal projections in $\mathcal{L}(H)$ with the property that for natural numbers $n$ and $m$, $P_n(H)$ and $P_m(H)$ are orthogonal finite dimensional subspaces of $H$.
    Let $\{\lambda_n\}$ be a bounded sequence of real numbers.
    Show that 
    \[
        K=\sum_{n=1}^\infty\lambda_n\cdot P_n
    \]
    is a properly defined symmetric operator in $\mathcal{L}(H)$ that is compact iff $\{\lambda_n\}$ converges to $0$.
    \item For $X$ a Banach space, define an operator $T\in\mathcal{L}(X)$ to be compact provided $T(B)$ has compact closure.
    Show that Proposition 18 holds for a general Banach space and Proposition 19 holds for a reflexive Banach space.
\end{enumerate}

% 16.6
\section{The Hilbert-Schmidt Theorem}
\begin{center}
	\textbf{PROBLEMS}
\end{center}
\begin{enumerate}
	\setcounter{enumi}{50}
    \item Let $H$ be a Hilbert space and $T\in\mathcal{L}(H)$ be compact and symmetric. Define
    \[
        \alpha=\underset{\|h\|=1}{\inf}\langle T(h),h\rangle\ \text{ and }\ \beta=\underset{\|h\|=1}{\sup}\langle T(h),h\rangle.
    \]
    Show that if $\alpha<0$, then $\alpha$ is an eigenvalue of $T$ and if $\beta>0$, then $\beta$ is an eigenvalue of $T$.
    Exhibit an example where $\alpha=0$ and yet $\alpha$ is not an eigenvalue of $T$; that is, $T$ is one-to-one (injective).
    \item Let $H$ be a Hilbert space and $K\in\mathcal{L}(H)$ be compact and symmetric.
    Suppose
    \[
        \underset{\|h\|=1}{\sup}\langle K(h),h\rangle=\beta>0.
    \] 
    Let $\{h_n\}$ be a sequence of unit vectors for which $\lim_{n\to\infty}\langle K(h_n),h_n\rangle=\beta$.
    Show that a subsequence of $\{h_n\}$ converges strongly to an eigenvector of $T$ with corresponding eigenvalue $\beta$.
\end{enumerate}

% 16.7
\section{The Riesz-Schauder Theorem: Characterization of Fredholm Operators}
\begin{center}
	\textbf{PROBLEMS}
\end{center}
\begin{enumerate}
	\setcounter{enumi}{52}
    \item Let $K\in\mathcal{L}(H)$ be compact.
    Show that $T=K^*K$ is compact and symmetric.
    Then use the Hilbert-Schmidt Theorem to show that there is an orthonormal sequence $\{\varphi_k\}$ of $H$ such that $T(\varphi_k)-\lambda_k\varphi_k$ for all $k$ and $T(h)=0$ if $h$ is orthogonal to $\{\varphi_k\}_{k=0}^\infty$.
    Conclude that if $h$ is orthogonal to $\{\varphi_k\}_{k=0}^\infty$, then
    \[
        \|K(h)\|^2\langle K(h),K(h)\rangle=\langle T(h),h\rangle=0.
    \]
    Define $H_0$ to be the closed linear span of $\{K^m(\varphi_k)\mid m\ge0,k\ge1\}$.
    Show that $H_0$ is closed and separable, $K(H_0)\subseteq H_0$ and $K=0$ on $H_0^\perp$.
    \item Let $\mathcal{K}(H)$ denote the set of compact operators in $\mathcal{L}(H)$.
    Show that $\mathcal{K}(H)$ is a closed subspace of $\mathcal{L}(H)$ that has the set of operators of finite rank as a dense subspace.
    Is $\mathcal{K}(H)$ an open subset of $\mathcal{L}(H)$?
    \item Show that the composition in either order of a Fredholm operator of index $0$ with an invertible operator is also Fredholm of index $0$.
    \item Show that the composition of two Fredholm operators of index $0$ is also Fredholm of index $0$.
    \item Show that an operator $T\in\mathcal{L}(H)$ is Fredholm of index $0$ iff it is the perturbation of an invertible operator by an operator of finite rank.
    \item Argue as follows to show that the collection of invertible operators in $\mathcal{L}(H)$ is an open subset of $\mathcal{L}(H)$.
    \begin{enumerate}[(i)]
        \item For $A\in\mathcal{L}(H)$ with $\|A\|<1$, use the completeness of $\mathcal{L}(H)$ to show that the so-called Neumann series $\sum_{n=0}^\infty A^n$ converges to an operator in $\mathcal{L}(H)$ that is the inverse of $Id-A$.
        \item For a invertible operator $S\in\mathcal{L}(H)$ show that for any $T\in\mathcal{L}(H)$, $T=S[Id+S^{-1}(T-S)]$.
        \item Use (i) and (ii) to show that if $S\in\mathcal{L}(H)$ is invertible then so is any $T\in\mathcal{L}(H)$ of which $\|S-T\|<1/\|S^{-1}\|$.
    \end{enumerate}
    \item Show that the set of operators in $\mathcal{L}(H)$ that are Fredholm of index $0$ is an open subset of $\mathcal{L}(H)$.
    \item By following the orthogonal approximation sequence method used in the proof of Proposition 22, provide another proof of Proposition 14 in case $H$ is separable.
    \item For $T\in\mathcal{L}(H)$, suppose that $\langle T(h),h\rangle\ge\|h\|^2$ for all $h\in H$.
    Assume that $K\in\mathcal{L}(H)$ is compact and $T+K$ is one-to-one.
    Show that $T+K$ is onto.
    \item Let $K\in\mathcal{L}(H)$ be compact an $\mu\in\mathbb{R}$ have $|\mu|>\|K\|$.
    Show that $\mu- K$ is invertible.
    \item Let $S\in\mathcal{L}(H)$ have $\|S\|<1$,$K\in\mathcal{L}(H)$ be compact and $(Id+S+K)(H)=H$.
    Show that $Id+S+K$ is one-to-one.
    \item Let $\mathcal{G}L(H)$ denote the set of invertible operators in $\mathcal{L}(H)$.
    \begin{enumerate}[(i)]
        \item Show that under the operation of composition of operators, $\mathcal{G}L(H)$ is a group: it is called the general linear group of $H$.
        \item An operator $T$ in $\mathcal{G}L(H)$ is said to be orthogonal, provided that $T^*=T^{-1}$.
        Show that the set of orthogonal operators is a subgroup of $\mathcal{G}L(H)$: it is called the orthogonal group.
    \end{enumerate} 
    \item Let $H$ be a Hilbert space, $T\in\mathcal{L}(H)$ be Fredholm of index zero, and $K\in\mathcal{L}(H)$ be compact.
    Show that $T+K$ is Fredholm of index zero.
    \item Let $X_0$ be a finite codimensional subspace of a Banach space $X$.
    Show that all finite dimensional linear complements of $X_0$ in $X$ have the same dimension.
\end{enumerate}


% Chapter 17
\chapter{General Measure Spaces: Their Properties and Construction}

\section{Measures and Measurable Sets}

\begin{center}
	\textbf{PROBLEMS}
\end{center}
\begin{enumerate}
	\setcounter{enumi}{0}
	\item Let $f$ be a nonnegative Lebesgue measurable function on $\mathbb{R}$. 
	For each Lebesgue measurable subset $E$ of $\mathbb{R}$, define $\mu(E) = \smallint_E f$, the Lebesgue integral of $f$ over $E$.
	Show that $\mu$ is a measure on the $\sigma$-algebra of Lebesgue measurable subsets of $\mathbb{R}$.
	\item Let $\mathcal{M}$ be a $\sigma$-algebra of subsets of a set $X$ and the set function $\mu : \mathcal{M} \to [0,\infty)$ be finitely additive.
	Prove that $\mu$ is a measure iff whenever $\{A_k\}_{k=1}^\infty$ is an ascending sequence of sets in $\mathcal{M}$, then
	\[
	\mu \biggl ( \bigcup_{k=1}^\infty A_k \biggr ) = \lim_{k \to \infty} \mu(A_k).	
	\]
\end{enumerate}

\section{Signed Measures: The Hahn and Jordan Decompositions}
\section{The Cath\'eodory Measure Induced by an Outer Measure}
\section{The Construction of Outer Measures}
\section{The Cath\'eodory-Hahn Theorem: The Extension of a Premeasure to a Measure}

% Chapter 18
\authoredby{inprogress}
\chapter{Integration Over General Measure Spaces}

% 18.1
\authoredby{inprogress}
\section{Measurable Functions}
Consider the measurable space $(X,\mathcal{M})$. 
For an extended real valued function $f$ of $X$ and a measurable subset $E$ of $X$, the restriction of $f$ to both $E$ and $X\setminus E$ are measurable iff $f$ is measurable on $X$.
\begin{namedthm*}{Proposition 3}
    Let $(X,\mathcal{M},\mu)$ be a complete measure space and $X_0$ be a measurable subset of $X$ for which $\mu(X\setminus X_0)=0$.
    Then an extended real valued function $f$ on $X$ is measurable iff its restriction to $X_0$ is measurable.
    In particular, if $g$ and $h$ are extended real valued functions on $X$ for which $g=h$ a.e. on $X$, then $g$ is measurable iff $h$ is measurable.
\end{namedthm*}
\begin{namedthm*}{Theorem 6}
    Let $(X,\mathcal{M},\mu)$ be a measure space and $\{f_n\}$ a sequence of measurable functions on $X$ for which $\{f_n\}\to f$ pointwise a.e. on $X$.
    If either the measure space $(X,\mathcal{M},\mu)$ is complete or the convergence is pointwise on all of $X$, then $f$ is measurable.
\end{namedthm*}
\begin{namedthm*}{Corollary 7}
    Let $(X,\mathcal{M},\mu)$ be a measure space and $\{f_n\}$ be a sequence of measurable function on $X$.
    Then the following functions are measurable:
    \[
        \sup\{f_n\},\ \inf\{f_n\},\ \limsup\{f_n\},\ \liminf\{f_n\}. 
    \]
\end{namedthm*}
\begin{namedthm*}{Egoroff's Theorem}
    Let $(X,\mathcal{M},\mu)$ be a finite measure space and $\{f_n\}$ a sequence of measurable functions on $X$ that converges pointwise a.e. on $X$ to a function $f$ that is finite a.e. on $X$.
    Then for each $\epsilon>0$, there is a measurable subset $X_\epsilon$ of $X$ for which 
    \[
        \{f_n\}\to f\text{ uniformly on }X_\epsilon\text{ and }\mu(X\setminus  X_\epsilon)<\epsilon.
    \]
\end{namedthm*}
\begin{center}
	\textbf{PROBLEMS}
\end{center}
In the following problems $(X,\mathcal{M},\mu)$ is a reference measure space and measurable means with respect to $\mathcal{M}$.
\begin{enumerate}
	\setcounter{enumi}{0}
    \item Show that an extended real valued function on $X$ is measurable iff $f^{-1}\{\infty\}$ and $f^{-1}\{-\infty\}$ are measurable and so is $f^{-1}(E)$ for every Borel set of real numbers.\\
    \\Let $f$ be an extended real valued function on $X$.\\
    \\$(\implies)$ Suppose that $f$ is measurable. 
    \\Then the set
    \[
        f^{-1}\{\infty\}=\{x\in X\mid f(x)=\infty\}=\bigcap_{n=1}^\infty\{x\in X\mid f(x)>n\},
    \]
    is measurable because it is a countable intersection of measurable sets.
    \\Similarly the set
    \[
        f^{-1}\{-\infty\}=\{x\in X\mid f(x)=-\infty\}=\bigcap_{n=1}^\infty\{x\in X\mid f(x)<n\},
    \]
    is measurable because it is a countable intersection of measurable sets.
    \\By definition of measurable function we have that the set
    \[
        \{x\in X\mid f(x)>a\}=\{x\in X\mid f(x)\in(a,\infty)\}
    \]
    is measurable for any real $a$.
    From Chapter 2 Problem 11, we have that that if a $\sigma$-algebra of subsets of R contains intervals of the form $(a,\infty)$, then it contains all intervals.
    Therefore by the properties of a $\sigma$-algebra, it must contain all open sets (Chapter 1 Proposition 9 - Every nonempty open set is the union of a countable, disjoint collection of open intervals).
    Then because the collection of Borel sets is the smallest $\sigma$-algebra that contains all of the open sets of real numbers, we have
    \[
        f^{-1}(E)\text{ is measurable for any Borel set }E.
    \]
    \\$(\impliedby)$ Suppose that the sets $f^{-1}\{\infty\}$ and $f^{-1}\{-\infty\}$ are measurable, and that $f^{-1}(E)$ is measurable for any Borel set $E$ of real numbers.
    \\Fix any real number $c$.
    \\The collection of Borel sets contain all intervals of the form $(c,\infty)$ so that the set
    \[
        \{x\in X\mid f(x)> c\}=\{x\in X\mid f(x)\in(c,\infty)\}=f^{-1}((c,\infty))
    \]
    is measurable.
    \\Therefore $f$ is a measurable function.
    \item Suppose $(X,\mathcal{M},\mu)$ is not complete.
    Let $E$ be a subset of a set of measure zero that does not belong to $\mathcal{M}$.
    Let $f=0$ on $X$ and $g=\chi_E$.
    Show that $f=g$ a.e. on $X$ while $f$ is measurable and $g$ is not.
    \item Suppose $(X,\mathcal{M},\mu)$ is not complete.
    Show that there is a sequence $\{f_n\}$ of measurable functions on $X$ that converges pointwise a.e. on $X$ to a function $f$ that is not measurable.
    \item Let $E$ be a measurable subset of $X$ and $f$ an extended real-valued function on $X$.
    Show that $f$ is measurable iff its restrictions to $E$ and $X\setminus E$ are measurable.
    \item Show that an extended real valued function $f$ on $X$ is measurable iff for each rational number $c$, $\{x\in X\mid f(x)<c\}$ is a measurable set.\\
    \\Let $f$ be an extended real valued function on $X$.\\
    \\$(\implies)$ Suppose that $f$ is measurable.
    \\Then trivially for any rational number $c$, $\{x\in X\mid f(x)<c\}$ is a measurable set by definition of measurable function.\\
    \\$(\impliedby)$ Suppose that for each rational number $c$, $\{x\in X\mid f(x)<c\}$ is a measurable set.
    \\Let $a$ be any real number.
    \\Then for each natural number $n$, by density of the rationals in the reals there exists a rational $c_n$ such that 
    \[
        a_n-\frac{1}{n}<c_n< a_n,
    \]
    and we have that the set $\{x\in X\mid f(x)<c_n\}$ is measurable.
    \\Then we have $\bigcup_{n=1}^\infty[\infty,c_n)=[\infty,a)$, so that we have the set
    \begin{align*}
        \bigcup_{n=1}^\infty\{x\in X\mid f(x)<c_n\}
        &=\bigcup_{n=1}^\infty\{x\in X\mid f(x)\in[\infty,c_n)\}\\
        &=\{x\in X\mid f(x)\in[\infty,a)\},\\
        &=\{x\in X\mid f(x)<a\},
    \end{align*}
    which is measurable because it is the countable union of measurable sets.
    \\Therefore $f$ is a measurable function.
    \item Consider two extended real valued measurable functions $f$ and $g$ on $X$ that are finite a.e. on $X$.
    Define $X_0$ to be the set of points in $X$ at which both $f$ and $g$ are finite.
    Show that $X_0$ is measurable and $\mu(X\setminus X_0)=0$.\\
    \\We have that $f$ and $g$ are finite a.e. on $X$, which means that there exist (measurable) subsets $X_f$, $X_g$ of $X$, both of measure zero, where the property holds on $X\setminus X_f$ and $X\setminus X_g$ respectively.
    \\Then $X_f$ and $X_g$ are measurable imply that $X\setminus X_f$ and $X\setminus X_g$ are also measurable by the properties of a $\sigma$-algebra.
    \\Therefore we have that the set
    \[
        X_0=[X\setminus X_f]\cap[X\setminus X_g]=\{x\in X\mid f(x)\text{ is finite},g(x)\text{ is finite}\}
    \]
    is measurable because it is the intersection of measurable sets.
    \\We see that 
    \[
        X\setminus X_0=X\cap([X^c\cup X_f]\cup[X^c\cup X_g])=X_f\cup X_g,
    \]
    and we use countable monotonicity to see that
    \[
        \mu(X\setminus X_0)=\mu(X_f\cup X_g)\le\mu(X_f)+\mu(X_g)=0+0.
    \]
    \item Let $X$ be a nonempty set.
    Show that every extended real valued function on $X$ is measurable w.r.t. the measurable space $(X,2^X)$.
    \begin{enumerate}[(i)]
        \item Let $x_0$ belong to $X$ and $\delta_{x_0}$ be the Dirac measure at $x_0$ on $2^X$.
        Show that two function on $X$ are equal a.e. $[\delta_{x_0}]$ iff they take the same value at $x_0$.
        \item Let $\eta$ be the counting measure on $2^X$.
        Show that two functions on $X$ are equal a.e. $[\eta]$ iff they take the same value at every point in $X$.
    \end{enumerate}
    \item Let $X$ be a topological space and $\mathcal{B}(X)$ be the smallest $\sigma$-algebra containing the topology on $X$.
    $\mathcal{B}(X)$ is called the Borel $\sigma$-algebra associated with the topological space $X$.
    Show that any continuous real valued function on $X$ is measurable w.r.t. the Borel measurable space $(X,\mathcal{B}(X))$.
    \item If a real valued function on $\mathbb{R}$ is measurable w.r.t. the $\sigma$-algebra of Lebesgue measurable sets, is it necessarily measurable w.r.t. the Borel measurable space $(\mathbb{R},\mathcal{B}(\mathbb{R}))$?
    \item Check that the proofs of Proposition 1 and Theorem 4 follow from the proofs of the corresponding results in the case of Lebesgue measure on the real line.
    \item Prove Corollary 7.\\
    \\Let $(X,\mathcal{M},\mu)$ be a measure space and $\{f_n\}$ be a sequence of measurable function on $X$.
    \\\begin{enumerate}[(i)]
        \item $f(x):=\sup_{n\in\mathbb{N}}\{f_n(x)\}$
        \\Fix any real number $c$.
        \\(1) Let $y\in\{x\in X\mid f(x)>c\}$.
        Then $f(y)>c$.
        By definition of supremum, there exists an index $k$ such that $f(y)\ge f_k(y)>c$, and therefore $y\in\{x\in X\mid f_k(x)>c\}$ for some $k$, which implies
        \[
            \{x\in X\mid f(x)>c\}\subseteq\bigcup_{n=1}^\infty\{x\in X\mid f_n(x)>c\}.\tag{1}
        \]
        (2) Let $y'\in\bigcup_{n=1}^\infty\{x\in X\mid f_n(x)>c\}$.
        Then there exists an index $k$ such that $y'\in\{x\in X\mid f_k(x)>c\}$.
        By definition of supremum, we have $f(y')\ge f_k(y')>c$, and therefore $y'\in\{x\in X\mid f(x)>c\}$, which implies
        \[
            \{x\in X\mid f(x)>c\}\supseteq\bigcup_{n=1}^\infty\{x\in X\mid f_n(x)>c\}\tag{2}
        \]
        Then by (1) and (2),
        \[
            \{x\in X\mid f(x)>c\}=\bigcup_{n=1}^\infty\{x\in X\mid f_n(x)>c\},
        \]
        which is measurable because it is the countable union of measurable sets.
        % \\We may also prove that $\{x\in X\mid f(x)<c\}=\bigcap_{n=1}^\infty\{x\in X\mid f_n(x)<c\}$ in a similar way.
        \item $f(x):=\inf_{n\in\mathbb{N}}\{f_n(x)\}$
        \\Fix any real number $c$.
        \\(1) Let $y\in\{x\in X\mid f(x)\ge c\}$.
        Then $f(y)\ge c$.
        By definition of infimum, for all indices $n$, we have that $c\le f(y)\le f_n(y)$, and therefore $y\in\{x\in X\mid f_n(x)\ge c\}$ for all $n$, which implies
        \[
            \{x\in X\mid f(x)\ge c\}\subseteq\bigcap_{n=1}^\infty\{x\in X\mid f_n(x)\ge c\}.\tag{1}
        \]
        (2) Let $y'\in\bigcap_{n=1}^\infty\{x\in X\mid f_n(x)\ge c\}$.
        Then for all indices $n$, we have $y'\in\{x\in X\mid f_n(x)\ge c\}$ so that $c$ is a lower bound to the set $\{f_n(y')\}_{n\in\mathbb{N}}$.
        Then by definition of infimum we have $f(y')\ge c$, and therefore $y'\in\{x\in X\mid f(x)\ge c\}$, which implies
        \[
            \{x\in X\mid f(x)\ge c\}\supseteq\bigcap_{n=1}^\infty\{x\in X\mid f_n(x)\ge c\}\tag{2}
        \]
        Then by (1) and (2),
        \[
            \{x\in X\mid f(x)\ge c\}=\bigcap_{n=1}^\infty\{x\in X\mid f_n(x)\ge c\},
        \]
        which is measurable because it is the countable intersection of measurable sets.
        % \\Fix any real number $c$.
        % \\(1) Let $y\in\{x\in X\mid f(x)<c\}$.
        % Then $f(y)<c$.
        % \\By definition of infimum, there exists an index $k$ such that 
        % \[
        %     f(y)\le f_k(y)<c,
        % \]
        % and therefore $y\in\{x\in X\mid f_k(x)<c\}$, which implies
        % \[
        %     \{x\in X\mid f(x)<c\}\subseteq\bigcup_{n=1}^\infty\{x\in X\mid f_n(x)<c\}.\tag{1}
        % \]
        % (2) Let $y'\in\bigcup_{n=1}^\infty\{x\in X\mid f_n(x)<c\}$.
        % \\Then there exists an index $k$ such that $y'\in\{x\in X\mid f_k(x)<c\}$.
        % \\By definition of infimum, we have 
        % \[
        %     f(y')\le f_k(y')<c,
        % \]
        % and therefore $y'\in\{x\in X\mid f(x)<c\}$, which implies
        
        % \[
        %     \{x\in X\mid f(x)<c\}\supseteq\bigcup_{n=1}^\infty\{x\in X\mid f_n(x)<c\}\tag{2}
        % \]
        % Then by (1) and (2),
        % \[
        %     \{x\in X\mid f(x)<c\}=\bigcup_{n=1}^\infty\{x\in X\mid f_n(x)<c\},
        % \]
        % which is measurable because it is the countable union of measurable sets.\\
        \item $f(x):=\limsup_n\{f_n(x)\}=\lim_{n\to\infty}\sup_{k\ge n}\{f_k(x)\}=\inf_{n\in\mathbb{N}}\sup_{k\ge n}\{f_k(x)\}$
        \\Fix any real number $c$.
        \\Pulling from (i), with a small modification, we have that the function $g_n(x):=\sup_{k\ge n}\{f_k(x)\}$ is measurable for each $n\in\mathbb{N}$.
        \\That is, the set
        \[
            \{x\in X\mid g_n(x)> c\}=\bigcup_{k=n}^\infty\{x\in X\mid f_k(x)> c\}
        \]
        is measurable, and thus each function $g_n$ is measurable.
        \\Then using the same process as (ii), the set
        \[
            \{x\in X\mid f(x)\ge c\}=\bigcap_{n=1}^\infty\{x\in X\mid g_n(x)\ge c\}%=\bigcap_{n=1}^\infty\bigcup_{k=n}^\infty\{x\in X\mid f_k(x)\ge c\}
        \]
        is measurable because it is a countable intersection of measurable sets.
        % \\(The first set equality can be proved using the same process as (ii).)
        \item $f(x):=\liminf_n\{f_n(x)\}=\lim_{n\to\infty}\inf_{k\ge n}\{f_k(x)\}=\sup_{n\in\mathbb{N}}\inf_{k\ge n}\{f_k(x)\}$
        \\Fix any real number $c$.
        \\Pulling from (ii), with modification, we have that the function $g_n(x):=\inf_{k\ge n}\{f_k(x)\}$ is measurable for each $n\in\mathbb{N}$.
        \\That is, the set
        \[
            \{x\in X\mid g_n(x)\ge c\}=\bigcap_{k=n}^\infty\{x\in X\mid f_k(x)\ge c\}
        \]
        is measurable, and thus each function $g_n$ is measurable.
        \\Then using the same process as (i), the set
        \[
            \{x\in X\mid f(x)> c\}=\bigcup_{n=1}^\infty\{x\in X\mid g_n(x)> c\}%=\bigcup_{n=1}^\infty\bigcap_{k=n}^\infty\{x\in X\mid f_k(x)> c\}
        \]
        is measurable because it is a countable union of measurable sets.
        % \\(The first set equality can be proved using the same process as (i).)

        % The strict vs unstrict inequality might be a problem for (iii) and (iv) in the second part.
    \end{enumerate}
    \item Prove Egoroff's Theorem.
    Is Egoroff's Theorem true in the absence of the assumption that the limit function is finite a.e.?
    \item Let $\{f_n\}$ be a sequence of real valued measurable functions on $X$ such that, for each natural number $n$, $\mu\{x\in X\mid |f_n(x)-f_{n+1}(x)|>1/2^n\}<1/2^n$.
    Show that $\{f_n\}$ is pointwise convergent a.e. on $X$. 
    (Hint: Use the Borel-Cantelli Lemma.)
    \item Under the assumptions of Egoroff's Theorem, show that $X=\bigcup_{k=0}^\infty X_k$, where each $X_k$ is measurable, $\mu(X_0)=0$ and, for $k\ge1$, $\{f_n\}$ converges uniformly to $f$ on $X_k$.
    \item A sequence $\langle f_n\rangle$ of measurable real-valued functions on $X$ is said to \textbf{converge in measure} to a measurable function $f$ provided that for each $\eta>0$,
    \[
        \lim_{n\to\infty}\mu\{x\in X\mid|f_n(x)-f(x)>\eta|\}=0.
    \]
    A sequence $\langle f_n\rangle$ of measurable functions is said to \textbf{Cauchy in measure} provided that for each $\epsilon>0$ and $\eta>0$, there is an index $N$ such that for each $m,n\ge N$,
    \[
        \mu\{x\in X\mid|f_n(x)-f_m(x)>\eta|\}<\epsilon.
    \]
    \begin{enumerate}[(i)]
        \item Show that if $\mu(X)<\infty$ and $\{f_n\}$ converges pointwise a.e. on $X$ to a measurable function $f$, then $\{f_n\}$ converges to $f$ in measure.
        (Hint: Use Egoroff's Theorem.)
        \item Show that if $\{f_n\}$ converges to $f$ in measure, then there is a subsequence of $\{f_n\}$ that converges pointwise a.e. on $X$ to $f$.
        (Hint: Use the Borel-Cantelli Lemma.)
        \item Show that if $\{f_n\}$ is Cauchy in measure, then there is a measurable function $f$ to which $\{f_n\}$ converges in measure.
    \end{enumerate}
    \item Assume $\mu(X)<\infty$.
    Show that $\{f_n\}\to f$ in measure iff each subsequence of $\{f_n\}$ has a further subsequence that converges pointwise a.e. on $X$ to $f$.
    Use this to show that for two sequences that converge in measure, the product sequence also converges in measure to the product of the limits.
\end{enumerate}

% 18.2
\authoredby{untouched}
\section{Integration of Nonnegative Measurable Functions}
\begin{center}
	\textbf{PROBLEMS}
\end{center}
\begin{enumerate}
	\setcounter{enumi}{16}
    \item 
\end{enumerate}

% 18.3
\section{Integration of General Measurable Functions}
\begin{center}
	\textbf{PROBLEMS}
\end{center}
\begin{enumerate}
	\setcounter{enumi}{26}
    \item 
\end{enumerate}

% 18.4
\authoredby{inprogress}
\section{The Radon-Nikodym Theorem}
Let $(X,\mathcal{M})$ be a measurable space.
For $\mu$ a measure on $(X,\mathcal{M})$ and $f$ a nonnegative function on $X$ that is measurable w.r.t. $\mathcal{M}$, define the set function $\nu$ on $\mathcal{M}$ by
\[
    \nu(E)=\int_Efd\mu\text{ for all }E\in\mathcal{M}.
\]

\begin{namedthm*}{The Radon-Nikodym Theorem}
    Let $(X,\mathcal{M},\mu)$ be a $\sigma$-finite measure space and $\nu$ a $\sigma$-finite measure defined on the measurable space $(X,\mathcal{M})$ that is absolutely continuous with respect to $\mu$.
    Then there is a nonnegative function $f$ on $X$ that is measurable with respect to $\mathcal{M}$ for which 
    \[
        \nu(E)=\int_Efd\mu\quad\text{for all }E\in\mathcal{M}.
    \]
    The function $f$ is unique in the sense that if $g$ is any nonnegative measurable function on $X$ that also has this property, then $g=f$ a.e. $[\mu]$.
\end{namedthm*}

\begin{center}
	\textbf{PROBLEMS}
\end{center}
\begin{enumerate}
	\setcounter{enumi}{48}
    \item Show that the Radon-Nikodym Theorem for finite measures $\mu$ and $\nu$ implies the theorem for $\sigma$-finite measures $\mu$ and $\nu$.
    \item Establish the uniqueness of the function $f$ in the Radon-Nikodym Theorem.
    
    Suppose that both $f$ and $g$ satisfy the RND property.
    Then 
    \[
        \int_Efd\mu=\nu(E)=\int_Egd\mu,\quad\text{for any }E\in\mathcal{M}.
    \]
    And so for $\{f>g\}\in\mathcal{M}$, we have
    \[
        \int_{\{f>g\}}(f-g)d\mu=0.
    \]
    Because $f-g>0$, then we must have that $\{f>g\}$ has measure zero.
    The same argument can be used for $\{f<g\}$.
    Therefore $f=g$ $\mu$-a.e., as $\{f\neq g\}=\{f>g\}\cup\{f<g\}$ has measure zero.
\end{enumerate}

% 18.5
\authoredby{untouched}
\section{The Nikodym Metric Space: The Vitali-Hahn-Saks Theorem}
\begin{center}
	\textbf{PROBLEMS}
\end{center}
\begin{enumerate}
	\setcounter{enumi}{60}
    \item 
\end{enumerate}

% Chapter 19
\chapter{General $L^p$ spaces: Completeness, Duality, and Weak Convergence}

\section{The Completeness of $L^p(X,\mu),1\le p \le \infty$}
\section{The Riesz Representation Theorem for the Dual of $L^p(X,\mu),1\le p < \infty$}
\section{The Kantorovitch Representation Theorem for the Dual of $L^\infty(X,\mu)$}
\section{Weak Sequential Compactness in $L^p(X,\mu),1< p < \infty$}
\section{Weak Sequential Compactness in $L^1(X,\mu)$: The Dunford-Pettis Theorem}

% Chapter 20
\authoredby{untouched}
\chapter{The Construction of Particular Measures}

% 20.1
\authoredby{untouched}
\section{Product Measures: The Theorems of Fubini and Tonelli}

% 20.2
\authoredby{untouched}
\section{Lebesgue Measure on Euclidean Space $\mathbb{R}^n$}

% 20.3
\authoredby{inprogress}
\section{Cumulative Distribution Functions and Borel Measures on $\mathbb{R}$}
For a bounded Lebesgue measurable function $f$ on $[a,b]$, denote the Lebesgue integral of $f$ over $[a,b]$ by $\int_{[a,b]}fdm$.
For a bounded function $f$ on $[a,b]$ whose set of discontinuities has Lebesgue measure zero, we proved that the Riemann integral $\int_a^bf(x)dx$ is defined and 
\[
    \int_{[a,b]}fdm=\int_a^bf(x)dx.
\]
There are two generalizations of these integrals, the Lebesgue-Stieltjes integral and Riemann-Stieltjes integral.

Let the function $g:I\to\mathbb{R}$ be increasing and continuous on the right.
For a bounded measurable function $f:I\to\mathbb{R}$, define the \textbf{Lebesgue-Stieltjes integral} of $f$ with respect to $g$ over $[a,b]$, which we denote by $\int_{[a,b]}fdg$, by

% 20.4
\authoredby{untouched}
\section{Carath\'eodory Outer Measures and Hausdorff Measures on a Metric Space}

% Chapter 21
\chapter{Measure and Topology}

\section{Locally Compact Topological Spaces}
\section{Separating Sets and Extending Functions}
\section{The Construction of Radon Measures}
\section{The Representation of Positive Linear Functionals on $C_c(X)$: The Riesz-Markov Theorem}
\section{The Riesz Representation Theorem for the Dual of $C(X)$}
\section{Regularity Properties of Baire Measures}

% Chapter 22
\chapter{Invariant Measures}

\section{Topological Groups: The General Linear Group}
\section{Kakutani's Fixed Point Theorem}
\section{Invariant Borel Measures on Compact Groups: von Neumann's Theorem}
\section{Measure-Preserving Transformations and Ergodicity: The Bogoliubov-Krilov Theorem}


% Chapter 3
\chapter{Lebesgue Measurable Functions}

\section{Sums, Products, and Compositions}
\section{Sequential Pointwise Limits and Simple Approximation}
\section{Littlewood's Three Principles, Ergoff's Theorem, and Lusin's Theorem}

% Chapter 4
\chapter{Lebesgue Integration}

\section{The Riemann Integral}
\section{The Lebesgue Integral of a Bounded Measurable Function over a Set of Finite Measure}
\section{The Lebesgue Integral of a Measurable Nonnegative Function}
\section{The General Lebesgue Integral}
\section{Countable Additivity and Continuity of Integration}
\section{Uniform Integrability: The Vitali Convergence Theorem}

% Chapter 5
\chapter{Lebesgue Integration: Further Topics}

\section{Uniform Integrability and Tightness: A General Vitali Convergence Theorem}
\section{Convergence in Measure}
\section{Characterizations of Riemann and Lebesgue Integrability}

% Chapter 6
\chapter{Differentiation and Integration}

\section{Continuity of Monotone Functions}
\section{Differentiability of Monotone Functions: Lebesgue's Theorem}
\section{Functions of Bounded Variation: Jordan's Theorem}
\section{Absolutely Continuous Functions}
\section{Integrating Derivatives: Differentiating Indefinite Integrals}
\section{Convex Functions}

% Chapter 7
\chapter{The $L^p$ Spaces: Completeness and Approximation}

\section{Normed Linear Spaces}
\begin{center}
	\textbf{PROBLEMS}
\end{center}
\begin{enumerate}
	\setcounter{enumi}{0}
	\item For $f$ in $C[a,b]$, Define
	\[
	\| f \|_1 = \int_a^b |f|.	
	\]
	Show that this is a norm on $C[a,b]$.
	Also show that there is no number $c \ge 0$ for which
	\[
	\| f \|_{\max}	\le c \| f \|_1 \text{ for all $f$ in $C[a,b]$},
	\]
	but there is a $c \ge 0$ for which 
	\[
	\| f \|_1	\le c \| f \|_{\max} \text{ for all $f$ in $C[a,b]$}.
	\]
	\item Let $X$ be the family of all polynomials with real coefficients defined on $\mathbb{R}$.
	Show that this is a linear space. For a polynomial $p$, define $\| p\|$ to be the sum of the absolute values of the coefficients of $p$.
	Is this a norm?
	\item For $f$ in $L^1[a,b]$, define $\|f\| = \smallint_a^b x^2 |f(x)|dx$.
	Show that this is a norm on $L^1[a,b]$.
	\item For $f$ in $L^\infty[a,b]$, show that 
	\[
	\| f\|_\infty = \min \biggl \{ M \ \biggl |\ m \{x \in [a,b]\ |\ |f(x)| > M \} =0 \biggr \},
	\] 
	and if, furthermore, $f$ is continuous on $[a,b]$, that
	\[
	\| f \|_{\infty} = \| f \|_{\max}.	
	\]
	\item Show that $\ell^\infty$ and $\ell^1$ are normed linear spaces.
\end{enumerate}

\section{The Inequalities of Young, H\"older, and Minkowski}
\section{$L^p$ is Complete: The Riesz-Fischer Theorem}
\section{Approximation and Separability}

% Chapter 8
\chapter{The $L^p$ Spaces: Duality and Weak Convergence}

\section{The Riesz Representation for the Dual of $L^p,a\le p\le \infty$}
\section{Weak Sequential Convergence in $L^p$}
\section{Weak Sequential Compactness}
\section{The Minimization of Convex Functionals}

\setcounter{chapter}{0}
\chapter*{II ABSTRACT SPACES: METRIC, TOPOLOGICAL, BANACH, AND HILBERT SPACES}
\addcontentsline{toc}{chapter}{II ABSTRACT SPACES: METRIC, TOPOLOGICAL, BANACH, AND HILBERT SPACES}
\setcounter{chapter}{8}

% Chapter 9
\chapter{Metric Spaces: General Properties}

\section{Examples of Metric Spaces}
\section{Open Sets, Closed Sets, and Convergent Sequences}
\section{Continuous Mappings Between Metric Spaces}
\section{Complete Metric Spaces}
\section{Compact Metric Spaces}
\section{Separable Metric Spaces}

% Chapter 10
\chapter{Metric Spaces: Three Fundamental Theorems}

\section{The Arzel\'a-Ascoli Theorem}
\section{The Baire Category Theorem}
\section{The Banach Contraction Principle}

% Chapter 11
\chapter{Topological Spaces: General Properties}

\section{Open Sets, Closed Sets, Bases, and Subbases}
\section{The Separation Properties}
\section{Countability and Separability}
\section{Continuous Mappings Between Topological Spaces}
\section{Compact Topological Spaces}
\section{Connected Topological Spaces}

% Chapter 12
\chapter{Topological Spaces: Three Fundamental Theorems}

\section{Urysohn's Lemma and the Tietze Extension Theorem}
\section{The Tychonoff Product Theorem}
\section{Thye Stone-Weierstrass Theorem}

% Chapter 13
\chapter{Continuous Linear Operators Between Banach Spaces}

\section{Normed Linear Spaces}
\section{Linear Operators}
\section{Compactness Lost: Infinite Dimensional Normed Linear Spaces}
\section{The Open Mapping and Closed Graph Theorems}\
\section{The Uniform Boundedness Principle}

% Chapter 14
\chapter{Duality for Normed Linear Spaces}

\section{Linear Functionals, Bounded Linear Functionals, and Weak Topologies}
\section{The Hahn-Banach Theorem}
\section{Reflexive Banach Spaces and Weak Sequential Convergence}
\section{Locally Convex Topological Vector Spaces}
\section{The Separation of Convex Sets and Mazur's Theorem}
\section{The Krein-Milman Theorem}

% Chapter 15
\chapter{Compactness Regained: The Weak Topology}

\section{Alaoglu's Extension of Helley's Theorem}
\section{Reflexivity and Weak Compactness: Kakutani's Theorem}
\section{Compactness and Weak Sequential Compactness: The Eberlein-\v Smulian Theorem}
\section{Metrizability of Weak Topologies}

% Chapter 16
\chapter{Continuous Linear Operators on Hilbert Spaces}

\section{The Inner Product and Orthogonality}
\section{The Dual Space and Weak Sequential Convergence}
\section{Bessel's Inequality and Orthonormal Bases}
\section{Adjoints and Symmetry for Linear Operators}
\section{Compact Operators}
\section{The Hilbert-Schmidt Theorem}
\section{The Riesz-Schauder Theorem: Characterization of Fredholm Operators}

\setcounter{chapter}{0}
\chapter*{III MEASURE AND INTEGRATION: GENERAL THEORY} 
\addcontentsline{toc}{chapter}{III MEASURE AND INTEGRATION: GENERAL THEORY}
\setcounter{chapter}{16}

% Chapter 17
\chapter{General Measure Spaces: Their Properties and Construction}

\section{Measures and Measurable Sets}

\begin{center}
	\textbf{PROBLEMS}
\end{center}
\begin{enumerate}
	\setcounter{enumi}{0}
	\item Let $f$ be a nonnegative Lebesgue measurable function on $\mathbb{R}$. 
	For each Lebesgue measurable subset $E$ of $\mathbb{R}$, define $\mu(E) = \smallint_E f$, the Lebesgue integral of $f$ over $E$.
	Show that $\mu$ is a measure on the $\sigma$-algebra of Lebesgue measurable subsets of $\mathbb{R}$.
	\item Let $\mathcal{M}$ be a $\sigma$-algebra of subsets of a set $X$ and the set function $\mu : \mathcal{M} \to [0,\infty)$ be finitely additive.
	Prove that $\mu$ is a measure iff whenever $\{A_k\}_{k=1}^\infty$ is an ascending sequence of sets in $\mathcal{M}$, then
	\[
	\mu \biggl ( \bigcup_{k=1}^\infty A_k \biggr ) = \lim_{k \to \infty} \mu(A_k).	
	\]
\end{enumerate}

\section{Signed Measures: The Hahn and Jordan Decompositions}
\section{The Cath\'eodory Measure Induced by an Outer Measure}
\section{The Construction of Outer Measures}
\section{The Cath\'eodory-Hahn Theorem: The Extension of a Premeasure to a Measure}

% Chapter 18
\chapter{Integration Over General Measure Spaces}

\section{Measurable Functions}
\section{Integration of Nonnegative Measurable Functions}
\section{Integration of General Measurable Functions}
\section{The Radon-Nikodym Theorem}
\section{The Nikodym Metric Space: The Vitali-Hahn-Saks Theorem}

% Chapter 19
\chapter{General $L^p$ spaces: Completeness, Duality, and Weak Convergence}

\section{The Completeness of $L^p(X,\mu),1\le p \le \infty$}
\section{The Riesz Representation Theorem for the Dual of $L^p(X,\mu),1\le p < \infty$}
\section{The Kantorovitch Representation Theorem for the Dual of $L^\infty(X,\mu)$}
\section{Weak Sequential Compactness in $L^p(X,\mu),1< p < \infty$}
\section{Weak Sequential Compactness in $L^1(X,\mu)$: The Dunford-Pettis Theorem}

% Chapter 20
\chapter{The Construction of Particular Measures}

\section{Product Measures: The Theorems of Fubini and Tonelli}
\section{Lebesgue Measure on Euclidean Space $\mathbb{R}^n$}
\section{Cumulative Distribution Functions and Borel Measures on $\mathbb{R}$}
\section{Carath\'eodory Outer Measures and Hausdorff Measures on a Metric Space}

% Chapter 21
\chapter{Measure and Topology}

\section{Locally Compact Topological Spaces}
\section{Separating Sets and Extending Functions}
\section{The Construction of Radon Measures}
\section{The Representation of Positive Linear Functionals on $C_c(X)$: The Riesz-Markov Theorem}
\section{The Riesz Representation Theorem for the Dual of $C(X)$}
\section{Regularity Properties of Baire Measures}

% Chapter 22
\chapter{Invariant Measures}

\section{Topological Groups: The General Linear Group}
\section{Kakutani's Fixed Point Theorem}
\section{Invariant Borel Measures on Compact Groups: von Neumann's Theorem}
\section{Measure-Preserving Transformations and Ergodicity: The Bogoliubov-Krilov Theorem}

\end{document}