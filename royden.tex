\documentclass[a4paper,10pt]{book}
\usepackage[utf8]{inputenc}
\usepackage[T1]{fontenc}
\usepackage[]{mdframed}
\usepackage{lipsum}
\usepackage{xcolor}
\usepackage{amsfonts} 
\usepackage{times}
\usepackage[shortlabels]{enumitem}
\usepackage{amsmath}
\usepackage{centernot}
\usepackage{amsthm,amssymb}
\usepackage{verbatim}
\usepackage{multicol}
\usepackage{titletoc}
\usepackage{graphicx}
\usepackage{minitoc}

%\newtheorem*{theorem*}{}
%\newtheorem{theorem}{Theorem}

\usepackage{amsthm}
%\usepackage{thmtools}

\newtheorem{theorem}{Theorem}[section] % the main one
%\newtheorem{lemma}[theorem]{Lemma}

\theoremstyle{plain} % just in case the style had changed
\newcommand{\thistheoremname}{}

\begin{comment}
\newtheorem{genericthm}[theorem]{\thistheoremname}
\newenvironment{namedthm}[1]
	{\renewcommand{\thistheoremname}{#1}%
	\begin{genericthm}}
	{\end{genericthm}}
\end{comment}

% https://tex.stackexchange.com/questions/12913/customizing-theorem-name

\newtheorem*{genericthm*}{\thistheoremname}
\newenvironment{namedthm*}[1]
	{\renewcommand{\thistheoremname}{#1}%
	\begin{genericthm*}}
	{\end{genericthm*}}


%\newmdtheoremenv{theo}{Theorem}

% https://tex.stackexchange.com/questions/43536/writing-steps-in-an-equation


\usepackage[T1]{fontenc}
\usepackage{geometry}
\usepackage{titlesec}
\usepackage{titletoc}
\usepackage{blindtext}  % drop in actual document

\titleformat{name=\chapter}[display]
{\normalfont\huge\bfseries}
{\chaptertitlename\ \thechapter}
{20pt}
{\Huge}
[\normalsize\normalfont\vspace*{1pc}%
\hbox{\large\bfseries\contentsname}\vspace{6pt}\titlerule\vspace{3pt}
\startcontents
\printcontents{l}{1}{\setcounter{tocdepth}{2}}\vspace{1pt}
\titlerule\vspace{1pc}]

\titleformat{name=\chapter,numberless}[display]
{\normalfont\huge\bfseries}
{}
{20pt}
{\Huge}

\usepackage{url}
\usepackage{hyperref}
\usepackage{geometry}
\geometry{a4paper}
\usepackage[english]{babel}
\title{Real Analysis Royden - Fourth Edition\\
	\large Notes + Solved Exercises :) \\
	\large \href{https://latex-programming.fandom.com/wiki/List_of_LaTeX_symbols}{Latex Symbols}
}
\author{J.B.}
\date{\small May 2024}

\begin{document}

\pagenumbering{roman}
\maketitle
\tableofcontents

\clearpage\pagenumbering{arabic}
%\pagenumbering{arabic}

\setcounter{chapter}{0}
\chapter*{I LEBESGUE INTEGRATION FOR FUNCTIONS OF A SINGLE REAL VARIABLE}
\addcontentsline{toc}{chapter}{I LEBESGUE INTEGRATION FOR FUNCTIONS OF A SINGLE REAL VARIABLE}
\setcounter{chapter}{0}

\setcounter{chapter}{0}
\chapter*{Preliminaries on Sets, Mappings, and Relations}
\addcontentsline{toc}{chapter}{Preliminaries on Sets, Mappings, and Relations}
\setcounter{chapter}{0}

\begin{flushleft}

\begin{namedthm*}{Definition}
A relation $R$ on a set $X$ is called an \textbf{equivalence relation} provided:
\begin{enumerate}[label=(\roman*),align=left]
	\item $xRx$ for all $x \in X$ (reflexive),
	\item $xRy$ implies $yRx$ for all $x,y \in X$ (symmetric),
	\item $xRy$ and $yRz$ imply $xRz$ for all $x,y,z \in X$ (transitive).
\end{enumerate}
\end{namedthm*}
\bigskip

\begin{namedthm*}{Partial Ordering on a set $X$}
A relation $R$ on a nonempty set $X$ is called a \textbf{partial ordering} provided:
\begin{enumerate}[label=(\roman*),align=left]
	\item $xRx$ for all $x \in X$ (reflexive),
	\item $xRy$ and $yRx$ imply $x=y$ for all $x,y \in X$ (antisymmetric),
	\item $xRy$ and $yRz$ imply $xRz$ for all $x,y,z \in X$ (transitive).
\end{enumerate}
A subset $E$ of $X$ is \textbf{totally ordered} provided either $xRy$ or $yRx$ for all $x,y \in E$.
A member $x$ of $X$ is said to be an \textbf{upper bound} for a subset $E$ of $X$ provided that 
\[ yRx \text{ for all } y \in E.\]
A member $x$ of $X$ is said to be \textbf{maximal} provided that 
\[ xRy \text{ implies that } y =x \text{ for } y \in X.\]
\end{namedthm*}
\medskip

\begin{namedthm*}{Strict Partial Ordering on a set $X$}
A relation $R$ on a nonempty set $X$ is called a \textbf{strict partial ordering} provided:
\begin{enumerate}[label=(\roman*),align=left]
	\item not $xRx$ for all $x \in X$ (irreflexive),
	\item $xRy$ implies not $yRx$ for all $x,y \in X$ (asymmetric),
	\item $xRy$ and $yRz$ imply $xRz$ for all $x,y,z \in X$ (transitive).
\end{enumerate}
A subset $E$ of $X$ is \textbf{strictly totally ordered} provided either $xRy$ or $yRx$ if $x\neq y$ for all $x,y \in E$.\par
\end{namedthm*}


\begin{namedthm*}{Zorn's Lemma}
	Let $X$ be a partially ordered set for which every totally ordered subset has an upper bound. Then $X$ has a maximal member.
\end{namedthm*}

\begin{namedthm*}{Every vector space has a basis}
\end{namedthm*}
\begin{proof}
Let V be any vector space, and let L be the collection of all linearly independent subsets of V. 
L is nonempty as the singleton sets are linearly independent. 
Define a partial order on L in the form $C \subseteq C'$ for $C,C' \in L$.
For any chain (a totally ordered subset of a partially ordered set) $\mathcal{C}$ of $L$, where $\mathcal{C}$ consists of the sets $C_1 \subseteq C_2 \subseteq \cdots$, we can construct a linearly independent set $C' = \bigcup_{C \in \mathcal{C}} C$ that is an upper bound of $\mathcal{C}$.
By Zorn's Lemma, L has a maximal element, say M.
This collection $M$ is a basis for $V$. To show this, suppose by contradiction that there exists a vector $v \in V$ s.t. $v \notin \text{ Span}\{M\}$.
Then $v \cup M$ is linearly independent and $M \subseteq v \cup M$, a contradiction to the fact that $M$ is maximal.
\end{proof}

\end{flushleft}

% Chapter 1
\chapter{The Real Numbers: Sets, Sequences, and Functions}

\section{The Field, Positivity, and Completeness Axioms}

\begin{flushleft}

\textbf{The field axioms}\par
Consider $a,b,c \in \mathbb{R}$:
\begin{enumerate}
	\item Closure of Addition: $a+b \in \mathbb{R}$.
	\item Associativity of Addition: $(a+b)+c = a+(b+c)$.
	\item Additive Identity: $0+a=a+0=a$.
	\item Additive Inverse: $(-a)+a=a+(-a)=0$.
	\item Commutativity of Addition: $a+b=b+a$.
	\item Closure of Multiplication: $ab \in \mathbb{R}$.
	\item Associativity of Multiplication: $(ab)c = a(bc)$.
	\item Distributive Property: $ a(b+c)=ab+ac$.
	\item Commutativity of Multiplication: $ab=ba$.
	\item Multiplicative Identity: $1a=a1=a$.
	\item No Zero Divisors: $ab=0 \implies a=0 \text{ or } b=0$.
	\item Multiplicative Inverse: $a^{-1}a=aa^{-1}=1$.
	\item Nontriviality: $1 \neq 0$.
\end{enumerate}
\medskip

\textbf{The positivity axioms}\par
The set of \textbf{positive numbers}, $\mathcal{P}$, has the following two properties:
\begin{itemize}
    \item [P1] If $a$ and $b$ are positive, then $ab$ and $a+b$ are both positive.
    \item [P2] For a real number $a$, exactly one of the three is true: $a$ is positive, $-a$ is positive, $a=0$.	
\end{itemize}

We call a nonempty set $I$ of real numbers an \textbf{interval} provided for any two points in $I$, all the points that lie between these two points also lie in $I$.
That is, $\forall x,y \in I, \lambda x + (1-\lambda)y \in I \text{ for } \lambda \in [0,1]$.
\medskip

\textbf{The completeness axiom}\par
A nonempty set $E$ of real numbers is said to be \textbf{bounded above} provided there is a real number $b$ such that $x \le b$ for all $x\in E$: the number $b$ is called an \textbf{upper bound} for $E$.
We can similarly define a set being \textbf{bounded below} and having a \textbf{lower bound}. A set that is bounded above need not have a largest member.
\begin{namedthm*}{The Completeness Axiom}
Let $E$ be a nonempty set of real numbers that is bounded above. The among the set of upper bounds for $E$ there is a smallest, or least, upper bound.
(This least upper bound is called the \textbf{supremum} of $E$. Also, it can be shown that any nonempty set $E$ that is bounded below has a greatest lower bound, called the \textbf{infimum} of $E$).	
\end{namedthm*}

\medskip
\textbf{The extended real numbers}\par
The extended real numbers: $\mathbb{R} \cup \{-\infty,\infty\}$\par
Every set of real numbers has a supremum and infimum that belongs to the extended real numbers.

\end{flushleft}

\begin{center}
	\textbf{PROBLEMS}
\end{center}
\begin{enumerate}
	\setcounter{enumi}{0}
	\item For $a\neq 0$ and $a\neq 0$, show that $(ab)^{-1} = a^{-1}b^{-1}$.\par

		\begin{align*}
			(ab)(ab)^{-1} & = 1 && \tag*{by multiplicative inverse}\\
			a(b(ab)^{-1}) & = 1 && \tag*{by associativity of multiplication} \\
			a^{-1}a(b(ab)^{-1}) & = a^{-1}1 \\
			1(b(ab)^{-1}) & = a^{-1}1 && \tag*{by multiplicative inverse} \\
			b(ab)^{-1} & = a^{-1} && \tag*{by multiplicative identity} \\
			b^{-1}b(ab)^{-1} & = b^{-1}a^{-1} \\
			1(ab)^{-1} & = b^{-1}a^{-1} && \tag*{by multiplicative inverse} \\
			(ab)^{-1} & = b^{-1}a^{-1} && \tag*{by multiplicative identity} \\
			(ab)^{-1} & = a^{-1}b^{-1} && \tag*{by commutativity of multiplication} \\
		\end{align*}

	\item Verify the following:
	\begin{enumerate}[label=(\roman*),align=left]
        \item For each real number $a\neq 0$, $a^2>0$. In particular, $1>0$ since $1 \neq 0$ and $1=1^2$.\par
        By positivity axiom P2, since $a\neq 0$, either $a$ is positive or $-a$ is positive.\par
		In the case $a$ is positive, $a^2$ is positive by positivity axiom P1.\par
		In the case $-a$ is positive, $(-a)(-a)$ is positive by P1.
		\begin{align*}
			(-a)(-a) & = (-a)(-a) + a(0) && \tag*{by additive identity}\\
			& = (-a)(-a) + a(-a+a) && \tag*{by additive inverse}\\
			& = (-a)(-a) + a(-a) + a(a) && \tag*{by distributive property} \\
			& = (-a + a)(-a) + a^2 && \tag*{by distributive property} \\
			& = 0(-a) + a^2 &&\tag*{by additive inverse} \\
			& = a^2 &&\tag*{by additive identity}
		\end{align*}
		Therefore $a^2$ is positive by equality.
        \item For each positive number $a$, its multiplicative inverse  $a^{-1}$ also is positive.\par
        The multiplication of two positive numbers is positive by positivity axiom P1.\par
		The multiplication of two non-positive numbers is positive: by reformulating the previous result from (i), we can see $0 < (-a)(-b) = ab$ for $a,b \neq 0$. \par
		The multiplication of a positive number and a non-positive number is not positive. 
		To see this, suppose $a$ is positive and $b$ is not positive, but $ab$ is positive. By P1 and P2, $a(-b)$ is also positive.
		By P1, $ab + a(-b)$ is positive. However,
		\[
		ab + a(-b) = a(b-b) = a(0) = 0.
		\]
		This is a contradiction to P2. Therefore $ab$ is not positive.
		\par
		The result from (i) shows that $1$ is positive. By multiplicative inverse, $aa^{-1} = 1 > 0$. Therefore $a^{-1}$ must be positive because $a$ is positive.
        \item If $a>b$, then \[ ac >bc \text{ if } c>0 \text{ and } ac < bc \text{ if } c<0. \]
        Proof that $a(-1)=-a$ for a real number $a$:
		\[a+(-1)a = 1a+(-1)a = (1+-1)a = 0a = 0.\]		
		$a>b$ implies that $a-b$ is positive.\par
		If $c$ is positive, then $(a-b)c = ac-bc$ is positive, and $ac>bc$.\par
		If $c$ is not positive, then $(a-b)c = ac-bc$ is not positive, and $-(ac-bc) = bc-ac$ is positive, so $bc>ac$.\par
    \end{enumerate}
	\item For a nonempty set of real numbers $E$, show that $\inf E = \sup E$ iff $E$ consists of a single point.\par
	$(\implies)$ Suppose $\inf E = \sup E$.\par
	Then $\inf E \le x \le \sup E$ for all $x\in E$. But this implies $x = \inf E = \sup E$ for all $x\in E$, so $E$ consists of the single point $x$.\par
	$(\impliedby)$ Suppose $E={x}$ is a singleton set.\par
	Clearly $x$ is an upper bound and a lower bound for $E$, as $x\le x$. 
	By completeness of the reals, there exists $\sup E$ and $\inf E$ s.t. $x \le \inf E \le x \le \sup E \le x$, as $\inf E$ is the greatest lower bound, and $\sup E$ is the least upper bound.
	Therefore $\inf E = \sup E$.
	\item Let $a$ and $b$ be real numbers.
	\begin{enumerate}[label=(\roman*),align=left]
        \item Show that if $ab = 0$, then $a=0$ or $b=0$.\par
        Contrapositive: Let $a\neq0$ and $b\neq0$. In 2.(ii), it was shown that the multiplication of two nonzero numbers is either positive or not positive. Therefore $ab\neq 0$.
        \item Verify that $a^2 -b^2 = (a-b)(a+b)$ and conclude from part (i) that if $a^2 = b^2$, then $a=b$ or $a=-b$.\par
        \begin{align*}
			(a-b)(a+b) & = (a-b)(a) + (a-b)(b) && \tag*{by distributive property}\\
			& = (a)(a)+(-b)(a) + (a)(b)+(-b)(b) && \tag*{by distributive property}\\
			& = (a)(a)+(-b+b)(a) +(-b)(b) && \tag*{by distributive property}\\
			& = (a)(a) +(-b)(b) && \tag*{by additive inverse}\\
			& = a^2 - b^2 
		\end{align*}
		Suppose $a^2=b^2$. Then $(a-b)(a+b)=a^2-b^2=0$, and by (i), $(a-b)=0 \implies a=b$ or $(a+b)=0 \implies a=-b$.
        \item Let $c$ be a positive real number. Define $E = \{ x \in \mathbb{R} \ |\  x^2 < c\}$. Verify that $E$ is nonempty and bounded above.
		Define $x_0 = \sup E$. Show that $x_0^2 = c$. Use part (ii) to show that there is a unique $x>0$ for which $x^2=c$. It is denoted by $\sqrt{c}$.\par
		We can consider $0\in \mathbb{R}$. $0^2=0<c$, so $0\in E$ and $E$ is nonempty. Also, $c+1$ is a real number and an upper bound for $E$; thus by the completeness axiom, $E$ has a supremum, say $x_0$.
		We can see that for any upper bound $b$ of $E$, $x \le x_0 \le b$ for all $x \in E$. Then $x^2 \le x_0^2 \le b^2$ implies $x_0^2=c$, else $x_0$ is not the supremum. \par
		Suppose there exists $x_1,x_2 > 0$ such that $x_1^2 = c$ and $x_2^2 = c$. This implies $x_1^2 = x_2^2$, and by part (ii), $x_1 = x_2$ or $x_1 = -x_2$. Because $x_1,x_2$ are positive, $x_1 = x_2$.
    \end{enumerate}
	\item Let $a,b,c$ be real numbers s.t. $a\neq 0$ and consider the quadratic equation \[ ax^2+bx+c=0, x \in \mathbb{R}.\]
	\begin{enumerate}[label=(\roman*),align=left]
        \item Suppose $b^2 - 4ac >0$. Use the Field Axioms and the preceding problem to complete the square and thereby show that this equation has exactly two solutions given by
		\[  x = \dfrac{-b + \sqrt{b^2-4ac}}{2a} \ \text{and}\  x = \dfrac{-b - \sqrt{b^2-4ac}}{2a}. \]
		\begin{align*}
			ax^2+bx+c & = 0 \\
			4a(ax^2+bx+c) & = 4a(0) \\
			4a^2x^2+4abx+4ac & = 0 && \tag*{by distributive property}\\
			4a^2x^2+4abx+4ac+b^2-b^2 & = 0 && \tag*{by additive inverse}\\
			4a^2x^2+4abx+b^2 & = b^2-4ac \\
			(2ax+b)^2 & = b^2-4ac 
		\end{align*}
		By 4(iii), because $b^2 - 4ac >0$, there is a unique $y>0$ for which $y^2 = b^2-4ac$. It is denoted by $y = \sqrt{b^2-4ac}$.\par
		By 4(ii), $(2ax+b)^2 = b^2-4ac = y^2$ implies $(2ax+b) = \sqrt{b^2-4ac} = y$ or $(2ax+b) = -\sqrt{b^2-4ac} = -y.$\par
		\begin{align*}
			2ax+b & = \pm \sqrt{b^2-4ac} \\
			2ax & = -b \pm \sqrt{b^2-4ac} \\
			x & = \dfrac{-b \pm \sqrt{b^2-4ac}}{2a}.
		\end{align*}
	\end{enumerate}
	\item Use the Completeness Axiom to show that every nonempty set of real numbers that is bounded below has an infimum and that
	\[\inf E =-\sup \{-x \ |\ x \in E\}.\]
	Let $E$ be a set that is bounded below; that is, there exists $l\in \mathbb{R}$ such that $l \le x$ for all $x\in E$.
	Then $-l \ge -x$ for all $x \in E$, and $-l$ is an upper bound for $-E=\{-x \ | \ x\in E\}$. 
	Therefore the set $-E$ is bounded above, and by the completeness axiom, there exists a least upper bound $c= \sup (-E)$.
	Then for any upper bound $u$ of $-E$, $u \ge c \ge -x$ for all $x \in E.$
	Then $-u$ is a lower bound of $E$, and $-u \le c \le x$ for all $x \in E$, and $c$ is the greatest lower bound and thus infimum of $E$.
	\item For real numbers $a$ and $b$, verify the following:
	\begin{enumerate}[label=(\roman*),align=left]
		\item $|ab| = |a||b|.$\par
		We have 
		\[ 
		|ab| =
		\begin{cases} 
			ab & \text{ if } ab \ge 0, \\
			-(ab) & \text{ if } ab < 0.
		\end{cases}
		\]
		The case where either $a$ or $b$ are zero is trivial.
		In problem 2(ii), it was shown that $ab>0$ if $a,b$ are the same sign, and $ab<0$ if $a,b$ are opposite signs.\par
		Case $a,b>0$: Then $ab>0$ so $|ab| = ab$, and $|a| = a$ and $|b|=b$ so $|a||b| = ab$.\par
		Case $a,b<0$: Then $ab>0$ so $|ab| = ab$, and $|a| = -a$ and $|b|=-b$ so $|a||b| = (-a)(-b)=ab$.\par
		Case $a<0,b>0$: Then $ab<0$ so $|ab| = -(ab) = (-1)ab$, and $|a| = -a = (-1)a$ and $|b|=b$ so $|a||b| = (-1)ab$.
		\item $|a+b| \le |a|+|b|.$\par
		The case where both $a,b=0$ is trivial.\par
		Case $a,b>0$: Then $a+b > 0 $, so $|a+b| = a+b$ and $|a|+|b| = a+ b$.\par
		Case $a>0,b=0$: Then $a+b = a+0=a > 0 $, so $|a+b| = a$ and $|a|+|b| = a+ 0 = a$.\par
		Case $a<0,b=0$: Then $a+b = a +0=a<0 $, so $|a+b| = -a$ and $|a|+|b| = -a +0 = -a$.\par
		Case $a,b<0$: Then $a+b < 0 $, so $|a+b| = -(a+b) = -a-b$ and $|a|+|b| = -a- b$.\par
		That is, equality holds except for the case where $a,b$ are nonzero opposite signs:\par
		Case $a>0,b<0$: $|a+b| \in \{a+b, -(a+b)\}$.\par
		$b<0<-b \implies a+b<a<a-b$, and $-a<0<a \implies -(a+b)=-a-b<-b<a-b$.\par
		$|a|+|b| = a- b$, so $|a+b| < |a|+|b|$.
		\item For $\epsilon >0,$
		\[ |x-a| < \epsilon \text{  iff  } a - \epsilon < x < a + \epsilon.\]
		We have
		\[ 
		|x-a| =
		\begin{cases} 
			x-a & \text{ if } x-a \ge 0, \\
			-(x-a) & \text{ if } x-a < 0.
		\end{cases}
		\]
		$(\implies)$ Suppose $|x-a| < \epsilon$.\par
		Then $x-a < \epsilon$ and $a-x < \epsilon$.\par
		Then $x< a+\epsilon$ and $a-\epsilon<x$.\par
		$(\impliedby)$ Suppose $a - \epsilon < x < a + \epsilon$.\par
		Then
		\begin{align*}
		a - \epsilon-a &< x-a < a + \epsilon-a \\
		- \epsilon &< x-a < \epsilon
		\end{align*}
		So $x-a < \epsilon$ and $- \epsilon < x-a \implies -(x-a)< \epsilon$, so $|x-a| < \epsilon$.
	\end{enumerate}
\end{enumerate}

\section{The Natural and Rational Numbers}
\begin{flushleft}

\begin{namedthm*}{Definition}
	A set $E$ of real numbers is said to be \textbf{inductive} provided it contains 1 and if the number $x$ belongs to $E$, the number $x+1$ also belongs to $E$.
\end{namedthm*}

The set of \textbf{natural numbers}, denoted by $\mathbb{N}$, is defined to be the intersection of all inductive subsets of $\mathbb{R}$. 

\begin{namedthm*}{Theorem 1}
	Every nonempty set of natural numbers has a smallest member.
\end{namedthm*}
\begin{proof}
	Let $E$ be a nonempty set of natural numbers. Since the set $\{x\in \mathbb{R}\ |\ x \ge 1 \}$ is an inductive set, by definition of intersection, $\mathbb{N} \subseteq \{x\in \mathbb{R}\ |\ x \ge 1 \}$, and the natural numbers are bounded below by $1$.
	Therefore $E$ is bounded below by $1$. By the Completeness Axiom, $E$ has an infimum; let $c=\inf E$.
	Since $c+1$ is not a lower bound for $E$, there is an $m \in E$ for which $m < c+1$.
	We claim that $m$ is the smallest member of $E$. Otherwise, there is an $n\in E$ for which $n<m$. Since $n \in E$, $c\le n$. Thus $c \le n < m < c+1$ and therefore $m-n<1$.
	Therefore the natural number $m$ belongs to the interval $(n,n+1)$. However, an induction argument shows that $(n, n+1) \cap \mathbb{N} = \emptyset$ (Problem 8). 
	This is a contradiction to $m \in E$. Therefore $m$ is the smallest member of $E$.
\end{proof}

\begin{namedthm*}{Archimedean Property}
	For each pair of positive real numbers $a$ and $b$, there is a natural number $n$ for which $na>b$. This can be reformulated: for each positive real number $\epsilon$, there is a natural number $n$ for which $\dfrac{1}{n} < \epsilon$.
\end{namedthm*}

The set of \textbf{integers}, denoted $\mathbb{Z}$, is defined to be the set of numbers consisting of the natural numbers, their negatives, and zero.
\par
\medskip
Consider the number $2$. From problem 4(iii), there is a unique $x>0$ for which $x^2=2$. It is denoted by $\sqrt{2}$. This number is not rational.
Suppose that $x$ is rational: then there exist $p,q \in \mathbb{Z}$ such that $(\dfrac{p}{q})^2=2$. Then $p^2=2q^2$. 
By the unique prime factorizations of $p$ and $q$, $p^2$ is divisible by $2^{2k}$ for some $k \in \mathbb{Z}_{\ge 0}$, while $2q^2$ is divisible by $2 \cdot 2^{2j} = 2^{2j+1}$ for some $j \in \mathbb{Z}_{\ge 0}$.
$2^{2k} \neq 2^{2j+1}$ for any combinations of $k,j$ so $p^2=2q^2$ is not possible, and $\sqrt{2}$ is not rational.

\begin{namedthm*}{Definition}
A set $E$ of real numbers is said to be \textbf{dense} in $\mathbb{R}$ provided that between any two real numbers there lies a member of $E$.	
\end{namedthm*}

\begin{namedthm*}{Theorem 2}
The rational numbers are dense in $\mathbb{R}$.	
\end{namedthm*}
\begin{proof}
Let $a,b \in \mathbb{R}$ with $a<b$.\par
Case $a>0$:\par
By the Archimedean Property, for $(b-a)>0$, there exists $q \in \mathbb{N}$ for which $\dfrac{1}{q} < b-a$. \par
Again by the Archimedean Property, for $b,\dfrac{1}{q}>0$, there exists $n \in \mathbb{N}$ for which $n(\dfrac{1}{q})>b$.\par
Therefore the set $S=\{n \in \mathbb{N} \ |\ \dfrac{n}{q} \ge b \}$ is nonempty. Because $S$ is a set of natural numbers, by Theorem 1, $S$ has a smallest member $p$.
Noticing $\dfrac{1}{q} < b-a < b$, we see that $1 \notin S$ and $p>1$. Therefore $p-1$ is a natural number (Problem 9).
Because $p$ is the smallest member of $S$, $p-1 \notin S$ and $\dfrac{(p-1)}{q} < b$.
Also, 
\[
a = b-(b-a) < \dfrac{p}{q} - (\dfrac{1}{q}) = \dfrac{(p-1)}{q}.
\]
Therefore the rational number $\dfrac{(p-1)}{q}$ lies between $a$ and $b$.\par
Case $a<0$:\par
By the Archimedean Property, for $1,-a>0$, there exists $n \in \mathbb{N}$ for which $n(1) > -a$, which implies $n+a>0$, and $b>a$ implies $n+b>n+a>0$.
Then we can use the first case to show that there exists a rational number $r$ such that $n+a<r<n+b$. Therefore the rational number $r-n$ lies between $a$ and $b$.
\end{proof}

\end{flushleft}

\begin{center}
	\textbf{PROBLEMS}
\end{center}
\begin{enumerate}
	\setcounter{enumi}{7}
	\item Use an induction argument to show that for each natural number $n$, the interval $(n, n+1)$ fails to contain any natural number.\par
	For $n \in \mathbb{N}$, let $P(n)$ be the assertion that $(n, n+1) \cap \mathbb{N} = \emptyset$.\par
	$P(1)$: $(1, 2)=\{x\ |\ 1<x<2\}$. Suppose there exists a natural number $q \in (1, 2)$. Then $q>1$ and by problem 9, $q-1$ is a natural number. 
	However, $1<q<2 \implies 0<q-1<1$, which is a contradiction to the fact that the natural numbers are bounded below by $1$ (Theorem 1). Therefore there are no natural numbers in $(1, 2)$.\par
	Suppose $P(k)$ is true for some natural number $k$.\par
	$P(k+1)$: Suppose there exists a natural number $p \in (k+1, (k+1)+1)$; that is, $k+1<p<k+2$.\par
	Case $p=1$: then $k+1<1<k+2$. but $k \in \mathbb{N}$ so $k+1 >1$. Thus $p=1$ is not possible.\par
	Case $p>1$: then by problem 9, $p-1 \in \mathbb{N}$, so $k+1<p<k+2 \implies k<p-1<k+1$. This is a contradiction to $P(k)$, the assumption that there are no natural numbers between $(k,k+1)$.
	Therefore $P(k+1)$ is true.
	\item Use an induction argument to show that if $n>1$ is a natural number, then $n-1$ also is a natural number. The use another induction argument to show that if $m$ and $n$ are natural numbers with $n>m$, then $n-m$ is a natural number.\par
	For $n \in \mathbb{N}$, let $P(n)$ be the assertion that $n=1$ or $n-1 \in \mathbb{N}$.\par
	$P(1)$: $1=1$, true.\par
	Suppose $P(k)$ is true for some $k \in \mathbb{N}$.\par
	$P(k+1)$: $(k+1)-1 = k \in \mathbb{N}$, true.\par
	\medskip
	For $n \in \mathbb{N}$, let $Q(n)$ be the assertion that for all $m \in \mathbb{N}$ such that $n>m$, then $n-m \in \mathbb{N}$.\par
	$Q(1)$: true trivially, because there are no natural numbers less than $1$.\par
	Suppose $Q(k)$ is true for some $k \in \mathbb{N}$; that is, for all $m \in \mathbb{N}$ such that $k>m$, then $k-m \in \mathbb{N}.$\par
	$Q(k+1)$: For all the $m$ from $Q(k)$, we have $(k+1)>k>m$.\par
	We want to show that $(k+1)-m \in \mathbb{N}$.\par
	This is clearly true for $m=1$ because $(k+1)-1 = k \in \mathbb{N}$. 
	Otherwise, $m>1$, so by $P(m)$, $m-1 \in \mathbb{N}$ and $(k+1)-m = k -(m-1)$.
	$Q(k)$ is true tells us that because $(m-1) \in \mathbb{N}$ and $k>m>m-1$, then $k-(m-1) \in \mathbb{N}$.
	Therefore $Q(k+1)$ is true.
	\item Show that for any real number $r$, there is exactly one integer in the interval $[r,r+1)$.\par
	This is trivial if $r \in \mathbb{Z}$.\par
	Consider the smallest integer $p$ less than $[r,r+1)$.
	Then $p<r<r+1$ (and $r<p+1$, because $r=p+1 \implies r \in \mathbb{Z}$ and $r>p+1 \implies p$ is not the smallest integer less than $[r,r+1)$), therefore $r<p+1<r+1$. Because the integers are inductive, $p+1 \in \mathbb{Z}$.\par
	To show that there is not more than one integer between $[r,r+1)$:
	let $q$ be a natural number such that $r \le q < r+1$. Then $q-1 < r \le q$ and $q< r+1 \le q+1$. 
	From problem 8, we see that there are no integers between $(q-1,q)$ and $(q,q+1)$, 
	so there is only one integer in $(q-1,q)\cup q \cup (q,q+1) \supseteq [r,r+1)$.
	\item Show that any nonempty set of integers that is bounded above has a largest member.\par
	Let $E$ be a nonempty set of integers that is bounded above. By the completeness axiom, there exists $c = \sup E$. 
	That is, $x\le c$ for all $x \in E$. Then $c-1 < z \le c$ for some $z \in E$ because $c-1$ is not an upper bound of $E$.
	Suppose $c$ is not in $E$. Then $c-1 < z < c$. 
	This implies that $c-1 < z < w \le c$ for some $w \in E$ because $z$ is not an upper bound of E.
	But then there exists two integers in the interval $(c-1,c]$, which is a contradiction to problem 10.
	Therefore $c$ is an element of $E$, and it is the largest member.
	\item Show that the irrational numbers are dense in $\mathbb{R}$.\par
	Choose any two real numbers $a,b$ and any irrational number $z$. Then $\dfrac{a}{z},\dfrac{b}{z}$ are real numbers. 
	By density of the rationals in $\mathbb{R}$, there exists a rational $r$ such that $\dfrac{a}{z}<r<\dfrac{b}{z}$. This implies $a<rz<b$, where $rz$ is an irrational number.\par
	Proof that $rz$ is irrational:\par
	Let $r = \dfrac{p}{q}$ and suppose that $rz$ is rational; then $rz = \dfrac{m}{n}$.
	\begin{align*}
		\dfrac{p}{q}z &= \dfrac{m}{n}\\
		z &=\dfrac{m}{n} \dfrac{q}{p}\\
		z &= \dfrac{mq}{np}
	\end{align*}
	Then $z$ is rational, a contradiction.
	\item Show that each real number is the supremum of a set of rational numbers and also the supremum of a set of irrational numbers.\par
	Choose any real number $a$. Let $S=\{ r \in \mathbb{Q}\ |\ r\le a\}$.
	Then $a$ is an upper bound for this set. To show that $a$ is the supremum, suppose by contradiction that it is not.
	Then there exists $c \in \mathbb{R}$ such that $r \le c < a$. 
	However, the rational numbers are dense in $\mathbb{R}$, so there exists a rational between $c$ and $a$, a contradiction to the assumption that $c$ is an upper bound to $S$. 
	\par
	The same argument can easily be shown for the irrational numbers.
	\item Show that if $r>0$, then, for each natural number $n$, $(1+r)^n \ge 1+n \cdot r$.\par
	Let $r>0$.\par
	For $n \in \mathbb{N}$, let $P(n)$ be the assertion that $(1+r)^n \ge 1+n \cdot r$.\par
	$P(1)$: $(1+r)^1 = 1+1 \cdot r$, true.\par
	Suppose $P(k)$ is true for some $k \in \mathbb{N}$. Then $(1+r)^k \ge 1+k \cdot r$. \par
	$P(k+1)$:\par
	$(1+r)^{k+1} = (1+r)^k(1+r) \ge (1+kr)(1+r) = 1+ kr + r +kr^2 > 1+ kr + r = 1+(k+1) \cdot r$.
	\item Use induction arguments to prove that for every natural number $n$,
	\begin{enumerate}[label=(\roman*),align=left]
        \item \[ \sum_{j=1}^n j^2 = \dfrac{n(n+1)(2n+1)}{6}, \]
        $P(1)$: $\sum_{j=1}^1 j^2 = 1 = \dfrac{1(1+1)(2+1)}{6}$.\par
		Suppose $P(k)$ is true for $k \in \mathbb{N}$.\par
		$P(k+1)$: 
		\begin{align*}
			\sum_{j=1}^{k+1} j^2 &= \sum_{j=1}^{k} j^2 + (k+1)^2 \\
			&= \dfrac{k(k+1)(2k+1)}{6} + (k+1)^2\\
			& = \dfrac{k(2k^2+k+2k+1)}{6} + \dfrac{6(k^2+2k+1)}{6} \\
			& = \dfrac{(2k^3+k^2+2k^2+k)+(6k^2+12k+6)}{6} \\
			&= \dfrac{2k^3+9k^2+13k+6}{6} \\
			&= \dfrac{(k+1)(2k^2+7k+6)}{6} \\
			&= \dfrac{(k+1)(k+2)(2(k+1)+1)}{6}.
		\end{align*}
        \item \[ 1^3 + 2^3 + \cdots + n^3 = (1+2+\cdots +n)^2, \]
        $P(1)$: $a^3 = 1 = (1)^3$.\par
		Suppose $P(k)$ is true for $k \in \mathbb{N}$.\par
		$P(k+1)$: 
		\begin{align*}
			1^3 + 2^3 + \cdots + (k+1)^3 &= 1^3 + 2^3 + \cdots + k^3 + (k+1)^3 \\
			&= (1+2+\cdots +k)^2 + (k+1)^3 \\
			&= \biggl (\dfrac{k(k+1)}{2} \biggr)^2 + (k+1)^3 \\
			&= \dfrac{k^2(k+1)^2}{4} + \dfrac{4(k+1)^3}{4} \\
			&= \dfrac{k^2(k+1)^2+(4k+4)(k+1)^2}{4} \\
			&= \dfrac{(k^2+4k+4)(k+1)^2}{4} \\
			&= \dfrac{(k+2)^2(k+1)^2}{2^2} \\
			&= \biggl (\dfrac{(k+2)(k+1)}{2} \biggr )^2 \\
			&= \biggl (\dfrac{((k+1)+1)(k+1)}{2} \biggr )^2 \\
			&= \biggl (1+2+ \cdots + (k+1) \biggr )^2.
		\end{align*}
        \item \[ 1+r+\cdots +r^n = \dfrac{1-r^{n+1}}{1-r} \text{ if } r \neq 1.\]
        $P(1)$: $1+r^1 = \dfrac{(1+r)(1-r)}{1-r} = \dfrac{1-r^2}{1-r}$.\par
		Suppose $P(k)$ is true for $k \in \mathbb{N}$.\par
		$P(k+1)$: 
		\begin{align*}
			1+r+\cdots +r^{k+1} &= 1+r+\cdots + r^k +r^{k+1} \\
			&= \dfrac{1-r^{k+1}}{1-r} +r^{k+1} \\
			&= \dfrac{1-r^{k+1}}{1-r} +\dfrac{(1-r)r^{k+1}}{1-r} \\
			&= \dfrac{1-r^{k+1} + r^{k+1} - r^{(k+1)+1} }{1-r} \\
			&= \dfrac{1 - r^{(k+1)+1} }{1-r}.
		\end{align*}
    \end{enumerate}
\end{enumerate}

\section{Countable and Uncountable Sets}

\begin{flushleft}

Two sets $A$ and $B$ are \textbf{equipotent} provided there exists a bijection between them.\par
A set $E$ is \textbf{countable} if it is equipotent to a set of natural numbers.\par
For a countably infinite set $X$, we say that $\{x_n \ |\ n \in \mathbb{N} \}$ is an \textbf{enumeration} of $X$ provided
\[
	X = \{x_n \ |\ n \in \mathbb{N} \} \ \text{ and }\ x_n \neq x_m \ \text{ if }\ n \neq m. 
\]
\par
\medskip
\begin{namedthm*}{Theorem 3}
	A subset of a countable set is countable. In particular, every set of natural numbers is countable.
\end{namedthm*}
\begin{namedthm*}{Corollary 4}
	The following sets are countably infinite:
	\begin{enumerate}[label=(\roman*),align=left]
		\item For each natural numbers $n$, the Cartesian product $\mathbb{N}^n = \mathbb{N} \times \cdots \times \mathbb{N}$.
		\item The set of natural numbers $\mathbb{Q}$.
	\end{enumerate}
\end{namedthm*}
\par
\medskip
The rationals are countable: 
$\mathbb{Q} = \{0,\dfrac{1}{1},-\dfrac{1}{1},\dfrac{1}{2},-\dfrac{1}{2},\dfrac{2}{1},-\dfrac{2}{1}, \dfrac{3}{1}, -\dfrac{3}{1},\dfrac{1}{3},-\dfrac{1}{3},\dfrac{1}{4},-\dfrac{1}{4},\dfrac{2}{3},-\dfrac{2}{3},\cdots \}$.
\par
\medskip
\begin{namedthm*}{Corollary 6}
The union of a countable collection of countable sets is countable.
\end{namedthm*}
An interval of real numbers is called degenerate if it is empty or contains a single member.
\begin{namedthm*}{Theorem 7}
A nondegenerate interval of real numbers is uncountable.	
\end{namedthm*}
\begin{proof}
Let $I$ be a nondegenerate interval of real numbers. Clearly $I$ is not finite. Suppose $I$ is countably infinite.
Let $\{x_n \ |\ n \in \mathbb{N} \}$ be an enumeration of $I$. 
For each $n \in \mathbb{N}$, choose a nondegenerate compact subinterval $[a_n,b_n] \subseteq I$ such that $x_n \notin [a_n,b_n]$. 
Let the set of such intervals $\{[a_n,b_n]\}_{n=1}^\infty$ be descending: $[a_{n+1},b_{n+1}] \subseteq [a_n,b_n]$ (That is, $a_n \le a_{n+1}<b_{n+1}\le b_n$.)
Now, the nonempty set $E = \{a_n \ |\ n \in \mathbb{N} \}$ is bounded above by $b_1$.
Then the Completeness Axiom implies that $E$ has a supremum, say $x^* = \sup E$. 
Then for each $n$, $a_n \le x^* \le b_n$ because $x^*$ is the supremum of $E$ and each $b_n$ is an upper bound for $E$.
Therefore $x^*$ belongs to $[a_n,b_n]$ for each $n$.
But then $x^*$ is an element of $I$ and thus has an index $n_0 \in \mathbb{N}$ such that $x^* = x_{n_0}$. But $x^* \in [a_{n_0},b_{n_0}]$, a contradiction.
Therefore $I$ is not countable.
\end{proof}

\end{flushleft}

\begin{center}
	\textbf{PROBLEMS}
\end{center}
\begin{enumerate}
	\setcounter{enumi}{15}
	\item Show that the set $\mathbb{Z}$ of integers is countable.\par
	There exists a bijection $\phi: \mathbb{Z} \to \mathbb{N}$ with
	\[ 
		\phi(x) =
		\begin{cases} 
			2x & \text{ if } x > 0, \\
			-2x+1 & \text{ if } x \le 0.
		\end{cases}
	\]
	\begin{align*}
		\mathbb{Z} &= \{0,1,-1,2,-2,3,-3,4,-4, \cdots\} \\
		\mathbb{N} &= \{1,2,3,4,5,6,7,8,9, \cdots\}
	\end{align*}
	\item Show that a set $A$ is countable iff there is an injective mapping of $A$ to $\mathbb{N}$.\par
	$(\implies)$ Suppose $A$ is countable.\par
	Then either $A$ is equipotent to $\mathbb{N}$, or there is an $n \in \mathbb{N}$ such that $A$ is equipotent to $\{1,2, \cdots, n \}$.
	In the case $A$ is countably infinite, we have a bijection with $\mathbb{N}$ and thus an injection. In the case $A$ is finite, we have an injection with a subset of $\mathbb{N}$, and thus an injection with $\mathbb{N}$
	(injection: $f(a)=f(b) \implies a=b$ for $a,b \in A$).
	\par
	$(\impliedby)$ Suppose there is an injective mapping of $A$ to $\mathbb{N}$.\par
	Then there is a bijection from $A$ to some subset $B$ of $\mathbb{N}$.
	By Theorem 3, every subset of natural numbers is countable, and because $A$ is equipotent to this countable set $B$, then $A$ is countable.
	\item Use an induction argument to complete the proof of part (i) of Corollary 4.\par
	(Not an induction argument)\par
	Consider the function $f:\mathbb{N}^2 \to \mathbb{N}$, where $f(m,n) = 2^m3^n$. 
	By the Fundamental Theorem of Arithmetic, $2^m3^n = 2^{m'}3^{n'} \implies m=m',n=n'$.
	Then clearly $f$ is an injection. By problem 17, we see that $\mathbb{N}^2$ is countable.
	\par
	For any $k\in \mathbb{N}$ we can construct a function $f:\mathbb{N}^k \to \mathbb{N}$, where we have $n$ primes such that $f(m_1,m_2, \cdots, m_k) = p_1^{m_1}p_2^{m_2} \cdots p_k^{m_k}$.
	By the fundamental theorem of arithmetic, this is an injection and thus $\mathbb{N}^k$ is countable.
	\item Prove Corollary 6 in the case of a finite family of countable sets.\par
	Let $\{S_n\}_{n=1}^k$ be a finite family of countable sets.
	Then each set $S_n$ is countable, and we can enumerate as follows: $S_n = \{s_{nm} \ | \ m \in \mathbb{N} \}$.
	Then because there is only a finite number of countable sets, we can construct a function $f: \bigcup_{n=1}^k S_n \to \mathbb{N}$ seeing that 
	\[
	\bigcup_{n=1}^k S_n = \{s_{11},s_{21},s_{31},\cdots, s_{k1}, s_{12}, s_{22},s_{32},\cdots,s_{k2},s_{13}, \cdots \}.
	\]
	\item Let both $f:A \to B$ and $g:B \to C$ be injective and surjective. Show that the composition $g \circ f:A \to B$ and the inverse $f^{-1}:B \to A$ are also injective and surjective.\par
	$g \circ f$:\par
	By surjectivity of $g$, for all $c \in C$, there exists a $b \in B$ such that $g(b)=c$.
	Then by surjectivity of $f$, there exists an $a \in A$ such that $f(a)=b$.\par
	Therefore for any $c \in C$:
	\begin{align*}
		c & = g(b) && \text{ for some $b \in B$}\\
		& = g(f(a))&& \text{ for some $a \in A$}\\
		& =g \circ f (a)
	\end{align*}
	Therefore $g \circ f$ is surjective.\par
	By injectivity of $g$, $g(b)=g(b') \implies b = b'$.\par
	By injectivity of $f$, $f(a)=f(a') \implies a = a'$.
	\begin{align*}
		g \circ f (a) & = g \circ f (a')\\
		g(f(a)) & = g(f(a')) \\
		f(a) & = f(a')&& \text{ by injectivity of $g$}\\
		a & = a'&& \text{ by injectivity of $f$}
	\end{align*}
	Therefore $g \circ f$ is injective.
	\par
	$f^{-1}$:\par
	Because $f$ is a function from $A$ to $B$, $f(a) \subseteq B$ is defined for all $a \in A$.
	That is, for all $a \in A$, there exists a $b \in B$ such that $f^{-1}(b) = a$.
	Thus $f^{-1}$ is surjective.\par
	Because $f$ is a function, for each $a \in A$, $f(a)=b$ and $f(a)=b'$ imply $b=b'$. That is, $f^{-1}(b)=f^{-1}(b') \implies b=b'$.
	Thus $f^{-1}$ is injective.   
	\item Use an induction argument to establish the pigeonhole principle.\par
	For $n \in \mathbb{N}$, let $P(n)$ be the assertion that for any $m \in \mathbb{N}$, the set $\{1,2, \cdots, n\}$ is not equipotent to the set $\{1,2, \cdots, n+m\}$.\par
	$P(1)$: We have the sets $A=\{1\}$ and $B=\{1,2, \cdots, 1+m\}$, for $m \in \mathbb{N}$.
	In the case $m=1$, $B=\{1,1+1\}=\{1,2\}$, and clearly $A$ is not equipotent to $B$. Clearly $A$ is also not equipotent to $B$ for any other natural number $m>1$.\par
	Suppose $P(k)$ is true for some $k \in \mathbb{N}$. Then $\{1,2, \cdots, k\}$ is not equipotent to the set $\{1,2, \cdots, k+m\}$, for any $m \in \mathbb{N}$.\par
	$P(k+1)$: Then clearly $\{1,2, \cdots, k+1\}$ is not equipotent to the set $\{1,2, \cdots, (k+1), \cdots, (k+1)+m\}$, for any $m \in \mathbb{N}$.
	\item Show that $2^{\mathbb{N}}$, the collection of all sets of natural numbers, is uncountable.\par
	(Cantor's Theorem: for a set $A$, any function $f:A\to \mathcal{P}(A)$ is not surjective.)\par
	Let $f:\mathbb{N}\to \mathcal{P}(\mathbb{N})$ be any map. Let $E = \{n \in \mathbb{N}\ | \ n \notin f(n) \}$. 
	Then $E$ is a subset of the naturals that is not in the image of $f$, so $f$ is not surjective. 
	Therefore there is no bijection between $\mathbb{N}$ and  $\mathcal{P}(\mathbb{N})$.
	\item Show that the Cartesian product of a finite collection of countable sets is countable. Use the preceding theorem to show that $\mathbb{N}^{\mathbb{N}}$, the collection of all mappings of $\mathbb{N}$ into $\mathbb{N}$, is not countable.\par
	In problem 18, we showed that for any $k \in \mathbb{N}$, the set $\mathbb{N}^k = \mathbb{N} \times \mathbb{N} \times \cdots \times \mathbb{N}$ is countable. 
	It is then trivial to see that the Cartesian product of any finite collection of countable sets $S_1 \times S_2 \times \cdots \times S_k$ is countable.\par
	Notation:
	\[
		0=\emptyset, 1= \{0\}, 2=\{0,1\}, 3 = \{0,1,2\}, \cdots
	\]
	We can let $2^{\mathbb{N}}= \{0,1\}^{\mathbb{N}}$ be the set of functions $f:\mathbb{N} \to \{0,1\}$.\par
	Then, for any subset $A \subseteq \mathbb{N}$, there exists a function $f \in \{0,1\}^{\mathbb{N}}$ such that 
	\[
		f(x) =
	\begin{cases}
		1 & \text{if } x \in A,\\
		0 & \text{if } x \notin A,
	\end{cases}	
	\]
	and we have a bijection between the elements of $\{0,1\}^{\mathbb{N}}$ and the subsets of $\mathbb{N}$ ("Two sets that are equipotent are, from a set-theoretic point of view, indistinguishable").
	Therefore $2^{\mathbb{N}}= \{0,1\}^{\mathbb{N}}$ can be used to represent the collection of subsets of $\mathbb{N}$.\par
	Now, because the set of functions $f:\mathbb{N} \to \{0,1\}$ is uncountable, then clearly the set of functions $f:\mathbb{N} \to \mathbb{N} \supseteq \{0,1\}$ is uncountable (including zero in the naturals for notation convenience).
	\item Show that a nondegenerate interval of real numbers fails to be finite.\par
	Theorem 7 tells us that a nondegenerate interval of real numbers is uncountable, and thus, finite.	
	\item Show that any two nondegenerate intervals of real numbers are equipotent.\par
	We can prove this by showing that any interval is equipotent to the interval $(0,1)$.\par
	For any bounded interval $(a,b),(a,b],[a,b),[a,b]$, there exists a bijection to $(0,1),(0,1],[0,1),[0,1]$ respectively,
	of the form $f(x) = \dfrac{1}{b-a}(x-a)$.\par
	\item Is the set $\mathbb{R} \times \mathbb{R}$ equipotent to $\mathbb{R}$?\par
	yes (Schr\"oder-Bernstein theorem	)
\end{enumerate}

\section{Open Sets, Closed Sets, and Borel Sets of Real Numbers}

\begin{namedthm*}{The Heine-Borel Theorem}
Let $F$ be a closed and bounded set of real numbers. Then every open cover of $F$ has a finite subcover. 	
\end{namedthm*}
\begin{proof}
	Let $F$ be the closed, bounded interval $[a,b]$. Let $\mathcal{F}$ be an open cover of $[a,b]$. 
	Define $E$ to be the set of numbers $x \in [a.b]$ with the property that the interval $[a,x]$ can be covered by a finite number of the sets of $\mathcal{F}$.
	Since $a\in [a,b] \subseteq \mathcal{F}$ implies that $a$ is in one of the sets $\mathcal{O}' \subseteq \mathcal{F}$ by definition of union, $\mathcal{O}'$ is a finite subcover of $[a,a]=\{a\}$, and thus $a \in E$ and $E$ is nonempty.
	Since $E \subseteq [a,b] = \{x\ |\ a \le x \le b\}$, $E$ is bounded above by $b$, so by the completeness of $\mathbb{R}$, $E$ has a supremum $c = \sup E$.
	Because $c \le b$, clearly $c$ belongs to $[a,b]$, and this implies that there is an $\mathcal{O} \subseteq \mathcal{F}$ that contains $c$.
	Since $\mathcal{O}$ is open, there is an $\epsilon >0$ such that that the interval $(c- \epsilon, c+ \epsilon) \subseteq \mathcal{O}$.
	Now $c-\epsilon$ is not an upper bound for $E$, and so there must be an $x \in E$ with $c-\epsilon < x$. Because $x \in E$, there exists a finite collection $\{ \mathcal{O}_1, \cdots, \mathcal{O}_k \}$ of sets in $\mathcal{F}$ that covers $[a,x]$.
	Then clearly the finite collection $\{ \mathcal{O}_1, \cdots, \mathcal{O}_k, \mathcal{O} \}$ covers the interval $[a,c+ \epsilon)$.
	Therefore $c=b$, otherwise there exists a number $c +\tfrac{1}{2}\epsilon$ that has a finite subcover and $c < c +\tfrac{1}{2}\epsilon$ implies that $c$ is not an upper bound for $E$.
	Thus $[a,b] \in E$ and $[a,b]$ can be covered by a finite number of sets of $\mathcal{F}$.
\end{proof}

\begin{namedthm*}{The Heine-Borel Theorem $(\impliedby)$}
	Let $F$ be a real set such that every open cover of $F$ has a finite subcover. Then $F$ is closed and bounded.
\end{namedthm*}
\begin{proof}
	Let $K$ be a compact subset of a metric space $X$. Proving that $X \setminus K$ is open will show that $K$ is closed.
	Consider any $p \in X \setminus K$. For a $k \in K$, let $O_k$ and $I_k$ be neighborhoods of $p$ and $k$ respectively, with radius less than $\tfrac{1}{2} d(p,q)$.
	Because $K$ is compact, there are finitely many points $k_1, \cdots, k_n$ in $K$ such that $K \subseteq I_{k_1} \cup \cdots \cup I_{k_n}$.
	Let $O = O_{k_1} \cap \cdots \cap O_{k_n}$ so that $O$ is an open neighborhood of $p$ that does not intersect $K$.
	Then $O \subseteq X \setminus K$ and $X\setminus K$ is open. Therefore $K$ is closed. 
\end{proof}

\begin{namedthm*}{The Nested Set Theorem}
Let $\{F_n\}_{n=1}^\infty$ be a descending countable collection of nonempty closed sets of real numbers for which $F_1$ is bounded.
Then
\[
    \bigcap_{n=1}^\infty F_n \neq \emptyset.
\]
\end{namedthm*}
\begin{proof} 
By contradiction, suppose that $\bigcap_{n=1}^\infty F_n = \emptyset$. 
Then $\bigcup_{n=1}^\infty F_n^c = (\bigcap_{n=1}^\infty F_n)^c  = \emptyset^c = \mathbb{R}$, and we have an open cover of $\mathbb{R}$ and thus an open cover of $F_1 \subseteq \mathbb{R}$. 
By the Heine-Borel Theorem, there exists an $N \in \mathbb {N}$ such that $F_1 \subseteq \bigcup_{n=1}^N F_n^c$.  
Because $\{F_n\}$ is descending, $F_n \supseteq F_{n+1}$ for any $n \ge 1$. 
This implies $F_{n}^c \subseteq F_{n+1}^c$, and thus $F_1 \subseteq \bigcup_{n=1}^N F_n^c = F_N^c = \mathbb{R}\setminus F_N$.
This is a contradiction to the assumption that $F_N$ is a nonempty subset of $F_1$.
\end{proof}

\begin{center}
	\textbf{PROBLEMS}
\end{center}
\begin{enumerate}
	\setcounter{enumi}{26}
	\item Is the set of rational numbers open or closed?\par
	The set of rationals is neither open nor closed.
	The rationals is not open because the irrationals are dense in the rationals; that is, between any two rationals there is an irrational.
	The rationals is not closed because it does not contain all its limit points; a sequence of rationals can be constructed that converges to an irrational.
	(Thus we see that the irrationals is neither open nor closed as well.)
	\item What are the sets of real numbers that are both open and closed?\par
	It is clear that $\mathbb{R}$ is open, and $\emptyset$ is open (vacuously).
	Then because the complement of an open set is closed, $\mathbb{R}$ and $\emptyset$ are both closed as well.
	\item Find two sets $A$ and $B$ such that $A \cap B = \emptyset$ and $\overline A \cap \overline B \neq \emptyset.$\par
	Let $A= (4,5)$ and $B = (5,20)$. Then $(4,5) \cap (5,20) = \emptyset$ and $\overline A= [4,5]$ and $\overline B = [5,20]$ so $[4,5] \cap [5,20]= \{5\} \neq \emptyset$.\par
	Let $A= \mathbb{Q}$ and $B = \mathbb{Q}^c$. Then $\mathbb{Q} \cap \mathbb{Q}^c = \emptyset$ and $\overline A= \mathbb{R}$ and $\overline B = \mathbb{R}$ so $\mathbb{R} \cap \mathbb{R}= \mathbb{R} \neq \emptyset$.\par
	\item A point $x$ is called an \textbf{accumulation point} of a set $E$ provided it is a point of closure of $E \setminus \{ x\}.$
	\begin{enumerate}[label=(\roman*),align=left]
        \item Show that the set $E'$ of accumulation points of $E$ is a closed set.\par
        Then for $x \in E'$, every open interval that contains $x$ also contains a point in $E \setminus \{x\}$.\par
		Suppose $E'$ is not closed. 
		Then there exists an element $y \notin E'$ such that every open interval that contains $y$ also contains a point $x \in E'$.
		Then every open interval that contains $x$ contains a point $z \in E \setminus \{x\}$... 
		It can be shown that $y \in E'$ and so $E'$ contains all its points of closure and is thus closed.
        \item Show that $\overline E = E \cup E'.$\par
        $E$ includes all the isolated points not included in $E'$. 
    \end{enumerate}
	\item A point $x$ is called an \textbf{ isolated point} of a set $E$ provided there is an $r>0$ for which $(x-r,x+r)\cap E = \{x\}.$ Show that if a set $E$ consists of isolated points, then it is countable.\par
	Each singleton set $\{x\}$ can be enumerated.
	\item A point $x$ is called an \textbf{interior point} of a set $E$ if there is an $r>0$ such that the open interval $(x-r,x+r)$ is contained in $E$. The set of interior points of $E$ is called the \textbf{interior} of $E$ denoted by int $E$. Show that
	\begin{enumerate}[label=(\roman*),align=left]
        \item $E$ is open iff $E = \text{ int } E$.\par
        $(\implies)$ Suppose $E$ is open.\par
		Then clearly every point of $E$ is an interior point.
		\par
		$(\impliedby)$ Suppose $E = \text{ int } E$.\par
		Then every point has an open neighborhood contained in $E$, so $E$ is open.
        \item $E$ is dense iff $ \text{ int } (\mathbb{R} \setminus E)= \emptyset$.
    \end{enumerate}
	\item Show that the nested set theorem is false if $F_1$ is unbounded.\par
	The nested set theorem works because the compactness of $F_1$ allows us to reach a contradiction to the fact that the intersection is empty (see the proof above).\par
	Consider
	\[
	\bigcap_{n=1}^\infty [n, \infty) = \emptyset.
	\]
	This intersection is empty because for any $x$, there exists an $n \in \mathbb{N}$ such that $x < n$ and thus $x \notin [n,\infty)$.
	\item Show that the assertion of the Heine-Borel Theorem is equivalent to the Completeness Axiom for the real numbers. Show that the assertion of the Nested Set Theorem is equivalent to the Completeness Axiom for the real numbers.\par
	The Heine-Borel Theorem States that Closed and bounded sets are compact; that is, every open cover of a closed and bounded set has a finite subcover.
	If a set $E$ is bounded, then for any open cover $E \subseteq \mathcal{F}$ there exists a finite open subcover $\mathcal{O} \subseteq \mathcal{F}$. 
	We can consider the intersection of all such $\mathcal{O}$ so that $E \subseteq \bigcap_{\mathcal{O} \subseteq \mathcal{F}} \mathcal{O} \subseteq \mathcal{O}$, and this intersection is the supremum. 
	\par
	Clearly the descending sets from the nested set theorem are closed and bounded, so the Heine-Borel Theorem discussed above can be used to imply the Completeness Axiom.
	\item Show that the collection of Borel sets is the smallest $\sigma$-algebra that contains the closed sets.\par
	The Borel sets is defined to be the smallest $\sigma$-algebra that contains all the open sets of real numbers.
	Any sigma-algebra that contains the closed sets contains the open sets by the complement property of a sigma-algebra, so the Borel sets is the smallest sigma-algebra that contains the closed sets as well. 
	\item Show that the collection of Borel sets is the smallest $\sigma$-algebra that contains the intervals of the form $[a,b)$, where $a<b.$\par
	Any interval $[a,b)$ can be written in the form
	\[
	[a,b) = \bigcup_{n=1}^\infty [a,b-\tfrac{1}{n}]	
	\] 
	\item Show that each open set is an $F_{\sigma}$ set.\par
	Any open set $(a,b)$ can be written in the form
	\[
		(a,b) = \bigcup_{n=1}^\infty [a+\tfrac{1}{n},b-\tfrac{1}{n}].	
	\] 
\end{enumerate}

\section{Sequences of Real Numbers}

\begin{namedthm*}{Proposition 14}
Let the sequence of real numbers $\{a_n\}$ converge to the real number $a$. 
Then the limit is unique, the sequence is bounded, and, for a real number $c$, 
\[
\text{if } a_n \le c \text{ for all } n, \text{ then } a\le c.	
\]	
\end{namedthm*}
\begin{proof}
	Suppose there exist $a$ and $b$ such that $\{a_n\}\to a$ and $\{a_n\}\to b$.
	Then For any $\epsilon >0$, there exists the index $N = \max \{N_a,N_b\}$ such that for all $n \ge N \ge N_a,N_b$, then $|a-a_n| < \epsilon$ and $|b-a_n| < \epsilon$.
	By the triangle inequality, $|a-b| \le |a-a_n| + |a_n-b| < \epsilon + \epsilon = 2 \epsilon = \epsilon ' $.
	Therefore $a=b$, and the limit is unique. \par
	Consider $\epsilon =1$. Then there exists an index $N \in \mathbb{N}$ such that for all $n \ge N$, $|a_n-a| < 1$.
	Also, $|a_n|-|a| \le |a_n-a| <1\implies |a_n| < |a| +1$.
	Let $M = \max \{|a_1|, |a_2|, \cdots, |a_N|, |a|+1 \}$. The maximum exists because this is a finite set of real numbers (totally ordered).
	Considering any $n \in \mathbb{N}$, if $n \ge N$, then $|a_n-a| <1\implies |a_n| < |a| +1 \le M$, and if $n<N$, then $|a_n| \le \max \{|a_1|, |a_2|, \cdots, |a_N|, |a|+1 \} =M$, so $M$ is a bound for this sequence.
	\par
	Suppose that for all $n$, $a_n \le c$ but $a>c$. 
	Then $a_n \le c < a$ for all $n$, and $0 \le c-a_n <a-a_n$.
	Choosing $\epsilon = c-a_n$, there exists an index such that $|a-a_n| < c-a_n$. But this is a contradiction. 
	Therefore $a \le c$.
\end{proof}

\begin{flushleft}

\begin{namedthm*}{Theorem 15}[the Monotone Convergence Criterion for Real Sequences]
	A monotone sequence of real numbers converges iff it is bounded.
\end{namedthm*}
\begin{proof}
	$(\implies)$ Suppose a monotone sequence converges.\par
	By the above proposition, it is bounded.\par
	$(\impliedby)$ Suppose a monotone sequence $\{a_n\}$ is bounded.\par
	By the Completeness Axiom, there exists a supremum say $a$ such that $a_n \le a$ for all $n$.
	Consider any $\epsilon >0$. Now, $a-\epsilon$ is not an upper bound, and because the sequence is increasing, there exists an index $N$ for which $a_n \ge a_N > a-\epsilon$ for all $n \ge N$.
	Then $\epsilon > a-a_n$ and the sequence converges to $a$. The proof is the same for a decreasing sequence. 
\end{proof}

\begin{namedthm*}{Theorem 16}[The Bolzano-Weierstrass Theorem]
Every bounded sequence of real numbers has a convergent subsequence.	
\end{namedthm*}
\begin{proof}
	Let $a_n$ be a bounded sequence of real numbers. Choose $M>0$ s.t. $|a_n| \le M$ for all $n$. 
	Define $E_n = \overline{\{a_j \ |\ j \ge n\}}$. Then we also have $E_n \subseteq [-M,M]$ and $E_n$ is closed since it is the closure of a set.
	Therefore $\{E_n\}$ is a descending sequence of nonempty closed bounded subsets of real numbers. 
	The Nested Set Theorem tells us that $\bigcap_{n=1}^\infty E_n \neq \emptyset$, so there exists $a \in \bigcap_{n=1}^\infty E_n$.
	For each natural number $k$, $a$ is a point of closure of $\{a_j \ |\ j \ge k\}$.
	Thus for infinitely many indices $j \ge n$, $a_j$ belongs to $(a-\tfrac{1}{k},a+\tfrac{1}{k})$.
	By induction, choose a strictly increasing subsequence of natural numbers $n_k$ such that $|a-a_{n_k}|< \tfrac{1}{k}$ for all $k$.
	From the Archimedean Property of the reals, the subsequence $\{ a_{n_k} \}$ converges to $a$.
\end{proof}

\begin{namedthm*}{Proposition 19}
Let $\{a_n \}$ and $\{b_n \}$ be sequences of real numbers.
\begin{enumerate}[label=(\roman*),align=left]
	\item $\lim \sup \{a_n \}=\ell \in \mathbb{R}$ iff for each $\epsilon >0$, there are infinitely many indices $n$ for which $a_n > l-\epsilon $ and only finitely many indices $n$ for which $a_n < l-\epsilon $.\par
	$(\implies)$ Suppose $\lim \sup \{a_n \}=\ell \in \mathbb{R}$.\par
	Then by problem 38, $\ell$ is a cluster point of the sequence. This means that for all $\epsilon > 0$, there exists a subsequence $\{a_{n_k} \}$ such that $ \ell - a_{n_k} < \epsilon$ for all $n_k$ greater than some index, and thus $ \ell - \epsilon < a_{n_k} $ for infinitely many indices $n_k$.\par
	Suppose by contradiction that for $\epsilon >0$, there are infinitely many indices $n$ for which $a_n < l-\epsilon $.
	That is, no matter how large the epsilon we choose, there exists a subsequence $\{a_{n_k} \}$ such that $\epsilon < l-a_{n_k} $ for all $n_k$ after a certain index.
	This implies that $\{a_n\}$ is not bounded, so by Proposition 14, the sequence does not converge to a real number.
	This is a contradiction to $\ell \in \mathbb{R}$.
	\par
	$(\impliedby)$ Suppose for $\epsilon >0$, there are infinitely many indices $n$ for which $a_n > l-\epsilon $ and only finitely many indices $n$ for which $a_n < l-\epsilon $.\par
	Then choosing specific indices $n_k$, there exists a subsequence $\{a_{n_k} \}$ such that $\ell - a_{n_k} <\epsilon $ for all $n_k$, and this implies the subsequence converges to $\ell$.
	If we suppose that $\ell \neq \lim \sup \{a_n \}$, then there exists some $\delta >0$ such that $\ell > \ell - \delta = \lim \sup \{a_n \}$.\par
	Now, $\ell - \delta = \lim \sup \{a_n \} = \lim_{n \to \infty} \sup \{ a_k\ |\ k \ge n\}$.
	That means for any $n$, $a_k \le \ell - \delta$ for $k \ge n$.
	However, this is a contradiction to the fact that there are only finitely many such indices $k$ for which this is true.
	Therefore $\ell = \lim \sup \{a_n \}$.
	\item $\lim \sup \{a_n \}=\infty$ iff $\{a_n \}$ is not bounded above.\par
	$(\implies)$ Suppose $\lim \sup \{a_n \}=\infty$.\par
	This implies that $\infty = \lim \sup \{a_n \}$ is a cluster point and there exists a subsequence that converges to infinity.
	Therefore $\{a_n \}$ is not bounded above.\par
	$(\impliedby)$ Suppose $\{a_n \}$ is not bounded above.\par
	By Proposition 4, $\{a_n \}$ does not converge to a real number.
	Also,$\{a_n \}$ is not bounded above implies that for any real number $c$, there exists an index such that $a_n > c$.
	Then the only upper bound of this sequence is $\infty$ and thus $\lim \sup \{a_n \}=\infty$.
	\item 
	\[
		\lim \sup \{a_n \}= -\lim \inf \{-a_n \}. 	
	\]
	Definitions of limsup and liminf:\par
	$\lim \sup \{a_n \} = \lim_{n \to \infty} [\sup \{ a_k\ |\ k \ge n\}]$
	$\implies$ for any $n \in \mathbb{N}$, $\sup \{ a_k\ |\ k \ge n\} \ge a_k$ for $k \ge n$.\par
	$\lim \inf \{a_n \} = \lim_{n \to \infty} [\inf \{ a_k\ |\ k \ge n\}]$.
	$\implies$ for any $n \in \mathbb{N}$, $\inf \{ a_k\ |\ k \ge n\} \le a_k$ for $k \ge n$.\par
	Now we have\par
	$\lim \inf \{-a_n \} = \lim_{n \to \infty} [\inf \{ -a_k\ |\ k \ge n\}]$.\par
	$\implies$ for any $n \in \mathbb{N}$, $\inf \{ -a_k\ |\ k \ge n\} \le -a_k$ for $k \ge n$.\par
	$\implies$ for any $n \in \mathbb{N}$, $-\inf \{ -a_k\ |\ k \ge n\} \ge a_k$ for $k \ge n$, the definition of limsup.\par
	\item A sequence of real numbers $\{ a_n\}$ converges to an extended real number $a$ iff 
	\[
		\lim \inf \{a_n \}= \lim \sup \{a_n \} = a.
	\]
	$(\implies)$ Suppose a sequence of real numbers $\{ a_n\}$ converges to an extended real number $a$.\par
	Clearly $\lim \inf \{a_n \} \le a \le \lim \sup \{a_n \} $.\par
	If $\lim \inf \{a_n \} < a < \sup \{a_n \} $, then we reach a contradiction to the infimum and supremum respectively.\par
	Therefore $\lim \inf \{a_n \} = a = \lim \sup \{a_n \} $.
	\par
	$(\impliedby)$ Suppose $\lim \inf \{a_n \}= \lim \sup \{a_n \} = a$.\par
	Then for any $n \in \mathbb{N}$, $\inf \{ a_k\ |\ k \ge n\} \le a_k \le \sup \{ a_k\ |\ k \ge n\}$ for $k \ge n$, which implies
	\[a= \lim \inf \{a_n \} = \lim_{n \to \infty} \inf \{ a_k\ |\ k \ge n\} \le \lim_{n \to \infty} a_k \le \lim_{n \to \infty} \sup \{ a_k\ |\ k \ge n\}= \lim \sup \{a_n \} = a\]
	Clearly $\{ a_n\}$ converges to $a$.
	\item If $a_n \le b_n$ for all $n$, then
	\[
		\lim \sup \{a_n \} \le \lim \sup \{b_n \}.	
	\]
	For any $n \in \mathbb{N}$, $a_k \le \sup \{ a_k\ |\ k \ge n\}$ and $b_k \le \sup \{ b_k\ |\ k \ge n\} $ for all $k \ge n$.\par
	If we suppose $\lim \sup \{a_n \} > \lim \sup \{b_n \}$, then there exists a natural number $n$ such that $\sup \{ a_k\ |\ k \ge n\} > \sup \{ b_k\ |\ k \ge n\} \ge b_k \ge a_k$ for all $k \ge n$.
	However, by problem 38, we see that $\lim \sup \{a_n \}$ is a cluster point of $\{a_n \}$, and we reach a contradiction. (or contradiction to def of supremum?)
\end{enumerate}	
\end{namedthm*}

\begin{namedthm*}{Proposition 20}
	Let $\{a_n \}$ be a sequence of real numbers.
	\begin{enumerate}[label=(\roman*),align=left]
		\item The series $\textstyle \sum_{k=1}^\infty a_k$ is summable iff for each $\epsilon >0$, there is an index $N$ for which
		\[
			\biggl | \sum_{k=n}^{n+m} a_k \biggr | < \epsilon \text{ for } n \ge N \text{ and any natural number } m.	
		\]
		\item If the series $\sum_{k=1}^\infty |a_k|$ is summable, then $\sum_{k=1}^\infty a_k$ is also summable.
		\item If each term $a_k$ is nonnegative, then the series $\sum_{k=1}^\infty a_k$ is summable iff the sequence of partial sums is bounded.
	\end{enumerate}	
\end{namedthm*}


\end{flushleft}

\begin{center}
	\textbf{PROBLEMS}
\end{center}
\begin{enumerate}
	\setcounter{enumi}{37}
	\item We call an extended real number a \textbf{cluster point} of a sequence $\{ a_n\}$ if a subsequence converges to this extended real number. Show that $\lim \inf \{a_n\}$ is the smallest cluster point of $\{a_n\}$ and $\lim \sup \{a_n\}$ is the largest cluster point of $\{a_n\}$.\par
	Let $s = \lim \sup \{a_n\} = \lim_{n \to \infty} \sup \{ a_k\ |\ k \ge n\}$.
	Suppose there exists a subsequence $\{ a_{n_k} \}$ that converges to an extended real number $a$.
	Fix $\epsilon >0$. Then there exists an index $M$ such that $|a-a_{n_m}| < \epsilon$ when $n_m \ge M$, and $a_{n_m} \le \sup \{ a_k\ |\ k \ge M\}$.\par
	Then $\lim_{M \to \infty} a_{n_m} \le \lim_{M \to \infty} \sup \{ a_k\ |\ k \ge M\} \implies a \le s$.\par
	Therefore $\lim \sup \{a_n\}$ is the largest cluster point of $\{a_n\}$.
	($\lim \sup \{a_n\}$ is itself a cluster point else we reach a contradiction to the supremum.)
	The same method can be used to prove $\lim \inf \{a_n\}$.
	\item Prove proposition 19.\par
	See above for proof.
	\item Show that a sequence $\{a_n\}$ is convergent to an extended real number iff there is exactly one extended real number that is a cluster point of the sequence.\par
	$(\implies)$ Suppose $\{a_n\}$ is convergent to an extended real number $a$.\par
	By Proposition 19(iv), we have $\lim \inf \{a_n \}= \lim \sup \{a_n \} = a$, so clearly any cluster point is equal to $a$.
	\par
	$(\impliedby)$ Suppose there is exactly one extended real number $a$ that is a cluster point of $\{a_n\}$.\par
	Then there exists a subsequence that converges to $a$.
	Suppose that $\{a_n\}$ does not converge to $a$.
	Then there exists an $\epsilon > 0$ such that there are infinitely many indices $n$ for which $a-a_n > \epsilon$.
	Collect these indices to construct a subsequence $\{a_{n_k}\}$.
	In the case that $\{a_{n_k}\}$ is bounded, there exists another subsequence of $\{a_{n_k}\}$ that converges to a real number $b \neq a$. 
	But this is also a subsequence of the original sequence $\{a_n\}$, which implies $\{a_n\}$ has two cluster points $a$ and $b$, a contradiction.
	In the case that $\{a_{n_k}\}$ is unbounded, then for any real number $c$, there exists an index $n$ such that $|a_n| >c$.
	Then we can construct a subsequence that converges to $+\infty \neq a$ or $-\infty \neq a$, which is again a contradiction to the fact that $\{a_n\}$ has only one cluster point.
	\item Show that $\lim \inf a_n \le \lim \sup a_n$.\par
	For any natural number $n$, we have $\inf \{ a_k\ |\ k \ge n\} \le a_k \le \sup \{ a_k\ |\ k \ge n\}$ for all $k \ge n$.
	Taking the limit with respect to n clearly proves the statement.
	\item Prove that if, for all $n$, $a_n \ge 0$ and $b_n \ge 0 $, then \[ \lim \sup [a_n \cdot b_n] \le (\lim \sup a_n) \cdot (\lim \sup b_n),\] provided the product on the right is not of the form $0 \cdot \infty.$
	\item Show that every real sequence has a monotone subsequence. Use this to provide another proof of the Bolzano-Weierstrass Theorem.
	\item Let $p$ be a natural number greater than 1, and $x$ a real number $0 \le x \le 1.$ Show that there is a sequence $\{a_n\}$ of integers with $0 \le a_n < p$ for each $n$ such that \[ x = \sum_{n=1}^\infty\dfrac{a_n}{p^n} \] 
	and that this sequence is unique except when $x$ is of the form $q/p^n$, $0<q<p^n$, in which case there are exactly two such sequences. Show that, conversely, if $\{a_n\}$ is any sequence of integers with $0\le a_n < p$, the series \[ x = \sum_{n=1}^\infty\dfrac{a_n}{p^n} \] 
	converges to a real number $x$ with $0 \le x \le 1$. If $p = 10$, this sequence is called the \textit{decimal} expansion of $x$. For $p=2$ it is called the \textit{binary} expansion; and for $p=3$, the \textit{ternary} expansion.
	\item Prove Proposition 20.
	\item Show that the assertion of the Bolzano-Weierstrass Theorem is equivalent to the Completeness Axiom for the real numbers. Show that the assertion of the Monotone Convergence Theorem is equivalent to the Completeness Axiom for the real numbers.
\end{enumerate}

\section{Continuous Real-Valued Functions of a Real Variable}

\begin{center}
	\textbf{PROBLEMS}
\end{center}
\begin{enumerate}
	\setcounter{enumi}{46}
	\item Let $E$ be a closed set of real numbers and $f$ a real-valued function that is defined and continuous on $E$. Show that there is a function $g$ defined and continuous on all of $\mathbb{R}$ such that $f(x) = g(x)$ for each $x \in E$. (Hint: Take $g$ to be linear on each of the intervals of which $\mathbb{R} \setminus E$ is composed.)
	\item Define the real-valued function $f$ on $\mathbb{R}$ by setting 
	\[ 
	f(x) =
	\begin{cases} 
		x & \text{if x irrational}\\
		p \sin \dfrac{1}{q} & \text{if } x = \dfrac{p}{q} \text{ in lowest terms.} \\
	\end{cases}
	\]
	At what points is $f$ continuous?
	\item Let $f$ and $g$ be continuous real-valued functions with a common domain $E$.
	\begin{enumerate}[label=(\roman*),align=left]
        \item Show that the sum, $f+g$, and product, $fg$, are also continuous functions.
        \item If $h$ is a continuous function with image contained in $E$, show that the composition $f \circ h$ is continuous.
        \item Let max$\{f,g\}$ be the function defined by max$\{f,g\}(x)$ = max$\{f(x),g(x)\}$, for $x \in E$. Show that max$\{f,g\}$ is continuous.
        \item Show that $|f|$ is continuous.
    \end{enumerate}
	\item Show that a Lipschitz function is uniformly continuous but there are uniformly continuous functions that are not Lipschitz.
	\item A continuous function $\phi$ on $[a,b]$ is called \textbf{piecewise linear} provided there is a partition $a=x_0<x_1< \cdots <x_n = b$ of $[a,b]$ for which $\phi$ is linear on each interval $[x_i, x_{i+1}]$. Let $f$ be a continuous function on $[a,b]$ and $\epsilon$ a positive number. 
	Show that there is a piecewise linear function $\phi$ on $[a,b]$ with $|f(x)-\phi (x)| < \epsilon$ for all $x \in [a,b]$.
	\item Show that a nonempty set $E$ of real numbers is closed and bounded if and only if every continuous real-valued function on $E$ takes a maximum value.
	\item Show that a set $E$ of real numbers is closed and bounded iff every open cover of $E$ has a finite subcover.
	\item Show that a nonempty set $E$ of real numbers is an interval iff every continuous real-valued function on $E$ has an interval as its image.
	\item Show that a monotone function on an open interval is continuous iff its image is an interval. 
	\item Let $f$ be a real-valued function defined on $\mathbb{R}$. Show that the set of points at which $f$ is continuous is a $G_\delta$ set.
	\item Let $\{ f_n\}$ be a sequence of continuous functions defined on $\mathbb{R}$. Show that the set of points $x$ at which the sequence $\{f_n(x)\}$ converges to a real number is the intersection of a countable collection of $F_\sigma$ sets.
	\item Let $f$ be a continuous real-valued function on $\mathbb{R}$. Show that the inverse image with respect to $f$ of an open set is open, of a closed set is closed, and of a Borel set is Borel.
	\item A sequence $\{f_n\}$ of real-valued functions defined on a set $E$ is said to converge uniformly on $E$ to a function $f$ iff given $\epsilon >0$, there is an $N$ such that for all $x \in E$ and all $n \ge N$, we have $|f_n(x) - f(x)| < \epsilon$. Let $\{f_n\}$ be a sequence of continuous functions defined on a set $E$. Prove that if $\{f_n\}$ converges uniformly to $f$ on $E$, then $f$ is continuous on $E$. 
\end{enumerate}

% Chapter 2
\chapter{Lebesgue Measure}

% 2.1
\section{Introduction}
In this chapter we construct a collection of sets called \textbf{Lebesgue measurable sets}, and a set function of this collection called \textbf{Lebesgue measure}, denoted by $m$. The collection of Lebesgue measurable sets is a $\sigma$-algebra which contains all open sets and all closed sets. The set function $m$ possesses the following three properties:
\begin{namedthm*}{The measure of an interval is its length}
Each nonempty interval $I$ is Lebesgue measurable and 
\[
m(I) = \ell(I).
\]
\end{namedthm*}
\begin{namedthm*}{Measure is translation invariant}
If $E$ is Lebesgue measurable and $y$ is any number then the translate of $E$ by $y$, $E+y = \{x+y \ |\ x \in E\}$, also is Lebesgue measurable and
\[
m(E+y) = m(E).
\]
\end{namedthm*}
\begin{namedthm*}{Measure is countably additive over countable disjoint unions of sets}
If $\{E_k\}_{k=1}^\infty$ is a countable disjoint collection of Lebesgue measurable sets, then
\[
m(\bigcup_{k=1}^\infty E_k) = \sum_{k=1}^\infty m(E_k).
\]
\end{namedthm*}
It is not possible to construct a set function that possesses the above three properties and is defined for all sets of real numbers (See Vitali sets).
We first construct a set function called \textbf{outer measure}, denoted by $m^*$, such that: 
\begin{enumerate}[label=(\roman*),align=left]
	\item the outer measure of an interval is its length,
	\item outer measure is translation invariant,
	\item outer measure is countably subadditive.
\end{enumerate}
Then the Lebesgue measure $m$ is the restriction of $m^*$ to the Lebesgue measurable sets.

\begin{center}
	\textbf{PROBLEMS}
\end{center}
In the first three problems, let $m$ be a set function defined for all sets in a $\sigma$-algebra $\mathcal{A}$ with values in $[0,\infty]$. Assume $m$ is countably additive over countable disjoint collections of sets in $\mathcal{A}$.
\begin{enumerate}
	\setcounter{enumi}{0}
	\item Prove that if $A$ and $B$ are two sets in $\mathcal{A}$ with $A \subseteq B$, then $m(A) \le m(B)$. This property is called \textit{monotonicity}.\par
	$A \subseteq B \implies B = A \cup (B \cap A^c),$ where $A \cap (B \cap A^c) = \emptyset$. The set $(B \cap A^c)$ is measurable because $A^c$ is measurable and countable intersection is measurable, so $m(B) = m(A \cup (B \cap A^c)) = 	m(A) + m(B \cap A^c)$ by countable additivity, and thus $m(B) \ge m(A)$.
	\item Prove that if there is a set $A$ in the collection $\mathcal{A}$ for which $m(A) < \infty$, then $m(\emptyset) = 0$.\par
	We have $A\cap\emptyset = \emptyset$ and $A\cup\emptyset = A$. 
    \begin{align*}
        m(A)&=m(A\cup\emptyset)\\
        m(A)&=m(A)+m(\emptyset)&&\text{ by disjoint additivity}\\
        0&=m(\emptyset).
    \end{align*}
	\item Let $\{E_k\}_{k=1}^\infty$ be a countable collection of sets in $\mathcal{A}$. Prove that $m(\bigcup_{k=1}^\infty E_k) \le \sum_{k=1}^\infty m(E_k).$
    For any two measurable sets $A,B$, we have $A\cup B=(A\setminus B)\cup(B)$.
    By disjoint additivity,
    \[
        m(A\cup B) = m(A\setminus B)+m(B)
    \]
    Now, by problem 1, $(A\setminus B)\subseteq A$ implies that $m(A\setminus B)\le m(A)$.
    Therefore
    \[
        m(A\cup B) \le m(A)+m(B).
    \]
    \item A set function $c$, defined on all subsets of $\mathbb{R}$, is defined as follows.
	Define $c(E)$ to be $\infty$ if $E$ has infinitely many members and $c(E)$ to be equal to the number of elements in $E$ if $E$ is finite; define $c(\emptyset)=0$. Show that $c$ is a countably additive and translation invariant set function. This set function is called the \textbf{counting measure}.\\
    Suppose $E=\{x_1,\cdots,x_n\}$.\\
    Then $m(E)=n$. For any real number $y$, $y+E = \{y+x_1,\cdots,y+x_n\}$, so $m(y+E)=n$.\\
    Suppose $E$ has infinitely many members.\\
    Then $y+E$ has infinitely members as well, so $m(E)=m(y+E)=\infty$.\\
    Let $\{E_k\}_{k=1}^\infty$ be a disjoint collection of sets of real numbers.
    In the case that there exists an $E_k$ with infinitely many members, then the countable additivity is clear.
    \\In the case that all sets $E_k$ are finite, for any two sets $E_i,E_j$:\\
    $E_i = \{x_1,\cdots,x_n\}$\\
    $E_j = \{y_1,\cdots,y_m\}$\\  
    Then $E_i\cup E_j = \{x_1,\cdots,x_n,y_1,\cdots,y_m\}$ and $m(E_i\cup E_j)=n+m=m(E_i) +m(E_j)$.
\end{enumerate}

% 2.2
\section{Lebesgue Outer Measure}

\begin{center}
	\textbf{PROBLEMS}
\end{center}
\begin{enumerate}
	\setcounter{enumi}{4}
	\item By using properties of outer measure, prove that the interval $[0,1]$ is not countable.
	\item Let $A$ be the set of irrational numbers in the interval $[0,1]$. Prove that $m^*(A)=1$.
	\item A set of real numbers is said to be a $G_\delta$ set provided it is the intersection of a countable collection of open sets.
	Show that for any bounded set $E$, there is a $G_\delta$ set $G$ for which
	\[E\subseteq G \ \text{and}\ m^*(G)=m^*(E).\]
	\item Let $B$ be the set of rational numbers in the interval $[0,1]$, and let $\{I_k\}_{k=1}^n$ be a finite collection of open intervals that covers $B$.
	Prove that $\textstyle \sum_{k=1}^n m^*(I_k) \ge 1$.
	\item Prove that if $m^*(A)=0$, then $m^*(A\cup B) = m^*(B)$.
	\item Let $A$ and $B$ be bounded sets for which there is an $\alpha >0$ such that $|a-b| \ge \alpha$ for all $a \in A, b \in B$.
	Prove that $m^*(A \cup B) = m^*(a)+m^*(B)$.
\end{enumerate}



% 2.3
\section{The $\sigma$-Algebra of Lebesgue Measurable Sets}

\begin{center}
	\textbf{PROBLEMS}
\end{center}
\begin{enumerate}
	\setcounter{enumi}{10}
	\item Prove that if a $\sigma$-algebra of subsets of $\mathbb{R}$ contains intervals of the form $(a,\infty)$, then it contains all intervals.\\
	\item Show that every interval is a Borel set.
	\item Show that 
	\begin{enumerate}[label=(\roman*),align=left]
        \item the translate of an $F_\sigma$ set is also $F_\sigma$,
        \item the translate of a $G_\delta$ set is also $G_\delta$,
        \item the translate of a set of measure zero also has measure zero.
    \end{enumerate}
    \item Show that if a set $E$ has positive outer measure, then there is a bounded subset of $E$ that also has positive outer measure.
    \item Show that if $E$ has finite measure and $\epsilon>0$, then $E$ is the disjoint union of a finite number of measurable sets, each of which has measure at most $\epsilon$.
\end{enumerate}

% 2.4
\section{Outer and Inner Approximation of Lebesgue Measurable Sets}

\begin{namedthm*}{Theorem 11}
	Let $E$ be any set of real numbers. Then each of the following four assertions is equivalent to the measurability of $E$.\\
	(Outer Approximation by Open Sets and $G_\delta$ sets)
	\begin{enumerate}[label=(\roman*),align=left]
        \item For each $\epsilon>0$, there is an open set $\mathcal{O}$ containing $E$ for which $m^*(\mathcal{O}\setminus E)<\epsilon$.
        \item There is a $G_\delta$ set $G$ containing $E$ for which $m^*(G\setminus E)=0$. 
    \end{enumerate}
	(Inner Approximation by Closed Sets and $F_\sigma$ sets)
	\begin{enumerate}[label=(\roman*),align=left]
        \setcounter{enumi}{2}
		\item For each $\epsilon>0$, there is a closed set $F$ contained in $E$ for which $m^*(E\setminus F)<\epsilon$.
        \item There is a $F_\sigma$ set $F$ contained in $E$ for which $m^*(E\setminus F)=0$.
    \end{enumerate}
\end{namedthm*}

\begin{center}
	\textbf{PROBLEMS}
\end{center}
\begin{enumerate}
	\setcounter{enumi}{15}
	\item Complete the proof of Theorem 11 by showing that measurability is equivalent to (iii) and also equivalent to (iv).
	\item Show that a set $E$ is measurable iff for each $\epsilon>0$, there is a closed set $F$ and open set $\mathcal{O}$ for which $F\subseteq E\subseteq \mathcal{O}$ and $m^*(\mathcal{O}\setminus F)<\epsilon$.
	\item Let $E$ have finite outer measure. Show that there is a $G_\delta$ set $G\supseteq E$ with $m(G)=m^*(E)$.
	Show that $E$ is measurable iff there is an $F_\sigma$ set $F \subseteq E$ with $m(F)=m^*(E)$. 
	\item Let $E$ have finite outer measure.
\end{enumerate}

% 2.5
\section{Countable Additivity, Continuity, and the Borel-Cantelli Lemma}

% 2.6
\section{Nonmeasurable Sets}

% 2.7
\section{The Cantor Set and the Cantor-Lebesgue Function}

% Chapter 3
\chapter{Lebesgue Measurable Functions}

% 3.1
\section{Sums, Products, and Compositions}
\begin{flushleft}
    \begin{namedthm*}{Proposition 1}
        Let the function $f$ have a measurable domain $E$. Then the following statements are equivalent:
        \begin{enumerate}[label=(\roman*),align=left]
            \item For each real number $c$, the set $\{x\in E\ |\ f(x)>c\}$ is measurable.
            \item For each real number $c$, the set $\{x\in E\ |\ f(x)\ge c\}$ is measurable.
            \item For each real number $c$, the set $\{x\in E\ |\ f(x)<c\}$ is measurable.
            \item For each real number $c$, the set $\{x\in E\ |\ f(x)\le c\}$ is measurable.
        \end{enumerate}
        Each of these properties implies that for each extended real number $c$,
        \begin{center}
            the set $\{x\in E\ |\ f(x)= c\}$ is measurable.
        \end{center}
    \end{namedthm*}

    \begin{namedthm*}{Definition}
        An extended real-valued function $f$ defined on $E$ is said to be \textbf{Lebesgue measurable}, or simply \textbf{measurable}, provided its domain $E$ is measurable and it satisfies one of the four statements of Proposition 1.        
    \end{namedthm*}

\begin{namedthm*}{Proposition 2}
    Let the real-valued function $f$ be defined on a measurable set $E$.
    Then the function $f$ is measurable iff for each open set $\mathcal{O}$, the inverse image of $\mathcal{O}$ under f, $f^{-1}(\mathcal{O})=\{x\in E\ |\ f(x)\in\mathcal{O}\}$, is a measurable set.
\end{namedthm*}
\begin{proof}
    Let $f:E\to\mathbb{R}$, where $E$ is a measurable set.\\
    $(\implies)$ Suppose that $f$ is measurable.\\
    Let $\mathcal{O}$ be open. Then by Chapter 1, Proposition 9, $\mathcal{O}$ can be written as the countable disjoint union of open intervals: $\mathcal{O}=\bigcup_{k=1}^\infty I_k$.
    We can construct these intervals in the following form:
    \[
        I_k=(a_k,b_k)=(-\infty,b_k)\cap(a_k,\infty)  
    \] 
    Therefore we see that
    \begin{align*}
        f^{-1}(\mathcal{O})&=f^{-1}(\bigcup_{k=1}^\infty I_k)\\
        &=f^{-1}(\bigcup_{k=1}^\infty (-\infty,b_k)\cap(a_k,\infty))\\
        &=\bigcup_{k=1}^\infty f^{-1}((-\infty,b_k)\cap(a_k,\infty))\\
        &=\bigcup_{k=1}^\infty f^{-1}(-\infty,b_k)\cap f^{-1}(a_k,\infty).
    \end{align*}
    Because $f$ is measurable, we see that $f^{-1}((-\infty,b_k))$ and $f^{-1}((a_k,\infty))$ are measurable sets, and countable union and intersection of measurable sets is also a measurable set, so $f^{-1}(\mathcal{O})$ is a measurable set.
    \\$(\impliedby)$ Suppose that for each open set $\mathcal{O}$, $f^{-1}(\mathcal{O})$ is a measurable set.\\
    Because for any real number $c$, the interval of the form $(c,\infty)$ is an open set, and therefore we have that the set $f^{-1}((c,\infty))=\{x\in E\ |\ f(x)\in(c,\infty)\}=\{x\in E\ |\ f(x)>c\}$ is measurable, which implies that $f$ is measurable.
\end{proof}

\begin{namedthm*}{Proposition 5}
    Let $f$ be an extended real-valued function on the measurable set $E$.
    \begin{enumerate}[label=(\roman*),align=right]
        \item If $f$ is measurable on $E$ and $f=g$ a.e. on $E$, then $g$ is measurable on $E$.
        \item For a measurable subset $D$ of $E$, $f$ is measurable on $E$ iff the restrictions of $f$ to $D$ and $E\setminus D$ are measurable.
    \end{enumerate}
\end{namedthm*}
\begin{proof}
    Let $f$ be an extended real-valued function on the measurable set $E$.
    \begin{enumerate}[label=(\roman*),align=right]
        \item Let $f$ be measurable on $E$ and $f=g$ a.e. on $E$.\\
        Define $A=\{x\in E\ |\ f(x)\neq g(x)\}\subseteq E$, so that $f=g$ on $E\setminus A$, and $m(A)=0$.
        \begin{align*}
            \{x\in E\ |\ g(x)>c\}&=(\{x\in E\ |\ g(x)>c\}\cap[E\cap A])\cup(\{x\in E\ |\ g(x)>c\}\cap[E\cap A^c])\\
            &=\{x\in A\ |\ g(x)>c\}\cup(\{x\in E\ |\ f(x)>c\}\cap[E\setminus A]).
        \end{align*}
        Now, $\{x\in A\ |\ g(x)>c\}\subseteq A$, and because $m(A)=0$, $\{x\in A\ |\ g(x)>c\}$ is measurable and has measure zero.
        The set $\{x\in E\ |\ f(x)>c\}$ is measurable because $f$ is measurable, and $E\cap A^c$ is measurable because $E$ and $A$ (and thus $A^c$) are measurable.
        Thus $\{x\in E\ |\ g(x)>c\}$ is measurable because it is the finite union and intersection of measurable sets; therefore $g$ is measurable on $E$.
        \item Let $D$ be a measurable subset of $E$.\\
        $(\implies)$ Suppose $f$ is measurable on $E$.\\
        Then for any real number $c$, we see that 
        \begin{align*}
            \{x\in D\ |\ f|_{D}(x)>c\}&=\{x\in E\ |\ f(x)>c\}\cap[E\cap D],\\
            \{x\in E\setminus D\ |\ f|_{E\setminus D}(x)>c\}&=\{x\in E\ |\ f(x)>c\}\cap[E\setminus D],
        \end{align*}
        where both are measurable because they are each intersections of measurable sets. Therefore the restrictions $f|_{D}$ and $f|_{E\setminus D}$ are measurable.\\
        $(\impliedby)$ Suppose the restrictions of $f$ to $D$ and $E\setminus D$ are measurable.\\
        Then for any real number $c$,
        \begin{align*}
            \{x\in E\ |\ f(x)>c\}&=\{x\in D\ |\ f|_{D}(x)>c\}\cup\{x\in E\setminus D\ |\ f|_{E\setminus D}(x)>c\},
        \end{align*}
        which is measurable because it is a union of measurable sets.
    \end{enumerate}
\end{proof}

\end{flushleft}
\begin{center}
	\textbf{PROBLEMS}
\end{center}
\begin{enumerate}
	\setcounter{enumi}{0}
	\item Suppose $f$ and $g$ are continuous functions on $[a,b]$. Show that if $f=g$ a.e. on $[a,b]$, then, in fact, $f=g$ on $[a,b]$.
    Is a similar assertion true if $[a,b]$ is replaced by a general measurable set $E$?\\
    \\Let $f,g$ be continuous functions on $[a,b]$, where $f=g$ on $[a,b]\setminus E_0$, where $E_0$ is a subset of $[a,b]$ and $m(E_0)=0$.
    \\Suppose that $E_0$ is nonempty.\\
    Consider any point $x_0\in E_0\subseteq[a,b]$. For any $\epsilon>0$, there exists a $c\in(x_0-\epsilon,x_0+\epsilon)\cap[a,b]$ such that $f(c)=g(c)$, else we reach a contradiction because $m((x_0-\epsilon,x_0+\epsilon)\cap[a,b])\neq0$.
    This means that we can construct a sequence $\{c_i\}_{i=1}^\infty$ that converges to $x_0$, where $f(c_i)=g(c_i)$ is defined for all $i$. 
    However, because $\{c_i\}\to x_0$, by continuity of $f,g$, we have $\{f(c_i)\}\to f(x_0)$ and $\{g(c_i)\}\to g(x_0)$, and because $f(c_i)=g(c_i)$ for all $i$, the limit is unique; that is,
    \[
        |f(x_0)-g(x_0)|\le|f(x_0)-f(c_i)|+|f(c_i)-g(x_0)|<\epsilon,
    \]
    and $f(x_0)=g(x_0)$.
    \\However, this is a contradiction to $f(x)\neq g(x)$ for all $x\in E_0$, and so $E_0=\emptyset$.\\
    \\In the case that $[a,b]$ is replaced by a general measurable set $E$, the assertion is not true.
    Consider the case where $E=\{a\}$, so that $f,g$ are continuous on $\{a\}$, and $f=g$ a.e. on $\{a\}$.
    This only implies that $f=g$ on $E$ except for a set of measure zero. But $E$ is already of measure zero, so $f(a)\neq g(a)$ is possible, and $f\neq g$ on $E$.
    \item Let $D$ and $E$ be measurable sets and $f$ a function with domain $D\cup E$. We proved that $f$ is measurable on $D\cup E$ iff its restrictions to $D$ and $E$ are measurable.
    Is the same true if "measurable" is replaced by "continuous"?\\
    \\No; consider the function $f:[-1,1]\to\mathbb{R}$, where $[-1,1]=[-1,0)\cup[0,1]$, and we define
    \[
    f(x)=
    \begin{cases}
        0&x\in[-1,0),\\
        1&x\in[0,1].
    \end{cases}    
    \]
    Clearly we have a point of discontinuity at $x=0$, so $f$ is not continuous even though $f|_{[-1,0)}$ and $f|_{[0,1]}$ are continuous.
    \item Suppose a function $f$ has a measurable domain and is continuous except at a finite number of points.
    Is $f$ necessarily measurable?\\
    \\Yes; let $f$ be a function on the measurable domain $E$, and suppose $f$ is continuous on $E\setminus E_0$, where $E_0=\{x_i\}_{i=1}^n \subseteq E$. Then $m(E_0)=0$ because countable sets are measurable and have measure zero.
    \\Now, $f|_{E\setminus E_0}$ is continuous and therefore measurable (Proposition 3), and $f|_{E_0}$ is defined on a set of measure zero, so any subset $\{x\in E_0\ |\ f|_{E_0}(x)>c\}\subseteq E_0$ has measure zero and is thus measurable, and therefore $f|_{E_0}$ is a measurable function.
    \\Recall Proposition 5 to see that for the measurable subset $E_0$ of $E$, $f$ is measurable because $f|_{E_0}$ and $f|_{E\setminus E_0}$ are both measurable functions.
    \item Suppose $f$ is a real-valued function on $\mathbb{R}$ such that $f^{-1}(c)$ is measurable for each number $c$. Is $f$ necessarily measurable?\\
    \\No; let $V\subseteq[0,1]$ be a Vitali set. Therefore $V$ is nonmeasurable (see Ch 2.6).
    Consider the function $f:\mathbb{R}\to\mathbb{R}$, defined as
    \[
        f(x)=
        \begin{cases}
            -e^x&x\in V\\
            e^x&x\notin V\\
        \end{cases}    
    \]
    For any real number $c$, we have
    \[
        f^{-1}(c)=
        \begin{cases}
            \ln(-c)& c<0\\
            \ln(c)& c>0\\
            \emptyset& c=0
        \end{cases}
    \]
    and so $f^{-1}(c)$ is a singleton set or is the empty set, which are measurable, so $f^{-1}(c)$ is measurable.\\
    Now, we know that $e^x:\mathbb{R}\to\mathbb{R}_{>0}$ and so $e^x>0$ for any real number $x$.\\
    Therefore $f(x)=-e^x<0$ only when $x\in V$.
    However, the set $\{x\in\mathbb{R}\ |\ f(x)<0\}=V$ is not measurable, and so $f$ is not a measurable function.
    \item Suppose the function $f$ is defined on a measurable set $E$ and $\{x\in E\ |\ f(x)>c\}$ is a measurable set for each rational number $c$. Is $f$ necessarily a measurable function?\\
    \\Yes. Let $f:E\to\mathbb{R}$ with $E$ a measurable set, and let $\{x\in E\ |\ f(x)>c\}=\{x\in E\ |\ f(x)\in(c,\infty)\}$ be measurable for each $c\in\mathbb{Q}$.\\
    Let $a$ be any real number. Then for any natural number $n$, there exists a rational number $c_n$ such that $a<c_n<a+\frac{1}{n}$, and therefore $\bigcup_{n=1}^\infty(c_n,\infty)=(a,\infty)$.
    Therefore we have
    \begin{align*}
        \{x\in E\ |\ f(x)>a\} &= f^{-1}((a,\infty))\\
        &=f^{-1}(\bigcup_{n=1}^\infty(c_n,\infty))\\
        &=\bigcup_{n=1}^\infty f^{-1}((c_n,\infty))\\
        &=\bigcup_{n=1}^\infty \{x\in E\ |\ f(x)>c_n\},
    \end{align*}
    which is a countable union of measurable sets, and therefore is also measurable.
    \item Let $f$ be a function with measurable domain $D$. Show that $f$ is measurable iff the function $g$ defined on $\mathbb{R}$ by $g(x)=f(x)$ for $x\in D$ and $g(x)=0$ for $x\notin D$ is measurable.\\
    \\Let $D\subseteq\mathbb{R}$ be a measurable set, let $f:D\to\mathbb{R}$, and let $g:\mathbb{R}\to\mathbb{R}$ be defined by
    \[
        g(x)=
        \begin{cases}
            f(x)&x\in D\\
            0&x\notin D
        \end{cases}
    \]
    $(\implies)$ Suppose that $f$ is measurable.\\
    For any real number $c$,
    \[
        \{x\in \mathbb{R}\ |\ g(x)>c\}=
        \begin{cases}
            \{x\in D\ |\ f(x)>c\}&c\ge0\\
            \{x\in D\ |\ f(x)>c\}\cup D^c&c<0\\
        \end{cases}
    \]
    Both of the sets $\{x\in D\ |\ f(x)>c\}$ and $\{x\in D\ |\ f(x)>c\}\cap D^c$ are measurable, so $ \{x\in \mathbb{R}\ |\ g(x)>c\}$ is measurable and thus $g$ is a measurable function.\\
    \\$(\impliedby)$ Suppose that $g$ is measurable.\\
    Recall Proposition 5 (ii) to see that for the measurable subset $D$ of $\mathbb{R}$, $g$ is measurable on $\mathbb{R}$, which implies that the restrictions $g|_{D}$ and $g|_{\mathbb{R}\setminus D}$ are measurable.
    Therefore for any real number $c$,
    \begin{align*}
        \{x\in D\ |\ f(x)>c\}&=\{x\in \mathbb{R}\ |\ g|_{D}(x)>c\}\cap D\text{ is measurable},
    \end{align*}
    and $f$ is measurable.
    \item Let the function $f$ be defined on a measurable set $E$. Show that $f$ is measurable iff for each borel set $A$, $f^{-1}(A)$ is measurable. (Hint: the collection of sets $A$ that have the property that $f^{-1}(A)$ is measurable is a $\sigma$-algebra.)\\
    \\Let $f:E\to\mathbb{R}$, where $E$ is a measurable set.\\
    $(\implies$) Suppose that $f$ is measurable.\\
    Let $\mathcal{M}=\{A\ |\ f^{-1}(A)\text{ is measurable}\}$.\\
    To show that $\mathcal{M}$ is a $\sigma$-algebra, know that the measurable sets is a $\sigma$-algebra.\\
	Observe that:
	\begin{enumerate}[label=(\roman*),align=left]
		\item $f^{-1}(\emptyset)=\emptyset\implies\emptyset\in \mathcal{M}$.
		\item $A\in \mathcal{M}\implies f^{-1}(A)\text{ is measurable }\implies f^{-1}(A)^c=f^{-1}(A^c)\text{ is measurable }\implies A^c\in \mathcal{M}$.
		\item $A_k\in \mathcal{M}\implies f^{-1}(A_k)\text{ is measurable }\implies\bigcup_{k=1}^\infty f^{-1}(A_k)=f^{-1}(\bigcup_{k=1}^\infty A_k)\text{ is measurable }\implies \bigcup_{k=1}^\infty A_k\in \mathcal{M}$.
	\end{enumerate}
    Then because $f$ is measurable, for any real number $a$, the set $f^{-1}((a,\infty))=\{x\in E\ |\ f(x)>a\}$ is measurable.
    Now, $(a,\infty)\in\mathcal{M}$ because $f^{-1}((a,\infty))$ is measurable.
    Because $(a,\infty)$ is a Borel set, all other Borel sets are in $\mathcal{M}$ because the Borel sets are a $\sigma$-algebra.\\
    \\$(\impliedby$) Suppose that for each borel set $A$, the set $f^{-1}(A)=\{x\in E\ |\ f(x)\in A\}$ is measurable.\\
    Every interval of the form $(a,\infty)$ is a borel set, so we have that for any real number $a$, the set $f^{-1}((a,\infty))=\{x\in E\ |\ f(x)>a\}$ is measurable. This is equivalent to the measurability of $f$.
    \item (Borel measurability) A function $f$ is said to be \textbf{Borel measurable} provided its domain $E$ is a Borel set and for each $c$, the set $\{x\in E\ |\ f(x)>c\}$ is a Borel set.
    Verify that Proposition 1 and Theorem 6 remain valid if we replace "(Lebesgue) measurable set" by "Borel set".
    Show that:
    \begin{enumerate}[label=(\roman*),align=left]
        \item every Borel measurable function is Lebesgue measurable,\\
        \\The Borel sets are a subset of the measurable sets.
        Therefore for a Borel measurable function $f$, its domain $E$ is a Borel set (and thus a measurable set), and for each $c$, the set $\{x\in E\ |\ f(x)>c\}$ is a Borel set (and thus a measurable set).
        Thus $f$ is a measurable function.
        \item if $f$ is Borel measurable and $B$ is a Borel set, then $f^{-1}(B)$ is a Borel set,\\
        \\Let $S=\{B\ |\ f^{-1}(B)\text{ is Borel}\}$.\\
        To show that $S$ is a $\sigma$-algebra, know that the Borel sets is a $\sigma$-algebra.\\
        Observe that:
        \begin{enumerate}[label=(\roman*),align=left]
            \item $f^{-1}(\emptyset)=\emptyset\implies\emptyset\in S$.
            \item $B\in S\implies f^{-1}(B)\text{ is Borel }\implies f^{-1}(B)^c=f^{-1}(B^c)\text{ is Borel }\implies B^c\in S$.
            \item $B_k\in S\implies f^{-1}(B_k)\text{ is Borel }\implies\bigcup_{k=1}^\infty f^{-1}(B_k)=f^{-1}(\bigcup_{k=1}^\infty B_k)\text{ is Borel }\implies \bigcup_{k=1}^\infty B_k\in S$.
        \end{enumerate}
        Now, because $f$ is Borel measurable, for any real number $a$, the set $f^{-1}((a,\infty))$ is a Borel set.
        This implies $(a,\infty)\in S$.
        Because $(a,\infty)$ is a Borel set, all other Borel sets are in $S$ because the Borel sets is a $\sigma$-algebra.
        \item if $f$ and $g$ are Borel measurable, so is $f\circ g$,\\
        \\Let $f,g$ be Borel measurable, with $g:E\to F$, and $f:F\to\mathbb{R}$, where $E,F$ are Borel sets. 
        Then $f\circ g:E\to\mathbb{R}$ has a Borel set as its domain.\\
        Recall that $\{x\ |\ f(x)>a\}=\{x\ |\ f(x)\in(a,\infty)\}=f^{-1}((a,\infty))$.
        \begin{align*}
            \{x\ |\ (f\circ g )(x)>a\}&=\{x\ |\ (f\circ g )(x)\in(a,\infty)\}\\
            &=(f\circ g )^{-1}((a,\infty))\\
            &=(g^{-1}\circ f^{-1})((a,\infty))\\
            &=g^{-1}(f^{-1}((a,\infty)))\\
            &=g^{-1}(B)&&\text{ where }B\text{ is Borel}\\
            &=B'.&&\text{ where }B'\text{ is Borel}
        \end{align*}
        Therefore $f\circ g$ is Borel measurable.
        \item if $f$ is Borel measurable and $g$ is Lebesgue measurable, then $f\circ g$ is Lebesgue measurable.\\
        \\Let $f$ be Borel measurable, with $f:F\to\mathbb{R}$, and let $g$ be Lebesgue measurable, with $g:E\to F$, where $F$ is a Borel set and $E$ is a measurable set. 
        Then $f\circ g:E\to\mathbb{R}$ has a measurable set as its domain.
        \begin{align*}
            \{x\ |\ (f\circ g )(x)>a\}&=\{x\ |\ (f\circ g )(x)\in(a,\infty)\}\\
            &=(f\circ g )^{-1}((a,\infty))\\
            &=(g^{-1}\circ f^{-1})((a,\infty))\\
            &=g^{-1}(f^{-1}((a,\infty)))\\
            &=g^{-1}(B)&&\text{ where }B\text{ is Borel}\\
            &=B'.&&\text{ where }B'\text{ is measurable (Problem 7)}
        \end{align*}
        Therefore $f\circ g$ is Lebesgue measurable.
    \end{enumerate}
    \item Let $\{f_n\}$ be a sequence of measurable functions defined on a measurable set $E$.
    Define $E_0$ to be the set of points of $x$ in $E$ at which $\{f_n(x)\}$ converges. Is the set $E_0$ measurable?\\
    \\Let $E_0=\{x\in E\ |\ \{f_n(x)\}\text{ converges}\}\subseteq \{x\in E\ |\ \{f_n(x)\}\text{ is Cauchy}\}$, because all convergent sequences are Cauchy.
    \\Therefore $E_0=\{x\in E\ |\ \forall k\in\mathbb{N},\exists N\in\mathbb{N}\text{ such that }|f_n(x)-f_m(x)|<\frac{1}{k}\text{ for all }n,m\ge N\}$.
    \\This is equivalent to writing 
    \[
        E_0=\bigcap_{k\in\mathbb{N}}\bigcup_{N\in\mathbb{N}}\bigcap_{n,m\ge N}\{x\in E\ |\ |f_n(x)-f_m(x)|<\frac{1}{k}\}.
    \]
    The functions $f_n$ and $f_m$ are measurable, so by Theorem 6, $f_n-f_m$ is also measurable.
    The absolute value function $|\cdot|$ is continuous, so by Proposition 7, the composition $|\cdot|\circ(f_n-f_m)=|f_n-f_m|$ is a measurable function.
    Therefore for the real number $\frac{1}{k}$, the set $\{x\in E\ |\ |f_n(x)-f_m(x)|<\frac{1}{k}\}$ is a measurable set.\\
    Then $E_0$ is a countable union and intersection of measurable sets, so $E_0$ is measurable.
    \item Suppose $f$ and $g$ are real-valued functions defined on all of $\mathbb{R}$, $f$ is measurable, and $g$ is continuous.
    Is the composition $f\circ g$ necessarily measurable?\\
    \\No; let $f$ be measurable, with $f:\mathbb{R}\to\mathbb{R}$, and let $g$ be continuous, with $g:\mathbb{R}\to \mathbb{R}$.
        Then $f\circ g:\mathbb{R}\to\mathbb{R}$.
        \begin{align*}
            \{x\ |\ (f\circ g )(x)>a\}&=\{x\ |\ (f\circ g )(x)\in(a,\infty)\}\\
            &=(f\circ g )^{-1}((a,\infty))\\
            &=(g^{-1}\circ f^{-1})((a,\infty))\\
            &=g^{-1}(f^{-1}((a,\infty)))\\
            &=g^{-1}(A)&&\text{ where }A\text{ is measurable}
        \end{align*}
        Recall Chapter 2 Problem 36 to see that for a continuous function $g$, the set $g^{-1}(A)$ is not always measurable when $A$ is measurable.
    \item Let $f$ be a measurable function and $g$ be a one-to-one function from $\mathbb{R}$ onto $\mathbb{R}$ which has a Lipschitz inverse. Show that the composition $f\circ g$ is measurable. (Hint: examine Problem 37 in Chapter 2.)\\
    \\Let $f$ be measurable, with $f:\mathbb{R}\to\mathbb{R}$, and let $g$ be a bijection from $\mathbb{R}$ to $\mathbb{R}$, where $g^{-1}$ is Lipschitz.
    From Chapter 2 Problem 37, we have that $g^{-1}$ maps a measurable set to a measurable set; that is, for the measurable set $A$, the set $g^{-1}(A)$ is measurable.
    \\We have $f\circ g:\mathbb{R}\to\mathbb{R}$.
    \begin{align*}
        \{x\ |\ (f\circ g )(x)>a\}&=\{x\ |\ (f\circ g )(x)\in(a,\infty)\}\\
        &=(f\circ g )^{-1}((a,\infty))\\
        &=(g^{-1}\circ f^{-1})((a,\infty))\\
        &=g^{-1}(f^{-1}((a,\infty)))\\
        &=g^{-1}(A)&&\text{ where }A\text{ is measurable}\\
        &=A'.&&\text{ where }A'\text{ is measurable (Chapter 2 Problem 37)}
    \end{align*}
    Therefore $f\circ g$ is measurable.
\end{enumerate}

% 3.2
\section{Sequential Pointwise Limits and Simple Approximation}
\begin{flushleft}
    \begin{namedthm*}{Definition}
        For a sequence $\{f_n\}$ of functions with common domain $E$, a function $f$ on $E$, and a subset $A$ of $E$, we say that  
        \begin{enumerate}[label=(\roman*),align=left]
            \item The sequence $\{f_n\}$ converges to $f$ pointwise on $A$ provided
            \[
                \lim_{n\to\infty}f_n(x)=f(x)\text{ for all }x\in A.
            \]
            a.k.a.
            \[
                \forall x\in A, \forall\epsilon>0, \exists N_x\in\mathbb{N}:\forall n\ge N_x, |f_n(x)-f(x)|<\epsilon.
            \]
            \item The sequence $\{f_n\}$ converges to $f$ pointwise a.e. on $A$ provided it converges to $f$ pointwise on $A\setminus B$, where $m(B)=0$.
            \item The sequence $\{f_n\}$ converges to $f$ uniformly on $A$ provided for each $\epsilon>0$, there is an index $N$ for which
            \[
                |f-f_n|<\epsilon\text{ on }A\text{ for all }n\ge N.    
            \]
            a.k.a.
            \[
                \forall\epsilon>0, \exists N\in\mathbb{N}:\forall n\ge N,\forall x\in A, |f_n(x)-f(x)|<\epsilon.
            \]
        \end{enumerate}
    \end{namedthm*}

    \begin{namedthm*}{Theorem}
        Let $A\subseteq\mathbb{R}^n$ and suppose $\{f_i\}$ is a sequence of functions $f_i:A\to\mathbb{R}^m$ such that 
        \begin{enumerate}[label=(\roman*),align=left]
            \item $\{f_i\}\to f$ uniformly on $A$
            \item Each $f_i$ is uniformly continuous on $A$.
        \end{enumerate}
        Then $f:A\to\mathbb{R}^m$ is uniformly continuous on $A$.
    \end{namedthm*}
    \begin{proof}
        Fix $\epsilon>0$.\\
        By (i), there exists an $N\in\mathbb{N}$ such that for all $n\ge N$, for all $x\in A$, $\|f_n(x)-f(x)\|<\epsilon/3$.
        \\Now, fix $N$ and fix $k\ge N$.\\
        By (ii), there exists a $\delta>0$ such that for all $x,y\in A$ with $\|x-y\|<\delta$, then $\|f_k(x)-f_k(y)\|<\epsilon/3$.
        \\Therefore we have
        \begin{align*}
            \|f(x)-f(y)\|&=\|f(x)-f_k(x)+f_k(x)-f_k(y)+f_k(y)-f(y)\|\\
            &\le\|f(x)-f_k(x)\|+\|f_k(x)-f_k(y)\|+\|f_k(y)-f(y)\|\\
            &<\frac{\epsilon}{3}+\frac{\epsilon}{3}+\frac{\epsilon}{3}\\
            &=\epsilon.
        \end{align*}
        Therefore for all $\epsilon>0$, there exists a $\delta>0$ such that for all $x,y \in A$ with $\|x-y\|<\delta$, then $\|f(x)-f(y)\|<\epsilon$.
        Thus $f$ is uniformly continuous on $A$.\\
        A similar proof can show this for a sequence of continuous functions converging uniformly to a continuous function.
    \end{proof}

    \begin{namedthm*}{Example}
        Consider the sequence of continuous functions $\{f_n\}_{n=2}^\infty:[0,1]\to\mathbb{R}$, defined by
        \[ 
		f_n(x) =
            \begin{cases} 
                \frac{n-0}{1/n-0}x& \text{ if } x \in [0,\frac{1}{n}]\\
                \frac{0-n}{2/n-1/n}(x-\frac{1}{n})+n & \text{ if } x \in (\frac{1}{n},\frac{2}{n}]\\
                0& \text{ if } x \in (\frac{2}{n},1]
            \end{cases}
            =
            \begin{cases} 
                n^2x& \text{ if } x \in [0,\frac{1}{n}]\\
                -n^2(x-\frac{1}{n})+n & \text{ if } x \in (\frac{1}{n},\frac{2}{n}]\\
                0& \text{ if } x \in (\frac{2}{n},1]
            \end{cases}
	    \]
        (Each $f_n$ is a triangle-shaped function that achieves its max $f(1/n)=n$ and base corners $f(0)=0$ and $f(2/n)=0$.)\\
        In addition, consider the continuous function $f:[0,1]\to\mathbb{R}$ defined by $f(x)=0$ for all $x\in[0,1]$.\\
        The sequence $\{f_n\}$ converges to $f$ pointwise but not uniformly on $[0,1]$.
        \\To see this, for any $\epsilon>0$, and for any $x\in[0,1]$, there exists an index $N_x\in\mathbb{N}$ such that for all $n\ge N_x$, we have $\frac{2}{n}<x$, that is, $x\in(\frac{2}{n},1]$, and so $|f_n(x)-f(x)|=0-0=0<\epsilon$.\\
        Therefore the sequence converges pointwise.\\
        To see that the sequence does not converge uniformly, we see that there exists an $\epsilon=1>0$ such that for all indices $N$, there exists an $n\ge N$ and the point $1/n$ such that $|f_n(1/n)-f(1/n)|=n-0=n>1$.
    \end{namedthm*}
    The pointwise limit of continuous functions may not be continuous.\\
    The pointwise limit of Riemann integrable functions may not be Riemann integrable.
    \begin{namedthm*}{Proposition 9}
        Let $\{f_n\}$ be a sequence of measurable functions on $E$ that converges pointwise a.e. on $E$ to the function $f$.
        Then $f$ is measurable.
    \end{namedthm*}
    \begin{proof}
        Let $E_0$ be a subset of $E$ such that $\{f_n\}$ converges pointwise to $f$ on $E\setminus E_0$, with $m(E_0)=0$. 
        Because $E_0$ has measure zero, then it is measurable; by Proposition 5, we have that $f$ is measurable iff the restrictions to $E_0$ and $E\setminus E_0$ are measurable.
        By monotonicity of measure, the measure of the set $\{x\in E_0\ |\ f(x)<c\}$ is always zero and thus trivially measurable.
        \\We want to show that $\{x\in E\setminus E_0\ |\ f(x)<c\}$ is measurable to use Proposition 5.
        \\Now, for any point $x\in E\setminus E_0$, we have that
        \[
            f(x)<c\iff\exists n,k\in\mathbb{N}\text{ s.t. }f_j(x)<c-1/n\ \forall j\ge k.    
        \]
        To see why, suppose that $\forall n,k\in\mathbb{N}, \exists j\ge k\text{ s.t. } f_j(x)\ge c-1/n$.
        \\In the case $f_j(x) > c-1/n$, we have $1/n+f_j(x)\ge c$ for all $n$, which implies that $f_j(x)\ge c>f$, a contradiction.
        \\In the case $f_j(x) = c-1/n$, we have that for any $n$, for any indices $k$, there exists an index $j\ge k$ such that $c-f_j(x)=1/n$, but because $f<c$, the convergence $\{f_n\}\to f$ is not possible, a contradiction.
        \\Then we can write 
        \[
            \{x\in E\setminus E_0\ |\ f(x)<c\}=\bigcup_{1\le n,k<\infty} \biggl[\bigcap_{j=k}^\infty\{x\in E\setminus E_0\ |\ f_j(x)<c-1/n\} \biggr],
        \]
        Which is a countable union and intersection of measurable sets, and thus is measurable.
    \end{proof}

    If $A$ is any set, the \textbf{characteristic function} of $A$, $\chi_A$, is the function on $\mathbb{R}$ defined by 
    \[
        \chi_A=
        \begin{cases}
            1&x\in A\\
            0&x\notin A
        \end{cases}  
    \]
    The function $\chi_A$ is measurable iff the set $A$ is measurable:\\
    $(\implies)$ Suppose $\chi_A$ is measurable.\\
    Then for the real number $c=0$, the set $\{x\in\mathbb{R}\ |\ \chi_A(x)>0\}=A$ is measurable.\\
    $(\impliedby)$ Suppose the set $A$ is measurable.\\
    Then for any real number $c$, we have
    \[
        \{x\in\mathbb{R}\ |\ \chi_A(x)\in(c,\infty)\}=
        \begin{cases}
            \emptyset&c\ge1\\
            A& 1>c\ge0\\
            \mathbb{R}&0>c\\
        \end{cases}
    \]
    Each of the sets $\emptyset$, $A$, and $\mathbb{R}$ are measurable, so $\{x\in\mathbb{R}\ |\ \chi_A(x)\in(c,\infty)\}$ is a measurable set and thus $\chi_A$ is a measurable function.\\
    Thus the existence of a nonmeasurable set $E$ implies the existence of a nonmeasurable function $\chi_E$.
        \begin{namedthm*}{Definition}
            A real-valued function $\varphi$ defined on a measurable set $E$ is called \textbf{simple} provided it is measurable and takes only a finite number of values.          
        \end{namedthm*}
        If $\varphi$ is simple, has domain $E$ and takes the distinct values $c_1,\cdots,c_n$, then
        \[
            \varphi=\sum_{k=1}^n c_k\cdot\chi_{E_k}\text{ on }E,\text{ where }E_k=\{x\in E\ |\ \varphi(x)=c_k\}.   
        \]
        (A linear combination of measurable functions)\\
        A simple function is called a \textbf{step function} in the special case that the sets $E_k$ are intervals. 
    \begin{namedthm*}{The Simple Approximation Lemma}
        Let $f$ be a measurable real-valued function on $E$. 
        Assume $f$ is bounded on $E$, that is, there is an $M\ge0$ for which $|f|\le M$ on $E$.
        Then for each $\epsilon>0$, there are simple functions $\varphi_\epsilon$ and $\psi_\epsilon$ defined on $E$ which have the following approximation properties:
        \[
            \varphi_\epsilon\le f\le\psi_\epsilon\text{ and }0\le\psi_\epsilon-\varphi_\epsilon<\epsilon\text{ on }E.    
        \]
    \end{namedthm*}
    \begin{proof}
        Because $f$ is bounded, we have that $f(x)\subseteq[-M,M]$ for all $x\in E$.
        \\Let $(c,d)$ be an open, bounded interval such that $f(E)\subseteq[-M,M]\subseteq(c,d)$, so that $(c,d)$ contains $f(E)$, the image of $E$ under $f$.
        \\Also, choose $\{y_0,\cdots,y_n\}$ to be a partition of the closed, bounded interval $[c,d]$ such that $y_k-y_{k-1}<\epsilon$ for $1\le k\le n$:
        \[
            c=y_0<y_1<\cdots<y_{n-1}<y_n=d
        \]
        Define 
        \[
            I_k=[y_{k-1},y_k)\text{ and }E_k=f^{-1}(I_k)\text{ for }1\le k \le n.
        \]
        Because $f$ is a measurable function and each interval $I_k=[y_{k-1},y_k)$ is Borel, then each set $E_k=f^{-1}(I_k)$ is measurable (see Problem 7).
        Because $E_k$ is measurable, then $\chi_{E_k}$ is a measurable function.
        \\Then we can define the simple functions $\varphi_\epsilon$ and $\psi_\epsilon$ on $E$ by
        \begin{align*}
            \varphi_\epsilon&=\sum_{k=1}^n y_{k-1}\cdot\chi_{E_k}\\
            \psi_\epsilon&=\sum_{k=1}^n y_k\cdot\chi_{E_k}
        \end{align*}
        Now, for any $x\in E$, because $f(E)\subseteq(c,d)$, and $\{[y_{k-1},y_k)\}_{k=1}^n$ is a partition that contains $(c,d)$, there exists a unique $k$, $1\le k\le n$, for which $f(x)\in[y_{k-1},y_k)$, and therefore
        \[
            \varphi_\epsilon(x)=y_{k-1}\le f(x)<y_k=\psi_\epsilon(x).    
        \]
        Also, for each $x\in E$, $\psi_\epsilon(x)-\varphi_\epsilon(x)=y_k-y_{k-1}<\epsilon$ for some $k$.
    \end{proof}
    \begin{namedthm*}{The Simple Approximation Theorem}
        An extended real-valued function $f$ on a measurable set $E$ is measurable iff there is a sequence $\{\varphi_n\}$ of simple functions on $E$ which converges pointwise on $E$ to $f$ and has the property that
        \[
            |\varphi_n|\le|f|\text{ on }E\text{ for all }n.
        \]
        If $f$ is nonnegative, we may choose $\{\varphi_n\}$ to be increasing.
    \end{namedthm*}
    \begin{proof}
        Let $f$ be an extended real-valued function on a measurable set $E$.\\
        $(\impliedby)$ Suppose $\{\varphi_n\}$ is a sequence of simple (and thus measurable) functions on $E$ that converges pointwise on $E$ to $f$ (and has the property that $|\varphi_n|\le|f|$ on $E$ for all $n$).\\
        Then by Proposition 9, $f$ is measurable.\\
        $(\implies)$ Suppose $f$ is measurable.\\
        We can assume $f\ge0$ on $E$ (See Problem 23 for the general case, as $f=f^+-f^-$ on $E$, a linear combination of two nonnegative measurable functions $f^+$ and $f^-$).
        \\For a natural number $n$, define $E_n=\{x\in E\ |\ f(x)\le n\}$. Because $f$ is a measurable function, the set $E_n\subseteq E$ is measurable.
        Then by Proposition 5 (ii), $f|_{E_n}$ is measurable.\\
        Also, $f|_{E_n}$ is bounded because $0\le f|_{E_n}(x)\le n$ for $x\in E_n$.\\
        Now, recall the Simple Approximation Lemma to see that for the measurable real-valued function $f|_{E_n}$ on $E_n$, where $f|_{E_n}$ is bounded on $E_n$, we have that for $\epsilon=1/n$, there exist simple functions $\varphi_n$ and $\psi_n$ defined on $E_n$ such that
        \[
            0\le\varphi_n\le f\le\psi_n\text{ on }E_n\text{ and }0\le\psi_n-\varphi_n<1/n\text{ on }E_n.    
        \]
        Then we can see that
        \[
            0\le\varphi_n\le f\text{ and }0\le f-\varphi_n\le\psi_n-\varphi_n<1/n\text{ on }E_n.    
        \]
        Now, $E=E_n\cup E_n^c= \{x\in E\ |\ f(x)\le n\}\cup\{x\in E\ |\ f(x)> n\}$ and $\varphi_n$ is defined on $E_n$. We can extend $\varphi_n$ to all of $E$ by setting $\varphi_n(x)=n$ on $E_n^c=\{x\in E\ |\ f(x)> n\}$.
        Then $\varphi_n$ is a simple function defined on $E$ and $0\le \varphi_n\le f$ on $E$.
        To see that $\{\varphi_n\}$ converges to $f$ pointwise on $E$:\\
        Consider any $x\in E$.\\
        Case $f(x)<\infty$:\\
        Then there exists a natural number $N$ such that $f(x)\le N$.
        Then for all $n\ge N$, We have that
        \[
            0\le f(x)-\varphi_n(x)<1/n,    
        \]
        and thus $\lim_{n\to\infty}\varphi_n(x)=f(x)$.\\
        Case $f(x)=\infty$:\\
        Then $\varphi_n(x)=n$ because $f(x)> n$ for all $n$, and $\lim_{n\to\infty}\varphi_n(x)=\infty=f(x)$.\\
        By replacing each $\varphi_n$ with $\max\{\varphi_1,\cdots,\varphi_n\}$, we have $\{\varphi_n\}$ increasing.
    \end{proof}
\end{flushleft}
\begin{center}
	\textbf{PROBLEMS}
\end{center}
\begin{enumerate}
	\setcounter{enumi}{11}
    \item Let $f$ be a bounded measurable function on $E$. Show that there are sequences of simple functions on $E$, $\{\varphi_n\}$ and $\{\psi_n\}$, such that $\{\varphi_n\}$ is increasing and $\{\psi_n\}$ is decreasing and each of these sequences converges to $f$ uniformly on $E$.\\
    \\Let $f$ be a bounded measurable function on the measurable set $E$.
    Because $f$ is bounded, there exists a real number $M$ such that $|f|\le M$ and thus $-M\le f\le M$.
    Then for all natural numbers $n\ge M$, the set $E_n=\{x\in E\ |\ f(x)\le n\}=E$.
    By the Simple Approximation Lemma, for $\epsilon=1/n$, there exist the simple function $\varphi_n$ and $\psi_n$ defined on $E_n=E$ such that 
    \[
            \varphi_n\le f\le\psi_n\text{ and }0\le\psi_n-\varphi_n<1/n\text{ on }E.    
    \]
    Then for each $n\ge M$, we have the sequences of functions $\{\varphi_n\}$ and $\{\psi_n\}$ such that
    \begin{align*}
        \varphi_n\le f&\text{ and }f-\varphi_n\le\psi_n-\varphi_n<1/n\text{ on }E,\\  
        f\le\psi_n&\text{ and }\psi_n-f\le\psi_n-\varphi_n<1/n\text{ on }E.    
    \end{align*}
    We can replace each $\varphi_n$ with $\max\{\varphi_1,\cdots,\varphi_n\}$ and each $\psi_n$ with $\min\{\psi_1,\cdots,\psi_n\}$ so that the sequences are increasing and decreasing respectively.
    The convergence is uniform because we showed that for any $\epsilon=1/n$, there exists the natural number $n$ such that for all $n'\ge n$, $f(x)-\varphi_{n'}(x)<1/n$ and $\psi_{n'}(x)-f(x)<1/n$ for all $x\in E$.
    \item A real-valued measurable function is said to be \textit{semisimple} provided it takes only a countable number of values. Let $f$ be any measurable function on $E$.
    Show that there is a sequence of semisimple functions $\{f_n\}$ on $E$ that converges to $f$ uniformly on $E$.\\
    \\We can define $f_n(x)=\frac{1}{n}\lfloor nf(x)\rfloor$ (where the floor function $\lfloor x\rfloor$ returns the largest integer less than or equal to $x$).
    Then because $f_n=\frac{\lfloor nf(x)\rfloor}{n}$ and $\lfloor nf(x)\rfloor$ and $n$ are integers, we have $f_n(E)\subseteq\mathbb{Q}$ which is countable.
    Because the floor function rounds down to the nearest integer, we have
    \[
        |nf(x)-\lfloor nf(x)\rfloor|<1\text{ for all }x\in E,
    \]
    and therefore
    \[
        |f(x)-\frac{1}{n}\lfloor nf(x)\rfloor|<1/n\text{ for all }x\in E.
    \]
    \item Let $f$ be a measurable function on $E$ that is finite a.e. on $E$ and $m(E)<\infty$.
    For each $\epsilon>0$, show that there is a measurable set $F$ contained in $E$ such that $f$ is bounded on $F$ and $m(E\setminus F)<\epsilon$.\\
    \\For each natural number $n$, let $F_n=\{x\in E\ |\ f(x)>n\}$. 
    Because $f$ is measurable, then each $F_n$ is a measurable set. 
    Then $\{F_n\}$ is descending because 
    \[
        F_n=\{x\in E\ |\ f(x)>n\}\supseteq\{x\in E\ |\ f(x)>n+1>n\}=F_{n+1}.
    \]
    Also, $F_1\subseteq E$, so by monotonicity of measure, $m(F_1)\le m(E)<\infty$.
    \\By the continuity of measure, for the descending collection of measurable sets $\{F_n\}$ for which $m(F_1)<\infty$, we have $m(\bigcap_{n=1}^\infty F_n)=\lim_{n\to\infty}m(F_n)$.
    \\Now, 
    \[
        \bigcap_{n=1}^\infty F_n = \bigcap_{n=1}^\infty\{x\in E\ |\ f(x)>n\}=\{x\in E\ |\ f(x)=\infty\},
    \]
    Because $f$ is finite a.e. on $E$, we have 
    \[
        0=m(\{x\in E\ |\ f(x)=\infty\})=m(\bigcap_{n=1}^\infty F_n)=\lim_{n\to\infty}m(F_n).
    \]
    Then for any $\epsilon>0$, there exists an $N\in\mathbb{N}$ such that for all $n\ge N$, we have $m(F_n)<\epsilon$.
    \\If we let $F=E\cap F_n^c=E\setminus F_n \subseteq E$, then $F=\{x\in E\ |\ f(x)\le n\}$, a measurable set, and 
    \[
        E\cap F^c=E\cap [E^c\cup F_n]=[E\cap E^c]\cup[E\cap F_n]=\emptyset\cup F_n=F_n,
    \]
    and thus we have
    \[
        m(E\setminus F)=m(E\cap F^c)=m(F_n)<\epsilon.
    \]
    \item Let $f$ be a measurable function on $E$ that if finite a.e. on $E$ and $m(E)<\infty$. 
    Show that for each $\epsilon>0$, there is a measurable set $F$ contained in $E$ and a sequence $\{\varphi_n\}$ of simple functions on $E$ such that $\{\varphi_n\}\to f$ uniformly on $F$ and $m(E\setminus F)<\epsilon$. (Hint: see the preceding problem.)\\
    \\Because $f$ is finite a.e. on $E$ and $m(E)<\infty$, by the previous problem, for any $\epsilon>0$, there exists a measurable set $F\subseteq E$ such that $f$ is bounded on $F$ and $m(E \setminus F)<\epsilon$.
    Furthermore, because $f$ is bounded and measurable on $F$, by Problem 12, there exists a sequence of simple functions $\{\varphi_n\}$ on $F$ that converges uniformly to $f$ on $F$.
    Each $\varphi_n$ can be extended to $E$ by setting $\varphi_n=n$ on $F^c$.
    \item Let $I$ be a closed, bounded interval and $E$ a measurable subset of $I$. Let $\epsilon>0$.
    Show that there is a step function $h$ on $I$ and a measurable subset $F$ of $I$ for which 
    \[
        h=\chi_E\text{ on }F\text{ and }m(I\setminus F)<\epsilon.    
    \]
    (Hint: use Theorem 12 of Chapter 2.)\\
    \\Theorem 12 of Chapter 2 states:
    \\Let $E$ be a measurable set of finite outer measure.
    Then for each $\epsilon>0$, there is a finite disjoint collection of open intervals $\{I_k\}_{k=1}^n$ for which if $\mathcal{O}=\bigcup_{k=1}^n I_k$, then
    \[
        m^*(E\setminus\mathcal{O})+m^*(\mathcal{O}\setminus E)<\epsilon.    
    \]
    $\cdots$\\
    Fix $\epsilon>0$.\\
    Because $m(E)$ is finite, by definition of infimum, for $\epsilon>0$, there exists a countable collection of open intervals $\{I_k\}_{k=1}^\infty$ whose union $\mathcal{O}=\bigcup_{k=1}^\infty I_k$ covers $E$ and 
    \[
        m(E)\le \sum_{k=1}^\infty \ell(I_k)< m(E)+\epsilon/2,
    \]
    and $\sum_{k=1}^\infty\ell(I_k)$ is finite as well.
    Because $E$ is a measurable subset of finite outer measure that is contained in $\mathcal{O}$, then we can use the excision property and subadditivity to see that
    \begin{equation}
        m(\mathcal{O}\setminus E)=m(\mathcal{O})-m(E)=m(\bigcup_{k=1}^\infty I_k)-m(E)\le\sum_{k=1}^\infty \ell(I_k)-m(E)< \epsilon/2,\tag{1}
    \end{equation}
    Because the series $\sum_{k=1}^\infty\ell(I_k)$ is finite, the sequence of partial sums $\sum_{k=1}^n\ell(I_k)$ converges to $\sum_{k=1}^\infty\ell(I_k)$, so that for $\epsilon/2$, there exists an index $N$ such that for all $n\ge N$, we have
    \begin{align*}
        \sum_{k=1}^\infty\ell(I_k)-\sum_{k=1}^n\ell(I_k) &<\epsilon/2\\
        \sum_{k=n+1}^\infty\ell(I_k)+\sum_{k=1}^n\ell(I_k)-\sum_{k=1}^n\ell(I_k) &<\epsilon/2\\
        \sum_{k=n+1}^\infty\ell(I_k)&<\epsilon/2,
    \end{align*}
    and then we can let $\mathcal{O}'=\bigcup_{k=1}^N I_k$ so that, because $E\cap(\mathcal{O}\setminus\mathcal{O}')\subseteq\mathcal{O}\setminus\mathcal{O}'$,
    \[
        m(E\cap(\mathcal{O}\setminus\mathcal{O}'))\le m(\mathcal{O}\setminus\mathcal{O}')=m(\bigcup_{k=1}^\infty I_k\setminus\bigcup_{k=1}^N I_k)=m(\bigcup_{k=N+1}^\infty I_k)\le\sum_{k=N+1}^\infty\ell(I_k)<\epsilon/2.
    \]
    Because $E\subseteq \mathcal{O}$, then $E\cap \mathcal{O}^c=\emptyset$, and we can derive
    \begin{align*}
        E\cap\mathcal{O}'&=[\emptyset\cup(E\cap\mathcal{O}')]\\
        &=[(E\cap\mathcal{O}^c)\cup(E\cap\mathcal{O}')]\\
        &=\emptyset\cup[(E\cap\mathcal{O}^c)\cup(E\cap\mathcal{O}')]\\
        &=[E\cap E^c]\cup[E\cap(\mathcal{O}^c\cup\mathcal{O}')]\\
        &=E\cap[E^c\cup(\mathcal{O}^c\cup\mathcal{O}')]\\
        &=E\cap[E^c\cup(\mathcal{O}\cap\mathcal{O}'^c)^c]\\
        &=E\cap[E\cap(\mathcal{O}\setminus\mathcal{O}')]^c\\
        &=E\setminus[E\cap(\mathcal{O}\setminus\mathcal{O}')].
    \end{align*}
    And then we see that by excision,
    \begin{equation}
        m(E\cap\mathcal{O}')=m(E\setminus[E\cap(\mathcal{O}\setminus\mathcal{O}')])=m(E)-m(E\cap(\mathcal{O}\setminus\mathcal{O}'))>m(E)-\epsilon/2.\tag{2}
    \end{equation}
    We will let $F=(E\cap\mathcal{O}')\cup(I\setminus\mathcal{O})$ so that 
    \begin{align*}
        I\setminus F&=I\cap F^c\\
        &=I\cap[(E\cap\mathcal{O}')\cup(I\setminus\mathcal{O})]^c\\
        &=I\cap[(I^c\cup\mathcal{O})\cap(E\cap\mathcal{O}')^c]\\
        &=[I\cap(I^c\cup\mathcal{O})]\cap(E\cap\mathcal{O}')^c\\
        &=[(I\cap I^c)\cup(I\cap\mathcal{O})]\cap(E\cap\mathcal{O}')^c\\
        &=[I\cap\mathcal{O}]\setminus(E\cap\mathcal{O}')\\
        &\subseteq\mathcal{O}\setminus[E\cap\mathcal{O}'].
    \end{align*}
    Therefore we can write
    \begin{align*}
        m(I\setminus F)&\le m(\mathcal{O}\setminus[E\cap\mathcal{O}'])&&\text{by monotonicity}\\
        &=m(\mathcal{O})-m(E\cap\mathcal{O}')&&\text{by excision}\\
        &<m(\mathcal{O})-m(E)+\epsilon/2&&\text{by (2)}\\
        &<\epsilon/2+\epsilon/2&&\text{by (1)}\\\
        &<\epsilon.
    \end{align*}
    We can let $h=\sum_{k=1}^n \chi_{J_k}=\chi_{\mathcal{O}'}$, with $J_k=I_k\setminus\bigcup_{j=1}^{k-1}I_j$, where each $J_k$ is a finite union of disjoint intervals, and so $h$ is a step function.
    Then $h=1$ on $E\cap\mathcal{O}'$ and $h=0$ on $I\setminus\mathcal{O}$, so that $h=\chi_E$ on $F$.
    \item Let $I$ be a closed, bounded interval and $\psi$ a simple function defined on $I$. Let $\epsilon>0$.
    Show that there is a step function $h$ on $I$ and a measurable subset $F$ of $I$ for which 
    \[
        h=\psi\text{ on }F\text{ and }m(I\setminus F)<\epsilon.    
    \] 
    (Hint: use the fact that a simple function is a linear combination of characteristic functions and the preceding problem.)\\
    \\We have
    \[
        \psi=\sum_{k=1}^n c_k\cdot\chi_{E_k}\text{ on }I,\text{ where }E_k=\{x\in I\ |\ \psi(x)=c_k\}\text{ measurable}.  
    \]
    For each $k=1,\cdots,n$, we have that $I$ is a closed, bounded interval and $E_k$ a measurable subset of $I$.
    For $\epsilon>0$, by the previous Problem 16, there is a step function $h_k$ on $I$ and a measurable subset $F_k$ of $I$ for which 
    \[
        h_k=\chi_{E_k}\text{ on }F_k\text{ and }m(I\setminus F_k)<\epsilon/n.    
    \]
    We can let $h=\sum_{k=1}^n c_k h_k$ and $F=\bigcap_{k=1}^n F_k$ so that $h=\psi$ on $F$ and $I\cap F^c=I\cap \bigcup_{k=1}^n F_k^c=\bigcup_{k=1}^n (I\cap F_k^c)$ which gives us
    \[
        m(I\setminus F)=m(I\cap F^c)=m(\bigcup_{k=1}^n (I\cap F_k^c))\le \sum_{k=1}^n m(I\cap F_k^c)=\sum_{k=1}^n m(I\setminus F_k)<\epsilon.
    \]
    \item Let $I$ be a closed, bounded interval and $f$ a bounded measurable function defined on $I$. Let $\epsilon>0$.
    Show that there is a step function $h$ on $I$ and a measurable subset $F$ of $I$ for which 
    \[
        |h-f|<\epsilon\text{ on }F\text{ and }m(I\setminus F)<\epsilon.    
    \]
    Because $f$ is bounded and measurable on $I$, by Problem 12, there exists a sequence of simple functions $\{\psi_n\}$ on $I$ that converges uniformly to $f$ on $I$.
    Then for any $\epsilon$, we can choose $\psi\in\{\psi_n\}$ such that $|\psi-f|<\epsilon$.
    By the previous Problem 17, there is a step function $h$ on $I$ and a measurable subset $F$ of $I$ for which 
    \[
        h=\psi\text{ on }F\text{ and }m(I\setminus F)<\epsilon.    
    \] 
    Therefore we have $|h-f|<\epsilon$.
    \item Show that the sum and product of two simple functions are simple as are the max and the min.\\
    \\Consider two simple functions $\varphi$ and $\psi$ on the measurable set $E$, with
    \[
        \varphi=\sum_{k=1}^n c_k\cdot\chi_{E_k}\text{ on }E,\text{ where }E_k=\{x\in E\ |\ \varphi(x)=c_k\}\text{ measurable}.  
    \]
    \[
        \psi=\sum_{k'=1}^m c'_{k'}\cdot\chi_{E'_{k'}}\text{ on }E,\text{ where }E'_{k'}=\{x\in E\ |\ \psi(x)=c'_{k'}\}\text{ measurable}.  
    \]
    Then for any $x\in E$, there exists an $i\in \{1,\cdots,n\}$ and $j\in \{1,\cdots,m\}$ such that
    \[
        (\varphi+\psi)(x)= c_i+c'_j\text{ on }E_i\cap E'_j,
    \]
    The intersection of measurable sets $E_i\cap E'_j$ is also measurable, so the function $\chi_{E_i\cap E'_j}$ is measurable.\\ 
    That is, we have the simple function
    \[
        \varphi+\psi=\sum_{k=1}^n \sum_{k'=1}^m (c_k+c_{k'})\cdot\chi_{E_k\cap E'_{k'}}\text{ on }E.
    \]
    Similarly for the product, we have
    \[
        \varphi\cdot\psi=\sum_{k=1}^n \sum_{k'=1}^m (c_k\cdot c_{k'})\cdot\chi_{E_k\cap E'_{k'}}\text{ on }E.
    \]
    We can recall Chapter 1 Problem 49 to see how we define max and min:
    \begin{align*}
        \max\{\varphi,\psi\}&=\frac{1}{2}(\varphi+\psi+|\varphi-\psi|),\\
        \min\{\varphi,\psi\}&=\frac{1}{2}(\varphi+\psi-|\varphi-\psi|).
    \end{align*}
    Clearly scaling a simple function is simple, and the absolute value of a simple function is simple, and we showed that the sum of simple functions is simple, and therefore the max and min of simple functions is simple.
    \item Let $A,B$ be any sets. Show that
    \begin{align*}
        \chi_{A\cap B}&=\chi_A\cdot\chi_B\\
        \chi_{A\cup B}&=\chi_A+\chi_B-\chi_A\cdot\chi_B\\
        \chi_{A^c}&=1-\chi_A
    \end{align*}
    \\We can use DeMorgan's laws to see that
    \[
        \chi_{A\cap B}=
        \begin{cases}
            1&x\in A\cap B\\
            0&x\notin A\cap B
        \end{cases}  
        =
        \begin{cases}
            1&x\in A\cap B\\
            0&x\in A^c\cup B^c
        \end{cases}
        =
        \begin{cases}
            1&x\in A \text{ and }x\in B\\
            0&x\notin A\text{ or }x\notin B
        \end{cases} 
    \]
    Then for any $x\in\mathbb{R}$, we have
    \begin{center}
        \begin{tabular}{|c c c c|} 
        \hline
        $\chi_A(x)$  & $\chi_B(x)$ & $\chi_{A\cap B}(x)$ & $\chi_A(x)\cdot\chi_B(x)$ \\ [0.5ex] 
        \hline\hline
        0 & 0 & 0 & $0\cdot0=0$  \\ 
        \hline
        0 & 1 & 0 & $0\cdot1=0$   \\
        \hline
        1 & 0 & 0 & $1\cdot0=0$   \\
        \hline
        1 & 1 & 1 & $1\cdot1=1$ \\[1ex] 
        \hline
        \end{tabular}
    \end{center}
    Similarly see that
    \[
        \chi_{A\cup B}=
        \begin{cases}
            1&x\in A\cup B\\
            0&x\notin A\cup B
        \end{cases}  
        =
        \begin{cases}
            1&x\in A\cup B\\
            0&x\in A^c\cap B^c
        \end{cases}
        =
        \begin{cases}
            1&x\in A \text{ or }x\in B\\
            0&x\notin A\text{ and }x\notin B
        \end{cases} 
    \]
    Then for any $x\in\mathbb{R}$, we have
    \begin{center}
        \begin{tabular}{|c c c c|} 
        \hline
        $\chi_A(x)$  & $\chi_B(x)$ & $\chi_{A\cup B}(x)$ & $\chi_A(x)+\chi_B(x)-\chi_A(x)\cdot\chi_B(x)$ \\ [0.5ex] 
        \hline\hline
        0 & 0 & 0 & $0+0-0\cdot0=0$   \\ 
        \hline
        0 & 1 & 1 & $0+1-0\cdot1=1$ \\
        \hline
        1 & 0 & 1 & $1+0-1\cdot0=1$  \\
        \hline
        1 & 1 & 1 & $1+1-1\cdot1=1$ \\[1ex] 
        \hline
        \end{tabular}
    \end{center}
    Finally see that
    \[
        \chi_{A^c}=
        \begin{cases}
            1&x\in A^c\\
            0&x\notin A^c
        \end{cases}  
        =
        \begin{cases}
            1&x\notin A\\
            0&x\in A
        \end{cases} 
        =
        \begin{cases}
            1-0&x\notin A\\
            1-1&x\in A
        \end{cases} 
        =
        1-
        \begin{cases}
            0&x\notin A\\
            1&x\in A
        \end{cases} 
        =
        1-\chi_A.
    \]
    That is, for any $x\in\mathbb{R}$, we have
    \begin{center}
        \begin{tabular}{|c c c|} 
        \hline
        $\chi_{A}(x)$ & $\chi_{A^c}(x)$ & 1-$\chi_{A}(x)$ \\ [0.5ex] 
        \hline\hline
        1 & 0 & $1-1=0$  \\ 
        \hline
        0 & 1 & $1-0=1$ \\[1ex] 
        \hline
        \end{tabular}
    \end{center}
    \item For a sequence $\{f_n\}$ of measurable functions with common domain $E$, show that each of the following functions is measurable:
    \begin{itemize}
        \item $\inf\{f_n\}$\\
        \\We have 
        \[
            \inf\{f_n\} = \{x\in E\ |\ \inf\{f_n\}> c\} = \bigcap_{n=1}^\infty\{x\in E\ |\ f_n(x)>c\}
        \]
        or
        \[
            \inf\{f_n\} = \{x\in E\ |\ \inf\{f_n\}< c\} = \bigcup_{n=1}^\infty\{x\in E\ |\ f_n(x)<c\}
        \]
        \item $\sup\{f_n\}$\\
        \\Similarly,
        \[
            \sup\{f_n\} = \{x\in E\ |\ \sup\{f_n\}> c\} = \bigcup_{n=1}^\infty\{x\in E\ |\ f_n(x)>c\}
        \]
        or
        \[
            \sup\{f_n\} = \{x\in E\ |\ \sup\{f_n\}< c\} = \bigcap_{n=1}^\infty\{x\in E\ |\ f_n(x)<c\}
        \]
        \item $\lim\inf\{f_n\}$
        \item $\lim\sup\{f_n\}$
    \end{itemize}
    \item (Dini's Theorem) Let $\{f_n\}$ be an increasing sequence of continuous functions on $[a,b]$ which converges pointwise on $[a,b]$ to the continuous function $f$ on $[a,b]$.
    Show that the convergence is uniform on $[a,b]$. (Hint: let $\epsilon>0$. For each natural number $n$, define $E_n=\{x\in[a,b]\ |\ f(x)-f_n(x)<\epsilon\}$. Show that $\{E_n\}$ is an open cover of $[a,b]$ and use the Heine-Borel Theorem.)\\
    \\Let $\epsilon>0$. For each natural number $n$, define 
    \begin{align*}
        E_n&=\{x\in[a,b]\ |\ f(x)-f_n(x)<\epsilon\}\\
        &=\{x\in[a,b]\ |\ f(x)-f_n(x)\in(-\infty,\epsilon)\}\\
        &=(f-f_n)^{-1}((-\infty,\epsilon)).
    \end{align*}
    The sum and product of continuous functions is continuous, so the function $f-f_n$ is continuous and therefore $E_n=(f-f_n)^{-1}((-\infty,\epsilon))$ is an open set.
    \\Because $\{f_n\}$ converges pointwise to $f$ on $[a,b]$, for any $\epsilon>0$, for any $x\in[a,b]$, there exists an index $N_x\in\mathbb{N}$ such that for all $n\ge N_x$, we have $|f(x)-f_n(x)|<\epsilon$.
    \\This means that for any $x\in[a,b]$, there exists an index $N_x\in\mathbb{N}$ such that $x\in\{x\in[a,b]\ |\ f(x)-f_{N_x}(x)<\epsilon\}=E_{N_x}$, and so $x\in\bigcup_{n\in\mathbb{N}} E_n$, which implies
    \[
        \forall x\in[a,b],\exists N_x\in\mathbb{N}\text{ s.t. }x\in\{x\in[a,b]\ |\ |f(x)-f_{N_x}(x)|<\epsilon\}=E_{N_x}\implies x\in\bigcup_{n\in\mathbb{N}} E_n,
    \]
    and by definition of subset, $[a,b]\subseteq \bigcup_{n\in\mathbb{N}} E_n$.
    Now, $\{E_n\}$ is an open cover of $[a,b]$ because it is a union of open sets $E_n$ and it covers $[a,b]$.
    Because $[a,b]$ is compact, there exists a finite subcover $\{E_{n_k}\}_{k=1}^m\subseteq\{E_n\}$. 
    \\This means that for any $x\in[a,b]$, there exists the index $k\in\{1,\cdots,m\}$ such that $x\in E_{n_k}=\{x\in[a,b]\ |\ |f(x)-f_{n_k}(x)|<\epsilon\}$.
    \\Then we can let $N_0=\max\{n_1,\cdots,n_m\}$.
    \\Therefore for any $\epsilon>0$, there exists the index $N_0$ such that for all $n\ge N_0\ge n_i$, $i\in\{1,\cdots,m\}$,
    \[
        |f(x)-f_n(x)|<\epsilon\text{ for all }x\in[a,b].
    \]
    Thus we have uniform convergence.
    \item Express a measurable function as the difference of nonnegative measurable functions and thereby prove the general Simple Approximation Theorem based on the special case of a nonnegative measurable function.\\
    \\Let $f$ be a measurable function on $E$, and we have $f=f^+-f^-$ on $E$, a linear combination of the two nonnegative measurable functions $f^+=\max\{f,0\}\ge0$ and $f^-=\max\{-f,0\}\ge0$.
    In our proof for the Simple Approximation Theorem, we proved the case for $f^+\ge0$ and $f^-\ge0$ that there exist sequences $\{\varphi^+_n\}$ and $\{\varphi^-_n\}$ of simple functions on $E$ that converge pointwise on $E$ to $f^+$ and $f^-$ respectively, and 
    \begin{align*}
        0\le\varphi^+_n&\le f^+\text{ on }E\text{ for all }n,\\
        0\le\varphi^-_n&\le f^-\text{ on }E\text{ for all }n.
    \end{align*}
    then
    \begin{align*}
        0\ge-\varphi^-_n&\ge -f^-\text{ on }E\text{ for all }n,
    \end{align*}
    so that $-f^-\le-\varphi^-_n\le f^-\implies -\varphi^-_n \le |f^-|$.
    \\We have the sets $E^+=\{x\in E\ |\ f(x)\ge0\}$ and $E^-=\{x\in E\ |\ f(x)\le0\}$, so that
    \[
        f(x)=|f(x)|=
        \begin{cases}
            0&\text{ if }x\in E^+\cap E^-\\
            f^+(x)&\text{ if }x\in E^+\text{ only}\\
            f^-(x)&\text{ if }x\in E^-\text{ only}
        \end{cases}    
    \]
    The function $\varphi^+_n-\varphi^-_n$ is simple and
    \[
        0\le\varphi^+_n\le f^+=0, 0\le\varphi^-_n\le f^-=0\text{ on }E^+\cap E^-\implies\varphi^+_n,\varphi^-_n =0,
    \]
    and so we have
    \[
        (\varphi^+_n-\varphi^-_n)(x)=
        \begin{cases}
            0\le f(x)&\text{ if }x\in E^+\cap E^-\\
            \varphi^+_n(x)\le f^+(x)&\text{ if }x\in E^+\text{ only}\\
            -\varphi^-_n(x)\le f^-(x)&\text{ if }x\in E^-\text{ only}
        \end{cases}    
    \]
    Then clearly $\varphi^+_n-\varphi^-_n$ converges pointwise to $f$ on $E$, and therefore we have
    \begin{align*}
        |\varphi^+_n-\varphi^-_n|&\le|f|\text{ on }E\text{ for all }n.
    \end{align*}
    \item Let $I$ be an interval and $f:I\to\mathbb{R}$ be increasing. Show that $f$ is measurable by first showing that, for each natural number $n$, the strictly increasing function $x\mapsto f(x)+x/n$ is measurable, and then taking pointwise limits.\\
    \\Let $f_n(x)=f(x)+x/n$.\\
    Then each $f_n$ is strictly increasing; that is, for $x,y\in I$, we have $f_n(x)<f_n(y)\iff x<y$. 
    This also tells us that each $f_n$ is injective because 
    \[
        x\neq y \implies x<y\text{ or }y<x \implies f_n(x)<f_n(y)\text{ or }f_n(y)<f_n(x)\implies f_n(x)\neq f_n(y).
    \]
    So for any element $x^*\in dom(f_n)$, we know that $f_n^{-1}(x^*)$ consists of a single element at most.
    \\Now, for any two $x^*,y^*\in(a,\infty)$, by definition of interval, any point $z^*$ between them is also in $(a,\infty)$. 
    \\We then have $x^*<z^*<y^* \iff f_n^{-1}(x^*)<f_n^{-1}(z^*)<f_n^{-1}(y^*)$, and so the set $f_n^{-1}((a,\infty))=\{x\in I\ |\ f_n(x)\in(a,\infty)\}$ is an interval for all $a$, and every interval is measurable, so $f_n$ is measurable.
    \\To see that $\{f_n\}$ converges pointwise to $f$, let $\epsilon>0$ and consider $x\in I$. 
    Then there exists an index $N$ such that for all $n\ge N$, 
    \[
        |f_n(x)-f(x)|=|x/n|<\epsilon.    
    \]
    Now, because $\{f_n\}$ is a sequence of measurable functions in $I$ that converges pointwise a.e. on $I$ to $f$, then by Proposition 9, $f$ is measurable.
\end{enumerate}

% 3.3
\section{Littlewood's Three Principles, Ergoff's Theorem, and Lusin's Theorem}
\begin{flushleft}
    \begin{namedthm*}{Egoroff's Theorem}
        Assume $E$ has finite measure. Let $\{f_n\}$ be a sequence of measurable functions on $E$ that converges pointwise on $E$ to the real-valued function $f$.
        Then for each $\epsilon>0$, there is a closed set $F$ contained in $E$ for which     
        \[
            \{f_n\}\to f\text{ uniformly on }F\text{ and }m(E\setminus F)<\epsilon.    
        \]
    \end{namedthm*}
    \begin{namedthm*}{Lemma 10}
        Under the assumptions of Egoroff's Theorem, for each $\eta>0$ and $\delta>0$, there is a measurable subset $A$ of $E$ and an index $N$ for which
        \[
            |f_n-f|<\eta\text{ on }A\text{ for all }n\ge N\text{ and }m(E\setminus A)<\delta.    
        \]
    \end{namedthm*}
    \begin{proof}
        First, we can see that because $\{f_n\}$ is a sequence of measurable functions on $E$ that converges pointwise on $E$ to $f$, by Proposition 9, $f$ is measurable.
        Then by Theorem 6, the function $f_n-f$ is measurable.
        Finally by Proposition 7, considering the continuous function $|\cdot|$ and the measurable function $f_n-f$, the composition $|f_n-f|$ is measurable.
        \\This means that the set $\{x\in E\ |\ |f_n-f|<\eta\}$ is measurable.
        \\Then we see that
        \[
            E_n=\{x\in E\ |\ |f_k-f|<\eta\text{ for all }k\ge n\}=\bigcap_{k=n}^\infty\{x\in E\ |\ |f_k-f|<\eta\},
        \]
        is also measurable.
        \\Then $\{E_n\}$ is an ascending sequence of measurable sets because 
        \[
            E_n=\{x\in E\ |\ |f_n-f|<\eta\}\cap\biggl[\bigcap_{k={n+1}}^\infty\{x\in E\ |\ |f_k-f|<\eta\}\biggr]\subseteq \bigcap_{k={n+1}}^\infty\{x\in E\ |\ |f_k-f|<\eta\}=E_{n+1}.
        \]
        Also, $E=\bigcup_{n=1}^\infty E_n$, because $\{f_n\}$ converges pointwise to $f$ on $E$. 
        \\That is, for $\eta>0$, for $x\in E$, there exists an index $N\in\mathbb{N}$ such that
        \[
            |f_k(x)-f(x)|<\eta\text{ for all }k\ge N,
        \]
        and thus $x\in E_N$.
        \\Now, by continuity of measure, we have
        \[
            m(E)=m(\bigcup_{n=1}^\infty E_n)=\lim_{n\to\infty}m(E_n).
        \]
        Then because $m(E)<\infty$, there exists an index $N_0$ for which $m(E)-m(E_{N_0})<\delta$.
        \\Define $A=E_{N_0}$ so we can use excision to see that
        \[
            m(E\setminus A)=m(E)-m(E_{N_0})<\delta,
        \]
        and 
        \[
            A=\{x\in E\ |\ |f_k-f|<\eta\text{ for all }k\ge N_0\}.
        \]
    \end{proof}
    \begin{proof}
        To prove Egoroff's Theorem:\\
        Assume $E$ has finite measure. Let $\{f_n\}$ be a sequence of measurable functions on $E$ that converges pointwise on $E$ to the real-valued function $f$.
        \\For each natural number $n$, we can let $\eta=1/n$ and $\delta=\epsilon/2^{n+1}$.
        \\By Lemma 10, there exists a subset $A_n$ of $E$ and an index $N_n$ for which
        \[
            |f_k-f|<1/n\text{ on }A_n\text{ for all }k\ge N_n\text{ and }m(E\setminus A_n)<\epsilon/2^{n+1}.    
        \]
        We define 
        \[
            A=\bigcap_{n=1}^\infty A_n.
        \]
        Then we see that
        \begin{align*}
            m(E\setminus A)&=m(E\setminus \biggl[\bigcap_{n=1}^\infty A_n\biggr])\\
            &=m(E\cap \biggl[\bigcup_{n=1}^\infty A_n^c\biggr])\\
            &=m(\bigcup_{n=1}^\infty [E\cap A_n^c])\\
            &\le \sum_{n=1}^\infty m(E\cap A_n^c)\\
            &< \sum_{n=1}^\infty \epsilon/2^{n+1}\\
            &= \frac{1}{2} \sum_{n=1}^\infty \epsilon/2^n\\
            &=\epsilon/2.
        \end{align*}
        To see that $\{f_n\}$ converges uniformly to $f$ on $A$:
        \\Let $\epsilon>0$. Then there exists an index $n_0$ such that $1/n_0<\epsilon$ and 
        \[
            |f_k-f|<1/n_0<\epsilon\text{ on }A\subseteq A_{n_0}\text{ for all }k\ge N_{n_0}.
        \]
        Finally we can use Chapter 2 Theorem 11 to choose a closed set $F$ contained in $A$ for which $m(A\setminus F)<\epsilon/2$.
        Then we have
        \begin{align*}
            m(E\setminus F)&=m(E)-m(F)&&\text{by excision}\\
            &= m(E)-m(A)+m(A)-m(F)\\
            &=m(E\setminus A)+m(A\setminus F)&&\text{by excision}\\
            &<\epsilon/2+\epsilon/2\\
            &=\epsilon.
        \end{align*}
        Therefore $\{f_n\}\to f$ uniformly on $F$ and $m(E\setminus F)<\epsilon$.
    \end{proof}
    \begin{namedthm*}{Proposition 11}
        Let $f$ be a simple function defined on $E$.
        Then for each $\epsilon>0$, there is a continuous function $g$ on $\mathbb{R}$ and a closed set $F$ contained in $E$ for which
        \[
            f=g\text{ on }F\text{ and }m(E\setminus F)<\epsilon.    
        \]
    \end{namedthm*}
    \begin{namedthm*}{Lusin's Theorem}
        Let $f$ be a real-valued measurable function on $E$.
        Then for each $\epsilon>0$, there is a continuous function $g$ on $\mathbb{R}$ and a closed set $F$ contained in $E$ for which 
        \[
            f=g\text{ on }F\text{ and }m(E\setminus F)<\epsilon.
        \]
    \end{namedthm*}
    \begin{proof}
        Consider the case that $m(E)<\infty$.
        \\The Simple Approximation Theorem tells us that there exists a sequence of simple functions $\{f_n\}$ defined on $E$ that converges to $f$ pointwise on $E$.
        \\By the previous Proposition 11, for each $n$, there exists a continuous function $g_n$ on $\mathbb{R}$ and a closed set $F_n$ contained in $E$ for which
        \[
            f_n=g_n\text{ on }F_n\text{ and }m(E\setminus F_n)<\epsilon/2^{n+1}.    
        \]
        Also, Egoroff's Theorem states that because $E$ has finite measure and that $\{f_n\}$ is a sequence of simple (measurable) function on $E$ that converges pointwise on $E$ to the real-valued function $f$, we have that there exists a closed set $F_0$ contained in $E$ such that
        \[
            \{f_n\}\to f\text{ uniformly on }F_0\text{ and }m(E\setminus F_0)<\epsilon/2.    
        \]
        We define $F=\bigcap_{n=0}^\infty F_n$, which is closed because it is an intersection of closed sets.
        Then we see that
        \begin{align*}
            m(E\setminus F)&=m(E\cap\biggr[\bigcup_{n=0}^\infty F_n^c\biggl])\\
            &=m(\bigcup_{n=0}^\infty [E\cap F_n^c])\\
            &=m([E\cap F_0^c]\cup\bigcup_{n=1}^\infty [E\cap F_n^c])\\
            &=m([E\setminus F_0]\cup\bigcup_{n=1}^\infty [E\setminus F_n])\\
            &\le m([E\setminus F_0])+\sum_{n=1}^\infty m( E\setminus F_n)\\
            &<\epsilon/2+\sum_{n=1}^\infty\epsilon/2^{n+1}\\
            &=\epsilon.
        \end{align*}
        Each $f_n$ is continuous on $F$ since $F\subseteq F_n$ and $f_n=g_n$ on $F_n$.
        \\Also, $\{f_n\}$ converges to $f$ uniformly on $F$ since $F\subseteq F_0$. 
        \\Then we can use the fact that because $\{f_n\}$ is a sequence of continuous functions on $F$ that converges uniformly on $F$ to $f$, then $f$ is continuous on $F$ as well.
        \\Finally see problem 25 to see that there exists a continuous function $g$ that extends $f$ to all of $\mathbb{R}$.
        Then $f=g$ on $F$ and $m(E\setminus F)<\epsilon$.
    \end{proof}
\end{flushleft}
\begin{center}
	\textbf{PROBLEMS}
\end{center}
\begin{enumerate}
	\setcounter{enumi}{24}
    \item Suppose $f$ is a function that is continuous on a closed set $F$ of real numbers. Show that $f$ has a continuous extension to all of $\mathbb{R}$. This is a special case of the forthcoming Tietze Extension Theorem.
    (Hint: express $\mathbb{R}\setminus F$ as the union of a countable disjoint collection of open intervals and define $f$ to be linear on the closure of each of these intervals.)\\
    \\(See Chapter 1 Problem 47.)
    \\Let $f$ be a function that is continuous on the closed set $F$.
    Consider the open set $F^c$.
    By Chapter 1 Proposition 9, this open  $F^c$ is the union of a countable, disjoint collection of intervals.
    \\In the case that $(-\infty,a)$ [or $(a,\infty)$] is in $F^c$, then $a \in F$ and $f(a)$ is defined.
	Simply let $f(x)=f(a)$ be the constant function on $(-\infty,a)$ [or $(a,\infty)$].
	\\In the case that $(a,b) \in F^c$, then $a,b\in F$ and $f(a)$,$f(b)$ are defined.
	Let 
	\[
		f(x)=\frac{f(b)-f(a)}{b-a}(x-a)+f(a)\text{ on } (a,b).
	\]
    Then we see that the extension of $f$ is continuous on $\mathbb{R}$.
    \item For the function $f$ and the set $F$ in the statement of Lusin's Theorem, show that the restriction of $f$ to $F$ is a continuous function.
    Must there be any points at which $f$, considered as a function of $E$, is continuous?\\
    \\See the proof for Lusin's Theorem; because $\{f_n\}$ is a sequence of continuous functions on $F$ that converges uniformly on $F$ to $f$, then $f$ is continuous on $F$ as well.
    \item Show that the conclusion of Egoroff's Theorem can fail if we drop the assumption that the domain has finite measure.\\
    \\Going back to the proof for Egoroff's Theorem, we see that we used the excision property:
    \begin{align*}
        m(E\setminus F)&=m(E)-m(F)&&\text{by excision}\\
        &= m(E)-m(A)+m(A)-m(F)\\
        &=m(E\setminus A)+m(A\setminus F)&&\text{by excision}\\
        &<\epsilon/2+\epsilon/2\\
        &=\epsilon.
    \end{align*}
    The excision property requires that
    \begin{align*}
        m(E\setminus F)&=m(E)-m(F)&&\text{if }m(F)<\infty\text{ and }F\subseteq E\\
        m(E\setminus A)&=m(E)-m(A)&&\text{if }m(A)<\infty\text{ and }A\subseteq E\\
        m(A\setminus F)&=m(A)-m(F)&&\text{if }m(F)<\infty\text{ and }F\subseteq A
    \end{align*}
    Specifically, we needed that $m(A),m(F)<\infty$. This was only possible because we assumed $m(E)<\infty$.
    \item Show that Egoroff's Theorem continues to hold if the convergence is pointwise a.e. and $f$ is finite a.e.\\
    \\In Lemma 10 on the way to proving Egoroff's Theorem, we used Proposition 9, which only requires the convergence to be pointwise a.e.
    \\In Lemma 10 we also used Theorem 6, which only requires $f_n$ and $f$ to be finite a.e. 
    \item Prove the extension of Lusin's Theorem to the case that $E$ has infinite measure.\\
    \\We needed to assume that $E$ had finite measure because we used Egoroff's Theorem in the proof for Lusin's Theorem, which requires finite measure (Problem 27).
    \item Prove the extension of Lusin's Theorem to the case that $f$ is not necessarily real-valued, but is finite a.e.\\
    \\We needed to assume that $f$ was real valued because we used Egoroff's Theorem in the proof for Lusin's Theorem, which requires $f$ to be real-valued.
    However, we showed that Egoroff's Theorem continues to hold if $f$ is finite a.e. (Problem 28).
    \item Let $\{f_n\}$ be a sequence of measurable functions on $E$ that converges to the real-valued $f$ pointwise on $E$. 
    Show that $E=\bigcup_{k=1}^\infty E_k$, where for each index $k$, $E_k$ is measurable, and $\{f_n\}$ converges uniformly to $f$ on each $E_k$ if $k>1$, and $m(E_1)=0$.\\
    \\Use Egoroff's Theorem.
\end{enumerate}
% Chapter 4
\chapter{Lebesgue Integration}
% 4.1
\section{The Riemann Integral}
\begin{flushleft}
    In this chapter the Lebesgue integral is defined in four stages:
    \begin{enumerate}
        \setcounter{enumi}{-1}
        \item (define the Riemann integral for bounded functions on a closed, bounded interval):\\\bigskip
        For a bounded real-valued function $f$ defined on the closed, bounded interval $[a,b]$, define the Riemann integral of $f$ over $[a,b]$ by
        \[
            (R)\int_a^bf=\sup\biggl\{(R)\int_a^b\varphi\ |\ \varphi\text{ step, }\varphi\le f\text{ on }[a,b]\biggr\}=\inf\biggl\{(R)\int_a^b\psi\ |\ \psi\text{ step, }\psi\ge f\text{ on }[a,b]\biggr\},
        \]
        where the Riemann integral of a step function is defined as
        \[
            (R)\int_E\psi=\sum_{k=1}^n c_k\cdot (c_k-c_{k-1}).
        \]
        \item define the (Lebesgue) integral for simple functions over a set of finite measure:\\\bigskip
        For a (measurable) simple function $\psi$ defined on a set of finite measure $E$, we define the integral of $\psi$ over $E$ by
        \[
            \int_E\psi=\sum_{i=1}^n a_i\cdot m(E_i).
        \]
        \item define the (Lebesgue) integral for bounded measurable functions $f$ over a set of finite measure, in terms of integrals of upper and lower approximations of $f$ by simple functions.\\\bigskip
        For a bounded measurable function $f$ defined on a set of finite measure $E$, we define the integral of $f$ over $E$ by
        \[
            \int_Ef=\sup\biggl\{\int_E\varphi\ |\ \varphi\text{ simple, }\varphi\le f\text{ on }E\biggr\}=\inf\biggl\{\int_E\psi\ |\ \psi\text{ simple, }\psi\ge f\text{ on }E\biggr\}.
        \]
        \item define the (Lebesgue) integral of a general nonnegative measurable function $f$ over $E$ to be the supremum of the integrals of lower approximations of $f$ by bounded measurable functions that vanish outside a set of finite measure;
        the integral of such a function is nonnegative, but may be infinite.\\\bigskip
        For a nonnegative measurable function $f$ on $E$, we define the integral of $f$ over $E$ by
        \[
            \int_Ef=\sup\biggl\{\int_Eh\ |\ h\text{ bounded, measurable, of finite support and }0\le h\le f\text{ on }E\biggr\}.
        \]
        \item define a general measurable function to be (Lebesgue) integrable over $E$ provided $\int_E |f|<\infty$.
    \end{enumerate}
    \bigskip \textbf{The Construction of the Riemann integral:}\\\bigskip
    Let $f$ be a bounded real-valued function defined on the closed, bounded interval $[a,b]$.
    Let $P=\{x_0,x_1,\cdots,x_n\}$ be a partition of $[a,b]$, that is,
    \[
        a=x_0<x_1<\cdots<x_n=b.
    \]
    Define the \textbf{lower and upper Darboux sums} for $f$ with respect to $P$, respectively, by
    \[
        L(f,P)=\sum_{i=1}^n m_i\cdot (x_i-x_{i-1})
    \]
    and
    \[
        U(f,P)=\sum_{i=1}^n M_i\cdot (x_i-x_{i-1}),
    \]
    where, for $1\le i\le n$,
    \[
        m_i=\inf\{f(x)\ |\ x_{i-1}<x<x_i\}\text{ and }M_i=\sup\{f(x)\ |\ x_{i-1}<x<x_i\}.
    \]
    We then define the \textbf{lower and upper Riemann integrals} of $f$ over $[a,b]$, respectively, by
    \[
        (R)\underline\int_a^bf=\sup\biggl\{L(f,P)\ |\ P\text{ a partition of }[a,b]\biggr\}
    \]
    and
    \[
        (R)\overline\int_a^bf=\inf\biggl\{U(f,P)\ |\ P\text{ a partition of }[a,b]\biggr\}.
    \]
    If the upper and lower integrals are equal we say that $f$ is \textbf{Riemann integrable} over $[a,b]$ and call this common value the Riemann integral of $f$ over $[a,b]$:
    \[
        (R)\int_a^bf
    \]
    A real-valued function $\psi$ defined on $[a,b]$ is called a \textbf{step function} provided there is a partition $P=\{x_0,x_1,\cdots,x_n\}$ of $[a,b]$ and numbers $c_1,\cdots,c_n$ such that for $1\le i\le n$,
    \[
        \psi(x)=c_i\text{ if }x_{i-1}<x<x_i.
    \]
    Clearly a step function is Riemann integrable:
    \[
        \sum_{i=1}^n c_i\cdot (x_i-x_{i-1})=L(\psi,P)=U(\psi,P)=(R)\int_a^b\psi
    \]
    Then we can reformulate the definition of the lower and upper Riemann integrals:
    \[
        (R)\underline\int_a^bf=\sup\biggl\{(R)\int_a^b\varphi\ |\ \varphi\text{ a step function and }\varphi\le f\text{ on }[a,b]\biggr\}
    \]
    and
    \[
        (R)\overline\int_a^bf=\inf\biggl\{(R)\int_a^b\psi\ |\ \psi\text{ a step function and }\varphi\ge f\text{ on }[a,b]\biggr\}.
    \]\\\bigskip
    \textbf{Example (Dirichlet's Function)}
    Define $f:[0,1]\to\mathbb{R}$ such that
    \[
        f(x)=
        \begin{cases}
            1&x\in\mathbb{Q}\\
            0&x\notin\mathbb{Q}
        \end{cases}    
    \] 
    Let $P$ be any partition of $[0,1]$.
    By the density of the rationals and the irrationals, for any open interval $(x_{i-1},x_i)$ generated by $P$, there exists both a rational $r$ and an irrational $s$ so that $f(r)=1$ and $f(s)=0$ and so $m_i=\inf\{f(x)\ |\ x_{i-1}<x<x_i\}\le f(r),f(s)\le\sup\{f(x)\ |\ x_{i-1}<x<x_i\}=M_i$, thus
    \[
        L(f,P)=0\text{ and }U(f,P)=1.
    \]
    Therefore
    \[
        (R)\underline\int_0^1f=0<1=(R)\overline\int_0^1f,
    \]
    so $f$ is not Riemann integrable.
    \\Consider the enumeration of the rationals in $[0,1]$: $\{q_k\}_{k=1}^\infty$.
    We can define a sequence of functions $f_n:[0,1]\to\mathbb{R}$ in the following way:
    \[
        f_n(x)=
        \begin{cases}
            1&x\in\{q_1,\cdots,q_n\}\\
            0&\text{else}
        \end{cases}
    \]
    Each $f_n$ is a step function (See any partition of the form $0=q_1<\cdots<q_n<\cdots<q_2=1$) and thus is Riemann integrable, and $\{f_n\}$ is an increasing sequence of Riemann integrable functions on $[0,1]$,
    \[
        |f_n|\le1\text{ on }[0,1]\text{ for all }n,
    \]
    and 
    \[
        \{f_n\}\to f\text{ pointwise on }[0,1].
    \]
    To see this, let $\epsilon>0$ and let $x\in[0,1]$. 
    Then $x$ is rational or irrational.
    \\If $x$ is irrational, then $|f_n(x)-f(x)|=|0-0|=0<\epsilon$.
    \\If $x$ is rational, there exists an index $N$ such that $x=q_N$ and for all $n\ge N$, we have $|f_n(q_N)-f(q_N)|=|1-1|=0<\epsilon$.
    \\Thus we have an increasing sequence of Riemann integrable functions on $[0,1]$ that converges pointwise to a function that is not Riemann integrable.
\end{flushleft}
\begin{center}
	\textbf{PROBLEMS}
\end{center}
\begin{enumerate}
	\setcounter{enumi}{0}
    \item Show that, in the above Dirichlet function example, $\{f_n\}$ fails to converge to $f$ uniformly on $[0,1]$.\\
    \\Let $\epsilon=1/2$. Then for any natural number $n$ we choose, there exists $n+1\ge n$ and $q_{n+1}\in[0,1]\cap\mathbb{Q}$ such that $|f_n(q_{n+1})-f(q_{n+1})|=|0-1|=1>1/2$.
    Therefore uniform convergence fails.
    \item A partition $P'$ of $[a,b]$ is called a refinement of a partition $P$ provided each partition point of $P$ is also a partition point of $P'$.
    For a bounded function $f$ on $[a,b]$, show that under refinement lower Darboux sums increase and upper Darboux sums decrease.\\
    \\(Ex: the partition $P'=\{a,b,c\}$ is a refinement of $P=\{a,c\}$.)
    \\Let $P=\{x_0,\cdots,x_n\}$ be any partition.
    Consider $P'$ to be a refinement of $P$ (and suppose $P'\neq P$). 
    Then for some $k\in\{1,\cdots,n\}$, there exists a point $y\in P'$ such that $x_{k-1}<y<x_k$.
    \\Now, we have 
    \begin{align*}
        \{f(x)\ |\ x_{k-1}<x<x_k\}&\supseteq\{f(x)\ |\ x_{k-1}<x<y\}\\
        \{f(x)\ |\ x_{k-1}<x<x_k\}&\supseteq\{f(x)\ |\ y<x<x_k\}
    \end{align*}
    so that
    \begin{align*}
        m_k=\inf\{f(x)\ |\ x_{k-1}<x<x_k\}&\le \inf\{f(x)\ |\ x_{k-1}<x<y\}:=m_k^l\\
        m_k=\inf\{f(x)\ |\ x_{k-1}<x<x_k\}&\le \inf\{f(x)\ |\ y<x<x_k\}:=m_k^r
    \end{align*}
    and 
    \begin{align*}
        M_k=\sup\{f(x)\ |\ x_{k-1}<x<x_k\}&\ge \sup\{f(x)\ |\ x_{k-1}<x<y\}:=M_k^l\\
        M_k=\sup\{f(x)\ |\ x_{k-1}<x<x_k\}&\ge \sup\{f(x)\ |\ x_{k-1}<x<y\}:=M_k^r
    \end{align*}
    and finally recall that the lower and upper Darboux sums with respect to $P$ are defined
    \begin{align*}
        L(f,P)=\sum_{i=1}^n m_i\cdot (x_i-x_{i-1})\\
        U(f,P)=\sum_{i=1}^n M_i\cdot (x_i-x_{i-1})
    \end{align*}
    so that at the index $k$, we have
    \begin{align*}
        m_k\cdot (x_k-x_{k-1})=m_k\cdot (y-x_{k-1})+m_k\cdot (x_k-y)\le m_k^l\cdot (y-x_{k-1})+m_k^r\cdot (x_k-y)\\
        M_k\cdot (x_k-x_{k-1})=M_k\cdot (y-x_{k-1})+M_k\cdot (x_k-y)\ge M_k^l\cdot (y-x_{k-1})+M_k^r\cdot (x_k-y)
    \end{align*}
    and then clearly the lower and upper Darboux sums $L(f,P'),U(f,P')$ with respect to $P'$ are such that
    \begin{align*}
        L(f,P)\le L(f,P'),\\
        U(f,P)\ge U(f,P').
    \end{align*}
    That is, the lower Darboux sum of any refinement is an increase, and the upper Darboux sum of any refinement is a decrease.
    \item Use the preceding problem to show that for a bounded function on a closed, bounded interval, each lower Darboux sum is no greater than each upper Darboux sum.
    From this conclude that the lower Riemann integral is no greater than the upper Riemann integral.\\
    \\Let $f$ be a bounded function on a closed, bounded interval $[a,b]$.
    \\Let $P=\{x_0,\cdots,x_n\}$ be any partition of $[a,b]$.
    \\Then for all $k\in\{1,\cdots,n\}$,
    \begin{align*}
        m_k=\inf\{f(x)\ |\ x_{k-1}<x<x_k\}&\le \sup\{f(x)\ |\ x_{k-1}<x<x_k\}=M_k,
    \end{align*}
    and therefore 
    \begin{equation}
        L(f,P)=\sum_{i=1}^n m_i\cdot (x_i-x_{i-1})\le\sum_{i=1}^n M_i\cdot (x_i-x_{i-1})=U(f,P).\tag{1}
    \end{equation}
    Then we show that the following holds:
    \[
        (R)\underline\int_a^bf=\sup\biggl\{L(f,P)\ |\ P\text{ a partition of }[a,b]\biggr\}\le \inf\biggl\{U(f,P)\ |\ P\text{ a partition of }[a,b]\biggr\}=(R)\overline\int_a^bf
    \]
    Suppose by contradiction that there exists partitions $P,B$ such that 
    \[
        \sup\biggl\{L(f,P)\ |\ P\text{ a partition of }[a,b]\biggr\}\ge L(f,P)>U(f,B)\ge \inf\biggl\{U(f,P)\ |\ P\text{ a partition of }[a,b]\biggr\}
    \]
    Then $P\cup B$ is a refinement of both $P$ and $B$, and so by the preceding Problem 2,
    \[
        L(f,P\cup B)\ge L(f,P)>U(f,B)\ge U(f,P\cup B).
    \]
    Furthermore, by (1),
    \[
        U(f,P\cup B)\ge L(f,P\cup B)\ge L(f,P)>U(f,B)\ge U(f,P\cup B),
    \]
    and we reach a contradiction.
    \item Suppose the bounded function $f$ on $[a,b]$ is Riemann integrable over $[a,b]$.
    Show that there is a sequence $\{P_n\}$ of partitions of $[a,b]$ for which $\lim_{n\to\infty}[U(f,P_n)-L(f,P_n)]=0$.\\
    \\Because $f$ is Riemann integrable, we have
    \[
        (R)\underline\int_a^bf=\sup\biggl\{L(f,P)\ |\ P\text{ a partition of }[a,b]\biggr\}=\inf\biggl\{U(f,P)\ |\ P\text{ a partition of }[a,b]\biggr\}=(R)\overline\int_a^bf.
    \]
    For each natural number $n$, let $\epsilon=1/2n$ so that, by definition of supremum and infimum, there exists partitions $P_n$ and $B_n$ such that 
    \[
        \left[(R)\underline\int_a^bf\right]-1/2n<L(f,P_n) \le\left[(R)\underline\int_a^bf\right]=(R)\int_a^bf=\left[(R)\overline\int_a^bf\right]\le U(f,B_n)<\left[(R)\overline\int_a^bf\right]+1/2n.
    \]
    Furthermore, because $P_n\cup B_n$ is a refinement of both $P_n$ and $B_n$, we have
    \[
        \left[(R)\int_a^bf\right]-1/2n<L(f,P_n)\le L(f,P_n\cup B_n) \le(R)\int_a^bf\le U(f,P_n\cup B_n)\le U(f,B_n)<\left[(R)\int_a^bf\right]+1/2n,
    \]
    and thus for each $n$, we have $U(f,P_n\cup B_n)-L(f,P_n\cup B_n)<1/n$.
    \\Therefore for the sequence $\{P_n\cup B_n\}$ of partitions of $[a,b]$, for any $\epsilon$, there exists an index $N$ such that for all $n\ge N$, then $U(f,P_n\cup B_n)-L(f,P_n\cup B_n)<1/n\le 1/N<\epsilon$.
    \item Let $f$ be a bounded function on $[a,b]$. Suppose there is a sequence $\{P_n\}$ of partitions of $[a,b]$ for which $\lim_{n\to\infty}[U(f,P_n)-L(f,P_n)]=0$. Show that $f$ is Riemann integrable over $[a,b]$.\\
    \\We say that there exists a sequence $\{P_n\}$ of partitions of $[a,b]$ such that for any $\epsilon$, there exists an index $N$ such that for all $n\ge N$, then $U(f,P_n)-L(f,P_n)<\epsilon$.
    \\In Problem 3 we showed that 
    \[
        (R)\underline\int_a^bf=\sup\biggl\{L(f,P)\ |\ P\text{ a partition of }[a,b]\biggr\}\le \inf\biggl\{U(f,P)\ |\ P\text{ a partition of }[a,b]\biggr\}=(R)\overline\int_a^bf,
    \]
    and so we have
    \[
        L(f,P_n)\le (R)\underline\int_a^bf\le (R)\overline\int_a^bf\le U(f,P_n).
    \]
    Then for any $\epsilon$, we have that $(R)\overline\int_a^bf-(R)\underline\int_a^bf<\epsilon$, and thus $(R)\overline\int_a^bf\le(R)\underline\int_a^bf$ so that $(R)\underline\int_a^bf=(R)\overline\int_a^bf$ and $f$ is Riemann integrable.
    \item Use the preceding problem to show that since a continuous function $f$ on a closed, bounded interval $[a,b]$ is uniformly continuous on $[a,b]$, it is Riemann integrable over $[a,b]$.\\
    \\(Review Chapter 1 Theorem 23 for the proof that a continuous function on a compact set is uniformly continuous.)
    \\Let $f$ be a continuous function on $[a,b]$. Then $f$ is uniformly continuous and bounded.
    That means that for any $\epsilon>0$, there exists a $\delta>0$ such that for $x,y\in[a,b]$ with $|x-y|<\delta$, then $|f(x)-f(y)|<\epsilon$.
    Therefore for each $\epsilon$, we can create a partition $P_\delta=\{x_0,\cdots,x_n\}$ of $[a,b]$ such that for any $k\in\{1,\cdots,n\}$, we have, for the interval $(x_{k-1},x_k)$, that $x_k-x_{k-1}<\delta$.
    Then for any $x,y\in(x_{k-1},x_k)$, we have $|f(x)-f(y)|<\epsilon/n$, and therefore for each $k$,
    \begin{align*}
        M_k-m_k=\sup\{f(x)\ |\ x_{k-1}<x<x_k\}-\inf\{f(x)\ |\ x_{k-1}<x<x_k\}<\epsilon/n,
    \end{align*}
    and thus
    \[
        U(f,P_\delta)-L(f,P_\delta)=\sum_{i=1}^n M_i\cdot (x_i-x_{i-1})-\sum_{i=1}^n m_i\cdot (x_i-x_{i-1})<\epsilon.
    \]
    This means that, for each natural number $m$, setting $\epsilon=1/m$, we can construct a partition $P_{\delta_m}$ such that $U(f,P_{\delta_m})-L(f,P_{\delta_m})<1/m$, and therefore $\lim_{m\to\infty}[U(f,P_{\delta_m})-L(f,P_{\delta_m})]=0$, and so by the preceding Problem 5, we have that $f$ is Riemann integrable.
    \item Let $f$ be an increasing real-valued function on $[0,1]$. For a natural number $n$, define $P_n$ to be the partition of $[0,1]$ into $n$ subintervals of length $1/n$. 
    Show that $U(f,P_n)-L(f,P_n)\le 1/n[f(1)-f(0)]$. Use Problem 5 to show that $f$ is Riemann integrable over $[0,1]$.\\
    \\Because $f$ is real-valued and increasing on $[0,1]$, we know $f$ is bounded and $0\le f(1)-f(0)<\infty$.
    Because $f$ is increasing, then for each $k$, we have
    \begin{align*}
        f(x_{k-1})\le\inf\{f(x)\ |\ x_{k-1}<x<x_k\}=m_k\le M_k=\sup\{f(x)\ |\ x_{k-1}<x<x_k\}\le f(x_k),
    \end{align*}
    so that
    \[
        M_k-m_k\le f(x_k)-f(x_{k-1}).
    \]
    Then we see that
    \begin{align*}
        U(f,P_n)-L(f,P_n)&=\sum_{i=1}^n M_i\cdot (x_i-x_{i-1})-\sum_{i=1}^n m_i\cdot (x_i-x_{i-1})\\
        &=\sum_{i=1}^n (M_i-m_i)\cdot (x_i-x_{i-1})\\
        &=\sum_{i=1}^n (M_i-m_i)\cdot (1/n)\\
        &=1/n\sum_{i=1}^n (M_i-m_i)\\
        &\le1/n \sum_{i=1}^n f(x_i)-f(x_{i-1})\\
        &=1/n [f(x_n)-f(x_0)]\\
        &=1/n [f(1)-f(0)].
    \end{align*}
    Then because we just proved that for each natural number $n$, we have $U(f,P_n)-L(f,P_n)\le 1/n[f(1)-f(0)]$, then $\lim_{n\to\infty}[U(f,P_n)-L(f,P_n)]=0$, so by Problem 5, $f$ is Riemann integrable.
    \item Let $\{f_n\}$ be a sequence of bounded functions that converges uniformly to $f$ on the closed, bounded interval $[a,b]$. 
    If each $f_n$ is Riemann integrable over $[a,b]$, show that $f$ also is Riemann integrable over $[a,b]$. Is it true that
    \[
        \lim_{n\to\infty}\int_a^bf_n=\int_a^bf?  
    \]
    For any partition $P=\{x_0,\cdots,x_\ell\}$, we have 
    \begin{align*}
        U(f,P)-L(f,P)&=U(f,P)-U(f_n,P)+U(f_n,P)-L(f_n,P)+L(f_n,P)-L(f,P).
    \end{align*}
    Uniform convergence means that for any $\epsilon>0$, then there exists an index $N$ such that for all $n\ge N$, then $|f(x)-f_n(x)|<\frac{\epsilon}{3\ell}$ for all $x\in[a,b]$.
    \\We see that for any $x\in[a,b]$,
    \[
        -\frac{\epsilon}{3\ell}<f(x)-f_n(x)<\frac{\epsilon}{3\ell}.
    \]
    \\Then for any $k\in\{1,\cdots,\ell\}$, we see that
    \begin{align*}
        M_k=\sup\{f(x)\ |\ x_{k-1}<x<x_k\}&=\sup\{f(x)-f_n(x)+f_n(x)\ |\ x_{k-1}<x<x_k\}\\
        &<\sup\{\frac{\epsilon}{3\ell}+f_n(x)\ |\ x_{k-1}<x<x_k\}\\
        &=\frac{\epsilon}{3\ell}+\sup\{f_n(x)\ |\ x_{k-1}<x<x_k\}\\
        &=\frac{\epsilon}{3\ell}+(M_n)_k
    \end{align*}
    and 
    \begin{align*}
        m_k=\inf\{f(x)\ |\ x_{k-1}<x<x_k\}&=\inf\{f(x)-f_n(x)+f_n(x)\ |\ x_{k-1}<x<x_k\}\\
        &>\inf\{-\frac{\epsilon}{3\ell}+f_n(x)\ |\ x_{k-1}<x<x_k\}\\
        &=-\frac{\epsilon}{3\ell}+\inf\{f_n(x)\ |\ x_{k-1}<x<x_k\}\\
        &=-\frac{\epsilon}{3\ell}+(m_n)_k
    \end{align*}
    Then the Darboux sums are such that
    \begin{align*}
        U(f,P)=\sum_{k=1}^\ell M_k(x_k-x_{k-1})&<\sum_{k=1}^\ell\frac{\epsilon}{3\ell}+\sum_{k=1}^\ell(M_n)_k(x_k-x_{k-1})\\
        &=\frac{\epsilon}{3}+U(f_n,P)
    \end{align*}
    and 
    \begin{align*}
        L(f,P)=\sum_{k=1}^\ell m_k(x_k-x_{k-1})&>-\sum_{k=1}^\ell\frac{\epsilon}{3\ell}+\sum_{k=1}^\ell(m_n)_k(x_k-x_{k-1})\\
        &=-\frac{\epsilon}{3}+L(f_n,P)
    \end{align*}
    Therefore we have
    \begin{align*}
        U(f,P)-L(f,P)&=U(f,P)-U(f_n,P)+U(f_n,P)-L(f_n,P)+L(f_n,P)-L(f,P)\\
        &=[U(f,P)-U(f_n,P)]+[U(f_n,P)-L(f_n,P)]+[L(f_n,P)-L(f,P)]\\
        &<\frac{\epsilon}{3}+[U(f_n,P)-L(f_n,P)]+\frac{\epsilon}{3}.
    \end{align*}
    Because each $f_n$ is bounded and Riemann integrable over $[a,b]$, then by Problem 4, we proved there is a sequence $\{P_m\}_{m=1}^\infty$ of partitions of $[a,b]$ s.t. $\lim_{m\to\infty}[U(f_n,P_m)-L(f_n,P_m)]=0$.
    Therefore for any $\epsilon>0$, there exists an index $N$ such that $U(f,P_m)-U(f_n,P_m)<\frac{\epsilon}{3}$ for $m\ge N$.
    \begin{align*}
        U(f,P_m)-L(f,P_m)&<\frac{\epsilon}{3}+[U(f_n,P_m)-L(f_n,P_m)]+\frac{\epsilon}{3}\\
        &<\frac{\epsilon}{3}+\frac{\epsilon}{3}+\frac{\epsilon}{3}\\
        &=\epsilon.
    \end{align*}
    Thus because there is a sequence $\{P_m\}_{m=1}^\infty$ of partitions of $[a,b]$ s.t. $\lim_{m\to\infty}[U(f,P_m)-L(f,P_m)]=0$, by Problem 5, $f$ is Riemann integrable.
    \\Also, yes, it is true that $\lim_{n\to\infty}\int_a^bf_n=\int_a^bf$.
\end{enumerate}

% 4.2
\section{The Lebesgue Integral of a Bounded Measurable Function over a Set of Finite Measure}
\begin{flushleft}
    Recall the definition of a simple function:
    If $\psi$ is simple, has domain $E$ and takes the distinct values $a_1,\cdots,a_n$, then
    \begin{equation}
        \psi=\sum_{k=1}^n a_i\cdot\chi_{E_i}\text{ on }E,\text{ where }E_i=\psi^{-1}(a_i)=\{x\in E\ |\ \psi(x)=a_i\}.\tag{1}   
    \end{equation}
    The canonical representation is characterized by the $E_i$'s being disjoint and the $a_i$'s being distinct.
    \begin{namedthm*}{Definition}
        For a simple function $\psi$ defined on a set of finite measure $E$, we define the integral of $\psi$ over $E$ by
        \[
            \int_E\psi=\sum_{i=1}^n a_i\cdot m(E_i),
        \]
        where $\psi$ has the canonical representation given by (1).
    \end{namedthm*}
    Let $f$ be a bounded real-valued function defined on a set of finite measure $E$.
    We define the \textbf{lower and upper Lebesgue integral}, respectively, of $f$ over $E$ to begin
    \[
        \sup\biggl\{\int_E\varphi\ |\ \varphi\text{ simple and }\varphi\le f\text{ on }E\biggr\},
    \]
    and
    \[
        \inf\biggl\{\int_E\psi\ |\ \psi\text{ simple and }\varphi\ge f\text{ on }E\biggr\}.
    \]
    \begin{namedthm*}{Definition}
        A bounded function $f$ on a domain $E$ of finite measure is said to be \textbf{Lebesgue measurable} over $E$ provided its upper and lower Lebesgue integrals over $E$ are equal.
        The common value of the upper and lower integrals is called the \textbf{Lebesgue integral}, or simply the integral, of $f$ over $E$ and is denoted by $\int_Ef$.
    \end{namedthm*}
    \begin{namedthm*}{Theorem 3}    
        Let $f$ be a bounded function defined on the closed, bounded interval $[a,b]$.
        If $f$ is Riemann integrable over $[a,b]$, then it is Lebesgue integrable over $[a,b]$ and the two integrals are equal.
    \end{namedthm*}
    \begin{proof}
        Saying that $f$ is Riemann integrable means that
        \[
            (R)\int_Ef=\sup\biggl\{(R)\int_a^b\varphi\ |\ \varphi\text{ step and }\varphi\le f\text{ on }[a,b]\biggr\}=\inf\biggl\{(R)\int_a^b\psi\ |\ \psi\text{ step and }\varphi\ge f\text{ on }[a,b]\biggr\}.
        \]
        The Riemann integral over a closed, bounded interval of a step function agrees with the Lebesgue integral.
        Therefore because all step functions are simple functions we have 
        \begin{align*}
            \biggl\{(R)\int_a^b\varphi\ |\ \varphi\text{ step and }\varphi\le f\text{ on }[a,b]\biggr\}
            &\subseteq \biggl\{\int_E\varphi\ |\ \varphi\text{ simple and }\varphi\le f\text{ on }E\biggr\}\\
            \biggl\{(R)\int_a^b\psi\ |\ \psi\text{ step and }\varphi\ge f\text{ on }[a,b]\biggr\}
            &\subseteq\biggl\{\int_E\psi\ |\ \psi\text{ simple and }\varphi\ge f\text{ on }E\biggr\}
        \end{align*}
        and thus
        \begin{align*}
            \sup\biggl\{(R)\int_a^b\varphi\ |\ \varphi\text{ step and }\varphi\le f\text{ on }[a,b]\biggr\}
            &\le \sup\biggl\{\int_E\varphi\ |\ \varphi\text{ simple and }\varphi\le f\text{ on }E\biggr\}\\
            \inf\biggl\{(R)\int_a^b\psi\ |\ \psi\text{ step and }\varphi\ge f\text{ on }[a,b]\biggr\}
            &\ge\inf\biggl\{\int_E\psi\ |\ \psi\text{ simple and }\varphi\ge f\text{ on }E\biggr\}
        \end{align*}
        Thus we can write
        \begin{align*}
            (R)\int_Ef&= \sup\biggl\{(R)\int_a^b\varphi\ |\ \varphi\text{ step and }\varphi\le f\text{ on }[a,b]\biggr\}\\
            &\le\sup\biggl\{\int_E\varphi\ |\ \varphi\text{ simple and }\varphi\le f\text{ on }E\biggr\}\\
            &\le\inf\biggl\{\int_E\psi\ |\ \psi\text{ simple and }\varphi\ge f\text{ on }E\biggr\}\\
            &\le\inf\biggl\{(R)\int_a^b\psi\ |\ \psi\text{ step and }\varphi\ge f\text{ on }[a,b]\biggr\}\\
            &=(R)\int_Ef,
        \end{align*}
        and $f$ is Lebesgue integrable with $\int_Ef=(R)\int_Ef$.
        \\\bigskip
        \textbf{Example} The Dirichlet function is a simple function because 
        \[
        f(x)=
        \begin{cases}
            1&x\in\mathbb{Q}\\
            0&x\notin\mathbb{Q}
        \end{cases}    
        \] 
        so that 
        \[
            f(x)=1\cdot\chi_{\mathbb{Q}\cap[0,1]}+0\cdot\chi_{\mathbb{Q}^c\cap[0,1]}=1\cdot\chi_{\mathbb{Q}\cap[0,1]}\text{ on }[0,1],
        \]
        where
        \begin{align*}
            \mathbb{Q}\cap[0,1]&=f^{-1}(1)=\{x\in [0,1]\ |\ f(x)=1\},\\
            \mathbb{Q}^c\cap[0,1]&=f^{-1}(0)=\{x\in [0,1]\ |\ f(x)=0\}.
        \end{align*}
        Then $f$ is Lebesgue integrable (but not Riemann integrable) with
        \[
            \int_{[0,1]}f=1\cdot m(\mathbb{Q}\cap[0,1])+0\cdot m(\mathbb{Q}^c\cap[0,1])=1\cdot0+0\cdot1=0.
        \]
    \end{proof}
    \begin{namedthm*}{Theorem 4}    
        Let $f$ be a bounded measurable function on a set of finite measure $E$.
        Then $f$ is integrable over $E$.
    \end{namedthm*}
    \begin{proof}
        For each natural number $n$, by the Simple Approximation Lemma, for $\epsilon=1/n>0$, there are two simple functions $\varphi_n$ and $\psi_n$ on $E$ for which
        \[
            \varphi_n\le f\le\psi_n\text{ and }0\le\psi_n-\varphi_n\le1/n\text{ on }E.    
        \]
        By the monotonicity of the integral for simple functions,
        \[
            0\le\int_E[\psi_n-\varphi_n]\le1/n\cdot m(E).       
        \]
        Then by linearity of the integral for simple functions,
        \[
            0\le\int_E\psi_n-\int_E\varphi_n\le1/n\cdot m(E).       
        \]
        Then 
        \begin{align*}
            0&\le\inf\biggl\{\int_E\psi\ |\ \psi\text{ simple, }\psi\ge f\biggr\}-\sup\biggl\{\int_E\varphi\ |\ \varphi\text{ simple, }\varphi\le f\biggr\}\\
            &\le\int_E\psi_n-\int_E\varphi_n\\
            &\le1/n\cdot m(E).       
        \end{align*}
        Then $\inf\{\int_E\psi\ |\ \psi\text{ simple, }\psi\ge f\}=\sup\{\int_E\varphi\ |\ \varphi\text{ simple, }\varphi\ge f\}$ and thus $f$ is integrable over $E$.
    \end{proof}
    $\cdots$\\\bigskip
    \begin{namedthm*}{Corollary 6}  
        Let $f$ be a bounded measurable function on a set of finite measure $E$.
        Suppose $A$ and $B$ are disjoint measurable subsets of $E$. 
        Then
        \[
            \int_{A\cup B}f=\int_Af+\int_Bf.
        \]   
    \end{namedthm*}
    \begin{proof}
        Both $f\cdot\chi_A$ and $f\cdot\chi_B$ are bounded measurable functions on $E$.
        Since $A$ and $B$ are disjoint, see Chapter 3 Problem 20 to see that
        \[
            f\cdot\chi_{A\cup B}=f\cdot(\chi_A+\chi_B+\chi_{A\cap B})=f\cdot(\chi_A+\chi_B+0)=f\cdot\chi_A+f\cdot\chi_B.    
        \]
        Furthermore, for any measurable subset $E_1$ of $E$ (see Problem 10),
        \[
            \int_{E_1}f=\int_{E}f\cdot\chi_{E_1}.    
        \]
        Therefore, by linearity of integration,
        \[
            \int_{A\cup B}f= \int_{E}f\cdot\chi_{A\cup B}=\int_{E}f\cdot\chi_{A}+\int_{E}f\cdot\chi_{B}=\int_Af+\int_Bf.
        \]
    \end{proof}
    \begin{namedthm*}{Corollary 7}
        Let $f$ be a bounded measurable function on a set of finite measure $E$.
        Then
        \[
            \biggl|\int_Ef\biggr|\le\int_E|f|.
        \]
    \end{namedthm*}
    \begin{proof}
        The function $|f|$ is measurable (see Chapter 3 Proposition 7) and bounded: $|f|\le|f|$ on $E$, so that
        \[
            -|f|\le f\le|f|\text{ on }E.
        \]
        Therefore by linearity and monotonicity of integration,
        \[
            -\int_E|f|\le \int_Ef\le\int_E|f|,
        \]
        and therefore $|\int_Ef|\le\int_E|f|$.
    \end{proof}
    \begin{namedthm*}{Proposition 8}
        Let $\{f_n\}$ be a sequence of bounded measurable functions on a set of finite measure $E$.
        \[
            \text{If }\{f_n\}\to f\text{ uniformly on }E,\text{ then }\lim_{n\to\infty}\int_Ef_n=\int_Ef.    
        \]
    \end{namedthm*}
    \begin{proof}
        Since the convergence is uniform and each $f_n$ is bounded, then the limit function $f$ is bounded.
        The function $f$ is measurable since it is the pointwise limit of a sequence of measurable functions (see Chapter 3 Proposition 9).
        \\Let $\epsilon>0$.
        By uniform convergence, there exists an index $N$ such that 
        \[
            |f-f_n|<\frac{\epsilon}{m(E)}\text{ on $E$ for all }n\ge N. 
        \]
        Then
        \begin{align*}
            |\int_Ef-\int_Ef_n|&=|\int_E[f-f_n]|&&\text{linearity of integration}\\
            &\le\int_E|f-f_n|&&\text{Corollary 7}\\
            &<\int_E[\frac{\epsilon}{m(E)}]\cdot1&&\text{monotonicity of integration}\\
            &=[\frac{\epsilon}{m(E)}]\cdot m(E)\\
            &=\epsilon.
        \end{align*}
        Therefore $\lim_{n\to\infty}\int_Ef_n=\int_Ef$.
    \end{proof}
    Recall an example from Chapter 3.2:
        \\Consider the sequence of continuous functions $\{f_n\}_{n=2}^\infty:[0,1]\to\mathbb{R}$, defined by
        \[ 
		f_n(x) =
            \begin{cases} 
                \frac{n-0}{1/n-0}x& \text{ if } x \in [0,\frac{1}{n}]\\
                \frac{0-n}{2/n-1/n}(x-\frac{1}{n})+n & \text{ if } x \in (\frac{1}{n},\frac{2}{n}]\\
                0& \text{ if } x \in (\frac{2}{n},1]
            \end{cases}
            =
            \begin{cases} 
                n^2x& \text{ if } x \in [0,\frac{1}{n}]\\
                -n^2(x-\frac{1}{n})+n & \text{ if } x \in (\frac{1}{n},\frac{2}{n}]\\
                0& \text{ if } x \in (\frac{2}{n},1]
            \end{cases}
	    \]
        (Each $f_n$ is a triangle-shaped function that achieves its max $f(1/n)=n$ and base corners $f(0)=0$ and $f(2/n)=0$.)\\
        In addition, consider the continuous function $f:[0,1]\to\mathbb{R}$ defined by $f(x)=0$ for all $x\in[0,1]$.\\\bigskip
        The sequence $\{f_n\}$ converges to $f$ pointwise (a.e) but not uniformly on $[0,1]$.
        \\Thus we have
        \[
            \{f_n\}\to f\text{ pointwise on }[0,1],\text{ but }\lim_{n\to\infty}\int_0^1f_n=1\neq0=\int_0^1f.
        \]
        \\Another example: we can define a sequence of nonnegative, measurable, continuous functions on $[0,1]$ by
        \[
            f_n(x)=
            \begin{cases}
                -n^2x+n&x\in[0,\frac{1}{n}]\\
                0&x\in(\frac{1}{n},1]
            \end{cases}
        \]
        Then $\{f_n\}\to f\equiv0$ pointwise a.e. on $[0,1]$, but $\int_{[0,1]}f_n=1/2$ for all $n$, and so
        \[
            \lim_{n\to\infty}\int_{[0,1]}f_n=1/2\neq0=\int_{[0,1]}f.
        \]
    \begin{namedthm*}{The Bounded Convergence Theorem}
        Let $\{f_n\}$ be a sequence of measurable functions on a set of finite measure $E$.
        Suppose $\{f_n\}$ is uniformly pointwise bounded on $E$; that is, there is a number $M\ge0$ for which
        \[
            |f_n|\le M\text{ on }E\text{ for all }n.
        \]
        \[
            \text{ If }\{f_n\}\to f\text{ pointwise on }E,\text{ then }\lim_{n\to\infty}\int_Ef_n=\int_Ef.   
        \]
    \end{namedthm*}
    \begin{proof}
        The pointwise limit of a sequence of measurable functions is measurable (Chapter 3 Proposition 9).
        Therefore $f$ is measurable.
        Also, clearly $|f|\le M$ on $E$.
        Let $A$ be any measurable subset of $E$ and $n$ a natural number.
        \\Then
        \begin{align*}
            \int_Ef_n-\int_Ef&=\int_E[f_n-f]\\
            &=\int_A[f_n-f]+\int_{E\setminus A}[f_n-f]&&\text{Corollary 6}\\
            &=\int_A[f_n-f]+\int_{E\setminus A}f_n+\int_{E\setminus A}(-f)&&\text{Linearity of integration}\\
            &\le\int_A[f_n-f]+\int_{E\setminus A}M+\int_{E\setminus A}M&&\text{Monotonicity of integration: }|f_n|,|f|\le M\\
            &=\int_A[f_n-f]+2M\cdot m(E\setminus A)&&\text{Integral of constant functions}\\
        \end{align*} 
        And by the triangle inequality and Corollary 7,
        \begin{align*}
            \left|\int_Ef_n-\int_Ef\right|
            &\le|\int_A[f_n-f]|+|2M\cdot m(E\setminus A)|
            \le\int_A|f_n-f|+2M\cdot m(E\setminus A)
        \end{align*}
        Now let $\epsilon>0$.
        \\Because $E$ has finite measure and $\{f_n\}$ converges pointwise to $f$ on $E$, then by Egoroff's Theorem, there is a measurable subset $A$ of $E$ for which $\{f_n\}\to f$ uniformly on $A$ and $m(E\setminus A)<\epsilon/4M$.
        Then by uniform convergence, there is an index $N$ for which 
        \[
            |f_n-f|<\frac{\epsilon}{2\cdot m(E)}\text{ on }A\text{ for all }n\ge N.
        \]
        Therefore, because $A\subseteq E\implies m(A)\le m(E)<\infty\implies \frac{m(A)}{m(E)}\le1$, using monotonicity of integration,
        \begin{align*}
            \left|\int_Ef_n-\int_Ef\right|&\le\int_A|f_n-f|+2M\cdot m(E\setminus A)\\
            &<\int_A\frac{\epsilon}{2\cdot m(E)}+2M\cdot\epsilon/4M\\
            &=\frac{\epsilon}{2\cdot m(E)}m(A)+\epsilon/2\\
            &\le\frac{\epsilon}{2}+\frac{\epsilon}{2}\\
            &=\epsilon.
        \end{align*}
        Therefore the sequence of integrals $\{\int_Ef\}$ converges to $\int_Ef$.
    \end{proof}
\begin{namedthm*}{Remark}
    Prior to the proof of the Bounded Convergence Theorem, no use was made of the countable additivity of the Lebesgue measure on the real line.
    Only finite additivity was used, and it was used just once, in the proof of Lemma 1. But for the proof of the Bounded Convergence Theorem we used Egoroff's Theorem.
    Egoroff's Theorem needed the continuity of Lebesgue measure, a consequence of countable additivity of Lebesgue measure.    
\end{namedthm*}

\end{flushleft}
\begin{center}
	\textbf{PROBLEMS}
\end{center}
\begin{enumerate}
	\setcounter{enumi}{8}
    \item Let $E$ have measure zero. Show that if $f$ is a bounded function on $E$, then $f$ is measurable and $\int_Ef=0$.\\
    \\First, consider any simple function $\psi$ defined on $E$, taking the values $a_1,\cdots,a_n$ on the subsets $E_1,\cdots,E_n$ of $E$.
    \\For any $i\in\{1,\cdots,n\}$, by monotonicity of Lebesgue measure, $m(E_i)\le m(E)=0\implies m(E_i)=0$.
    Therefore the integral of any simple function on $E$ is zero:
    \[
        \int_E\psi=\sum_{i=1}^n a_i\cdot m(E_i)=\sum_{i=1}^n a_i\cdot 0=0
    \]
    Thus for the bounded function $f$ on the set of finite measure $E$, we have
    \[
        \sup\biggl\{\int_E\varphi\ |\ \varphi\text{ simple, }\varphi\le f\text{ on }E\biggr\}=0=\inf\biggl\{\int_E\psi\ |\ \psi\text{ simple, }\psi\ge f\text{ on }E\biggr\},
    \]
    and $f$ is Lebesgue integrable with $\int_Ef=0$.\\
    \\To see that $f$ is measurable, consider any sequence of simple (measurable) functions $\{\varphi_n\}$ on $E$.
    This sequence trivially converges pointwise $a.e.$ on $E$ to the function $f$ because it converges pointwise on the set $\emptyset=E\setminus E$, where $E\subseteq E$ with $m(E)=0$.
    Then by Chapter 3 Proposition 9, $f$ is measurable.\\
    \\(A bounded function on a set of finite measure is Lebesgue integrable iff it is measurable;
    \\$(\implies)$ Chapter 5 Theorem 7, 
    \\$(\impliedby)$ Chapter 4 Theorem 4)
    \item Let $f$ be a bounded measurable function on a set of finite measure $E$. For a measurable subset $A$ of $E$, show that $\int_Af=\int_Ef\cdot\chi_A$.\\
    \\For any simple function $\psi$ on $E$, we have
    \[
        \psi=\sum_{k=1}^nc_k\cdot \chi_{E_k},
    \]
    Then for the measurable subset $A$ of $E$, the restriction of $\psi$ to $A$ is measurable and so
    \[
        \psi_{|_A}=\sum_{k=1}^nc_k\cdot \chi_{E_k\cap A},\text{ where }E_k\cap A=\{x\in E\ |\ \psi(x)=c_k\}\cap A =\{x\in A\ |\ \psi_{|_A}(x)=c_k\}
    \]
    Also, we consider the measurable function $\chi_A$ on $E$, and clearly the product $\psi\cdot\chi_A$ is measurable (and simple) so that
    \[
        \psi\cdot\chi_A=\sum_{k=1}^nc_k\cdot \chi_{E_k}\cdot\chi_{A}=\sum_{k=1}^nc_k\cdot \chi_{E_k\cap A},
    \]
    and therefore 
    \[
        \int_A\psi=\sum_{k=1}^n c_k\cdot m(E_k\cap A)=\int_E\psi\cdot\chi_A.
    \]
    Now, to show $\int_Af=\int_Ef\cdot\chi_A$ it is sufficient to see that $\int_Af\le\int_Ef\cdot\chi_A$ and $\int_Af\ge\int_Ef\cdot\chi_A$.\\ 
    \\For any simple function $\varphi'$ on $E$ such that $\varphi'\le f$, we have $\varphi'\cdot\chi_A\le f\cdot\chi_A$ on $E$, and so
    \[
        \int_A\varphi'=\int_E\varphi'\cdot\chi_A\le\sup\biggl\{\int_E\varphi\ |\ \varphi\text{ simple, }\varphi\le f\cdot\chi_A\text{ on }E\biggr\}=\int_Ef\cdot\chi_A
    \]
    Then the supremum of all such $\varphi'$ shows that
    \[
        \int_Af=\sup\biggl\{\int_A\varphi'\ |\ \varphi'\text{ simple, }\varphi'\le f\text{ on }A\biggr\}\le\int_Ef\cdot\chi_A\tag{1}
    \]
    Again, for any simple function $\psi'$ on $E$ such that $\psi'\ge f$, we have $\psi'\cdot\chi_A\ge f\cdot\chi_A$ on $E$, and so
    \[
        \int_A\psi'=\int_E\psi'\cdot\chi_A\ge\inf\biggl\{\int_E\psi\ |\ \psi\text{ simple, }\psi\ge f\cdot\chi_A\text{ on }E\biggr\}=\int_Ef\cdot\chi_A
    \]
    Then the infimum of all such $\psi'$ shows that
    \[
        \int_Af=\inf\biggl\{\int_A\psi'\ |\ \psi'\text{ simple, }\psi'\ge f\text{ on }A\biggr\}\ge\int_Ef\cdot\chi_A\tag{2}
    \]
    Therefore by $(1)$ and $(2)$ we have $\int_Af=\int_Ef\cdot\chi_A$.
    \item Does the Bounded Convergence Theorem hold for the Riemann integral?\\
    \\No. Recall the Dirichlet function example. 
    The sequence of measurable functions $\{f_n\}$ on the set of finite measure $[0,1]$ is uniformly bounded on $[0,1]$ as each $f_n\in\{0,1\}$ so that 
    \[
        |f_n|\le1\text{ on }[0,1]\text{ for all }n. 
    \]
    Then for the Dirichlet function $f$,
    \[
        \{f_n\}\to f\text{ pointwise on }[0,1],\text{ but }(R)\underline\int_0^1f=0<1=(R)\overline\int_0^1f,
    \]
    so that $f$ is not Riemann integrable and so $(R)\int_0^1f$ is not defined, and we cannot say anything about if $\lim_{n\to\infty}(R)\int_0^1f_n=(R)\int_0^1f$.
    \item Let $f$ be a bounded measurable function on a set of finite measure $E$. Assume $g$ is bounded and $f=g$ a.e. on $E$. Show that $\int_Ef=\int_Eg$.\\
    \\Because $f=g$ a.e. on $E$, then $f=g$ on $E\setminus E_0$, where $m(E_0)=0$.
    Recall from Chapter 3 Proposition 5 $(i)$ that $f$ is measurable on $E$ and $f=g$ a.e. on $E$ implies that $g$ is measurable on $E$.
    \\Then 
    \begin{align*}
        \int_Ef&=\int_{E\setminus E_0}f+\int_{E_0}f&&\text{Corollary 6}\\
        &=\int_{E\setminus E_0}f&&\text{Problem 9: }\int_{E_0}f=0\\
        &=\int_{E\setminus E_0}g&&\text{$f=g$ on $E\setminus E_0$}\\
        &=\int_Eg-\int_{E_0}g&&\text{Corollary 6}\\
        &=\int_Eg.&&\text{Problem 9: }\int_{E_0}g=0
    \end{align*}
    \item Does the Bounded Convergence Theorem hold if $m(E)<\infty$ but we drop the assumption that the sequence $\{|f_n|\}$ is uniformly bounded on $E$?\\ 
    \\No, see the example of the sequence of continuous (and thus measurable) triangular functions $\{f_n\}$ on $E=[0,1]$ so that $m(E)=1<\infty$.
    But $\{f_n\}=\{|f_n|\}$ is not uniformly bounded on $E$ because for every number $M\in\mathbb{N}$ we choose, there always exists the function $f_{N+1}$ with $f_{N+1}(\frac{1}{N+1})=N+1>N$.
    \\Then for the function $f\equiv0$ on $[0,1]$,
    \[
        \{f_n\}\to f\text{ pointwise on }[0,1],\text{ but }\lim_{n\to\infty}\int_0^1f_n=1\neq0=\int_0^1f,
    \]
    and the Bounded Convergence Theorem does not hold.
    \item Show that Proposition 8 is a special case of the Bounded Convergence Theorem.\\
    \\We see that since the convergence is uniform, then for $\epsilon>0$, there exists an index $N$ such that for all $n\ge N$,
    \[
        |f|-|f_n|\le|f-f_n|<\epsilon\text{ on }E.\tag{1}
    \]
    Also, since each $f_n$ is bounded, we have 
    \[
        |f_n|\le M_n\text{ on }E.
    \]
    Therefore at the index $N$, we have for any $\epsilon>0$,
    \[
        |f|<|f_N|+\epsilon\le M_N+\epsilon,
    \]
    and thus $f$ is bounded: $|f|\le M_N$ on $E$.\\
    \\To show that $\{f_n\}$ is uniformly bounded, we can set $\epsilon=1$ so that there exists an index $N'$ such that for all $n\ge N'$, by $(1)$, we have
    \[
        |f_n|<|f|+1\le M_N+1\text{ on }E.
    \]
    Then we have for all $n\in\mathbb{N}$,
    \[
        |f_n|\le\max\{M_1,\cdots,M_{N'},M_N+1\}\text{ on }E,
    \]
    and the sequence $\{f_n\}$ is uniformly bounded on $E$.\\
    \\Then Proposition 8 is a special case of the Bounded Convergence Theorem because it requires $\{f_n\}$ to converge uniformly to $f$.
    \item Verify the assertions in the last Remark of this section.\\
    \\This is true; we have:
    \begin{itemize}
        \item Continuity of measure uses countable additivity of measure,
        \item Lemma 10 uses continuity of measure,
        \item Egoroff's Theorem uses Lemma 10,
        \item Bounded Convergence Theorem uses Egoroff's Theorem.
    \end{itemize}
    \item Let $f$ be a nonnegative bounded measurable function on a set of finite measure $E$. Assume $\int_Ef=0$. Show that $f=0$ a.e. on $E$.\\
    \\We can suppose by contradiction that $f=0$ on $E\setminus E_0$, but $m(E_0)\neq0$.
    \\So $f>0$ on $E_0$, or in other words, 
    \[
        0<m(E_0)=m(\{x\in E\ |\ f(x)>0\})=m(\bigcup_{n=1}^\infty\{x\in E\ |\ f(x)\ge1/n\})=m(\bigcup_{n=1}^\infty E_n)\le \sum_{n=1}^\infty m(E_n).
    \]
    Then there must exist an index $k$ for which $m(E_k)\neq0$, else we have $0<m(E_0)\le\sum_{n=1}^\infty 0=0$ and reach a contradiction.
    \\Thus we have that $f\ge 1/k$ on $E_k$, with $m(E_k)>0$.
    \\Therefore it is possible to define the simple function $\varphi_k$ on $E$ such that $\varphi_n \le f$ on $E$:
    \[
        \varphi_k(x)=
        \begin{cases}
            1/k&x\in E_k\\
            0&x\notin E_k
        \end{cases}    
    \]
    where, because $1/k>0$ and $m(E_k)>0$, 
    \[
        \int_E\varphi_k=\frac{1}{k}\cdot m(E_k)+0\cdot m(E_k^c)=\frac{1}{k}\cdot m(E_k)>0.
    \]
    But then we have
    \[
        \int_E\varphi_k>0=\int_Ef=\sup\biggl\{\int_E\varphi\ |\ \varphi\text{ simple, }\varphi\le f\text{ on }E\biggr\},
    \]
    a contradiction to the supremum.
\end{enumerate}

% 4.3
\section{The Lebesgue Integral of a Measurable Nonnegative Function}
\begin{flushleft}
    A function $f$ on $E$ is said to be of finite support provided it vanishes outside a set of finite measure, that is, there exists a set $E_0$ such that $m(E_0)<\infty$ and $f\equiv0$ on $E\setminus E_0$.
    Therefore $f=f\cdot\chi_{E_0}$ so that 
    \[
        \int_{E_0}f=\int_Ef\cdot\chi_{E_0}=\int_Ef.
    \]
    \begin{namedthm*}{Definition}
        For a nonnegative measurable function $f$ on $E$, we define the integral of $f$ over $E$ by
        \[
            \int_Ef=\sup\biggl\{\int_Eh\ |\ h\text{ bounded, measurable, of finite support and }0\le h\le f\text{ on }E\biggr\}.
        \]
    \end{namedthm*}
    \begin{namedthm*}{Chebyshev's Inequality}
        Let $f$ be a nonnegative measurable function on $E$.
        Then for any $\lambda>0$, 
        \[
            m\{x\in E\ |\ f(x)\ge\lambda\}\le\frac{1}{\lambda}\cdot\int_Ef.\tag{9}
        \]
    \end{namedthm*}
    \begin{proof}
        Define $E_\lambda=\{x\in E\ |\ f(x)\ge\lambda\}$. 
        \\First suppose $m(E_\lambda)=\infty$.
        \\Then for a natural number $n$, define $E_{\lambda,n}=E_\lambda\cap[-n,n]$ and $\psi_n=\lambda\cdot\chi_{E_{\lambda,n}}$.
        Then $\psi_n$ is a bounded measurable function of finite support, 
        \[
            \lambda\cdot m(E_{\lambda,n})=\int_{E_{\lambda,n}}\lambda\cdot1=\int_E\lambda\cdot\chi_{E_{\lambda,n}}=\int_E\psi_n\text{ and }0\le\psi_n\le f\text{ on }E\text{ for all }n.
        \]
        From the continuity of measure, because $\{E_{\lambda,n}\}_{n=1}^\infty$ is ascending and $E_\lambda=\bigcup_{n=1}^\infty E_{\lambda,n}$,
        \[
            \infty=\lambda\cdot m(E_\lambda)=\lambda\cdot \lim_{n\to\infty}m(E_{\lambda,n})=\lim_{n\to\infty}\int_E\psi_n\le\int_Ef.
        \]
        Therefore the inequality $(9)$ holds since both sides equal $\infty$.
        \\Now suppose $m(E_\lambda)<\infty$.
        \\Define $h=\lambda\cdot\chi_{E_\lambda}$.
        Then $h$ is a bounded measurable function of finite support, and $0\le h\le f$ on $E$.
        Then 
        \[
            \lambda\cdot m(E_\lambda)=\int_{E_{\lambda}}\lambda\cdot1=\int_E\lambda\cdot\chi_{E_{\lambda}}=\int_Eh\le\int_Ef.
        \]
        Thus we have $m(E_\lambda)\le\frac{1}{\lambda}\int_Ef$.
    \end{proof}
    \begin{namedthm*}{Fatou's Lemma}
        Let $\{f_n\}$ be a sequence of nonnegative measurable functions on $E$.
        \[
            \text{ If }\{f_n\}\to f\text{ pointwise a.e. on }E,\text{ then }\int_Ef\le\lim\inf\int_Ef_n.    
        \]
    \end{namedthm*}
    \begin{proof}
        Let $\{f_n\}$ converge pointwise on $E\setminus E_0$, where $m(E_0)=0$. 
        We know that sets of measure zero do not contribute to the integral. 
        That is,
        \[
            \int_Ef=\int_{E\setminus E_0}f+\int_{E_0}f=\int_{E\setminus E_0}f.
        \]
        Then by excising this set of measure zero, we can assume pointwise convergence on all of $E$ (use $E$ instead of $E\setminus E_0$ for ease of writing).
        \\Then because each $f_n$ is nonnegative, we have $0\le f_n,\ \forall n\implies 0\le f$.
        Also, because $\{f_n\}$ is a sequence of measurable functions that converges pointwise to $f$, then $f$ is also measurable.
        \\Then to verify the inequality of Fatou's Lemma, by the definition of the integral of the nonnegative measurable function $f$, it is necessary and sufficient to show that if $h$ is any bounded measurable function of finite support for which $0\le h\le f$ on $E$, then
        \[
            \int_Eh\le\lim\inf\int_Ef_n.
        \]
        % To see this, suppose that $\int_Eh\le\lim\inf\int_Ef_n$ holds for all such $h$, but $\int_Ef>\lim\inf\int_Ef_n$.
        % \\Let $L=\lim\inf\int_Ef_n$. Then because $\int_Ef=\sup\biggl\{\int_Eh \biggr\}$:
        % \\If $\int_Ef<\infty$, there must exist an $h^*$ (bounded, measurable, finite support, $0\le h^*\le f$) such that
        % \[
        %     \int_Ef\ge\int_Eh^*>\int_Ef-\frac{\int_Ef-L}{2}>L.
        % \]
        % If $\int_Ef=\infty$, then $L<\int_Ef=\infty$, and for any real number $N\ge L$, there must exist an $h^*$ (bounded, measurable, finite support, $0\le h^*\le f$) such that
        % \[
        %     \int_Eh^*>N\ge L.
        % \]
        % However, both are a contradiction to $\int_Eh^*\le L=\lim\inf\int_Ef_n$.
        % Therefore $\int_Ef\le\lim\inf\int_Ef_n$.
        This is because $\int_Ef=\sup\biggl\{\int_Eh \biggr\}$; that is, the least upper bound.
        \\Now, consider a function $h$ that is bounded, measurable, finite support, and $0\le h\le f$.
        Then there exists $M\ge0$ for which $|h|\le M$ on $E$.
        Let $E_0=\{x\in E\ |\ h(x)\neq0\}$, so because $h$ is of finite support, $m(E_0)<\infty$.
        For $n\in\mathbb{N}$, define a function $h_n$ on $E$ by
        \[
            h_n=\min\{h,f_n\}\text{ on }E.    
        \]
        Then the function $h_n$ is measurable and
        \[
            0\le h_n\le M\text{ on }E_0\text{ and }h_n\equiv0\text{ on }E\setminus E_0\text{ (finite support)}.
        \]
        Also, for each $x$ in $E$, since $h(x)\le f(x)$ and $\{f_n(x)\}\to f(x)$, then $\{h_n(x)\}\to h(x)$.
        \\Then we have a sequence of measurable functions $\{h_n\}$ on a set of finite measure $E_0$.
        Also, there exists an $M$ such that $|h_n|\le M$ on $E_0$ for all $n$, and $\{h_n\}\to h$ pointwise on $E_0$. 
        Thus by the Bounded Convergence Theorem, we have $\lim_{n\to\infty}\int_{E_0}h_n=\int_{E_0}h$, and so
        \[
            \lim_{n\to\infty}\int_{E}h_n=\lim_{n\to\infty}\int_{E_0}h_n=\int_{E_0}h=\int_Eh.
        \]
        However, for each $n$, we have $h_n\le f_n$ on $E$ and by monotonicity, $\int_Eh_n\le\int_Ef_n$.
        Thus
        \[
            \int_Eh=\lim_{n\to\infty}\int_{E}h_n=\lim\inf\int_{E}h_n\le\lim\inf\int_{E}f_n.    
        \]
    \end{proof}
    \begin{namedthm*}{The Monotone Convergence Theorem}
        Let $\{f_n\}$ be an increasing sequence of nonnegative measurable functions on $E$.
        \[
            \text{If }\{f_n\}\to f\text{ pointwise a.e. on }E,\text{ then }\lim_{n\to\infty}\int_Ef_n=\int_Ef.
        \]
    \end{namedthm*}
    \begin{proof}
        By Fatou's Lemma, we have
        \[
            \int_Ef\le\lim\inf\int_Ef_n.  
        \]
        However, for each index $n$, because $\{f_n\}$ is increasing, we have $f_n\le f$ a.e. on $E$, and so $\int_Ef_n\le\int_Ef$.
        Therefore
        \[
            \lim\sup\int_Ef_n\le\int_Ef.
        \]
        Hence, $\lim\sup\int_Ef_n\le\int_Ef\le\lim\inf\int_Ef_n$ implies that
        \[
            \int_Ef=\lim_{n\to\infty}\int_Ef_n.
        \]
    \end{proof}
    \begin{namedthm*}{Corollary 12}
        Let $\{u_n\}$ be a sequence of nonnegative measurable functions on $E$.
        \[
            \text{If }f=\sum_{n=1}^\infty u_n\text{ pointwise a.e. on }E,\text{ then }\int_Ef=\sum_{n=1}^\infty\int_Eu_n.
        \]
    \end{namedthm*}
    \begin{proof}
        Let $f_k=\sum_{n=1}^ku_n$ so that $\{f_k\}$ is an increasing sequence of nonnegative measurable functions on $E$, and
        \[
            f=\sum_{n=1}^\infty u_n=\lim_{k\to\infty}\sum_{n=1}^ku_n=\lim_{k\to\infty}f_k\text{ pointwise a.e. on }E.
        \]
        Then by the Monotone Convergence Theorem and the linearity of integration, we have
        \[
            \int_Ef=\lim_{k\to\infty}\int_Ef_k=\lim_{k\to\infty}\int_E\sum_{n=1}^ku_n=\lim_{k\to\infty}\sum_{n=1}^k\int_Eu_n=\sum_{n=1}^\infty\int_Eu_n.
        \]
    \end{proof}
    \begin{namedthm*}{Definition}
        A nonnegative measurable function $f$ on a measurable set $E$ is said to be \textbf{integrable} over $E$ provided
        \[
            \int_Ef<\infty.    
        \]
    \end{namedthm*}
\end{flushleft}
\begin{center}
	\textbf{PROBLEMS}
\end{center}
\begin{enumerate}
	\setcounter{enumi}{16}
    \item Let $E$ be a set of measure zero and define $f\equiv\infty$ on $E$. Show that $\int_Ef=0$.\\
    \\(Recall Problem 9).
    If we suppose there exists a simple function that does not have an integral of zero, then there must exist some subset of $E_i$ of $E$ such that $m(E_i)>0$. 
    But this is a contradiction to the monotonicity of Lebesgue measure: $m(E_i)\le m(E)=0$.
    Therefore it must be that $\sup\biggl\{\int_E\varphi\ |\ \varphi\text{ simple, }\varphi\le f\text{ on }E\biggr\}=0=\inf\biggl\{\int_E\psi\ |\ \psi\text{ simple, }\psi\ge f\text{ on }E\biggr\}$ and $\int_Ef=0$.
    \item Show that the integral of a bounded measurable function of finite support is properly defined.\\
    \\Let $f$ on $E$ be a bounded measurable function of finite support. 
    \\That is, $m(E_0)=m(\{x\in E\ |\ f(x)\neq0\})<\infty$.
    \\We want to show that
    \[
        \int_Ef=\int_{E_0}f.
    \]
    First consider any simple function on $E_0$;
    we can write
    \[
        \varphi=\sum_{k=1}^nc_k\cdot \chi_{E_k}\text{ and }\int_{E_0}\varphi=\sum_{k=1}^n c_k\cdot m(E_k),
    \]
    where $E_k=\{x\in E_0\ |\ \varphi(x)=c_k\}$.
    \\Then extending $\varphi$ to $E$ by setting $\varphi(x)=0$ for $x\in E\setminus E_0$, see that
    \[
        \varphi=\sum_{k=1}^nc_k\cdot \chi_{E_k}+0\cdot\chi_{E\setminus E_0}\text{ and }\int_{E}\varphi=\sum_{k=1}^n c_k\cdot m(E_k) + 0\cdot m(E\setminus E_0). 
    \]
    However, $m(E\setminus E_0)=\infty$.
    We can use the $\sigma$-finiteness (see Chapter 17.1) of the Lebesgue measure to partition $E\setminus E_0$ into a countable union of disjoint measurable sets, each of finite measure.
    That is, $E\setminus E_0=\bigcup_{i=1}^\infty A_i$, where $m(A_i)<\infty$ for all $i$.
    We see that
    \begin{align*}
        \int_{E}\varphi&=\sum_{k=1}^n c_k\cdot m(E_k) + 0\cdot m(E\setminus E_0)\\
        &=\sum_{k=1}^n c_k\cdot m(E_k) + 0\cdot m(\bigcup_{i=1}^\infty A_i)\\
        &=\sum_{k=1}^n c_k\cdot m(E_k) + 0\cdot \sum_{i=1}^\infty m(A_i)\\
        &=\sum_{k=1}^n c_k\cdot m(E_k) + \sum_{i=1}^\infty 0\cdot m(A_i)\\
        &=\sum_{k=1}^n c_k\cdot m(E_k)\\
        &=\int_{E_0}\varphi.
    \end{align*}
    and therefore any simple function of finite support has $\int_{E}\varphi=\int_{E_0}\varphi$.\\
    \\Now, for any simple functions $\varphi$ and $\psi$ on on $E_0$ such that $\varphi\le f_{|_{E_0}}$ and $\psi\ge f_{|_{E_0}}$, there exists the extension $\varphi(x),\psi(x)=0$ for $x\in E\setminus E_0$ so that $0=\varphi(x)\le f(x)=0$ and $0=f(x)\le\psi(x)=0$ on $x\in E\setminus E_0$ and $\varphi\le f$ and $f\le \psi$ on all of $E$.
    \\Then because $\int_{E}\varphi=\int_{E_0}\varphi$ and $\int_{E}\psi=\int_{E_0}\psi$,
    \begin{align*}
        \biggl\{\int_{E}\varphi\ |\ \varphi\text{ simple, }\varphi\le f\text{ on }E\biggr\}\supseteq\biggl\{\int_{E_0}\varphi\ |\ \varphi\text{ simple, }\varphi\le f_{|_{E_0}}\text{ on }E_0\biggr\}\\
        \biggl\{\int_{E}\psi\ |\ \psi\text{ simple, }\psi\ge f\text{ on }E\biggr\}\supseteq\biggl\{\int_{E_0}\psi\ |\ \psi\text{ simple, }\psi\ge f_{|_{E_0}}\text{ on }E_0\biggr\}
    \end{align*}
    Then because $\sup\biggl\{\int_{E}\varphi\ |\ \varphi\text{ simple, }\varphi\le f\text{ on }E\biggr\}\le\inf\biggl\{\int_{E}\psi\ |\ \psi\text{ simple, }\psi\ge f\text{ on }E\biggr\}$ and 
    \begin{align*}
        \sup\biggl\{\int_{E}\varphi\ |\ \varphi\text{ simple, }\varphi\le f\text{ on }E\biggr\}\ge\sup\biggl\{\int_{E_0}\varphi\ |\ \varphi\text{ simple, }\varphi\le f_{|_{E_0}}\text{ on }E_0\biggr\}=\int_{E_0}f_{|_{E_0}}\\
        \inf\biggl\{\int_{E}\psi\ |\ \psi\text{ simple, }\psi\ge f\text{ on }E\biggr\}\le\inf\biggl\{\int_{E_0}\psi\ |\ \psi\text{ simple, }\psi\ge f_{|_{E_0}}\text{ on }E_0\biggr\}=\int_{E_0}f_{|_{E_0}}
    \end{align*}
    we have $\int_Ef=\int_{E_0}f_{|_{E_0}}$.
    \item For a number $\alpha$, define $f(x)=x^\alpha$ for $0<x\le1$, and $f(0)=0$. Compute $\int_0^1f$.\\
    \\Case $0\le\alpha$:
    \\Then the function $f(x)=x^\alpha$ is positive monotone on $[0,1]$, that is, 
    \[
        0\le x\le1\implies 0=0^\alpha\le x^\alpha\le1^\alpha=1,
    \]
    and $f$ is bounded by $1$ on the closed bounded interval $[0,1]$.\\
    \\Now we see that $f$ is Riemann integrable:
    \\For each natural number $m$, consider the partition $P_m=\{0,\frac{1}{m},\frac{2}{m},\cdots,\frac{m-1}{m},1\}=\{x_0,x_1,\cdots,x_m\}$.
    Then because $f$ is increasing, we have $f(x_{k-1})=\inf\{f(x)\ |\ x_{k-1}<x<x_k\}$ and $f(x_k)=\sup\{f(x)\ |\ x_{k-1}<x<x_k\}$, so that
    \begin{align*}
        L(f,P_m)&=\sum_{k=1}^m f(x_{k-1})\cdot\frac{1}{m}\\
        &=\sum_{k=1}^{m-1} f(x_k)\cdot\frac{1}{m}+f(x_0)\cdot\frac{1}{m}\\
        &=\sum_{k=1}^{m-1} f(x_k)\cdot\frac{1}{m}+0\cdot\frac{1}{m}\\
        &=\sum_{k=1}^{m-1} f(x_k)\cdot\frac{1}{m}
    \end{align*}
    and
    \begin{align*}
        U(f,P_m)&=\sum_{k=1}^m f(x_k)\cdot\frac{1}{m}\\
        &=\sum_{k=1}^{m-1} f(x_k)\cdot\frac{1}{m}+f(x_m)\cdot\frac{1}{m}\\
        &=\sum_{k=1}^{m-1} f(x_k)\cdot\frac{1}{m}+1\cdot\frac{1}{m}
    \end{align*}
    Then clearly we get $\lim_{m\to\infty}[U(f,P_m)-L(f,P_m)]=0$, with 
    \[
        U(f,P_m)=\sum_{k=1}^{m} f(x_k)\cdot\frac{1}{m}=\sum_{k=1}^{m} (\frac{k}{m})^\alpha\cdot\frac{1}{m},
    \]
    and $\lim_{m\to\infty}\sum_{k=1}^{m} (\frac{k}{m})^\alpha\cdot\frac{1}{m}=\frac{1}{\alpha+1}$.
    That is, we can see $\lim_{m\to\infty} \frac{\sum_{k=1}^{m}k^\alpha}{m^{\alpha+1}}-\frac{1}{\alpha+1}=0$.
    % \begin{align*}
    %     \frac{\sum_{k=1}^{m}k^\alpha}{m^{\alpha+1}}-\frac{1}{\alpha+1}
    %     =\frac{1}{\alpha+1}\left(\frac{\sum_{k=1}^{m}k^\alpha(\alpha+1)}{m^{\alpha+1}}-1\right)
    %     =\frac{1}{\alpha+1}\left(\frac{\sum_{k=1}^{m}[k^\alpha(\alpha+1)]-m^{\alpha+1}}{m^{\alpha+1}}\right)
    % \end{align*}
    \\(We can use integration to see:)
    \[
        \int_0^1x^\alpha = \left[\frac{x^{\alpha+1}}{\alpha+1}\right]_{x=0}^{x=1}=\left(\frac{1^{\alpha+1}}{\alpha+1}-\frac{0^{\alpha+1}}{\alpha+1}\right)=\left(\frac{1}{\alpha+1}\right).
    \]
    \\Case $-1<\alpha<0$:
    \\For each natural number $n$, define $f_n$ on $[0,1]$ such that
    \[
        f_n(x)=
        \begin{cases}
            f(x)=x^\alpha&\text{if }x\in[1/n,1]\\
            0&\text{if }x\in[0,1/n)
        \end{cases}
    \]
    Then $\{f_n\}$ is an increasing sequence of nonnegative measurable functions on $[0,1]$, and $\{f_n\}\to f$ pointwise on $[0,1]$.
    By the Monotone Convergence Theorem, $\lim_{n\to\infty}\int_0^1f_n=\int_0^1f$.\\
    \\Now,  because $f(x)=x^\alpha=\frac{1}{x^{-\alpha}}$ is negative monotone on $(0,1]$, that is, for any $n\in\mathbb{N}$,
    \[
        1/n\le x\implies (1/n)^{-\alpha}\le x^{-\alpha}\implies\frac{1}{x^{-\alpha}}\le\frac{1}{(1/n)^{-\alpha}}\implies x^\alpha\le(1/n)^\alpha,
    \] 
    then $f_n$ is bounded by $(1/n)^\alpha$ on the closed bounded interval $[0,1]$.\\
    \\Now we see that $f$ is Riemann integrable:
    \\Case $\alpha\le-1$:
    \\Consider the same sequence $\{f_n\}$ from the previous case. 
    We can use the Monotone Convergence Theorem.
    See again that $f_n$ is bounded by $(1/n)^\alpha$ on the closed bounded interval $[0,1]$, and is thus Riemann integrable.
    \item Let $\{f_n\}$ be a sequence of nonnegative measurable functions that converges to $f$ pointwise on $E$.
    Let $M\ge0$ be such that $\int_Ef_n\le M$ for all $n$. Show that $\int_Ef\le M$. Verify that this property is equivalent to the statement of Fatou's Lemma.\\
    \\Let $\{f_n\}$ be a sequence of nonnegative measurable functions that converges to $f$ pointwise on $E$.\\
    \\(Fatou's Lemma $\implies$ Property) Let $M\ge0$ be such that $\int_Ef_n\le M$ for all $n$.
    \\Then by Fatou's Lemma,
    \[
        \int_Ef\le\lim\inf\int_Ef_n\le M.
    \]
    (Property $\implies$ Fatou's Lemma) Suppose that $M\ge0$ with $\int_Ef_n\le M$ $\forall n$ implies $\int_Ef\le M$.
    \\Then, by Chapter 1 Problem 38, we know that $\lim\inf\left\{\int_Ef_n\right\}$ is the smallest cluster point of $\left\{\int_Ef_n\right\}$.
    That is, there exists a subsequence $\left\{\int_Ef_{n_k}\right\}$ that converges to $\lim\inf\left\{\int_Ef_n\right\}$.
    \\Fix $\epsilon>0$.
    \\Then there exists a natural number $N$ such that for $k\ge N$, then $\left|\int_Ef_{n_k}-\lim\inf\left\{\int_Ef_n\right\}\right|<\epsilon$.
    Then we have that $\{f_{n_k}\}$ is a sequence of measurable functions that converges to $f$ pointwise on $E$, and $\lim\inf\left\{\int_Ef_n\right\}+\epsilon\ge0$ with $\int_Ef_{n_k}<\lim\inf\left\{\int_Ef_n\right\}+\epsilon$,
    so the Property implies that $\int_E f\le\lim\inf\left\{\int_Ef_n\right\}+\epsilon$.
    \\Because this is true for any $\epsilon$, then $\int_E f\le\lim\inf\left\{\int_Ef_n\right\}$ holds.
    \item Let the function $f$ be nonnegative and integrable over $E$ and $\epsilon>0$. Show there is a simple function $\eta$ on $E$ that has finite support, $0\le\eta\le f$ on $E$ and $\int_E|f-\eta|<\epsilon$.
    If $E$ is a closed, bounded interval, show there is a step function $h$ on $E$ that has finite support and $\int_E|f-h|<\epsilon$.\\
    \\Fix $\epsilon>0$.
    \\For a nonnegative measurable function $f$ on $E$, we define the integral of $f$ over $E$ by
    \[
        \int_Ef:=\sup\biggl\{\int_Eh\ |\ h\text{ bounded, measurable, of finite support and }0\le h\le f\text{ on }E\biggr\},
    \]
    Then by definition of supremum, there exists $h\in\{ \text{h bounded, measurable, of finite support and }0\le h\le f\text{ on }E\}$ such that
    \[
        \int_Ef-\frac{\epsilon}{2}<\int_Eh\le\int_Ef.
    \]
    We can write this as
    \[
        \int_Ef-\int_Eh<\frac{\epsilon}{2}\tag{1}
    \]
    We have the definition of the integral of the bounded measurable function $h$ of finite support:
    \[
        \int_Eh:=\sup\biggl\{\int_E\varphi\ |\ \varphi\text{ simple, }0\le\varphi\le f\text{ on }E\biggr\}.
    \]
    Then by definition of supremum, there exists $\eta\in\{ \varphi\text{ simple, }0\le\varphi\le f\text{ on }E\}$ such that
    \[
        \int_Eh-\frac{\epsilon}{2}<\int_E\eta\le\int_Eh.
    \]
    We can write this as
    \[
        -\int_E\eta<-\int_Eh+\frac{\epsilon}{2}.\tag{2}
    \]
    Then by (1) and (2),
    \[
        \int_E[f-\eta]=\int_Ef-\int_E\eta<\int_Ef-\int_Eh+\frac{\epsilon}{2}<\frac{\epsilon}{2}+\frac{\epsilon}{2}=\epsilon.
    \]
    \item Let $\{f_n\}$ be a sequence of nonnegative measurable functions on $\mathbb{R}$ that converges pointwise on $\mathbb{R}$ to $f$ and $f$ be integrable over $\mathbb{R}$. Show that
    \[
        \text{if }\int_{\mathbb{R}}f=\lim_{n\to\infty}\int_{\mathbb{R}}f_n,\text{ then }\int_Ef=\lim_{n\to\infty}\int_Ef_n\text{ for any measurable set }E.    
    \]
    \\Let $E$ be a measurable set of real numbers.
    \\By Theorem 11, we have, for each $n$,
    \[
        \int_\mathbb{R}f_n=\int_Ef_n+\int_{E^c}f_n.\tag{1}
    \]
    Then because $f$ is integrable, then its integral $\int_{\mathbb{R}}f$ is finite, which by equality to $\lim_{n\to\infty}\int_{\mathbb{R}}f_n$ implies that the sequence $\left\{\int_{\mathbb{R}}f_n\right\}$ converges; that is,
    \[
        \infty>\int_{\mathbb{R}}f=\lim_{n\to\infty}\int_{\mathbb{R}}f_n=\liminf\int_{\mathbb{R}}f_n=\limsup\int_{\mathbb{R}}f_n.\tag{2}
    \]
    In particular, we have by (1), (2), and Fatou's Lemma for $\{f_n\}$ on $E^c$,
    % \[
    %     \int_\mathbb{R}f=\limsup\int_Ef_n+\liminf\int_{E^c}f_n.
    % \]
    % By Fatou's Lemma,
    \[
        \int_\mathbb{R}f-\limsup\int_Ef_n=\liminf\int_{E^c}f_n\ge\int_{E^c}f,
    \]
    so that, rearranging,
    \[
        \int_Ef=\int_\mathbb{R}f-\int_{E^c}f\ge\limsup\int_Ef_n.\tag{a}
    \]
    Then by Fatou's Lemma for $\{f_n\}$ on $E$,
    \[
        \int_Ef\le\liminf\int_Ef_n.\tag{b}
    \]
    Then (a), (b), and the fact that $\liminf\int_Ef_n\le\limsup\int_Ef_n$ imply equality:
    \[
        \int_Ef=\lim_{n\to\infty}\int_Ef_n.
    \]
    \item Let $\{a_n\}$ be a sequence of nonnegative real numbers. Define the function $f$ on $E=[1,\infty)$ by setting $f(x)=a_n$ if $n\le x<n+1$. Show that $\int_Ef=\sum_{n=1}^\infty a_n$.\\
    \\Because each $a_n$ is nonnegative, we can define a sequence of increasing functions $\{f_k\}$ on $[1,\infty)$ by
    \[
        f_k(x)=
        \begin{cases}
            a_n&x\in[n,n+1)\text{ for each }n\in\{1,\dots,k\}\\
            0&\text{else}
        \end{cases}
    \]
    Then each $f_k$ is a simple function of finite support $\bigcup_{n=1}^k[n,n+1)$ so that
    \[
        \int_{[1,\infty)}f_k=\sum_{n=1}^ka_n\cdot m([n,n+1))=\sum_{n=1}^ka_n.
    \]
    Also, $\{f_k\}\to f$ pointwise so that, by the Monotone Convergence Theorem,
    \[
        \sum_{n=1}^\infty a_n=\lim_{k\to\infty}\sum_{n=1}^k a_n=\lim_{k\to\infty}\int_{[1,\infty)}f_k=\int_{[1,\infty)}f.
    \]
    \item Let $f$ be a nonnegative measurable function on $E$.
    \begin{enumerate}[label=(\roman*),align=left]
        \item Show there is an increasing sequence $\{\varphi_n\}$ of nonnegative simple functions on $E$, each of finite support, which converges pointwise on $E$ to $f$.\\
        \\See the Simple Approximation Theorem from Chapter 3.
        \item Show that $\int_Ef=\sup\{\int_E\varphi\ |\ \varphi\text{ simple, of finite support and }0\le\varphi\le f\text{ on }E\}$.\\
        % \\For a nonnegative measurable function $f$ on $E$, we define the integral of $f$ over $E$ by
        % \[
        %     \int_Ef:=\sup\biggl\{\int_Eh\ |\ h\text{ bounded, measurable, of finite support and }0\le h\le f\text{ on }E\biggr\},
        % \]
        % where the integral of each bounded measurable function $h$ of finite support has an integral defined by
        % \[
        %     \int_Eh:=\sup\biggl\{\int_E\varphi\ |\ \varphi\text{ simple, }\varphi\le f\text{ on }E\biggr\}=\inf\biggl\{\int_E\psi\ |\ \psi\text{ simple, }\psi\ge f\text{ on }E\biggr\}.
        % \]
        \\By the Monotone Convergence Theorem with (i), we have that
        \[
            \sup_n\int_E\varphi_n=\lim_{n\to\infty}\int_E\varphi_n=\int_Ef.
        \]
    \end{enumerate}
    \item Let $\{f_n\}$ be a sequence of nonnegative measurable functions on $E$ that converges pointwise on $E$ to $f$. Suppose $f_n\le f$ on $E$ for each $n$. Show that
    \[
        \lim_{n\to\infty}\int_Ef_n=\int_Ef.
    \]
    \\(similar to Monotone Convergence Theorem)
    \\By monotonicity of integration, we get
    \[
        \lim\sup\int_Ef_n\le \int_Ef.
    \]
    By Fatou's Lemma,
    \[
        \int_Ef\le\lim\inf\int_Ef_n.     
    \]
    Then $\lim\inf\int_Ef_n\le\lim\sup\int_Ef_n$ (see Chapter 1 Problem 41) implies equality:
    \[
        \lim_{n\to\infty}\int_Ef_n=\int_Ef.
    \]
    \item Show that the Monotone Convergence Theorem may not hold for decreasing sequences of functions.\\
    \\We can define a decreasing sequence of nonnegative measurable functions $f_n:[0,\infty)\to\mathbb{R}$ by
    \[
        f_n(x)=
        \begin{cases}
            0&x<n\\
            1&x\ge n
        \end{cases}
    \]
    Then $\{f_n\}\to f\equiv0$ pointwise on $[0,\infty)$, but $\int_{[0,\infty)}f_n=\infty$ (not integrable over $[0,\infty)$) for all $n$, and so
    \[
        \lim_{n\to\infty}\int_{[0,\infty)}f_n=\infty\neq0=\int_{[0,\infty)}f.
    \]
    \item Prove the following generalization of Fatou's Lemma: If $\{f_n\}$ is a sequence of nonnegative measurable functions on $E$, then 
    \[
        \int_E\lim\inf f_n\le\lim\inf\int_Ef_n.     
    \]
    \\We have, for each $n$,
    \[
        0\le\inf_{k\ge n} f_k(x)\le f_n(x)\text{ for all }x,
    \]
    and so by monotonicity of integration,
    \[
    \int_E\inf_{k\ge n} f_k(x)\le\int_Ef_n,
    \]
    and by the Monotone Convergence Theorem for the sequence $\{\inf_{k\ge n} f_k\}\to\lim_n\inf_{k\ge n} f_k$,
    \[
    \int_E\lim_n\inf_{k\ge n} f_k(x)=\lim_n\int_E\inf_{k\ge n} f_k(x)\le\lim\inf_n\int_Ef_n.
    \]
\end{enumerate}
    
% 4.4
\authoredby{inprogress}
\section{The General Lebesgue Integral}
For an extended real-valued function $f$ on $E$, we have defined the positive part $f^+$ and the negative part $f^-$ of $f$, respectively, by
\[
    f^+(x)=\max\{f(x),0\},\text{ and }f^-(x)=\max\{-f(x),0\}\text{ for all }x\in E.
\]
Then $f^+$ and $f^-$ are nonnegative functions on $E$,
\[
    f=f^+-f^-\text{ on }E,
\]
and
\[
    |f|=f^++f^-\text{ on }E.
\]
\begin{namedthm*}{Definition}
    A measurable function $f$ on $E$ is said to be \textbf{integrable} over $E$ provided $|f|$ is integrable over $E$.
    When this is so we define the integral of $f$ over $E$ by
    \[
        \int_Ef=\int_Ef^+-\int_Ef^-.
    \]  
\end{namedthm*}
\begin{namedthm*}{Proposition 16}[the Integral Comparison Test]
    Let $f$ be a measurable function on $E$.
    Suppose there is a nonnegative function $g$ that is integrable over $E$ and dominates $f$ in the sense that
    \[
        |f|\le g\text{ on }E.
    \]
    Then $f$ is integrable over $E$ and 
    \[
        \left|\int_Ef\right|\le\int_E|f|.
    \]
\end{namedthm*}
\begin{proof}
    By the monotonicity of integration for nonnegative functions, we have $\int_E|f|\le\int_Eg<\infty$, which implies $|f|$ is integrable, and thus $f$ is integrable.
    \\Then by the triangle inequality and the linearity of integration,
    \[
        \left|\int_Ef\right|=\left|\int_Ef^+-\int_Ef^-\right|\le\int_Ef^++\int_Ef^-=\int_E|f|.
    \]
\end{proof}
\begin{namedthm*}{The Lebesgue Dominated Convergence Theorem}
    Let $\{f_n\}$ be a sequence of measurable functions on $E$.
    Suppose there is a function $g$ that is integrable over $E$ and dominates $\{f_n\}$ on $E$ in the sense that $|f_n|\le g$ on $E$ for all $n$.
    \[
        \text{If }\{f_n\}\to f\text{ pointwise a.e. on }E,\text{ then }f\text{ is integrable over }E\text{ and }\lim_{n\to\infty}\int_Ef_n=\int_Ef.
    \]
\end{namedthm*}
\begin{proof}
    Since $|f_n|\le g$ on $E$ and $|f|\le g$ a.e. on $E$ and $g$ is integrable over $E$, by the integral comparison test, $f$ and each $f_n$ also are integrable over $E$.
    We infer from Proposition 15 that, by possibly excising from $E$ a countable collection of sets of measure zero and using the countable additivity of Lebesgue measure, we may assume that $f$ and each $f_n$ is finite on $E$.
    The function $g-f$ and for each $n$, the function $g-f_n$, are properly defined, nonnegative, and measurable.
    Morever, the sequence $\{g-f_n\}$ converges pointwise a.e. on $E$ to $g-f$.
    \\By Fatou's Lemma,
    \[
        \int_E(g-f)\le\liminf\int_E(g-f_n).
    \]
    Thus, by the linearity of integration for measurable functions and $\liminf(-a_n)=-\limsup(a_n)$ (Chapter 1 Proposition 19 (iii)),
    \[
        \int_Eg-\int_Ef=\int_E(g-f)\le\liminf\int_E(g-f_n)=\int_Eg+\liminf\left(-\int_Ef_n\right)=\int_Eg-\limsup\int_Ef_n,
    \]
    which tells us that
    \[
        \limsup\int_Ef_n\le\int_Ef.
    \]
    Similarly by Fatou's Lemma for the sequence $\{g+f_n\}$,
    \[
        \int_Eg+\int_Ef=\int_E(g+f)\le\liminf\int_E(g+f_n)=\int_Eg+\liminf\int_Ef_n,
    \]
    which tells us that
    \[
        \int_Ef\le\liminf\int_Ef_n.
    \]
\end{proof}
\begin{namedthm*}{Theorem 19}[General Lebesgue Dominated Convergence Theorem]
    Let $\{f_n\}$ be a sequence of measurable functions on $E$ that converges pointwise a.e. on $E$ to $f$.
    Suppose there is a sequence $\{g_n\}$ of nonnegative measurable functions on $E$ that converges pointwise a.e. on $E$ to $g$ and dominates $\{f_n\}$ on $E$ in the sense that
    \[
        |f_n|\le g_n\text{ on }E\text{ for all }n. 
    \]
    \[
        \text{If }\lim_{n\to\infty}\int_Eg_n=\int_Eg<\infty,\text{ then }\lim_{n\to\infty}\int_Ef_n=\int_Ef.
    \]
\end{namedthm*}

\begin{center}
	\textbf{PROBLEMS}
\end{center}
\begin{enumerate}
	\setcounter{enumi}{27}
    \item Let $f$ be integrable over $E$ and let $C$ be a measurable subset of $E$. Show that $\int_Cf=\int_Ef\cdot\chi_C$.\\
    \\We have
    \item For a measurable function $f$ on $[1,\infty)$ which is bounded on bounded sets, define $a_n=\int_n^{n+1}f$ for each natural number $n$.
    Is it true that $f$ is integrable over $[1,\infty)$ iff the series $\sum_{n=1}^\infty a_n$ converges?
    Is it true that $f$ is integrable over $[1,\infty)$ iff the series $\sum_{n=1}^\infty a_n$ converges absolutely?\\
    \\
    \item Let $g$ be a nonnegative integrable function over $E$ and suppose $\{f_n\}$ is a sequence of measurable functions on $E$ such that for each $n$, $|f_n|\le g$ a.e. on $E$. Show that
    \[
        \int_E\lim\inf f_n \le \lim\inf\int_E f_n \le \lim\sup\int_E f_n \le \int_E\lim\sup f_n.
    \]
    \item Let $f$ be a measurable function on $E$ which can be expressed as $f=g+h$ on $E$, where $g$ is finite and integrable over $E$ and $h$ is nonnegative on $E$.
    Define $\int_Ef=\int_Eg+\int_Eh$. Show that this is properly defined in the sense that it is independent of the particular choice of finite integrable function $g$ and nonnegative function $h$ whose sum is $f$.
    \item Prove the General Lebesgue Dominated Convergence Theorem by following the proof of the Lebesgue Dominated Convergence Theorem, but replacing the sequences $\{g-f_n\}$ and $\{g+f_n\}$, respectively, by $\{g_n-f_n\}$ and $\{g_n+f_n\}$.
    \item Let $\{f_n\}$ be a sequence of integrable functions on $E$ for which $f_n\to f$ a.e. on $E$ and $f$ is integrable over $E$. Show that $\int_E|f-f_n|\to0$ iff $\lim_{n\to\infty}\int_E|f_n|=\int_E|f|$.
    (Hint: use the General Lebesgue Dominated Convergence Theorem.)
    \item Let $f$ be a nonnegative measurable function on $\mathbb{R}$. Show that 
    \[
        \lim_{n\to\infty}\int_{-n}^nf=\int_{\mathbb{R}}f.
    \]  
    \\Define, for each $n$, the function
    \[
        f_n=f\cdot\chi_{[-n,n]},
    \] 
    so that because $f$ is nonnegative and measurable, $\{f_n\}$ is an increasing sequence of nonnegative measurable functions on $\mathbb{R}$ that converges pointwise to $f$ on $\mathbb{R}$.
    \\Then by the Monotone Convergence Theorem, 
    \[
        \lim_{n\to\infty}\int_{[-n,n]}f=\lim_{n\to\infty}\int_Ef\cdot\chi_{[-n,n]}=\lim_{n\to\infty}\int_Ef_n=\int_Ef.
    \] 
    \item Let $f$ be a real-valued function of two variables $(x,y)$ that is defined on the square $Q=\{(x,y)\ |\ 0\le x\le 1,0\le y\le 1\}$ and is a measurable function of $x$ for each fixed value of $y$.
    Suppose for each fixed value of $x$, $\lim_{y\to0}f(x,y)=f(x)$ and that for all $y$, we have $|f(x,y)|\le g(x)$, where $g$ is integrable over $[0,1]$. Show that
    \[
    \lim_{y\to0}\int_0^1f(x,y)dx=\int_0^1f(x)dx.    
    \] 
    Also show that if the function $f(x,y)$ is continuous in $y$ for each $x$, then 
    \[
        h(y)=\int_0^1f(x,y)dx 
    \]
    is a continuous function of $y$.
    \item Let $f$ be a real-valued function of two variables $(x,y)$ that is defined on the square $Q=\{(x,y)\ |\ 0\le x\le 1,0\le y\le 1\}$ and is a measurable function of $x$ for each fixed value of $y$.
    For each $(x,y)\in Q$ let the partial derivative $\partial f/\partial y$ exist. Suppose there is a function $g$ that is integrable over $[0,1]$ and such that 
    \[
        \biggl | \frac{\partial f}{\partial y}(x,y) \biggr | \le g(x)\text{ for all }(x,y)\in Q.
    \]
    Prove that 
    \[
        \frac{d}{dy}\biggl[\int_0^1f(x,y)dx\biggr]=\int_0^1\frac{\partial f}{\partial y}(x,y)dx\text{ for all }y\in [0,1].
    \]
\end{enumerate}

% 4.5
\authoredby{untouched}
\section{Countable Additivity and Continuity of Integration}
\begin{center}
	\textbf{PROBLEMS}
\end{center}
\begin{enumerate}
	\setcounter{enumi}{36}
    \item Let $f$ be an integrable function on $E$. Show that for each $\epsilon>0$, there is a natural number $N$ for which if $n\ge N$, then $|int_{E_n}f|<\epsilon$ where $E_n=\{x\in E\ |\ |x|\ge n\}$.
    \item For each of the two functions $f$ on $[1,\infty)$ defined below, show that $\lim_{n\to\infty}\int_1^nf$ exists while $f$ is not integrable over $[1,\infty)$. Does this contradict the continuity of integration?
    \begin{enumerate}[label=(\roman*),align=left]
        \item Define $f(x)=\frac{(-1)^n}{n}$, for $n\le x < n+1$.
        \item Define $f(x) = \frac{(\sin x)}{x}$ for $1\le x<\infty$.
    \end{enumerate}
    \item Prove the theorem regarding the continuity of integration.
\end{enumerate}

% 4.6
\section{Uniform Integrability: The Vitali Convergence Theorem}
\begin{center}
	\textbf{PROBLEMS}
\end{center}
\begin{enumerate}
	\setcounter{enumi}{39}
    \item Let $f$ be integrable over $\mathbb{R}$. Show that the function $F$ defined by 
    \[
        F(x) = \int_{-\infty}^xf\text{ for all }x\in\mathbb{R}
    \]
    is properly defined and continuous. Is it necessarily Lipschitz?
    \item Show that Proposition 25 is false if $E=\mathbb{R}$.
    \item Show that Theorem 26 is false without the assumption that the $h_n$'s are nonnegative.
    \item Let the sequences of functions $\{h_n\}$ and $\{g_n\}$ be uniformly integrable over $E$. Show that for any $\alpha$ and $\beta$, the sequence of linear combinations $\{\alpha f_n + \beta g_n\}$ also is uniformly integrable over $E$.
    \item Let $f$ be integrable over $\mathbb{R}$ and let $\epsilon>0$. Establish the following three approximation properties. 
    \begin{enumerate}[label=(\roman*),align=left]
        \item There is a simple function $\eta$ on $\mathbb{R}$ which has finite support and $\int_{\mathbb{R}}|f-\eta|<\epsilon$. (Hint: first verify this if $f$ is nonnegative.)
        \item There is a step function $s$ on $\mathbb{R}$ which vanishes outside a closed, bounded interval and $\int_{\mathbb{R}}|f-s|<\epsilon$. (Hint: apply part (i) and Problem 18 of Chapter 3.)
        \item There is a continuous function $g$ on $\mathbb{R}$ which vanishes outside a bounded set and $\int_{\mathbb{R}}|f-g|<\epsilon$.
    \end{enumerate}
    \item Let $f$ be integrable over $E$. Define $\hat f$ to be the extension of $f$ to all of $\mathbb{R}$ obtained by setting $\hat f \equiv 0$ outside of $E$. 
    Show that $\hat f$ is integrable over $\mathbb{R}$ and $\int_Ef=\int_{\mathbb{R}}\hat f$.
    Use this and part (i) and (iii) of the preceding problem to show that for $\epsilon>0$, there is a simple function $\eta$ on $E$ and a continuous function $g$ on $E$ for which $\in_E|f-\eta|<\epsilon$ and $\in_E|f-g|<\epsilon$.
    \item (Riemann-Lebesgue) Let $f$ be integrable over $(-\infty,\infty)$. Show That
    \[
        \lim_{n\to\infty}\int_{-\infty}^\infty f(x) \cos nxdx=0.    
    \]
    (Hint: first show this for $f$ is a step function that vanishes outside a closed, bounded interval and then use the approximation property (ii) of Problem 44.)
    \item Let $f$ be integrable over $(-\infty,\infty)$.
    \begin{enumerate}[label=(\roman*),align=left]
        \item Show that for each $t$,
        \[
            \int_{-\infty}^\infty f(x)dx=\int_{-\infty}^\infty f(x+t)dx.  
        \]
        \item Let $g$ be a bounded measurable function on $\mathbb{R}$. Show that 
        \[
            \lim_{t\to0}\int_{-\infty}^\infty g(x)\cdot[f(x)-f(x+t)]=0.
        \] 
        (Hint: first show this, using uniform continuity of $f$ on $\mathbb{R}$, if $f$ is continuous and vanishes outside a bounded set. Then use the approximation property (iii) of Problem 44.)
    \end{enumerate}
    \item Let $f$ be integrable over $E$ and let $g$ be a bounded measurable function on $E$. Show that $f\cdot g$ is integrable over $E$.
    \item Let $f$ be integrable over $\mathbb{R}$. Show that the following four assertions are equivalent:
    \begin{enumerate}[label=(\roman*),align=left]
        \item $f=0$ a.e. on $\mathbb{R}$.
        \item $\int_{\mathbb{R}}fg=0$ for every bounded measurable function $g$ on $\mathbb{R}$.
        \item $\int_Af=0$ for every measurable set $A$.
        \item $\int_{\mathcal{O}}f=0$ for every open set $\mathcal{O}$.
    \end{enumerate}
    \item Let $\mathcal{F}$ be a family of functions, each of which is integrable over $E$. Show that $\mathcal{F}$ is uniformly integrable over $E$ iff for each $\epsilon>0$, there is a $\delta>0$ such that for each $f\in\mathcal{F}$,
    \[
        \text{if }A\subseteq E\text{ is measurable and }m(A)<\delta,\text{ then }\biggl |\int_Af\biggr |<\epsilon.
    \]
    \item Let $\mathcal{F}$ be a family of functions, each of which is integrable over $E$. Show that $\mathcal{F}$ is uniformly integrable over $E$ iff for each $\epsilon>0$, there is a $\delta>0$ such that for each $f\in\mathcal{F}$,
    \[
        \text{if }\mathcal{U}\text{ is open and }m(E\cap\mathcal{U})<\delta,\text{ then }\int_{E\cap\mathcal{U}}|f|<\epsilon.
    \]
\end{enumerate}
% Chapter 5
\chapter{Lebesgue Integration: Further Topics}

% 5.1
\section{Uniform Integrability and Tightness: A General Vitali Convergence Theorem}
\begin{center}
	\textbf{PROBLEMS}
\end{center}
\begin{enumerate}
	\setcounter{enumi}{0}
    \item Let $\{f_n\}_{k=1}^n$ be a finite family of functions, each of which is integrable over $E$.
    Show that $\{f_n\}_{k=1}^n$ is uniformly integrable and tight over $E$.
    \item Prove Corollary 2.
    \item Let the sequences of functions $\{h_n\}$ and $\{g_n\}$ be uniformly integrable and tight over $E$.
    Show that for any $\alpha$ and $\beta$, $\{\alpha f_n +\beta g_n\}$ is also uniformly integrable and tight over $E$.
    \item Let $\{f_n\}$ be a sequence of measurable functions on $E$. Show that f$\{f_n\}$ is uniformly integrable and tight over $E$ iff for each $\epsilon>0$, there is a measurable subset $E_0$ of $E$ that has finite measure and a $\delta>0$ such that for each measurable subset $A$ of $E$ and index $n$,
    \[
        \text{if }m(A\cap E_0)<\delta,\text{ then }\int_A|f_n|<\epsilon.  
    \]
    \item Let $\{f_n\}$ be a sequence of measurable functions on $\mathbb{R}$. Show that f$\{f_n\}$ is uniformly integrable and tight over $\mathbb{R}$ iff for each $\epsilon>0$, there are positive numbers $r$ and $\delta$ such that for each open subset $\mathcal{O}$ of $\mathbb{R}$ and index $n$
    \[
        \text{if }m(\mathcal{O}\cap(-r,r))<\delta,\text{ then }\int_{\mathcal{O}}|f_n|<\epsilon.  
    \]
\end{enumerate}

% 5.2
\section{Convergence in Measure}
\begin{center}
	\textbf{PROBLEMS}
\end{center}
\begin{enumerate}
	\setcounter{enumi}{5}
    \item Let $\{f_n\}\to f$ in measure on $E$ and let $g$ be a measurable function on $E$ that is finite a.e. on $E$. 
    Show that $\{f_n\}\to g$ in measure on $E$ iff $f=g$ a.e. on $E$.
    \item Let $E$ have finite measure, let $\{f_n\}\to f$ in measure on $E$ and let $g$ be a measurable function on $E$ that is finite a.e. on $E$.
    Prove that $\{f_n\cdot g\}\to f\cdot g$ in measure, and use this to show that $\{f_n^2\}\to f^2$ in measure.
    Infer from this that if $\{g_n\}\to g$ in measure, then $\{f_n\cdot g_n\}\to f\cdot g$ in measure.
    \item Show that Fatou's Lemma, the Monotone Convergence Theorem, the Lebesgue Dominated Convergence Theorem, and the Vitali Convergence Theorem remain valid if "pointwise convergence a.e." is replaced by "convergence in measure".
    \item Show that Proposition 3 does not necessarily hold for sets $E$ of infinite measure.
    \item Show that linear combinations of sequences that converge in measure on a set of finite measure also converge in measure.
    \item Assume $E$ has finite measure. Let $\{f_n\}$ be a sequence of measurable functions on $E$ and let $f$ be a measurable function on $E$ for which $f$ and each $f_n$ is finite a.e. on $E$.
    Prove that $\{f_n\}\to f$ in measure on $E$ iff every subsequence of $\{f_n\}$ has in turn a further subsequence that converges to $f$ pointwise a.e. on $E$.
    \item Show that a sequence $\{a_j\}$ of real numbers converges to a real number if $|a_{j+1}-a_j|\le\frac{1}{2^j}$ for all $j$ by showing that the sequence $\{a_j\}$ must be Cauchy.
    \item A sequence $\{f_n\}$ of measurable functions on $E$ is said to be \textbf{Cauchy in measure} provided that given $\eta>0$ and $\epsilon>0$, there is an index $N$ such that for all $m,n\ge N$, 
    \[
        m\{x\in E\ |\ |f_n(x)-f_m(x)|\ge\eta\}<\epsilon.  
    \]
    Show that if $\{f_n\}$ is Cauchy in measure, then there is a measurable function $f$ on $E$ to which the sequence $\{f_n\}$ converges in measure.
    (Hint: choose a strictly increasing sequence of natural numbers $\{n_j\}$ such that for each index $j$, if $E_j=\{x\in E\ |\ |f_{n_{j+1}}(x)-f_{n_j}(x)|>\frac{1}{2^j}\}$, then $m(E_j)<\frac{1}{2^j}$. Now use the Borel-Cantelli Lemma and the preceding problem.)
    \item Assume $m(E)<\infty$. For two measurable functions $g$ and $h$ on $E$, Define
    \[
        \rho(g,h)=\int_E\frac{|g-h|}{1+|g-h|}.
    \]
    Show that $\{f_n\}\to f$ in measure on $E$ iff $\lim_{n\to\infty}\rho(f_n,f)=0$.
\end{enumerate}

% 5.3
\section{Characterizations of Riemann and Lebesgue Integrability}
\begin{center}
	\textbf{PROBLEMS}
\end{center}
\begin{enumerate}
	\setcounter{enumi}{14}
    \item Let $f$ and $g$ be bounded functions that are Riemann integrable over $[a,b]$. Show that the product $fg$ is also Riemann integrable over $[a,b]$.
    \item Let $f$ be a bounded function on $[a,b]$ whose set of discontinuities has measure zero. Show that $f$ is measurable. Then show that the same holds without the assumption of boundedness.
    \item Let $f$ be a function on $[0,1]$ that is continuous on $(0,1]$. Show that it is possible for the sequence $\{\int_{[1/n,1]}f\}$ to converge and yet $f$ is not Lebesgue integrable over $[0,1]$. Can this happen if $f$ is nonnegative?
\end{enumerate}
% Chapter 6
\chapter{Differentiation and Integration}

% 6.1
\section{Continuity of Monotone Functions}
\begin{center}
	\textbf{PROBLEMS}
\end{center}
\begin{enumerate}
	\setcounter{enumi}{0}
    \item Let $C$ be a countable subset of the nondegenerate closed, bounded interval $[a,b]$. Show that there is an increasing function on $[a,b]$ that is continuous only at points in $[a,b]\setminus C$.
    \item Show that there is a strictly increasing function on $[0,1]$ that is continuous only at the irrational numbers in $[0,1]$.
    \item Let $f$ be a monotone function on a subset $E$ of $\mathbb{R}$. Show that $f$ is continuous except possibly at a countable number of points in $E$.
    \item Let $E$ be a subset of $\mathbb{R}$ and let $C$ be a countable subset of $E$. Is there a monotone function on $E$ that is continuous only at points in $E\setminus C$?
\end{enumerate}

% 6.2
\section{Differentiability of Monotone Functions: Lebesgue's Theorem}
\begin{center}
	\textbf{PROBLEMS}
\end{center}
\begin{enumerate}
	\setcounter{enumi}{4}
    \item Show that the Vitali Covering Lemma does not extend to the case in which the covering collection has degenerate closed intervals.
    \item Show that the Vitali Covering Lemma does extend to the case in which the covering collection consists of nondegenerate general intervals.
    \item let $f$ be continuous on $\mathbb{R}$. Is there an open interval on which $f$ is monotone?
    \item Let $I$ and $J$ be closed, bounded intervals and $\gamma>0$ be such that $\ell(I)>\gamma\cdot\ell(J)$.
    Assume $I\cap J\neq\emptyset$. Show that if $\gamma\ge 1/2$, then $J\subseteq 5*I$, where $5*I$ denotes the interval with the same center as $I$ and five times its length.
    Is the same true if $0<\gamma<1/2$?
    \item Show that a set $E$ of real numbers has measure zero iff there is a countable collection of open intervals $\{I_k\}_{k=1}^\infty$ for which each point in $E$ belongs to infinitely many of the $I_k's$ and $\sum_{k=1}^\infty\ell(I_k)<\infty$.
    \item (Riesz-Nagy) Let $E$ be a set of measure zero contained in the open interval $(a,b)$.
    According to the preceding problem, there is a countable collection of open intervals contained in $(a,b)$, $\{(c_k,d_k)\}_{k=1}^\infty$, for which each point in $E$ belongs to infinitely many intervals in the collection and $\sum_{k=1}^\infty(d_k-c_k)<\infty$.
    Define
    \[
        f(x)=\sum_{k=1}^\infty\ell((c_k,d_k)\cap(-\infty,x))\text{ for all }x\text{ in }(a,b).
    \]
    Show that $f$ is increasing and fails to be differentiable at each point in $E$.
    \item For real numbers $\alpha<\beta$ and $\gamma>0$, show that if $g$ is integrable over $[\alpha+\gamma,\beta+\gamma]$, Then
    \[
        \int_{\alpha}^{\beta}g(t+\gamma)dt=\int_{\alpha+\gamma}^{\beta+\gamma}g(t)dt.  
    \]
    Prove this change of variables formula by successively considering simple functions, bounded measurable functions, nonnegative integrable functions, and general integrable functions.
    Use it to prove (14).
    \item Compute the upper and lower derivatives of the characteristic function of the rationals.
    \item Let $E$ be a set of finite outer measure and $\mathcal{F}$ a collection of closed, bounded intervals that cover $E$ in the sense of Vitali.
    Show that there is a countable disjoint collection $\{I_k\}_{k=1}^\infty$ of intervals in $\mathcal{F}$ for which
    \[
        m^*\biggl[E\setminus\bigcup_{k=1}^\infty I_k\biggr]=0.    
    \]
    \item Use the Vitali Covering Lemma to show that the union of any collection (countable or uncountable) of closed, bounded, nondegenerate intervals is measurable.
    \item Define $f$ on $\mathbb{R}$ by
    \[
        f(x)=
        \begin{cases}
            x\sin(1/x)&x\neq0\\
            0&x=0
        \end{cases}    
    \]
    Find the upper and lower derivatives of $f$ at $x=0$.
    \item Let $g$ be integrable over $[a,b]$. Define the antiderivative of $g$ of $g$ to be the function $f$ defined on $[a,b]$ by
    \[
        f(x)=\int_a^xg\text{ for all }x\in[a,b].    
    \]
    Show that $f$ is differentiable almost everywhere on $(a,b)$.
    \item Let $f$ be an increasing bounded function on the open, bounded interval $(a,b)$. Verify (18).
    \item Show that if $f$ is defined on $(a,b)$ and $c\in(a,b)$ is a local minimizer for $f$, then $\underline{D}f(c)\le0\le\overline{D}f(c)$.
    \item Let $f$ be continuous on $[a,b]$ with $\underline{D}f\ge0$ on $(a,b)$. Show that $f$ is increasing on $[a,b]$.
    (Hint: first show this for a function $g$ for which $\underline{D}g\ge\epsilon>0$ on $(a,b)$. Apply this to the function $g(x)=f(x)+\epsilon x$.)
    \item Let $f$ and $g$ be real-valued functions on $(a,b)$. Show That
    \[
        \underline{D}f+\underline{D}g\le\underline{D}(f+g)\le\overline{D}(f+g)\le \overline{D}(f)+\overline{D}(g)\text{ on }(a,b).
    \]
    \item Let $f$ be defined on $[a,b]$ and $g$ a continuous function on $[\alpha,\beta]$ that is differentiable at $\gamma\in(\alpha,\beta)$ with $g(\gamma)=c\in(a,b)$. Verify the following.
    \begin{enumerate}[label=(\roman*),align=left]
        \item If $g'(\gamma)>0$, then $\overline{D}(f\circ g)(\gamma)=\overline{D}f(c)\cdot g'(\gamma)$.
        \item If $g'(\gamma)=0$ and the upper and lower derivatives of $f$ at $c$ are finite, then $\overline{D}(f\circ g)(\gamma)=0$.
    \end{enumerate}
    \item Show that a strictly increasing function that is defined on an interval is measurable and then use this to show that a monotone function that is defined on an interval is measurable.
    \item Show that a continuous function $f$ on $[a,b]$ is Lipschitz if its upper and lower derivatives are bounded on $(a,b)$.
    \item Show that for $f$ defined in the last remark of this section, $f'$ is not integrable over $[0,1]$.
\end{enumerate}

% 6.3
\section{Functions of Bounded Variation: Jordan's Theorem}
\begin{center}
	\textbf{PROBLEMS}
\end{center}
\begin{enumerate}
	\setcounter{enumi}{24}
    \item Suppose $f$ is continuous on $[0,1]$. Must there be a nondegenerate closed subinterval $[a,b]$ of $[0,1]$ for which the restriction of $f$ to $[a,b]$ is of bounded variation?
    \item Let $f$ be the Dirichlet function, the characteristic function of the rationals in $[0,1]$. Is $f$ of bounded variation on $[0,1]$?
    \item Define $f(x)=\sin x$ on $[0,2\pi]$. Find two increasing functions $h$ and $g$ for which $f=h-g$ on $[0,2\pi]$.
    \item Let $f$ be a step function on $[a,b]$. Find a formula for its total variation.
    \item
    \begin{enumerate}[label=(\roman*),align=left]
        \item Define
        \[
            f(x)=
            \begin{cases}
                x^2\cos(1/x^2)&x\neq0,x\in[-1,1]\\
                0&x=0
            \end{cases}  
        \]
        Is $f$ of bounded variation on $[-1,1]$?
        \item Define
        \[
            g(x)=
            \begin{cases}
                x^2\cos(1/x)&x\neq0,x\in[-1,1]\\
                0&x=0
            \end{cases}  
        \]
        Is $g$ of bounded variation on $[-1,1]$?
    \end{enumerate}
    \item Show that the linear combination of two functions of bounded variation is also of bounded variation.
    Is the product of two such functions also of bounded variation?
    \item Let $P$ be a partition of $[a,b]$ that is a refinement of the partition $P'$. For a real-valued function $f$ on $[a,b]$, show that $V(f,P')\le V(f,P)$.
    \item Assume $f$ is of bounded variation on $[a,b]$. Show that there is a sequence of partitions $\{P_n\}$ of $[a,b]$ for which the sequence $\{V(f,P_n)\}$ is increasing and converges to $TV(f)$.
    \item Let $\{f_n\}$ be a sequence of real-valued functions on $[a,b]$ that converges pointwise on $[a,b]$ to the real-valued function $f$. Show that 
    \[
        TV(f)\le\lim\inf TV(f_n).  
    \]
    \item Let $f$ and $g$ be of bounded variation on $[a,b]$. Show that 
    \[
        TV(f+g)\le TV(f)+TV(g)\text{ and }TV(\alpha f)=|\alpha|TV(f).
    \]
    \item For $\alpha$ and $\beta$ positive numbers, define the function $f$ on $[0,1]$ by
    \[
        f(x)=
        \begin{cases}
            x^\alpha\sin(1/x^\beta)&\text{for }0<x\le1\\
            0&\text{for }x=0
        \end{cases}    
    \]
    Show that if $\alpha>\beta$, then $f$ is of bounded variation on $[0,1]$, by showing that $f'$ is integrable over $[0,1]$. Then show that if $\alpha\le\beta$, then $f$ is not of bounded variation on $[0,1]$.
    \item Let $f$ fail to be of bounded variation on $[0,1]$. Show that there is a point $x_0$ in $[0,1]$ such that there are subintervals of $[0,1]$ that contain $x_0$ and have arbitrarily small length on which $f$ fails to be of bounded variation.
\end{enumerate}

% 6.4
\section{Absolutely Continuous Functions}
\begin{center}
	\textbf{PROBLEMS}
\end{center}
\begin{enumerate}
	\setcounter{enumi}{36}
    \item Let $f$ be a continuous function on $[0,1]$ that is absolutely continuous on $[\epsilon,1]$ for each $0<\epsilon<1$.
    \begin{enumerate}[label=(\roman*),align=left]
        \item Show that $f$ may not be absolutely continuous on $[0,1]$.
        \item Show that $f$ is absolutely continuous on $[0,1]$ if it is increasing.
        \item Show that the function $f$ on $[0,1]$, defined by $f(x)=\sqrt(x)$ for $0\le x\le1$, is absolutely continuous, but not Lipschitz, on $[0,1]$.
    \end{enumerate}
    \item Show that $f$ is absolutely continuous on $[a,b]$ iff for each $\epsilon>0$, there is a $\delta>0$ such that for every countable disjoint collection $\{(a_k,b_k)\}_{k=1}^\infty$ of open intervals in $(a,b)$,
    \[
        \sum_{k=1}^\infty|f(b_k)-f(a_k)|<\epsilon\text{ if }\sum_{k=1}^\infty[b_k-a_k]<\delta.  
    \]
    \item Use the preceding problem to show that if $f$ is continuous and increasing on $[a,b]$, then $f$ is absolutely continuous on $[a,b]$ iff for each $\epsilon$, there is a $\delta>0$ such that for a measurable subset $E$ of $[a,b]$,
    \[
        m^*(f(E))<\epsilon\text{ if }m(E)<\delta.  
    \]
    \item Use the preceding problem to show that an increasing absolutely continuous function $f$ on $[a,b]$ maps sets of measure zero onto sets of measure zero.
    Conclude that the Cantor-Lebesgue function $\varphi$ is not absolutely continuous on $[0,1]$ since the function $\psi$, defined by $\psi(x)=x+\varphi(x)$ for $0\le x\le1$, maps the Cantor set to a set of measure 1 (page 52).
    \item Let $f$ be an increasing absolutely continuous function on $[a,b]$. Use (i) and (ii) below to conclude that $f$ maps measurable sets to measurable sets.
    \begin{enumerate}[label=(\roman*),align=left]
        \item Infer from the continuity of $f$ and the compactness of $[a,b]$ that $f$ maps closed sets to closed sets and therefore maps $F_\sigma$ sets to $F_\sigma$ sets.
        \item The preceding problem tells us that $f$ maps sets of measure zero to sets of measure zero. 
    \end{enumerate}
    \item Show that both the sum and product of absolutely continuous functions are absolutely continuous.
    \item Define the functions $f$ and $g$ on $[-1,1]$ by $f(x)=x^{\frac{1}{3}}$ for $-1\le x\le 1$ and
    \[
        g(x)=
        \begin{cases}
            x^2\cos(\pi/2x)&\text{if }x\neq0,x\in[-1,1]\\
            0&\text{if }x=0
        \end{cases}  
    \]
    \begin{enumerate}[label=(\roman*),align=left]
        \item Show that both $f$ and $g$ are absolutely continuous on $[-1,1]$.
        \item For the partition $P_n=\{-1,0,1/2n,1/[2n-1],\cdots,1/3,1/2,1\}$ of $[-1,1]$, examine $V(f\circ g,P_n)$.
        \item Show that $f\circ g$ fails to be of bounded variation, and hence also fails to be absolutely continuous, on $[-1,1]$. 
    \end{enumerate}
    \item Let $f$ be Lipschitz on $\mathbb{R}$ and $g$ be absolutely continuous on $[a,b]$. Show that the composition $f \circ g$ is absolutely continuous on $[a,b]$.
    \item Let $f$ be absolutely continuous on $\mathbb{R}$ and $g$ be absolutely continuous and strictly monotone on $[a,b]$. Show that the composition $f\circ g$ is absolutely continuous on $[a,b]$.
    \item Verify the assertions made in the final remark of this section.
    \item Show that a function $f$ is absolutely continuous on $[a,b]$ iff for each $\epsilon>0$, there is a $\delta>0$ such that for every finite disjoint collection $\{(a_k,b_k)\}_{k=1}^n$ of open intervals in $(a,b)$,
    \[
        \biggl|\sum_{k=1}^n[f(b_k)-f(a_k)]\biggr|<\epsilon\text{ if }\sum_{k=1}^n[b_k-a_k]<\delta.  
    \]
\end{enumerate}

% 6.5
\section{Integrating Derivatives: Differentiating Indefinite Integrals}

\begin{center}
	\textbf{PROBLEMS}
\end{center}
\begin{enumerate}
	\setcounter{enumi}{47}
    \item The Cantor-Lebesgue function $\varphi$ is continuous and increasing on $[0,1]$. Conclude from Theorem 10 that $\varphi$ is not absolutely continuous on $[0,1]$.
    Compare this reasoning with that proposed in Problem 40.
    \item Let $f$ be continuous on $[a,b]$ and differentiable almost everywhere on $(a,b)$. Show that 
    \[
        \int_a^bf'=f(b)-f(a)  
    \]
    if and only if 
    \[
        \int_a^b[\lim_{n\to\infty}\text{Diff}_{1/n}f]=\lim_{n\to\infty}[\int_a^b\text{Diff}_{1/n}f].
    \]
    \item Let $f$ be continuous on $[a,b]$ and differentiable almost everywhere on $(a,b)$. Show that if $\{\text{Diff}_{1/n}f\}$ is uniformly integrable over $[a,b]$, then
    \[
        \int_a^bf'=f(b)-f(a).  
    \]
    \item Let $f$ be continuous on $[a,b]$ and differentiable almost everywhere on $(a,b)$. Suppose there is a nonnegative function $g$ that is integrable over $[a,b]$ and 
    \[
        |\text{Diff}_{1/n}f|\le g\text{ a.e. on }[a,b]\text{ for all }n.  
    \]
    Show that 
    \[
        \int_a^bf'=f(b)-f(a).  
    \]
    \item Let $f$ and $g$ be absolutely continuous on $[a,b]$. Show that
    \[
        \int_a^bf\cdot g'=f(b)g(b)-f(a)g(a)-\int_a^bf'\cdot g.  
    \] 
    \item Let the function $f$ be absolutely continuous on $[a,b]$. Show that $f$ is Lipschitz on $[a,b]$ iff there is a $c>0$ for which $|f'|\le c$ a.e. on $[a,b]$.
    \item 
    \begin{enumerate}[label=(\roman*),align=left]
        \item Let $f$ be a singular increasing function on $[a,b]$. Use the Vitali Covering Lemma to show that $f$ has the following property: Given $\epsilon>0,\delta>0$, there is a finite disjoint collection $\{(a_k,b_k)\}_{k=1}^n$ of open intervals in $(a,b)$ for which
        \[
            \sum_{k=1}^n[b_k-a_k]<\delta\text{ and }\sum_{k=1}^n[f(b_k)-f(a_k)]>f(b)-f(a)-\epsilon.  
        \]
        \item Let $f$ be an increasing function on $[a,b]$ with the property described in part (i). Show that $f$ is singular.
        \item Let $\{f_n\}$ be a sequence of singular increasing functions on $[a,b]$ for which the series $\sum_{n=1}^\infty f_n(x)$ converges to a finite value for each $x\in[a,b]$. Define 
        \[
            f(c)=\sum_{n=1}^\infty f_n(x)\text{ for }x\in[a,b].  
        \]
        Show that $f$ is also singular.
    \end{enumerate}
    \item Let $f$ be of bounded variation on $[a,b]$, and define $v(x)=TV(f_{[a,x]})$ for all $x\in[a,b]$.
    \begin{enumerate}[label=(\roman*),align=left]
        \item Show that $|f'|\le v'$ a.e. on $[a,b]$, and infer from this that 
        \[
            \int_a^b|f'|\le TV(f).  
        \]
        \item Show that the above is an equality iff $f$ is absolutely continuous on $[a,b]$.
        \item Compare parts (i) and (ii) with Corollaries 4 and 12, respectively.
    \end{enumerate}
    \item Let $g$ be strictly increasing and absolutely continuous on $[a,b]$.
    \begin{enumerate}[label=(\roman*),align=left]
        \item Show that for any open subset $\mathcal{O}$ of $(a,b)$,
        \[
            m(g(\mathcal{O}))=\int_{\mathcal{O}}g'(x)dx.  
        \]
        \item Show that for any $G_\delta$ subset $E$ of $(a,b)$,
        \[
            m(g(E))=\int_Eg'(x)dx.  
        \]
        \item Show that for any subset $E$ of $[a,b]$ that has measure 0, its image $g(E)$ also has measure 0, so that
        \[
            m(g(E))=0=\int_{E}g'(x)dx.  
        \]
        \item Show that for any measurable subset $A$ of $[a,b]$,
        \[
            m(g(A))=\int_{A}g'(x)dx.  
        \]
        \item Let $c=g(a)$ and $d=g(b)$. Show that for any simple function $\varphi$ on $[c,d]$,
        \[
            \int_c^d\varphi(y)dy=\int_a^b\varphi(g(x))g'(x)dx.
        \]
        \item Show that for any nonnegative integrable function $f$ over $[c,d]$,
        \[
            \int_c^df(y)dy=\int_a^bf(g(x))g'(x)dx.
        \]
        \item Show that part (i) follows from (vi) in the case that $f$ is the characteristic function of $g(\mathcal{O})$ and the composition is defined.
    \end{enumerate}
    \item Is the change of variables formula in part (vi) of the preceding problem true if we just assume $g$ is increasing, not necessarily strictly?
    \item Construct an absolutely continuous strictly increasing function $f$ on $[0,1]$ for which $f'=0$ on a set of positive measure.
    (Hint: Let $E$ be the relative complement in $[0,1]$ of a generalized Cantor set of positive measure and $f$ the indefinite integral of $\chi_E$. See Problem 39 of Chapter 2 for the construction of such a Cantor set.)
    \item For a nonnegative integrable function $f$ over $[c,d]$, and a strictly increasing absolutely continuous function $g$ on $[a,b]$ such that $g([a,b])\subseteq [c,d]$, is it possible to justify the change of variables formula
    \[
        \int_{g(a)}^{g(b)}f(y)dy=\int_a^bf(g(x))g'(x)dx  
    \]
    by showing that
    \[
        \frac{d}{dx}\biggl[\int_{g(a)}^{g(x)}f(s)ds-\int_a^xf(g(t))g'(t)dt\biggr]=0\text{ for almost all }x\in(a,b)?  
    \]\
    \item Let $f$ be absolutely continuous and singular on $[a,b]$. Show that $f$ is constant. Also show that the Lebesgue decomposition of a function of bounded variation is unique if the singular function is required to vanish a t $x=a$.
\end{enumerate}

% 6.6
\section{Convex Functions}

\begin{center}
	\textbf{PROBLEMS}
\end{center}
\begin{enumerate}
	\setcounter{enumi}{60}
	\item Show that a real-valued function $\varphi$ on $(a,b)$ is convex iff for points $x_1,\cdots,x_n$ in $(a,b)$ and nonnegative numbers $\lambda_1,\cdots,\lambda_n$ such that $\sum_{k=1}^n \lambda_k=1$,
	\[
        \varphi \biggr ( \sum_{k=1}^n \lambda_k x_k \biggl ) \le \sum_{k=1}^n \lambda_k \varphi(x_k).   
    \]
    Use this to directly prove Jensen's Inequality for $f$ a simple function.
    \item Show that a continuous function on $(a,b)$ is convex iff
    \[
        \varphi(\frac{x_1+x_2}{2})\le\frac{\varphi(x_1)+\varphi(x_2)}{2}\text{ for all }x_1,x_2\in (a,b).  
    \]
    \item A function on a general interval $I$ is said to be convex provided it is continuous on $I$ and (38) holds for all $x_1,x_2\in I$.
    Is a convex function on a closed, bounded interval $[a,b]$ necessarily Lipschitz on $[a,b]$?
    \item Let $\varphi$ have a second derivative at each point in $(a,b)$.
    Show that $\varphi$ is convex iff $\varphi''$ is nonnegative.
    \item Suppose $a\ge 0$ and $b\ge 0$.
    Show that the function $\varphi(t)=(a+bt)^p$ is convex on $[0,\infty)$ for $1\le p < \infty$.
    \item For what functions $\varphi$ is Jensen's Inequality always an equality?
    \item State and prove a version of Jensen's Inequality on a general closed, bounded interval $[a,b]$.
    \item Let $f$ be integrable over $[0,1]$. Show that 
    \[
    \exp\biggl [\int_0^1f(x)dx\biggr ] \le \int_0^1 \exp(f(x))dx.   
    \]
    \item Let $\{\alpha_n\}$ be a sequence of nonnegative numbers whose sum is $1$ and $\{\zeta_n\}$ is a sequence of positive numbers. Show that
    \[
    \prod_{n=1}^\infty \zeta_n^{\alpha_n} \le \sum_{n=1}^\infty \alpha_n \zeta_n. 
    \]
    \item Let $g$ be a positive measurable function on $[0,1]$. Show that $\log(\int_0^1g(x)dx) \ge\int_0^1\log(g(x))dx$ whenever each side is defined.
    \item (Nemytskii) Let $\varphi$ be a continuous function on $\mathbb{R}$.
    Show that if there are constants for which (43) holds, then $\varphi\circ f$ is integrable over $[0,1]$ whenever $f$ is.
    Then show that if $\varphi\circ f$ is integrable over $[0,1]$ whenever $f$ is, then there are constants $c_1$ and $c_2$ for which (43) holds.
\end{enumerate}
% Chapter 7
\authoredby{finished}
\chapter{The $L^p$ Spaces: Completeness and Approximation}

% 7.1
%\authoredby{finished}
\section{Normed Linear Spaces}
\begin{flushleft}
	
	Throughout this chapter $E$ denotes a measurable set of real numbers.
	Define $\mathcal{F}$ to be the collection of all measurable extended real-valued functions on $E$ that are finite a.e. on $E$.
	We can say that two functions $f,g\in\mathcal{F}$ are equivalent, denoted by $f\cong g$, provided
	\[
	f(x)=g(x)\text{ for almost all }x\in E.	
	\]
	This is an equivalence relation and induces a partition of $\mathcal{F}$ into a disjoint collection of equivalence classes, denoted by $\mathcal{F}/\cong$, which is a linear space.
	There is a natural family $\{L^p(E)\}_{1\le p\le\infty}$ of subspaces of $\mathcal{F}/\cong$.
	\\
	For $1\le p<\infty$, define $L^p(E)$ to be the collection of equivalence class $[f]$ for which 
	\[
	\int_E|f|^p<\infty.	
	\]
	Then if $f\cong g$, then $\int_E|f|^p=\int_E|g|^p$.
	Showing that $L^p(E)$ is closed under linear combinations will prove that $L^p(E)$ is a linear subspace.
	To do this, let $c=\max\{|a|,|b|\}$ so that
	\[
	|a+b|\le|a|+|b|\le 2c,
	\]
	which implies
	\[
	|a+b|^p\le 2^pc^p\le2^p(|a|^p+|b|^p).	
	\]
	This inequality, together with the linearity and monotonicity of integration tells us that 
	\[
		\int_E|\alpha f+\beta g|^p \le 2^p(|\alpha|^p\int_E|f|^p+|\beta|^p\int_E| g|^p)<\infty.
	\]
	That is, for $[f],[g]\in L^p(E)$, then $\alpha[f]+\beta[g]\in L^p(E)$.\\
	We call a function $f\in\mathcal{F}$ \textbf{essentially bounded} provided there is some $M\ge 0$, called an \textbf{essential upper bound} for $f$, for which
	\[
	|f(x)|\le M\text{ for almost all }x\in E.	
	\]
	Then we can define $L^\infty(E)$ to be the collection of equivalence classes $[f]$ for which $f$ is essentially bounded. 
	Clearly $L^\infty(E)$ is a linear subspace because
	\[
	|\alpha f(x)+\beta g(x)|\le|\alpha| |f(x)|+|\beta|| g(x)|\le |\alpha| M + |\beta|M' = M'' \text{ a.e. on $E$}
	\]
	To state that a function $f$ in $L^p[a,b]$ is continuous means that there is a continuous function that agrees with $f$ a.e. on $[a,b]$.
	There is only one such continuous function and it is often convenient to consider this unique continuous function as the representative of $[f]$.\\
	It is useful to consider real-valued functions that have as their domain linear spaces of functions: such functions are called \textbf{functionals}.

	\begin{namedthm*}{Definition}
		Let $X$ be a linear space.
		A real-valued functional $\|\cdot\|$  on $X$ is called a \textbf{norm} provided for each $f$ and $g$ in $X$ and each real number $\alpha$,\\
		(The Triangle Inequality)
		\[
			\|f+g\|\le\|f\|+\|g\|,
		\]
		(Positive Homogeneity)
		\[
			\|\alpha f\|=|\alpha|\|f\|,	
		\]
		(Nonnegativity)
		\[
			\|f\|\ge 0\text{ and }\|f\|=0 \iff f=0.
		\]
	\end{namedthm*}
	A \textbf{normed linear space} is a linear space together with a norm.
	If $X$ is a linear space normed by $\|\cdot\|$ we say that a function $f$ in $X$ is a \textbf{unit function} provided $\|f\|=1$.
	For any $f\in X,f\neq 0$, the function $\frac{f}{\|f\|}$ is a unit function: it is a scalar multiple of $f$ which we call the \textbf{normalization} of $f$.

	\begin{namedthm*}{Example}[The Normed Linear Space $L^1(E)$]
		For a function $f$ in $L^1(E)$, define
		\[
		\|f\|_1=\int_E|f|.	
		\]
		Then $\|\cdot\|$ is a norm on $L^1(E)$.\\
		For $f,g\in L^1(E)\subseteq \mathcal{F}$, since $f$ and $g$ are finite a.e. on $E$, the triangle inequality for real numbers tells us that
		\[
		|f+g|\le|f|+|g|\text{ a.e. on }E.	
		\]
		Then by the monotonicity and linearity of integration, we have subadditivity:
		\[
			\|f+g\|_1=\int_E|f+g| \le \int_E[|f|+|g|] = \int_E|f| +\int_E|g| = \|f\|_1+\|g\|_1.	
		\]
		By the linearity of integration, clearly we have absolute homogeneity:
		\[
			\|\alpha f\|_1 = \int_E|\alpha f| = \int_E|\alpha| |f|=|\alpha|\int_E| f| = |\alpha|\|f\|_1.	
		\]
		Clearly $\|f\|$ is nonnegative. Finally, if $f\in L^1(E)$ and $\|f\|_1=0$, then $f=0$ a.e. on $E$. 
		Therefore $[f]$ is the zero element of the linear space $L^1(E)\subseteq \mathcal{F}/\cong$, that is $f=0$.
	\end{namedthm*}

	\begin{namedthm*}{Example}[The Normed Linear Space $L^\infty(E)$]
		For a function $f$ in $L^\infty(E)$, define $\|f\|_\infty$ to be the infimum of the essential upper bounds for $f$.
		\[
			\|f\|_\infty = \inf\{M\ :\ |f(x)|\le M\text{ a.e. on }E\}.
		\]
		We call $\|f\|_\infty$ the \textbf{essential supremum} of $f$ and claim that $\|\cdot\|_\infty$ is a norm on $L^\infty(E)$.
		\\Nonnegativity and positive homogeneity are clear. 
		\\To show that the triangle inequality holds, we see that for each natural number $n$, there is a subset $E_n$ of $E$ for which
		\[
		|f|\le \|f\|_\infty + \frac{1}{n}\text{ on }E\setminus E_n\text{ and }m(E_n)=0.	
		\]
		This is true because $\|f\|_\infty$ is the infimum, the greatest lower bound, so $\|f\|_\infty + \frac{1}{n}$ is not a lower bound and thus there exists a real number $M$ in the set of upper bounds a.e. of $f$ for which $\|f\|_\infty\le M < \|f\|_\infty + \frac{1}{n}$ a.e. on $E$, and so $|f|\le M < \|f\|_\infty + \frac{1}{n}$ a.e. on $E$.
		\\Accepting that the union of sets of measure zero is also measure zero, we can let $E_\infty = \bigcup_{n=1}^\infty E_n$, and so 
		\[
		|f|\le \|f\|_\infty \text{ on }E\setminus E_\infty\text{ and }m(E_n\infty)=0.	
		\]
		Thus we have that $|f|\le \|f\|_\infty$ a.e. on $E$; i.e., ess. sup$f$ is the smallest essential upper bound for $f$.
		\\Now, for $f,g\in L^\infty(E)$, 
		\[
			|f+g|\le|f|+|g|\le\|f\|_\infty+\|g\|_\infty\text{ a.e. on }E.
		\]
		Therefore $\|f\|_\infty+\|g\|_\infty$ is an essential bound for $f+g$ and thus the smallest essential upper bound, $\|f+g\|_\infty$, is such that
		\[
		\|f+g\|_\infty \le \|f\|_\infty+\|g\|_\infty.	
		\]
	\end{namedthm*}

	\begin{namedthm*}{Example}[The Normed Linear Spaces $\ell^1$ and $\ell^\infty$]
		For $1\le p<\infty$, define $\ell^p$ to be the collection of real sequences $a=(a_1,a_2,\cdots)$ for which 
		\[
		\sum_{k=1}^\infty|a_k|^p<\infty .	
		\]
		Let $a,b \in \ell^p$, and let $\alpha , \beta$ be real numbers.
		Then we have that $\sum_{k=1}^\infty|a_k|^p<\infty$ and $\sum_{k=1}^\infty|b_k|^p<\infty$.
		Using the inequality $|a+b|^p\le2^p(|a|^p+|b|^p)$, we have
		\begin{align*}
			\sum_{k=1}^\infty|\alpha a_k + \beta b_k|^p &\le \sum_{k=1}^\infty[2^p(|\alpha a_k|^p + |\beta b_k|^p)]\\ 
			&=\sum_{k=1}^\infty2^p|\alpha|^p |a_k|^p+\sum_{k=1}^\infty2^p|\beta|^p| b_k|^p\\
			&=2^p|\alpha|^p\sum_{k=1}^\infty |a_k|^p+2^p|\beta|^p\sum_{k=1}^\infty| b_k|^p\\
			&< 2^p|\alpha|^p\infty+2^p|\beta|^p\infty\\
			&=\infty.
		\end{align*}
		Thus $\ell^p$ is a linear space.\\
		We define $\ell^\infty$ to be the linear space of real bounded sequences: that is, for any $\{a_k\}$ in $\ell^\infty$, there exists a real number $M$ for which $|a_k|\le M$ for all $k$.
		We can define the following norms:
		\\For $\{a_k\}\in\ell^1$:
		\[
			\|\{a_k\}\|_1 = \sum_{k=1}^\infty|a_k|
		\]
		For $\{a_k\}\in\ell^\infty$:
		\[
			\|\{a_k\}\|_\infty = \sup_{1\le k<\infty}|a_k|
		\]
	\end{namedthm*}

	\begin{namedthm*}{Example}[The Normed Linear Space $C\lbrack a,b\rbrack$]
		Let $[a,b]$ be a closed, bounded interval. 
		The the linear space of continuous real-valued functions on $[a,b]$ is denoted by $C[a,b]$. 
		Since a continuous function on a compact set takes on a maximum value (ch1 problem 52), we can define
		\[
		\|f\|_{\max}=\max_{x\in [a,b]}|f(x)|.	
		\]		
	\end{namedthm*}

\end{flushleft}
\begin{center}
	\textbf{PROBLEMS}
\end{center}
\begin{enumerate}
	\setcounter{enumi}{0}
	\item For $f$ in $C[a,b]$, Define
	\[
	\| f \|_1 = \int_a^b |f|.	
	\]
	Show that this is a norm on $C[a,b]$.
	Also show that there is no number $c \ge 0$ for which
	\[
	\| f \|_{\max}	\le c \| f \|_1 \text{ for all $f$ in $C[a,b]$},
	\]
	but there is a $c \ge 0$ for which 
	\[
	\| f \|_1	\le c \| f \|_{\max} \text{ for all $f$ in $C[a,b]$}.
	\]
	\\
	Let $f,g\in C[a,b]$. For each $x\in [a.b]$, we have the inequality $|f(x)+g(x)|\le|f(x)|+|g(x)|$, so by monotonicity and linearity of integration,
	\[
	\|f+g\|_1=\int_a^b|f(x)+g(x)| \le \int_a^b[|f(x)|+|g(x)|] = \int_a^b|f(x)| +\int_a^b|g(x)| = \|f\|_1+\|g\|_1.	
	\]
	Therefore subadditivity holds.\\
	Also, by linearity of integration, we have
	\[
	\|\alpha f\|_1 = \int_a^b|\alpha f| = \int_a^b|\alpha| |f|=|\alpha|\int_a^b| f| = |\alpha|\|f\|_1.	
	\]
	Therefore absolute homogeneity holds.\\
	Finally, by definition of absolute value, $0 \le |f(x)|$ for all $x\in [a,b]$, and by monotonicity of integration,
	\[
	0=\int_a^b 0 \le \int_a^b |f| = \|f\|_1.	
	\] 
	Clearly $\int_a^b |f| = 0$ iff $f\equiv 0$ on $[a,b]$.
	Therefore positive definiteness holds.\\
	Thus $\|\cdot\|_1$ is a norm on $C[a,b]$.\\
	\\Consider the interval $[a,b]=[0,1]$.
	For any $c>0$ we choose, there exists an $n\in \mathbb{N}$ such that $n> c$, with the continuous function $f_n:[0,1]\to \mathbb{R}$ defined as
	\[ 
		f_n(x) =
		\begin{cases} 
			\frac{n-0}{1/n-0}x& \text{ if } x \in [0,\frac{1}{n}]\\
			\frac{0-n}{2/n-1/n}(x-\frac{1}{n})+n & \text{ if } x \in (\frac{1}{n},\frac{2}{n}]\\
			0& \text{ if } x \in (\frac{2}{n},1]
		\end{cases}
		=
		\begin{cases} 
			n^2x& \text{ if } x \in [0,\frac{1}{n}]\\
			-n^2(x-\frac{1}{n})+n & \text{ if } x \in (\frac{1}{n},\frac{2}{n}]\\
			0& \text{ if } x \in (\frac{2}{n},1]
		\end{cases}
	\]
	(This is a triangle-shaped function that reaches its peak $n$ at $x=\frac{1}{n}$.)\\
	Now, for any $n$, we have $\|f_n\|_1 = \int_0^1|f_n|=1$, and $\|f_n\|_{\max} = n$.\\
	Then $\| f_n \|_{\max}=n > c =  c \| f_n\|_1$.\\
	\\
	Finally, we can see that for any $f$ in $C[a,b]$, by monotonicity of the integral, 
	\begin{align*}
	\|f\|_1 &= \int_a^b|f(x)|\\ 
	&\le \int_a^b\max_{x\in [a,b]}|f(x)|\\
	&=\max_{x\in [a,b]}|f(x)| \int_a^b1\\
	&= \max_{x\in [a,b]}|f(x)| \cdot m([a,b]) \\
	&= \|f\|_{\max} \cdot m([a,b]).
	\end{align*}
	Therefore $\|f\|_1\le m([a,b])\|f\|_{\max}$ for all $f\in C[a,b]$.
	\item Let $X$ be the family of all polynomials with real coefficients defined on $\mathbb{R}$.
	Show that this is a linear space. For a polynomial $p$, define $\| p\|$ to be the sum of the absolute values of the coefficients of $p$.
	Is this a norm?\\
	For any two polynomials $p,q\in X$, there exists natural numbers $n,m$ (suppose without loss of generality that $n\le m$) such that
	\begin{align*}
	p(x) &= a_0+a_1x+a_2x^2+\cdots+a_{n-1}x^{n-1}+a_nx^n+\cdots+0x^m	\\
	q(x) &= b_0+b_1x+b_2x^2+\cdots+b_{n-1}x^{n-1}+b_nx^n+\cdots+b_mx^m	
	\end{align*}
	Now, considering any scalars $\alpha,\beta \in \mathbb{R}$, we have
	\begin{align*}
		\alpha p(x) + \beta q(x) &= \alpha (a_0+a_1x+a_2x^2+\cdots+a_{n-1}x^{n-1}+a_nx^n)\\
		&+ \beta (b_0+b_1x+b_2x^2+\cdots+b_{m-1}x^{m-1}+b_mx^m)\\
		&=(\alpha a_0)+(\alpha a_1)x+(\alpha a_2)x^2+\cdots+(\alpha a_{n-1})x^{n-1}+(\alpha a_n)x^n\\
		&+ (\beta b_0)+(\beta b_1)x+(\beta b_2)x^2+\cdots+(\beta b_{n-1})x^{n-1}+(\beta b_n)x^n+\cdots+(\beta b_m)x^m\\
		&=(\alpha a_0+\beta b_0)+(\alpha a_1+\beta b_1)x+\cdots+(\alpha a_n+\beta b_n)x^n+\cdots+(\beta b_m)x^m
	\end{align*}
	This is also a polynomial, as for each $i$, we have $(\alpha a_i+\beta b_i) \in \mathbb{R}$, so $X$ is a linear space.\\
	Now, for any polynomial
	\[
		p(x) = a_0+a_1x+a_2x^2+\cdots+a_nx^n,	
	\]
	we can define $\|p\| = |a_0|+|a_1|+|a_2|+\cdots+|a_n| = \sum_{i=0}^n|a_i|$.\\
	The triangle inequality is clear because 
	\[
		\|p+q\| = \sum_{i=0}^{m}|a_i+b_i|\le\sum_{i=0}^{m}[|a_i|+|b_i|]=\sum_{i=0}^{m}|a_i|+\sum_{i=0}^{m}|b_i|=\|p\|+\|q\|.
	\]
	Absolute homogeneity is clear because
	\[
		\|\alpha p\| = \sum_{i=0}^n|\alpha a_i|= \sum_{i=0}^n|\alpha|| a_i|=|\alpha|\sum_{i=0}^n| a_i|=|\alpha|\|p\|.
	\]
	Finally, positive definiteness is clear because
	\[
		0 \le |a_i| \implies 0\le \sum_{i=0}^n| a_i|=\|p\|,
	\]
	And $\|p\|=0$ if and only if $p(x)= 0+0x+0x^2+\cdots0x^n=0$.
	\item For $f$ in $L^1[a,b]$, define $\|f\| = \smallint_a^b x^2 |f(x)|dx$.
	Show that this is a norm on $L^1[a,b]$.\\
	For $f\in L^1[a,b]$, then $f$ is measurable and finite a.e. on $[a,b]$, and $\int_a^b|f(x)|dx<\infty$.\\
	Let $f,g\in L^1[a,b]$, and let $\alpha$ be a real number.\\
	Because the triangle inequality holds a.e. on $[a,b]$, by monotonicity and linearity of the integral, we have
	\begin{align*}
	\|f+g\|&=\int_a^b x^2 |f(x)+g(x)|dx\\
	&\le\int_a^b x^2 [|f(x)|+|g(x)|]dx\\
	&=\int_a^b [x^2 |f(x)|+x^2|g(x)|]dx\\
	&=\int_a^b x^2 |f(x)|dx+\int_a^b x^2 |g(x)|dx\\
	&= \|f\|+\|g\|.	
	\end{align*}
	Therefore $\|\cdot\|$ is subadditive.\\
	By linearity of the integral, we have
	\[
	\|\alpha f\| = \int_a^bx^2|\alpha f(x)|dx = \int_a^bx^2|\alpha| |f(x)|dx=|\alpha |\int_a^bx^2 |f(x)|dx	=|\alpha |\|f\|.
	\]
	Therefore $\|\cdot\|$ satisfies absolute homogeneity.\\
	We can use the fact that $0\le x^2$ and $0\le |f(x)|$ implies $0\le x^2|f(x)|$.
	By monotonicity of the integral, we have
	\[
	0=\int_a^b0dx \le \int_a^bx^2|f(x)|dx = \|f\|.
	\]
	Clearly $\|f\|=0$ if and only if $f=0$ a.e. on $[a,b]$ because $x^2\cdot 0 = 0$.\\
	Therefore $\|\cdot \|$ satisfies positive definiteness.
	\item For $f$ in $L^\infty[a,b]$, show that 
	\[
	\| f\|_\infty = \min \biggl \{ M \ \biggl |\ m \{x \in [a,b]\ |\ |f(x)| > M \} =0 \biggr \},
	\] 
	\\
	That is, the sup norm is the smallest real number $M$ such that $|f(x)|>M$ only on a set of measure zero.
	In an above example, we showed that $\|f\|_\infty$ is the smallest essential upper bound for $f$.
	That is, $|f|\le\|f\|_\infty$ a.e. on $E$ (That is, the inequality is true for $E\setminus E_0$, where $m(E_0)=0$.)
	\\
	and if, furthermore, $f$ is continuous on $[a,b]$, that
	\[
	\| f \|_{\infty} = \| f \|_{\max}.	
	\]
	If $f$ is continuous, then there are no jump discontinuities ($f$ is continuous at $x_0$ iff $f(x_0^-)=f(x_0)=f(x_0^+)$).
	Then $|f|\le\|f\|_\infty$ everywhere on $E$.
	\item Show that $\ell^\infty$ and $\ell^1$ are normed linear spaces.\\
	$\ell^\infty$:\\
	Let $a,b \in \ell^\infty$, and let $\alpha , \beta$ be real numbers.\\
	Then for some real numbers $M,N$, we have that $|a_k|\le M$ and $|b_k|\le N$ for all $k$.
	\begin{align*}
		\alpha a + \beta b &= \alpha (a_1,a_2,\cdots)+ \beta (b_1,b_2,\cdots)\\		
		&= (\alpha a_1,\alpha a_2,\cdots)+ (\beta b_1,\beta b_2,\cdots)\\
		&= (\alpha a_1+\beta b_1,\alpha a_2+\beta b_2,\cdots)
	\end{align*}
	Then $|\alpha a_k+\beta b_k|\le \alpha M + \beta N$ for all $k$, and $\ell^\infty$ is a linear space.\\
	To show that $\|a\|_\infty = \sup_{1\le k<\infty}|a_k|$ is a norm:
	\[
		\|a+b\|_\infty = \sup_{1\le k<\infty}|a_k+b_k|\le \sup_{1\le i<\infty}|a_i| + \sup_{1\le j<\infty}|b_j| = \|a\|_\infty + \|b\|_\infty,
	\]
	\[
		\|\alpha a\|_\infty = \sup_{1\le k<\infty}|\alpha a_k| = \sup_{1\le k<\infty}|\alpha|| a_k|= |\alpha|\sup_{1\le k<\infty}| a_k|=|\alpha|\|a\|_\infty, 
	\]
	\[
		0 \le \sup_{1\le k<\infty}| a_k| = \|a\|_\infty,\text{ and }\sup_{1\le k<\infty}| a_k|=0\text{ iff }a_k=0\text{ for all }k.	
	\]
	\\
	$\ell^1$:\\
	Let $a,b \in \ell^1$, and let $\alpha , \beta$ be real numbers.\\
	Then we have that $\sum_{k=1}^\infty|a_k|<\infty$ and $\sum_{k=1}^\infty|b_k|<\infty$.\\
	By the triangle inequality for real numbers, we have
	\[
		\sum_{k=1}^\infty |\alpha a_k + \beta b_k| \le\sum_{k=1}^\infty[ |\alpha|| a_k| + | \beta ||b_k|]= |\alpha|\sum_{k=1}^\infty | a_k| + |\beta |\sum_{k=1}^\infty |b_k| <|\alpha|\infty+|\beta|\infty = \infty.
	\]
	Therefore $\ell^1$ is a linear space.\\
	To show that $\|a\|_1 = \sum_{k=1}^\infty|a_k|$ is a norm:
	\[
		\|a+b\|_1 = \sum_{k=1}^\infty|a_k+b_k|\le\sum_{k=1}^\infty[|a_k|+|b_k|]=\sum_{k=1}^\infty|a_k|+\sum_{k=1}^\infty|b_k|<\infty +\infty = \infty,
	\]
	\[
		\|\alpha a\|_1 = \sum_{k=1}^\infty|\alpha a_k| = \sum_{k=1}^\infty|\alpha|| a_k|=|\alpha|\sum_{k=1}^\infty| a_k|=|\alpha|\|a\|_1, 
	\]
	\[
		0 \le | a_k| \implies 0 \le \sum_{k=1}^\infty |a_k| = \|a\|_1,\text{ and }\sum_{k=1}^\infty| a_k|=0\text{ iff }a_k=0\text{ for all }k.	
	\]
\end{enumerate}

% 7.2
\authoredby{inprogress}
\section{The Inequalities of Young, H\"older, and Minkowski}

\begin{center}
	\textbf{PROBLEMS}
\end{center}
\begin{enumerate}
	\setcounter{enumi}{5}
	\item Show that if H\"older's Inequality is true for normalized functions it is true in general.
	\item Verify the assertions in the above two examples regarding the membership of the function $f$ in $L^p(E)$. 
	\item Let $f$ and $g$ belong to $L^2(E)$. From the linearity of integration show that for any number $\lambda$,
	\[
		\lambda^2\int_Ef^2+2\lambda\int_Ef\cdot g+\int_Eg^2=\int_E(\lambda f+g)^2\ge0.	
	\] 
	From this and the quadratic formula directly derive the Cauchy-Schwarz Inequality.
	\item Show that in Young's Inequality there is equality iff $a^p=b^q$.
	\item Show that in H\"older's Inequality there is equality iff there are constants $\alpha,\beta$ not both zero, for which
	\[
		\alpha|f|^p=\beta|g|^q\text{ a.e. on }E.	
	\]
	For a point $x=(x_1,x_2,\cdots,x_n)$ in $\mathbb{R}^n$, define $T_x$ to be the step function on the interval ...
\end{enumerate}

% 7.3
\authoredby{inprogress}
\section{$L^p$ is Complete: The Riesz-Fischer Theorem}
\begin{namedthm*}{Definition}
	A sequence $\{f_n\}$ in a linear space $X$ that is normed by $\|\cdot\|$ is said to \textbf{converge to $f$ in $X$} provided
	\[
		\lim_{n\to\infty}\|f-f_n\|=0.
	\]
	We write 
	\[
		\{f_n\}\to f\text{ in }X\text{ or }\lim_{n\to\infty}f_n=f\text{ in }X
	\]
	to mean that each $f_n$ and $f$ belong to $X$ and $\lim_{n\to\infty}\|f-f_n\|=0$.
\end{namedthm*}
For a sequence $\{f_n\}$ and a function $f$ in $C[a,b]$, $\{f_n\}\to f$ in $C[a,b]$, normed by the maximum norm, iff $\{f_n\}\to f$ uniformly on $[a,b]$.

For a sequence $\{f_n\}$ and a function $f$ in $L^\infty(E)$, $\{f_n\}\to f$ in $L^\infty(E)$ iff $\{f_n\}\to f$ uniformly on the complement of a set of measure zero.

For a sequence $\{f_n\}$ and a function $f$ in $L^p(E)$, $1\le p<\infty$, $\{f_n\}\to f$ in $L^p(E)$ iff $\lim_{n\to\infty}\int_E|f_n-f|^p=0.$
\begin{namedthm*}{Definition}
	A sequence $\{f_n\}$ in a linear space $X$ that is normed by $\|\cdot\|$ is said to be \textbf{Cauchy} in $X$ provided for each $\epsilon>0$, there is a natural number $N$ such that
	\[
		\|f_n-f_m\|<\epsilon\text{ for all }m,n\ge N.
	\]
	A normed linear space $X$ is said to be \textbf{complete} provided every Cauchy sequence in $X$ converges to a function in $X$.
	A complete normed linear space is called a \textbf{Banach space}.
\end{namedthm*}
\begin{namedthm*}{Proposition 4}
	Let $X$ be a normed linear space.
	Then every convergent sequence in $X$ is Cauchy.
	Moreover, a Cauchy sequence in $X$ converges if it has a convergent subsequence.
\end{namedthm*}
\begin{proof}
	Let $\{f_n\}\to f$ in $X$.
	By the triangle inequality of the norm,
	\[
		\|f_n-f_m\|=\|f_n-f+f-f_m\|\le\|f_n-f\|+\|f-f_m\|\text{ for all }m,n.
	\]
	Therefore $\{f_n\}$ is Cauchy.

	Now let $\{f_n\}$ be a Cauchy sequence in $X$ that has a subsequence $\{f_{n_k}\}$ which converges in $X$ to $f$.
	Let $\epsilon>0$.
	Since $\{f_n\}$ is Cauchy, there exists an $N\in\mathbb{N}$ such that $\|f_n-f_m\|<\epsilon/2$ for all $n,m\ge N$.
	Then, since $\{f_{n_k}\}$ converges to $f$ we can choose $k$ such that $n_k>N$ so that $\|f_{n_k}-f\|<\epsilon/2$.
	Therefore we have
	\[
		\|f_n-f\|=\|f_n-f_{n_k}+f_{n_k}-f\|\le\|f_n-f_{n_k}\|+\|f_{n_k}-f\|<\epsilon/2+\epsilon/2\text{ for }n\ge N.
	\]
	Therefore $\{f_n\}\to f$ in $X$.
\end{proof}
\begin{namedthm*}{Definition}
	Let $X$ be a linear space normed by $\|\cdot\|$.
	A sequence $\{f_n\}$ in $X$ is said to be \textbf{rapidly Cauchy} provided there is a convergent series of positive numbers $\sum_{k=1}^\infty\epsilon_k$ for which 
	\[
		\|f_{k+1}-f_k\|\le\epsilon_k^2\text{ for all }k.
	\]
\end{namedthm*}
\begin{namedthm*}{Proposition 5}
	Let $X$ be a normed linear space.
	Then every rapidly Cauchy sequence in $X$ is Cauchy.
	Furthermore, every Cauchy sequence has a rapidly Cauchy subsequence.
\end{namedthm*}
\begin{proof}
	Let $\{f_n\}$ be a rapidly Cauchy sequence in $X$ and $\sum_{k=1}^\infty\epsilon_k$ a convergent series of positive numbers for which 
	\[
		\|f_{k+1}-f_k\|\le\epsilon_k^2\text{ for all }k.
	\]
	We have by telescoping, 
	\[
		f_{n+k}-f_n=\sum_{j=n}^{n+k-1}[f_{j+1}-f_j]\text{ for all }n,k,
	\]
	and therefore by subadditivity of the norm,
	\[
		\|f_{n+k}-f_n\|=\sum_{j=n}^{n+k-1}\|f_{j+1}-f_j\|\le\sum_{j=n}^{n+k-1}\epsilon_j^2\le\sum_{j=n}^\infty\epsilon_j^2\text{ for all }n,k.
	\]
	For $\varepsilon_1=1>0$, there exists $N_1\in\mathbb{N}$ such that $\sum_{j=p}^\infty\epsilon_j<1$ for all $p\ge N_1$.
	This implies $0<\epsilon_j<1$ for each $j\ge p$, which gives us $\epsilon_j^2<\epsilon_j$ so that $\sum_{j=p}^\infty\epsilon_j^2<\sum_{j=p}^\infty\epsilon_j<\infty$, which implies that $\sum_{j=1}^\infty\epsilon_j^2$ is a convergent series of positive numbers.
	
	Fix $\varepsilon_2>0$.
	Then because $\sum_{j=1}^\infty\epsilon_j^2$ is convergent, there exists an index $N_2\in\mathbb{N}$ such that for all $n\ge N_2$, then 
	\[
		\|f_{n+k}-f_n\|\le\sum_{j=n}^\infty\epsilon_j^2=\sum_{j=1}^\infty\epsilon_j^2-\sum_{j=1}^{n-1}\epsilon_j^2<\varepsilon_2\text{ for all }k.
	\]
	Therefore $\{f_n\}$ is Cauchy.

	Now assume that $\{f_n\}$ is a Cauchy sequence in $X$.
	\\\underline{Base case: $P(1)$}:
	\\Because $\{f_n\}$ is Cauchy, there exists an index $N$ such that for $n_1,n_2\ge N$ with $n_1<n_2$, then 
	\[
		\|f_{n_2}-f_{n_1}\|\le\frac{1}{2^3}<\frac{1}{2^1},
	\]
	and for all $n\ge n_{2}$, we have
	\[
		\|f_{n_2}-f_n\|<\frac{1}{2^3}.
	\]
	\underline{Inductive hypothesis: Suppose $P(k)$}: 
	\\suppose that we already have $n_1,\dots,n_k$ such that
	\[
		\|f_{n_{k}}-f_{n_{k-1}}\|<1/2^{k-1},
	\]
	and for all $n\ge n_{k}$, we have
	\[
		\|f_{n_{k}}-f_{n}\|<1/2^{k+1}.\tag{1}
	\]
	\underline{$P(k+1)$}:
	\\Because $\{f_n\}$ is Cauchy, there exists an index $N$ such that for $m,n\ge N$, then 
	\[
		\|f_m-f_n\|<1/2^{k+2}.\tag{2}
	\]
	Then choose $n_{k+1}>\max\{N,n_k\}$ so that by (2) and (1), for all $n\ge n_{k+1}$ we have
	\[
		\|f_{n_{k+1}}-f_{n_{k}}\|\le\|f_{n_{k+1}}-f_n\|+\|f_n-f_{n_{k}}\|\le\frac{1}{2^{k+2}}+\frac{1}{2^{k+1}}=(\frac{1}{2^2}+\frac{1}{2})\cdot\frac{1}{2^{k}}\le\frac{1}{2^{k}},
	\]
	and because $n_{k+1}\ge N$, for all $n\ge n_{k+1}$ we have
	\[
		\|f_{n_{k+1}}-f_{n}\|<1/2^{k+2}.
	\]
	Therefore we have shown that we can inductively choose a strictly increasing sequence $\{n_k\}$ such that 
	\[
		\|f_{n_{k+1}}-f_{n_k}\|\le\frac{1}{2^k}\text{ for all }k.
	\]
	and $\sqrt{2^k}=(2^k)^{1/2}=\sqrt{2}^k$ for each $k$ tells us that $\sum_{k=1}^\infty\frac{1}{\sqrt{2^k}}=\sum_{k=1}^\infty(\frac{1}{\sqrt{2}})^k$, which converges because $\left|\frac{1}{\sqrt{2}}\right|<1$.
	Therefore $\{f_{n_k}\}$ is rapidly Cauchy.
\end{proof}
\begin{namedthm*}{Theorem 6}
	Let $E$ be a measurable set and $1\le p\le\infty$.
	Then every rapidly Cauchy sequence in $L^p(E)$ converges both w.r.t. the $L^p(E)$ norm and pointwise a.e. on $E$ to a function in $L^p(E)$.
\end{namedthm*}

\begin{center}
	\textbf{PROBLEMS}
\end{center}
\begin{enumerate}
	\setcounter{enumi}{22}
	\item Provide an example of a Cauchy sequence of real numbers that is not rapidly Cauchy.
	\item Let $X$ be a normed linear space.
	Assume that $\{f_n\}\to f$ in $X$, $\{g_n\}\to g$ in $X$, and $\alpha$ and $\beta$ are real numbers.
	Show that
	\[
		\{\alpha f_n+\beta g_n\}\to\alpha f+\beta g\text{ in }X.
	\]
	\item Assume that $E$ has finite measure and $1\le p_1< p_2\le\infty$.
	Show that if $\{f_n\}\to f$ in $L^{p_2}(E)$, then $\{f_n\}\to f$ in $L^{p_1}(E)$.
	\item (The $L^p$ Dominated Convergence Theorem) Let $\{f_n\}$ be a sequence of measurable functions that converges pointwise a.e. on $E$ to $f$.
	For $1\le p<\infty$, suppose there is a function $g$ in $L^p(E)$ such that for all $n$, $|f_n|\le g$ a.e. on $E$.
	Prove that $\{f_n\}\to f$ in $L^p(E)$.
	\item For $E$ a measurable set and $1\le p<\infty$, assume $\{f_n\}\to f$ in $L^p(E)$.
	Show that there is a subsequence $\{f_{n_k}\}$ and a function $g\in L^p(E)$ for which $|f_{n_k}|\le g$ a.e. on $E$ for all $k$.
	\item Assume $E$ has finite measure and $1\le p<\infty$.
	Suppose $\{f_n\}$ is a sequence of measurable functions that converges pointwise a.e. on $E$ to $f$.
	For $1\le p<\infty$, show that $\{f_n\}\to f$ in $L^p(E)$ if there is a $\theta>0$ such that $\{f_n\}$ belongs to and is bounded as a subset of $L^{p+\theta}(E)$.
	\item Consider the linear space of polynomials on $[a,b]$ normed by $\|\cdot\|_{\max}$ norm.
	Is this normed linear space a Banach space?
	\item Let $\{f_n\}$ be a sequence in $C[a,b]$ and $\sum_{k=1}^\infty a_k$ a convergent series of positive numbers such that
	\[
		\|f_{k+1}-f_k\|_{\max}\le a_k\text{ for all }k.
	\]
	Prove that
	\[
		|f_{n+k}(x)-f_n(x)|\le\|f_{n+k}-f_n\|_{\max}\le\sum_{j=n}^\infty a_j\text{ for all }k,n\text{ and all }x\in[a,b].
	\]
	Conclude that there is a function $f\in C[a,b]$ such that $\{f_n\}\to f$ uniformly on $[a,b]$.
	\item Use the preceding problem to show that $C[a,b]$, normed by the maximum norm, is a Banach space.\\
	\\Note: See Chapter 16 Problem 1 that $C[a,b]$ normed by the $L^2[a,b]$ norm is not a Banach space.
	In particular, we had a sequence of continuous functions in $[a,b]$ that was Cauchy (w.r.t. the $L^2$ norm ) but converged ($L^2$) to a discontinuous function in $[a,b]$.
	We again consider this sequence of functions $\{f_n\}$ and now show that it is no longer Cauchy (w.r.t. the maximum norm) and thus does not converge.
	Recall the definition of each function:
	\[
        f_n(x):=
        \begin{cases}
            0 &x\le t\\
            n(x-t)&t<x<t+\frac{1}{n}\\
            1 &x\ge t+\frac{1}{n}
        \end{cases}
    \]
	Fix $\epsilon=\frac{1}{2}$.
	For any $N\in\mathbb{N}$, consider $n\ge N$ and $m=3n>n\ge N$. 
	\\Then we have
	\[
        (f_m-f_n)(x)=
        \begin{cases}
            0-0 &x\le t\\
            m(x-t)-n(x-t)&x\in(t,t+\frac{1}{m})\\
            1-n(x-t)&x\in[t+\frac{1}{m},t+\frac{1}{n})\\
            1-1 &x\ge t+\frac{1}{n}
        \end{cases}
	\]
	then we can clearly see that the maximum occurs at $t+\frac{1}{m}$ so that 
	\[
		\|f_m-f_n\|_{\max}=\max_{x\in[a,b]}|(f_m-f_n)(x)|=1-n((t+\frac{1}{m})-t)=1-\frac{n}{3n}=\frac{2}{3}>\frac{1}{2},
	\]
	and the sequence is not Cauchy.\\
	\\Back to the proof of Problem 31:
	\\Consider the linear space $C[a,b]$ with the maximum norm. 
	Suppose that $\{f_n\}$ is a Cauchy sequence of functions in this space.
	By Proposition 5, there exists a rapidly Cauchy subsequence $\{f_{n_k}\}$.
	That is, there is a convergent series of positive numbers $\sum_{k=1}^\infty\epsilon_k$ for which
	\[
		\|f_{n_{k+1}}-f_{n_k}\|_{\max}\le\epsilon_k^2\text{ for all }k.
	\]
	Then $\sum_{k=1}^\infty\epsilon_k$ converges implies that $\sum_{k=1}^\infty\epsilon_k^2$ also converges.
	\\Therefore by the previous Problem 30, there is a function $f\in C[a,b]$ such that $\{f_{n_k}\}\to f$ uniformly on $[a,b]$.
	Then by Proposition 4, $\{f_n\}$ converges because it has a convergent subsequence $\{f_{n_k}\}$. 
	\item Let $\{f_n\}$ be a sequence in $L^\infty(E)$ and $\sum_{k=1}^\infty a_k$ a convergent series of positive numbers such that
	\[
		\|f_{k+1}-f_k\|_{\infty}\le a_k\text{ for all }k.
	\]
	Prove that there is a subset $E_0$ of $E$ which has measure zero and 
	\[
		|f_{n+k}(x)-f_n(x)|\le\|f_{n+k}-f_n\|_{\infty}\le\sum_{j=n}^\infty a_j\text{ for all }k,n\text{ and all }x\in E\setminus E_0.
	\]
	Conclude that there is a function $f\in L^\infty(E)$ such that $\{f_n\}\to f$ uniformly on $E\setminus E_0$.
	\item Use the preceding problem to show that $L^\infty(E)$ is a Banach space.\\
	\\Consider the linear space $L^\infty(E)$ with the supremum norm. 
	Suppose that $\{f_n\}$ is a Cauchy sequence of functions in this space.
	By Proposition 5, there exists a rapidly Cauchy subsequence $\{f_{n_k}\}$.
	That is, there is a convergent series of positive numbers $\sum_{k=1}^\infty\epsilon_k$ for which
	\[
		\|f_{n_{k+1}}-f_{n_k}\|_{\infty}\le\epsilon_k^2\text{ for all }k.
	\]
	Then $\sum_{k=1}^\infty\epsilon_k$ converges implies that $\sum_{k=1}^\infty\epsilon_k^2$ also converges.
	\\Therefore by the previous Problem 32, there is a function $f\in L^\infty(E)$ and a set $E_0\subseteq E$ of measure zero such that $\{f_{n_k}\}\to f$ uniformly on $E\setminus E_0$.
	Then by Proposition 4, $\{f_n\}$ converges because it has a convergent subsequence $\{f_{n_k}\}$. 
	\item Prove that for $1\le p\le\infty$, $\ell^p$ is a Banach space.
	\item Show that the space $c$ of all convergent sequences of real numbers and the space $c_0$ of all sequences that converge to zero are Banach spaces w.r.t. the $\ell^\infty$ norm. 
\end{enumerate}

% 7.4
\authoredby{untouched}
\section{Approximation and Separability}

% Chapter 8
\authoredby{inprogress}
\chapter{The $L^p$ Spaces: Duality and Weak Convergence}

% 8.1
\authoredby{inprogress}
\section{The Riesz Representation for the Dual of $L^p,a\le p\le \infty$}

\textbf{Example}
Let $E$ be a measurable set, $1\le p<\infty$, $q$ the conjugate of $p$, and $g$ belong to $L^q(E)$.
Define the functional $T$ on $L^p(E)$ by
\[
    T(f)=\int_Eg\cdot f\text{ for all }f\in L^p(E).
\]
H\"older's Inequality tells us that for $f\in L^p(E)$, the product $g\cdot f$ is integrable over $E$ so the functional $T$ is properly defined.
By the linearity of integration, $T$ is linear.
Observe that H\"older's inequality is the statement that 
\[
    |T(f)|\le\|g\|_q\cdot\|f\|_p\text{ for all }f\in L^p(E).
\]
\begin{flushleft}

\textbf{Example}
Let $[a,b]$ be a closed, bounded interval and the function $g$ be of bounded variation on $[a,b]$.
Define the functional $T$ on $C[a,b]$ by
\[
    T(f)=\int_a^bf(x)dg(x)\text{ for all }f\in C[a,b],
\]  
where the integral is in the sense of Riemann-Stieltjes.
The functional $T$ is properly defined and linear.
Moreover, it follows immediately from the definition of this integral that 
\[
    |T(f)|\le TV(g)\cdot\|f\|_{\max}\text{ for all }f\in C[a,b],
\]  
where $TV(g)$ is the total variation of $g$ over $[a,b]$.
\end{flushleft}
\begin{namedthm*}{Definition}
    For a normed linear space $X$, a linear functional $T$ on $X$ is said to be \textbf{bounded} provided there is an $M\ge0$ for which 
\[
    |T(f)|\le M\|f\|\text{ for all }f\in X.
\]
The infimum of all such $M$ is called the \textbf{norm} of $T$ and denoted by $\|T\|_*$.
\end{namedthm*}
The inequalities in the first and second example above tell us that the linear functionals are bounded.

Let $T$ be a bounded linear functional on the normed linear space $X$.
It is easy to see that the above equation holds for $M=\|T\|_*$.
Hence, by the linearity of $T$,
\[
    |T(f)-T(h)|\le \|T\|_*\|f-h\|\text{ for all }f,h\in X.
\]
From this we infer the following continuity property of a bounded linear functional $T$:
\[
    \text{if }\{f_n\}\to f\text{ in }X,\text{ then }\{T(f_n)\}\to T(f).\tag{7}
\]
We leave it as an exercise (see chapter 13 problem 11) to show that 
\[
    \|T\|_*=\sup\{T(f)\mid f\in X,\|f\|\le1\},\tag{8}
\]
and use this characterization of $\|\cdot\|_*$ to prove the following proposition.
\begin{namedthm*}{Proposition 1}
    Let $X$ be a normed linear space.
    Then the collection of bounded linear functionals on $X$ is a linear space on which $\|\cdot\|_*$ is a norm.
    This normed linear space is called the \textbf{dual space} of $X$ and denoted by $X^*$.
\end{namedthm*}
\begin{namedthm*}{Proposition 2}
    Let $E$ be a measurable set, $1\le p<\infty$, $q$ the conjugate of $p$, and $g$ belong to $L^q(E)$.
    Define the functional $T$ on $L^p(E)$ by
    \[
        T(f)=\int_Eg\cdot f\text{ for all }f\in L^p(E).
    \]
    Then $T$ is a bounded linear functional on $L^p(E)$ and $\|T\|_*=\|g\|_q$.
\end{namedthm*}
\begin{proof}
    By linearity of integration, $T$ is linear, and from H\"older's inequality, we infer that $T$ is bounded and $\|T\|_*\le\|g\|_q$.
    For $p>1$, according to Chapter 7 Theorem 1, the conjugate function of $g$, $g^*=\|g\|_q^{1-q}\text{sgn}(g)|g|^{q-1}$, belongs to $L^p(E)$, and 
    \[
        T(g^*)=\int_Eg\cdot g^*=\|g\|_q\quad\text{and}\quad\|g^*\|_p=1.
    \] 
    Then by definition of the operator norm as a supremum (8), we have $\|T\|_*\ge T(g^*)=\|g\|_q$.
    Thus from the two inequalities we obtain $\|T\|_*=\|g\|_q$.
    For $p=1$, we argue by contradiction.

\end{proof}

\begin{center}
	\textbf{PROBLEMS}
\end{center}
\begin{enumerate}
	\setcounter{enumi}{0}
    \item Verify (8).
    
    \ \\(See Chapter 13 Problem 11) 
    Define 
    \[
        \begin{split}
        M'&:=\inf\{M\ge0\mid\|T(f)\|\le M\|f\|\text{ for all }f\in X\},\\
        N'&:=\sup\{\|T(f)\|\mid f\in X, \|f\|\le1\}.
        \end{split}
    \]
    We aim to show that they are equal.
    
    First, for $f\neq0$, then $\|\frac{f}{\|f\|}\|\le1$ so that by linearity of $T$,
    \[\|T(\frac{f}{\|f\|})\|\le N'\implies\|Tf\|\le N'\|f\|\implies M'\le N'.\]
    On the other hand, for $f$ such that $\|f\|\le1$, then
    \[\|Tf\|\le M'\|f\|\le M'\implies N'\le M.\]
    Therefore $M'=N'$.
    \\\item Prove Proposition 1.
    \item Let $T$ be a linear functional on a normed linear space $X$. Show that $T$ is bounded iff the continuity property (7) holds.
    \item A functional $T$ on a normed linear space $X$ is said to be Lipschitz provided there is a $c\ge0$ such that
    \[
        |T(g)-T(h)|\le c\|g-h\|\text{ for all }g,h\in X.  
    \]
    The infimum of such $c$'s is called the Lipschitz constant for $T$. Show that a linear functional is bounded iff it is Lipschitz, in which case its Lipschitz constant is $\|T\|_*$.
    \item Let $E$ be a measurable set and $1\le 0<\infty$. Show that the functions in $L^p(E)$ that vanish outside a bounded set are dense in $L^p(E)$. Show that this is false for $L^\infty(\mathbb{R})$.
    \item Establish the Riesz Representation Theorem in the case $p=1$ by first showing, in the notation of the proof of the theorem, that the function $\Phi$ is Lipschitz and therefore it is absolutely continuous. Then follow the $p>1$ proof.
    \item State and prove a Riesz Representation Theorem for the bounded linear functionals on $\ell^p$, $1\le p<\infty$.
    \item Let $c$ be the linear space of real sequences that converge to a real number and $c_0$ the subspace of $c$ comprising sequences that converge to $0$. Norm each of these linear spaces with the $\ell^\infty$ norm. Determine the dual space of $c$ and $c_0$.
    \item Let $[a,b]$ be a closed, bounded interval and $C[a,b]$ be normed by the maximum norm. Let $x_0$ belong to $[a,b]$. Define the linear functional $T$ on $C[a,b]$ by $T(f)=f(x_0)$. Show that $T$ is bounded and is given by Riemann-Stieltjes integration against a function of bounded variation.
    \item Let $f$ belong to $C[a,b]$. Show that there is a function $g$ that is of bounded variation on $[a,b]$ for which 
    \[
        \int_a^bfdg=\|f\|_{\max}\text{ and }TV(g)=1.  
    \]
    \item Let $[a,b]$ be a closed, bounded interval and $C[a,b]$ be normed by the maximum norm. Let $T$ be a bounded linear functional on $C[a,b]$.
    For $x\in[a,b]$, let $g_x$ be the member of $C[a,b]$ that is linear on $[a,x]$ and on $[x,b]$ with $g_x(a)=0,g_x(x)=x-a$ and $g_x(b)=x-a$. Define $\Phi(x)=T(g_x)$ for $x\in[a,b]$. Show that $\Phi$ is Lipschitz on $[a,b]$.
\end{enumerate}

% 8.2
\authoredby{untouched}
\section{Weak Sequential Convergence in $L^p$}
\begin{center}
	\textbf{PROBLEMS}
\end{center}
\begin{enumerate}
	\setcounter{enumi}{11}
    \item f
\end{enumerate}

% 8.3
\section{Weak Sequential Compactness}
\begin{center}
	\textbf{PROBLEMS}
\end{center}
\begin{enumerate}
	\setcounter{enumi}{36}
    \item f
\end{enumerate}

% 8.4
\section{The Minimization of Convex Functionals}
\begin{center}
	\textbf{PROBLEMS}
\end{center}
\begin{enumerate}
	\setcounter{enumi}{36}
    \item f
\end{enumerate}

% Chapter 9
\chapter{Metric Spaces: General Properties}

\section{Examples of Metric Spaces}
\section{Open Sets, Closed Sets, and Convergent Sequences}
\section{Continuous Mappings Between Metric Spaces}
\section{Complete Metric Spaces}
\section{Compact Metric Spaces}
\section{Separable Metric Spaces}


% Chapter 10
\chapter{Metric Spaces: Three Fundamental Theorems}

\section{The Arzel\'a-Ascoli Theorem}
\section{The Baire Category Theorem}
\section{The Banach Contraction Principle}

% Chapter 11
\chapter{Topological Spaces: General Properties}

\section{Open Sets, Closed Sets, Bases, and Subbases}
\section{The Separation Properties}
\section{Countability and Separability}
\section{Continuous Mappings Between Topological Spaces}
\section{Compact Topological Spaces}
\section{Connected Topological Spaces}

% Chapter 12
\chapter{Topological Spaces: Three Fundamental Theorems}

\section{Urysohn's Lemma and the Tietze Extension Theorem}
\section{The Tychonoff Product Theorem}
\section{Thye Stone-Weierstrass Theorem}

% Chapter 13
\authoredby{finished}
\chapter{Continuous Linear Operators Between Banach Spaces}

% 13.1
\authoredby{finished}
\section{Normed Linear Spaces}

\begin{namedthm*}{Definition}
    Two norms $\|\cdot\|_1$ and $\|\cdot\|_2$ on a linear space $X$ are said to be \textbf{equivalent} provided there are constants $c_1,c_2\ge0$ for which
    \[
        c_1\|x\|_1\le\|x\|_2\le c_2\|x\|_1\quad\text{for all }x\in X.
    \]
    We immediately see that two norms are equivalent iff their induced metrics are equivalent.
\end{namedthm*}

Given vectors $x_1,\dots,x_n$ in a linear space $X$ and real numbers $\lambda_1,\dots,\lambda_n$, the vector
\[
    x=\sum_{k=1}^n\lambda_kx_k
\]
is called a \textbf{linear combination} of the $x_i$'s.
A nonempty subset $Y$ of $X$ is called a \textbf{linear subspace}, or simply a subspace, provided every linear combination of vectors in $Y$ also belongs to $Y$.
For a nonempty subset $S$ of $X$, by the \textbf{span} of $S$ we mean the set of all linear combinations of vectors in $S$, we denote this by $\text{span }S$.

For any two nonempty subspaces $Y,Z$ of $X$, then the sum $Y+Z=\{y+z\mid y\in Y,z\in Z\}$ is also a subspace of $X$.
In the case $Y\cap Z=\{0\}$, we denote $Y+Z$ by $Y\oplus Z$ and call this subspace of $X$ the \textbf{direct sum} of $Y$ and $Z$.

Almost all the important theorems for metric space require completeness:
\begin{namedthm*}{Definition}
    A normed linear space is called a \textbf{Banach space} provided it is complete as a metric space with the metric induced by the norm.
\end{namedthm*}
The Riesz-Fischer Theorem tells us that for $E$ a measurable set of real numbers and $1\le p\le\infty$, $L^p(E)$ is a Banach space.
Also, for $X$ a compact topological space, $C(X)$ with the maximum norm is a Banach space (Chapter 7 Problem 31).
From the Completeness Axiom for $\mathbb{R}$, each Euclidean space $\mathbb{R}^n$ is a Banach space.

\begin{center}
	\textbf{PROBLEMS}
\end{center}
\begin{enumerate}
	\setcounter{enumi}{0}
    \item Show that a nonempty subset $S$ of a linear space $X$ is a subspace iff $S+S=S$ and $\lambda \cdot S=S$ for each $\lambda\in\mathbb{R},\lambda\neq0$.
    
    \ \\Let $X$ be a linear space and consider the nonempty subset $S$.

    \ \\$(\implies)$ Suppose that $S$ is a subspace of $X$.
    Then by definition of subspace, every linear combination of vectors in $S$ also belongs to $S$.
    In particular, 
    \begin{align*}
        x&\in S+S=\{y+z\mid y\in S,z\in S\}\implies \text{ $x$ is a linear combination of vectors in $S$ }\implies x\in S\\
        x&\in S\implies x=x+0\in S+S\text{ because }0\in S.
    \end{align*}
    Then the first implies $S+S\subset S$ and the second implies $S\subset S+S$.
    Therefore $S=S+S$.
    \begin{align*}
        x&\in \lambda S=\{\lambda y\mid y\in S\}\implies \text{ $x$ is a linear combination of vectors in $S$ }\implies x\in S\\
        x&\in S\implies x=\lambda \frac{x}{\lambda}\in \lambda S\text{ because }\frac{x}{\lambda}\in S.
    \end{align*}
    Then the first implies $\lambda S\subset S$ and the second implies $S\subset \lambda S$.
    Therefore $\lambda S=S$.

    \ \\$(\impliedby)$ Suppose that $S+S=S$ and $\lambda\cdot S=S$ for each nonzero $\lambda\in\mathbb{R}$.
    Consider any $y,z\in S$ and scalars $\lambda_1,\lambda_2$. 
    Then 
    \[
        \lambda_1 y+\lambda_1z\in \lambda_1 S+\lambda_2 S=S+S=S.
    \]
    Therefore any linear combination of vectors in $S$ also belongs to $S$, and thus $S$ is a subspace of $X$.
    \ \\\item If $Y$ and $Z$ are subspaces of the linear space $X$, show that $Y+Z$ is also a subspace and $Y+Z=\text{span}[Y\cup Z]$. 

    \ \\Let $x_1,x_2\in Y+Z$, and let $\alpha,\beta$ be any scalars.
    Then $x_1=\alpha_1y_1+\beta_1z_1$ and $x_2=\alpha_2y_2+\beta_2z_2$.
    Therefore the linear combination of $x_1$ and $x_2$ are in $Y+Z$:
    \[
        \alpha x_1+\beta x_2
        =\alpha(\alpha_1y_1+\beta_1z_1)+\beta(\alpha_2y_2+\beta_2z_2)
        =(\alpha\alpha_1)y_1+(\beta\alpha_2)y_2+(\alpha\beta_1)z_1+(\beta\beta_2)z_2
        \in Y+Z,
    \]
    because $(\alpha\alpha_1)y_1+(\beta\alpha_2)y_2\in Y$ and $(\alpha\beta_1)z_1+(\beta\beta_2)z_2\in Z$.

    Also, $x\in Y+Z$ trivially implies that $x\in\text{span}[Y\cup Z]$. 
    To show the other side, $x\in\text{span}[Y\cup Z]$ implies that $x$ is a linear combination of vectors in $Y\cup Z$.
    But we showed above that any linear combination of vectors in $Y\cup Z$ is in $Y+Z$, and so $x\in Y+Z$.
    Therefore $Y+Z=\text{span}[Y\cup Z]$.
    \ \\\item Let $S$ be a subset of a normed linear space $X$.
    \begin{enumerate}[label=(\roman*),align=left]
        \item Show that the intersection of a collection of linear subspaces of $X$ is also a linear subspace of $X$.
        
        \ \\Let $\{Y_\alpha\mid \alpha\in \mathcal{A}\}$ be a collection of linear subspaces of $X$.
        Consider $x_1,x_2\in \cap_\alpha Y_\alpha$ and scalars $\alpha,\beta$.
        (Then $x_1,x_2\in Y_\alpha$ for each $\alpha$ by definition of intersection.)
        Now we have that $\alpha x_1+\beta x_2\in Y_\alpha$ for each $\alpha\in\mathcal{A}$ because each $Y_\alpha$ is a subspace.
        Therefore $\alpha x_1+\beta x_2\in Y_\alpha\in\cap_\alpha Y_\alpha$ by definition of intersection.
        Thus we have that $\cap_\alpha Y_\alpha$ is a subspace of $X$ because it contains all linear combinations of its elements.
        \\\item Show that span$[S]$ is the intersection of all the linear subspaces of $X$ that contain $S$ and therefore is a linear subspace of $X$.
        
        \ \\Let $\mathcal{S}=\{Y\text{ subspace of }X,S\subset Y\}$.
        Consider $\cap\mathcal{S}$, the intersection of all elements of $\mathcal{S}$.
        Then by (i), $\cap\mathcal{S}$ is a subspace of $X$.
        Recall that $\text{span}[S]$ is the collection of all linear combinations of elements of $S$.
        It is clear to see that $\text{span}[S]$ is a linear subspace of $X$ such that $S\subset\text{span}[S]$, and therefore $\text{span}[S]\in\mathcal{S}$.
        Then by definition of intersection, 
        \[
            \cap\mathcal{S}\subset \text{span}[S].\tag{a}
        \]
        Also, for any $Y\in\mathcal{S}$, any linear combination of elements of $S$ is in $Y$, which implies
        \[
            \text{span}[S]\subset\cap\mathcal{S}.\tag{b}
        \]
        Therefore by (a) and (b), $\text{span}[S]=\cap\mathcal{S}$.
        \\\item Show that $\overline{\text{span}}[S]$ is the intersection of all the closed linear subspaces of $X$ that contain $S$ and is therefore a closed linear subspace of $X$. 

        \ \\We can simply use (i) and (ii) along with the additional fact that any arbitrary intersection of closed sets (subspaces) is also a closed set (subspace).
    \end{enumerate}
    \ \\\item For a normed linear space $X$, show that the function $\|\cdot\|:X\to\mathbb{R}$ is continuous.\\
    \\Fix $\epsilon>0$.
    \\Then let $\delta:=\epsilon>0$.
    \\Consider any $x,y\in X$ such that $\|x-y\|<\delta$.
    \\Then by the reverse triangle inequality ($\ast$),
    \[
        |\|x\|-\|y\||\le\|x-y\|<\delta=\epsilon,
    \]
    and $\|\cdot\|$ is continuous.\\
    \\($\ast$) Proof of reverse triangle inequality:
    \[
        \|x\|=\|x-y+y\|\le\|x-y\|+\|y\|.
    \]
    \ \\\item For two normed linear spaces $(X,\|\cdot\|_1)$ and $(Y,\|\cdot\|_2)$, define a linear structure on the Cartesian product $X\times Y$ by $\lambda\cdot(x,y)=(\lambda x,\lambda y)$ and $(x_1,y_1)+(x_2,y_2)=(x_1+x_2,y_1+y_2)$.
    Define the product norm $\|\cdot\|$ by $\|(x,y)\|=\|x\|_1+\|y\|_2$, for $x\in X$ and $y\in Y$.
    Show that this is a norm with respect to which a sequence converges if and only if each of the two component sequences converges.
    Furthermore, show that if $X$ and $Y$ are Banach spaces, then so is $X\times Y$.\\
    \\Let $(X\times Y,\|\cdot\|)$ be a normed linear space.\\
    \\$(\implies)$ Let $\{(x_n,y_n)\}$ be a sequence in $X\times Y$, and suppose that it converges to some $(x,y)\in X\times Y$ with respect to the norm $\|\cdot\|$.\\
    \\Fix $\epsilon>0$.
    \\Then there exists an index $N$ such that for all $n\ge N$,
    \[
        0\le\|x_n-x\|_1+\|y_n-y\|_2=\|(x_n-x,y_n-y)\|=\|(x_n,y_n)-(x,y)\|<\epsilon,
    \]
    which implies that $\{x_n\}\to x$ w.r.t. the norm $\|\cdot\|_1$ and $\{y_n\}\to y$ w.r.t. the norm $\|\cdot\|_2$.\\
    \\$(\impliedby)$ Let $\{x_n\}$ be a sequence in $X$, and $\{y_n\}$ be a sequence in $Y$, and suppose that there exist $x\in X$, $y\in Y$ such that $\{x_n\}\to x$ w.r.t. the norm $\|\cdot\|_1$ and $\{y_n\}\to y$ w.r.t. the norm $\|\cdot\|_2$.\\
    \\Fix $\epsilon>0$.
    \\Then there exists an index $N_x$ such that for all $n\ge N_x$,
    \[
        \|x_n-x\|_1<\frac{\epsilon}{2},
    \]
    and there also exists an index $N_y$ such that for all $n\ge N_y$,
    \[
        \|y_n-y\|_1<\frac{\epsilon}{2}.
    \]
    Thus for all $n\ge\max\{N_x,N_y\}$,
    \[
        \|(x_n,y_n)-(x,y)\|=\|(x_n-x,y_n-y)\|=\|x_n-x\|_1+\|y_n-y\|_2<\frac{\epsilon}{2}+\frac{\epsilon}{2}=\epsilon,
    \]
    and therefore the sequence $\{(x_n,y_n)\}$ in $X\times Y$ converges to $(x,y)\in X\times Y$ with respect to the norm $\|\cdot\|$.\\
    \\Finally, suppose that $X$ and $Y$ are Banach spaces.
    \\Let $\{(x_n,y_n)\}$ be any sequence in $X\times Y$ that is Cauchy.
    \\Then for any $\epsilon$, there exists an index $N$ such that for all $n,m\ge N$, then
    \[
        0\le\|x_n-x_m\|_1+\|y_n-y_m\|_2=\|(x_n-x_m,y_n-y_m)\|=\|(x_n,y_n)-(x_m,y_m)\|<\epsilon,
    \]
    which implies that the sequences $\{x_n\}$ and $\{y_n\}$ are also Cauchy.
    \\Then because both $X$ and $Y$ are Banach spaces, then $\{x_n\}\to x$ and $\{y_n\}\to y$ for some $x\in X$ and $y\in Y$, and therefore we proved in $(\impliedby)$ that $\{(x_n,y_n)\}\to(x,y)$, which implies that $X\times Y$ is a Banach space.
    \ \\\item Let $X$ be a normed linear space.
    \begin{enumerate}[(i)]
        \item Let $\{x_n\}$ and $\{y_n\}$ be sequences in $X$ such that $\{x_n\}\to x$ and $\{y_n\}\to y$.
        Show that for any real numbers $\alpha$ and $\beta$, $\{\alpha x_n+\beta y_n\}\to\alpha x+\beta y$.

        \ \\Because $\{x_n\}\to x$ and $\{y_n\}\to y$, choose an index $N$ such that for $n\ge N$, we have $\|x_n-x\|<\frac{\epsilon}{2|\alpha|}$ and $\|y_n-y\|<\frac{\epsilon}{2|\beta|}$ (assuming $|\alpha|,|\beta|>0$).
        Then by subadditivity and absolute homogeneity of the norm,
        \[
            \|\alpha x_n+\beta y_n-(\alpha x+\beta y)\|
            % =\|\alpha (x_n-x)+\beta (y_n-y)\|
            \le|\alpha|\| x_n-x\|+|\beta| \|y_n-y\|
            <|\alpha|\frac{\epsilon}{2|\alpha|}+|\beta|\frac{\epsilon}{2|\beta|}
            =\epsilon,
        \]
        which implies that $\{\alpha x_n+\beta y_n\}\to\alpha x+\beta y$.
        \\\item Use (i) to show that if $Y$ is a subspace of $X$, then its closure $\overline{Y}$ also is a linear subspace of $X$.
    
        \ \\(Recall that for a metric space $X$ and a subset $C\subset X$, then $x\in\overline{C}\iff$ there exists a sequence $\{x_n\}\text{ in }C\text{ such that }\{x_n\}\to x$.)
        
        \ \\Let $x,y\in\overline{Y}$.
        Then there exist sequences $\{x_n\}$ and $\{y_n\}$ in $Y$ such that $\{x_n\}\to x$ and $\{y_n\}\to y$.
        Let $\alpha,\beta$ be any two real numbers.
        Then $\alpha x_n+\beta y_n\in Y$ for each $n$, and by (i), we have $\{\alpha x_n+\beta y_n\}\to\alpha x+\beta y$. 
        Therefore $\alpha x+\beta y\in\overline{Y}$, and $\overline{Y}$ is a linear subspace because it contains all linear combinations of its elements.
        \\\item Use (i) to show that the vector sum is continuous from $X\times X$ to $X$ and scalar multiplication is continuous from $\mathbb{R}\times X$ to $X$.

        \ \\We use the product norm $\|(x,y)\|_P:=\|x\|_X+\|y\|_X$ (see problem 5).
        
        \ \\Define $f:X\times X\to X$ by $f(x,y)=x+y$.
        
        Fix $\epsilon>0$. 
        Then for $(x_1,y_1),(x_2,y_2)\in X\times X$ such that 
        \[
           \epsilon> \|(x_1,y_1)-(x_2,y_2)\|_P=\|(x_1-x_2,y_1-y_2)\|_P=\|x_1-x_2\|_X+\|y_1-y_2\|_X,
        \]
        we have
        \[
            \begin{split}
            \|f(x_1,y_1)-f(x_2,y_2)\|_X
            &=\|x_1+y_1-(x_2+y_2)\|_X\\
            &=\|x_1-x_2+y_1-y_2\|_X\\
            &\le\|x_1-x_2\|_X+\|y_1-y_2\|_X\\
            &<\epsilon,
            \end{split}
        \]
        and thus $f$ is continuous.

        \ \\Define $g:\mathbb{R}\times X\to X$ by $g(\alpha,x)=\alpha x$.
        
        Fix $\epsilon>0$. 
        Then let $\delta:=\min\left\{\frac{\epsilon}{2(\|x_1\|_X+1)},\frac{\epsilon}{2(|\alpha_2|+1)}\right\}>0$.
        Then for $(\alpha_1,x_1),(\alpha_2,x_2)\in \mathbb{R}\times X$ such that
        \[
           \delta> \|(\alpha_1,x_1)-(\alpha_2,x_2)\|_P=\|(\alpha_1-\alpha_2,x_1-x_2)\|_P=|\alpha_1-\alpha_2|+\|x_1-x_2\|_X,
        \]
        we have
        \[
            \begin{split}
            \|g(\alpha_1,x_1)-g(\alpha_2,x_2)\|_X
            &=\|\alpha_1 x_1-\alpha_2 x_2\|_X\\
            &=\|(\alpha_1 -\alpha_2)x_1+\alpha_2(x_1- x_2)\|_X\\
            &\le|\alpha_1 -\alpha_2|\|x_1\|_X+|\alpha_2|\|x_1- x_2\|_X\\
            &<\frac{\epsilon}{2(\|x_1\|_X+1)}\|x_1\|_X+|\alpha_2|\frac{\epsilon}{2(|\alpha_2|+1)}\\
            &<\frac{\epsilon}{2}+\frac{\epsilon}{2},
            \end{split}
        \]
        and thus $g$ is continuous.
    \end{enumerate} 
    \ \\\item Show that the set $\mathcal{P}$ of all polynomials on $[a,b]$ is a linear space.
    For $\mathcal{P}$ considered as a subset of the normed linear space $C[a,b]$, show that $\mathcal{P}$ fails to be closed.
    For $\mathcal{P}$ considered as a subset of the normed linear space $L^1[a,b]$, show that $\mathcal{P}$ fails to be closed.

    \ \\To see that $\mathcal{P}$ is a linear space, look at chapter 7 problem 2.
    
    Let $t\in(a,b)$ and consider the continuous function $f:[a,b]\to\mathbb{R}$ defined by
    \[
        f(x):=
        \begin{cases}
            0 &x\in[a, t]\\
            (\frac{1}{b-t})(x-t)&x\in(t,b]\\
        \end{cases}
    \]
    Then $f$ is not differentiable at $t$, and is thus not a polynomial.
    We can write $f\in C[a,b]$, $f\notin\mathcal{P}$.
    However, by Chapter 12.3 - The Stone-Weierstrass Theorem, there exists a sequence of polynomials $\{p_n\}$ in $\mathcal{P}$ that converges uniformly to $f\notin\mathcal{P}$.
    Therefore $\mathcal{P}$ is not closed.
    Because $\int_a^b|f|=\frac{b-t}{2}<\infty$, then $f\in L^1[a,b]$, and we can use the same argument to say that $\mathcal{P}$ is not closed.
    \ \\\item A nonnegative real-valued function $\|\cdot\|$ defined on a vector space $X$ is called a \textbf{pseudonorm} if $\|x+y\|\le\|x\|+\|y\|$ and $\|\alpha x\|=|\alpha|\|x\|$.
    Define $x\cong y$, provided $\|x-y\|=0$.
    Show that this is an equivalence relation.
    Define $X/\cong$ to be the set of equivalence classes of $X$ under $\cong$ and for $x\in X$ define $[x]$ to be the equivalence class of $x$.
    Show that $X/\cong$ is a normed vector space if we define $\alpha[x]+\beta[y]$ to be the equivalence class of $\alpha x+\beta y$ and define $\|[x]\|=\|x\|$.
    Illustrate this procedure with $X=L^p[a,b],1\le p<\infty$.

    \ \\To see that this is an equivalence relation:
    \begin{enumerate}[(i)]
        \item $x\cong x$ because $\|x-x\|=0$ for all $x\in X$.
        \item $x\cong y\iff0=\|x-y\|=\|y-x\|\iff y\cong x$.
        \item Suppose $x\cong y$ and $y\cong z$. 
        Then $\|x-z\|\le\|x-y\|+\|y-z\|=0$, which implies $x\cong z$.
    \end{enumerate}
\end{enumerate}

% 13.2
\authoredby{inprogress}
\section{Linear Operators}

\begin{namedthm*}{Definition}
    Let $X$ and $Y$ be linear spaces.
    A mapping $T:X\to Y$ is said to be \textbf{linear} provided for each $u,v\in X$, and real numbers $\alpha,\beta$, we have
    \[
        T(\alpha u+\beta v)=\alpha T(u)+\beta T(v).
    \]
\end{namedthm*}
Linear mappings are often called linear operators or linear transformations.
\begin{namedthm*}{Definition}
    Let $X$ and $Y$ be normed linear spaces.
    A linear operator $T:X\to Y$ is said to be \textbf{bounded} provided there is a constant $M\ge0$ for which 
    \[
        \|T(u)\|\le M\|u\|\quad\text{for all }u\in X.
    \]
    The infimum of all such $M$ is called the \textbf{operator norm} of $T$ and denoted by $\|T\|$.
    The collection of bounded linear operators from $X$ to $Y$ is denoted by $\mathcal{L}(X,Y)$.
\end{namedthm*}

\begin{namedthm*}{Theorem 1}
    A linear operator between normed linear spaces is continuous iff it is bounded.    
\end{namedthm*}
\begin{proof}
    Let $T:(X,\|\cdot\|_X)\to(Y,\|\cdot\|_Y)$ be a linear operator.\\
    \\$(\implies)$ Suppose that $T$ is continuous.
    \\Then for $\epsilon=1$, there exists a $\delta>0$ such that, for any $x\in X$ such that $\|x-0\|_X=\|x\|_X\le\delta$, then
    \[
        \|T(x)-T(0)\|_Y=\|T(x)\|_Y<1.
    \]
    (Where $T(0)=0$ by linearity.)
    \\Therefore consider any $u\in X$, $u\neq0$.
    \begin{align*}
        \|T(u)\|_Y&=\left\|T\left(\frac{\delta\cdot\|u\|_X}{\delta\cdot\|u\|_X}u\right)\right\|_Y\\
        &=\frac{\|u\|_X}{\delta}\left\|T\left(\frac{\delta}{\|u\|_X}u\right)\right\|_Y&&\text{by linearity of $T$ and absolute homogeneity of $\|\cdot\|_Y$.}\\
        &<\frac{\|u\|_X}{\delta}\cdot1,&&\text{because }\left\|\frac{\delta}{\|u\|_X}u\right\|_X=\frac{\delta\|u\|_X}{\|u\|_X}=\delta\le\delta.\\
    \end{align*}
    that is, there exists the positive constant $\frac{1}{\delta}$ such that 
    \[
        \|T(u)\|_Y\le\frac{1}{\delta}\|u\|_X\text{ for all }u\in X,
    \]
    which implies that $T$ is bounded.\\
    \\$(\impliedby)$ Suppose that $T$ is bounded.
    \\Then there exists an $M\ge0$ such that 
    $
        \|T(x)\|_Y\le M\|x\|_X\text{ for all }x\in X.
    $
    \\Fix $\epsilon>0$.
    \\Consider any $x,x'\in X$ such that $\|x-x'\|_X<\frac{\epsilon}{M+1}$.
    \\Then by linearity of $T$,
    \[
        \|T(x)-T(x')\|_Y=\|T(x-x')\|_Y\le M\|x-x'\|_X<M\frac{\epsilon}{M+1}<\epsilon,
    \]
    which implies that $T$ is continuous.
\end{proof}
The continuity property of a bounded linear operator $T:X\to Y$ says that:
\[
    \text{if }\{u_n\}\to u\text{ in }X,\text{ then }\{T(u_n)\}\to T(u)\text{ in }Y.
\]
The collection of linear operators between two linear spaces is a linear space. 
\begin{namedthm*}{Proposition 2}
    Let $X$ and $Y$ be normed linear spaces.
    Then the collection of bounded linear operators from $X$ to $Y$, $\mathcal{L}(X,Y)$, is a normed linear space.
\end{namedthm*}
\begin{proof}
    Let $T,S\in\mathcal{L}(X,Y)$, and let $\alpha,\beta$ be real numbers.
    Then using the norm $\|\cdot\|$ on $Y$, we have by subadditivity, absolute homogeneity, and the definition of the operator norm that 
    \[
        \|(\alpha T+\beta S)(u)\|=\|\alpha T(u)+\beta S(u)\|\le|\alpha|\| T(u)\|+|\beta|\| S(u)\|\le
        % |\alpha|\|T\|\|u\|+|\beta|\|S\|\|u\|
        (|\alpha|\|T\|+|\beta|\|S\|)\|u\|\quad\text{for all }u\in X,
    \]
    which implies that $\alpha T+\beta S$ is bounded (and clearly linear) and is thus in $\mathcal{L}(X,Y)$.
    To show that the operator norm is a norm on $\mathcal{L}(X,Y)$:
    \begin{enumerate}[(i)]
        \item $\|T\|=0\iff T(u)=0$ follows from $\|T\|=\inf\{M\ge0\mid \|T(u)\|\le M\|u\|,u\in X\}$
        \item $\|\alpha T\|=|\alpha|\|T\|$ because $\|\alpha T(u)\|=|\alpha|\|T(u)\|$ implies that 
        \[
            \|\alpha T\|=\inf\{M\mid |\alpha|\| T(u)\|\le M\|u\|\}=\inf\{M\mid \| T(u)\|\le \frac{M}{|\alpha|}\|u\|\}=|\alpha|\|T\|.
        \]
        \item $\|T+S\|\le\|T\|+\|S\|$ follows from above:
        \[
            \|(T+S)(u)\|%\le\|T(u)\|+\|S(u)\|
            \le (\|T\|+\|S\|)\|u\|,
        \]
        and therefore $\|T+S\|=\inf\{M\ge0\mid \|(T+S)(u)\|\le M\|u\|,u\in X\}\le \|T\|+\|S\|$.
    \end{enumerate}
\end{proof}
\begin{namedthm*}{Theorem 3}
    Let $X$ and $Y$ be normed linear spaces.
    If $Y$ is a Banach space, then so is $\mathcal{L}(X,Y)$.
\end{namedthm*}
\begin{proof}
    Let $\{T_n\}$ be a Cauchy sequence in $\mathcal{L}(X,Y)$.
    Consider any $u\in X$.
    Then for all indices $n,m$,
    \[
        \|T_n(u)-T_m(u)\|=\|(T_n-T_m)(u)\|\le\|T_n-T_m\|\|u\|.
    \]  
    Thus $\{T_n(u)\}$ is a Cauchy sequence in $Y$.
    Since $Y$ is complete, then the sequence $\{T_n(u)\}$ converges to a member of $Y$, which we can denote $T(u)$.
    This $T(u)$ is defined for any $u\in X$, and so $T:X\to Y$ is well-defined.
    We must show that $T$ belongs to $\mathcal{L}(X,Y)$ and that $\{T_n\}\to T$ in $\mathcal{L}(X,Y)$.
    
    To show linearity of $T$:
    Using the fact that $T_n(u)$ converges to $T(u)$ for any $u$, and that each $T_n$ is linear, for any $u_1,u_2\in X$ and scalars $\lambda_1,\lambda_2$, we have 
    \[
        \lambda_1T(u_1)+\lambda_2T(u_2)=\lambda_1\lim_{n\to\infty}T_n(u_1)+\lambda_2\lim_{n\to\infty}T_n(u_2)=\lim_{n\to\infty}T_n(\lambda_1u_1+\lambda_2u_2)=T(\lambda_1u_1+\lambda_2u_1).
    \]
    
    To show boundedness of $T$ and convergence of $\{T_n\}$ to $T$:
    Fix $\epsilon>0$.
    Because $\{T_n\}$ is a Cauchy sequence, there exists an index $N$ such that for all $n\ge N$ and $k\ge1$, we have $\|T_n-T_{n+k}\|<\epsilon/2$.
    Therefore for all $u\in X$, using the definition of operator norm,
    \[
        \|T_n(u)-T_{n+k}(u)\|=\|(T_n-T_{n+k})(u)\|\le\|T_n-T_{n+k}\|\|u\|<\epsilon/2\|u\|.
    \]  
    Now fix $n\ge N$ and $u\in X$.
    Because $\lim_{k\to\infty}T_{n+k}(u)=T(u)$ and the norm is continuous, we have
    \[
        \|T_n(u)-T(u)\|=\|T_n(u)-\lim_{k\to\infty}T_{n+k}(u)\|\le\epsilon/2\|u\|.
    \]  
    Thus the linear operator $T_N-T$ is bounded, and because $T_N$ is also bounded, so is $T$.
    Moreover, by the definition of the operator norm, we have for $n\ge N$, that 
    $
        \|T_n-T\|\le\epsilon/2<\epsilon,
    $
    and thus $\{T_n\}\to T$ in $\mathcal{L}(X,Y)$.

\end{proof}

For two normed linear spaces $X$ and $Y$, an operator $T\in\mathcal{L}(X,Y)$ is called an \textbf{isomorphism} provided it is one-to-one, onto, and has a continuous inverse.
For $T$ in $\mathcal{L}(X,Y)$, if it is one-to-one and onto, its inverse is linear.
Therefore (because bounded $\iff$ continuous for linear operators,) to be an isomorphism requires the inverse to be bounded; that is, the inverse belong to $\mathcal{L}(X,Y)$.
Two normed linear spaces are said to be \textbf{isomorphic} provided there is an isomorphism between them.
An isomorphism that also preserves the norm is called an \textbf{isometric isomorphism}: it is an isometry of the metric structures associated with the norms.

For a linear operator $T:X\to Y$, the subspace of $X$, $\{x\in X\mid T(x)=0\}$, is called the \textbf{kernel} of $T$ and denoted by $\text{ker }T$.
Observe that $T$ is one-to-one iff $\text{ker }T=\{0\}$.
We denote the \textbf{image} of $T$, $T(X)$, by $\text{Im }T$.

\begin{center}
	\textbf{PROBLEMS}
\end{center}
\begin{enumerate}
	\setcounter{enumi}{8}
    \item Let $X$ and $Y$ be normed linear spaces and $T:X\to Y$ be linear.
    \begin{enumerate}[(i)]
        \item Show that $T$ is continuous iff it is continuous at a single point $u_0$ in $X$.
        
        \ \\$(\implies)$ Suppose that $T$ is continuous.
        Then it is trivial that $T$ is continuous for any point $u_0$ in $X$.

        \ \\$(\impliedby)$ Suppose that $T$ is continuous at the point $u_0$ in $X$.
        Then there exists a $\delta>0$ such that for any $x\in X$ such that $\|u_0-x\|_X<\delta$, then $\|Tu_0-Tx\|_X<\epsilon$. 
        Let $y:=u_0-x$, and by linearity, we have
        \[
            \|y-0\|_X=\|y\|_X<\delta\implies \|Ty-T0\|_Y=\|Ty\|_Y<\epsilon,
        \]
        which tells us that $T$ is continuous at $0$.
        Now consider any point $x'\in X$.
        Then by continuity at zero, there exists a $\delta>0$ such that 
        \[
            \text{ for }x\in X\text{ s.t. }\|x'-x\|_X=\|(x'-x)-0\|_X<\delta\implies\|T(x'-x)-T0\|_Y=\|Tx'-Tx\|_Y<\epsilon.
        \]
        Therefore $T$ is continuous at every point.
        \ \\\item Show that $T$ is Lipschitz iff it is continuous.
        \\$(\implies)$ Suppose that $T$ is Lipschitz with Lipschitz constant $L\ge0$.
        Then for any $x\in X$, by linearity of $T$, we have
        \[  
            \|Tx\|=\|Tx-T0\|\le L\|x-0\|=L\|x\|,
        \]
        and $T$ is bounded and thus continuous.
        \\$(\impliedby)$ Suppose that $T$ is continuous.
        Then $T$ is bounded and there exists an $M\ge0$ for which 
        \[  
            \|Tx-Ty\|=\|T(x-y)\|\le M\|x-y\|,
        \]
        and thus $T$ is Lipschitz.
        \\\item Show that neither $(i)$ nor $(ii)$ hold in the absence of the linearity assumption on $T$. 

        \ \\$(i)$ The function $f(x):=1$ for $x\ge0$ and $f(x):=0$ for $x<0$ is continuous at the point $2$ but not continuous on all of $X$ (namely, $0$).
        \ \\$(ii)$ The function $g(x):=\sqrt{|x|}$ is continuous but it is not Lipschitz.
    \end{enumerate}
    \ \\\item For $X$ and $Y$ normed linear spaces and $T\in\mathcal{L}(X,Y)$, show that $\|T\|$ is the smallest Lipschitz constant for the mapping $T$; that is, the smallest number $c\ge0$ for which
    \[
        \|T(u)-T(v)\|\le c\cdot\|u-v\|\text{ for all }u,v\in X.
    \]
    \\Consider any $x,y\in X$, and consider the vector $(x-y)\in X$.
    \\Suppose that $T$ is Lipschitz; that is, there exists a $c\ge0$ such that 
    \[
        \|T(u-v)\|_Y=\|T(u)-T(v)\|_Y\le L\cdot\|u-v\|_X.
    \]
    Because $T$ is a Lipschitz function it is thus continuous (previous Problem 9(ii)), and because $T$ is linear, it is also thus bounded (Theorem 1), and so the operator norm of $T$ is well-defined.
    In particular, $\|T\|$ is the infimum of all such $c\ge0$.
    \ \\\item For $X$ and $Y$ normed linear spaces and $Y\in\mathcal{L}(X,Y)$, show that 
    \[
        \|T\|=\sup\{\|T(u)\|\mid u\in X, \|u\|\le1\}.
    \]

    \ \\Define 
    \[
        \begin{split}
        M'&:=\inf\{M\ge0\mid\|T(u)\|\le M\|u\|\text{ for all }u\in X\},\\
        N'&:=\sup\{\|T(u)\|\mid u\in X, \|u\|\le1\}.
        \end{split}
    \]
    We aim to show that they are equal.
    
    First, for $u\neq0$, then $\|\frac{u}{\|u\|}\|\le1$ so that by linearity of $T$,
    \[\|T(\frac{u}{\|u\|})\|\le N'\implies\|Tu\|\le N'\|u\|\implies M'\le N'.\]
    On the other hand, for $u$ such that $\|u\|\le1$, then
    \[\|Tu\|\le M'\|u\|\le M'\implies N'\le M.\]
    Therefore $M'=N'$.
    \ \\\item For $X$ and $Y$ normed linear spaces, let $\{T_n\}\to T$ in $\mathcal{L}(X,Y)$ and $\{u_n\}\to u$ in $X$.
    Show that $\{T_n(u_n)\}\to T(u)$ in $Y$.

    \ \\
    \item Let $X$ be a Banach space and $T\in\mathcal{L}(X,Y)$ have $\|T\|<1$.
    \begin{enumerate}[(i)]
        \item Use the Contraction Mapping Principle to show that $I-T\in\mathcal{L}(X,Y)$ is one-to-one and onto.
        \item Show that $I-T$ is an isomorphism.
    \end{enumerate}
    \item (Neumann Series) Let $X$ be a Banach space and $Y\in\mathcal{L}(X,Y)$ have $\|T\|<1$.
    Define $T^0=Id$.
    \begin{enumerate}[(i)]
        \item Use the completeness of $\mathcal{L}(X,X)$ to show that $\sum_{n=1}^\infty T^n$ converges in $\mathcal{L}(X,X)$.
        \item Show that $(I-T)^{-1}=\sum_{n=0}^\infty T^n$
    \end{enumerate}
    \item For $X$ and $Y$ normed linear spaces and $T\in\mathcal{L}(X,Y)$, show that $T$ is an isomorphism iff there is an operator $S\in\mathcal{L}(Y,X)$ such that for each $u\in X$ and $v\in Y$,
    \[
        S(T(u))=u\text{ and }T(S(v))=v.
    \]
    \item For $X$ and $Y$ normed linear spaces and $T\in\mathcal{L}(X,Y)$, show that $\text{ker }T$ is a closed subspace of $X$ and that $T$ is one-to-one iff $\text{ker }T=\{0\}$.
    \item Let $(X,\rho)$ be a metric space containing the point $x_0$.
    Define $\text{Lip}_0(X)$ to be the set of real-valued Lipschitz functions $f$ on $X$ that vanish at $x_0$.
    Show that $\text{Lip}_0(X)$ is a linear space that is normed by defining, for $f\in\text{Lip}_0(X)$,
    \[
        \|f\|=\sup_{x\neq y}\frac{|f(x)-f(y)|}{\rho(x,y)}.
    \]
    Show that $\text{Lip}_0(X)$ is a Banach space.
    For each $x\in X$, define the linear functional $F_x$ on $\text{Lip}_0(X)$ by setting $F_x(f)=f(x)$.
    Show that $F_x$ belongs to $\mathcal{L}(\text{Lip}_0(X),\mathbb{R})$ and that for $x,y\in X$, $\|F_x-F_y\|=\rho(x,y)$.
    Thus $X$ is isometric to a subset of the Banach space $\mathcal{L}(\text{Lip}_0(X),\mathbb{R})$.
    Since any closed subset of a complete metric space is complete, this provides another proof of the existence of a completion for any metric space $X$.
    It also shows that any metric space is isometric to a subset of a normed linear space.
    \item Use the preceding problem to show that every normed linear space is a dense subspace of a Banach space.
    \item For $X$ a normed linear space and $T,S\in\mathcal{L}(X,X)$, show that the composition $S\circ T$ also belongs to $\mathcal{L}(X,X)$ and $\|S\circ T\|\le\|S\|\cdot\|T\|$.
    \item Let $X$ be a normed linear space and $Y$ a closed linear subspace of $X$.
    Show that $\|x\|_1=\inf_{y\in Y}\|x-y\|$ defines a pseudonorm on $X$.
    The normed linear space induced by the pseudonorm $\|\cdot\|_1$ (see Problem 8) is denoted by $X/Y$ and called the \textbf{quotient space} of $X$ modulo $Y$.
    Show that the natural map $\varphi$ of $X$ onto $X/Y$ takes open sets into open sets.
    \item Show that if $X$ is a Banach space and $Y$ a closed linear proper subspace of $X$, then the quotient $X/Y$ also is a Banach space and the natural map $\varphi:X\to X/Y$ has norm 1.
    \item Let $X$ and $Y$ be normed linear spaces, $T\in\mathcal{L}(X,Y)$ and $\text{ker }T=Z$.
    Show that there is a unique bounded linear operator $S$ from $X/Z$ onto $Y$ such that $T=S\circ\varphi$ where $\varphi:X\to X/Z$ is the natural map.
    Moreover, show that $\|T\|=\|S\|$.
\end{enumerate}

% 13.3
\authoredby{inprogress}
\section{Compactness Lost: Infinite Dimensional Normed Linear Spaces}

A linear space $X$ is said to be finite dimensional provided there is a subset $\{e_1,\dots,e_n\}$ of $X$ that spans $X$.
If no proper subset also spans $X$, we call the set $\{e_1,\dots,e_n\}$ a basis for $X$ and call $n$ the dimension of $X$.
If $X$ is not spanned by a finite collection of vectors it is said to be finite dimensional.
Observe that a basis $\{e_1,\dots,e_n\}$ for $X$ is linearly independent in the sense that 
\[
    \text{if }\sum_{i=1}^nx_ie_i=0,\ \text{then }x_i=0\ \text{for all }1\le i\le n,
\]
for otherwise a proper subset of $\{e_1,\dots,e_n\}$ would span $X$.

For example, if there exists a nontrivial solution to the above homogenous equation, then there exists an index $j$ such that
\[
    \sum_{i=1,i\neq j}^n \frac{x_i}{-x_j}e_i=e_j,
\]
and so for any $y\in X$, we can essentially remove the $e_j$: 
\[
    y=\sum_{i=1}^ny_ie_i= \sum_{i=1,i\neq j}^ny_ie_i+y_je_j=\sum_{i=1,i\neq j}^ny_ie_i+\sum_{i=1,i\neq j}^n y_j\frac{x_i}{-x_j}e_i=\sum_{i=1,i\neq j}^n(y_i-y_j\frac{x_i}{x_j})e_i
\]
That is, the proper subset $\{e_1,\dots,e_n\}\setminus \{e_j\}$ spans $X$.
\begin{namedthm*}{Theorem 4}
    Any two norms on a finite dimensional linear space are equivalent.
\end{namedthm*}
\begin{proof}
    Since equivalnece of norms is an equivalence relation on the set of norms on $X$, it suffices to select a particular norm $\|\cdot\|_*$ on $X$ and show that any norm on $X$ is equivalent to $\|\cdot\|_*$.
    Let $\text{dim} X=n$ and $\{e_1,\dots,e_n\}$ be a basis for $X$.
    For any $x=x_1e_1+\dots x_ne_n\in X$, define
    \[
        \|x\|_*=\sqrt{x_1^2+\dots+x_n^2}.
    \]
    This is a norm on $X$ because the Euclidean norm is a norm on $\mathbb{R}^n$.

    Now let $\|\cdot\|$ be any norm on $X$.
    We must show the existance of $c_1,c_2\ge0$ such that (1) $ \|x\|\le c_1\|x\|_*$ and (2) $ c_2\|x\|_*\le \|x\|$ for all $x\in X$.
    
    (1) Consider any $x=\sum_{i=1}^n x_ie_i\in X$, where $\{e_i\}$ is a basis for $X$.
    Then by subadditivity and absolute homogeneity of $\|\cdot\|$, and by Cauchy-Schwarz, we get
    \[
      \|x\|
      \le\sum_{i=1}^n|x_i|\|e_i\|
    %   =\sum_{i=1}^n|x_i|\sqrt{\|e_i\|^2}
    %   \le\sum_{i=1}^n|x_i|\sqrt{\sum_{j=1}^n\|e_j\|^2}
      \le\left(\sqrt{\sum_{j=1}^n\|e_j\|^2}\right)\sqrt{\sum_{j=1}^n|x_i|^2}
      =\left(\sqrt{\sum_{j=1}^n\|e_j\|^2}\right)\|x\|_*
      :=c_1\|x\|_*\tag{1}
    \]
    (2) Now define the real-valued function $f:\mathbb{R}^n\to\mathbb{R}$ by
    \[
        f(x_1,\dots,x_n):=\|\sum_{i=1}^n x_ie_i\|
    \]
    Then by (1), we see that $f$ is Lipschitz continuous with respect to the topology induced by the $\|\cdot\|_*$ norm on $\mathbb{R}^n$:
    \[
      |f(x)-f(x')|=|\sum_{i=1}^n (x_i-x_i')e_i|=\|x-x'\|\le c_1\|x-x'\|_*
    \]
    Because $\{e_1,\dots,e_n\}$ is linearly independent, then $f(x)=\|\sum_{i=1}^n x_ie_i\|=\|0\|=0\iff x_i=0$ for each $i$, which tells us that $f$ takes positive values on the boundary of the unit ball $S=\{x\in\mathbb{R}^n\mid \|x\|_*=(\sum_{i=1}^n x_i^2)^{1/2}=1\}$, which is compact because it is closed and bounded.
    A continuous real-valued function on a compact topological space takes a minimum value.
    Therefore there exists the value $m>0$ and the point $x_m\in S$ such that $f(x_m)=m$, and 
    \[
        0<m=f(x_m)\le f(x)\quad\text{for all }x\in S.
    \]
    Now, for any $x\neq0$, we have $\frac{x}{\|x\|_*}\in S$, so that 
    $
        m\le f(\frac{x}{\|x\|_*})=\|\frac{x}{\|x\|_*}\|,
    $
    and by absolute homogeneity of the norm $\|\cdot\|$, we have
    \[
        m\|x\|_*\le \|x\|\quad\text{for all }x\in X.\tag{2}
    \]  
    Therefore by (1) and (2), there exists $c_1,m\ge0$ such that 
    \[
        m\|x\|_*\le \|x\|\le c_1\|x\|_*\quad\text{for all }x\in X.
    \] 

\end{proof}
\begin{namedthm*}{Corollary 5}
    Any two normed space of the same finite dimension are isomorphic.
\end{namedthm*}
\begin{namedthm*}{Corollary 6}
    Any finite dimensional normed linear space is complete and therefore any finite dimensional subspace of a normed linear space is closed.
\end{namedthm*}
``completeness is preserved under isomorphisms''
\begin{namedthm*}{Corollary 7}
    The closed unit ball in a finite dimensional normed linear space is compact.
\end{namedthm*}
\begin{namedthm*}{Riesz's Theorem}
    The closed unit ball of a normed linear space $X$ is compact iff $X$ is finite dimensional.
\end{namedthm*}
\begin{namedthm*}{Riesz's Lemma}
    Let $Y$ be a closed proper linear subspace of a normed linear space $X$.
    Then for each $\epsilon>0$, there is a unit vector $x_0\in X$ for which 
    \[
        \|x_0-y\|>1-\epsilon\text{ for all }y\in Y.
    \]
\end{namedthm*}
\begin{proof}[Proof of Riesz's Theorem]
    By Corollary 7, the closed unit ball in a finite dimensional normed linear space is compact.
    It remains to show that $B$ fails to be compact if $X$ is finite dimensional.
    Assume $X$ is finite dimensional.
    We will inductively choose a sequence $\{x_n\}$ in $B$ such that $\|x_n-x_m\|>1/2$ for $n\neq m$.
    This sequence has no Cauchy subsequence and therefore no convergent subsequence.
    Thus $B$ is not sequentially compact (see chapter 9.5), and therefore, since $B$ is a metric space, not compact.

    It remains to choose this subsequence.
    Choose any vector $x_1\in B$.
    For a natural number $n$, suppose we have chosen $n$ vectors in $B$, $\{x_1,\dots,x_n\}$, each pair of which are more than a distance $1/2$ apart.
    Let $X_n$ be the linear space spanned by these $n$ vectors.
    Then $X_n$ is a finite dimensional subspace of $X$, and so by Corollary 6, it is closed.
    Moreover, $X_n$ is a proper subspace of $X$ since $\text{dim }X=\infty$.
    By the preceding lemma we may choose $x_{n+1}$ in $B$ such that $\|x_{n+1}-x_i\|>1/2$ for all $1\le i\le n$.
    Thus we have inductively chosen a sequence in $B$, any two terms of which are more than a distance $1/2$ apart.
\end{proof}

\begin{center}
	\textbf{PROBLEMS}
\end{center}
\begin{enumerate}
	\setcounter{enumi}{22}
    \item Show that a subset of a finite dimensional normed linear space $X$ is compact iff it is closed and bounded.
    \\\item Complete the proof of Riesz's Lemma for $\epsilon\neq1/2$.
    
    \ \\The case where $\epsilon\ge1$ holds trivially, so consider $0<\epsilon<1$.
    Since $Y$ is a proper subset of $X$, there exists $x\in X\setminus Y$.
    Then, because $Y$ is closed, then $X\setminus Y$ is open, so that there exists an open ball centered at $x$ that is disjoint from $Y$.
    That is, 
    % That is, $y\notin\mathbb{B}(x,\delta)=\{z\in X\mid \|x-z\|<\delta\}\implies$
    \[
        0<d:=\inf\{\|x-y\|, y\in Y\}.\tag{1}
    \]
    Because $1-\epsilon<1$, by definition of infimum, there exists a vector $y_1\in Y$ for which 
    \[
        d\le\|x-y_1\|<\frac{d}{1-\epsilon}.\tag{2}
    \]
    Now define 
    \[
        x_0:=\frac{x-y_1}{\|x-y_1\|},
    \]
    so that $x_0$ is a unit vector.
    Moreover, for any $y\in Y$, we have
    \[
        x_0-y=\frac{x-(y_1+y\|x-y_1\|)}{\|x-y_1\|}:=\frac{1}{\|x-y_1\|}(x-y'),
    \] 
    where $y'=y_1+y\|x-y_1\|$ is a linear combination of elements of $Y$, and so is in $Y$.
    
    Therefore by (1) and (2),
    \[
        \|x_0-y\|=\frac{1}{\|x-y_1\|}\|x-y'\|>\frac{1-\epsilon}{d}\|x-y'\|\ge1-\epsilon.
    \]
    \\\item Exhibit an open cover of the closed unit ball of $X=\ell^2$ that has no finite subcover.
    Then do the same for $X=C[0,1]$ and $X=L^2[0,1]$.
    \\\item For normed linear spaces $X$ and $Y$, let $T:X\to Y$ be linear.
    If $X$ is finite dimensional, show that $T$ is continuous. 
    If $Y$ is finite dimensional, show that $T$ is continuous iff $\text{ker }T$ is closed.

    From Chapter 13 Theorem 1, we showed that continuous is equivalent to bounded for the linear operator $T$.
    Therefore it is sufficient to show that $T$ is bounded.
    Because $X$ is finite dimensional of some dimension $n$, choose a normalized basis $\{e_i\}_{i\in[n]}$ of $X$.
    Thus for any $x\in X$, then $x=\sum_{i=1}^nx_ie_i$, and then by linearity, $T(x)=\sum_{i=1}^nx_if(e_i)$.
    By subadditivity and absolute homogeneity,
    \[
        \|f(x)\|_Y\le\sum_{i=1}^n|x_i|\|f(e_i)\|_Y.
    \]
    \begin{enumerate}
        \item Let $M:=\sup_{i\in[n]}\{\|f(e_i)\|_Y\}=\max_{i\in[n]}\{\|f(e_i)\|_Y\}$.
        \item Note that any two norms on a finite-dimensional space are equivalent, so that 
        \[
            \sum_{i=1}^n|x_i|=\|x\|_1\le C\|x\|_X\quad\text{for some }C>0.
        \]
    \end{enumerate}
    Therefore by (i) and (ii),
    \begin{align*}
        \|f(x)\|_Y
        &\le\sum_{i=1}^n|x_i|\|f(e_i)\|_Y\\
        &\le C\|x\|_XM,
    \end{align*}
    which implies $f$ is bounded.
    \\\item (Another proof of Riesz's Theorem) Let $X$ be an infinite dimensional normed linear space, $B$ the closed unit ball in $X$, and $B_0$ the unit open ball in $X$.
    Suppose $B$ is compact.
    Then the open cover $\{x+(1/3)B_0\}_{x\in B}$ of $B$ has a finite subcover $\{x_i+(1/3)B_0\}_{1\le i\le n}$.
    Use Riesz's Lemma with $Y=\text{span}[\{x_1,\dots,x_n\}]$ to derive a contradiction.
    \\\item Let $X$ be a normed linear space.
    Show that $X$ is separable iff there is a compact subset $K$ of $X$ for which $\overline{\text{span}}[K]=X$.
\end{enumerate}

% 13.4
\authoredby{inprogress}
\section{The Open Mapping and Closed Graph Theorems}\

\begin{center}
	\textbf{PROBLEMS}
\end{center}
\begin{enumerate}
	\setcounter{enumi}{28}
    \item Let $X$ be a finite dimensional normed linear space and $Y$ a normed linear space.
    Show that every linear operator $T:X\to Y$ is continuous and open.\\
    \\Let $\|\cdot\|_X$ and $\|\cdot\|_Y$ be the norms on $X$ and $Y$ respectively.
    \\Because $X$ is finite dimensional, we can choose an orthonormal basis $\{e_1,\dots,e_n\}$ of $X$.
    \\Define 
    \[
        M:=\max\{\|T(e_1)\|_Y,\dots,\|T(e_n)\|_Y\}\ge0.
    \]
    By Chapter 13 Theorem 4, any two norms on the finite dimensional linear space $X$ are equivalent; 
    therefore in particular there exists $c\ge0$ such that
    \[
        \sum_{i=1}^n|x_i|=\|x\|_1\le c\|x\|_X\text{ for all }x\in X\tag{1}
    \] 
    Now consider any $x\in X$.
    \begin{align*}
        \|T(x)\|_Y&=\|T(\sum_{i=1}^nx_ie_i)\|_Y&&\text{using the orthonormal basis}\\
        &=\|\sum_{i=1}^nx_iT(e_i)\|_Y&&\text{by linearity of }T\\
        &\le\sum_{i=1}^n\|x_iT(e_i)\|_Y&&\text{by subadditivity of the norm }\|\cdot\|_Y\\
        &=\sum_{i=1}^n|x_i|\|T(e_i)\|_Y&&\text{by absolute homogeneity of the norm }\|\cdot\|_Y\\
        &\le\sum_{i=1}^n|x_i|M&&\text{by definition of }M\\
        &=\|x\|_1M&&\text{by definition of the 1-norm}\\
        &\le c\|x\|_XM,&&\text{by equivalence of norms (1)}
    \end{align*}
    and therefore there exists the constant $c\cdot M\ge0$ such that 
    \[
        \|T(x)\|_Y\le(c\cdot M)\|x\|_X\text{ for all }x\in X,
    \]
    which implies that $T$ is bounded, and thus by Chapter 13 Theorem 1, it is continuous.
    \\(it remains to show that $T$ is open)
    \item Let $X$ be a Banach space and $P\in\mathcal{L}(X,X)$ be a projection.
    Show that $P$ is open.
    \item Let $T:X\to Y$ be a continuous linear operator between the Banach spaces $X$ and $Y$.
    Show that $T$ is open if the image under $T$ of the open unit ball in $X$ is dense in a neighborhood of the origin in $Y$.
    \item Let $\{u_n\}$ be a sequence in a Banach space $X$.
    Suppose that $\sum_{k=1}^\infty\|u_k\|<\infty$.
    Show that there is an $x\in X$ for which
    \[
        \lim_{n\to\infty}\sum_{k=1}^\infty u_k=x.
    \]
    \item Let $T$ be a linear operator from a normed linear space $X$ to a finite-dimensional normed linear space $Y$.
    Show that $T$ is continuous iff $\text{ker }T$ is a closed subspace of $X$.
    \item Let $X$ be a Banach space, the operator $T\in\mathcal{L}(X,X)$ be open and $X_0$ be a closed subspace of $X$.
    The restriction $T_0$ of $T$ to $X_0$ is continuous.
    Is $T_0$ necessarily open?
    \item Let $V$ be a linear subspace of a linear space $X$.
    Argue as follows to show that $V$ has a linear complement in $X$.
    \begin{enumerate}[(i)]
        \item If $\text{dim }X<\infty$, let $\{e_i\}_{i=1}^n$ be a basis for $V$.
        Extend this basis for $V$ to a basis $\{e_i\}_{i=1}^{n+k}$ for $X$.
        Then define $W=\text{span}[\{e_{n+1},\dots,e_{n+k}\}]$.
        \item If $\text{dim }X=\infty$, apply Zorn's Lemma to the collection $\mathcal{F}$ of all subspaces $Z$ of $X$ for which $V\cap Z=\{0\}$, ordered by set inclusion.
        \item Verify (15) and (16).
        \item Let $Y$ be a normed linear space.
        Show that $Y$ is a Banach space iff there is a Banach space $X$ and a continuous, open mapping of $X$ onto $Y$.
    \end{enumerate}
\end{enumerate}

% 13.5
\authoredby{untouched}
\section{The Uniform Boundedness Principle}
\begin{center}
	\textbf{PROBLEMS}
\end{center}
\begin{enumerate}
	\setcounter{enumi}{37}
    \item As a consequence of the Baire Category Theorem we showed that a real-valued mapping that is the pointwise limit of a sequence of continuous mapping on a complete metric space must be continuous at all points of a dense subset of its domain.
    Adapt that proof so that it applies to mapping into any metric space.
    Use this to prove that the pointwise limit of a sequence of continuous linear operators on a Banach space has a limit that is continuous at some point and hence, by linearity, is continuous.
    \item Let $\{f_n\}$ be a sequence in $L^\infty[a,b]$.
    Suppose that for each $g\in L^1[a,b]$, $\lim_{n\to\infty}\int_a^bg\cdot f_n$ exists.
    Show that there is a function $f\in L^\infty[a,b]$ such that $\lim_{n\to\infty}\int_a^bg\cdot f_n=\int_a^bg\cdot f$ for all $g\in L^1[a,b]$.
    \item Let $X$ be the linear space of all polynomials defined on $\mathbb{R}$.
    For $p\in X$, define $\|p\|$ to be the sum of the absolute values of the coefficients of $p$.
    Show that this is a norm on $X$.
    For each $n$, define $\psi_n:X\to\mathbb{R}$ by $\psi_n(p)=p^{(n)}(0)$.
    Use the properties of the sequence $\{\psi_n\}$ in $\mathcal{L}(X,\mathbb{R})$ to show that $X$ is not a Banach space.
\end{enumerate}

% Chapter 14
\authoredby{inprogress}
\chapter{Duality for Normed Linear Spaces}

For a normed linear space $X$, we denoted the normed linear space of continuous linear real-valued functions of $X$ by $X^*$ and called it the \textbf{dual space} of $X$.
We aim to explore properties of the mapping from $X\times X^*$ to $\mathbb{R}$ defined by
\[
    (x,\psi)\mapsto\psi(x)\text{ for all }x\in X,\psi\in X^*.
\]

Hahn-Banach Theorem: extension of certain linear functionals on subspaces of an unnormed linear space to linear functionals on the whole space.
Hahn-Banach implies:
\begin{enumerate}
    \item for a normed linear space $X$, any bounded linear functional on a subspace of $X$ may be extended to a bounded linear functional on all of $X$, without increasing its norm
    \item for a locally convex Topological vector space $X$, any two disjoint closed convex sets of $X$ may be separated by a closed hyperplane
    \item for a reflexive Banach space $X$, any bounded sequence in $X$ has a weakly convergent subsequence.
\end{enumerate}

% 14.1
\authoredby{inprogress}
\section{Linear Functionals, Bounded Linear Functionals, and Weak Topologies}



\begin{center}
	\textbf{PROBLEMS}
\end{center}
\begin{enumerate}
	\setcounter{enumi}{0}
    \item hellos
\end{enumerate}

% 14.2
\authoredby{untouched}
\section{The Hahn-Banach Theorem}

% 14.3
\section{Reflexive Banach Spaces and Weak Sequential Convergence}

% 14.4
\section{Locally Convex Topological Vector Spaces}

% 14.5
\section{The Separation of Convex Sets and Mazur's Theorem}

% 14.6
\section{The Krein-Milman Theorem}

% Chapter 15
\chapter{Compactness Regained: The Weak Topology}

\section{Alaoglu's Extension of Helley's Theorem}
\section{Reflexivity and Weak Compactness: Kakutani's Theorem}
\section{Compactness and Weak Sequential Compactness: The Eberlein-\v Smulian Theorem}
\section{Metrizability of Weak Topologies}

% Chapter 16
\chapter{Continuous Linear Operators on Hilbert Spaces}

% 16.1
\section{The Inner Product and Orthogonality}
\begin{center}
	\textbf{PROBLEMS}
\end{center}
\begin{enumerate}
	\setcounter{enumi}{0}
\end{enumerate}

% 16.2
\section{The Dual Space and Weak Sequential Convergence}

% 16.3
\section{Bessel's Inequality and Orthonormal Bases}

% 16.4
\section{Adjoints and Symmetry for Linear Operators}

% 16.5
\section{Compact Operators}

% 16.6
\section{The Hilbert-Schmidt Theorem}

% 16.7
\section{The Riesz-Schauder Theorem: Characterization of Fredholm Operators}


% Chapter 17
\authoredby{finished}
\chapter{General Measure Spaces: Their Properties and Construction}

% 17.1
\authoredby{finished}
\section{Measures and Measurable Sets}
\begin{flushleft}
	\begin{namedthm*}{Definition}
		By a \textbf{measurable space} we mean a couple $(X,\mathcal{M})$ consisting of a set $X$ and a $\sigma$-algebra $\mathcal{M}$ of subsets of $X$.
		A subset $E$ of $X$ is called \textbf{measurable} (or measurable with respect to $\mathcal{M}$) provided $E$ belongs to $\mathcal{M}$.
	\end{namedthm*}
	\begin{namedthm*}{Definition}
		By a \textbf{measure} $\mu$ on a measurable space $(X,\mathcal{M})$ we mean an extended real-valued nonnegative set function $\mu:\mathcal{M}\to[0,\infty]$ for which $\mu(\emptyset)=0$ and which is \textbf{countably additive} in the sense that for any countable disjoint collection $\{E_k\}_{k=1}^\infty$ of measurable sets,
		\[
			\mu\biggl(\bigcup_{k=1}^\infty E_k\biggr)=\sum_{k=1}^\infty\mu(E_k).	
		\]
	\end{namedthm*}
	\begin{namedthm*}{Definition}
		By a \textbf{measure space} $(X,\mathcal{M},\mu)$ we mean a measurable space $(X,\mathcal{M})$ together with a measure $\mu$ defined on $\mathcal{M}$.
	\end{namedthm*}
	\begin{namedthm*}{Proposition 1}
		Let $(X,\mathcal{M},\mu)$ be a measure space.
		\begin{enumerate}[\indent {}]
			\item (Finite Additivity) For any finite disjoint collection $\{E_k\}_{k=1}^n$ of measurable sets,
			\[
				\mu\biggl(\bigcup_{k=1}^n E_k\biggr)=\sum_{k=1}^n\mu(E_k).	
			\]
			\item (Monotonicity) If $A$ and $B$ are measurable sets and $A\subseteq B$, then
			\[
				\mu(A)\le\mu(B).
			\]
			\item (Excision) If, moreover, $A\subseteq B$ and $\mu(A)<\infty$, then
			\[
				\mu(B\setminus A)=\mu(B)-\mu(A),
			\]
			so that if $\mu(A)=0$, then
			\[
				\mu(B\setminus A)=\mu(B).
			\]
			\item (Countable Monotonicity) For any countable collection $\{E_k\}_{k=1}^n$ of measurable sets that covers a measurable set $E$,
			\[
				\mu(E)\le\sum_{k=1}^\infty\mu(E_k).	
			\]
		\end{enumerate}
	\end{namedthm*}
	\begin{namedthm*}{Definition}
		Let $(X,\mathcal{M},\mu)$ be a measure space. 
		The measure $\mu$ is called \textbf{finite} provided $\mu(X)<\infty$. 
		It is called \textbf{$\sigma$-finite} provided $X$ is the union of a countable collection of measurable sets, each of which has finite measure.
		A measurable set $E$ is said to be of \textbf{finite measure} provided $\mu(E)<\infty$, and is said to be \textbf{$\sigma$-finite} provided $E$ is the union of a countable collection of measurable sets, each of which has finite measure.
	\end{namedthm*}
	\begin{namedthm*}{Definition}
		A measure space $(X,\mathcal{M},\mu)$ is said to be \textbf{complete} provided $\mathcal{M}$ contains all subsets of sets of measure zero, that is, if $E$ belongs to $\mathcal{M}$ and $\mu(E)=0$, then every subset of $E$ also belongs to $\mathcal{M}$.
	\end{namedthm*}
	For example, the Lebesgue measure $m$ on the real line is complete. 
	Moreover, in Chapter 2 Proposition 22, we showed that the Cantor set $C$, a Borel set that has Lebesgue measure zero, contains a Lebesgue measurable set that is not a Borel set.
	Therefore the Lebesgue measure restricted to the Borel $\sigma$-algebra $\mathcal{B}$ is not complete because $C$ belongs to $\mathcal{B}$ and $m(C)=0$ but there exists a subset $A\subseteq C$ such that $A\notin\mathcal{B}$.\\
	\medskip
	The following proposition tells us that each measure space can be completed.
	\begin{namedthm*}{Proposition 3}
		Let $(X,\mathcal{M},\mu)$ be a measure space.
		Define $\mathcal{M}_0$ to be the collection of subsets $E$ of $X$ of the form $E=A\cup B$ where $B\in\mathcal{M}$ and $A\subseteq C$ for some $C\in\mathcal{M}$ for which $\mu(C)=0$.
		For such a set $E$ define $\mu_0(E)=\mu(B)$. 
		Then $\mathcal{M}_0$ is a $\sigma$-algebra that contains $\mathcal{M}$, $\mu_0$ is a measure that extends $\mu$, and $(X,\mathcal{M}_0,\mu_0)$ (the \textbf{completion} of $(X,\mathcal{M},\mu)$) is a complete measure space.
	\end{namedthm*}
\end{flushleft}
\begin{center}
	\textbf{PROBLEMS}
\end{center}
\begin{enumerate}
	\setcounter{enumi}{0}
	\item Let $f$ be a nonnegative Lebesgue measurable function on $\mathbb{R}$. 
	For each Lebesgue measurable subset $E$ of $\mathbb{R}$, define $\mu(E) = \smallint_E f$, the Lebesgue integral of $f$ over $E$.
	Show that $\mu$ is a measure on the $\sigma$-algebra of Lebesgue measurable subsets of $\mathbb{R}$.\\
	\\Because $f$ is nonnegative, by monotonicity of integration, for any Lebesgue measurable set $E$, 
	\[
		0\le f\implies 0=\int_E 0\le \int_E f=\mu(E).
	\]
	Check Chapter 4 Problem 28 to see that for $f$ Lebesgue integrable over $\mathbb{R}$ and $\emptyset$ a Lebesgue measurable subset of $\mathbb{R}$, we have that
	\[
		\mu(\emptyset)=\int_\emptyset f=\int_\mathbb{R} f\cdot\chi_\emptyset=\int_\mathbb{R} f\cdot\chi_\emptyset=\int_\mathbb{R} 0 = 0.
	\]
	Even more simply, even if $f$ is not Lebesgue integrable, check Chapter 4 Problem 17 to see that
	\[
		\mu(\emptyset)=\int_\emptyset f=0.
	\]
	Let $\{E_n\}_{n=1}^\infty$ be a disjoint countable collection of Lebesgue measurable sets so that each $\mu(E_n) = \smallint_{E_n} f$ is defined.
	Then $E=\bigcup_{n=1}^\infty E_n$ is Lebesgue measurable, and $\mu(E) = \smallint_E f$ is defined.
	\\For $f$ Lebesgue integrable over $\mathbb{R}$, by Chapter 4 Theorem 20,
	\[
		\mu(\bigcup_{n=1}^\infty E_n)=\mu(E)=\int_E f =\sum_{n=1}^\infty\int_{E_n}f=\sum_{n=1}^\infty\mu(E_n).
	\]
	If $f$ is not Lebesgue integrable, we can still consider the sequence of functions $f_n$ on $E$ defined by
	\[
		f_n:=f\cdot\chi_{A_n},\text{ where }A_n:=\bigcup_{k=1}^nE_k.
	\]
	Then because $f$ is nonnegative, $\{f_n\}$ is an increasing sequence of nonnegative measurable functions on $E$, where $\{f_n\}\to f$ pointwise on $E$.
	By the Monotone Convergence Theorem (Chapter 4), and (finite) Additivity Over Domains of Integration for each $f_n$ (Chapter 4 Theorem 11), we have
	\[
		\infty=\int_E f=\lim_{n\to\infty}\int_Ef_n=\lim_{n\to\infty}\sum_{k=1}^n\int_{E_k}f_n=\sum_{k=1}^\infty\int_{E_k}f_n=\sum_{k=1}^\infty\mu(E_k),
	\] 
	so that $\mu(\bigcup_{n=1}^\infty E_n)=\mu(E)=\int_E f =\infty=\sum_{n=1}^\infty\mu(E_n)$.
	\\Therefore $\mu$ is a measure on the $\sigma$-algebra of Lebesgue measurable sets. 
	\item Let $\mathcal{M}$ be a $\sigma$-algebra of subsets of a set $X$ and the set function $\mu : \mathcal{M} \to [0,\infty)$ be finitely additive.
	Prove that $\mu$ is a measure iff whenever $\{A_k\}_{k=1}^\infty$ is an ascending sequence of sets in $\mathcal{M}$, then
	\[
	\mu \biggl ( \bigcup_{k=1}^\infty A_k \biggr ) = \lim_{k \to \infty} \mu(A_k).	
	\]
	\\$(\implies)$ Suppose that $\mu$ is a measure.\\
	Then by Continuity of Measure, the conclusion follows.\\
	\\$(\impliedby)$ Suppose that whenever $\{A_k\}_{k=1}^\infty$ is an ascending sequence of sets in $\mathcal{M}$, then $\mu ( \bigcup_{k=1}^\infty A_k ) = \lim_{k \to \infty} \mu(A_k)$.
	(See Chapter 2 Problem 28.)\\
	Finite additivity of $\mu$ means that for any finite disjoint collection $\{E_k\}_{k=1}^n$ of measurable sets, we have $\mu(\bigcup_{k=1}^n E_k)=\sum_{k=1}^n\mu(E_k)$.
	\\We define $F_n=\bigcup_{k=1}^n E_k$ so that $\{F_n\}_{n=1}^\infty$ is an ascending sequence of sets in $\mathcal{M}$, and thus $\mu ( \bigcup_{n=1}^\infty F_n ) = \lim_{n \to \infty} \mu(F_n)$.
	\\Thus we see
	\[
		\mu(\bigcup_{n=1}^\infty E_n)=\mu(\bigcup_{n=1}^\infty F_n)=\lim_{n \to \infty} \mu(F_n)=\lim_{n \to \infty} \mu(\bigcup_{k=1}^n E_k)=\lim_{n \to \infty}\sum_{k=1}^n\mu(E_k)=\sum_{k=1}^\infty\mu(E_k),
	\]
	that is, $\mu$ satisfies countable additivity, and thus $\mu$ is a measure.
	\item Let $\mathcal{M}$ be a $\sigma$-algebra of subsets of a set $X$. Formulate and establish a correspondent of the preceding problem for descending sequences of sets in $\mathcal{M}$.\\
	\\Let $\mathcal{M}$ be a $\sigma$-algebra of subsets of a set $X$ and the set function $\mu : \mathcal{M} \to [0,\infty)$ be finitely additive.
	Prove that $\mu$ is a measure iff whenever $\{A_k\}_{k=1}^\infty$ is a descending sequence of sets in $\mathcal{M}$ with $m(A_1)<\infty$, then
	\[
	\mu \biggl ( \bigcap_{k=1}^\infty A_k \biggr ) = \lim_{k \to \infty} \mu(A_k).	
	\]
	\\$(\implies)$ Suppose that $\mu$ is a measure.\\
	Then by Continuity of Measure, the conclusion follows.\\
	\\$(\impliedby)$ Suppose that whenever $\{A_k\}_{k=1}^\infty$ is a descending sequence of sets in $\mathcal{M}$ with $\mu(A_1)<\infty$, then $\mu ( \bigcap_{k=1}^\infty A_k ) = \lim_{k \to \infty} \mu(A_k)$.\\
	Finite additivity of $\mu$ means that for any finite disjoint collection $\{E_k\}_{k=1}^n$ of measurable sets, we have $\mu(\bigcup_{k=1}^n E_k)=\sum_{k=1}^n\mu(E_k)$.
	\\We can consider $\{\bigcup_{k={n+1}}^\infty E_k\}_{n=1}^\infty$, a descending sequence of sets in $\mathcal{M}$ with $\mu(\bigcup_{k=2}^\infty E_k)<\infty$, and then because $\{E_k\}_{k=1}^n$ is disjoint,
	\[
		\mu ( \bigcap_{n=1}^\infty [\bigcup_{k={n+1}}^\infty E_k]) = \lim_{n \to \infty} \mu(\bigcup_{k={n+1}}^\infty E_k)=0.
	\]
	\\Thus we see
	\begin{align*}
		\mu(\bigcup_{k=1}^\infty E_k)&=\mu([\bigcup_{k=1}^n E_k]\cup[\bigcup_{k={n+1}}^\infty E_k])\\
		&=\mu(\bigcup_{k=1}^n E_k)+\mu(\bigcup_{k={n+1}}^\infty E_k)&&\text{ by disjoint additivity}\\
		&=\sum_{k=1}^n\mu( E_k)+\mu(\bigcup_{k={n+1}}^\infty E_k)&&\text{ by disjoint additivity}
	\end{align*}
	The left hand side is independent of $n$, so taking the limit, we have
	\begin{align*}
		\mu(\bigcup_{k=1}^\infty E_k)&=\lim_{n\to\infty}\sum_{k=1}^n\mu( E_k)+\lim_{n\to\infty}\mu(\bigcup_{k={n+1}}^\infty E_k)\\
		&=\sum_{k=1}^\infty\mu( E_k)+0\\
		&=\sum_{k=1}^\infty\mu( E_k),
	\end{align*}
	that is, $\mu$ satisfies countable additivity, and thus $\mu$ is a measure.
	\item Let $\{(X_\lambda,\mathcal{M}_\lambda,\mu_\lambda)\}_{\lambda\in\Lambda}$ be a collection of measure spaces parametrized by the set $\Lambda$.
	Assume the collection of sets $\{X_\lambda\}_{\lambda\in\Lambda}$ is disjoint.
	Then we can form a new measure space (called their union) $(X,\mathcal{B},\mu)$ by letting $X=\bigcup_{\lambda\in\Lambda}X_\lambda$, letting $\mathcal{B}$ be the collection of subsets $B$ of $X$ such that $B\cap X_\lambda\in\mathcal{M}_\lambda$ for all $\lambda\in\Lambda$, and defining $\mu(B)=\sum_{\lambda\in\Lambda}\mu_\lambda[B\cap X_\lambda]$ for $B\in\mathcal{B}$.
	\begin{enumerate}[label=(\roman*),align=left]   
		\item Show that $\mathcal{B}$ is a $\sigma$-algebra.\\
		\\We have:
		\begin{enumerate}[label=(\roman*),align=left]   
			\item $X\in\mathcal{B}$ because $X\subseteq X$ such that for any $\lambda'\in\Lambda$,
			\begin{align*}
				X\cap X_{\lambda'}&= \bigcup_{\lambda\in\Lambda}X_\lambda\cap X_{\lambda'}= X_{\lambda'},
			\end{align*}
			where $X_{\lambda'}\in\mathcal{M}_{\lambda'}$ because $(X_{\lambda'},\mathcal{M}_{\lambda'},\mu_{\lambda'})$ is a measure space.
			\item if $B\in\mathcal{B}$, then $B\subseteq X$ such that for any $\lambda'\in\Lambda$, $B\cap X_{\lambda'}\in\mathcal{M}_{\lambda'}$.\\
			Then $B^c\subseteq X$ and because $ B\cap X_{\lambda'}\in\mathcal{M}_{\lambda'}$ and $X_{\lambda'}\in\mathcal{M}_{\lambda'}$,
			\[
				\mathcal{M}_{\lambda'}\ni[ B\cap X_{\lambda'}]^c\cap X_{\lambda'}=[B^c\cup X_{\lambda'}^c]\cap X_{\lambda'}=[B^c\cap X_{\lambda'}]\cup[X_{\lambda'}^c\cap X_{\lambda'}]=B^c\cap X_{\lambda'}.
			\]
			Therefore $B^c\in\mathcal{B}$.
			\item if $B_i\in\mathcal{B}$, then $B_i\in X$ such that for any $\lambda'\in\Lambda$, $B_i\cap X_{\lambda'}\in\mathcal{M}_{\lambda'}$ for all $i$.\\
			Then $\bigcup_{i=1}^\infty B_i\in X$ and $[\bigcup_{i=1}^\infty B_i]\cap X_{\lambda'} =\bigcup_{i=1}^\infty [B_i\cap X_{\lambda'}]\in\mathcal{M}_{\lambda'}$.
			\\Therefore $\bigcup_{i=1}^\infty B_i\in\mathcal{B}$.
		\end{enumerate}
		\item Show that $\mu$ is a measure.\\
		\\For $B\in\mathcal{B}$, we have $B\cap X_{\lambda'}\in\mathcal{M}_{\lambda'}$ for all $\lambda'\in\Lambda$, and so $\mu_{\lambda'}[B\cap X_{\lambda'}]$ is defined.
		\\Then $\mu_{\lambda'}[B\cap X_{\lambda'}]\ge0$ for all $\lambda'\in\Lambda$, which implies $\mu(B)=\sum_{\lambda\in\Lambda}\mu_\lambda[B\cap X_\lambda]\ge0$.
		\\Then because $\emptyset=X^c$ is in $\mathcal{B}$, then $\emptyset=\emptyset\cap X_{\lambda'}\in\mathcal{M}_{\lambda'}$ for all $\lambda'\in\Lambda$, and then $\mu(\emptyset)=\sum_{\lambda\in\Lambda}\mu_\lambda[\emptyset\cap X_\lambda]=\sum_{\lambda\in\Lambda}0=0$.
		\\Finally, consider any countable disjoint collection $\{B_k\}_{k=1}^\infty$ in $\mathcal{B}$.
		Then for any $\lambda'\in\Lambda$, the collection $\{B_k\cap X_{\lambda'}\}_{k=1}^\infty$ is disjoint so that 
		\begin{align*}
			\mu(\bigcup_{k=1}^\infty B_k)&= \sum_{\lambda\in\Lambda}\mu_\lambda[(\bigcup_{k=1}^\infty B_k)\cap X_\lambda]\\
			&=\sum_{\lambda\in\Lambda}\mu_\lambda[\bigcup_{k=1}^\infty (B_k\cap X_\lambda)]\\
			&=\sum_{\lambda\in\Lambda}\sum_{k=1}^\infty \mu_\lambda(B_k\cap X_\lambda)\\
			&=\sum_{k=1}^\infty \sum_{\lambda\in\Lambda} \mu_\lambda(B_k\cap X_\lambda)\\
			&=\sum_{k=1}^\infty \mu(B_k).
		\end{align*}
		Therefore $\mu$ is a measure.
		\item Show that $\mu$ is $\sigma$-finite iff all but a countable number of the measures $\mu_\lambda$ have $\mu(X_\lambda)=0$ and the remainder are $\sigma$-finite.\\
		\\$(\implies)$ Suppose $\mu$ is $\sigma$-finite.\\
		Then $X$ can be written as the countable union of disjoint measurable sets, each of which has finite measure under $\mu$.
		\\That is, we have $X=\bigcup_{k=1}^\infty A_k$, with $A_k\in\mathcal{B}$ s.t. $\mu(A_k)<\infty$ for each $k$.
		\\So $A_k\in\mathcal{B}\implies A_k\cap X_\lambda\in\mathcal{M}_\lambda$ for each $\lambda$, and $\sum_{\lambda\in\Lambda} \mu_\lambda(A_k\cap X_\lambda)=\mu(A_k)<\infty\implies\mu_\lambda(A_k\cap X_\lambda)<\infty$ for each $\lambda$.
		\\Then 
		\[
			\mu(X)=\mu(\bigcup_{k=1}^\infty A_k)=\sum_{k=1}^\infty\mu(A_k)=\sum_{k=1}^\infty\sum_{\lambda\in\Lambda} \mu_\lambda(A_k\cap X_\lambda)
		\]
		\\Then all but a countable number of the measures $\mu_\lambda$ can be nonzero, and the remainder must be $\sigma$-finite.\\
		\\$(\impliedby)$ Suppose all but a countable number of the measures $\mu_\lambda$ have $\mu(X_\lambda)=0$ and the remainder are $\sigma$-finite.\\
		Let $\Lambda^*\subseteq\Lambda$ be the set of measures $\mu_\lambda$ such that $\mu_\lambda(X_\lambda)=0$.
		\\Let $\Lambda^{*c}=\{\lambda_k\}_{k=1}^\infty\subseteq\Lambda$ be a countable collection such that each $\mu_{\lambda_k}$ is $\sigma$-finite.
		\\By definition of $\sigma$-finite, for each $k$, we have $X_{\lambda_k}=\bigcup_{i=1}^\infty [A_{\lambda_k}]_i$, with $[A_{\lambda_k}]_i\in\mathcal{M}_{\lambda_k}$ s.t. $\mu_{\lambda_k}([A_{\lambda_k}]_i)<\infty$ for each $i$.
		\\Then because the collection $\{X_\lambda\}_{\lambda\in\Lambda}$ is disjoint, then $[A_{\lambda_k}]_i\cap X_\lambda=([A_{\lambda_k}]_i\cap X_{\lambda_k})\cap X_\lambda=\emptyset$ for $\lambda\neq\lambda_k$.
		\\Also $[A_{\lambda_k}]_i=[A_{\lambda_k}]_i\cap X_{\lambda_k}\in\mathcal{M}_{\lambda_k}$ so that $[A_{\lambda_k}]_i\in\mathcal{B}$ and $\mu([A_{\lambda_k}]_i)$ is defined.
		\\Then we have for each $i$,
		\begin{align*}
			\mu([A_{\lambda_k}]_i)&=\sum_{\lambda\in\Lambda} \mu_\lambda([A_{\lambda_k}]_i\cap X_\lambda)\\
			&=\sum_{\lambda\neq\lambda_k} \mu_\lambda([A_{\lambda_k}]_i\cap X_\lambda)+\mu_{\lambda_k}([A_{\lambda_k}]_i\cap X_\lambda)\\
			&=\sum_{\lambda\neq\lambda_k} 0+\mu_{\lambda_k}([A_{\lambda_k}]_i)\\
			&=\mu_{\lambda_k}([A_{\lambda_k}]_i).
		\end{align*}
		Therefore $\mu_{\lambda_k}([A_{\lambda_k}]_i)=\mu([A_{\lambda_k}]_i)<\infty$.
		\\Then we can write
		\begin{align*}
			\mu(X)&=\sum_{\lambda\in\Lambda} \mu_\lambda(X_\lambda)\\		
			&=\sum_{\lambda\in\Lambda*} \mu_\lambda(X_\lambda)+\sum_{\lambda\notin\Lambda^*} \mu_\lambda(X_\lambda)\\	
			&=\sum_{\lambda\in\Lambda*} 0+\sum_{k=1}^\infty \mu_{\lambda_{k}}(X_{\lambda_{k}})\\
			&=\sum_{k=1}^\infty \mu_{\lambda_{k}}(\bigcup_{i=1}^\infty [A_{\lambda_k}]_i)\\
			&=\sum_{k=1}^\infty \sum_{i=1}^\infty \mu_{\lambda_{k}}( [A_{\lambda_k}]_i)\\
			&=\sum_{k=1}^\infty\sum_{i=1}^\infty \mu([A_{\lambda_k}]_i),
		\end{align*}
		and thus $X$ can be written as a countable disjoint union of measurable sets $[A_{\lambda_k}]_i$, each of which has finite measure under $\mu$.
		\\Therefore $\mu$ is $\sigma$-finite.
	\end{enumerate}
	\item Let $(X,\mathcal{M},\mu)$ be a measure space. The symmetric difference, $E_1\Delta E_2$, of two subsets $E_1$ and $E_2$ of $X$ is defined by
	\[
		E_1\Delta E_2=[E_1\setminus E_2]\cup[E_2\setminus E_1].
	\]
	\begin{enumerate}[label=(\roman*),align=left]   
		\item Show that if $E_1$ and $E_2$ are measurable and $\mu(E_1\Delta E_2)=0$, then $\mu(E_1)=\mu(E_2)$.\\
		\\We can see that
		\begin{align*}
			\mu(E_1\cup E_2)=\mu([E_1\Delta E_2]\cup[E_1\cap E_2])=\mu(E_1\Delta E_2)+\mu(E_1\cap E_2)=\mu(E_1\cap E_2).
		\end{align*}
		Then we also know that by monotonicity we have
		\begin{align*}
			E_1\cap E_2\subseteq E_1,E_2\subseteq E_1\cup E_2\implies\mu(E_1\cap E_2)\le\mu(E_1),\mu(E_2)\le\mu(E_1\cup E_2),
		\end{align*}
		and therefore $\mu(E_1)=\mu(E_2)$.
		\item Show that if $\mu$ is complete and $E_1\in\mathcal{M}$, then $E_2\in\mathcal{M}$ if $\mu(E_1\Delta E_2)=0$.\\
		\\Because $\mu(E_1\Delta E_2)=0$, then because $\mu$ is complete, the subsets $[E_1\setminus E_2]\subseteq E_1\Delta E_2$ and $[E_2\setminus E_1]\subseteq E_1\Delta E_2$ are measurable.
		Therefore the set $[E_2\setminus E_1]\cup[E_1]\cap[E_1\setminus E_2]^c$ is also measurable, and
		\begin{align*}
			[E_2\setminus E_1]\cup[E_1]\cap[E_1\setminus E_2]^c&=[E_2\cup E_1]\cap[E_1^c\cup E_1]\cap[E_1^c\cup E_2]\\
			&=[E_2\cup E_1]\cap[E_1^c\cup E_2]\\
			&=([E_2\cup E_1]\cap E_1^c)\cup ([E_2\cup E_1]\cap E_2)\\
			&=([E_2\cap E_1^c]\cup [E_1\cap E_1^c])\cup E_2\\
			&=(E_2\cap E_1^c)\cup E_2\\
			&=E_2,
		\end{align*}
		therefore $E_2=[E_2\setminus E_1]\cup[E_1]\cap[E_1\setminus E_2]^c$ is measurable.
	\end{enumerate}
	\item Let $(X,\mathcal{M},\mu)$ be a measure space and $X_0$ belong to $\mathcal{M}$.
	Define $\mathcal{M}_0$ to be the collection of sets in $\mathcal{M}$ that are subsets of $X_0$ and $\mu_0$ the restriction of $\mu$ to $\mathcal{M}_0$.
	Show that $(X_0,\mathcal{M}_0,\mu_0)$ is a measure space.\\
	\\We want to show that $(X_0,\mathcal{M}_0)$ is a measurable space (i.e., that $\mathcal{M}_0$ is a $\sigma$-algebra of subsets of $X_0$) and that $\mu_0$ is a measure on $\mathcal{M}_0$.
	\\To see that $\mathcal{M}_0$ is a $\sigma$-algebra:
	\begin{enumerate}[label=(\roman*),align=left]   
		\item $X_0\in\mathcal{M}_0$ because $X_0\in\mathcal{M}$ and $X_0\subseteq X_0$.
		\item if $A\in\mathcal{M}_0$, then $A\in\mathcal{M}$ and $A\subseteq X_0$.\\
		Then $X_0\cap A^c\in\mathcal{M}$ and $X_0\cap A^c\subseteq X_0$ imply that $X_0\cap A^c\in\mathcal{M}_0$.
		\item if $A_i\in\mathcal{M}_0$, then $A_i\in\mathcal{M}$ and $A_i\subseteq X_0$ for all $i$.\\
		Then $\bigcup_{i=1}^\infty A_i\in\mathcal{M}$ and $\bigcup_{i=1}^\infty A_i\subseteq X_0$ imply that $\bigcup_{i=1}^\infty A_i\in\mathcal{M}_0$.
	\end{enumerate}
	Therefore $(X_0,\mathcal{M}_0)$ is a measurable space.\\
	Clearly $\mu_0$ is a measure on $\mathcal{M}_0$, because it inherits the properties of a measure from $\mu$.\\
	Thus $(X_0,\mathcal{M}_0,\mu_0)$ is a measure space.
	\item Let $(X,\mathcal{M})$ be a measurable space. Verify the following:
	\begin{enumerate}[label=(\roman*),align=left]  
		\item If $\mu$ and $\nu$ are measures defined on $\mathcal{M}$, then set set function $\lambda$ defined on $\mathcal{M}$ by $\lambda(E)=\mu(E)+\nu(E)$ also is a measure. We denote $\lambda$ by $\mu+\nu$.\\
		\\Because $\mu(E)\ge 0$ and $\nu(E)\ge 0$ for any $E\in\mathcal{M}$, then $\lambda(E)=\mu(E)+\nu(E)\ge 0$.
		\\Also, $\mu(\emptyset)= 0$ and $\nu(\emptyset)= 0$ imply that $\lambda(\emptyset)=\mu(\emptyset)+\nu(\emptyset)= 0$.
		\\Finally, for any countable disjoint collection $\{E_k\}_{k=1}^\infty$ of measurable sets,
		\begin{align*}
			\lambda\left(\bigcup_{k=1}^\infty E_k\right)&=\mu\left(\bigcup_{k=1}^\infty E_k\right)+\nu\left(\bigcup_{k=1}^\infty E_k\right)\\
			&=\sum_{k=1}^\infty\mu(E_k)+\sum_{k=1}^\infty\nu(E_k)\\
			&=\sum_{k=1}^\infty[\mu(E_k)+\nu(E_k)]\\
			&=\sum_{k=1}^\infty\lambda(E_k).
		\end{align*}
		Therefore $\lambda$ is a measure.
		\item If $\mu$ and $\nu$ are measures on $\mathcal{M}$ and $\mu\ge\nu$, then there is a measure $\lambda$ on $\mathcal{M}$ for which $\mu=\nu+\lambda$.\\
		\\In the case $\mu(E)<\infty$, then we also have $\nu(E)\le\mu(E)<\infty$, and we can let $\lambda = \mu-\nu$.
		\\We clearly see that $\mu\ge\nu\implies\mu-\nu\ge0$ so that $\lambda(E)=\mu(E)-\nu(E)\ge 0$ for any $E\in\mathcal{M}$ (of finite measure under $\mu$).
		\\Also, $\mu(\emptyset)= 0$ and $\nu(\emptyset)= 0$ imply that $\lambda(\emptyset)=\mu(\emptyset)-\nu(\emptyset)= 0$.
		\\Finally, for any countable disjoint collection $\{E_k\}_{k=1}^\infty$ of measurable sets each of finite measure,
		\begin{align*}
			\lambda\left(\bigcup_{k=1}^\infty E_k\right)&=\mu\left(\bigcup_{k=1}^\infty E_k\right)-\nu\left(\bigcup_{k=1}^\infty E_k\right)\\
			&=\sum_{k=1}^\infty\mu(E_k)-\sum_{k=1}^\infty\nu(E_k)\\
			&=\sum_{k=1}^\infty[\mu(E_k)-\nu(E_k)]\\
			&=\sum_{k=1}^\infty\lambda(E_k).
		\end{align*}
		\\In the case $\mu(E)=\infty$, we can let $\lambda(E) = \infty$ so that $\nu(E)+\lambda(E)=\mu(E)$.
		\\Then $\lambda(E)=\infty\ge0$.
		\\For any countable disjoint collection $\{E_k\}_{k=1}^\infty$ of measurable sets, supposing there exists an index $j$ such that $\mu(E_j)=\infty$, then we defined $\lambda(E_j) = \infty$ so that by monotonicity, we have
		\begin{align*}
			\infty=\lambda(E_j)\le\lambda\left(\bigcup_{k=1}^\infty E_k\right),
		\end{align*}
		so $\lambda\left(\bigcup_{k=1}^\infty E_k\right)=\infty=\sum_{k=1}^\infty\lambda(E_k)$.
		\\Then we also have $\sum_{k=1}^\infty\mu(E_k)=\infty$ and 
		\begin{align*}
			\mu\left(\bigcup_{k=1}^\infty E_k\right)&=\mu\left(\bigcup_{k=1}^\infty E_k\right)+\lambda\left(\bigcup_{k=1}^\infty E_k\right)\\
			&=\mu\left(\bigcup_{k=1}^\infty E_k\right)+\infty\\
			&=\infty.
		\end{align*}
		In conclusion, we have defined
		\[
		\lambda(E)=
		\begin{cases}
			\mu(E)-\nu(E)&\text{if }\mu(E)<\infty\\\
			\infty&\text{if }\mu(E)=\infty,
		\end{cases}
		\]
		and we have proved that $\lambda$ is a measure.
		\item If $\nu$ is $\sigma$-finite, the measure $\lambda$ in (ii) is unique.\\
		\\Because $\nu$ is $\sigma$-finite, then $X$ is the union of a countable collection of measurable sets (may be taken to be disjoint), each of which has finite measure under $\nu$.
		That is, $X=\bigcup_{k=1}^\infty X_k$, where $\nu(X_k)<\infty$.
		Then for any $E\in\mathcal{M}$, we have 
		\[
			E=E\cap X = E\cap \bigcup_{k=1}^\infty X_k = \bigcup_{k=1}^\infty[E\cap X_k],
		\]
		where by monotonicity of measure we have $\nu(E\cap X_k)\le\nu(X_k)<\infty$, and thus any measurable set $E$ is also $\sigma$-finite when $\nu$ is $\sigma$-finite.
		\\Now, suppose there exist measures $\lambda_1$ and $\lambda_2$ such that $\mu=\nu+\lambda_1$ and $\mu=\nu+\lambda_2$.
		Then $\nu+\lambda_1=\nu+\lambda_2$ and thus $\nu-\nu=\lambda_2-\lambda_1$.
		\\For any $E\in\mathcal{M}$ such that $\nu(E)<\infty$, then clearly $\lambda_1(E)=\lambda_2(E)$.
		\\For any $E\in\mathcal{M}$ such that $\nu(E)=\infty$, $\nu(E)-\nu(E)=\infty-\infty$ is not defined. 
		\\However, because $\nu$ is $\sigma$-finite, there exists a countable disjoint collection $\{E_k\}_{k=1}^\infty$ such that $E=\bigcup_{k=1}^\infty E_k$ and $\nu(E_k)<\infty$ for each $k$. 
		Then we see that $\nu(E_k)-\nu(E_k)$ is defined for all $k$, and
		\[
			\nu(E)=\nu(\bigcup_{k=1}^\infty E_k)=\sum_{k=1}^\infty \nu(E_k)=\lim_{n\to\infty}\sum_{k=1}^n \nu(E_k).
		\] 
		Then we can write 
		\begin{align*}
			\lim_{n\to\infty}\sum_{k=1}^n \nu(E_k)-\lim_{n\to\infty}\sum_{k=1}^n \nu(E_k)&=\lambda_2(E)-\lambda_1(E)\\
			\lim_{n\to\infty}\sum_{k=1}^n [\nu(E_k)-\nu(E_k)]&=\lambda_2(E)-\lambda_1(E)\\
			\lim_{n\to\infty}\sum_{k=1}^n 0&=\lambda_2(E)-\lambda_1(E)\\
			0&=\lambda_2(E)-\lambda_1(E).
		\end{align*}
		Therefore $\lambda_1(E)=\lambda_2(E)$, and the measure $\lambda$ is unique.
		\item Show that in general the measure $\lambda$ need not be unique but that there is always a smallest such $\lambda$.\\
		\\Suppose there exists a set $E\in\mathcal{M}$ such that $\mu(E)=\infty$ and $\nu(E)=\infty$. 
		Then regardless of the number $\lambda(E)\in[0,\infty]$ we define $\lambda$ to be, we always have $\infty=\mu(E)=\nu(E)+\lambda(E)$.
		Then $\lambda(E)=0$ is the smallest value that we can set $\lambda$ to be, and we can define the smallest $\lambda$ in the following way:
		\[
		\lambda(E)=
		\begin{cases}
			\mu(E)-\nu(E)&\text{if }\mu(E)<\infty\ (\text{ forces }\nu(E)<\infty)\\
			\infty&\text{if }\mu(E)=\infty,\nu(E)<\infty\\
			0&\text{if }\mu(E)=\infty,\nu(E)=\infty
		\end{cases}
		\]
	\end{enumerate}
	\item Let $(X,\mathcal{M},\mu)$ be a measure space.
	The measure $\mu$ is said to be \textbf{semifinite} provided each measurable set of infinite measure contains measurable sets of arbitrarily large finite measure.
	\begin{enumerate}[label=(\roman*),align=left]  
		\item Show that each $\sigma$-finite measure is semifinite.\\
		\\If we suppose $\mu$ is $\sigma$-finite, then we can write any $E\in\mathcal{M}$ as the countable disjoint union of measurable sets of finite measure under $\mu$:
		$E=\bigcup_{k=1}^\infty E_k$ with $\mu(E_k)<\infty$.
		\\Consider any measurable set $E$ such that $\mu(E)=\infty$.
		Then 
		\[
			\mu(E)=\mu(\bigcup_{k=1}^\infty E_k)=\sum_{k=1}^\infty \mu(E_k)=\infty.
		\]
		Then the sequence of partial sums $\sum_{k=1}^n \mu(E_k)$ converges to infinity.
		That is, for any real number $x$, there exists an index $j$ such that $\sum_{k=1}^j \mu(E_k)>x$.
		Because each $E_k$ is disjoint and measurable, we have that $E_x:=\bigcup_{k=1}^j E_k\in\mathcal{M}$, and we can write
		\[
			x<\sum_{k=1}^j \mu(E_k)=\mu(\bigcup_{k=1}^j E_k)=\mu(E_x)<\infty.
		\]
		That is, for any real number we choose, there exists a measurable set $E_x\subseteq E$ of finite measure that is larger than $x$.
		\\Therefore $\mu$ is semifinite.
		\item For $E\in\mathcal{M}$, define $\mu_1(E)=\sup\{\mu(F)\ |\ F\subseteq E,\mu(F)<\infty\}$. 
		Show that $\mu_1$ is a semifinite measure: it is called the semifinite part of $\mu$.\\
		\\Consider any measurable set $E$ such that $\mu_1(E)=\infty$.
		Then for any subset $F$ of $E$ such that $\mu(F)<\infty$, we have that $\mu_1(E)\ge\mu(F)=\mu_1(F)$ by definition of supremum.
		(We have $\mu_1(F)=\mu(F)$ because $F$ is the largest subset of itself).
		However, because $\mu_1$ is the least upper bound, for any real number $x$, there exists a subset $F_x$ of $E$ such that $x<\mu_1(F_x)\le\mu_1(E)$, else we reach a contradiction to the supremum.
		Therefore for any real number $x$ we choose, there exists a measurable set $F_x\subseteq E$ of finite measure that is larger than $x$.  
		\item Find a measure $\mu_2$ on $\mathcal{M}$ that only takes the values $0$ and $\infty$ and $\mu=\mu_1+\mu_2$.\\
		\\We can define, for any $E\in\mathcal{M}$, 
		\[
			\mu_2(E)=
			\begin{cases}
				0&\text{if }\mu_1(E)<\infty\\
				\infty&\text{if }\mu_1(E)=\infty
			\end{cases}
		\]
		So that we have
		\[
			\mu(E)=
			\begin{cases}
				\mu_1(E)+\mu_2(E)=\mu(E)+0&\text{if }\mu_1(E)<\infty\\
				\mu_1(E)+\mu_2(E)=\mu(E)+\infty&\text{if }\mu_1(E)=\infty
			\end{cases}
		\]
	\end{enumerate}
	\item Prove Proposition 3; that is, show that $\mathcal{M}_0$ is a $\sigma$-algebra, $\mu_0$ is properly defined, and $(X,\mathcal{M}_0,\mu_0)$ is complete. In what sense is $\mathcal{M}_0$ minimal?\\
	\\We can see
	\begin{enumerate}[label=(\roman*),align=left]
		\item $X\in\mathcal{M}_0$ because $X\subseteq X$, and $X=\emptyset\cup X$ with $X\in\mathcal{M}$ and $\emptyset\subseteq\emptyset$ for $\emptyset\in\mathcal{M}$ where $\mu(\emptyset)=0$.
		\item If $E\in\mathcal{M}_0$, then $E\subseteq X$, and $E=A\cup B$ with $B\in\mathcal{M}$ and $A\subseteq C$ for $C\in\mathcal{M}$ where $\mu(C)=0$.\\
		Then $A\subseteq C\implies A^c\supseteq C^c$, and $A^c=[A^c\cap C]\cup[A^c\cap C^c]=[A^c\cap C]\cup C^c$.
		\\Now, $X\cap E^c\subseteq X$.
		\\We can write
		\begin{align*}
			E^c&=A^c\cap B^c\\
			&=([A^c\cap C]\cup C^c)\cap B^c\\
			&=([A^c\cap C]\cap B^c)\cup (C^c\cap B^c),
		\end{align*}
		Where $C^c\cap B^c\in\mathcal{M}$ and $[A^c\cap C]\cap B^c\subseteq C$ for $C\in\mathcal{M}$ where $\mu(C)=0$.
		\\Therefore $E^c\in\mathcal{M}_0$.
		\item If $E_k\in\mathcal{M}_0$, then $E_k\subseteq X$, and $E_k=A_k\cup B_k$ with $B_k\in\mathcal{M}$ and $A_k\subseteq C_k$ for $C_k\in\mathcal{M}$ where $\mu(C_k)=0$ for all $k$.\\
		Then $\bigcup_{k=1}^\infty E_k\subseteq X$, and 
		\[
			\bigcup_{k=1}^\infty E_k=\bigcup_{k=1}^\infty [A_k\cup B_k]=[\bigcup_{k=1}^\infty A_k]\cup[\bigcup_{k=1}^\infty B_k],
		\]
		Where $\bigcup_{k=1}^\infty B_k\in\mathcal{M}$ and $A_k\subseteq C_k\implies\bigcup_{k=1}^\infty A_k\subseteq \bigcup_{k=1}^\infty C_k$ for $\bigcup_{k=1}^\infty C_k\in\mathcal{M}$ with $\mu(\bigcup_{k=1}^\infty C_k)\le\sum_{k=1}^\infty\mu(C_k)=\sum_{k=1}^\infty0=0$.
		\\Thus $\bigcup_{k=1}^\infty E_k\in\mathcal{M}_0$.
	\end{enumerate}
	Thus $\mathcal{M}_0$ is a $\sigma$-algebra of subsets of $X$.\\
	\\To see that $\mu_0$ is a measure on the measurable space $(X,\mathcal{M}_0)$:
	\\For any $E\in\mathcal{M}_0$, we have $E=A\cup B,B\in\mathcal{M}$, so that $\mu_0(E)=\mu(B)\ge0$.
	\\Then for $\emptyset\in\mathcal{M}_0$, we have $\emptyset=\emptyset\cup \emptyset,\emptyset\in\mathcal{M}$, so that $\mu_0(\emptyset)=\mu(\emptyset)=0$.
	\\Finally, consider a disjoint collection $\{E_k\}_{k=1}^\infty$ of sets in $\mathcal{M}_0$.
	\\See (iii) to see that $\bigcup_{k=1}^\infty E_k=[\bigcup_{k=1}^\infty A_k]\cup[\bigcup_{k=1}^\infty B_k]$, where $\{E_k\}_{k=1}^\infty$ disjoint implies $\{B_k\}_{k=1}^\infty$ disjoint and we have $\bigcup_{k=1}^\infty B_k\in\mathcal{M}$.
	\\Then 
	\[
		\mu_0(\bigcup_{k=1}^\infty E_k)=\mu(\bigcup_{k=1}^\infty B_k)=\sum_{k=1}^\infty\mu(B_k)=\sum_{k=1}^\infty\mu_0(E_k).
	\]
	Therefore $(X,\mathcal{M}_0,\mu_0)$ is a measure space.\\
	\\To see that $(X,\mathcal{M}_0,\mu_0)$ is complete, consider any set $E\in\mathcal{M}_0$ such that $\mu_0(E)=0$.
	\[
		E\in\mathcal{M}_0\implies E\subseteq X,E=A\cup B, B\in\mathcal{M}\text{ and }A\subseteq C\text{ with }C\in\mathcal{M}, \mu(C)=0.
	\]
	Then $A\subseteq C\implies A\cup B\subseteq C\cup B$, and $C,B\in\mathcal{M}\implies C\cup B\in\mathcal{M}$.
	Thus $\mu(C\cup B)\le \mu(C)+\mu(B)=0$ is well-defined.
	\\Consider any $D\subseteq E$.
	\[
		D\subseteq E\subseteq X,D=D\cup \emptyset, \emptyset\in\mathcal{M}\text{ and }D\subseteq A\cup B\subseteq C\cup B\text{ with }C\cup B\in\mathcal{M}, \mu(C\cup B)=0.
	\]
	Therefore $D\in \mathcal{M}_0$ and $(X,\mathcal{M}_0,\mu_0)$ is complete.
	\item If $(X,\mathcal{M},\mu)$ is a measure space, we say that a subset $E$ of $X$ is \textbf{locally measurable} provided for each $B\in\mathcal{M}$ with $\mu(B)<\infty$, the intersection $E\cap B$ belongs to $\mathcal{M}$.
	The measure $\mu$ is called \textbf{saturated} provided every locally measurable set is measurable.
	\begin{enumerate}[label=(\roman*),align=left]  
		\item Show that each $\sigma$-finite measure is saturated.\\
		\\Suppose $\mu$ is $\sigma$-finite, then $X$ can be taken to be the union of a countable collection of measurable sets, each of which has finite measure under $\mu$.
		\\That is, $X=\bigcup_{k=1}^\infty X_k$, where $\mu(X_k)<\infty$.
		\\Then for any $E\in X$, we have 
		\[
			E=E\cap X = E\cap \bigcup_{k=1}^\infty X_k = \bigcup_{k=1}^\infty[E\cap X_k],
		\]
		In the case that $E$ is locally measurable, then each intersection $E\cap X_k$ is measurable.
		Then the countable intersection of measurable sets $\bigcup_{k=1}^\infty[E\cap X_k]=E$ is measurable.
		\\Thus when $\mu$ is $\sigma$-finite, every locally measurable set is measurable, and thus $\mu$ is saturated.
		\item Show that the collection $\mathcal{C}$ of locally measurable sets is a $\sigma$-algebra.\\
		\\We have
		\begin{enumerate}[label=(\roman*),align=left]   
			\item $X\in\mathcal{C}$ because for all $B\in\mathcal{M}$ with $\mu(B)<\infty$, then $X\cap B=B\in\mathcal{M}$.
			\item if $E\in\mathcal{C}$, then for all $B\in\mathcal{M}$ with $\mu(B)<\infty$, then $E\cap B\in\mathcal{M}$.\\
			Then we have the two measurable sets $E\cap B$ and $B$ so that $[E\cap B]^c\cap B$ is also measurable, and
			\[
				\mathcal{M}\ni[E\cap B]^c\cap B=[E^c\cup B^c]\cap B=[E^c\cap B]\cup[B^c\cap B]=E^c\cap B,
			\]
			and thus $E^c\in\mathcal{C}$. 
			\item if $E_i\in\mathcal{C}$, then for all $B\in\mathcal{M}$ with $\mu(B)<\infty$, then $E_i\cap B\in\mathcal{M}$ for all $i$.\\
			Then $\left[\bigcup_{i=1}^\infty E_i\right]\cap B=\bigcup_{i=1}^\infty [E_i\cap B]\in\mathcal{M}$ and thus $\bigcup_{i=1}^\infty E_i\in\mathcal{C}$.
		\end{enumerate}
		\item Let $(X,\mathcal{M},\mu)$ be a measure space and $\mathcal{C}$ the $\sigma$-algebra of locally measurable sets.
		For $E\in\mathcal{C}$, define $\overline\mu(E)=\mu(E)$ if $E\in\mathcal{M}$ and $\overline\mu(E)=\infty$ if $E\notin\mathcal{M}$.
		Show that $(X,\mathcal{C},\overline\mu)$ is a saturated measure space.\\
		\\In (ii) we showed that $\mathcal{C}$ is a $\sigma$-algebra of subsets of $X$.\\
		Therefore $(X,\mathcal{C})$ is a measurable space.\\
		\\We have defined
		\[
			\overline\mu(E)=
			\begin{cases}
				\mu(E)&\text{ if }E\in\mathcal{M}\\
				\infty&\text{ if }E\notin\mathcal{M}
			\end{cases}	
		\]
		\\We have $\overline\mu(E)\in\{\mu(E),\infty\}\ge0$ for all $E\in\mathcal{C}$.
		We have $\overline\mu(\emptyset)=\mu(\emptyset)=0$ because $\emptyset\in\mathcal{M}$ and $\emptyset\in\mathcal{C}$.
		Finally, consider a countable disjoint collection of sets $\{E_k\}_{k=1}^\infty$ in $\mathcal{C}$.
		\begin{enumerate}[label=(\roman*),align=left]  
			\item If for all $k$ we have $E_k\in\mathcal{M}$, then $\mu(\bigcup_{k=1}^\infty E_k)$ is measurable, $\overline\mu(E_k)=\mu(E_k)$, and
			\[
				\overline\mu(\bigcup_{k=1}^\infty E_k)=\mu(\bigcup_{k=1}^\infty E_k)=\sum_{k=1}^\infty \mu(E_k)=\sum_{k=1}^\infty \overline\mu(E_k).
			\]
			\item If there exists an index $j$ such that $E_j\notin\mathcal{M}$, then $\overline\mu(E_j)=\infty$ and $\sum_{k=1}^\infty \overline\mu(E_k)=\infty$.
		\end{enumerate}
		Consider the case that $\bigcup_{k=1}^\infty E_k\in\mathcal{C}$ is measurable and $\mu(\bigcup_{k=1}^\infty E_k)<\infty$. 
		\\Then because for any $j$, we have $E_j\in\mathcal{C}$, then $E_j=E_j\cap\bigcup_{k=1}^\infty E_k\in\mathcal{M}$.\\
		Then (i) must hold.\\
		\\Consider the case that $\bigcup_{k=1}^\infty E_k\in\mathcal{C}$ is measurable and $\mu(\bigcup_{k=1}^\infty E_k)=\infty$. \\
		If (i) holds, then $\overline\mu(\bigcup_{k=1}^\infty E_k)=\sum_{k=1}^\infty\overline\mu(E_k)$.\\
		If (ii) holds, then $\overline\mu(\bigcup_{k=1}^\infty E_k)=\mu(\bigcup_{k=1}^\infty E_k)=\infty=\sum_{k=1}^\infty \overline\mu(E_k)$.\\\bigskip
		Consider the case that $\bigcup_{k=1}^\infty E_k\in\mathcal{C}$ is not measurable. Then $\mu(\bigcup_{k=1}^\infty E_k)=\infty$. \\
		Then (ii) must hold else we reach a contradiction to $\bigcup_{k=1}^\infty E_k\notin\mathcal{M}$.\\\bigskip
		Therefore $(X,\mathcal{C},\overline\mu)$ is a measure space.\\
		\\We can use the definition of $\overline\mu$ to see that 
		\[
			B\in\mathcal{C}\text{ with }\overline\mu(B)<\infty\iff B\in\mathcal{M}\text{ with }\mu(B)<\infty.
		\]
		Consider a set $E\subseteq X$ such that $E\cap B\in\mathcal{C}$ for any such $B$.
		\\Then by monotonicity, $\overline\mu(E\cap B)\le \overline\mu(B)<\infty$.
		\\Because $\overline\mu(E\cap B)<\infty$, see the definition of $\overline\mu$ to see that $E\cap B\in\mathcal{M}$.
		\\Then we see that $E\in\mathcal{C}$ because for each $B\in\mathcal{M}$ with $\mu(B)<\infty$, we have $E\cap B\in\mathcal{M}$.\\\bigskip
		Therefore $(X,\mathcal{C},\overline\mu)$ is a saturated measure space.
		\item If $\mu$ is semifinite and $E\in\mathcal{C}$, the set $\underline\mu(E)=\sup\{\mu(B)\ |\ B\in\mathcal{M},B\subseteq E\}$.
		Show that $(X,\mathcal{C},\underline\mu)$ is a saturated measure space and that $\underline\mu$ is an extension of $\mu$.
		Give an example to show that $\overline\mu$ and $\underline\mu$ may be different.\\
		\\We first want to show that $\underline\mu$ is a measure on the measurable space $(X,\mathcal{C})$:
		\\For any $E\in\mathcal{C}$, we have $\underline\mu(E)\ge\mu(B)\ge0$ for $B\in\mathcal{M},B\subseteq E$.
		\\For $\emptyset\in\mathcal{C}$, we have $\underline\mu(\emptyset)=\mu(\emptyset)=0$ because $\{\emptyset\}=\{B\in\mathcal{M}\ |\ B\subseteq\emptyset\}$.
		\\Finally, for any disjoint collection $\{E_k\}_{k=1}^\infty$ in $\mathcal{C}$,\\
		\\Therefore $(X,\mathcal{C},\underline\mu)$ is a measure space.\\
		\\Consider any $E\subseteq X$ such that $E\cap B\in\mathcal{C}$ whenever $B\in\mathcal{C}$ with $\underline\mu(B)<\infty$.
		\\Then $E\cap B\in\mathcal{C}$ implies that $[E\cap B]\cap B'\in\mathcal{M}$ whenever $B'\in\mathcal{M}$ with $\mu(B')<\infty$.
	\end{enumerate}
	\item Let $\mu$ and $\eta$ be measures on the measurable space $(X,\mathcal{M})$.
	For $E\in\mathcal{M}$, define $\nu(E)=\max\{\mu(E),\eta(E)\}$. Is $\nu$ a measure on $(X,\mathcal{M})$?\\
	\\We have $0\le\mu(E),\eta(E)\le\max\{\mu(E),\eta(E)\}$ for any $E\in\mathcal{M}$.
	\\We have $\max\{\mu(E),\eta(E)\}\in[0,\infty]$ for any $E\in\mathcal{M}$.
	\\Counterexample:
	Let $E_1,E_2$ be nonempty disjoint measurable (singleton) sets such that
	\[
		\mu(E)=
		\begin{cases}
			1&E\supseteq E_1\\
			0&E\not\supseteq E_1\\
		\end{cases}
		\ \text{ and }\ \eta(E)=
		\begin{cases}
			1&E\supseteq E_2\\
			0&E\not\supseteq E_2\\
		\end{cases}
	\]
	Then $\mu(E)\in\{0,1\}\ge0$, $\mu(\emptyset)=0$ because $\emptyset\not\supseteq E_1$, and for any countable disjoint collection $\{A_k\}_{k=1}^\infty$ of measurable sets,
	in the case that for all $k$, $A_k\not\supseteq E_1$, then $\bigcup_{k=1}^\infty A_k\not\supseteq E_1$ and 
	\[
		\mu\biggl(\bigcup_{k=1}^\infty A_k\biggr)=\sum_{k=1}^\infty\mu(A_k)=0.
	\]
	In the case that there exists an index $j$ such that $A_j\supseteq E_1$, then $\bigcup_{k=1}^\infty A_k\supseteq E_1$, and because the sets are disjoint, $A_i\not\supseteq E_1$ for $i\neq j$.
	\[
		\mu\biggl(\bigcup_{k=1}^\infty A_k\biggr)=\sum_{k=1}^\infty\mu(A_k)=\sum_{k\in\mathbb{N}\setminus\{j\}}\mu(A_k)+\mu(A_j)=0+\mu(A_j)=1.
	\]
	Then $\eta$ can also be shown to be a measure in the exact same way.
	\\Then we see that
	\begin{align*}
		\nu(E_1\cup E_2)&=\max\{\mu(E_1\cup E_2),\eta(E_1\cup E_2)\}=\max\{1,1\}=1,\\
		\nu(E_1)+\nu(E_2)&=\max\{\mu(E_1),\eta(E_1)\}+\max\{\mu(E_2),\eta(E_2)\}=\max\{1,0\}+\max\{0,1\}=2.
	\end{align*}
	Thus $\nu$ is not a measure because it does not satisfy countable additivity.
\end{enumerate}

% 17.2
\authoredby{finished}
\section{Signed Measures: The Hahn and Jordan Decompositions}
\begin{flushleft}
	\begin{namedthm*}{Definition}
		By a \textbf{signed measure} $\nu$ on the measurable space $(X,\mathcal{M})$ we mean an extended real-valued set function $\nu:\mathcal{M}\to[-\infty,\infty]$ that possesses the following properties:
		\begin{enumerate}[label=(\roman*),align=left]  
			\item $\nu$ assumes at most one of the values $+\infty,-\infty$.
			\item $\nu(\emptyset)=0$.
			\item For any countable collection $\{E_k\}_{k=1}^\infty$ of disjoint measurable sets,
			\[
				\nu\left(\bigcup_{k=1}^\infty E_k\right)=\sum_{k=1}^\infty\nu(E_k),
			\]
			where the series $\sum_{k=1}^\infty\nu(E_k)$ converges absolutely if $\nu(\bigcup_{k=1}^\infty E_k)$ is finite (convergence must hold for any rearrangement).
		\end{enumerate}
	\end{namedthm*}
	A set $A$ is \textbf{positive} (with respect to $\nu$) provided that $A$ is measurable and for every measurable subset $E$ of $A$ we have $\nu(E)\ge0$.
	The restriction of $\nu$ to the measurable subsets of a positive set is a measure.
	(See Problem 6).
	Similarly, a set $B$ is called \textbf{negative} (with respect to $\nu$) provided that $B$ is measurable and for every measurable subset $E$ of $B$ we have $\nu(E)\le0$.
	The restriction of $-\nu$ to the measurable subsets of a negative set is a measure.
	A measurable set is called \textbf{null} with respect to $\nu$ provided every measurable subset of it also has measure zero.
	(Clearly a null set is both positive and negative.)
	\\Monotonicity for signed measures:
	\[
	A\subseteq B\text{ and }|\nu(B)|<\infty\implies|\nu(A)|<\infty.
	\]
	It is not possible for a signed measure to take on both $\pm\infty$ at the same time.
		\\To see this, suppose there exist two subsets $E_1,E_2$ of $X$ such that $\nu(E_1)=\infty$ and $\nu(E_2)=-\infty$.
		\\If $-\infty<\nu(E_1\cap E_2)<\infty$, then 
		\begin{align*}
			\nu(E_1)&=\nu(E_1\cap E_2)+\nu(E_1\setminus E_2)=\infty\text{ so that }\nu(E_1\setminus E_2)=\infty,\\
			\nu(E_2)&=\nu(E_2\cap E_1)+\nu(E_2\setminus E_1)=-\infty\text{ so that }\nu(E_2\setminus E_1)=-\infty,
		\end{align*}
		and $E_1\setminus E_2$ and $E_2\setminus E_1$ are disjoint but $\infty-\infty$ is not defined.
		\\If $\nu(E_1\cap E_2)=\infty$, then 
		\begin{align*}
			-\infty=(E_2)=\nu(E_2\cap E_1)+\nu(E_2\setminus E_1)=\infty+\nu(E_2\setminus E_1),
		\end{align*}
		and we cannot find an $E_2\setminus E_1$ that satisfies this because $-\infty+\infty$ is not defined.
		\\If $\nu(E_1\cap E_2)=-\infty$, then 
		\begin{align*}
			\infty=(E_1)=\nu(E_1\cap E_2)+\nu(E_1\setminus E_2)=-\infty+\nu(E_1\setminus E_2),
		\end{align*}
		and we cannot find an $E_1\setminus E_2$ that satisfies this because $\infty-\infty$ is not defined.
	\begin{namedthm*}{The Hahn Decomposition Theorem}
		Let $\nu$ be a signed measure on the measurable space $(X,\mathcal{M})$.
		Then there is a positive set $A$ for $\nu$ and a negative set $B$ for $\nu$ for which
		\[
			X=A\cup B\text{ and }A\cap B=\emptyset.	
		\]
	\end{namedthm*}
	\begin{proof}
		Without loss of generality assume $+\infty$ is the infinite value omitted by $\nu$.
		(Otherwise let $A$ be the negative set and $B$ be the positive set. We need this to use Hahn's Lemma: $0<\nu(E)<\infty$.)
		\\Let $\mathcal{P}$ be the collection of positive subsets of $X$.
		Define $\lambda=\sup\{\nu(E)\ |\ E\in\mathcal{P}\}$, and $\lambda\ge0$ and it is nonempty because $\emptyset\in\mathcal{P}$.
		Then by definition of supremum, for each natural number $k$, there exists an element $A_k\in\lambda$ such that
		\[
			\lambda-\frac{1}{k}<\nu(A_k)\le\lambda.
		\]
		Then $\{A_k\}_{k=1}^\infty$ is a countable collection of positive subsets of $X$ for which $\lambda=\lim_{k\to\infty}\nu(A_k)$.
		We can let $A=\cup_{k=1}^\infty A_k$, and by Proposition 4, the countable union of a positive collection of positive sets is positive, and thus $A$ is a positive subset as well; then $\nu(A)\le\lambda$.
		Then for each $k$, we have $A\cap A_k^c\subseteq A$, and because $A$ is positive, all of its subsets including $A\setminus A_k$ is positive, so $\nu(A\setminus A_k)\ge0$.
		Then by countable disjoint additivity,
		\[
			\nu(A)=\nu(A_k)+\nu(A\setminus A_k)\ge\nu(A_k).
		\]
		Therefore $\nu(A)\ge\lim_{k\to\infty}\nu(A_k)=\lambda$ and $\nu(A)\le\lambda$ implies $\nu(A)=\lambda$.
		Also, $\lambda<\infty$ because $\nu$ does not take on the value $\infty$.
		\\Let $B=X\setminus A$.
		Supposing by contradiction that $B$ is not negative, then there exists a subset $E$ of $B$ of nonnegative measure: $\nu(E)\not\le0\text{ and }\nu(E)\neq\infty\implies 0<\nu(E)<\infty$, and therefore, by Hahn's Lemma, there exists a measurable subset $E_0$ of $E$ that is both positive and of positive measure.
		Then $A\cup E_0$ is a positive set and 
		\[
			\nu(A\cup E_0)=\nu(A)+\nu(E_0)>\lambda,
		\]
		a contradiction to the choice of $\lambda$ as the supremum.
	\end{proof}
	If $\{A,B\}$ is a Hahn decomposition for $\nu$, then we define two measures $\nu^+$ and $\nu^-$ (the positive and negative variations of $\nu$) with $\nu=\nu^+-\nu^-$ by setting
	\[
		\nu^+(E)=\nu(E\cap A)\text{ and }\nu^-(E)=-\nu(E\cap B).	
	\]
	(The disjoint sets $E\cap A$ and $E\cap B$ have positive and negative measure, respectively, because $A$ is positive (and thus all subsets have positive measure), and $B$ is negative (and thus all subsets have negative measure).) 
	\\\bigskip Two measures $\nu_1$ and $\nu_2$ on $(X,\mathcal{M})$ are said to be \textbf{mutually singular} ($\nu_1\perp\nu_2$) if there are disjoint measurable sets $A,B$ with $X=A\cup B$ for which $\nu_1(A)=\nu_2(B)=0$.
	\begin{namedthm*}{The Jordan Decomposition Theorem}
		Let $\nu$ be a signed measure on the measurable space $(X,\mathcal{M})$. 
		Then there are two mutually singular measures $\nu^+$ and $\nu^-$ on $(X,\mathcal{M})$ for which $\nu=\nu^+-\nu^-$.
		Moreover, there is only one such pair of mutually singular measures.
	\end{namedthm*}
	\begin{proof}
		Existence:
		\\By the Hahn Decomposition Theorem, there exists a positive set $A$ and a negative set $B$ for which $X=A\cup B$ and $A\cap B=\emptyset$.
		Then we can define $\nu^+$ and $\nu^-$ on $(X,\mathcal{M})$ such that
		\begin{align*}
			\nu^+(E)&=\nu(A\cap E)\text{ for all }E\in\mathcal{M}\\
			\nu^-(E)&=-\nu(B\cap E)\text{ for all }E\in\mathcal{M}
		\end{align*}
		Clearly  $\nu^+$ and $\nu^-$ are measures because 
			\begin{align*}
				\nu^+(E)&\ge0&&\text{because }A\cap E\subseteq A,\text{ and }A\text{ is positive, so for any }C\subseteq A,\text{ then }\nu(C)\ge0\\
				\nu^-(E)&\ge0&&\text{because }B\cap E\subseteq B,\text{ and }B\text{ is negative, so for any }C\subseteq B,\text{ then }\nu(C)\le0\implies-\nu(C)\ge0
			\end{align*}
			\begin{align*}
				\nu^+(\emptyset)&=\nu(A\cap\emptyset)=\nu(\emptyset)=0\\
				\nu^-(\emptyset)&=-\nu(B\cap\emptyset)=-\nu(\emptyset)=0
			\end{align*}
			For disjoint measurable collection $\{E_k\}_{k=1}^\infty$,
			\begin{align*}
				\nu^+(\bigcup_{k=1}^\infty E_k)&=\nu(A\cap\bigcup_{k=1}^\infty E_k)=\nu(\bigcup_{k=1}^\infty [A\cap E_k])=\sum_{k=1}^\infty\nu(A\cap E_k)=\sum_{k=1}^\infty\nu^+(E_k)\\
				\nu^-(\bigcup_{k=1}^\infty E_k)&=-\nu(B\cap\bigcup_{k=1}^\infty E_k)=-\nu(\bigcup_{k=1}^\infty [B\cap E_k])=\sum_{k=1}^\infty-\nu(B\cap E_k)=\sum_{k=1}^\infty\nu^-(E_k)
			\end{align*}
		The measures $\nu^+$ and $\nu^-$ are mutually singular because $X=A\cup B$, $A\cap B=\emptyset$, and 
		\begin{align*}
			\nu^+(B)&=\nu(A\cap B)=\nu(\emptyset)=0\\
			\nu^-(A)&=-\nu(B\cap A)=-\nu(\emptyset)=0
		\end{align*}
		\\Then for any measurable set $E$, we have $E=[E\cap A]\cup[E\cap B]$ so that 
		\[
			\nu(E)=\nu([E\cap A]\cup[E\cap B])=\nu(E\cap A)+\nu(E\cap B)=\nu^+(E)-\nu^-(E).
		\]
	\end{proof}
	We define $|\nu|$ on $\mathcal{M}$ by
	\[
		|\nu|(E)=\nu^+(E)+\nu^-(E)\text{ for all }E\in\mathcal{M}.
	\]
	In Problem 16, prove that 
	\begin{equation*}
		|\nu|(X)=\sup\sum_{k=1}^n|\nu(E_k)|,\tag{4}
	\end{equation*}
	where the supremum is taken over all finite disjoint collections $\{E_k\}_{k=1}^n$ of measurable subsets of $X$.
	For this reason $|\nu|(X)$ is called the \textbf{total variation} of $\nu$ and denoted by $\|\nu\|_{var}$.
	\\\bigskip
	\textbf{Example}
	Let $f:\mathbb{R}\to\mathbb{R}$ be a function that is Lebesgue integrable over $\mathbb{R}$.
	For a Lebesgue measurable set $E$, define $\nu(E)=\int_E f dm$.
	\\Then $\nu$ is a signed measure on the measurable space $(\mathbb{R},\mathcal{L})$:
	\begin{enumerate}[(i)]
		\item $f$ is integrable means that we have $\int_\mathbb{R}|f|<\infty$.
		Then because $|f\cdot\chi_E|\le|f|$ on $\mathbb{R}$, by the integral comparison test, $f\cdot\chi_E$ is integrable over $\mathbb{R}$.
		That is,
		\[
			|\nu(E)|=\left|\int_Efdm\right|=\left|\int_\mathbb{R}f\cdot\chi_Edm\right|<\infty.
		\]
		\item $\nu(\emptyset)=\int_\emptyset f dm=0$ (See Chapter 4 Problem 17).
		\item Let $E=\{E_k\}_{k=1}^\infty$ be a countable collection of disjoint measurable sets.
		Then by Chapter 4 Theorem 20,
		\[
			\nu\left(\bigcup_{k=1}^\infty E_k\right)=\nu(E)=\int_{E}fdm=\sum_{k=1}^\infty\int_{E_k}fdm=\sum_{k=1}^\infty\nu(E_k).
		\]
	\end{enumerate}
	Thus $\nu$ is a signed measure.
	\\Define 
	\begin{align*}
		A&:=\{x\in\mathbb{R}\mid f(x)\ge0\}\\
		B&:=\{x\in\mathbb{R}\mid f(x)<0\}
	\end{align*}
	and define for each Lebesgue measurable set $E$,
	\begin{align*}
		\nu^+(E)&:=\nu(A\cap E)=\int_{A\cap E} f dm\\
		\nu^-(E)&:=-\nu(B\cap E)=-\int_{B\cap E} f dm
	\end{align*}
	Then $\{A,B\}$ is a Hahn decomposition of $\mathbb{R}$ w.r.t. the signed measure $\nu$.
	Moreover, $\nu=\nu^+-\nu^-$ is a Jordan decomposition of $\nu$.
\end{flushleft}
\begin{center}
	\textbf{PROBLEMS}
\end{center}
\begin{enumerate}
	\setcounter{enumi}{11}
	\item In the above example, let $E$ be a Lebesgue measurable set such that $0<\nu(E)<\infty$.
	Find a positive set $A$ contained in $E$ for which $\nu(A)>0$.\\
	\\Consider the set $A'=A\cap E$, which is a positive set because it is a measurable subset of the positive set $A$.
	\\Then
	\[
		\nu(A')=\nu(A\cap E)=\int_{A\cap A'} f dm-\int_{B\cap A'} f dm=\int_{A\cap E} f dm\ge0.
	\]
	If we suppose that $\nu(A')=0$, then we have
	\[
		\nu(E)=\int_{A\cap E} f dm-\int_{B\cap E} f dm=-\int_{B\cap E} f dm\le0,
	\]
	which is a contradiction to $\nu(E)>0$.
	\\Therefore $\nu(A')>0$.
	\item Let $\mu$ be a measure and $\mu_1$ and $\mu_2$ be mutually singular measures on a measurable space $(X,\mathcal{M})$ for which $\mu=\mu_1-\mu_2$.
	Show that $\mu_2=0$.
	Use this to establish the uniqueness assertion of the Jordan Decomposition Theorem.\\
	\\Because $\mu_1$ and $\mu_2$ are mutually singular, then there exist disjoint measurable sets $A,B$ with $X=A\cup B$ for which $\mu_1(A)=\mu_2(B)=0$.
	\\Consider $E\in\mathcal{M}$.
	\\In the case $E\subseteq B$, we have $\mu_2(E)=0$ by monotonicity of measure.
	\\In the case $E\subseteq A$, we have $\mu_1(E)=0$, so that
	\[
		\mu(E)=\mu_1(E)-\mu_2(E)=-\mu_2(E)\le0,
	\]
	where $\mu(E)\ge0$ implies $\mu_2(E)=0$.
	\\Therefore for any measurable set $E$,
	\[
		\mu_2(E)=\mu_2(E\cap A)+\mu_2(E\cap B)=0+0=0.
	\]
	\\Suppose $\nu$ is a signed measure on the measurable space $(X,\mathcal{M})$, and suppose we have any two pairs of mutually singular measures $(\nu_1^+,\nu_1^-)$ and $(\nu_2^+,\nu_2^-)$ so that $\nu=\nu_1^+-\nu_1^-=\nu_2^+-\nu_2^-$.
	\\By definition of mutually singular, there exist disjoint pairs $(A_1,B_1)$ and $(A_2,B_2)$ with $X=A_1\cup B_1=A_2\cup B_2$ for which $\nu_1^+(A_1)=\nu_1^-(B_1)=0$ and $\nu_2^+(A_2)=\nu_2^-(B_2)=0$.
	\\Let $E\in\mathcal{M}$.
	\\In the case $E\subseteq B_1$,
	\[
		\nu_1^+(E)=\nu_2^+(E)-\nu_2^-(E).
	\]
	Restricting $\nu_1^+$ to all measurable subsets of $B_1$ (see Problem 6), we have that $\nu_2^-=0$ so that $\nu_1^+=\nu_2^+$.
	\\In the case $E\subseteq A_1$,
	\[
		\nu_1^-(E)=-\nu_2^+(E)+\nu_2^-(E).
	\]
	Restricting $\nu_1^-$ to all measurable subsets of $A_1$, we have that $\nu_2^+=0$ so that $\nu_1^-=\nu_2^-$.\\
	\\Therefore for any measurable set $E$,
	\[
		\nu_1^+(E)=\nu_1^+(E\cap B_1)=\nu_2^+(E\cap B_1)=\nu_2^+(E),
	\]
	and Similarly,
	\[
		\nu_1^-(E)=\nu_1^-(E\cap A_1)=\nu_2^-(E\cap A_1)=\nu_2^-(E).
	\]
	\item Show that if $E$ is any measurable set, then
	\[
		-\nu^-(E)\le\nu(E)\le\nu^+(E)\text{ and }|\nu(E)|\le|\nu|(E).	
	\]
	\\We have the Jordan decomposition
	\[
		\nu(E)=\nu^+(E)-\nu^-(E),
	\]
	where $\nu^+$ and $\nu^-$ are measures so that they are nonnegative; therefore
	\[
		0\le\nu^-(E)\implies \nu(E)\le\nu^+(E),
	\]
	and
	\[
		\nu^+(E)\ge0\implies \nu(E)\ge-\nu^-(E).
	\]
	By the triangle inequality,
	\[
		|\nu(E)|=|\nu^+(E)-\nu^-(E)|\le\nu^+(E)+\nu^-(E)=|\nu|(E).
	\]
	\item Show that if $\nu_1$ and $\nu_2$ are any two finite signed measures, then so is $\alpha\nu_1+\beta\nu_2$, where $\alpha$ and $\beta$ are real numbers. Show that
	\[
		|\alpha\nu|=|\alpha||\nu|\text{ and }|\nu_1+\nu_2|\le|\nu_1|+|\nu_2|,
	\]
	where $\nu\le\mu$ means $\nu(E)\le\mu(E)$ for all measurable sets $E$.\\
	\\Let $E$ be any measurable set.
	\begin{enumerate}[(i)]
		\item $|\alpha\nu_1(E)+\beta\nu_2(E)|\le|\alpha||\nu_1(E)|+|\beta||\nu_2(E)|<|\alpha|\cdot\infty+|\beta|\cdot\infty=\infty$
		\item $\alpha\nu_1(\emptyset)+\beta\nu_2(\emptyset)=\alpha\cdot0+\beta\cdot0=0$
		\item Let $\{E_k\}_{k=1}^\infty$ be a countable collection of disjoint measurable sets.
		\[
			\alpha\nu_1(\bigcup_{k=1}^\infty E_k)+\beta\nu_2(\bigcup_{k=1}^\infty E_k)=\alpha\sum_{k=1}^\infty\nu_1(E_k)+\beta\sum_{k=1}^\infty\nu_2(E_k)=\sum_{k=1}^\infty\left[\alpha\nu_1(E_k)+\beta\nu_2(E_k)\right].
		\]
	\end{enumerate}
	Therefore $\alpha\nu_1+\beta\nu_2$ is itself a finite signed measure.
	\\Homogeneity and subadditivity of the absolute value come from the properties of the real numbers. 
	\item Prove (4).\\
	\\Let $\{E_k\}_{k=1}^n$ be a finite, measurable partition of $X$.
	\\Recall the definition $|\nu|=\nu^++\nu^-$.
	\\From problem 14, we have $|\nu(E_k)|\le|\nu|(E_k)$ for each $k$.
	\\Then 
	\begin{align*}
		\sum_{k=1}^n|\nu(E_k)|
		&\le\sum_{k=1}^n|\nu|(E_k)\\
		&=\sum_{k=1}^n\nu^+(E_k)+\sum_{k=1}^n\nu^-(E_k)\\
		&=\nu^+(\bigcup_{k=1}^n E_k)+\nu^-(\bigcup_{k=1}^n E_k)\\
		&=\nu^+(X)+\nu^-(X)\\
		&=|\nu|(X),
	\end{align*}
	so that
	\[
		\sup\sum_{k=1}^n|\nu(E_k)|\le|\nu|(X).\tag{a}
	\]
	\\Because $\nu^+$ and $\nu^-$ are mutually singular measures, consider the disjoint sets $E_1,E_2$ such that $X=E_1\cup E_2$ and $\nu^+(E_1)=\nu^-(E_2)=0$.
	\\Therefore $\nu(E_1)=\nu^+(E_1)-\nu^-(E_1)=-\nu^-(E_1)$ and $\nu(E_2)=\nu^+(E_2)-\nu^-(E_2)=\nu^+(E_2)$ so that 
	\begin{align*}
		|\nu|(X)&=|\nu|(E_1\cup E_2)\\
		&=\nu^+(E_1\cup E_2)+\nu^-(E_1\cup E_2)\\
		%&=\nu^+(E_1)+\nu^+(E_2)+\nu^-(E_1)+\nu^-(E_2)&&\text{disjoint additivity}\\
		&=\nu^-(E_1)+\nu^+(E_2)\\
		&=|\nu(E_1)|+|\nu(E_2)|\\
		&=\sum_{k=1}^2|\nu(E_k)|\\
		&\le\sup\sum_{k=1}^n|\nu(E_k)|.\tag{b}
	\end{align*}
	Then (a) and (b) imply equality:
	\[
		|\nu|(X)=\sup\sum_{k=1}^n|\nu(E_k)|.
	\]
	\item Let $\mu$ and $\nu$ be finite signed measures.
	Define $\mu\land\nu=\frac{1}{2}(\mu+\nu-|\mu-\nu|)$ and $\mu\lor\nu=\mu+\nu-\mu\land\nu$.
	\begin{enumerate}[label=(\roman*),align=left]  
		\item Show that the signed measure $\mu\land\nu$ is smaller than $\mu$ and $\nu$ but larger than any other signed measure that is smaller than $\mu$ and $\nu$.
		\item Show that the signed measure $\mu\lor\nu$ is larger than $\mu$ and $\nu$ but smaller than any other signed measure that is larger than $\mu$ and $\nu$.
		\item If $\mu$ and $\nu$ are positive measures, show that they are mutually singular iff $\mu\land\nu=0$.\\
	\end{enumerate}
	We have the identities
	\begin{align*}
		\max(\mu,\nu)+\min(\mu,\nu)&=\mu+\nu\\
		\max(\mu,\nu)-\min(\mu,\nu)&=|\mu-\nu|\\
		\max(\mu,\nu)&=\frac{1}{2}(\mu+\nu+|\mu-\nu|)\\
		\min(\mu,\nu)&=\frac{1}{2}(\mu+\nu-|\mu-\nu|)
	\end{align*}
	\begin{enumerate}[label=(\roman*),align=left]  
		\item $\mu(E)\land\nu(E)=\min(\mu(E),\nu(E))\le\mu(E),\nu(E)$.
		\\If $\lambda(E)\le\mu(E),\nu(E)$, then $\lambda(E)\le\min\{\mu(E),\nu(E)\}=\mu(E)\land\nu(E)$.
		\item We can see
		\[
			\mu\lor\nu=\frac{1}{2}(2\mu+2\nu)-\frac{1}{2}(\mu+\nu-|\mu-\nu|)=\frac{1}{2}(\mu+\nu+|\mu-\nu|),
		\]
		So that $\mu(E)\lor\nu(E)=\max(\mu(E),\nu(E))\ge\mu(E),\nu(E)$.
		\\If $\lambda(E)\ge\mu(E),\nu(E)$, then $\lambda(E)\ge\max\{\mu(E),\nu(E)\}=\mu(E)\lor\nu(E)$.
		\item Let $\mu$ and $\nu$ be positive measures.\\
		\\$(\implies)$ Suppose that $\mu$ and $\nu$ are mutually singular.
		\\Then there exist disjoint measurable sets $A,B$ with $X=A\cup B$ s.t. $\mu(A)=\nu(B)=0$.
		\\Let $E$ be a measurable set.
		\\In the case $E\subseteq A$,
		\[
			(\mu\land\nu)(E)=\frac{1}{2}(0+\nu(E)-|0-\nu(E)|)=0.
		\]
		In the case $E\subseteq B$,
		\[
			(\mu\land\nu)(E)=\frac{1}{2}(\mu(E)+0-|\mu(E)-0|)=0.
		\]
		Therefore, for any measurable set $E$,
		\[
			(\mu\land\nu)(E)=(\mu\land\nu)(E\cap A)+(\mu\land\nu)(E\cap B)=0+0=0.
		\]
		\\$(\impliedby)$ Suppose that $\mu\land\nu=0$.
		\\This implies that for any measurable set $E$, at least one of $\mu(E)$ and $\nu(E)$ must equal zero. 
		\\Consider the finite signed measure $\lambda=\mu-\nu$. 
		% By the Jordan Decomposition Theorem, there exists a pair of mutually singular measures $\lambda^+$ and $\lambda^-$ for which $\lambda=\lambda^+-\lambda^-$.
		% By definition of mutually singular, there exist disjoint measurable sets $P,N$ with $X=P\cup N$ s.t. $\lambda^+(N)=\lambda^-(P)=0$.
		\\By the Hahn Decomposition Theorem, there is a positive set $P$ for $\lambda$ and a negative set $N$ for $\lambda$ for which
		\[
			X=P\cup N\text{ and }P\cap N=\emptyset.	
		\]
		%By the Jordan Decomposition Theorem, we have $\lambda=\lambda^+-\lambda^-$.
		Let $E$ be a measurable set.
		\\In the case $E\subseteq P$,
		\[
			\mu(E)-\nu(E)\ge0.
		\]
		\begin{itemize}
			\item If $\mu(E)-\nu(E)>0$, then $\mu(E)>0$ and $\nu(E)=0$.
			\item If $\mu(E)-\nu(E)=0$, then $\mu(E)=\nu(E)=0$.
		\end{itemize}
		In the case $E\subseteq N$,
		\[
			\mu(E)-\nu(E)\le0.
		\]
		\begin{itemize}
			\item If $\mu(E)-\nu(E)<0$, then $\mu(E)=0$ and $-\nu(E)<0$.
			\item If $\mu(E)-\nu(E)=0$, then $\mu(E)=\nu(E)=0$.
		\end{itemize}
		Then $\mu(N)=\nu(P)=0$ so that $\mu$ and $\nu$ are mutually singular.
	\end{enumerate}
\end{enumerate}

% 17.3
\authoredby{finished}
\section{The Cath\'eodory Measure Induced by an Outer Measure}
\begin{namedthm*}{Definition}
	A set function $\mu^*:2^X\to[0,\infty]$ is called an \textbf{outer measure} provided $\mu^*(\emptyset)=0$ and $\mu^*$ is countably monotone in the sense that whenever a set $E\in2^X$ is covered by a countable collection $\{E_k\}_{k=1}^\infty$ of sets in $2^X$, then
	\[
		\mu^*(E)\le\sum_{k=1}^\infty\mu^*(E_k).
	\]
\end{namedthm*}
Clearly an outer measure is finitely monotone, which can be seen by setting $E_k=\emptyset$ for all $k>n$.
\begin{namedthm*}{Definition}
	For an outer measure $\mu^*:2^X\to[0,\infty]$, we call a subset $E$ of $X$ \textbf{measurable} (with respect to $\mu^*$) provided for every subset $A$ of $X$,
	\[
		\mu^*(A)=\mu^*(A\cap E)+\mu^*(A\cap E^c).
	\]
\end{namedthm*}
Since $\mu^*$ is finitely monotone so that $\mu^*(A)\le\mu^*(A\cap E)+\mu^*(A\cap E^c)$, to show that $E\subseteq X$ is measurable, it is only necessary to prove that
\[
	\mu^*(A)\ge\mu^*(A\cap E)+\mu^*(A\cap E^c)\text{ for all }A\subseteq X\text{ such that }\mu^*(A)<\infty.
\]
\begin{namedthm*}{Theorem 8}
	Let $\mu^*$ be an outer measure on $2^X$.
	Then the collection $\mathcal{M}$ of sets that are measurable with respect to $\mu^*$ is a $\sigma$-algebra.
	If $\overline{\mu}$ is the restriction of $\mu^*$ to $\mathcal{M}$, then $(X,\mathcal{M},\overline{\mu})$ is a complete measure space.
\end{namedthm*}

% no problems

% 17.4
\authoredby{finished}
\section{The Construction of Outer Measures}
\begin{namedthm*}{Theorem 9}
	Let $\mathcal{S}$ be a collection of subsets of a set $X$ and $\mu:\mathcal{S}\to[0,\infty]$ a set function.
	Define $\mu^*(\emptyset)=0$ and for $E\subseteq X$, $E\neq\emptyset$, define
	\[
		\mu^*(E)=\inf\sum_{k=1}^\infty\mu(E_k),
	\]
	where the infimum is taken over all countable collections $\{E_k\}_{k=1}^\infty$ of sets in $\mathcal{S}$ that cover $E$.
	Then the set function $\mu^*:2^X\to[0,\infty]$ is an outer measure called the \textbf{outer measure induced by} $\mu$.
\end{namedthm*}
(If a subset $E$ of $X$ cannot be covered by a countable subcollection of $\mathcal{S}$, then it has outer measure equal to $\infty$.)
\begin{proof}
	We have $\mu^*(\emptyset)=0$ by definition so it remains to show countable monotonicity.
	\\Consider any set $E\subseteq X$ that is covered by a countable collection $\{E_k\}_{k=1}^\infty$ of sets in $X$. 
	% \[
	% 	\mu^*(E)\le\sum_{k=1}^\infty\mu^*(E_k.)
	% \]
	\\We suppose that $\mu^*(E_k)<\infty$ for each $k$, otherwise the result is trivial.
	\\Fix $\epsilon>0$.
	\\For each $k$, by definition of infimum, there exists a countable collection $\{E_{ik}\}_{i=1}^\infty$ of sets in $\mathcal{S}$ that covers $E_k$ and
	\[
		\mu^*(E_k)\le\sum_{i=1}^\infty\mu(E_{ik})<\mu^*(E_k)+\frac{\epsilon}{2^k}.\tag{1}
	\]
	Then $\{E_{ik}\}_{1\le k,i<\infty}$ is a countable collection of sets in $S$ that covers $\bigcup_{k=1}^\infty E_k$ and therefore covers $E$.
	Therefore because $\mu^*$ is defined as the infimum of all such collections,
	\begin{align*}
		\mu^*(E)&\le\sum_{1\le k,i<\infty}\mu(E_{ik})\\
		&=\sum_{k=1}^\infty\sum_{i=1}^\infty\mu(E_{ik})\\
		&<\sum_{k=1}^\infty\mu^*(E_k)+\sum_{k=1}^\infty\frac{\epsilon}{2^k}&&\text{by (1)}\\
		&=\sum_{k=1}^\infty\mu^*(E_k)+\epsilon,
	\end{align*}
	and since this holds for all $\epsilon>0$, then $\mu^*(E)\le\sum_{k=1}^\infty\mu^*(E_k)$.
\end{proof}
\begin{namedthm*}{Definition}
	Let $\mathcal{S}$ be a collection of subsets of $X$, let $\mu:\mathcal{S}\to[0,\infty]$ be a set function, and let $\mu^*$ be the outer measure induced by $\mu$.
	The measure $\overline{\mu}$ that is the restriction of $\mu^*$ to the $\sigma$-algebra $\mathcal{M}$ of $\mu^*$-measurable sets is called the \textbf{Carath\'eodory measure induced by} $\mu$.
\end{namedthm*}
\[
	\underset{\text{a general set function}}{\mu:\mathcal{S}\to[0,\infty]}\longrightarrow
	\underset{\text{the induced outer measure}}{\mu^*:2^X\to[0,\infty]}\longrightarrow
	\underset{\text{the induced Carath\'eodory measure}}{\overline{\mu}:\mathcal{M}\to[0,\infty]}
\]
For a collection $\mathcal{S}$ of subsets of $X$, we use $\mathcal{S}_\sigma$ to denote those sets that are countable unions of sets of $\mathcal{S}$ and use $\mathcal{S}_{\sigma\delta}$ to denote those sets that are countable intersections of sets in $\mathcal{S}_\sigma$.
Observe that if $\mathcal{S}$ is the collection of open intervals of real numbers, the $\mathcal{S}_\sigma$ is the collection of open subsets of $\mathbb{R}$ (Chapter 1 Proposition 9: Every nonempty open sets is the union of a countable, disjoint collection of open intervals), and $\mathcal{S}_{\sigma\delta}$ is the collection of $G_\delta$ subsets of $\mathbb{R}$.
\begin{namedthm*}{Proposition 10}
	Let $\mu:\mathcal{S}\to[0,\infty]$ be a set function defined on a collection $\mathcal{S}$ of subsets of a set $X$ and $\overline{\mu}:\mathcal{M}\to[0,\infty]$ be the Carath\'eodory measure induced by $\mu$.
	Let $E$ be a subset of $X$ for which $\mu^*(E)<\infty$.
	Then there is a subset $A$ of $X$ for which 
	\[
		A\in\mathcal{S}_{\sigma\delta},\ E\subseteq A\text{ and }\mu^*(E)=\mu^*(A).
	\]
	Furthermore, if $E$ and each set in $\mathcal{S}$ is measurable with respect to $\mu^*$, then so is $A$ and 
	\[
		\overline{\mu}(A\setminus E)=0.
	\]
\end{namedthm*}
(See Chapter 2 Theorem 11 (ii))
\begin{center}
	\textbf{PROBLEMS}
\end{center}
\begin{enumerate}
	\setcounter{enumi}{17}
	\item Let $\mu^*:2^X\to[0,\infty]$ be an outer measure.
	Let $A\subseteq X, \{E_k\}_{k=1}^\infty$ be a disjoint countable collection of measurable sets and $E=\bigcup_{k=1}^\infty E_k$. Show that
	\[
		\mu^*(A\cap E)=\sum_{k=1}^\infty \mu^*(A\cap E_k).
	\]
	\\By definition, outer measure is countably monotone so that
	\[
		\mu^*(A\cap E)=\mu^*\left(A\cap \left[\bigcup_{k=1}^\infty E_k\right]\right)=\mu^*\left(\bigcup_{k=1}^\infty [A\cap E_k]\right)\le\sum_{k=1}^\infty \mu^*(A\cap E_k).
	\]
	By monotonicity of outer measure and Propositon 6 (finite additivity of outer measure),
	\[
		\mu^*(A\cap E)=\mu^*\left(A\cap \left[\bigcup_{k=1}^\infty E_k\right]\right)\ge\mu^*\left(A\cap \left[\bigcup_{k=1}^n E_k\right]\right)=\sum_{k=1}^n \mu^*(A\cap E_k).
	\]
	The left-hand side of the inequality is independent of $n$ so that $\mu^*(A\cap E)\ge\sum_{k=1}^\infty \mu^*(A\cap E_k)$.
	\item Show that any measure that is induced by an outer measure is complete.\\
	% \\Recall that a measure space $(X,\mathcal{M},\mu)$ is said to be \textbf{complete} provided $\mathcal{M}$ contains all subsets of sets of measure zero, that is, if $E$ belongs to $\mathcal{M}$ and $\mu(E)=0$, then every subset of $E$ also belongs to $\mathcal{M}$.
	\\Let $X$ be a set, let $\mu^*$ be an outer measure on $2^X$, and let $\mathcal{M}$ be the $\sigma$-algebra of sets that are measurable w.r.t. $\mu^*$.
	Then consider the measure $\overline{\mu}$ that is the restriction of $\mu^*$ to $\mathcal{M}$.
	We aim to show that $(X,\mathcal{M},\overline{\mu})$ is complete; that is, $\mathcal{M}$ contains all subsets of sets of measure zero (under $\overline{\mu}$).\\
	\\Let $E\in\mathcal{M}$ such that $\mu^*(E)=\overline{\mu}(E)=0$, and let $A\subseteq X$ be any set.
	\\Consider any subset $E'$ of $E$.
	\\By monotonicity of outer measure for $A\cap E'\subseteq E'\subseteq E$,
	\[
		\mu^*(A\cap E')\le\mu^*(E')\le\mu^*(E)=0,
	\]
	and by monotonicity of outer measure for $A\supseteq A\cap E'^c$,
	\[
		\mu^*(A)\ge\mu^*(A\cap E'^c)=\mu^*(A\cap E')+\mu^*(A\cap E'^c).
	\]
	Thus $E'$ is measurable.
	\item Let $X$ be any set.
	Define $\eta:2^X\to[0,\infty]$ by defining $\eta(\emptyset)=0$ and for $E\subseteq X,E\neq\emptyset$, defining $\eta(E)=\infty$.
	Show that $\eta$ is an outer measure.
	Also show that the set function that assigns $0$ to every subset of $X$ is an outer measure.\\
	\\We have $\eta(\emptyset)=0$ by definition so it remains to show countable monotonicity.
	\\Let $E\subseteq X$ and let $\{E_k\}_{k=1}^\infty$ be a countable collection of sets that cover $E$.
	\\If $E=\emptyset$, then the inequality is trivial, so assume $E\neq\emptyset$.
	\\Then there exists a $k$ such that $E_k\neq\emptyset$, which implies $\eta(E_k)=\infty$ so that we have 
	\[
		\eta(E)=\infty\le\infty=\sum_{k=1}^\infty\eta(E_k)
	\]
	\\The set function $\mu$ that assigns $0$ to every subset of $X$ is an outer measure trivially because $\mu(\emptyset)=0$ and
	\[
		\mu(E)=0\le0=\sum_{k=1}^\infty\mu(E_k)
	\]
	for any set $E$ that is covered by a countable collection $\{E_k\}_{k=1}^\infty$.
	\item Let $X$ be a set, $\mathcal{S}=\{\emptyset,X\}$, and define $\mu(\emptyset)=0,\mu(X)=1$. 
	Determine the outer measure $\mu^*$ induced by the set function $\mu:\mathcal{S}\to[0,\infty)$ and the $\sigma$-algebra of measurable sets.\\
	\\Recall that for any $E\subseteq X$, $E\neq\emptyset$, we have the definition of induced outer measure:
	\[
		\mu^*(E)=\inf\left\{\sum_{k=1}^\infty\mu(E_k)\mid E_k\in\mathcal{S},\ \bigcup_{k=1}^\infty E_k\supseteq E\right\}.
	\]
	For $E\neq\emptyset$, we have that $\mathcal{S}=\{\emptyset,X\}$ implies $\mu(E_k)\in\{0,1\}$ implies $\sum_{k=1}^\infty E_k\in\{1,2,\dots\}$, so that
	\[
		\mu^*(E)=1.
	\]
	($\sum_{k=1}^\infty E_k\neq0$ because otherwise we have a contradiction: $\emptyset=\bigcup_{k=1}^\infty E_k\supseteq E\neq\emptyset$)
	\\Therefore for any $E\subseteq X$, the induced outer measure $\mu^*$ is defined by
	\[
		\mu^*(E)=
		\begin{cases}
			0&E=\emptyset\\
			1&E\neq\emptyset
		\end{cases}
	\]
	We have defined that a subset $E$ of $X$ is measurable provided for every subset $A$ of $X$,
	\[
		\mu^*(A)=\mu^*(A\cap E)+\mu^*(A\cap E^c).
	\]
	Consider $A=X$, so that we have
	\[
		1=\mu^*(X)=\mu^*(E)+\mu^*(E^c),
	\]
	and the only sets $E$ that satisfy this are $E=\emptyset$ and $E=X$.
	\\Therefore the $\sigma$-algebra $\mathcal{M}$ of measurable sets (w.r.t. $\mu^*$) is simply
	\[
		\mathcal{M}=\{\emptyset,X\}.
	\]
	\item On the collection $\mathcal{S}=\{\emptyset,[1,2]\}$ of subsets of $\mathbb{R}$, define the set function $\mu:\mathcal{S}\to[0,\infty)$ as follows: $\mu(\emptyset)=0,\mu([1,2])=1$. 
	Determine the outer measure $\mu^*$ induced by $\mu$ and the $\sigma$-algebra of measurable sets.\\
	\\Similarly to the previous Problem 21, we deduce that the induced outer measure $\mu^*$ will be defined by
	\[
		\mu^*(E)=
		\begin{cases}
			0&E=\emptyset\\
			1&E\subseteq[1,2]\\
			\infty&\text{else}\\
		\end{cases}
	\]
	Now let $A$ be any set.
	\\If $A=\emptyset$, then trivially $\mu^*(A)=0\ge0=\mu^*(A\cap E)+\mu^*(A\cap E^c)$.
	\\If $A\not\subseteq[1,2]$, then trivially $\mu^*(A)=\infty\ge\mu^*(A\cap E)+\mu^*(A\cap E^c)$.
	\\Therefore consider the case where $A\subseteq[1,2]$, so that we have
	\[
		1=\mu^*([1,2])=\mu^*([1,2]\cap E)+\mu^*([1,2]\cap E^c).
	\]
	However, because both $[1,2]\cap E$ and $[1,2]\cap E^c$ are subsets of $[1,2]$, then the sets $E$ that satisfy the inequality are such that $E=\emptyset$ or $E\supseteq[1,2]$.
	\\Therefore the $\sigma$-algebra $\mathcal{M}$ of measurable sets (w.r.t. $\mu^*$) is
	\[
		\mathcal{M}=\{E\subseteq X\mid E=\emptyset\text{ or }E\supseteq[1,2]\}.
	\]
	\item On the collection $\mathcal{S}$ of all subsets of $\mathbb{R}$, define the set function $\mu:\mathcal{S}\to\mathbb{R}$ by setting $\mu(A)$ to be the number of integers in $A$.
	Determine the outer measure $\mu^*$ induced by $\mu$ and the $\sigma$-algebra of measurable sets.\\
	\\We can determine the induced outer measure to be
	\[
		\mu^*(E)=\mu(E)\text{ for any }E\subseteq X,
	\]
	and the $\sigma$-algebra of measurable sets to be
	\[
		\mathcal{M}=2^\mathbb{R}.
	\]
	\item Let $\mathcal{S}$ be a collection of subsets of $X$ and $\mu:\mathcal{S}\to[0,\infty]$ a set function.
	Is every set in $\mathcal{S}$ measurable with respect to the outer measure induced by $\mu$?\\
	\\Let $\mathcal{S}=2^{[1,2]}$, the set of all subsets of $[1,2]$.
	\\Define $\mu:\mathcal{S}\to[0,\infty)$ as follows: $\mu(\emptyset):=0$, $\mu([1,2]):=3$, and $\mu(E):=2,E\in\mathcal{S}\setminus\{\emptyset,[1,2]\}$. 
	\\Then we can define the extension $\mu^*$ to $2^{\mathbb{R}}$ to be
	\[
		\mu^*(E)=
		\begin{cases}
			0&E=\emptyset\\
			2&E\subset[1,2]\ \text{(strict subset, nonempty)}\\
			3&E=[1,2]\\
			\infty&\text{else}\\
		\end{cases}
	\]
	We can see that countable monotonicity holds:
	\[
		\mu^*(E)\le\sum_{k=1}^\infty\mu^*(E_k)\quad\text{for any }E\subset\mathbb{R}
	\]
	and so $\mu^*$ is an outer measure.
	\\However, the original set function $\mu$ is countably monotone but it is not finitely additive:
	\[
		\mu([1,2])=3\neq 4=\mu([1,1.5])+\mu((1.5,2])
	\]
	To see the effect on the outer measure, consider $E=[1,1.5]\in\mathcal{S}$.
	\\Then $[1,2]\cap E=[1,1.5]\subset[1,2]$ and $[1,2]\cap E^c=(1.5,2]\subset[1,2]$ so that for $A=[1,2]$,
	\[
		\mu^*([1,2])=3\neq2+2=\mu^*([1,2]\cap E)+\mu^*([1,2]\cap E^c),
	\]
	which implies that $E$ is not measurable.
	That is, because $\mu$ is not finitely additive, then every set in $\mathcal{S}$ is not measurable w.r.t the induced outer measure $\mu^*$.
	Therefore $\overline{\mu}$ is not an extension of $\mu$.
\end{enumerate}

% 17.5
\authoredby{inprogress}
\section{The Cath\'eodory-Hahn Theorem: The Extension of a Premeasure to a Measure}
Let $\mu:\mathcal{S}\to[0,\infty]$ be a set function that is defined on a nonempty collection $\mathcal{S}$ of subsets of a set $X$.
We ask the following question:
What properties must the collection $\mathcal{S}$ and set function $\mu$ possess in order that the Carath\'eodory measure $\overline{\mu}$ induced by $\mu$ be an extension of $\mu$:
that is; every set $E$ in $\mathcal{S}$ is measurable w.r.t. the outer measure $\mu^*$ induced by $\mu$ and, moreover, $\mu(E)=\mu^*(E)$?

In other words: see Problem 24 to see that not every set in $\mathcal{S}$ is measurable w.r.t. the outer measure induced by $\mu$ in general.
Therefore we want to know what additional properties of $\mathcal{S}$ and $\mu$ we must have to guarantee that $\mathcal{S}\subseteq\mathcal{M}$, so that $\overline{\mu}|_\mathcal{S}=\mu$. 
\begin{namedthm*}{Proposition 11}
	Let $\mathcal{S}$ be a collection of subsets of a set $X$ and let $\mu:\mathcal{S}\to[0,\infty]$ be a set function.
	In order that the Carath\'eodory measure induced by $\mu$ be an extension of $\mu$ it is necessary that $\mu$ be both finitely additive and countably monotone and, if $\emptyset$ belongs to $\mathcal{S}$, that $\mu(\emptyset)=0$.
\end{namedthm*}
\begin{namedthm*}{Definition}
	Let $\mathcal{S}$ be a collection of subsets of a set $X$ and $\mu:\mathcal{S}\to[0,\infty]$ be a set function.
	Then $\mu$ is called a \textbf{premeasure} provided $\mu$ is both finitely additive and countably monotone and, if $\emptyset$ belongs to $\mathcal{S}$, then $\mu(\emptyset)=0$.
\end{namedthm*}
Being a premeasure is a necessary but not sufficient condition for the Carath\'eodory measure induced by $\mu$ to be an extension of $\mu$. 
(See problems 25, 26).
However,  if we impose on $\mathcal{S}$ finer set-theoretic structure, this necessary condition is also sufficient.
\begin{namedthm*}{Definition}
	A collection $\mathcal{S}$ of subsets of $X$ is said to be closed with respect to the formulation of relative complements provided whenever $A$ and $B$ belong to $\mathcal{S}$, the relative complement $A\sim B$ belongs to $\mathcal{S}$.
	The collection $\mathcal{S}$ is said to be closed with respect to the formulation of finite intersections provided whenever $A$ and $B$ belong to $\mathcal{S}$, the intersection $A\cap B$ belongs to $\mathcal{S}$.
\end{namedthm*}
\begin{center}
	\textbf{PROBLEMS}
\end{center}
\begin{enumerate}
	\setcounter{enumi}{24}
	\item Let $X$ be any set containing more than one point and $A$ a proper nonempty subset of $X$.
	Define $\mathcal{S}=\{A,X\}$ and the set function $\mu:\mathcal{S}\to[0,\infty]$ by $\mu(A)=1$ and $\mu(X)=2$.
	Show that $\mu:\mathcal{S}\to[0,\infty]$ is a premeasure.
	Can $\mu$ be extended to a measure?
	What are the subsets of $X$ that are measurable with respect to the outer measure $\mu^*$ induced by $\mu$?
	\item Consider the collection $\mathcal{S}=\{\emptyset,[0,1],[0,3],[2,3]\}$ of subsets of $\mathbb{R}$ and define $\mu(\emptyset)=0,\mu([0,1])=1,\mu([0,3])=1,\mu([2,3])=1$.
	Show that $\mu:\mathcal{S}\to[0,\infty]$ is a premeasure.
	Can $\mu$ be extended to a measure?
	What are the subsets of $\mathbb{R}$ that are measurable with respect to the outer measure $\mu^*$ induced by $\mu$?
	\item Let $\mathcal{S}$ be a collection of subsets of a set $X$ and $\mu:\mathcal{S}\to[0,\infty]$ a set function.
	Show that $\mu$ is countably monotone iff $\mu^*$ is an extension of $\mu$.
	\item Show that a set function is a premeasure if it has an extension that is a measure.
	\item Show that a set function on a $\sigma$-algebra is a measure iff it is a premeasure.
	\item Let $\mathcal{S}$ be a collection of sets that is closed with respect to the formation of finite unions and finite intersections.
	\begin{enumerate}[label=(\roman*),align=left]  
		\item Show that $\mathcal{S}_\sigma$ is closed with respect to the formation of countable unions and finite intersections.
		\item Show that each set in $\mathcal{S}_{\sigma\delta}$ is the intersection of a decreasing sequence of $\mathcal{S}_\sigma$ sets.
	\end{enumerate}
	\item Let $\mathcal{S}$ be a semialgebra of subsets of a set $X$ and $\mathcal{S}'$ the collection of unions of finite disjoint collections of sets in $\mathcal{S}$.
	\begin{enumerate}[label=(\roman*),align=left]  
		\item Show that $\mathcal{S}'$ is an algebra.
		\item Show that $\mathcal{S}_\sigma=\mathcal{S}_\sigma'$ and therefore $\mathcal{S}_{\sigma\delta}=\mathcal{S}_{\sigma\delta}'$.
		\item Let $\{E_k\}_{k=1}^\infty$ be a collection of sets in $\mathcal{S}'$. Show that we can express 
		\[
			\sum_{k=1}^\infty \mu'(E_k')\ge\sum_{k=1}^\infty \mu(E_k).
		\]
		\item Let $A$ belong to $\mathcal{S}_{\sigma\delta}'$. Show that $A$ is the intersection of a descending sequence $\{A_k\}_{k=1}^\infty$ of sets in $\mathcal{S}_\sigma$.
	\end{enumerate}
	\item Let $\mathbb{Q}$ be the set of rational numbers and and $\mathcal{S}$ the collection of all finite unions of intervals of the form $(a,b]\cap\mathbb{Q}$, where $a,b\in\mathbb{Q}$ and $a\le b$. 
	Define $\mu((a,b])=\infty$ if $a<b$ and $\mu(\emptyset)=0$.
	Show that $\mathcal{S}$ is closed with respect to the formation of relative complements and $\mu:\mathcal{S}\to[0,\infty]$ is a premeasure.
	Then show that the extension of $\mu$ to the smallest $\sigma$-algebra containing $\mathcal{S}$ is not unique.
	\item By a bounded interval of real numbers we mean a set of the form $[a,b],[a,b),(a,b]$, or $(a,b)$ for real numbers $a\le b$.
	Thus we consider the empty-set and a set consisting of a single point to be a bounded interval.
	Show that each of the following three collections of sets $\mathcal{S}$ is a semiring.
	\begin{enumerate}[label=(\roman*),align=left]  
		\item Let $\mathcal{S}$ be the collection of all bounded intervals of real numbers.
		\item Let $\mathcal{S}$ be the collection of all subsets of $\mathbb{R}\times\mathbb{R}$ that are products of bounded intervals of real numbers.
		\item Let $n$ be a natural number an $X$ be the $n$-fold Cartesian product of $\mathbb{R}$:
		\[
			X=\mathbb{R}\times\cdots\times\mathbb{R}=\mathbb{R}^n.
		\]
		Let $\mathcal{S}$ be the collection of all subsets of $X$ that are $n$-fold Cartesian products of bounded intervals of real numbers.
	\end{enumerate}
	\item If we start with an outer measure $\mu^*$ on $2^X$ and form the induced measure $\overline\mu$ on the $\mu^*$-measurable sets, we can view $\overline\mu$ as a set function and denote by $\mu^+$ the outer measure induced by $\overline\mu$.
	\begin{enumerate}[label=(\roman*),align=left]  
		\item Show that for each set $E\subset X$ we have $\mu^+(E)\ge\mu^*(E)$.
		\item For a given set $E$, show that $\mu^+(E)=\mu^*(E)$ iff there is a $\mu^*$-measurable set $A\supseteq E$ with $\mu^*(A)=\mu^*(E)$.
	\end{enumerate}
	\item Let $\mathcal{S}$ be a $\sigma$-algebra of subsets of $X$ and $\mu:\mathcal{S}\to[0,\infty]$ a measure.
	Let $\overline\mu:\mathcal{M}\to[0,\infty]$ be the measure induced by $\mu$ via the Carath\'eodory construction.
	Show that $\mathcal{S}$ is a subcollection of $\mathcal{M}$ and it may be a proper subcollection.
	\item Let $\mu$ be a finite premeasure on an algebra $\mathcal{S}$, and $\mu^*$ the induced outer measure.
	Show that a subset $E$ of $X$ is $\mu^*$-measurable iff for each $\epsilon>0$ there is a set $A\in\mathcal{S}_\delta,A\subseteq E$, such that $\mu^*(E\setminus A)<\epsilon$.
\end{enumerate}

% Chapter 18
\authoredby{inprogress}
\chapter{Integration Over General Measure Spaces}

% 18.1
\authoredby{inprogress}
\section{Measurable Functions}
Consider the measurable space $(X,\mathcal{M})$. 
For an extended real valued function $f$ of $X$ and a measurable subset $E$ of $X$, the restriction of $f$ to both $E$ and $X\setminus E$ are measurable iff $f$ is measurable on $X$.
\begin{namedthm*}{Proposition 3}
    Let $(X,\mathcal{M},\mu)$ be a complete measure space and $X_0$ be a measurable subset of $X$ for which $\mu(X\setminus X_0)=0$.
    Then an extended real valued function $f$ on $X$ is measurable iff its restriction to $X_0$ is measurable.
    In particular, if $g$ and $h$ are extended real valued functions on $X$ for which $g=h$ a.e. on $X$, then $g$ is measurable iff $h$ is measurable.
\end{namedthm*}
\begin{namedthm*}{Theorem 6}
    Let $(X,\mathcal{M},\mu)$ be a measure space and $\{f_n\}$ a sequence of measurable functions on $X$ for which $\{f_n\}\to f$ pointwise a.e. on $X$.
    If either the measure space $(X,\mathcal{M},\mu)$ is complete or the convergence is pointwise on all of $X$, then $f$ is measurable.
\end{namedthm*}
\begin{namedthm*}{Corollary 7}
    Let $(X,\mathcal{M},\mu)$ be a measure space and $\{f_n\}$ be a sequence of measurable function on $X$.
    Then the following functions are measurable:
    \[
        \sup\{f_n\},\ \inf\{f_n\},\ \limsup\{f_n\},\ \liminf\{f_n\}. 
    \]
\end{namedthm*}
\begin{namedthm*}{Egoroff's Theorem}
    Let $(X,\mathcal{M},\mu)$ be a finite measure space and $\{f_n\}$ a sequence of measurable functions on $X$ that converges pointwise a.e. on $X$ to a function $f$ that is finite a.e. on $X$.
    Then for each $\epsilon>0$, there is a measurable subset $X_\epsilon$ of $X$ for which 
    \[
        \{f_n\}\to f\text{ uniformly on }X_\epsilon\text{ and }\mu(X\setminus  X_\epsilon)<\epsilon.
    \]
\end{namedthm*}
\begin{center}
	\textbf{PROBLEMS}
\end{center}
In the following problems $(X,\mathcal{M},\mu)$ is a reference measure space and measurable means with respect to $\mathcal{M}$.
\begin{enumerate}
	\setcounter{enumi}{0}
    \item Show that an extended real valued function on $X$ is measurable iff $f^{-1}\{\infty\}$ and $f^{-1}\{-\infty\}$ are measurable and so is $f^{-1}(E)$ for every Borel set of real numbers.
    \item Suppose $(X,\mathcal{M},\mu)$ is not complete.
    Let $E$ be a subset of a set of measure zero that does not belong to $\mathcal{M}$.
    Let $f=0$ on $X$ and $g=\chi_E$.
    Show that $f=g$ a.e. on $X$ while $f$ is measurable and $g$ is not.
    \item Suppose $(X,\mathcal{M},\mu)$ is not complete.
    Show that there is a sequence $\{f_n\}$ of measurable functions on $X$ that converges pointwise a.e. on $X$ to a function $f$ that is not measurable.
    \item Let $E$ be a measurable subset of $X$ and $f$ an extended real-valued function on $X$.
    Show that $f$ is measurable iff its restrictions to $E$ and $X\setminus E$ are measurable.
    \item Show that an extended real valued function $f$ on $X$ is measurable iff for each rational number $c$, $\{x\in X\mid f(x)<c\}$ is a measurable set.\\
    \\Let $f$ be an extended real valued function on $X$.\\
    \\$(\implies)$ Suppose that $f$ is measurable.
    \\Then trivially for any rational number $c$, $\{x\in X\mid f(x)<c\}$ is a measurable set by definition of measurable function.\\
    \\$(\impliedby)$ Suppose that for each rational number $c$, $\{x\in X\mid f(x)<c\}$ is a measurable set.
    \\Let $a$ be any real number.
    \\Then for each natural number $n$, by density of the rationals in the reals there exists a rational $c_n$ such that 
    \[
        a-\frac{1}{n}<c_n< a_n,
    \]
    and we have that the set $\{x\in X\mid f(x)<c_n\}$ is measurable.
    \\Then we have $\bigcup_{n=1}^\infty[\infty,c_n)=[\infty,a)$, so that we have the set
    \begin{align*}
        \bigcup_{n=1}^\infty\{x\in X\mid f(x)<c_n\}
        &=\bigcup_{n=1}^\infty\{x\in X\mid f(x)\in[\infty,c_n)\}\\
        &=\{x\in X\mid f(x)\in[\infty,a)\},\\
        &=\{x\in X\mid f(x)<a\},
    \end{align*}
    which is measurable because it is the countable union of measurable sets.
    \\Therefore $f$ is a measurable function.
    \item Consider two extended real valued measurable functions $f$ and $g$ on $X$ that are finite a.e. on $X$.
    Define $X_0$ to be the set of points in $X$ at which both $f$ and $g$ are finite.
    Show that $X_0$ is measurable and $\mu(X\setminus X_0)=0$.
    \item Let $X$ be a nonempty set.
    Show that every extended real valued function on $X$ is measurable w.r.t. the measurable space $(X,2^X)$.
    \begin{enumerate}[(i)]
        \item Let $x_0$ belong to $X$ and $\delta_{x_0}$ be the Dirac measure at $x_0$ on $2^X$.
        Show that two function on $X$ are equal a.e. $[\delta_{x_0}]$ iff they take the same value at $x_0$.
        \item Let $\eta$ be the counting measure on $2^X$.
        Show that two functions on $X$ are equal a.e. $[\eta]$ iff they take the same value at every point in $X$.
    \end{enumerate}
    \item Let $X$ be a topological space and $\mathcal{B}(X)$ be the smallest $\sigma$-algebra containing the topology on $X$.
    $\mathcal{B}(X)$ is called the Borel $\sigma$-algebra associated with the topological space $X$.
    Show that any continuous real valued function on $X$ is measurable w.r.t. the Borel measurable space $(X,\mathcal{B}(X))$.
    \item If a real valued function on $\mathbb{R}$ is measurable w.r.t. the $\sigma$-algebra of Lebesgue measurable sets, is it necessarily measurable w.r.t. the Borel measurable space $(\mathbb{R},\mathcal{B}(\mathbb{R}))$?
    \item Check that the proofs of Proposition 1 and Theorem 4 follow from the proofs of the corresponding results in the case of Lebesgue measure on the real line.
    \item Prove Corollary 7.\\
    \\Let $(X,\mathcal{M},\mu)$ be a measure space and $\{f_n\}$ be a sequence of measurable function on $X$.
    \\\begin{enumerate}[(i)]
        \item $f(x):=\sup_n\{f_n(x)\}$
        \\Fix any real number $c$.
        \\Let $y\in\{x\in X\mid f(x)>c\}$.
        Then $f(y)>c$.
        \\By definition of supremum, there exists an index $k$ such that 
        \[
            f(y)\ge f_k(y)>c,
        \]
        and therefore $y\in\{x\in X\mid f_k(x)>c\}$, which implies
        \[
            \{x\in X\mid f(x)>c\}\subseteq\bigcup_{n=1}^\infty\{x\in X\mid f_n(x)>c\}.\tag{1}
        \]
        Let $y'\in\bigcup_{n=1}^\infty\{x\in X\mid f_n(x)>c\}$.
        \\Then there exists an index $k$ such that $y'\in\{x\in X\mid f_k(x)>c\}$.
        \\By definition of supremum, we have 
        \[
            f(y')\ge f_k(y')>c,
        \]
        and therefore $y'\in\{x\in X\mid f(x)>c\}$, which implies
        
        \[
            \{x\in X\mid f(x)>c\}\supseteq\bigcup_{n=1}^\infty\{x\in X\mid f_n(x)>c\}\tag{2}
        \]
        Then by (1) and (2),
        \[
            \{x\in X\mid f(x)>c\}=\bigcup_{n=1}^\infty\{x\in X\mid f_n(x)>c\},
        \]
        which is measurable because it is the countable union of measurable sets.
        \item $f(x):=\inf_n\{f_n(x)\}$
        \\a
        \item $f(x):=\limsup_n\{f_n(x)\}$
        \\a
        \item $f(x):=\liminf_n\{f_n(x)\}$
        \\a
    \end{enumerate}
    \item Prove Egoroff's Theorem.
    Is Egoroff's Theorem true in the absence of the assumption that the limit function is finite a.e.?
    \item Let $\{f_n\}$ be a sequence of real valued measurable functions on $X$ such that, for each natural number $n$, $\mu\{x\in X\mid |f_n(x)-f_{n+1}(x)|>1/2^n\}<1/2^n$.
    Show that $\{f_n\}$ is pointwise convergent a.e. on $X$. 
    (Hint: Use the Borel-Cantelli Lemma.)
    \item Under the assumptions of Egoroff's Theorem, show that $X=\bigcup_{k=0}^\infty X_k$, where each $X_k$ is measurable, $\mu(X_0)=0$ and, for $k\ge1$, $\{f_n\}$ converges uniformly to $f$ on $X_k$.
    \item A sequence $\langle f_n\rangle$ of measurable real-valued functions on $X$ is said to \textbf{converge in measure} to a measurable function $f$ provided that for each $\eta>0$,
    \[
        \lim_{n\to\infty}\mu\{x\in X\mid|f_n(x)-f(x)>\eta|\}=0.
    \]
    A sequence $\langle f_n\rangle$ of measurable functions is said to \textbf{Cauchy in measure} provided that for each $\epsilon>0$ and $\eta>0$, there is an index $N$ such that for each $m,n\ge N$,
    \[
        \mu\{x\in X\mid|f_n(x)-f_m(x)>\eta|\}<\epsilon.
    \]
    \begin{enumerate}[(i)]
        \item Show that if $\mu(X)<\infty$ and $\{f_n\}$ converges pointwise a.e. on $X$ to a measurable function $f$, then $\{f_n\}$ converges to $f$ in measure.
        (Hint: Use Egoroff's Theorem.)
        \item Show that if $\{f_n\}$ converges to $f$ in measure, then there is a subsequence of $\{f_n\}$ that converges pointwise a.e. on $X$ to $f$.
        (Hint: Use the Borel-Cantelli Lemma.)
        \item Show that if $\{f_n\}$ is Cauchy in measure, then there is a measurable function $f$ to which $\{f_n\}$ converges in measure.
    \end{enumerate}
    \item Assume $\mu(X)<\infty$.
    Show that $\{f_n\}\to f$ in measure iff each subsequence of $\{f_n\}$ has a further subsequence that converges pointwise a.e. on $X$ to $f$.
    Use this to show that for two sequences that converge in measure, the product sequence also converges in measure to the product of the limits.
\end{enumerate}

% 18.2
\authoredby{untouched}
\section{Integration of Nonnegative Measurable Functions}
\begin{center}
	\textbf{PROBLEMS}
\end{center}
\begin{enumerate}
	\setcounter{enumi}{16}
    \item 
\end{enumerate}

% 18.3
\section{Integration of General Measurable Functions}
\begin{center}
	\textbf{PROBLEMS}
\end{center}
\begin{enumerate}
	\setcounter{enumi}{26}
    \item 
\end{enumerate}

% 18.4
\section{The Radon-Nikodym Theorem}
\begin{center}
	\textbf{PROBLEMS}
\end{center}
\begin{enumerate}
	\setcounter{enumi}{48}
    \item 
\end{enumerate}

% 18.5
\section{The Nikodym Metric Space: The Vitali-Hahn-Saks Theorem}
\begin{center}
	\textbf{PROBLEMS}
\end{center}
\begin{enumerate}
	\setcounter{enumi}{60}
    \item 
\end{enumerate}

% Chapter 19
\chapter{General $L^p$ spaces: Completeness, Duality, and Weak Convergence}

\section{The Completeness of $L^p(X,\mu),1\le p \le \infty$}
\section{The Riesz Representation Theorem for the Dual of $L^p(X,\mu),1\le p < \infty$}
\section{The Kantorovitch Representation Theorem for the Dual of $L^\infty(X,\mu)$}
\section{Weak Sequential Compactness in $L^p(X,\mu),1< p < \infty$}
\section{Weak Sequential Compactness in $L^1(X,\mu)$: The Dunford-Pettis Theorem}

% Chapter 20
\chapter{The Construction of Particular Measures}

\section{Product Measures: The Theorems of Fubini and Tonelli}
\section{Lebesgue Measure on Euclidean Space $\mathbb{R}^n$}
\section{Cumulative Distribution Functions and Borel Measures on $\mathbb{R}$}
\section{Carath\'eodory Outer Measures and Hausdorff Measures on a Metric Space}

% Chapter 21
\chapter{Measure and Topology}

\section{Locally Compact Topological Spaces}
\section{Separating Sets and Extending Functions}
\section{The Construction of Radon Measures}
\section{The Representation of Positive Linear Functionals on $C_c(X)$: The Riesz-Markov Theorem}
\section{The Riesz Representation Theorem for the Dual of $C(X)$}
\section{Regularity Properties of Baire Measures}

% Chapter 22
\chapter{Invariant Measures}

\section{Topological Groups: The General Linear Group}
\section{Kakutani's Fixed Point Theorem}
\section{Invariant Borel Measures on Compact Groups: von Neumann's Theorem}
\section{Measure-Preserving Transformations and Ergodicity: The Bogoliubov-Krilov Theorem}


% Chapter 3
\chapter{Lebesgue Measurable Functions}

\section{Sums, Products, and Compositions}
\section{Sequential Pointwise Limits and Simple Approximation}
\section{Littlewood's Three Principles, Ergoff's Theorem, and Lusin's Theorem}

% Chapter 4
\chapter{Lebesgue Integration}

\section{The Riemann Integral}
\section{The Lebesgue Integral of a Bounded Measurable Function over a Set of Finite Measure}
\section{The Lebesgue Integral of a Measurable Nonnegative Function}
\section{The General Lebesgue Integral}
\section{Countable Additivity and Continuity of Integration}
\section{Uniform Integrability: The Vitali Convergence Theorem}

% Chapter 5
\chapter{Lebesgue Integration: Further Topics}

\section{Uniform Integrability and Tightness: A General Vitali Convergence Theorem}
\section{Convergence in Measure}
\section{Characterizations of Riemann and Lebesgue Integrability}

% Chapter 6
\chapter{Differentiation and Integration}

\section{Continuity of Monotone Functions}
\section{Differentiability of Monotone Functions: Lebesgue's Theorem}
\section{Functions of Bounded Variation: Jordan's Theorem}
\section{Absolutely Continuous Functions}
\section{Integrating Derivatives: Differentiating Indefinite Integrals}
\section{Convex Functions}

% Chapter 7
\chapter{The $L^p$ Spaces: Completeness and Approximation}

\section{Normed Linear Spaces}
\begin{center}
	\textbf{PROBLEMS}
\end{center}
\begin{enumerate}
	\setcounter{enumi}{0}
	\item For $f$ in $C[a,b]$, Define
	\[
	\| f \|_1 = \int_a^b |f|.	
	\]
	Show that this is a norm on $C[a,b]$.
	Also show that there is no number $c \ge 0$ for which
	\[
	\| f \|_{\max}	\le c \| f \|_1 \text{ for all $f$ in $C[a,b]$},
	\]
	but there is a $c \ge 0$ for which 
	\[
	\| f \|_1	\le c \| f \|_{\max} \text{ for all $f$ in $C[a,b]$}.
	\]
	\item Let $X$ be the family of all polynomials with real coefficients defined on $\mathbb{R}$.
	Show that this is a linear space. For a polynomial $p$, define $\| p\|$ to be the sum of the absolute values of the coefficients of $p$.
	Is this a norm?
	\item For $f$ in $L^1[a,b]$, define $\|f\| = \smallint_a^b x^2 |f(x)|dx$.
	Show that this is a norm on $L^1[a,b]$.
	\item For $f$ in $L^\infty[a,b]$, show that 
	\[
	\| f\|_\infty = \min \biggl \{ M \ \biggl |\ m \{x \in [a,b]\ |\ |f(x)| > M \} =0 \biggr \},
	\] 
	and if, furthermore, $f$ is continuous on $[a,b]$, that
	\[
	\| f \|_{\infty} = \| f \|_{\max}.	
	\]
	\item Show that $\ell^\infty$ and $\ell^1$ are normed linear spaces.
\end{enumerate}

\section{The Inequalities of Young, H\"older, and Minkowski}
\section{$L^p$ is Complete: The Riesz-Fischer Theorem}
\section{Approximation and Separability}

% Chapter 8
\chapter{The $L^p$ Spaces: Duality and Weak Convergence}

\section{The Riesz Representation for the Dual of $L^p,a\le p\le \infty$}
\section{Weak Sequential Convergence in $L^p$}
\section{Weak Sequential Compactness}
\section{The Minimization of Convex Functionals}

\setcounter{chapter}{0}
\chapter*{II ABSTRACT SPACES: METRIC, TOPOLOGICAL, BANACH, AND HILBERT SPACES}
\addcontentsline{toc}{chapter}{II ABSTRACT SPACES: METRIC, TOPOLOGICAL, BANACH, AND HILBERT SPACES}
\setcounter{chapter}{8}

% Chapter 9
\chapter{Metric Spaces: General Properties}

\section{Examples of Metric Spaces}
\section{Open Sets, Closed Sets, and Convergent Sequences}
\section{Continuous Mappings Between Metric Spaces}
\section{Complete Metric Spaces}
\section{Compact Metric Spaces}
\section{Separable Metric Spaces}

% Chapter 10
\chapter{Metric Spaces: Three Fundamental Theorems}

\section{The Arzel\'a-Ascoli Theorem}
\section{The Baire Category Theorem}
\section{The Banach Contraction Principle}

% Chapter 11
\chapter{Topological Spaces: General Properties}

\section{Open Sets, Closed Sets, Bases, and Subbases}
\section{The Separation Properties}
\section{Countability and Separability}
\section{Continuous Mappings Between Topological Spaces}
\section{Compact Topological Spaces}
\section{Connected Topological Spaces}

% Chapter 12
\chapter{Topological Spaces: Three Fundamental Theorems}

\section{Urysohn's Lemma and the Tietze Extension Theorem}
\section{The Tychonoff Product Theorem}
\section{Thye Stone-Weierstrass Theorem}

% Chapter 13
\chapter{Continuous Linear Operators Between Banach Spaces}

\section{Normed Linear Spaces}
\section{Linear Operators}
\section{Compactness Lost: Infinite Dimensional Normed Linear Spaces}
\section{The Open Mapping and Closed Graph Theorems}\
\section{The Uniform Boundedness Principle}

% Chapter 14
\chapter{Duality for Normed Linear Spaces}

\section{Linear Functionals, Bounded Linear Functionals, and Weak Topologies}
\section{The Hahn-Banach Theorem}
\section{Reflexive Banach Spaces and Weak Sequential Convergence}
\section{Locally Convex Topological Vector Spaces}
\section{The Separation of Convex Sets and Mazur's Theorem}
\section{The Krein-Milman Theorem}

% Chapter 15
\chapter{Compactness Regained: The Weak Topology}

\section{Alaoglu's Extension of Helley's Theorem}
\section{Reflexivity and Weak Compactness: Kakutani's Theorem}
\section{Compactness and Weak Sequential Compactness: The Eberlein-\v Smulian Theorem}
\section{Metrizability of Weak Topologies}

% Chapter 16
\chapter{Continuous Linear Operators on Hilbert Spaces}

\section{The Inner Product and Orthogonality}
\section{The Dual Space and Weak Sequential Convergence}
\section{Bessel's Inequality and Orthonormal Bases}
\section{Adjoints and Symmetry for Linear Operators}
\section{Compact Operators}
\section{The Hilbert-Schmidt Theorem}
\section{The Riesz-Schauder Theorem: Characterization of Fredholm Operators}

\setcounter{chapter}{0}
\chapter*{III MEASURE AND INTEGRATION: GENERAL THEORY} 
\addcontentsline{toc}{chapter}{III MEASURE AND INTEGRATION: GENERAL THEORY}
\setcounter{chapter}{16}

% Chapter 17
\chapter{General Measure Spaces: Their Properties and Construction}

\section{Measures and Measurable Sets}

\begin{center}
	\textbf{PROBLEMS}
\end{center}
\begin{enumerate}
	\setcounter{enumi}{0}
	\item Let $f$ be a nonnegative Lebesgue measurable function on $\mathbb{R}$. 
	For each Lebesgue measurable subset $E$ of $\mathbb{R}$, define $\mu(E) = \smallint_E f$, the Lebesgue integral of $f$ over $E$.
	Show that $\mu$ is a measure on the $\sigma$-algebra of Lebesgue measurable subsets of $\mathbb{R}$.
	\item Let $\mathcal{M}$ be a $\sigma$-algebra of subsets of a set $X$ and the set function $\mu : \mathcal{M} \to [0,\infty)$ be finitely additive.
	Prove that $\mu$ is a measure iff whenever $\{A_k\}_{k=1}^\infty$ is an ascending sequence of sets in $\mathcal{M}$, then
	\[
	\mu \biggl ( \bigcup_{k=1}^\infty A_k \biggr ) = \lim_{k \to \infty} \mu(A_k).	
	\]
\end{enumerate}

\section{Signed Measures: The Hahn and Jordan Decompositions}
\section{The Cath\'eodory Measure Induced by an Outer Measure}
\section{The Construction of Outer Measures}
\section{The Cath\'eodory-Hahn Theorem: The Extension of a Premeasure to a Measure}

% Chapter 18
\chapter{Integration Over General Measure Spaces}

\section{Measurable Functions}
\section{Integration of Nonnegative Measurable Functions}
\section{Integration of General Measurable Functions}
\section{The Radon-Nikodym Theorem}
\section{The Nikodym Metric Space: The Vitali-Hahn-Saks Theorem}

% Chapter 19
\chapter{General $L^p$ spaces: Completeness, Duality, and Weak Convergence}

\section{The Completeness of $L^p(X,\mu),1\le p \le \infty$}
\section{The Riesz Representation Theorem for the Dual of $L^p(X,\mu),1\le p < \infty$}
\section{The Kantorovitch Representation Theorem for the Dual of $L^\infty(X,\mu)$}
\section{Weak Sequential Compactness in $L^p(X,\mu),1< p < \infty$}
\section{Weak Sequential Compactness in $L^1(X,\mu)$: The Dunford-Pettis Theorem}

% Chapter 20
\chapter{The Construction of Particular Measures}

\section{Product Measures: The Theorems of Fubini and Tonelli}
\section{Lebesgue Measure on Euclidean Space $\mathbb{R}^n$}
\section{Cumulative Distribution Functions and Borel Measures on $\mathbb{R}$}
\section{Carath\'eodory Outer Measures and Hausdorff Measures on a Metric Space}

% Chapter 21
\chapter{Measure and Topology}

\section{Locally Compact Topological Spaces}
\section{Separating Sets and Extending Functions}
\section{The Construction of Radon Measures}
\section{The Representation of Positive Linear Functionals on $C_c(X)$: The Riesz-Markov Theorem}
\section{The Riesz Representation Theorem for the Dual of $C(X)$}
\section{Regularity Properties of Baire Measures}

% Chapter 22
\chapter{Invariant Measures}

\section{Topological Groups: The General Linear Group}
\section{Kakutani's Fixed Point Theorem}
\section{Invariant Borel Measures on Compact Groups: von Neumann's Theorem}
\section{Measure-Preserving Transformations and Ergodicity: The Bogoliubov-Krilov Theorem}

\end{document}