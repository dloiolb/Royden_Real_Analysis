\documentclass[a4paper,10pt]{book}
\usepackage[utf8]{inputenc}
\usepackage[T1]{fontenc}
\usepackage[]{mdframed}
\usepackage{lipsum}
\usepackage{xcolor}
\usepackage{amsfonts} 
\usepackage{times}
\usepackage[shortlabels]{enumitem}
\usepackage{amsmath}
\usepackage{centernot}
\usepackage{amsthm,amssymb}
\usepackage{verbatim}
\usepackage{multicol}
\usepackage{titletoc}
\usepackage{graphicx}
\usepackage{minitoc}

%\newtheorem*{theorem*}{}
%\newtheorem{theorem}{Theorem}

\usepackage{amsthm}
%\usepackage{thmtools}

\newtheorem{theorem}{Theorem}[section] % the main one
%\newtheorem{lemma}[theorem]{Lemma}

\theoremstyle{plain} % just in case the style had changed
\newcommand{\thistheoremname}{}

\begin{comment}
\newtheorem{genericthm}[theorem]{\thistheoremname}
\newenvironment{namedthm}[1]
	{\renewcommand{\thistheoremname}{#1}%
	\begin{genericthm}}
	{\end{genericthm}}
\end{comment}

% https://tex.stackexchange.com/questions/12913/customizing-theorem-name

\newtheorem*{genericthm*}{\thistheoremname}
\newenvironment{namedthm*}[1]
	{\renewcommand{\thistheoremname}{#1}%
	\begin{genericthm*}}
	{\end{genericthm*}}


%\newmdtheoremenv{theo}{Theorem}

% https://tex.stackexchange.com/questions/43536/writing-steps-in-an-equation


\usepackage[T1]{fontenc}
\usepackage{geometry}
\usepackage{titlesec}
\usepackage{titletoc}
\usepackage{blindtext}  % drop in actual document

\titleformat{name=\chapter}[display]
{\normalfont\huge\bfseries}
{\chaptertitlename\ \thechapter}
{20pt}
{\Huge}
[\normalsize\normalfont\vspace*{1pc}%
\hbox{\large\bfseries\contentsname}\vspace{6pt}\titlerule\vspace{3pt}
\startcontents
\printcontents{l}{1}{\setcounter{tocdepth}{2}}\vspace{1pt}
\titlerule\vspace{1pc}]

\titleformat{name=\chapter,numberless}[display]
{\normalfont\huge\bfseries}
{}
{20pt}
{\Huge}

\usepackage{url}
\usepackage{hyperref}
\usepackage{geometry}
\geometry{a4paper}
\usepackage[english]{babel}
\title{Real Analysis Royden - Fourth Edition\\
	\large Notes + Solved Exercises :) \\
	\large \href{https://latex-programming.fandom.com/wiki/List_of_LaTeX_symbols}{Latex Symbols}
}
\author{J.B.}
\date{\small May 2024}

\begin{document}
\maketitle
\tableofcontents

\setcounter{chapter}{0}
\chapter*{I LEBESGUE INTEGRATION FOR FUNCTIONS OF A SINGLE REAL VARIABLE}
\addcontentsline{toc}{chapter}{I LEBESGUE INTEGRATION FOR FUNCTIONS OF A SINGLE REAL VARIABLE}
\setcounter{chapter}{0}

\setcounter{chapter}{0}
\chapter*{Preliminaries on Sets, Mappings, and Relations}
\addcontentsline{toc}{chapter}{Preliminaries on Sets, Mappings, and Relations}
\setcounter{chapter}{0}

\begin{flushleft}

\begin{namedthm*}{Definition}
A relation $R$ on a set $X$ is called an \textbf{equivalence relation} provided:
\begin{enumerate}[label=(\roman*),align=left]
	\item $xRx$ for all $x \in X$ (reflexive),
	\item $xRy$ implies $yRx$ for all $x,y \in X$ (symmetric),
	\item $xRy$ and $yRz$ imply $xRz$ for all $x,y,z \in X$ (transitive).
\end{enumerate}
\end{namedthm*}
\bigskip

\begin{namedthm*}{Partial Ordering on a set $X$}
A relation $R$ on a nonempty set $X$ is called a \textbf{partial ordering} provided:
\begin{enumerate}[label=(\roman*),align=left]
	\item $xRx$ for all $x \in X$ (reflexive),
	\item $xRy$ and $yRx$ imply $x=y$ for all $x,y \in X$ (antisymmetric),
	\item $xRy$ and $yRz$ imply $xRz$ for all $x,y,z \in X$ (transitive).
\end{enumerate}
A subset $E$ of $X$ is \textbf{totally ordered} provided either $xRy$ or $yRx$ for all $x,y \in E$.
A member $x$ of $X$ is said to be an \textbf{upper bound} for a subset $E$ of $X$ provided that 
\[ yRx \text{ for all } y \in E.\]
A member $x$ of $X$ is said to be \textbf{maximal} provided that 
\[ xRy \text{ implies that } y =x \text{ for } y \in X.\]
\end{namedthm*}
\medskip

\begin{namedthm*}{Strict Partial Ordering on a set $X$}
A relation $R$ on a nonempty set $X$ is called a \textbf{strict partial ordering} provided:
\begin{enumerate}[label=(\roman*),align=left]
	\item not $xRx$ for all $x \in X$ (irreflexive),
	\item $xRy$ implies not $yRx$ for all $x,y \in X$ (asymmetric),
	\item $xRy$ and $yRz$ imply $xRz$ for all $x,y,z \in X$ (transitive).
\end{enumerate}
A subset $E$ of $X$ is \textbf{strictly totally ordered} provided either $xRy$ or $yRx$ if $x\neq y$ for all $x,y \in E$.\par
\end{namedthm*}


\begin{namedthm*}{Zorn's Lemma}
	Let $X$ be a partially ordered set for which every totally ordered subset has an upper bound. Then $X$ has a maximal member.
\end{namedthm*}

\begin{namedthm*}{Every vector space has a basis}
\end{namedthm*}
\begin{proof}
Let V be any vector space, and let L be the collection of all linearly independent subsets of V. 
L is nonempty as the singleton sets are linearly independent. 
Define a partial order on L in the form $C \subseteq C'$ for $C,C' \in L$.
For any chain (a totally ordered subset of a partially ordered set) $\mathcal{C}$ of $L$, where $\mathcal{C}$ consists of the sets $C_1 \subseteq C_2 \subseteq \cdots$, we can construct a linearly independent upper bound $C' = \bigcup_{C \in \mathcal{C}} C$ of $\mathcal{C}$.
By Zorn's Lemma, L has a maximal element, say M.
This collection $M$ is a basis for $V$. To show this, suppose by contradiction that there exists a vector $v \in V$ s.t. $v \notin \text{ Span}\{M\}$.
Then $v \cup M$ is linearly independent and $M \subseteq v \cup M$, a contradiction to the fact that $M$ is maximal.
\end{proof}

\end{flushleft}

\chapter{The Real Numbers: Sets, Sequences, and Functions}
\begin{flushleft}

\textbf{The field axioms}\par
Consider $a,b,c \in \mathbb{R}$:
\begin{enumerate}
	\item Closure of Addition: $a+b \in \mathbb{R}$.
	\item Associativity of Addition: $(a+b)+c = a+(b+c)$.
	\item Additive Identity: $0+a=a+0=a$.
	\item Additive Inverse: $(-a)+a=a+(-a)=0$.
	\item Commutativity of Addition: $a+b=b+a$.
	\item Closure of Multiplication: $ab \in \mathbb{R}$.
	\item Associativity of Multiplication: $(ab)c = a(bc)$.
	\item Distributive Property: $ a(b+c)=ab+ac$.
	\item Commutativity of Multiplication: $ab=ba$.
	\item Multiplicative Identity: $1a=a1=a$.
	\item No Zero Divisors: $ab=0 \implies a=0 \text{ or } b=0$.
	\item Multiplicative Inverse: $a^{-1}a=aa^{-1}=1$.
	\item Nontriviality: $1 \neq 0$.
\end{enumerate}
\medskip

\textbf{The positivity axioms}\par
The set of \textbf{positive numbers}, $\mathcal{P}$, has the following two properties:
\begin{itemize}
    \item [P1] If $a$ and $b$ are positive, then $ab$ and $a+b$ are both positive.
    \item [P2] For a real number $a$, exactly one of the three is true: $a$ is positive, $-a$ is positive, $a=0$.	
\end{itemize}

We call a nonempty set $I$ of real numbers an \textbf{interval} provided for any two points in $I$, all the points that lie between these two points also lie in $I$.
That is, $\forall x,y \in I, \lambda x + (1-\lambda)y \in I \text{ for } \lambda \in [0,1]$.
\medskip

\textbf{The completeness axiom}\par
A nonempty set $E$ of real numbers is said to be \textbf{bounded above} provided there is a real number $b$ such that $x \le b$ for all $x\in E$: the number $b$ is called an \textbf{upper bound} for $E$.
We can similarly define a set being \textbf{bounded below} and having a \textbf{lower bound}. A set that is bounded above need not have a largest member.
\begin{namedthm*}{The Completeness Axiom}
Let $E$ be a nonempty set of real numbers that is bounded above. The among the set of upper bounds for $E$ there is a smallest, or least, upper bound, called the \textbf{supremum} of $E$.
Also, any nonempty set $E$ that is bounded below has a greatest lower bound, called the \textbf{infimum} of $E$.	
\end{namedthm*}

\end{flushleft}

\section{The Field, Positivity, and Completeness Axioms}
\begin{center}
	\textbf{PROBLEMS}
\end{center}
\begin{enumerate}
	\setcounter{enumi}{0}
	\item For $a\neq 0$ and $a\neq 0$, show that $(ab)^{-1} = b^{-1}a^{-1}$.\par

		\begin{align*}
			(ab)(ab)^{-1} & = 1 && \tag*{by multiplicative inverse}\\
			a(b(ab)^{-1}) & = 1 && \tag*{by associativity of multiplication} \\
			a^{-1}a(b(ab)^{-1}) & = a^{-1}1 \\
			1(b(ab)^{-1}) & = a^{-1}1 && \tag*{by multiplicative inverse} \\
			b(ab)^{-1} & = a^{-1} && \tag*{by multiplicative identity} \\
			b^{-1}b(ab)^{-1} & = b^{-1}a^{-1} \\
			1(ab)^{-1} & = b^{-1}a^{-1} && \tag*{by multiplicative inverse} \\
			(ab)^{-1} & = b^{-1}a^{-1} && \tag*{by multiplicative identity} \\
		\end{align*}

	\item Verify the following:
	\begin{enumerate}[label=(\roman*),align=left]
        \item For each real number $a\neq 0$, $a^2>0$. In particular, $1>0$ since $1 \neq 0$ and $1=1^2$.
        \item For each positive number $a$, its multiplicative inverse  $a^{-1}$ also is positive.
        \item If $a>b$, then \[ ac >bc \text{ if } c>0 \text{ and } ac < bc \text{ if } c<0. \]
    \end{enumerate}
	\item For a nonempty set of real numbers $E$, show that $\inf E = \sup E$ iff $E$ consists of a single point.
	\item Let $a$ and $b$ be real numbers.
	\begin{enumerate}[label=(\roman*),align=left]
        \item Show that if $ab = 0$, then $a=0$ or $b=0$.
        \item Verify that $a^2 -b^2 = (a-b)(a+b)$ and conclude from part (i) that if $a^2 = b^2$, then $a=b$ or $a=-b$.
        \item Let $c$ be a positive real number. Define $E = \{ x \in \mathbb{R} \ |\  x^2 < c\}$. Verify that $E$ is nonempty and bounded above. 
		Define $x_0 = \sup E$. Show that $x_0^2 = c$. Use part (ii) to show that there is a unique $x>0$ for which $x^2=c$. It is denoted by $\sqrt{c}$.
    \end{enumerate}
	\item Let $a,b,c$ be real numbers s.t. $a\neq 0$ and consider the quadratic equation \[ ax^2+bx+c=0, x \in \mathbb{R}.\]
	\begin{enumerate}[label=(\roman*),align=left]
        \item Suppose $b^2 - 4ac >0$. Use the Field Axioms and the preceding problem to complete the square and thereby show that this equation has exactly tqo solutions given by
		\[  x = \dfrac{-b + \sqrt{b^2-4ac}}{2a} \text{  and  } x = \dfrac{-b - \sqrt{b^2-4ac}}{2a}. \]
	\end{enumerate}
	\item Use the Completeness Axiom to show that every nonempty set of real numbers that is bounded below has an infimum and that
	\[\inf E =\sup \{-x \ |\ x \in E.\]
	\item For real numbers $a$ and $b$, verify the following:
	\begin{enumerate}[label=(\roman*),align=left]
		\item $|ab| = |a||b|.$
		\item $|a+b| \le |a|+|b|.$
		\item For $\epsilon >0,$
		\[ |x-a| < \epsilon \text{  iff  } a - \epsilon < x < a + \epsilon.\]
	\end{enumerate}
\end{enumerate}

\section{The Natural and Rational Numbers}

\begin{center}
	\textbf{PROBLEMS}
\end{center}
\begin{enumerate}
	\setcounter{enumi}{7}
	\item Use an induction argument to show that for each natural number $n$, the interval $(n, n+1)$ fails to contain any natural number.
	\item Use an induction argument to show that if $n>1$ is a natural number, then $n-1$ also is a natural number. The use another induction argument to show that if $m$ and $n$ are natural numbers with $n>m$, then $n-m$ is a natural number.
	\item Show that for any real number $r$, there is exactly one integer in the interval $[r,r+1)$.
	\item Show that ay nonempty set of integers that is bounded above has a largest member.
	\item Show that the irrational numbers are dense in $\mathbb{R}$.
	\item Show that each real number is the supremum of a set of rational numbers and also the supremum of a set of irrational numbers.
	\item Show that if $r>0$, then, for each natural number $n$, $(1+r)^n \ge 1+n \cdot r$.
	\item Use induction arguments to prove that for every natural number $n$,
	\begin{enumerate}[label=(\roman*),align=left]
        \item \[ \sum_{j=1}^n j^2 = \dfrac{n(n+1)(2n+1)}{6}, \]
        \item \[ 1^3 + 2^3 + \cdots + n^3 = (1+2+\cdots +n)^2, \]
        \item \[ 1+r+\cdots +r^n = \dfrac{1-r^{n+1}}{1-r} \text{ if } r \neq 1.\]
    \end{enumerate}
\end{enumerate}

\section{Countable and Uncountable Sets}

\begin{center}
	\textbf{PROBLEMS}
\end{center}
\begin{enumerate}
	\setcounter{enumi}{15}
	\item Show that the set $\mathbb{Z}$ of integers is countable.
	\item Show that a set $A$ is countable iff there is an injective mapping of $A$ to $\mathbb{N}$.
	\item Use an induction argument to complete the proof of part (i) of Corollary 4.
	\item Prove Corollary 6 in the case of a finite family of countable sets.
	\item Let both $f:A \to B$ and $g:B \to C$ be injective and surjective. Show that the composition $g \circ f:A \to B$ and the inverse $f^{-1}:B \to A$ are also injective and surjective.
	\item Use an induction argument to establish the pigeonhole principle.
	\item Show that $2^{\mathbb{N}}$, the collection of all sets of natural numbers, is uncountable.
	\item Show that the Cartesian product of a finite collection of countable sets is countable. Use the preceding theorem to show that $\mathbb{N}^{\mathbb{N}}$, the collection of all mappings of $\mathbb{N}$ into $\mathbb{N}$, is not countable.
	\item Show that a degenerate interval of real numbers fails to be finite.
	\item Show that any two nondegenerate intervals of real numbers are equipotent.
	\item Is the set $\mathbb{R} \times \mathbb{R}$ equipotent to $\mathbb{R}$?
\end{enumerate}

\section{Open Sets, Closed Sets, and Borel Sets of Real Numbers}

\begin{namedthm*}{The Nested Set Theorem}
Let $\{F_n\}_{n=1}^\infty$ be a descending countable collection of nonempty closed sets of real numbers for which $F_1$ is bounded.
Then
\[
    \bigcap_{n=1}^\infty F_n \neq \emptyset.
\]
\end{namedthm*}
\begin{proof} 
By contradiction, suppose that $\bigcap_{n=1}^\infty F_n = \emptyset$. 
Then $\bigcup_{n=1}^\infty F_n^c = (\bigcap_{n=1}^\infty F_n)^c  = \emptyset^c = \mathbb{R}$, and we have an open cover of $\mathbb{R}$ and thus an open cover of $F_1 \subseteq \mathbb{R}$. 
By the Heine-Borel Theorem, there exists an $N \in \mathbb {N}$ such that $F_1 \subseteq \bigcup_{n=1}^N F_n^c$.  
Because $\{F_n\}$ is descending, $F_n \supseteq F_{n+1}$ for any $n \ge 1$. 
This implies $F_{n}^c \subseteq F_{n+1}^c$, and thus $F_1 \subseteq \bigcup_{n=1}^N F_n^c = F_N^c = \mathbb{R}\setminus F_N$.
This is a contradiction to the assumption that $F_N$ is a nonempty subset of $F_1$.
\end{proof}

\begin{center}
	\textbf{PROBLEMS}
\end{center}
\begin{enumerate}
	\setcounter{enumi}{26}
	\item Is the set of rational numbers open or closed?
	\item What are the sets of real numbers that are both open and closed?
	\item Find two sets $A$ and $B$ such that $A \cap B = \emptyset$ and $\overline A \cap \overline B \neq \emptyset.$
	\item A point $x$ is called an \textbf{accumulation point} of a set $E$ provided it is a point of closure of $E \setminus \{ x\}.$
	\begin{enumerate}[label=(\roman*),align=left]
        \item Show that the set $E'$ of accumulation points of $E$ is a closed set.
        \item Show that $\overline E = E \cup E'.$
    \end{enumerate}
	\item A point $x$ is called an \textbf{ isolated point} of a set $E$ provided there is an $r>0$ for which $(x-r,x+r)\cap E = \{x\}.$ Show that if a set $E$ consists of isolated points, then it is countable.
	\item A point $x$ is called an \textbf{interior point} of a set $E$ if there is an $r>0$ such that the open interval $(x-r,x+r)$ is contained in $E$. The set of interior points of $E$ is called the \textbf{interior} of $E$ denoted by int $E$. Show that
	\begin{enumerate}[label=(\roman*),align=left]
        \item $E$ is open iff $E = \text{ int } E$.
        \item $E$ is dense iff $ \text{ int } (\mathbb{R} \setminus E)= \emptyset$.
    \end{enumerate}
	\item Show that the nested set theorem is false if $F_1$ is unbounded.
	\item Show that the assertion of the Heine-Borel Theorem is equivalent to the Completeness Axiom for the real numbers. Show that the assertion of the Nested Set Theorem is equivalent to the Completeness Axiom for the real numbers.
	\item Show that the collection of Borel sets is the smallest $\sigma$-algebra that contains the closed sets.
	\item Show that the collection of Borel sets is the smallest $\sigma$-algebra that contains the intervals of the form $[a,b)$, where $a<b.$
	\item Show that each open set is an $F_{\sigma}$ set.
\end{enumerate}

\section{Sequences of Real Numbers}

\begin{center}
	\textbf{PROBLEMS}
\end{center}
\begin{enumerate}
	\setcounter{enumi}{37}
	\item We call an extended real number a \textbf{cluster point} of a sequence $\{ a_n\}$ if a subsequence converges to this extended real number. Show that $\lim \inf \{a_n\}$ is the smallest cluster point of $\{a_n\}$ and $\lim \sup \{a_n\}$ is the largest cluster point of $\{a_n\}$.
	\item Prove proposition 19.
	\item Show that a sequence $\{a_n\}$ is convergent to an extended real number iff there is exactly one extended real number that is a cluster point of the sequence.
	\item Show that $\lim \inf a_n \le \lim \sup a_n$.
	\item Prove that if, for all $n$, $a_n \ge 0$ and $b_n \ge 0 $, then \[ \lim \sup [a_n \cdot b_n] \le (\lim \sup a_n) \cdot (\lim \sup b_n),\] provided the product on the right is not of the form $0 \cdot \infty.$
	\item Show that every real sequence has a monotone subsequence. Use this to provide another proof of the Bolzano-Weierstrass Theorem.
	\item Let $p$ be a natural number greater than 1, and $x$ a real number $0 \le x \le 1.$ Show that there is a sequence $\{a_n\}$ of integers with $0 \le a_n < p$ for each $n$ such that \[ x = \sum_{n=1}^\infty\dfrac{a_n}{p^n} \] 
	and that this sequence is unique except when $x$ is of the form $q/p^n$, $0<q<p^n$, in which case there are exactly two such sequences. Show that, conversely, if $\{a_n\}$ is any sequence of integers with $0\le a_n < p$, the series \[ x = \sum_{n=1}^\infty\dfrac{a_n}{p^n} \] 
	converges to a real number $x$ with $0 \le x \le 1$. If $p = 10$, this sequence is called the \textit{decimal} expansion of $x$. For $p=2$ it is called the \textit{binary} expansion; and for $p=3$, the \textit{ternary} expansion.
	\item Prove Proposition 20.
	\item Show that the assertion of the Bolzano-Weierstrass Theorem is equivalent to the Completeness Axiom for the real numbers. Show that the assertion of the Monotone Convergence Theorem is equivalent to the Completeness Axiom for the real numbers.
\end{enumerate}

\section{Continuous Real-Valued Functions of a Real Variable}

\begin{center}
	\textbf{PROBLEMS}
\end{center}
\begin{enumerate}
	\setcounter{enumi}{46}
	\item Let $E$ be a closed set of real numbers and $f$ a real-valued function that is defined and continuous on $E$. Show that there is a function $g$ defined and continuous on all of $\mathbb{R}$ such that $f(x) = g(x)$ for each $x \in E$. (Hint: Take $g$ to be linear on each of the intervals of which $\mathbb{R} \setminus E$ is composed.)
	\item Define the real-valued function $f$ on $\mathbb{R}$ by setting 
	\[ 
	f(x) =
	\begin{cases} 
		x & \text{if x irrational}\\
		p \sin \dfrac{1}{q} & \text{if } x = \dfrac{p}{q} \text{ in lowest terms.} \\
	\end{cases}
	\]
	At what points is $f$ continuous?
	\item Let $f$ and $g$ be continuous real-valued functions with a common domain $E$.
	\begin{enumerate}[label=(\roman*),align=left]
        \item Show that the sum, $f+g$, and product, $fg$, are also continuous functions.
        \item If $h$ is a continuous function with image contained in $E$, show that the composition $f \circ h$ is continuous.
        \item Let max$\{f,g\}$ be the function defined by max$\{f,g\}(x)$ = max$\{f(x),g(x)\}$, for $x \in E$. Show that max$\{f,g\}$ is continuous.
        \item Show that $|f|$ is continuous.
    \end{enumerate}
	\item Show that a Lipschitz function is uniformly continuous but there are uniformly continuous functions that are not Lipschitz.
	\item A continuous function $\phi$ on $[a,b]$ is called \textbf{piecewise linear} provided there is a partition $a=x_0<x_1< \cdots <x_n = b$ of $[a,b]$ for which $\phi$ is linear on each interval $[x_i, x_{i+1}]$. Let $f$ be a continuous function on $[a,b]$ and $\epsilon$ a positive number. 
	Show that there is a piecewise linear function $\phi$ on $[a,b]$ with $|f(x)-\phi (x)| < \epsilon$ for all $x \in [a,b]$.
	\item Show that a nonempty set $E$ of real numbers is closed and bounded if and only if every continuous real-valued function on $E$ takes a maximum value.
	\item Show that a set $E$ of real numbers is closed and bounded iff every open cover of $E$ has a finite subcover.
	\item Show that a nonempty set $E$ of real numbers is an interval iff every continuous real-valued function on $E$ has an interval as its image.
	\item Show that a monotone function on an open interval is continuous iff its image is an interval. 
	\item Let $f$ be a real-valued function defined on $\mathbb{R}$. Show that the set of points at which $f$ is continuous is a $G_\delta$ set.
	\item Let $\{ f_n\}$ be a sequence of continuous functions defined on $\mathbb{R}$. Show that the set of points $x$ at which the sequence $\{f_n(x)\}$ converges to a real number is the intersection of a countable collection of $F_\sigma$ sets.
	\item Let $f$ be a continuous real-valued function on $\mathbb{R}$. Show that the inverse image with respect to $f$ of an open set is open, of a closed set is closed, and of a Borel set is Borel.
	\item A sequence $\{f_n\}$ of real-valued functions defined on a set $E$ is said to converge uniformly on $E$ to a function $f$ iff given $\epsilon >0$, there is an $N$ such that for all $x \in E$ and all $n \ge N$, we have $|f_n(x) - f(x)| < \epsilon$. Let $\{f_n\}$ be a sequence of continuous functions defined on a set $E$. Prove that if $\{f_n\}$ converges uniformly to $f$ on $E$, then $f$ is continuous on $E$. 
\end{enumerate}

\chapter{Lebesgue Measure}

\section{Introduction}
In this chapter we construct a collection of sets called \textbf{Lebesgue measurable sets}, and a set function of this collection called \textbf{Lebesgue measure}, denoted by $m$. The collection of Lebesgue measurable sets is a $\sigma$-algebra which contains all open sets and all closed sets. The set function $m$ possesses the following three properties:
\begin{namedthm*}{The measure of an interval is its length}
Each nonempty interval $I$ is Lebesgue measurable and 
\[
m(I) = \ell(I).
\]
\end{namedthm*}
\begin{namedthm*}{Measure is translation invariant}
If $E$ is Lebesgue measurable and $y$ is any number then the translate of $E$ by $y$, $E+y = \{x+y \ |\ x \in E\}$, also is Lebesgue measurable and
\[
m(E+y) = m(E).
\]
\end{namedthm*}
\begin{namedthm*}{Measure is countably additive over countable disjoint unions of sets}
IF $\{E_k\}_{k=1}^\infty$ is a countable disjoint collection of Lebesgue measurable sets, then
\[
m(\bigcup_{k=1}^\infty E_k) = \sum_{k=1}^\infty m(E_k).
\]
\end{namedthm*}
It is not possible to construct a set function that possesses the above three properties and is defined for all sets of real numbers (See Vitali sets).
We first construct a set function called \textbf{outer measure}, denoted by $m^*$, such that: 
\begin{enumerate}[label=(\roman*),align=left]
	\item the outer measure of an interval is its length,
	\item outer measure is translation invariant,
	\item outer measure is countably subadditive.
\end{enumerate}
Then the Lebesgue measure $m$ is the restriction of $m^*$ to the Lebesgue measurable sets.

\begin{center}
	\textbf{PROBLEMS}
\end{center}
In the first three problems, let $m$ be a set function defined for all sets in a $\sigma$-algebra $\mathcal{A}$ with values in $[0,\infty]$. Assume $m$ is countably additive over countable disjoint collections of sets in $\mathcal{A}$.
\begin{enumerate}
	\setcounter{enumi}{0}
	\item Prove that if $A$ and $B$ are two sets in $\mathcal{A}$ with $A \subseteq B$, then $m(A) \le m(B)$. This property is called \textit{monotonicity}.
	\item Prove that if there is a set $A$ in the collection $\mathcal{A}$ for which $m(A) < \infty$, then $m(\emptyset) = 0$.
	\item Let $\{E_k\}_{k=1}^\infty$ be a countable collection of sets in $\mathcal{A}$. Prove that $m(\bigcup_{k=1}^\infty E_k) \le \sum_{k=1}^\infty m(E_k).$
	\item A set function $c$, defined on all subsets of $\mathbb{R}$, is defined as follows.
	Define $c(E)$ to be $\infty$ if $E$ has infinitely many members and $c(E)$ to be equal to the number of elements in $E$ if $E$ is finite; define $c(\emptyset)=0$. Show that $c$ is a countably additive and translation invariant set function. This set function is called the \textbf{counting measure}.
\end{enumerate}


\section{Lebesgue Outer Measure}
\section{The $\sigma$-Algebra of Lebesgue Measurable Sets}
\section{Outer and Inner Approximation of Lebesgue Measurable Sets}
\section{Countable Additivity, Continuity, and the Borel-Cantelli Lemma}
\section{Nonmeasurable Sets}
\section{The Cantor Set and the Cantor-Lebesgue Function}

\chapter{Lebesgue Measurable Functions}
\chapter{Lebesgue Integration}
\chapter{Lebesgue Integration: Further Topics}
\chapter{Differentiation and Integration}
\chapter{The $L^p$ Spaces: Completeness and Approximation}
\chapter{The $L^p$ Spaces: Duality and Weak Convergence}

\setcounter{chapter}{0}
\chapter*{II ABSTRACT SPACES: METRIC, TOPOLOGICAL, BANACH, AND HILBERT SPACES}
\addcontentsline{toc}{chapter}{II ABSTRACT SPACES: METRIC, TOPOLOGICAL, BANACH, AND HILBERT SPACES}
\setcounter{chapter}{8}

\chapter{Metric Spaces: General Properties}
\chapter{Metric Spaces: Three Fundamental Theorems}
\chapter{Topological Spaces: General Properties}
\chapter{Topological Spaces: Three Fundamental Theorems}
\chapter{Continuous Linear Operators Between Banach Spaces}
\chapter{Duality for Normed Linear Spaces}
\chapter{Compactness Regained: The Weak Topology}
\chapter{Continuous Linear Operators on Hilbert Spaces}

\setcounter{chapter}{0}
\chapter*{III MEASURE AND INTEGRATION: GENERAL THEORY} 
\addcontentsline{toc}{chapter}{III MEASURE AND INTEGRATION: GENERAL THEORY}
\setcounter{chapter}{16}

\chapter{General Measure Spaces: Their Properties and Construction}
\chapter{Integration Over General Measure Spaces}
\chapter{General $L^p$ spaces: Completeness, Duality, and Weak Convergence}
\chapter{The Construction of Particular Measures}
\chapter{Measure and Topology}
\chapter{Invariant Measures}

\end{document}