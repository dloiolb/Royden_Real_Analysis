\documentclass[a4paper,10pt]{book}
\usepackage[utf8]{inputenc}
\usepackage[T1]{fontenc}
\usepackage[]{mdframed}
\usepackage{lipsum}
\usepackage{xcolor}
\usepackage{amsfonts} 
\usepackage{times}
\usepackage[shortlabels]{enumitem}
\usepackage{amsmath}
\usepackage{centernot}
\usepackage{amsthm,amssymb}
\usepackage{verbatim}
\usepackage{multicol}
\usepackage{titletoc}
\usepackage{graphicx}
\usepackage{minitoc}

%\newtheorem*{theorem*}{}
%\newtheorem{theorem}{Theorem}

\usepackage{amsthm}
%\usepackage{thmtools}

\newtheorem{theorem}{Theorem}[section] % the main one
%\newtheorem{lemma}[theorem]{Lemma}

\theoremstyle{plain} % just in case the style had changed
\newcommand{\thistheoremname}{}

\begin{comment}
\newtheorem{genericthm}[theorem]{\thistheoremname}
\newenvironment{namedthm}[1]
	{\renewcommand{\thistheoremname}{#1}%
	\begin{genericthm}}
	{\end{genericthm}}
\end{comment}

% https://tex.stackexchange.com/questions/12913/customizing-theorem-name

\newtheorem*{genericthm*}{\thistheoremname}
\newenvironment{namedthm*}[1]
	{\renewcommand{\thistheoremname}{#1}%
	\begin{genericthm*}}
	{\end{genericthm*}}


%\newmdtheoremenv{theo}{Theorem}


\usepackage[T1]{fontenc}
\usepackage{geometry}
\usepackage{titlesec}
\usepackage{titletoc}
\usepackage{blindtext}  % drop in actual document

\titleformat{name=\chapter}[display]
{\normalfont\huge\bfseries}
{\chaptertitlename\ \thechapter}
{20pt}
{\Huge}
[\normalsize\normalfont\vspace*{1pc}%
\hbox{\large\bfseries\contentsname}\vspace{6pt}\titlerule\vspace{3pt}
\startcontents
\printcontents{l}{1}{\setcounter{tocdepth}{2}}\vspace{1pt}
\titlerule\vspace{1pc}]

\titleformat{name=\chapter,numberless}[display]
{\normalfont\huge\bfseries}
{}
{20pt}
{\Huge}

\usepackage{url}
\usepackage{hyperref}
\usepackage{geometry}
\geometry{a4paper}
\usepackage[english]{babel}
\title{Real Analysis Royden - Fourth Edition\\
	\large Notes + Solved Exercises :) \\
	\large \href{https://latex-programming.fandom.com/wiki/List_of_LaTeX_symbols}{Latex Symbols}
}
\author{J.B.}
\date{\small May 2024}

\begin{document}
\maketitle
\tableofcontents

\setcounter{chapter}{0}
\chapter*{I LEBESGUE INTEGRATION FOR FUNCTIONS OF A SINGLE REAL VARIABLE}
\addcontentsline{toc}{chapter}{I LEBESGUE INTEGRATION FOR FUNCTIONS OF A SINGLE REAL VARIABLE}
\setcounter{chapter}{0}

\setcounter{chapter}{0}
\chapter*{Preliminaries on Sets, Mappings, and Relations}
\addcontentsline{toc}{chapter}{Preliminaries on Sets, Mappings, and Relations}
\setcounter{chapter}{0}

\chapter{The Real Numbers: Sets, Sequences, and Functions}
\minitoc

\section{The Field, Positivity, and Completeness Axioms}
\begin{center}
	\textbf{PROBLEMS}
\end{center}
\begin{enumerate}
	\setcounter{enumi}{0}
	\item For $a\neq 0$ and $a\neq 0$, show that $(ab)^{-1} = a^{-1}b^{-1}$.
	\item Verify the following:
	\begin{enumerate}[label=(\roman*),align=left]
        \item For each real number $a\neq 0$, $a^2>0$. In particular, $1>0$ since $1 \neq 0$ and $1=1^2$.
        \item For each positive number $a$, its multiplicative inverse  $a^{-1}$ also is positive.
        \item If $a>b$, then \[ ac >bc \text{ if } c>0 \text{ and } ac < bc \text{ if } c<0. \]
    \end{enumerate}
\end{enumerate}

\section{The Natural and Rational Numbers}

\begin{center}
	\textbf{PROBLEMS}
\end{center}
\begin{enumerate}
	\setcounter{enumi}{7}
	\item Use an induction argument to show that for each natural number $n$, the interval $(n, n+1)$ fails to contain any natural number.
	\item Use an induction argument to show that if $n>1$ is a natural number, then $n-1$ also is a natural number. The use another induction argument to show that if $m$ and $n$ are natural numbers with $n>m$, then $n-m$ is a natural number.
	\item Show that for any real number $r$, there is exactly one integer in the interval $[r,r+1)$.
	\item Show that ay nonempty set of integers that is bounded above has a largest member.
	\item Show that the irrational numbers are dense in $\mathbb{R}$.
	\item Show that each real number is the supremum of a set of rational numbers and also the supremum of a set of irrational numbers.
	\item Show that if $r>0$, then, for each natural number $n$, $(1+r)^n \ge 1+n \cdot r$.
	\item Use induction arguments to prove that for every natural number $n$,
	\begin{enumerate}[label=(\roman*),align=left]
        \item \[ \sum_{j=1}^n j^2 = \dfrac{n(n+1)(2n+1)}{6}, \]
        \item \[ 1^3 + 2^3 + \cdots + n^3 = (1+2+\cdots +n)^2, \]
        \item \[ 1+r+\cdots +r^n = \dfrac{1-r^{n+1}}{1-r} \text{ if } r \neq 1.\]
    \end{enumerate}
\end{enumerate}

\section{Countable and Uncountable Sets}
\section{Open Sets, Closed Sets, and Borel Sets of Real Numbers}

\begin{namedthm*}{The Nested Set Theorem}
Let $\{F_n\}_{n=1}^\infty$ be a descending countable collection of nonempty closed sets of real numbers for which $F_1$ is bounded.
Then
\[
    \bigcap_{n=1}^\infty F_n \neq \emptyset.
\]
\end{namedthm*}
\begin{proof} 
By contradiction, suppose that $\bigcap_{n=1}^\infty F_n = \emptyset$. 
Then $\bigcup_{n=1}^\infty F_n^c = (\bigcap_{n=1}^\infty F_n)^c  = \emptyset^c = \mathbb{R}$, and we have an open cover of $\mathbb{R}$ and thus an open cover of $F_1 \subseteq \mathbb{R}$. 
By the Heine-Borel Theorem, there exists an $N \in \mathbb {N}$ such that $F_1 \subseteq \bigcup_{n=1}^N F_n^c$.  
Because $\{F_n\}$ is descending, $F_n \supseteq F_{n+1}$ for any $n \ge 1$. 
This implies $F_{n}^c \subseteq F_{n+1}^c$, and thus $F_1 \subseteq \bigcup_{n=1}^N F_n^c = F_N^c = \mathbb{R}\setminus F_N$.
This is a contradiction to the assumption that $F_N$ is a nonempty subset of $F_1$.
\end{proof}

\begin{center}
	\textbf{PROBLEMS}
\end{center}
\begin{enumerate}
	\setcounter{enumi}{26}
	\item Is the set of rational numbers open or closed?
	\item What are the sets of real numbers that are both open and closed?
	\item Find two sets $A$ and $B$ such that 
\end{enumerate}

\section{Sequences of Real Numbers}
\section{Continuous Real-Valued Functions of a Real Variable}

\chapter{Lebesgue Measure}
\chapter{Lebesgue Measurable Functions}
\chapter{Lebesgue Integration}
\chapter{Lebesgue Integration: Further Topics}
\chapter{Differentiation and Integration}
\chapter{The $L^p$ Spaces: Completeness and Approximation}
\chapter{The $L^p$ Spaces: Duality and Weak Convergence}

\setcounter{chapter}{0}
\chapter*{II ABSTRACT SPACES: METRIC, TOPOLOGICAL, BANACH, AND HILBERT SPACES}
\addcontentsline{toc}{chapter}{II ABSTRACT SPACES: METRIC, TOPOLOGICAL, BANACH, AND HILBERT SPACES}
\setcounter{chapter}{8}

\chapter{Metric Spaces: General Properties}
\chapter{Metric Spaces: Three Fundamental Theorems}
\chapter{Topological Spaces: General Properties}
\chapter{Topological Spaces: Three Fundamental Theorems}
\chapter{Continuous Linear Operators Between Banach Spaces}
\chapter{Duality for Normed Linear Spaces}
\chapter{Compactness Regained: The Weak Topology}
\chapter{Continuous Linear Operators on Hilbert Spaces}

\setcounter{chapter}{0}
\chapter*{III MEASURE AND INTEGRATION: GENERAL THEORY} 
\addcontentsline{toc}{chapter}{III MEASURE AND INTEGRATION: GENERAL THEORY}
\setcounter{chapter}{16}

\chapter{General Measure Spaces: Their Properties and Construction}
\chapter{Integration Over General Measure Spaces}
\chapter{General $L^p$ spaces: Completeness, Duality, and Weak Convergence}
\chapter{The Construction of Particular Measures}
\chapter{Measure and Topology}
\chapter{Invariant Measures}

\end{document}