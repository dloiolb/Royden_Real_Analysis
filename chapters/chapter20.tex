% Chapter 20
\authoredby{inprogress}
\chapter{The Construction of Particular Measures}

% 20.1
\authoredby{inprogress}
\section{Product Measures: The Theorems of Fubini and Tonelli}

Throughout this section $(X,\mathcal{A},\mu)$ and $Y,\mathcal{B},\nu$ are two reference measure spaces.
Consider the Cartesian product $X\times Y$ of $X$ and $Y$.
If $A\subset X$ and $B\subset Y$, we call $A\times B$ a rectangle.
If $A\in\mathcal{A}$ and $B\in\mathcal{B}$, we call $A\times B$ a \textbf{measurable rectangle}.

\begin{namedthm*}{Proposition 2}
    Let $\mathcal{R}$ be the collection of measurable rectangles in $X\times Y$ and for a measurable rectangle $A\times B$, define
    \[
        \lambda(A\times B):=\mu(A)\cdot\nu(B).
    \]
    Then $\mathcal{R}$ is a semiring and $\lambda:\mathcal{R}\to[0,\infty]$ is a premeasure.
\end{namedthm*}

\begin{namedthm*}{Definition}
    Let $(X,\mathcal{A},\mu)$ and $(Y,\mathcal{B},\nu)$ be measure spaces, $\mathcal{R}$ the collection of measurable rectangles contained in $X\times Y$, and $\lambda$ the premeasure defined on $\mathcal{R}$ by
    \[
        \lambda(A\times B):=\mu(A)\cdot\nu(B)\quad\text{for }A\times B\in\mathcal{R}.
    \]
    By the \textbf{product measure} $\lambda=\mu\times\nu$ we mean the Carath\'eodory extension of $\lambda:\mathcal{R}\to[0,\infty]$ defined on the $\sigma$-algebra of $(\mu\times\nu)^*$-measurable subsets of $X\times Y$.
\end{namedthm*}

\begin{center}
	\textbf{PROBLEMS}
\end{center}
\begin{enumerate}
	\setcounter{enumi}{0}
    \item Let $A\subset X$ and let $B$ be $\nu$-measurable subset of $Y$.
    If $A\times B$ is measurable w.r.t. the product measure $\mu\times\nu$, is $A$ necessarily measurable w.r.t. $\mu$?
    
    No.
    Consider the measurable spaces $(X=[0,1],\mathcal{A}=\mathcal{L}(\mathbb{R})|_X,\mu)$, $(Y=\mathbb{R},\mathcal{B}=\mathcal{L}(\mathbb{R}),\nu)$, where $\mathcal{L}(\mathbb{R})$ is the $\sigma$-algebra of Lebesgue-measurable sets, $\mathcal{L}(\mathbb{R})|_X=\{E\cap X:E\in\mathcal{L}(\mathbb{R})\}$, and both $\mu$ and $\nu$ are the Lebesgue measure.
    Then defining $\mathcal{R}:=\{A\times B:A\in\mathcal{A},B\in\mathcal{B}\}$ and $\lambda(A\times B):=\mu(A)\nu(B)$ for each $(A\times B)\in\mathcal{R}$ as in Proposition 2, we have that $\mathcal{R}$ is a semiring and $\lambda$ is a premeasure on $\mathcal{R}$.
    By the Carath\'eodory-Hahn Theorem from Chapter 17.5, then the Carath\'eodory measure $\overline\lambda$ induced by $\lambda$ is an extension of $\lambda$.
    Recall That the Carath\'eodory measure is the restriction of the induced outer measure $\lambda^*$ (For $E\in(X\times Y)$, define $\lambda^*(E):=\inf\{\sum_{k=1}^\infty\lambda((A\times B)_k):\{(A\times B)_k\}_k\subset (X\times Y)\text{ countable, covers }E $) to the $\sigma$-algebra $\mathcal{M}$ of $\lambda^*$-measurable sets of $X\times Y$.
    
    Then note that for the Vitali set $V$ (see definition in Chapter 2.6) and any real number $x\in B$, then the set $(V\times \{x\})\subset(X\times Y)$ is $\lambda^*$-measurable because it has outer measure zero: 
    \[
        \lambda^*(V\times\{x\})\le\lambda([0,1]\times\{x\})=\mu([0,1])\cdot\nu(\{x\})=1\cdot0=0.
    \]
    However, the Vitali set $V$ is proved to be not measurable w.r.t. $\lambda$.

\end{enumerate}

% 20.2
\authoredby{untouched}
\section{Lebesgue Measure on Euclidean Space $\mathbb{R}^n$}

% 20.3
\authoredby{inprogress}
\section{Cumulative Distribution Functions and Borel Measures on $\mathbb{R}$}

Let $I=[a,b]$ be a closed, bounded interval of real numbers and $\mathcal{B}(I)$ the collection of Borel subsets of $I$.
We call a finite measure $\mu$ on $\mathcal{B}(I)$ a \textbf{Borel measrure}.
For such a measure, define the function $g_\mu:I\to\mathbb{R}$ by
\[
    g_\mu(x):=\mu[a,x]\quad\text{for all }x\in I.
\]
The function $g_\mu$ is called the \textbf{cumulative distribution function} of $\mu$.

\begin{namedthm*}{Proposition 26}
    Let $\mu$ be a Borel measure on $\mathcal{B}(I)$ and $g_\mu$ its cumulative distribution function.
    Then the measure $\mu$ is absolutely continuous with respect to Lebesgue measure if and only if the function $g_\mu$ is absolutely continuous.
\end{namedthm*}

For a bounded Lebesgue measurable function $f$ on $[a,b]$, denote the Lebesgue integral of $f$ over $[a,b]$ by $\int_{[a,b]}fdm$.
For a bounded function $f$ on $[a,b]$ whose set of discontinuities has Lebesgue measure zero, we proved that the Riemann integral $\int_a^bf(x)dx$ is defined and 
\[
    \int_{[a,b]}fdm=\int_a^bf(x)dx.
\]
There are two generalizations of these integrals, the Lebesgue-Stieltjes integral and Riemann-Stieltjes integral.

Let the function $g:I\to\mathbb{R}$ be increasing and continuous on the right.
For a bounded measurable function $f:I\to\mathbb{R}$, define the \textbf{Lebesgue-Stieltjes integral} of $f$ with respect to $g$ over $[a,b]$, which we denote by $\int_{[a,b]}fdg$, by

\begin{center}
	\textbf{PROBLEMS}
\end{center}
\begin{enumerate}
	\setcounter{enumi}{34}
    \item Prove Propositon 26.
    \item Suppose $f$ is a Bounded Borel measurable function on $[a,b]$ and $g$ is increasing and absolutely continuous on $[a,b]$.
    Prove that if $m$ denotes Lebesgue measure, then 
    \[
        \int_{[a,b]}fdg=\int_{[a,b]}fg'dm.
    \]

    % Because $g$ is absolutely continuous, then by Proposition 26, we have that the measure $\mu$ is a.c. w.r.t. the Lebesgure measure $m$.
    % Notation:
    % \[
    %     \int_{E}dg=\int_{E}d\mu_g=\int_Ed\mu
    % \]
\end{enumerate}

% 20.4
\authoredby{inprogress}
\section{Carath\'eodory Outer Measures and Hausdorff Measures on a Metric Space}

Lebesgue outer measure on Euclidean space $\mathbb{R}^n$ has the property that if $A$ and $B$ are subsets of $\mathbb{R}^n$ and there is a $\delta>0$ for which $\|u-v\|\ge\delta$ for all $u\in A$ and $v\in B$, then 
\[
    \mu_n^*(A\cup B)=\mu_n^*(A)+\mu_n^*(B).
\]
This section is devoted to the study of measures induced by outer measures on a metric space that possesses this property, and a particular class of such measures called Hausdorff measures.

\begin{namedthm*}{Definition}
    Let $(X,\rho)$ be a metric space.
    An outer measure $\mu^*:2^X\to[0,\infty]$ is called a \textbf{Carath\'eodory outer measure} provided whenever $A$ and $B$ are subsets of $X$ for which $\rho(A,B)>0$, then 
    \[
        \mu(A\cup B)=\mu^*(A)+\mu^*(B).
    \] 
\end{namedthm*}

\begin{namedthm*}{Theorem 28}
    Let $\mu^*$ be a Carath\'eodory outer measure on a metric space $(X,\rho)$.
    Then every Borel subset of $X$ is measurable w.r.t. $\mu^*$.
\end{namedthm*}

Consider the metric space $(X,\rho)$.
We define the diameter of a subset $A$ of $X$, $\text{diam}(A)$, by
\[
    \text{diam}(A):=\sup_{u,v\in A}\rho(u,v).
\]
Fix any positive real number $\alpha>0$.
Then we may define a measure $H_\alpha$ on the Borel $\sigma$-algebra $\mathcal{B}(X)$ called the Hausdorff measure on $X$ of dimension $\alpha$.

Fix $\alpha>0$.
Take $\epsilon>0$ and for a subset $E$ of $X$, define 
\[
    H_\alpha^{(\epsilon)}(E):=\inf
    \left\{
        \sum_{k=1}^\infty[\text{diam}(A_k)]^\alpha:\{A_k\}_{k=1}^\infty\subset X\text{ countable, covers }E,\text{diam}(A_k)<\epsilon.
    \right\}
\]
Observe that $H_\alpha^{(\epsilon)}$ increases as $\epsilon$ decreases;
% as $\epsilon$ decreases, as
% \[
%     \sum_{k=1}^\infty[\text{diam}(A_k)]^\alpha
%     \le 
%     \sum_{k=1}^\infty[\text{diam}(B_k)]^\alpha
% \]
see that for $0<\delta<\epsilon$, note that 
\begin{align*}
    \mathcal{C}_\epsilon:=\left\{
        A_k:\text{diam}(A_k)<\epsilon
    \right\}
    \supset
    \left\{
        A_k:\text{diam}(A_k)<\delta
    \right\}
    :=\mathcal{C}_{\delta}
\end{align*}
implies that 
\[
    \inf\mathcal{C}_\epsilon\le\inf\mathcal{C}_\delta.
\]
% in general when $\text{diam}(A_k)<\epsilon$ and $\text{diam}(B_k)<\epsilon+\delta$ for $\epsilon<\epsilon+\delta$
Thus define 
\[
    H_\alpha^*(E):=\sup_{\epsilon>0}H_\alpha^{(\epsilon)}(E)=\lim_{\epsilon\to0}H_\alpha^{(\epsilon)}(E).
\]

\begin{namedthm*}{Proposition 29}
    Let $(X,\rho)$ be a metric space and $\alpha$ a positive real number.
    Then $H_\alpha^*:2^X\to[0,\infty]$ is a Carath\'eodory outer measure.
\end{namedthm*}
We infer from Theorem 28 that $H_\alpha^*$ induces a measure on a $\sigma$-algebra that contains the Borel subsets of $X$.
We denote the restriction of this measure to $\mathcal{B}(X)$ by $H_\alpha$ and call it \textbf{Hausdorff $\alpha$-dimensional measure} on the metric space $X$.

\begin{namedthm*}{Proposition 30}   
    Let $(X,\rho)$ be a metric space, $A$ a Borel subset of $X$, and $\alpha,\beta$ positive real numbers for which $\alpha<\beta$.
    If $H_\alpha(A)<\infty$, then $H_\beta(A)=0$.
\end{namedthm*}
\begin{proof}
    Let $\epsilon>0$.
    Choose $\{A_k\}_k$ to be a countable convering of $A$ by sets of diameter less than $\epsilon$ for which 
    \[
        \sum_{k=1}^\infty[\text{diam}(A_k)]^\alpha\le H_\alpha(A)+1.
    \] 
    Then because $\text{diam}(A_k)^\beta=\text{diam}(A_k)^{\beta-\alpha}\text{diam}(A_k)^\alpha\le\epsilon^{\beta-\alpha}\text{diam}(A_k)^\alpha$, we have
    \[
        H_\beta^{(\epsilon)}(A)\le\sum_{k=1}^\infty[\text{diam}(A_k)]^\beta\le\epsilon^{\beta-\alpha}\sum_{k=1}^\infty[\text{diam}(A_k)]^\alpha
        \le\epsilon^{\beta-\alpha}[H_\alpha(A)+1].
    \]
    Because $H_\alpha(A)+1$ is a fixed (finite) number, take the limit as $\epsilon\to0$ to conclude that $H_\beta(A)=0$.
\end{proof}
For a subset $E$ of $\mathbb{R}^n$, we define the Hausdorff dimension of $E$, $\text{dim}_H(E)$, by
\[
    \text{dim}_H(E):=\inf\{\beta\ge0:H_\beta(E)=0\}.
\]  

\begin{center}
	\textbf{PROBLEMS}
\end{center}
\begin{enumerate}
	\setcounter{enumi}{45}
    \item Show that in the definition of Hausdorff measure one can take the coverings to be by open sets or by closed sets.
    

    \item Show that the set function outer Hausdorff measure $H_\alpha^*$ is countably monotone.
    
    Recall that $H_\alpha^*$ is countably monotone if whenever a set $E\in2^X$ is covered by a countable collection $\{E_k\}_{k=1}^\infty$ of sets in $2^X$, then
	\[
		H_\alpha^*(E)\le\sum_{k=1}^\infty H_\alpha^*(E_k).
	\]
    % Indeed,
    % \[
    % H_\alpha^{(\epsilon)}:=\inf
    % \left\{
    %     \sum_{k=1}^\infty[\text{diam}(A_k)]^\alpha:\{A_k\}_{k=1}^\infty\subset X\text{ countable, covers }E,\text{diam}(A_k)<\epsilon.
    % \right\}
    % \]

    \item In the plane $\mathbb{R}^2$ show that a bounded set may be enclosed in a ball of the same diameter.
    Use this to show that for a bounded subset $A$ of $\mathbb{R}^2$, $H_2(A)\ge4/\pi\cdot\mu_2(A)$, where $\mu_2$ is Lebesgue measure on $\mathbb{R}^2$.
    \item Let $(X,\rho)$ be a metric space and $\alpha>0$.
    For $E\subset X$, define 
    \[
        H_\alpha'(E):=\inf\sum_{k=1}^\infty[\text{diam}(A_k)]^\alpha,
    \]
    where $\{A_k\}_{k=1}^\infty$ is a countable collection of subsets of $X$ that covers $E$: there is no restriction regarding the size of the diameters of the sets in the cover.
    Compare the set functions $H_\alpha'$ and $H_\alpha$.
    \item Show that each Hausdorff measure $H_\alpha$ on Euclidean space $\mathbb{R}^n$ is invariant w.r.t. rigid motions.
    \item Give a direct proof to show that if $I$ is a nontrivial interval in $\mathbb{R}^n$, then $H_n(I)>0$.
    
    
    \item Show that in any metric space, $H_0$ is counting measure.
    
    (needs to be fixed but the idea is there) Note that for any $\epsilon>0$,
    \[
    H_0^{(\epsilon)}(E):=\inf
    \left\{
        \sum_{k=1}^\infty[\text{diam}(A_k)]^0=\sum_{k=1}^\infty 1:\{A_k\}_{k=1}^\infty\subset X\text{ countable, covers }E,\text{diam}(A_k)<\epsilon.
    \right\}
    \]
    Consider $E=\{x_1,\dots,x_n\}$ finite with $n$ elements, and consider any $\epsilon>0$ small enough so that $\rho(x_i,x_j)\ge\epsilon$ for $i\neq j\in[n]$.
    Then any covering $\{A_k\}_{k=1}^\infty$ must have at least $n$ sets $A_k$ of diameter less than $\epsilon$, because the distance between any two points is too large.
    Further, there exists the open sets $\{A_k=(x_k-\epsilon/3,x_k+\epsilon/3)\}_{k=1}^n$ so that $E\subset\cup_{k=1}^nA_k$ and $\sum_{k=1}^n[\text{diam}(A_k)]^0=\sum_{k=1}^n1=n$.
    In particular, for any $\delta>0$, then there exists $\{A_k\}_{k=1}^\infty$ so that
    \[
        H_0^{(\epsilon)}(E)\le n=\sum_{k=1}^n1\le\sum_{k=1}^\infty \text{diam}(A_k)^0<H_0^{(\epsilon)}+\delta,
    \]
    and $n-H_0^{(\epsilon)}(E)<\delta$.
    Taking $\epsilon\to0$ maintains $H_0(E)=n$.
    If $E$ is countably infinite, then in a similar fashion $\sum_{k=1}^\infty1=\infty$ and so $H_0(E)=\infty$.
    If $E$ is uncountably infinite but unbounded, then there exists a countably infinite subset of $E$ and so $H_0(E)=\infty$.
    Finally, if $E$ is uncountably infinite but bounded, then for any $\epsilon>0$, there exists the finite covering $\{A_k\}_{k=1}^p$ that covers $E$.
    However, when $\epsilon\to\infty$ then it must be that $p\to\infty$ and so $H_0(E)=\infty$. 
    
    \item Let $[a,b]$ be a closed, bounded interval of real numbers and $R=\{(x,y)\in\mathbb{R}^2:a\le x\le b,y=0\}$.
    Show that $H_2(R)=0$.
    Then show that $H_1(R)=b-a$.
    Conclude that the Hausdorff dimension of $R$ is 1.
    \item Let $f:[a,b]\to\mathbb{R}$ be a continuous bounded function on the closed, bounded interval $[a,b]$ that has continuous bounded derivative on the open interval $(a,b)$.
    Consider the graph $G$ of $f$ as a subset of the plane.
    Show that $H_1(G)=\int_a^b\sqrt{1+|f'(x)|^2}dx$.
    \item Let $J$ be an interval in $\mathbb{R}^n$, each of whose sides has length $1$.
    Define $\gamma_n:=H_n(J)$.
    Show that if $I$ is any bounded interval in $\mathbb{R}^n$, then $H_n(I)=\gamma_n\cdot\mu_n(I)$.
    From this infer, using the uniqueness assertion of the Carath\'eodory-Hahn Theorem, that $H_n=\gamma_n\cdot\mu_n$ on the Borel subsets of $\mathbb{R}^n$.
\end{enumerate}
