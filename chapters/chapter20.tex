% Chapter 20
\authoredby{inprogress}
\chapter{The Construction of Particular Measures}

% 20.1
\authoredby{inprogress}
\section{Product Measures: The Theorems of Fubini and Tonelli}

Throughout this section $(X,\mathcal{A},\mu)$ and $Y,\mathcal{B},\nu$ are two reference measure spaces.
Consider the Cartesian product $X\times Y$ of $X$ and $Y$.
If $A\subset X$ and $B\subset Y$, we call $A\times B$ a rectangle.
If $A\in\mathcal{A}$ and $B\in\mathcal{B}$, we call $A\times B$ a \textbf{measurable rectangle}.

\begin{namedthm*}{Proposition 2}
    Let $\mathcal{R}$ be the collection of measurable rectangles in $X\times Y$ and for a measurable rectangle $A\times B$, define
    \[
        \lambda(A\times B):=\mu(A)\cdot\nu(B).
    \]
    Then $\mathcal{R}$ is a semiring and $\lambda:\mathcal{R}\to[0,\infty]$ is a premeasure.
\end{namedthm*}

\begin{namedthm*}{Definition}
    Let $(X,\mathcal{A},\mu)$ and $(Y,\mathcal{B},\nu)$ be measure spaces, $\mathcal{R}$ the collection of measurable rectangles contained in $X\times Y$, and $\lambda$ the premeasure defined on $\mathcal{R}$ by
    \[
        \lambda(A\times B):=\mu(A)\cdot\nu(B)\quad\text{for }A\times B\in\mathcal{R}.
    \]
    By the \textbf{product measure} $\lambda=\mu\times\nu$ we mean the Carath\'eodory extension of $\lambda:\mathcal{R}\to[0,\infty]$ defined on the $\sigma$-algebra of $(\mu\times\nu)^*$-measurable subsets of $X\times Y$.
\end{namedthm*}

\begin{center}
	\textbf{PROBLEMS}
\end{center}
\begin{enumerate}
	\setcounter{enumi}{0}
    \item Let $A\subset X$ and let $B$ be $\nu$-measurable subset of $Y$.
    If $A\times B$ is measurable w.r.t. the product measure $\mu\times\nu$, is $A$ necessarily measurable w.r.t. $\mu$?
    
    No.
    Consider the measurable spaces $(X=[0,1],\mathcal{A}=\mathcal{L}(\mathbb{R})|_X,\mu)$, $(Y=\mathbb{R},\mathcal{B}=\mathcal{L}(\mathbb{R}),\nu)$, where $\mathcal{L}(\mathbb{R})$ is the $\sigma$-algebra of Lebesgue-measurable sets, $\mathcal{L}(\mathbb{R})|_X=\{E\cap X:E\in\mathcal{L}(\mathbb{R})\}$, and both $\mu$ and $\nu$ are the Lebesgue measure.
    Then defining $\mathcal{R}:=\{A\times B:A\in\mathcal{A},B\in\mathcal{B}\}$ and $\lambda(A\times B):=\mu(A)\nu(B)$ for each $(A\times B)\in\mathcal{R}$ as in Proposition 2, we have that $\mathcal{R}$ is a semiring and $\lambda$ is a premeasure on $\mathcal{R}$.
    By the Carath\'eodory-Hahn Theorem from Chapter 17.5, then the Carath\'eodory measure $\overline\lambda$ induced by $\lambda$ is an extension of $\lambda$.
    Recall That the Carath\'eodory measure is the restriction of the induced outer measure $\lambda^*$ (For $E\in(X\times Y)$, define $\lambda^*(E):=\inf\{\sum_{k=1}^\infty\lambda((A\times B)_k):\{(A\times B)_k\}_k\subset (X\times Y)\text{ countable, covers }E $) to the $\sigma$-algebra $\mathcal{M}$ of $\lambda^*$-measurable sets of $X\times Y$.
    
    Then note that for the Vitali set $V$ (see definition in Chapter 2.6) and any real number $x\in B$, then the set $(V\times \{x\})\subset(X\times Y)$ is $\lambda^*$-measurable because it has outer measure zero: 
    \[
        \lambda^*(V\times\{x\})\le\lambda([0,1]\times\{x\})=\mu([0,1])\cdot\nu(\{x\})=1\cdot0=0.
    \]
    However, the Vitali set $V$ is proved to be not measurable w.r.t. $\lambda$.

\end{enumerate}

% 20.2
\authoredby{untouched}
\section{Lebesgue Measure on Euclidean Space $\mathbb{R}^n$}

% 20.3
\authoredby{inprogress}
\section{Cumulative Distribution Functions and Borel Measures on $\mathbb{R}$}
For a bounded Lebesgue measurable function $f$ on $[a,b]$, denote the Lebesgue integral of $f$ over $[a,b]$ by $\int_{[a,b]}fdm$.
For a bounded function $f$ on $[a,b]$ whose set of discontinuities has Lebesgue measure zero, we proved that the Riemann integral $\int_a^bf(x)dx$ is defined and 
\[
    \int_{[a,b]}fdm=\int_a^bf(x)dx.
\]
There are two generalizations of these integrals, the Lebesgue-Stieltjes integral and Riemann-Stieltjes integral.

Let the function $g:I\to\mathbb{R}$ be increasing and continuous on the right.
For a bounded measurable function $f:I\to\mathbb{R}$, define the \textbf{Lebesgue-Stieltjes integral} of $f$ with respect to $g$ over $[a,b]$, which we denote by $\int_{[a,b]}fdg$, by

% 20.4
\authoredby{untouched}
\section{Carath\'eodory Outer Measures and Hausdorff Measures on a Metric Space}
