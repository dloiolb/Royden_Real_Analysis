% Chapter 18
\authoredby{inprogress}
\chapter{Integration Over General Measure Spaces}

% 18.1
\authoredby{inprogress}
\section{Measurable Functions}
Consider the measurable space $(X,\mathcal{M})$. 
For an extended real valued function $f$ of $X$ and a measurable subset $E$ of $X$, the restriction of $f$ to both $E$ and $X\setminus E$ are measurable iff $f$ is measurable on $X$.
\begin{namedthm*}{Proposition 3}
    Let $(X,\mathcal{M},\mu)$ be a complete measure space and $X_0$ be a measurable subset of $X$ for which $\mu(X\setminus X_0)=0$.
    Then an extended real valued function $f$ on $X$ is measurable iff its restriction to $X_0$ is measurable.
    In particular, if $g$ and $h$ are extended real valued functions on $X$ for which $g=h$ a.e. on $X$, then $g$ is measurable iff $h$ is measurable.
\end{namedthm*}
\begin{namedthm*}{Theorem 6}
    Let $(X,\mathcal{M},\mu)$ be a measure space and $\{f_n\}$ a sequence of measurable functions on $X$ for which $\{f_n\}\to f$ pointwise a.e. on $X$.
    If either the measure space $(X,\mathcal{M},\mu)$ is complete or the convergence is pointwise on all of $X$, then $f$ is measurable.
\end{namedthm*}
\begin{namedthm*}{Corollary 7}
    Let $(X,\mathcal{M},\mu)$ be a measure space and $\{f_n\}$ be a sequence of measurable function on $X$.
    Then the following functions are measurable:
    \[
        \sup\{f_n\},\ \inf\{f_n\},\ \limsup\{f_n\},\ \liminf\{f_n\}. 
    \]
\end{namedthm*}
\begin{namedthm*}{Egoroff's Theorem}
    Let $(X,\mathcal{M},\mu)$ be a finite measure space and $\{f_n\}$ a sequence of measurable functions on $X$ that converges pointwise a.e. on $X$ to a function $f$ that is finite a.e. on $X$.
    Then for each $\epsilon>0$, there is a measurable subset $X_\epsilon$ of $X$ for which 
    \[
        \{f_n\}\to f\text{ uniformly on }X_\epsilon\text{ and }\mu(X\setminus  X_\epsilon)<\epsilon.
    \]
\end{namedthm*}
\begin{center}
	\textbf{PROBLEMS}
\end{center}
In the following problems $(X,\mathcal{M},\mu)$ is a reference measure space and measurable means with respect to $\mathcal{M}$.
\begin{enumerate}
	\setcounter{enumi}{0}
    \item Show that an extended real valued function on $X$ is measurable iff $f^{-1}\{\infty\}$ and $f^{-1}\{-\infty\}$ are measurable and so is $f^{-1}(E)$ for every Borel set of real numbers.\\
    \\Let $f$ be an extended real valued function on $X$.\\
    \\$(\implies)$ Suppose that $f$ is measurable. 
    \\Then the set
    \[
        f^{-1}\{\infty\}=\{x\in X\mid f(x)=\infty\}=\bigcap_{n=1}^\infty\{x\in X\mid f(x)>n\},
    \]
    is measurable because it is a countable intersection of measurable sets.
    \\Similarly the set
    \[
        f^{-1}\{-\infty\}=\{x\in X\mid f(x)=-\infty\}=\bigcap_{n=1}^\infty\{x\in X\mid f(x)<n\},
    \]
    is measurable because it is a countable intersection of measurable sets.
    \\By definition of measurable function we have that the set
    \[
        \{x\in X\mid f(x)>a\}=\{x\in X\mid f(x)\in(a,\infty)\}
    \]
    is measurable for any real $a$.
    From Chapter 2 Problem 11, we have that that if a $\sigma$-algebra of subsets of R contains intervals of the form $(a,\infty)$, then it contains all intervals.
    Therefore by the properties of a $\sigma$-algebra, it must contain all open sets (Chapter 1 Proposition 9 - Every nonempty open set is the union of a countable, disjoint collection of open intervals).
    Then because the collection of Borel sets is the smallest $\sigma$-algebra that contains all of the open sets of real numbers, we have
    \[
        f^{-1}(E)\text{ is measurable for any Borel set }E.
    \]
    \\$(\impliedby)$ Suppose that the sets $f^{-1}\{\infty\}$ and $f^{-1}\{-\infty\}$ are measurable, and that $f^{-1}(E)$ is measurable for any Borel set $E$ of real numbers.
    \\Fix any real number $c$.
    \\The collection of Borel sets contain all intervals of the form $(c,\infty)$ so that the set
    \[
        \{x\in X\mid f(x)> c\}=\{x\in X\mid f(x)\in(c,\infty)\}=f^{-1}((c,\infty))
    \]
    is measurable.
    \\Therefore $f$ is a measurable function.
    \item Suppose $(X,\mathcal{M},\mu)$ is not complete.
    Let $E$ be a subset of a set of measure zero that does not belong to $\mathcal{M}$.
    Let $f=0$ on $X$ and $g=\chi_E$.
    Show that $f=g$ a.e. on $X$ while $f$ is measurable and $g$ is not.
    \item Suppose $(X,\mathcal{M},\mu)$ is not complete.
    Show that there is a sequence $\{f_n\}$ of measurable functions on $X$ that converges pointwise a.e. on $X$ to a function $f$ that is not measurable.
    \item Let $E$ be a measurable subset of $X$ and $f$ an extended real-valued function on $X$.
    Show that $f$ is measurable iff its restrictions to $E$ and $X\setminus E$ are measurable.
    \item Show that an extended real valued function $f$ on $X$ is measurable iff for each rational number $c$, $\{x\in X\mid f(x)<c\}$ is a measurable set.\\
    \\Let $f$ be an extended real valued function on $X$.\\
    \\$(\implies)$ Suppose that $f$ is measurable.
    \\Then trivially for any rational number $c$, $\{x\in X\mid f(x)<c\}$ is a measurable set by definition of measurable function.\\
    \\$(\impliedby)$ Suppose that for each rational number $c$, $\{x\in X\mid f(x)<c\}$ is a measurable set.
    \\Let $a$ be any real number.
    \\Then for each natural number $n$, by density of the rationals in the reals there exists a rational $c_n$ such that 
    \[
        a_n-\frac{1}{n}<c_n< a_n,
    \]
    and we have that the set $\{x\in X\mid f(x)<c_n\}$ is measurable.
    \\Then we have $\bigcup_{n=1}^\infty[\infty,c_n)=[\infty,a)$, so that we have the set
    \begin{align*}
        \bigcup_{n=1}^\infty\{x\in X\mid f(x)<c_n\}
        &=\bigcup_{n=1}^\infty\{x\in X\mid f(x)\in[\infty,c_n)\}\\
        &=\{x\in X\mid f(x)\in[\infty,a)\},\\
        &=\{x\in X\mid f(x)<a\},
    \end{align*}
    which is measurable because it is the countable union of measurable sets.
    \\Therefore $f$ is a measurable function.
    \item Consider two extended real valued measurable functions $f$ and $g$ on $X$ that are finite a.e. on $X$.
    Define $X_0$ to be the set of points in $X$ at which both $f$ and $g$ are finite.
    Show that $X_0$ is measurable and $\mu(X\setminus X_0)=0$.\\
    \\We have that $f$ and $g$ are finite a.e. on $X$, which means that there exist (measurable) subsets $X_f$, $X_g$ of $X$, both of measure zero, where the property holds on $X\setminus X_f$ and $X\setminus X_g$ respectively.
    \\Then $X_f$ and $X_g$ are measurable imply that $X\setminus X_f$ and $X\setminus X_g$ are also measurable by the properties of a $\sigma$-algebra.
    \\Therefore we have that the set
    \[
        X_0=[X\setminus X_f]\cap[X\setminus X_g]=\{x\in X\mid f(x)\text{ is finite},g(x)\text{ is finite}\}
    \]
    is measurable because it is the intersection of measurable sets.
    \\We see that 
    \[
        X\setminus X_0=X\cap([X^c\cup X_f]\cup[X^c\cup X_g])=X_f\cup X_g,
    \]
    and we use countable monotonicity to see that
    \[
        \mu(X\setminus X_0)=\mu(X_f\cup X_g)\le\mu(X_f)+\mu(X_g)=0+0.
    \]
    \item Let $X$ be a nonempty set.
    Show that every extended real valued function on $X$ is measurable w.r.t. the measurable space $(X,2^X)$.
    \begin{enumerate}[(i)]
        \item Let $x_0$ belong to $X$ and $\delta_{x_0}$ be the Dirac measure at $x_0$ on $2^X$.
        Show that two function on $X$ are equal a.e. $[\delta_{x_0}]$ iff they take the same value at $x_0$.
        \item Let $\eta$ be the counting measure on $2^X$.
        Show that two functions on $X$ are equal a.e. $[\eta]$ iff they take the same value at every point in $X$.
    \end{enumerate}
    \item Let $X$ be a topological space and $\mathcal{B}(X)$ be the smallest $\sigma$-algebra containing the topology on $X$.
    $\mathcal{B}(X)$ is called the Borel $\sigma$-algebra associated with the topological space $X$.
    Show that any continuous real valued function on $X$ is measurable w.r.t. the Borel measurable space $(X,\mathcal{B}(X))$.
    \item If a real valued function on $\mathbb{R}$ is measurable w.r.t. the $\sigma$-algebra of Lebesgue measurable sets, is it necessarily measurable w.r.t. the Borel measurable space $(\mathbb{R},\mathcal{B}(\mathbb{R}))$?
    \item Check that the proofs of Proposition 1 and Theorem 4 follow from the proofs of the corresponding results in the case of Lebesgue measure on the real line.
    \item Prove Corollary 7.\\
    \\Let $(X,\mathcal{M},\mu)$ be a measure space and $\{f_n\}$ be a sequence of measurable function on $X$.
    \\\begin{enumerate}[(i)]
        \item $f(x):=\sup_{n\in\mathbb{N}}\{f_n(x)\}$
        \\Fix any real number $c$.
        \\(1) Let $y\in\{x\in X\mid f(x)>c\}$.
        Then $f(y)>c$.
        \\By definition of supremum, there exists an index $k$ such that 
        \[
            f(y)\ge f_k(y)>c,
        \]
        and therefore $y\in\{x\in X\mid f_k(x)>c\}$ for some $k$, which implies
        \[
            \{x\in X\mid f(x)>c\}\subseteq\bigcup_{n=1}^\infty\{x\in X\mid f_n(x)>c\}.\tag{1}
        \]
        (2)Let $y'\in\bigcup_{n=1}^\infty\{x\in X\mid f_n(x)>c\}$.
        \\Then there exists an index $k$ such that $y'\in\{x\in X\mid f_k(x)>c\}$.
        \\By definition of supremum, we have 
        \[
            f(y')\ge f_k(y')>c,
        \]
        and therefore $y'\in\{x\in X\mid f(x)>c\}$, which implies
        
        \[
            \{x\in X\mid f(x)>c\}\supseteq\bigcup_{n=1}^\infty\{x\in X\mid f_n(x)>c\}\tag{2}
        \]
        Then by (1) and (2),
        \[
            \{x\in X\mid f(x)>c\}=\bigcup_{n=1}^\infty\{x\in X\mid f_n(x)>c\},
        \]
        which is measurable because it is the countable union of measurable sets.
        % \\We may also prove that $\{x\in X\mid f(x)<c\}=\bigcap_{n=1}^\infty\{x\in X\mid f_n(x)<c\}$ in a similar way.
        \item $f(x):=\inf_{n\in\mathbb{N}}\{f_n(x)\}$
        \\Fix any real number $c$.
        \\(1) Let $y\in\{x\in X\mid f(x)\ge c\}$.
        Then $f(y)\ge c$.
        \\By definition of infimum, for all indices $n$, we have that
        \[
            c\le f(y)\le f_n(y),
        \]
        and therefore $y\in\{x\in X\mid f_n(x)\ge c\}$ for all $n$, which implies
        \[
            \{x\in X\mid f(x)\ge c\}\subseteq\bigcap_{n=1}^\infty\{x\in X\mid f_n(x)\ge c\}.\tag{1}
        \]
        (2) Let $y'\in\bigcap_{n=1}^\infty\{x\in X\mid f_n(x)\ge c\}$.
        \\Then for all indices $n$, we have $y'\in\{x\in X\mid f_n(x)\ge c\}$ so that $c$ is a lower bound to the set $\{f_n(y')\}_{n\in\mathbb{N}}$.
        \\Then by definition of infimum we have 
        \[
            c\le f(y'),
        \]
        and therefore $y'\in\{x\in X\mid f(x)\ge c\}$, which implies
        \[
            \{x\in X\mid f(x)\ge c\}\supseteq\bigcap_{n=1}^\infty\{x\in X\mid f_n(x)\ge c\}\tag{2}
        \]
        Then by (1) and (2),
        \[
            \{x\in X\mid f(x)\ge c\}=\bigcap_{n=1}^\infty\{x\in X\mid f_n(x)\ge c\},
        \]
        which is measurable because it is the countable intersection of measurable sets.
        % \\Fix any real number $c$.
        % \\(1) Let $y\in\{x\in X\mid f(x)<c\}$.
        % Then $f(y)<c$.
        % \\By definition of infimum, there exists an index $k$ such that 
        % \[
        %     f(y)\le f_k(y)<c,
        % \]
        % and therefore $y\in\{x\in X\mid f_k(x)<c\}$, which implies
        % \[
        %     \{x\in X\mid f(x)<c\}\subseteq\bigcup_{n=1}^\infty\{x\in X\mid f_n(x)<c\}.\tag{1}
        % \]
        % (2) Let $y'\in\bigcup_{n=1}^\infty\{x\in X\mid f_n(x)<c\}$.
        % \\Then there exists an index $k$ such that $y'\in\{x\in X\mid f_k(x)<c\}$.
        % \\By definition of infimum, we have 
        % \[
        %     f(y')\le f_k(y')<c,
        % \]
        % and therefore $y'\in\{x\in X\mid f(x)<c\}$, which implies
        
        % \[
        %     \{x\in X\mid f(x)<c\}\supseteq\bigcup_{n=1}^\infty\{x\in X\mid f_n(x)<c\}\tag{2}
        % \]
        % Then by (1) and (2),
        % \[
        %     \{x\in X\mid f(x)<c\}=\bigcup_{n=1}^\infty\{x\in X\mid f_n(x)<c\},
        % \]
        % which is measurable because it is the countable union of measurable sets.\\
        \item $f(x):=\limsup_n\{f_n(x)\}=\lim_{n\to\infty}\sup_{k\ge n}\{f_n(x)\}=\inf_{n\in\mathbb{N}}\sup_{k\ge n}\{f_n(x)\}$
        \\Fix any real number $c$.
        \\Pulling from (i), with a small modification, we have that the function $g_n(x):=\sup_{k\ge n}\{f_k(x)\}$ is measurable for each $n\in\mathbb{N}$.
        \\That is, the set
        \[
            \{x\in X\mid g_n(x)> c\}=\bigcup_{k=n}^\infty\{x\in X\mid f_k(x)> c\}
        \]
        is measurable, and thus each function $g_n$ is measurable.
        \\Then using the same process as (ii), the set
        \[
            \{x\in X\mid f(x)\ge c\}=\bigcap_{n=1}^\infty\{x\in X\mid g_n(x)\ge c\}%=\bigcap_{n=1}^\infty\bigcup_{k=n}^\infty\{x\in X\mid f_k(x)\ge c\}
        \]
        is measurable because it is a countable intersection of measurable sets.
        % \\(The first set equality can be proved using the same process as (ii).)
        \item $f(x):=\liminf_n\{f_n(x)\}=\lim_{n\to\infty}\inf_{k\ge n}\{f_n(x)\}=\sup_{n\in\mathbb{N}}\inf_{k\ge n}\{f_n(x)\}$
        \\Fix any real number $c$.
        \\Pulling from (ii), with modification, we have that the function $g_n(x):=\inf_{k\ge n}\{f_k(x)\}$ is measurable for each $n\in\mathbb{N}$.
        \\That is, the set
        \[
            \{x\in X\mid g_n(x)\ge c\}=\bigcap_{k=n}^\infty\{x\in X\mid f_k(x)\ge c\}
        \]
        is measurable, and thus each function $g_n$ is measurable.
        \\Then using the same process as (i), the set
        \[
            \{x\in X\mid f(x)> c\}=\bigcup_{n=1}^\infty\{x\in X\mid g_n(x)> c\}%=\bigcup_{n=1}^\infty\bigcap_{k=n}^\infty\{x\in X\mid f_k(x)> c\}
        \]
        is measurable because it is a countable union of measurable sets.
        % \\(The first set equality can be proved using the same process as (i).)

        % The strict vs unstrict inequality might be a problem for (iii) and (iv) in the second part.
    \end{enumerate}
    \item Prove Egoroff's Theorem.
    Is Egoroff's Theorem true in the absence of the assumption that the limit function is finite a.e.?
    \item Let $\{f_n\}$ be a sequence of real valued measurable functions on $X$ such that, for each natural number $n$, $\mu\{x\in X\mid |f_n(x)-f_{n+1}(x)|>1/2^n\}<1/2^n$.
    Show that $\{f_n\}$ is pointwise convergent a.e. on $X$. 
    (Hint: Use the Borel-Cantelli Lemma.)
    \item Under the assumptions of Egoroff's Theorem, show that $X=\bigcup_{k=0}^\infty X_k$, where each $X_k$ is measurable, $\mu(X_0)=0$ and, for $k\ge1$, $\{f_n\}$ converges uniformly to $f$ on $X_k$.
    \item A sequence $\langle f_n\rangle$ of measurable real-valued functions on $X$ is said to \textbf{converge in measure} to a measurable function $f$ provided that for each $\eta>0$,
    \[
        \lim_{n\to\infty}\mu\{x\in X\mid|f_n(x)-f(x)>\eta|\}=0.
    \]
    A sequence $\langle f_n\rangle$ of measurable functions is said to \textbf{Cauchy in measure} provided that for each $\epsilon>0$ and $\eta>0$, there is an index $N$ such that for each $m,n\ge N$,
    \[
        \mu\{x\in X\mid|f_n(x)-f_m(x)>\eta|\}<\epsilon.
    \]
    \begin{enumerate}[(i)]
        \item Show that if $\mu(X)<\infty$ and $\{f_n\}$ converges pointwise a.e. on $X$ to a measurable function $f$, then $\{f_n\}$ converges to $f$ in measure.
        (Hint: Use Egoroff's Theorem.)
        \item Show that if $\{f_n\}$ converges to $f$ in measure, then there is a subsequence of $\{f_n\}$ that converges pointwise a.e. on $X$ to $f$.
        (Hint: Use the Borel-Cantelli Lemma.)
        \item Show that if $\{f_n\}$ is Cauchy in measure, then there is a measurable function $f$ to which $\{f_n\}$ converges in measure.
    \end{enumerate}
    \item Assume $\mu(X)<\infty$.
    Show that $\{f_n\}\to f$ in measure iff each subsequence of $\{f_n\}$ has a further subsequence that converges pointwise a.e. on $X$ to $f$.
    Use this to show that for two sequences that converge in measure, the product sequence also converges in measure to the product of the limits.
\end{enumerate}

% 18.2
\authoredby{untouched}
\section{Integration of Nonnegative Measurable Functions}
\begin{center}
	\textbf{PROBLEMS}
\end{center}
\begin{enumerate}
	\setcounter{enumi}{16}
    \item 
\end{enumerate}

% 18.3
\section{Integration of General Measurable Functions}
\begin{center}
	\textbf{PROBLEMS}
\end{center}
\begin{enumerate}
	\setcounter{enumi}{26}
    \item 
\end{enumerate}

% 18.4
\authoredby{inprogress}
\section{The Radon-Nikodym Theorem}
Let $(X,\mathcal{M})$ be a measurable space.
For $\mu$ a measure on $(X,\mathcal{M})$ and $f$ a nonnegative function on $X$ that is measurable w.r.t. $\mathcal{M}$, define the set function $\nu$ on $\mathcal{M}$ by
\[
    \nu(E)=\int_Efd\mu\text{ for all }E\in\mathcal{M}.
\]
\begin{center}
	\textbf{PROBLEMS}
\end{center}
\begin{enumerate}
	\setcounter{enumi}{48}
    \item 
\end{enumerate}

% 18.5
\authoredby{untouched}
\section{The Nikodym Metric Space: The Vitali-Hahn-Saks Theorem}
\begin{center}
	\textbf{PROBLEMS}
\end{center}
\begin{enumerate}
	\setcounter{enumi}{60}
    \item 
\end{enumerate}
