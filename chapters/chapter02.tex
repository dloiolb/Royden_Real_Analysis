% Chapter 2
\chapter{Lebesgue Measure}

% 2.1
\section{Introduction}
In this chapter we construct a collection of sets called \textbf{Lebesgue measurable sets}, and a set function of this collection called \textbf{Lebesgue measure}, denoted by $m$. The collection of Lebesgue measurable sets is a $\sigma$-algebra which contains all open sets and all closed sets. The set function $m$ possesses the following three properties:
\begin{namedthm*}{The measure of an interval is its length}
Each nonempty interval $I$ is Lebesgue measurable and 
\[
m(I) = \ell(I).
\]
\end{namedthm*}
\begin{namedthm*}{Measure is translation invariant}
If $E$ is Lebesgue measurable and $y$ is any number then the translate of $E$ by $y$, $E+y = \{x+y \ |\ x \in E\}$, also is Lebesgue measurable and
\[
m(E+y) = m(E).
\]
\end{namedthm*}
\begin{namedthm*}{Measure is countably additive over countable disjoint unions of sets}
If $\{E_k\}_{k=1}^\infty$ is a countable disjoint collection of Lebesgue measurable sets, then
\[
m(\bigcup_{k=1}^\infty E_k) = \sum_{k=1}^\infty m(E_k).
\]
\end{namedthm*}
It is not possible to construct a set function that possesses the above three properties and is defined for all sets of real numbers (See Vitali sets).
We first construct a set function called \textbf{outer measure}, denoted by $m^*$, such that: 
\begin{enumerate}[label=(\roman*),align=left]
	\item the outer measure of an interval is its length,
	\item outer measure is translation invariant,
	\item outer measure is countably subadditive.
\end{enumerate}
Then the Lebesgue measure $m$ is the restriction of $m^*$ to the Lebesgue measurable sets.

\begin{center}
	\textbf{PROBLEMS}
\end{center}
In the first three problems, let $m$ be a set function defined for all sets in a $\sigma$-algebra $\mathcal{A}$ with values in $[0,\infty]$. Assume $m$ is countably additive over countable disjoint collections of sets in $\mathcal{A}$.
\begin{enumerate}
	\setcounter{enumi}{0}
	\item Prove that if $A$ and $B$ are two sets in $\mathcal{A}$ with $A \subseteq B$, then $m(A) \le m(B)$. This property is called \textit{monotonicity}.\par
	$A \subseteq B \implies B = A \cup (B \cap A^c),$ where $A \cap (B \cap A^c) = \emptyset$. The set $(B \cap A^c)$ is measurable because $A^c$ is measurable and countable intersection is measurable, so $m(B) = m(A \cup (B \cap A^c)) = 	m(A) + m(B \cap A^c)$ by countable additivity, and thus $m(B) \ge m(A)$.
	\item Prove that if there is a set $A$ in the collection $\mathcal{A}$ for which $m(A) < \infty$, then $m(\emptyset) = 0$.\par
	We have $A\cap\emptyset = \emptyset$ and $A\cup\emptyset = A$. 
    \begin{align*}
        m(A)&=m(A\cup\emptyset)\\
        m(A)&=m(A)+m(\emptyset)&&\text{ by disjoint additivity}\\
        0&=m(\emptyset).
    \end{align*}
	\item Let $\{E_k\}_{k=1}^\infty$ be a countable collection of sets in $\mathcal{A}$. Prove that $m(\bigcup_{k=1}^\infty E_k) \le \sum_{k=1}^\infty m(E_k).$
    For any two measurable sets $A,B$, we have $A\cup B=(A\setminus B)\cup(B)$.
    By disjoint additivity,
    \[
        m(A\cup B) = m(A\setminus B)+m(B)
    \]
    Now, by problem 1, $(A\setminus B)\subseteq A$ implies that $m(A\setminus B)\le m(A)$.
    Therefore
    \[
        m(A\cup B) \le m(A)+m(B).
    \]
    \item A set function $c$, defined on all subsets of $\mathbb{R}$, is defined as follows.
	Define $c(E)$ to be $\infty$ if $E$ has infinitely many members and $c(E)$ to be equal to the number of elements in $E$ if $E$ is finite; define $c(\emptyset)=0$. Show that $c$ is a countably additive and translation invariant set function. This set function is called the \textbf{counting measure}.\\
    Suppose $E=\{x_1,\cdots,x_n\}$.\\
    Then $m(E)=n$. For any real number $y$, $y+E = \{y+x_1,\cdots,y+x_n\}$, so $m(y+E)=n$.\\
    Suppose $E$ has infinitely many members.\\
    Then $y+E$ has infinitely members as well, so $m(E)=m(y+E)=\infty$.\\
    Let $\{E_k\}_{k=1}^\infty$ be a disjoint collection of sets of real numbers.
    In the case that there exists an $E_k$ with infinitely many members, then the countable additivity is clear.
    \\In the case that all sets $E_k$ are finite, for any two sets $E_i,E_j$:\\
    $E_i = \{x_1,\cdots,x_n\}$\\
    $E_j = \{y_1,\cdots,y_m\}$\\  
    Then $E_i\cup E_j = \{x_1,\cdots,x_n,y_1,\cdots,y_m\}$ and $m(E_i\cup E_j)=n+m=m(E_i) +m(E_j)$.
\end{enumerate}

% 2.2
\section{Lebesgue Outer Measure}

\begin{center}
	\textbf{PROBLEMS}
\end{center}
\begin{enumerate}
	\setcounter{enumi}{4}
	\item By using properties of outer measure, prove that the interval $[0,1]$ is not countable.
	\item Let $A$ be the set of irrational numbers in the interval $[0,1]$. Prove that $m^*(A)=1$.
	\item A set of real numbers is said to be a $G_\delta$ set provided it is the intersection of a countable collection of open sets.
	Show that for any bounded set $E$, there is a $G_\delta$ set $G$ for which
	\[E\subseteq G \ \text{and}\ m^*(G)=m^*(E).\]
	\item Let $B$ be the set of rational numbers in the interval $[0,1]$, and let $\{I_k\}_{k=1}^n$ be a finite collection of open intervals that covers $B$.
	Prove that $\textstyle \sum_{k=1}^n m^*(I_k) \ge 1$.
	\item Prove that if $m^*(A)=0$, then $m^*(A\cup B) = m^*(B)$.
	\item Let $A$ and $B$ be bounded sets for which there is an $\alpha >0$ such that $|a-b| \ge \alpha$ for all $a \in A, b \in B$.
	Prove that $m^*(A \cup B) = m^*(a)+m^*(B)$.
\end{enumerate}



% 2.3
\section{The $\sigma$-Algebra of Lebesgue Measurable Sets}

\begin{center}
	\textbf{PROBLEMS}
\end{center}
\begin{enumerate}
	\setcounter{enumi}{10}
	\item Prove that if a $\sigma$-algebra of subsets of $\mathbb{R}$ contains intervals of the form $(a,\infty)$, then it contains all intervals.\\
	\item Show that every interval is a Borel set.
	\item Show that 
	\begin{enumerate}[label=(\roman*),align=left]
        \item the translate of an $F_\sigma$ set is also $F_\sigma$,
        \item the translate of a $G_\delta$ set is also $G_\delta$,
        \item the translate of a set of measure zero also has measure zero.
    \end{enumerate}
    \item Show that if a set $E$ has positive outer measure, then there is a bounded subset of $E$ that also has positive outer measure.
    \item Show that if $E$ has finite measure and $\epsilon>0$, then $E$ is the disjoint union of a finite number of measurable sets, each of which has measure at most $\epsilon$.
\end{enumerate}

% 2.4
\section{Outer and Inner Approximation of Lebesgue Measurable Sets}

\begin{namedthm*}{Theorem 11}
	Let $E$ be any set of real numbers. Then each of the following four assertions is equivalent to the measurability of $E$.\\
	(Outer Approximation by Open Sets and $G_\delta$ sets)
	\begin{enumerate}[label=(\roman*),align=left]
        \item For each $\epsilon>0$, there is an open set $\mathcal{O}$ containing $E$ for which $m^*(\mathcal{O}\setminus E)<\epsilon$.
        \item There is a $G_\delta$ set $G$ containing $E$ for which $m^*(G\setminus E)=0$. 
    \end{enumerate}
	(Inner Approximation by Closed Sets and $F_\sigma$ sets)
	\begin{enumerate}[label=(\roman*),align=left]
        \setcounter{enumi}{2}
		\item For each $\epsilon>0$, there is a closed set $F$ contained in $E$ for which $m^*(E\setminus F)<\epsilon$.
        \item There is a $F_\sigma$ set $F$ contained in $E$ for which $m^*(E\setminus F)=0$.
    \end{enumerate}
\end{namedthm*}

\begin{center}
	\textbf{PROBLEMS}
\end{center}
\begin{enumerate}
	\setcounter{enumi}{15}
	\item Complete the proof of Theorem 11 by showing that measurability is equivalent to (iii) and also equivalent to (iv).
	\item Show that a set $E$ is measurable iff for each $\epsilon>0$, there is a closed set $F$ and open set $\mathcal{O}$ for which $F\subseteq E\subseteq \mathcal{O}$ and $m^*(\mathcal{O}\setminus F)<\epsilon$.
	\item Let $E$ have finite outer measure. Show that there is a $G_\delta$ set $G\supseteq E$ with $m(G)=m^*(E)$.
	Show that $E$ is measurable iff there is an $F_\sigma$ set $F \subseteq E$ with $m(F)=m^*(E)$. 
	\item Let $E$ have finite outer measure.
\end{enumerate}

% 2.5
\section{Countable Additivity, Continuity, and the Borel-Cantelli Lemma}

% 2.6
\section{Nonmeasurable Sets}

% 2.7
\section{The Cantor Set and the Cantor-Lebesgue Function}
