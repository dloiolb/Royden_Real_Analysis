% Chapter 2
\chapter{Lebesgue Measure}

% 2.1
\section{Introduction}
In this chapter we construct a collection of sets called \textbf{Lebesgue measurable sets}, and a set function of this collection called \textbf{Lebesgue measure}, denoted by $m$. 
(A \textit{set function} is a function that associates an extended real number to each set in a collection of sets.)
The collection of Lebesgue measurable sets is a $\sigma$-algebra which contains all open sets and all closed sets. The set function $m$ possesses the following three properties:
\begin{namedthm*}{The measure of an interval is its length}
Each nonempty interval $I$ is Lebesgue measurable and 
\[
m(I) = \ell(I).
\]
\end{namedthm*}
\begin{namedthm*}{Measure is translation invariant}
If $E$ is Lebesgue measurable and $y$ is any number then the translate of $E$ by $y$, $E+y = \{x+y \ |\ x \in E\}$, also is Lebesgue measurable and
\[
m(E+y) = m(E).
\]
\end{namedthm*}
\begin{namedthm*}{Measure is countably additive over countable disjoint unions of sets}
If $\{E_k\}_{k=1}^\infty$ is a countable disjoint collection of Lebesgue measurable sets, then
\[
m(\bigcup_{k=1}^\infty E_k) = \sum_{k=1}^\infty m(E_k).
\]
\end{namedthm*}
It is not possible to construct a set function that possesses the above three properties and is defined for all sets of real numbers (See Vitali sets).
We first construct a set function called \textbf{outer measure}, denoted by $m^*$, such that: 
\begin{enumerate}[label=(\roman*),align=left]
	\item the outer measure of an interval is its length:
	\[
		m^*(I)=\ell(I).
	\]
	\item outer measure is translation invariant:
	\[
		m^*(A+y)=m^*(A).
	\]
	\item outer measure is countably subadditive:
	\[
		m(\bigcup_{k=1}^\infty E_k) \le \sum_{k=1}^\infty m(E_k).
	\]
\end{enumerate}
Outer measure is defined for all sets of real numbers.
However, outer measure fails to be countably additive: there exists $A,B$ disjoint s.t. $m^*(A\cup B)<m^*(A)+m^*(B)$.\\
Then the Lebesgue measure $m$ is the restriction of $m^*$ to the Lebesgue measurable sets.

\begin{center}
	\textbf{PROBLEMS}
\end{center}
In the first three problems, let $m$ be a set function defined for all sets in a $\sigma$-algebra $\mathcal{A}$ with values in $[0,\infty]$. Assume $m$ is countably additive over countable disjoint collections of sets in $\mathcal{A}$.
\begin{enumerate}
	\setcounter{enumi}{0}
	\item Prove that if $A$ and $B$ are two sets in $\mathcal{A}$ with $A \subseteq B$, then $m(A) \le m(B)$. This property is called \textit{monotonicity}.\par
	$A \subseteq B \implies B = A \cup (B \cap A^c),$ where $A \cap (B \cap A^c) = \emptyset$. The set $(B \cap A^c)$ is measurable because $A^c$ is measurable and countable intersection is measurable, so $m(B) = m(A \cup (B \cap A^c)) = 	m(A) + m(B \cap A^c)$ by countable additivity, and thus $m(B) \ge m(A)$.
	\item Prove that if there is a set $A$ in the collection $\mathcal{A}$ for which $m(A) < \infty$, then $m(\emptyset) = 0$.\par
	We have $A\cap\emptyset = \emptyset$ and $A\cup\emptyset = A$. 
    \begin{align*}
        m(A)&=m(A\cup\emptyset)\\
        m(A)&=m(A)+m(\emptyset)&&\text{ by disjoint additivity}\\
        0&=m(\emptyset).
    \end{align*}
	\item Let $\{E_k\}_{k=1}^\infty$ be a countable collection of sets in $\mathcal{A}$. Prove that $m(\bigcup_{k=1}^\infty E_k) \le \sum_{k=1}^\infty m(E_k).$
    For any two measurable sets $A,B$, we have $A\cup B=(A\setminus B)\cup(B)$.
    By disjoint additivity,
    \[
        m(A\cup B) = m(A\setminus B)+m(B)
    \]
    Now, by problem 1, $(A\setminus B)\subseteq A$ implies that $m(A\setminus B)\le m(A)$.
    Therefore
    \[
        m(A\cup B) \le m(A)+m(B).
    \]
    \item A set function $c$, defined on all subsets of $\mathbb{R}$, is defined as follows.
	Define $c(E)$ to be $\infty$ if $E$ has infinitely many members and $c(E)$ to be equal to the number of elements in $E$ if $E$ is finite; define $c(\emptyset)=0$. Show that $c$ is a countably additive and translation invariant set function. This set function is called the \textbf{counting measure}.\\
    Suppose $E=\{x_1,\cdots,x_n\}$.\\
    Then $m(E)=n$. For any real number $y$, $y+E = \{y+x_1,\cdots,y+x_n\}$, so $m(y+E)=n$.\\
    Suppose $E$ has infinitely many members.\\
    Then $y+E$ has infinitely members as well, so $m(E)=m(y+E)=\infty$.\\
    Let $\{E_k\}_{k=1}^\infty$ be a disjoint collection of sets of real numbers.
    In the case that there exists an $E_k$ with infinitely many members, then the countable additivity is clear.
    \\In the case that all sets $E_k$ are finite, for any two sets $E_i,E_j$:\\
    $E_i = \{x_1,\cdots,x_n\}$\\
    $E_j = \{y_1,\cdots,y_m\}$\\  
    Then $E_i\cup E_j = \{x_1,\cdots,x_n,y_1,\cdots,y_m\}$ and $m(E_i\cup E_j)=n+m=m(E_i) +m(E_j)$.
\end{enumerate}

% 2.2
\section{Lebesgue Outer Measure}
\begin{flushleft}

Let $I$ be a nonempty interval of real numbers. We define its length:
\[
	\ell(I)=
	\begin{cases}
		\infty&\text{if $I$ is unbounded}\\
		b-a&\text{endpoints $a,b$}
	\end{cases}	
\]
For a set $A$ of real numbers, consider the countable collections $\{I_k\}_{k=1}^\infty$ of nonempty open, bounded intervals that cover $A$; that is, collections for which $A\subseteq\bigcup_{k=1}^\infty I_k$.
For each such collection, consider the sum of the lengths of the intervals in the collection. Since the lengths are positive numbers, each sum is uniquely defined independently of the order of the terms.
We define the \textbf{outer measure} of $A$, $m^*(A)$, to be the infimum of all such sums, that is
\[
	m^*(A)=\inf\biggl\{\sum_{k=1}^\infty\ell(I_k)\ \biggl|\ A\subseteq\bigcup_{k=1}^\infty I_k\biggr\}.
\]
It follows immediately from the definition of outer measure that $m^*(\emptyset)=0$.
Moreover, since any cover of a set $B$ is also a cover of any subset of $B$, outer measure is \textbf{monotone} in the sense that
\[
	A\subseteq B\implies m^*(A)\le m^*(B).	
\]
Then because $\emptyset\subseteq A$ for any set $A$, we have $0=m^*(\emptyset)\le m^*(A)$.
\begin{namedthm*}{Example}
	A countable set $C$ has outer measure zero.\\
	Because $C$ is countable, enumerate $C$ such that $C=\{c_k\}_{k=1}^\infty$.
	Fix $\epsilon>0$. 
	For each $k\in\mathbb{N}$, define an open interval $I_k= (c_k-\frac{\epsilon}{2^{k+1}},c_k+\frac{\epsilon}{2^{k+1}})$.
	Then $C\subseteq\bigcup_{k=1}^\infty I_k$.
	Therefore we have, by definition of infimum,
	\[
	0\le m^*(C)\le\sum_{k=1}^\infty\ell(I_k)=\sum_{k=1}^\infty\frac{2\epsilon}{2^{k+1}} = \sum_{k=1}^\infty\frac{\epsilon}{2^k}=\epsilon.
	\]
	This inequality holds for each $\epsilon>0$; thus $m^*(C)=0$.
\end{namedthm*}
\begin{namedthm*}{Lemma}
	$\sum_{k=1}^\infty\frac{1}{2^k}=1$.
\end{namedthm*}
\begin{proof}
	To show that $\sum_{k=1}^n\frac{1}{2^k}=1-\frac{1}{2^n}$ (induction).\\
	Let $P(n)$ be the assertion that $\sum_{k=1}^n\frac{1}{2^k}=1-\frac{1}{2^n}$ for $n\in\mathbb{N}$.\\
	$P(1)$:
	\begin{align*}
		\sum_{k=1}^1\frac{1}{2^k}=\frac{1}{2}=1-\frac{1}{2^1}.
	\end{align*}
	$P(2)$:
	\begin{align*}
		\sum_{k=1}^2\frac{1}{2^k}=\frac{1}{2}+\frac{1}{4}=\frac{3}{4}=1-\frac{1}{2^2}.
	\end{align*}
	Suppose $P(m)$ is true for $m\ge1$; that is, $\sum_{k=1}^m\frac{1}{2^k}=1-\frac{1}{2^m}$.\\
	$P(m+1)$:
	\begin{align*}
		\sum_{k=1}^{m+1}\frac{1}{2^k}&=\sum_{k=1}^m\frac{1}{2^k}+\frac{1}{2^{m+1}}\\
		&=1-\frac{1}{2^m}+\frac{1}{2^{m+1}}\\
		&=1-\frac{2}{2^{m+1}}+\frac{1}{2^{m+1}}\\
		&=1-\frac{1}{2^{m+1}}.
	\end{align*}
	Therefore $P(m)$ is true for all $m\ge1$.
	\[
		\sum_{k=1}^\infty\frac{1}{2^k}=\lim_{n\to\infty}\sum_{k=1}^n\frac{1}{2^k}=\lim_{n\to\infty}(1-\frac{1}{2^n})=1.
	\]
	\\(An alternate proof would be to see that we have a sequence of partial sums that is monotonic with $1$ as the supremum. Then the sequence of partial sums converges to $1$ and the series is summable to $1$.)
\end{proof}

\end{flushleft}
\begin{center}
	\textbf{PROBLEMS}
\end{center}
\begin{enumerate}
	\setcounter{enumi}{4}
	\item By using properties of outer measure, prove that the interval $[0,1]$ is not countable.\\
	Suppose that the interval $[0,1]$ is countable. By an example above, we showed that a countable set has outer measure zero, so $m^*([0,1])=0$.
	Also, the outer measure of an interval is its length. Then $m^*([0,1])=1$, and we reach a contradiction.
	\item Let $A$ be the set of irrational numbers in the interval $[0,1]$. Prove that $m^*(A)=1$.\\
	Let $A=[0,1]\cap\mathbb{Q}^c$.\\
	Then $A\subseteq[0,1]$, so by monotonicity of outer measure, 
	\begin{align*}
		m^*(A)&\le m^*([0,1])\\
		m^*(A)&\le 1.
	\end{align*}
	Also, we have 
	\begin{align*}
		[0,1]&=([0,1]\cap\mathbb{Q}^c)\cup([0,1]\cap\mathbb{Q})\\
		[0,1]&=A\cup([0,1]\cap\mathbb{Q})\\
		[0,1]&\subseteq A\cup([0,1]\cap\mathbb{Q})&&A=B\implies A\subseteq B\text{ and }A\supseteq B\\
		m^*([0,1])&\le m^*(A\cup (m^*([0,1]\cap\mathbb{Q}))&&\text{by monotonicity}\\
		m^*([0,1])&\le m^*(A)+m^*([0,1]\cap\mathbb{Q})&&\text{by countable subadditvity}\\
		m^*([0,1])&\le m^*(A)+0&&\text{countable set has outer measure zero}\\
		1&\le m^*(A).&&\text{outer measure of interval is length}
	\end{align*}
	Then $m^*(A)\le 1$ and $1\le m^*(A)$ imply that $m^*(A)=1$.
	\item A set of real numbers is said to be a $G_\delta$ set provided it is the intersection of a countable collection of open sets.
	Show that for any bounded set $E$, there is a $G_\delta$ set $G$ for which
	\[E\subseteq G \ \text{and}\ m^*(G)=m^*(E).\]
	Suppose $E$ is a bounded set of real numbers.\\
	Then there exists a real number $M$ for which $|x|\le M$ for all $x\in E$; that is, $E\subseteq[-M,M]$.
	By monotonicity of outer measure, $m^*(E)\le m^*([-M,M])=2M<\infty$, and the outer measure of $E$ is finite.\\
	Now, because outer measure is defined as $m^*(E)=\inf\{\sum_{k=1}^\infty\ell(I_k)\ |\ E\subseteq\bigcup_{k=1}^\infty I_k\}$, we have that $m^*(E)$ is the greatest lower bound, so for a natural number $n$, $m^*(E)+\frac{1}{n}$ is not a lower bound.
	That is, there exists a countable sequence of open intervals $\{(I_n)_k\}_{k=1}^\infty$ such that $E\subseteq\bigcup_{k=1}^\infty (I_n)_k$ and 
	\begin{equation}
		m^*(E)\le\sum_{k=1}^\infty\ell((I_n)_k)<m^*(E)+\frac{1}{n}.\tag{1}
	\end{equation}
	Now, for each natural number $n$, we can define the open set
	\begin{equation}
		\mathcal{O}_n:=\bigcup_{k=1}^\infty (I_n)_k.\tag{2}
	\end{equation}
	Also define the countable intersection of open sets; i.e., a $G_\delta$ set:
	\[
		\mathcal{O}:=\bigcap_{n=1}^\infty\mathcal{O}_n.
	\]
	Then because we have $E\subseteq\mathcal{O}_n$ for every $n$, this implies $E\subseteq\bigcap_{n=1}^\infty\mathcal{O}_n=\mathcal{O}$.
	\begin{align*}
		m^*(E)&\le m^*(\mathcal{O})&&\text{by monotonicity of outer measure: }E\subseteq\mathcal{O}\\
		&\le m^*(\mathcal{O}_n)&&\text{by monotonicity of outer measure: }\mathcal{O}=\bigcap_{n=1}^\infty\mathcal{O}_n\subseteq\mathcal{O}_n\\
		&=m^*(\bigcup_{k=1}^\infty (I_n)_k)&&\text{by (2)}\\
		&\le\sum_{k=1}^\infty\ell((I_n)_k)&&\text{by countable subadditivity of outer measure}\\
		&<m^*(E)+\frac{1}{n}.&&\text{by (1)}
	\end{align*}
	Therefore for any natural number $n$,
	\[
		m^*(E)\le m^*(\mathcal{O})<m^*(E)+\frac{1}{n}.	
	\]
	Taking the limit as $n\to\infty$, we get that $m^*(E)= m^*(\mathcal{O})$.\\
	Therefore there exists a $G_\delta$ set $\mathcal{O}$ such that $E\subseteq \mathcal{O}$ and $m^*(E)= m^*(\mathcal{O})$.
	\item Let $B$ be the set of rational numbers in the interval $[0,1]$, and let $\{I_k\}_{k=1}^n$ be a finite collection of open intervals that covers $B$.
	Prove that $\textstyle \sum_{k=1}^n m^*(I_k) \ge 1$.\\
	The rational numbers are dense in the reals; that is, between any two real numbers, there exists a rational number.
	Therefore, the rational numbers are also dense in the real subset $[0,1]$: between any two numbers in $[0,1]$, there exists a rational number. \\
	In the case that $[0,1]\subseteq \bigcup_{k=1}^nI_k$, the inequality is clear by monotonicity and subadditivity: 
	\[
		1=m^*([0,1])\le m^*(\bigcup_{k=1}^nI_k)\le\sum_{k=1}^nm^*(I_k).
	\]
	In the case that $[0,1]\not \subseteq \bigcup_{k=1}^nI_k$, then \[(\bigcup_{k=1}^nI_k)^c\cap[0,1]=(\bigcap_{k=1}^nI_k^c)\cap[0,1]=\bigcap_{k=1}^n(I_k^c\cap[0,1])\neq\emptyset.\]
	\\We want to show that $\bigcap_{k=1}^nI_k^c\cap[0,1]$ is countable so that $m^*(\bigcap_{k=1}^nI_k^c\cap[0,1])=0$.\\
	Because each $I_k^c\cap[0,1]$ is a closed interval (of irrational numbers), the intersection is also a closed interval (nonempty by assumption); that is, $\bigcap_{k=1}^n(I_k^c\cap[0,1])=[a,b]$ for some $a\le b$.
	Suppose by contradiction that $\bigcap_{k=1}^n(I_k^c\cap[0,1])$ is not countable. 
	Then we have that $a<b$.
	However, by density of the rationals, there exists a rational between $[a,b]$, leading to a contradiction.
	\\Therefore $\bigcap_{k=1}^n(I_k^c\cap[0,1])=\{x\}$, where $x\in\mathbb{Q}^c$, and $\bigcap_{k=1}^n(I_k^c\cap[0,1])$ is countable.
	\\Now we can write
	\begin{align*}
		[0,1]&=(\bigcup_{k=1}^nI_k\cap[0,1])\cup(\bigcap_{k=1}^nI_k^c\cap[0,1])\\
		[0,1]&\subseteq(\bigcup_{k=1}^nI_k\cap[0,1])\cup(\bigcap_{k=1}^nI_k^c\cap[0,1])&&A=B\implies A\subseteq B\text{ and }A\supseteq B\\
		m^*([0,1])&\le m^*((\bigcup_{k=1}^nI_k\cap[0,1])\cup(\bigcap_{k=1}^nI_k^c\cap[0,1]))&&\text{by monotonicity}\\
		m^*([0,1])&\le m^*(\bigcup_{k=1}^nI_k\cap[0,1])+m^*(\bigcap_{k=1}^nI_k^c\cap[0,1])&&\text{by countable subadditivity}\\
		m^*([0,1])&\le m^*(\bigcup_{k=1}^nI_k\cap[0,1])+0&&\text{the outer measure of a countable set is zero}\\
		1&\le m^*(\bigcup_{k=1}^nI_k\cap[0,1])\\
		1&\le m^*(\bigcup_{k=1}^nI_k)&&\text{by monotonicity: }\bigcup_{k=1}^nI_k\cap[0,1]\subseteq[0,1]\\
		1&\le\sum_{k=1}^nm^*(I_k).&&\text{by countable subadditivity}
	\end{align*}
	\item Prove that if $m^*(A)=0$, then $m^*(A\cup B) = m^*(B)$.
	\begin{align*}
		m^*(A\cup B)&\le m^*(A)+m^*(B)&&\text{by countable subadditivity}\\
		m^*(A\cup B)&\le m^*(B)&&\text{because }m^*(A)=0.
	\end{align*}
	Also, we have $B\subseteq A\cup B$, so by monotonicity of outer measure,
	\begin{align*}
		m^*(B)&\le m^*(A\cup B).
	\end{align*}
	Then $m^*(A\cup B)\le m^*(B)$ and $m^*(B)\le m^*(A\cup B)$ imply that $m^*(A\cup B) = m^*(B)$.
	\item Let $A$ and $B$ be bounded sets for which there is an $\alpha >0$ such that $|a-b| \ge \alpha$ for all $a \in A, b \in B$.
	Prove that $m^*(A \cup B) = m^*(A)+m^*(B)$.\\
	By countable subadditivity of outer measure, $m^*(A \cup B) \le m^*(A)+m^*(B)$.\\
	We can see that $A$ and $B$ are disjoint: Suppose by contradiction that $A,B$ are not disjoint. Then there exists a real number $x$ such that $x\in A$ and $x\in B$.
	But then $|x-x|=0<\alpha$, a contradiction.\\
	Let $\epsilon$ such that $\alpha/2>\epsilon>0$. By definition of outer measure and infimum, there exists a countable sequence of open intervals $\{I_k\}_{k=1}^\infty$ such that $(A\cup B)\subseteq\bigcup_{k=1}^\infty I_k$ and 
	\begin{equation}
		m^*(A\cup B)\le\sum_{k=1}^\infty\ell(I_k)<m^*(A\cup B)+\epsilon.\tag{1}	
	\end{equation}
	Now, each $I_k$ is such that $A\cap I_k\neq\emptyset$ or $B\cap I_k\neq\emptyset$, but not both.\\
	To see this, suppose by contradiction that there exists an $I_k$ such that $A\cap I_k\neq\emptyset$ and $B\cap I_k\neq\emptyset$.
	Then there exists $a,b\in I_k$ such that $a\in A$ and $b\in B$.
	Without loss of generality, suppose that these are the closest two elements of $A$ and $B$, and suppose $a<b$. Then the interval $(a,b)$ contains no elements of $A$ or $B$, and $m^*(b-a)\ge\alpha>\alpha/2$.
	This is a contradiction to the fact that $\sum_{k=1}^\infty\ell(I_k)$ is within $\alpha/2$ of $m^*(A\cup B)$.\\
	We can then separate $\{I_k\}_{k=1}^\infty$ into two subsequences $\{(I_A)_i\}_{i=1}^\infty$ and $\{(I_B)_j\}_{j=1}^\infty$ such that $A\subseteq\bigcup_{i=1}^\infty (I_A)_i$ and $B\subseteq\bigcup_{j=1}^\infty (I_B)_j$.
	Then because the sum is uniquely defined independently of the order of terms, $\sum_{k=1}^\infty\ell(I_k)=\sum_{i=1}^\infty\ell((I_A)_i)+\sum_{j=1}^\infty\ell((I_B)_j)$.\\
	Therefore we can write
	\begin{align*}
		m^*(A \cup B)&\le m^*(A)+m^*(B)&&\text{by countable subadditivity of outer measure}\\
		&\le m^*(\bigcup_{i=1}^\infty (I_A)_i)+m^*(\bigcup_{j=1}^\infty (I_B)_j)&&\text{by monotonicity of outer measure}\\
		&\le \sum_{i=1}^\infty\ell((I_A)_i)+\sum_{j=1}^\infty\ell((I_B)_j)&&\text{by countable subadditivity of outer measure}\\
		&=\sum_{k=1}^\infty\ell(I_k)&&\text{rearranging the sum}\\
		&<m^*(A\cup B)+\epsilon&&\text{by (1)}
	\end{align*}
	Therefore for any $\epsilon$,
	\[
		m^*(A \cup B)\le m^*(A)+m^*(B)<m^*(A\cup B)+\epsilon,
	\]
	thus $m^*(A \cup B) = m^*(A)+m^*(B)$.
\end{enumerate}



% 2.3
\section{The $\sigma$-Algebra of Lebesgue Measurable Sets}
\begin{flushleft}
	Outer measure is defined for all sets of real numbers, the outer measure of an interval is its length, outer measure is countably subadditive, and outer measure is translation invariant.
	However, outer measure fails to be countably additive or even finitely additive.
	That is, there exists disjoint sets $A,B$ such that 
	\begin{equation}
		m^*(A\cup B)<m^*(A)+m^*(B).\tag{1}	
	\end{equation}
	We identify a $\sigma$-algebra of sets, called the Lebesgue measurable sets, which contains all intervals and all open sets and has the property that the restriction of the set function outer measure to the collection of Lebesgue measurable sets is countably additive.
	\begin{namedthm*}{Definition}
		A set $E$ is said to be \textbf{measurable} provided for any set $A$,
		\[
			m^*(A)=m^*(A\cap E)+m^*(A\cap E^c).	
		\]		
	\end{namedthm*}  
	We see that the strict inequality (1) cannot occur if one of the sets is measurable:\\
	Suppose $A$ is measurable and $B$ is any set disjoint from $A$.
	\begin{align*}
		m^*(A\cup B)&=m^*([A\cup B]\cap A)+m^*([A\cup B]\cap A^c)&&\text{by definition of $A$ measurable}\\
		&=m^*(A)+m^*([A\cap A^c]\cup [B\cap A^c])&&\text{left: absorbtion, right: distributive property}\\
		&=m^*(A)+m^*(\emptyset\cup [B\setminus A])&&\text{complement and def of set difference}\\
		&=m^*(A)+m^*(B).&&\text{identity of union and set difference of disjoint sets}\\
	\end{align*}
	
	Suppose we want to prove that a set $E$ is measurable.\\
	We already have that for any set $A$,
	\begin{align*}
		m^*(A)&=m^*([A\cap E]\cup[A\cap E^c])&&\text{by set properties}\\
		m^*(A)&\le m^*(A\cap E)+m^*(A\cap E^c).&&\text{by subadditivity of outer measure}\\
	\end{align*}
	Therefore to show that $E$ is measurable, it suffices to show the other inequality:
	\begin{equation}
		m^*(A)\ge m^*(A\cap E)+m^*(A\cap E^c).\tag{2}
	\end{equation}
	This inequality holds trivially if $m^*(A)=\infty$. Therefore we need only prove (2) for sets $A$ that have finite outer measure.

	\begin{namedthm*}{Proposition 4}
		Any set of outer measure zero is measurable. In particular, any countable set is measurable.
	\end{namedthm*}
	\begin{proof}
		Let $E$ be such that $m^*(E)=0$. Let $A$ be any set.\\
		\begin{itemize}
			\item $A\cap E\subseteq E$
			\item $A\cap E^c\subseteq A$
		\end{itemize}
		By monotonicity of outer measure,
		\begin{align*}
			m^*(A\cap E)&\le m^*(E)=0\\
			m^*(A\cap E^c)&\le m^*(A)
		\end{align*}
		Therefore
		\begin{align*}
			m^*(A)&\ge m^*(A\cap E^c)+0\\
			m^*(A)&\ge m^*(A\cap E^c)+m^*(A\cap E).
		\end{align*}
	\end{proof}

	Every open set is the disjoint union of a countable collection of open intervals. Every interval is measurable, and the countable union of measurable sets is measurable, so all open sets are measurable.
	By complement, all closed sets are measurable. In the same way, all $G_\delta$ sets and all $F_\sigma$ sets are measurable.
	\par\medskip
	The intersection of all the $\sigma$-algebras of subsets of $\mathbb{R}$ that contain the open sets is a $\sigma$-algebra called the \textbf{Borel $\sigma$-algebra}, members of this collection are called \textbf{Borel sets}.
	That is, the Borel sigma-algebra is the sigma-algebra generated by the open sets.

	\begin{namedthm*}{Lemma 1}
		The set of all subsets of $X$, $\mathcal{P}(X)$ (or $2^X$), is a $\sigma$-algebra of subsets of $X$.
	\end{namedthm*}
	\begin{proof}
		Let $X$ be any set.
		\begin{enumerate}[label=(\roman*),align=left]   
			\item $X\in\mathcal{P}(X)$.
			\item if $A\in\mathcal{P}(X)$, then $A^c=X\setminus A = \{x\in X\ |\ x\notin A\}\in\mathcal{P}(X)$.
			\item if $A_i\in\mathcal{P}(X)$, then $\bigcup_{i=1}^\infty A_i = \{x\in X\ |\ x\in A_i\text{ for some }i\}$.
		\end{enumerate}
	\end{proof}

	\begin{namedthm*}{Lemma 2}
		Given any collection of $\sigma$-algebras $\{\mathcal{F}_\alpha\}_{\alpha\in\mathcal{A}}$ of $X$, the intersection $\bigcap_{\alpha\in\mathcal{A}}\mathcal{F}_\alpha$ is also a $\sigma$-algebra.
	\end{namedthm*}
	\begin{proof}
		Let $X$ be any set.
		\begin{enumerate}[label=(\roman*),align=left]   
			\item $X\in\mathcal{F}_\alpha,\forall\alpha\in\mathcal{A}\implies X\in\bigcap_{\alpha\in\mathcal{A}}\mathcal{F}_\alpha$.
			\item $A\in\bigcap_{\alpha\in\mathcal{A}}\mathcal{F}_\alpha\implies A\in\mathcal{F}_\alpha,\forall\alpha\in\mathcal{A}\implies A^c\in\mathcal{F}_\alpha,\forall\alpha\in\mathcal{A}\implies A^c\in\bigcap_{\alpha\in\mathcal{A}}\mathcal{F}_\alpha$.
			\item $A_i\in\bigcap_{\alpha\in\mathcal{A}}\mathcal{F}_\alpha\implies A_i\in\mathcal{F}_\alpha,\forall\alpha\in\mathcal{A}\implies \bigcup_{i=1}^\infty A\in\mathcal{F}_\alpha,\forall\alpha\in\mathcal{A}\implies \bigcup_{i=1}^\infty A\in\bigcap_{\alpha\in\mathcal{A}}\mathcal{F}_\alpha$.
		\end{enumerate}
	\end{proof}

	\begin{namedthm*}{Theorem}
		Given any collection $\mathcal{C}$ of subsets of $X$, there exists a smallest $\sigma$-algebra containing $\mathcal{C}$.
		(This is called the $\sigma$-algebra generated by $\mathcal{C}$.)
	\end{namedthm*}
	\begin{proof}
		Consider $S=\{\mathcal{F}\ |\ \mathcal{C}\subseteq\mathcal{F},\mathcal{F}\text{ is a }\sigma\text{-algebra of }X\}$.\\
		Now, $S$ is nonempty because $\mathcal{C}\in\mathcal{P}(X)$ and by Lemma 1, $\mathcal{P}(X)$ is a $\sigma$-algebra of $X$; therefore $\mathcal{P}(X)\in S$.\\
		Consider $\bigcap S$, the intersection of all the elements of $S$.\\
		\begin{enumerate}
			\item By Lemma 2, $\bigcap S$ is a $\sigma$-algebra,
			\item $\mathcal{C}\in\mathcal{F},\forall\mathcal{F}\in S\implies\mathcal{C}\in\bigcap S$, so $\bigcap S$ is a $\sigma$-algebra that contains $\mathcal{C}$,
			\item  $\bigcap S\subseteq \mathcal{F}$ for any $\mathcal{F}\in S$ by def of intersection, so $\bigcap S$ is the smallest $\sigma$-algebra containing $\mathcal{C}$.
		\end{enumerate}
	\end{proof}

	\begin{namedthm*}{Proposition 10}
		The translate of a measurable set is measurable.		
	\end{namedthm*}
	\begin{proof}
		Let $E$ be measurable, let $A$ be any set, and let $y$ be any real number.
		\\First we need to see that 
		\begin{align*}
			(A\cap [E+y])-y&=\{x:x\in A,\text{ and }x\in E+y\}-y=\{x:x\in A-y\text{ and }x\in E\}=[A-y]\cap E\\
			(A\cap [E+y]^c)-y&=\{x:x\in A,\text{ and }x\notin E+y\}-y=\{x:x\in A-y\text{ and }x\notin E\}=[A-y]\cap E^c
		\end{align*}
		Now, we have
		\begin{align*}
			m^*(A)&=m^*(A-y)&&\text{outer measure is translation invariant}\\
			&=m^*([A-y]\cap E)+m^*([A-y]\cap E^c)&&\text{because $E$ is measurable}\\
			&=m^*(A\cap [E+y]-y)+m^*(A\cap [E+y]^c-y)&&\text{by above}\\
			&=m^*(A\cap [E+y])+m^*(A\cap [E+y]^c).&&\text{outer measure is translation invariant}
		\end{align*}
		Therefore $E+y$ is measurable.
	\end{proof}

\end{flushleft}
\begin{center}
	\textbf{PROBLEMS}
\end{center}
\begin{enumerate}
	\setcounter{enumi}{10}
	\item Prove that if a $\sigma$-algebra of subsets of $\mathbb{R}$ contains intervals of the form $(a,\infty)$, then it contains all intervals.\\
	Let $\mathcal{M}$ be a $\sigma$-algebra of subsets of $\mathbb{R}$.\\
	Suppose that for any real number $a$, the interval $(a,\infty)\in\mathcal{M}$.\\
	For any real number $b$, because $\mathcal{M}$ is closed under complements,
	\begin{align*}
		(b,\infty)\in\mathcal{M}&\implies(b,\infty)^c=(-\infty,b]\in\mathcal{M}.
	\end{align*}
	For any natural number $n$, because $\mathcal{M}$ is closed under intersections:
	\begin{align*}
		(a-\frac{1}{n},\infty), (-\infty,b]\in\mathcal{M}&\implies(a-\frac{1}{n},\infty)\cap(-\infty,b]=(a-\frac{1}{n},b]\in\mathcal{M},\\
		(a,\infty), (-\infty,b-\frac{1}{n}]\in\mathcal{M}&\implies(a,\infty)\cap(-\infty,b-\frac{1}{n}]=(a,b-\frac{1}{n}]\in\mathcal{M}.
	\end{align*}
	Because $\mathcal{M}$ is closed under countable intersections and countable unions:
	\begin{align*}
		\text{for any $n\in\mathbb{N}$, }(a-\frac{1}{n},b]\in\mathcal{M}\implies\bigcap_{n=1}^\infty(a-\frac{1}{n},b]=[a,b]\in\mathcal{M},\\
		\text{for any $n\in\mathbb{N}$, }(a,b-\frac{1}{n}]\in\mathcal{M}\implies\bigcup_{n=1}^\infty(a,b-\frac{1}{n}]=(a,b)\in\mathcal{M}.
	\end{align*}
	In short, for any real numbers $a,b$, we have
	\begin{align*}
		[a,b]&=\bigcap_{n=1}^\infty(a-\frac{1}{n},b]=\bigcap_{n=1}^\infty(a-\frac{1}{n},\infty)\cap(-\infty,b]=\bigcap_{n=1}^\infty(a-\frac{1}{n},\infty)\cap(b,\infty)\\
		(a,b)&=\bigcup_{n=1}^\infty(a,b-\frac{1}{n}]=\bigcup_{n=1}^\infty(a,\infty)\cap(-\infty,b-\frac{1}{n}]=\bigcup_{n=1}^\infty(a,\infty)\cap(b-\frac{1}{n},\infty)
	\end{align*}
	The construction of intervals of the form $(a,b]$ and $[a,b)$ is similar.
	\item Show that every interval is a Borel set.\\
	Because any interval of the form $(a,\infty)$ is open, $(a,\infty)$ is a Borel set; i.e., it is a member of the Borel sigma-algebra.
	By the previous problem 11, any sigma-algebra that contains intervals of the form $(a,\infty)$ contains all intervals. 
	Therefore the Borel sigma-algebra contains all intervals and thus all intervals are Borel sets.
	\item Show that 
	\begin{enumerate}[label=(\roman*),align=left]                                                                                                         
        \item the translate of an $F_\sigma$ set is also $F_\sigma$,\\
        Let $F$ be an $F_\sigma$ set, that is, $F=\bigcup_{n=1}^\infty F_n$, with $F_n$ closed. \\
		For any real number $y$,
        \begin{align*}
			F+y&=(\bigcup_{n=1}^\infty F_n)+y\\
			&=\{x:x\in F_n\text{ for some }n\}+y\\
			&=\{x:x\in F_n+y\text{ for some }n\}\\
			&=\bigcup_{n=1}^\infty (F_n+y)
		\end{align*}
		The translate of any closed set is closed, so this is still an $F_\sigma$ set.
        \item the translate of a $G_\delta$ set is also $G_\delta$,\\
        Let $\mathcal{O}$ be a $G_\delta$ set, that is, $\mathcal{O}=\bigcap_{n=1}^\infty \mathcal{O}_n$, with $\mathcal{O}_n$ open.\\
        For any real number $y$,
		\begin{align*}
			\mathcal{O}+y&=(\bigcap_{n=1}^\infty \mathcal{O}_n)+y\\
			&=\{x:x\in \mathcal{O}_n\text{ for all }n\}+y\\
			&=\{x:x\in \mathcal{O}_n+y\text{ for all }n\}\\
			&=\bigcap_{n=1}^\infty (\mathcal{O}_n+y)
		\end{align*}
		The translate of any open set is open, so this is still a $G_\delta$ set.
        \item the translate of a set of measure zero also has measure zero.\\
        Let $E$ be a set of measure zero. That is, $m^*(E)=0$.\\
		For any $\epsilon>0$, by definition of infimum, there exists a countable collection of open intervals $\{I_k\}_{k=1}^\infty$ such that  
		\[
		m^*(E)\le\sum_{k=1}^\infty \ell(I_k) <m^*(E)+\epsilon,
		\]
		Thus because the outer measure is zero, 
		\[
			\sum_{k=1}^\infty \ell(I_k) <\epsilon.
		\]
		Now, for any real number $y$,
		\[
			E+y \subseteq (\bigcup_{k=1}^\infty I_k)+y=\bigcup_{k=1}^\infty (I_k+y).
		\]
		By monotonicity of outer measure, 
		\[
			m^*(E+y)\le \sum_{k=1}^\infty\ell(I_k+y)=\sum_{k=1}^\infty\ell(I_k)<\epsilon.
		\]
		Therefore $m^*(E+y)=0$.
    \end{enumerate}
    \item Show that if a set $E$ has positive outer measure, then there is a bounded subset of $E$ that also has positive outer measure.\\
    Suppose $E$ has positive outer measure.\\
	If $E$ is bounded, then clearly $E$ itself is a bounded subset of $E$ with positive outer measure.\\
	If $E$ is unbounded:\\
	First, we can partition the real numbers:
	\[
		\mathbb{R}=\bigcup_{n\in\mathbb{Z}}[n,n+1)	
	\]
	Then we have that 
	\[
		E=E\cap \mathbb{R}=E\cap(\bigcup_{n\in\mathbb{Z}}[n,n+1))=\bigcup_{n\in\mathbb{Z}}E\cap[n,n+1).
	\]
	By countable subadditivity of outer measure,
	\begin{align*}
		0<m^*(E)=m^*(\bigcup_{n\in\mathbb{Z}}E\cap[n,n+1))\le\sum_{n\in\mathbb{Z}}m^*(E\cap[n,n+1))
	\end{align*}
	Then there exists an $n\in \mathbb{Z}$ such that $m^*(E\cap[n,n+1))>0$, else we reach a contradiction.
	Therefore we have $E\cap[n,n+1)\subseteq E$ that is bounded and has positive outer measure.
    \item Show that if $E$ has finite measure and $\epsilon>0$, then $E$ is the disjoint union of a finite number of measurable sets, each of which has measure at most $\epsilon$.\\
    (We are letting $E$ be a measurable set because we are talking about measure specifically, not outer measure.)\\
	If $E$ is countable, then $E$ has measure zero and $E$ itself is the measurable set whose measure is less than any $\epsilon$: $m(E)=0<\epsilon$.
	In fact, if $E$ has measure zero then the conclusion is trivial.\\
	Suppose $E$ has positive measure.\\
	Fix $\epsilon>0$.\\
	In the case that $E$ is not bounded, there exists an $M$ such that 
	\[
		m(E\setminus[-M,M])<\epsilon.\ (\star)
	\]
	$(\star)$: To prove this we can partition $\mathbb{R}$: 
	\[
		\mathbb{R}=\bigcup_{n=0}^\infty\biggl([-(n+1),-n)\cup(n,n+1]\biggr)=\bigcup_{n=0}^\infty I_n.
	\]
	That is, $I_0=[-1,1],I_1=[-2,-1)\cup(1,2],I_2=[-3,-2)\cup(2,3],\cdots$\\
	Therefore $E=E\cap\mathbb{R}=E\cap(\bigcup_{n=0}^\infty I_n)=\bigcup_{n=0}^\infty (E\cap I_n)$.\\
	By countable additivity of measure, and the fact that $E$ has finite measure,
	\[
		m(E)=m(\bigcup_{n=0}^\infty (E\cap I_n))=\sum_{n=0}^\infty m(E\cap I_n)<\infty.
	\]
	Thus we have a sequence of partial sums that converges so there exists an index $M$ such that 
	\[
		\sum_{n=M}^\infty m(E\cap I_n)=\biggl|\sum_{n=0}^\infty m(E\cap I_n)-\sum_{n=0}^{M-1} m(E\cap I_n)\biggr|<\epsilon.
	\]
	We see that $m(E\setminus[-M,M]) = m(\bigcup_{n=M}^\infty (E\cap I_n))=\sum_{n=M}^\infty m(E\cap I_n)<\epsilon$.
	\\
	Therefore $E=(E\cap[-M,M])\cup(E\cap[-M,M]^c))$, a disjoint union, and $m(E\cap[-M,M]^c)<\epsilon$, so we need only worry now about $E\cap[-M,M]$.\\
	Else if $E$ is bounded, then there exists an $M$ such that $E\subseteq[-M,M]$, and $E=E\cap[-M,M]$.\\
    Now, for this $\epsilon$, we can partition the real numbers into a countable collection of disjoint measurable intervals $I_k$ of the form $[x,x+\epsilon)$.\\
	When we choose a natural number $l$ such that $\frac{2M}{\epsilon}<l$, we get $M<-M+l\epsilon$ so that 
	\[
		E\cap[-M,M]\subseteq [-M,M]\subseteq\bigcup_{k=1}^l[-M+(k-1)\epsilon,-M+k\epsilon)=\bigcup_{k=1}^l I_k.
	\]
	Then 
	\[
		E\cap[-M,M]=E \cap (\bigcup_{k=1}^l I_k) = \bigcup_{k=1}^l (E \cap I_k).
	\]
	Thus $E$ is the union of a finite number of disjoint measurable sets, each of which has measure at most $\epsilon$.\\
	(If $E$ is not bounded, $E=(\bigcup_{k=1}^l (E \cap I_k))\cup(E\setminus[-M,M])$, which still satisfies the conclusion.)
\end{enumerate}

% 2.4
\section{Outer and Inner Approximation of Lebesgue Measurable Sets}
\begin{flushleft}
	
	Measurable sets possess the following \textbf{excision property}: If $A$ is a measurable set of finite outer measure that is contained in $B$, then 
	\[
		m^*(B\setminus A)=m^*(B)-m^*(A).	
	\]
	This holds because
	\[
		m^*(B)=m^*(B\cap A)+m^*(B\cap A^c)=m^*(A)+m^*(B\cap A^c).	
	\]

\end{flushleft}
\begin{namedthm*}{Theorem 11}
	Let $E$ be any set of real numbers. Then each of the following four assertions is equivalent to the measurability of $E$.\\
	(Outer Approximation by Open Sets and $G_\delta$ sets)
	\begin{enumerate}[label=(\roman*),align=left]
        \item For each $\epsilon>0$, there is an open set $\mathcal{O}$ containing $E$ for which $m^*(\mathcal{O}\setminus E)<\epsilon$.
        \item There is a $G_\delta$ set $G$ containing $E$ for which $m^*(G\setminus E)=0$. 
    \end{enumerate}
	(Inner Approximation by Closed Sets and $F_\sigma$ sets)
	\begin{enumerate}[label=(\roman*),align=left]
        \setcounter{enumi}{2}
		\item For each $\epsilon>0$, there is a closed set $F$ contained in $E$ for which $m^*(E\setminus F)<\epsilon$.
        \item There is a $F_\sigma$ set $F$ contained in $E$ for which $m^*(E\setminus F)=0$.
    \end{enumerate}
\end{namedthm*}
\begin{proof}
	($E$ is measurable $\implies$ (i)):\\
	Assume $E$ is measurable and fix $\epsilon>0$.\\
	Case: $m^*(E)<\infty$:\\
	By definition of outer measure and infimum, there exists a countable collection of intervals $\{I_k\}_{k=1}^\infty$ such that $E\subseteq\bigcup_{k=1}^\infty I_k$ and
	\[
		m^*(E)\le\sum_{k=1}^\infty\ell(I_k)<m^*(E)+\epsilon.	
	\]
	Defining $\mathcal{O}=\bigcup_{k=1}^\infty I_k$, we see that $\mathcal{O}$ is an open set containing $E$.
	By subadditivity of outer measure,
	\begin{align*}
		m^*(\mathcal{O})=m^*(\bigcup_{k=1}^\infty I_k)\le\sum_{k=1}^\infty\ell(I_k)<m^*(E)+\epsilon,
	\end{align*}
	so that 
	\begin{align*}
		m^*(\mathcal{O})-m^*(E)<\epsilon.
	\end{align*}
	Because $E$ is measurable, has finite outer measure, and is contained in $\mathcal{O}$, we have the excision property:
	\[
		m^*(\mathcal{O}\setminus E)=m^*(\mathcal{O})-m^*(E)<\epsilon.
	\]
	Case: $m^*(E)=\infty$:\\
	Then $E$ may be expressed as the disjoint union of a countable collection $\{E_k\}_{k=1}^\infty$ of measurable sets, each of which has finite outer measure
	(See Problems 14 and 15 for an example of partitioning $\mathbb{R}$).
	Now, for each index $k$, because each $E_k$ is measurable and has finite outer measure, we showed above that there exists an open set $\mathcal{O}_k$ containing $E_k$ for which $m^*(\mathcal{O}_k\setminus E_k)<\epsilon/2^k$.
	The set $\mathcal{O}=\bigcup_{k=1}^\infty \mathcal{O}_k$ is open, it contains $E$ (because $E=\bigcup_{k=1}^\infty E_k \subseteq \bigcup_{k=1}^\infty \mathcal{O}_k =\mathcal{O}$), and we have $E\supseteq E_k \implies E^c\subseteq E_k^c$, so that
	\[
		\mathcal{O}\setminus E=\mathcal{O}\cap E^c = (\bigcup_{k=1}^\infty \mathcal{O}_k)\cap E^c = \bigcup_{k=1}^\infty (\mathcal{O}_k\cap E^c)\subseteq\bigcup_{k=1}^\infty (\mathcal{O}_k\cap E_k^c)=\bigcup_{k=1}^\infty (\mathcal{O}_k\setminus E_k).
	\]
	Therefore by monotonicity and subadditivity of outer measure,
	\[
		m^*(\mathcal{O}\setminus E)	\le m^*(\bigcup_{k=1}^\infty (\mathcal{O}_k\setminus E_k))\le\sum_{k=1}^\infty m^*(\mathcal{O}_k\setminus E_k)<\sum_{k=1}^\infty\epsilon/2^k=\epsilon.
	\]
	Thus property (i) holds for $E$.\\
	\\((i) $\implies$ (ii)):\\
	Now, assume property (i) holds for $E$. 
	Then for each natural number $k$, there exists an open set $\mathcal{O}_k$ that contains $E$ for which $m^*(\mathcal{O}_k\setminus E)<1/k$.
	\\Define $G=\bigcap_{k=1}^\infty\mathcal{O}_k$ so that $E\subseteq\mathcal{O}_k$ for all $k\implies E\subseteq\bigcap_{k=1}^\infty\mathcal{O}_k=G$.
	Then $G$ is a $G_\delta$ set that contains $E$.\\
	Then because for all $k$, $G\subseteq\mathcal{O}_k\implies G\setminus E\subseteq\mathcal{O}_k\setminus E$, by monotonicity of outer measure,
	\[
		m^*(G\setminus E)\subseteq m^*(\mathcal{O}_k\setminus E)<1/k.
	\]
	Thus $m^*(G\setminus E)=0$, and (ii) holds.\\
	\\((ii) $\implies$ $E$ is measurable):\\
	Assume property (ii) holds for $E$.\\
	We can write
	\begin{align*}
		E&=G\cap E\\
		&=\emptyset\cup(G\cap E)\\
		&=(G\cap G^c)\cup(G\cap E)\\
		&=G\cap (G^c\cup E)\\
		&=G\cap (G\cap E^c)^c\\
		&=G\cap (G\setminus E)^c.
	\end{align*}
	Now, $m^*(G\setminus E)=0$, and any set of measure zero is measurable, so $G\setminus E$ is measurable and also $(G\setminus E)^c$ is measurable by complement.
	Also, $G$ is a $G_\delta$ set, and all $G_\delta$ sets are measurable. Finally, the intersection of measurable sets is measurable so $G\cap (G\setminus E)^c$ is measurable.
	Thus $E$ is measurable.
\end{proof}

\begin{center}
	\textbf{PROBLEMS}
\end{center}
\begin{enumerate}
	\setcounter{enumi}{15}
	\item Complete the proof of Theorem 11 by showing that measurability is equivalent to (iii) and also equivalent to (iv).\\
	($E$ is measurable $\implies$ (iii)):\\
	Fix $\epsilon>0$.
	Suppose $E$ is measurable. Then $E^c$ is also measurable.
	Also, stating that $E^c$ is measurable is equivalent to property (i), that is, there exists an open set $\mathcal{O}$ containing $E^c$ such that $m^*(\mathcal{O}\setminus E^c)<\epsilon$.\\
	Then because $\mathcal{O}$ is open, we can define the closed set $F=\mathcal{O}^c$, and we have that 
	\[
		\mathcal{O}\setminus E^c = \mathcal{O}\cap E = F^c\cap E = E\setminus F,
	\]
	and $E^c\subseteq \mathcal{O}\implies E\supseteq \mathcal{O}^c=F$.
	Therefore $F$ is a closed set contained in $E$ for which $m^*(E\setminus F)=m^*(\mathcal{O}\setminus E^c)<\epsilon$, and (iii) holds.\\
	\\((iii) $\implies$ (iv)):\\
	Suppose that property (iii) holds for $E$.
	Then for each natural number $k$, there exists a closed set $F_k$ contained in $E$ for which $m^*(E\setminus F_k)<1/k$.
	\\Then defining $F=\bigcup_{k=1}^\infty F_n$, we have that $F_k\subseteq E,\forall k\implies F=\bigcup_{k=1}^\infty F_k \subseteq E$.
	Then $F$ is an $F_\sigma$ set that is contained in $E$.
	Then $F\supseteq F_k\implies F^c\subseteq F_k^c$ and thus $E\cap F^c\subseteq E\cap F_k^c$ and $E\setminus F\subseteq E\setminus F_k$.
	By monotonicity of outer measure, for all $k$, we have 
	\[
		m^*(E\setminus F)\le m^*(E\setminus F_k)<1/k.	
	\]
	Therefore $m^*(E\setminus F)=0$, and (iv) holds.\\
	\\((iv) $\implies$ $E$ is measurable):\\
	Suppose that property (iv) holds for $E$.\\
	We can write
	\begin{align*}
		E&=E\cap\mathbb{R}\\
		&=[F\cup E]\cap[F\cup F^c]\\
		&=F\cup [E\cap F^c]\\
		&=F\cup [E\setminus F].
	\end{align*}
	Now, $m^*(E\setminus F)=0$ implies $E\setminus F$ is measurable because all sets of measure zero are measurable.
	Also, $F$ is an $F_\sigma$ set, which is measurable. Therefore $F\cup [E\setminus F]$, the intersection of measurable sets, is measurable.
	Thus $E$ is measurable.
	\item Show that a set $E$ is measurable iff for each $\epsilon>0$, there is a closed set $F$ and open set $\mathcal{O}$ for which $F\subseteq E\subseteq \mathcal{O}$ and $m^*(\mathcal{O}\setminus F)<\epsilon$.\\
	Let $E$ be a set, and let $\epsilon>0$.\\
	(This case we assuming $E$ has finite measure to assume excision, maybe proof not complete)\\
	$(\implies)$ Suppose $E$ is measurable.\\
	Then by Theorem 11 (i), (iii), there is an open set $\mathcal{O}$ containing $E$ for which $m^*(\mathcal{O}\setminus E)<\epsilon/2$, and a closed set $F$ contained in $E$ for which $m^*(E\setminus F)<\epsilon/2$.
	That is, $F\subseteq E\subseteq \mathcal{O}$.\\
	By excision, $m^*(E\setminus F)=m^*(E)-m^*(F)$, and we can write
	\begin{align*}
		m^*(E)-m^*(F)&<\epsilon/2\\
		m^*(E)&<m^*(F)+\epsilon/2\\
		-m^*(E)&>-m^*(F)-\epsilon/2
	\end{align*}
	Also by excision, we have  $m^*(\mathcal{O}\setminus E)=m^*(\mathcal{O})-m^*(E)$, and 
	\[
		m^*(\mathcal{O})-m^*(F)-\epsilon/2<m^*(\mathcal{O})-m^*(E)<\epsilon/2
	\]
	Therefore $m^*(\mathcal{O}\setminus F)=m^*(\mathcal{O})-m^*(F)<\epsilon$.\\
	$(\impliedby)$ Suppose there is a closed set $F$ and open set $\mathcal{O}$ for which $F\subseteq E\subseteq \mathcal{O}$ and $m^*(\mathcal{O}\setminus F)<\epsilon$.\\
	By excision and monotonicity of outer measure, we have that 
	\[
		m^*(E\setminus F)=m^*(E)-m^*(F)\le m^*(\mathcal{O})-m^*(F)=m^*(\mathcal{O}\setminus F)<\epsilon.
	\]
	Therefore we have a closed set $F$ contained in $E$ for which $m^*(E\setminus F)<\epsilon$, i.e., proposition (iii), which implies that $E$ is measurable.
	\item Let $E$ have finite outer measure. Show that there is a $G_\delta$ set $G\supseteq E$ with $m(G)=m^*(E)$.
	Show that $E$ is measurable iff there is an $F_\sigma$ set $F \subseteq E$ with $m(F)=m^*(E)$.\\
	Let $E$ be a set with finite outer measure.\\
	Then for each natural number $k$, by definition of infimum, there exists a countable collection of open intervals $\{(I_k)_n\}_{n=1}^\infty$ whose union contains $E$ for which 
	\[
		\sum_{n=1}^\infty \ell((I_k)_n) <m^*(E)+1/k.
	\]
	Now, $\mathcal{O}_k=\bigcup_{n=1}^\infty (I_k)_n$ is an open set, and we can define $G=\bigcap_{k=1}^\infty\mathcal{O}_k$ so that $E\subseteq\mathcal{O}_k$ for all $k\implies E\subseteq\bigcap_{k=1}^\infty\mathcal{O}_k=G$.
	Then $G$ is a $G_\delta$ set that contains $E$. Because $G\subseteq\mathcal{O}_k$ for all $k$, by monotonicity,
	\[
		m^*(G)\le m^*(\mathcal{O}_k)=m^*(\bigcup_{n=1}^\infty (I_k)_n)\le\sum_{n=1}^\infty \ell((I_k)_n) <m^*(E)+1/k.
	\]
	Then we have $m^*(G)<m^*(E)+1/k$ for any natural number $k$, which implies $m^*(G)\le m^*(E)$. \\
	Also, by monotonicity, $E\subseteq G\implies m^*(E)\le m^*(G)$.\\
	Therefore $m^*(G)=m^*(E)$.\\
	\\Let $E$ be a set with finite outer measure.\\
	$(\implies)$ Suppose that $E$ is measurable.\\
	By Theorem 11 (iv), there is an $F_\sigma$ set $F$ contained in $E$ for which $m^*(E\setminus F)=0$. Because $E$ has finite outer measure, then $F$ has finite outer measure by monotonicity of outer measure.
	Then by excision, we have $m^*(E)-m^*(F)=m^*(E\setminus F)=0$, which implies $m^*(E)=m^*(F)$.\\
	$(\impliedby)$ Suppose there is an $F_\sigma$ set $F \subseteq E$ with $m(F)=m^*(E)$.\\
	Then $0=m^*(E)-m^*(F)$. Because $E$ has finite outer measure, then $F$ has finite outer measure by monotonicity of outer measure.
	Therefore by excision we have $0=m^*(E)-m^*(F)=m^*(E\setminus F)$ and Theorem 11 (iv) holds, which implies that $E$ is measurable.
	\item Let $E$ have finite outer measure.
	Show that if $E$ is not measurable, then there is an open set $\mathcal{O}$ containing $E$ that has finite outer measure and for which 
	\[m^*(\mathcal{O}\setminus E)>m^*(\mathcal{O})-m^*(E).\]
	Suppose $E$ is not measurable. 
	However, suppose by contradiction that for all open sets $\mathcal{O}$ containing $E$ that have finite outer measure, we have $m^*(\mathcal{O}\setminus E) \le m^*(\mathcal{O})-m^*(E)$.
	\\Let $\epsilon>0$.
	By definition of outer measure, there exists a countable collection of open intervals $\{I_k\}$ whose union contains $E$ and 
	\[
		\sum_{k=1}^\infty \ell(I_k)<m^*(E)+\epsilon.
	\]
	We can define $\mathcal{O}:=\bigcup_{k=1}^\infty I_k$, which is an open set that contains $E$, and by subadditivity of outer measure, we have that
	\[
		m^*(\mathcal{O})=m^*(\bigcup_{k=1}^\infty I_k)\le\sum_{k=1}^\infty \ell(I_k)<m^*(E)+\epsilon
	\]
	Therefore $m^*(\mathcal{O})-m^*(E)<\epsilon$, and $\mathcal{O}$ has finite outer measure.\\
	By assumption, we have that $m^*(\mathcal{O}\setminus E) \le m^*(\mathcal{O})-m^*(E)<\epsilon$.
	However, this means that we have an open set $\mathcal{O}$ containing $E$ for which $m^*(\mathcal{O}\setminus E) <\epsilon$, Theorem 11 (i), which is equivalent to saying that $E$ is measurable, which is a contradiction.
	\item (Lebesgue). Let $E$ have finite outer measure. Show that $E$ is measurable iff for each open, bounded interval $(a,b)$,
	\[
		b-a=m^*((a,b)\cap E)+m^*((a,b)\setminus E).
	\]
	Let $E$ be a set of finite outer measure.\\
	$(\implies)$ Suppose that $E$ is measurable.\\
	Then for any interval $(a,b)$, we have 
	\[
		b-a=\ell((a,b))=m^*((a,b))=m^*((a,b)\cap E)+m^*((a,b)\cap E^c)=m^*((a,b)\cap E)+m^*((a,b)\setminus E).
	\]
	$(\impliedby)$ Suppose that for each open, bounded interval $(a,b)$, we have $b-a=m^*((a,b)\cap E)+m^*((a,b)\setminus E)$.\\
	Then we have
	\[
		m^*((a,b))=\ell((a,b))=b-a=m^*((a,b)\cap E)+m^*((a,b)\setminus E)=m^*((a,b)\cap E)+m^*((a,b)\cap E^c).
	\]
	(This is only proved for any open interval; measurability of $E$ implies this is true for any set)
	\item Use property (ii) of Theorem 11 as the primitive definition of a measurable set and prove that the union of two measurable sets is measurable. Then do the same for property (iv).\\
	\\(ii) Let $E$ be any set of real numbers. Define $E$ to be measurable if there is a $G_\delta$ set $G$ containing $E$ for which $m^*(G\setminus E)=0$.\\
	Let $A$ and $B$ be two measurable sets under this definition.
	Then there exist $G_\delta$ sets $G_A,G_B$ containing $A,B$ respectively for which $m^*(G_A\setminus A)=0$ and $m^*(G_B\setminus B)=0$.\\
	Now, by definition of $G_\delta$ set:
	\begin{align*}
		G_A &= \bigcap_{k=1}^\infty \mathcal{O}_k,\text{ for }\mathcal{O}_k\text{ open}\\
		G_B &= \bigcap_{n=1}^\infty \mathcal{U}_n,\text{ for }\mathcal{U}_n\text{ open}
	\end{align*}
	Therefore 
	\begin{align*}
		G_A\cup G_B&=(\bigcap_{k=1}^\infty \mathcal{O}_k)\cup(\bigcap_{n=1}^\infty \mathcal{U}_n)\\
		&=\bigcap_{k=1}^\infty (\mathcal{O}_k\cup(\bigcap_{n=1}^\infty \mathcal{U}_n)\\
		&=\bigcap_{k=1}^\infty (\bigcap_{n=1}^\infty(\mathcal{O}_k\cup \mathcal{U}_n))
	\end{align*}
	For each $k,n$ pair, $\mathcal{O}_k\cup \mathcal{U}_n$ is an open set, so $G_A\cup G_B$ is a countable intersection of open sets and thus a $G_\delta$ set.
	Also, $G_A\supseteq A$ and $G_B\supseteq B$ imply that $G_A\cup G_B\supseteq A\cup B$, so $G_A\cup G_B$ is a $G_\delta$ set that contains $A\cup B$.\\
	We can write
	\begin{align*}
		(G_A\cup G_B)\setminus(A\cup B)&=(G_A\cup G_B)\cap(A\cup B)^c\\
		&=(G_A\cup G_B)\cap(A^c\cap B^c)\\
		&=[G_A\cap(A^c\cap B^c)]\cup[G_B\cap(A^c\cap B^c)]\\
		&=[G_A\cap A^c\cap B^c]\cup[G_B\cap B^c\cap A^c]\\
		&\subseteq[G_A\cap A^c]\cup[G_B\cap B^c]\\
		&\subseteq[G_A\setminus A]\cup[G_B\setminus B].\\
	\end{align*}
	By monotonicity of outer measure and subadditivity, 
	\begin{align*}
		m^*((G_A\cup G_B)\setminus(A\cup B))&\le m^*([G_A\setminus A]\cup[G_B\setminus B])\\
		&\le m^*(G_A\setminus A)+m^*(G_B\setminus B)\\
		&=0.
	\end{align*}
	Therefore $A\cup B$ is measurable.\\
	\\(iv) Let $E$ be any set of real numbers. Define $E$ to be measurable if there is an $F_\sigma$ set $F$ contained in $E$ for which $m^*(E\setminus F)=0$.\\
	Let $A$ and $B$ be two measurable sets under this definition.
	Then there exist $F_\sigma$ sets $F_A,F_B$ contained in $A,B$ respectively for which $m^*(A\setminus F_A)=0$ and $m^*(B\setminus F_B)=0$.\\
	Now, by definition of $F_\sigma$ set:
	\begin{align*}
		F_A &= \bigcup_{k=1}^\infty I_k,\text{ for }I_k\text{ closed}\\
		F_B &= \bigcup_{n=1}^\infty J_n,\text{ for }J_n\text{ closed}
	\end{align*}
	Therefore 
	\begin{align*}
		F_A\cup F_B&=(\bigcup_{k=1}^\infty I_k)\cup(\bigcup_{n=1}^\infty J_n),
	\end{align*}
	which is clearly a countable union of closed sets, so $F_A\cup F_B$ is an $F_\sigma$ set.
	Also, $F_A\subseteq A$ and $F_B\subseteq B$ imply that $F_A\cup F_B\subseteq A\cup B$, so $F_A\cup F_B$ is an $F_\sigma$ set that is contained in $A\cup B$.\\
	We can write
	\begin{align*}
		(A\cup B)\setminus(F_A\cup F_B)&=(A\cup B)\cap(F_A\cup F_B)^c\\
		&=(A\cup B)\cap(F_A^c\cap F_B^c)\\
		&=[A\cap(F_A^c\cap F_B^c)]\cup[B\cap(F_A^c\cap F_B^c)]\\
		&=[A\cap F_A^c\cap F_B^c]\cup[B\cap F_B^c\cap F_A^c]\\
		&\subseteq[A\cap F_A^c]\cup[B\cap F_B^c]\\
		&\subseteq[A\setminus F_A]\cup[B\setminus F_B].\\
	\end{align*}
	By monotonicity of outer measure and subadditivity, 
	\begin{align*}
		m^*((A\cup B)\setminus(F_A\cup F_B))&\le m^*([A\setminus F_A]\cup[B\setminus F_B])\\
		&\le m^*(A\setminus F_A)+m^*(B\setminus F_B)\\
		&=0.
	\end{align*}
	Therefore $A\cup B$ is measurable.\\
	\item For any set $A$, define $m^{**}(A)\in[0,\infty]$ by 
	\[
		m^{**}(A)=\inf\{m^*(\mathcal{O})\ |\ \mathcal{O}\supseteq A, \mathcal{O}\text{ open.}\}	
	\]
	How is this set function $m^{**}$ related to outer measure $m^*$?\\
	\\Consider any open set $\mathcal{O}$ such that $A\subseteq\mathcal{O}$.
	By monotonicity of outer measure, $m^*(A)\le m^*(\mathcal{O})$, and therefore $m^*(A)$ is a lower bound to the set $\{m^*(\mathcal{O})\ |\ \mathcal{O}\supseteq A, \mathcal{O}\text{ open.}\}$.
	Because $m^{**}$ is defined as the greatest lower bound, we get
	\[
		m^*(A)\le m^{**}(A).
	\]
	Now, if $m^*(A)=\infty$, then trivially we have
	\[
		m^*(A)\ge m^{**}(A),
	\]
	which implies $m^*(A)= m^{**}(A)$.\\
	Thus we consider the case where $m^*(A)<\infty$.\\
	Then for any $\epsilon>0$, by definition of infimum, there exists a countable collection of open intervals $\{I_n\}_{n=1}^\infty$ whose union contains $A$ for which 
	\[
		\sum_{n=1}^\infty \ell(I_n) <m^*(A)+\epsilon.
	\]
	Now, $\mathcal{O}=\bigcup_{n=1}^\infty I_n$ is an open set that contains $A$, so by definition of $m^{**}$,
	\[
		m^{**}(A)\le m^*(\mathcal{O})=m^*(\bigcup_{n=1}^\infty I_n)\le\sum_{n=1}^\infty \ell(I_n) <m^*(A)+\epsilon.
	\]
	Then $m^{**}(A)<m^*(A)+\epsilon$ implies $m^{**}(A)\le m^*(A)$.\\
	Therefore $m^*(A)= m^{**}(A)$.
	\item For any set $A$, def=ine $m^{***}(A)\in[0,\infty]$ by
	\[
		m^{***}(A)=\sup\{m^*(F)\ |\ F\subseteq A, F\text{ closed.}\}	
	\]
	How is this set function $m^{***}$ related to outer measure $m^*$?\\
	\\Consider any closed set $F$ such that $F\subseteq A$.
	By monotonicity of outer measure, $m^*(F)\le m^*(A)$, and therefore $m^*(A)$ is an upper bound to the set $\{m^*(F)\ |\ F\subseteq A, F\text{ closed.}\}$.
	Because $m^{**}$ is defined as the least upper bound, we get
	\[
		m^{***}(A)\le m^*(A).
	\]
	(In addition, if $A$ is measurable, then $m^{***}(A)= m^*(A)$.)
\end{enumerate}

% 2.5
\section{Countable Additivity, Continuity, and the Borel-Cantelli Lemma}
\begin{flushleft}
	\begin{namedthm*}{Theorem 15}[the Continuity of Measure]
		Lebesgue measure possesses the following continuity properties:
		\begin{enumerate}[label=(\roman*),align=left]
			\item If $\{A_k\}_{k=1}^\infty$ is an ascending collection of measurable sets, then
			\[
				m\biggl(\bigcup_{k=1}^\infty A_k\biggr)=\lim_{k\to\infty}m(A_k).
			\]
			\item If $\{B_k\}_{k=1}^\infty$ is a descending collection of measurable sets and $m(B_1)<\infty$, then
			\[
				m\biggl(\bigcap_{k=1}^\infty B_k\biggr)=\lim_{k\to\infty}m(B_k).
			\]
		\end{enumerate}
	\end{namedthm*}
	\begin{proof}
		Let $\{A_k\}_{k=1}^\infty$ be ascending and measurable.\\
		If there exists an index $k$ such that $m(A_k)>\infty$, then by monotonicity of measure, $m(\bigcup_{k=1}^\infty A_k)=\infty$.
		Also, because this collection is ascending, we have $A_k\subseteq A_n$ whenever $k\le n$; therefore by monotonicity, $\infty=m(A_k)\le m(A_n)$ for all $n$ such that $k\le n$, and thus (i) holds.\\
		Therefore it remains to prove the case that $m(A_k)<\infty$ for all $k$.\\
		Define $A_0=\emptyset$, and define $C_k=A_k\setminus A_{k-1}$. Then $\{C_k\}_{k=1}^\infty$ is disjoint and $\bigcup_{k=1}^\infty C_k=\bigcup_{k=1}^\infty A_k$.
		Now we can write
		\begin{align*}
			m(\bigcup_{k=1}^\infty A_k)&=m(\bigcup_{k=1}^\infty C_k)\\
			&=\sum_{k=1}^\infty m(C_k)&&\text{countable (disjoint) monotonicity}\\
			&=\sum_{k=1}^\infty m(A_k\setminus A_{k-1})\\
			&=\sum_{k=1}^\infty [m(A_k)-m(A_{k-1})]&&\text{by excision: }m(A_{k-1})<\infty\\
			&=\lim_{n\to\infty}\sum_{k=1}^n [m(A_k)-m(A_{k-1})]\\
			&=\lim_{n\to\infty}m(A_n)-m(A_0)&&\text{by telescoping}\\
			&=\lim_{n\to\infty}m(A_n).&&\text{because }A_0=\emptyset
		\end{align*}
		\\Let $\{B_k\}_{k=1}^\infty$ be descending and measurable.\\
		Define $D_k=B_1\setminus B_k=B_1\cap B_k^c$.\\
		Then because $\{B_k\}_{k=1}^\infty$ is descending, 
		\[
			B_k\supseteq B_{k+1}\implies B_1\cap B_k^c\subseteq B_1\cap B_{k+1}^c\implies D_k\subseteq D_{k+1},
		\]
		and $\{D_k\}_{k=1}^\infty$ is ascending.\\
		Now we have
		\[
			\bigcup_{k=1}^\infty D_k=\bigcup_{k=1}^\infty [B_1\cap B_k^c]=B_1\cap[\bigcup_{k=1}^\infty  B_k^c]=(B_1\cap[\bigcap_{k=1}^\infty  B_k]^c=B_1\setminus[\bigcap_{k=1}^\infty  B_k].
		\]
		Then by part (i), we can write
		\begin{align*}
			m(\bigcup_{k=1}^\infty D_k)&=\lim_{k\to\infty}m(D_k)\\
			m(B_1\setminus[\bigcap_{k=1}^\infty  B_k])&=\lim_{k\to\infty}m(B_1\setminus B_k)\\
			m(B_1)-m(\bigcap_{k=1}^\infty  B_k)&=\lim_{k\to\infty}[m(B_1)-m( B_k)]\\
			m(B_1)-m(\bigcap_{k=1}^\infty  B_k)&=m(B_1)-\lim_{k\to\infty}[m( B_k)]\\
			m(\bigcap_{k=1}^\infty  B_k)&=\lim_{k\to\infty}[m( B_k)].
		\end{align*}
	\end{proof}
	\begin{namedthm*}{The Borel-Cantelli Lemma}
		Let $\{E_k\}_{k=1}^\infty$ be a countable collection of measurable sets for which $\sum_{k=1}^\infty m(E_k)<\infty$.
		Then almost all $x\in\mathbb{R}$ belong to at most finitely many of the $E_k$'s.
	\end{namedthm*}
	\begin{proof}
		By countable subadditivity, for each $n$,
		\[
			m(\bigcup_{k=n}^\infty E_k)\le\sum_{k=n}^\infty m(E_k)<\infty.
		\]
		Because $\sum_{k=1}^\infty m(E_k)<\infty$, we have a sequence of partial sums such that for any $\epsilon>0$, there exists an index n for which 
		\[
			\sum_{k=n}^\infty m(E_k)=|\sum_{k=1}^\infty m(E_k)-\sum_{k=1}^{n-1} m(E_k)|<\epsilon.
		\] 
		Therefore there exists an $n$ such that $|\sum_{k=n}^\infty m(E_k)-0|<\epsilon$, and $\lim_{n\to\infty}\sum_{k=n}^\infty m(E_k)=0$.\\
		By continuity of measure (ii),
		\[
			m(\bigcap_{n=1}^\infty[\bigcup_{k=1}^\infty E_k])=\lim_{n\to\infty}m(\bigcup_{k=1}^\infty E_k)\le\lim_{n\to\infty}\sum_{k=n}^\infty m(E_k)=0.
		\]
		Therefore almost all $x\in\mathbb{R}$ fail to belong to $\bigcap_{n=1}^\infty[\bigcup_{k=1}^\infty E_k]$ and therefore belong to at most finitely many $E_k$'s.
	\end{proof}
	Let $\{A_k\}_{k=1}^\infty$ be a countable collection of sets that belong to a $\sigma$-algebra $\mathcal{A}$. Since $\mathcal{A}$ is closed w.r.t. countable unions and intersections, the following two sets belong to $\mathcal{A}$:
	\begin{align*}
		\lim\sup\{A_k\}_{k=1}^\infty&=\bigcap_{n=1}^\infty[\bigcup_{k=n}^\infty A_k]\\
		\lim\inf\{A_k\}_{k=1}^\infty&=\bigcup_{n=1}^\infty[\bigcap_{k=n}^\infty A_k]
	\end{align*}
	The set $\lim\sup\{A_k\}_{k=1}^\infty$ is the set of points that belong to $A_n$ for countably infinitely many indices $n$ while the set $\lim\inf\{A_k\}_{k=1}^\infty$ is the set of points that belong to $A_n$ except for at most finitely many indices $n$.
\end{flushleft}
\begin{center}
	\textbf{PROBLEMS}
\end{center}
\begin{enumerate}
	\setcounter{enumi}{23}
	\item Show that if $E_1$ and $E_2$ are measurable, then
	\[
		m(E_1\cup E_2)+m(E_1\cap E_2) = m(E_1)+m(E_2).	
	\]
	\begin{align*}
		m(E_1\cup E_2)+m(E_1\cap E_2)&=m([E_1\cup E_2]\cap E_1)+m([E_1\cup E_2]\cap E_1^c)+m(E_1\cap E_2)\\
		&=m(E_1)+m([E_1\cap E_1^c]\cup[E_2\cap E_1^c])+m(E_1\cap E_2)\\
		&=m(E_1)+m(\emptyset\cup[E_2\cap E_1^c])+m(E_1\cap E_2)\\
		&=m(E_1)+m(E_2\cap E_1^c)+m(E_1\cap E_2)\\
		&=m(E_1)+m([E_2\cap E_1^c]\cup[E_1\cap E_2])\\
		&=m(E_1)+m([E_2\cup(E_1\cap E_2)]\cap [E_1^c\cup(E_1\cap E_2)])\\
		&=m(E_1)+m(E_2\cap [E_1^c\cup(E_1\cap E_2)])\\
		&=m(E_1)+m(E_2\cap [E_1^c\cup E_1]\cap [E_1^c\cup E_2])\\
		&=m(E_1)+m(E_2\cap \mathbb{R} \cap [E_1^c\cup E_2])\\
		&=m(E_1)+m(E_2\cap [E_1^c\cup E_2])\\
		&=m(E_1)+m(E_2).
	\end{align*}
	\item Show that the assumption that $m(B_1)<\infty$ is necessary in part (ii) of the theorem regarding continuity of measure.\\
	In the proof of (ii), we get to the point 
	\[
		m(B_1)-m(\bigcap_{k=1}^\infty  B_k)=m(B_1)-\lim_{k\to\infty}[m( B_k)].
	\]
	If $m(B_1)=\infty$, then we have
	\[
		\infty-m(\bigcap_{k=1}^\infty  B_k)=\infty-\lim_{k\to\infty}[m( B_k)],
	\]
	and we cannot reach the conclusion we want because $\infty-\infty$ is not defined.
	\item Let $\{E_k\}_{k=1}^\infty$ be a countable disjoint collection of measurable sets. Prove that for any set $A$, 
	\[
		m^*(A\cap\bigcup_{k=1}^\infty E_k)=\sum_{k=1}^\infty m^*(A\cap E_k).	
	\]
	We have by countable subadditivity:
	\begin{align*}
		m^*(A\cap\bigcup_{k=1}^\infty E_k)&=m^*(\bigcup_{k=1}^\infty (A\cap E_k))\le\sum_{k=1}^\infty m^*(A\cap E_k).
	\end{align*}
	Now, for any $n$, we have $A\cap\bigcup_{k=1}^\infty E_k\supseteq A\cap\bigcup_{k=1}^n E_k$, so by monotonicity and Proposition 6,
	\begin{align*}
		m^*(A\cap\bigcup_{k=1}^\infty E_k)&\ge m^*(A\cap\bigcup_{k=1}^n E_k)=\sum_{k=1}^n m^*(A\cap E_k)
	\end{align*}
	The left hand side is independent of $n$, so taking the limit as $n\to\infty$, we get
	\begin{align*}
		m^*(A\cap\bigcup_{k=1}^\infty E_k)&\ge \sum_{k=1}^\infty m^*(A\cap E_k).
	\end{align*}
	\item Let $\mathcal{M}'$ be any $\sigma$-algebra of subsets of $\mathbb{R}$ and $m'$ a set function on $\mathcal{M}'$ which takes values in $[0,\infty]$, is countably additive, and such that $m'(\emptyset)=0$.
	\begin{enumerate}[label=(\roman*),align=left]
		\item Show that $m'$ is finitely additive, monotone, countably monotone, and possesses the excision property.\\
		Countable additivity implies that for any disjoint collection of measurable sets $\{E_k\}_{k=1}^\infty$, we have $m'(\bigcup_{k=1}^\infty E_k)=\sum_{k=1}^\infty m'(E_k)$.\\
		Now, any finite disjoint collection $\{E_k\}_{k=1}^n$ can be extended to the infinite disjoint collection $\{E_k'\}_{k=1}^\infty$, where $E_k'=E_k$ for $k\in\{1,\cdots,n\}$, and $E_k'=\emptyset$ for $k>n$.
		Clearly from this we have finite additivity.\\
		In Problem 1 of this chapter, it was shown that a countably additive set function possesses the monotonicity property.
		Thus $m'$ is monotone. It can clearly be shown that $m'$ is also countably monotone.\\
		To see excision, simply use countable additivity to see that for measurable sets $A,B$ such that $A\subseteq B$, we have
		\[
			m'(B) = m'([B\cap A]\cup[B\cap A^c])=m'(B\cap A)+m'(B\cap A^c)=m'(A)+m'(B\setminus A).
		\]
		\item Show that $m'$ possesses the same continuity properties as Lebesgue measure.\\
		Check Theorem 15 and the Borel-Cantelli Lemma above.
	\end{enumerate}
	\item Show that continuity of measure together with finite additivity of measure implies countable additivity of measure.\\
	Let $\{E_k\}_{k=1}^\infty$ be a disjoint collection of measurable sets. (if any $E_k$ has infinite measure, countable additivity is clear, so we need only consider sets of finite measure for all $E_k$.)\\
	Finite additivity implies that for the disjoint collection of measurable sets $\{E_k\}_{k=1}^n$, we have $m(\bigcup_{k=1}^n E_k)=\sum_{k=1}^n m(E_k)$.\\
	We can define $F_n=\bigcup_{k=1}^n E_k$ so that continuity of measure implies that for the ascending collection $\{F_n\}_{n=1}^\infty$ of measurable sets, we have $m(\bigcup_{n=1}^\infty F_n)=\lim_{n\to\infty}m(F_n)$.\\
	Therefore we can write
	\[
		m(\bigcup_{n=1}^\infty E_n)=m(\bigcup_{n=1}^\infty F_n)=\lim_{n\to\infty}m(F_n)=\lim_{n\to\infty}m(\bigcup_{k=1}^n E_k)=\lim_{n\to\infty}\sum_{k=1}^n m(E_k)=\sum_{k=1}^\infty m(E_k).
	\]
\end{enumerate}

% 2.6
\section{Nonmeasurable Sets}
\begin{flushleft}
	Consider the subgroup under addition $\mathbb{Q}\subseteq\mathbb{R}$.
	Now, $\mathbb{Q}$ is a normal subgroup, and we have the quotient group $\mathbb{R}/\mathbb{Q}$, with the (disjoint) cosets written as $r+\mathbb{Q}$ where $r\in\mathbb{R}$.
	A Vitali set $V\subseteq[0,1]$ is defined to be a set such that for all $r\in\mathbb{R}$, there exists exactly one unique $v\in V$ such that $v-r\in\mathbb{Q}$.
	Every Vitali set is uncountable, and $v-u\notin\mathbb{Q}$ for $u,v\in V$, $u\neq v$.
	\begin{namedthm*}{Theorem}
		A Vitali set is non-measurable.
	\end{namedthm*}
	\begin{proof}
		Suppose by contradiction that a Vitali set V is measurable.\\
		Let $\{q_k\}_{k=1}^\infty$ be an enumeration of the rational numbers in $[-1,1]$:\\
		recall that $\mathbb{Q}$ looks like
		\[
			\mathbb{Q} = \{0,\frac{1}{1},-\frac{1}{1},\frac{1}{2},-\frac{1}{2},\frac{2}{1},-\frac{2}{1}, \frac{3}{1}, -\frac{3}{1},\frac{1}{3},-\frac{1}{3},\frac{1}{4},-\frac{1}{4},\frac{2}{3},-\frac{2}{3},\cdots \},
		\]
		therefore
		\[
			\{q_k\}_{k=1}^\infty= \{0,\frac{1}{1},-\frac{1}{1},\frac{1}{2},-\frac{1}{2},\frac{1}{3},-\frac{1}{3},\frac{1}{4},-\frac{1}{4},\frac{2}{3},-\frac{2}{3},\cdots \}.
		\]
		For each natural number $k$, let $V_k=V+q_k=\{v+q_k:v\in V\}$.\\
		First we will show the following:
		\begin{enumerate}[label=(\roman*),align=left]
			\item $V_i\cap V_j=\emptyset$ for $i\neq j$
			\item $[0,1]\subseteq\bigcup_{k=1}^\infty V_k \subseteq [-1,2]$
		\end{enumerate}
		(i) Suppose by contradiction that $V_i\cap V_j=\emptyset$ for some $i\neq j$.\\
		That is, there exists $x\in V_i,y\in V_j$ such that $x=y$.\\
		Also, there exists $v,u\in V$ such that $x=v+q_i$ and $y=u+q_j$.\\
		By equality, we have $v+q_i=u+q_j$.\\
		In the case that $v=u$, we get $q_i=q_j$, a contradiction.\\
		In the case that $v\neq u$, we can write $v-u=q_j-q_i\in\mathbb{Q}$, a contradiction.\\\medskip
		(ii) For any real $r\in[0,1]$, there exists a $v\in V\subseteq[0,1]$ such that $r-v\in\mathbb{Q}$.\\
		We can see that
		\begin{align*}
			\max(r-v)&=1-0=1,\\
			\min(r-v)&=0-1=-1.
		\end{align*}
		which implies $r-v=q_i\in[-1,1]\cap\mathbb{Q}$ for some $i$, and thus $r=v+q_i\in V_i$.\\
		In short, we can write this as
		\[
			r\in[0,1]\implies r\in V_i\text{ for some }i\implies r\in\bigcup_{k=1}^\infty V_k\implies[0,1]\subseteq\bigcup_{k=1}^\infty V_k. 
		\]
		Now, $V_k=V+q_k,V\subseteq[0,1],q_k\in[-1,1]$, therefore
		\begin{align*}
			\max(v+q_k)&=1+1=2,\\
			\min(v+q_k)&=0-1=-1.
		\end{align*}
		Therefore $V_k\subseteq[-1,2]$ for all $k$, and thus $\bigcup_{k=1}^\infty V_k\subseteq[-1,2]$.\\\medskip
		Then we can write
		\begin{align*}
			m^*([0,1])&\le m^*(\bigcup_{k=1}^\infty V_k) \le m^*([-1,2])&&\text{by monotonicity of outer measure}\\
			1&\le\sum_{k=1}^\infty m^*(V_k)\le 3&&\text{countable additivity (measurability of $V$) }\star\\
			1&\le\sum_{k=1}^\infty m^*(V+q_k)\le 3\\
			1&\le\sum_{k=1}^\infty m^*(V)\le 3&&\text{by translation invariance of outer measure}\\
		\end{align*}
		However, $m^*(V)\ge0$ is a constant, so $\sum_{k=1}^\infty m^*(V)=0$ or $\sum_{k=1}^\infty m^*(V)=\infty$, neither of which is in $[1,3]$, and we reach a contradiction.
	\end{proof}
	For any nonempty set $E$ of real numbers, we define two points in $E$ to be \textbf{rationally equivalent} provided their difference belongs to $\mathbb{Q}$. By a \textbf{choice set} for the rational equivalence relation on $E$ we mean a set $\mathcal{C}_E$ consisting of exactly one member of each equivalence class.
	A choice set $\mathcal{C}_E$ is characterized by the following two properties:
	\begin{enumerate}
		\item the difference of two points in $\mathcal{C}_E$ is not rational;
		\item for each point $x$ in $E$, there is a point $c$ in $\mathcal{C}_E$ for which $x=c+q$, $q\in\mathbb{Q}$.
	\end{enumerate}
	The first property can be reformulated as
	\[
		\text{For any set }\Lambda\subseteq\mathbb{Q},\{\lambda+\mathcal{C}_E\}_{\lambda\in\Lambda}\text{ is disjoint}.
	\]
	We also have that 
	\[
		E\subseteq\bigcup_{\lambda\in\mathbb{Q}}(\lambda+\mathcal{C}_E).
	\]
\end{flushleft}
\begin{center}
	\textbf{PROBLEMS}
\end{center}
\begin{enumerate}
	\setcounter{enumi}{28}
	\item 
	\begin{enumerate}[label=(\roman*),align=left]
		\item Show that rational equivalence defines an equivalence relation on any set.\\
		\\Let $X$ be any set and define $x\sim y$ when $x-y\in\mathbb{Q}$ for $x,y\in X$.
		\begin{enumerate}
			\item $x-x=0\in\mathbb{Q}\iff x\sim x$ for all $x\in X$.
			\item $x\sim y \iff x-y=q \in\mathbb{Q}\iff y-x=-q\in\mathbb{Q}\iff y\sim x$ for all $x,y\in X$.
			\item $x\sim y,y\sim z \iff x-y=q\in\mathbb{Q},y-z=q'\in\mathbb{Q}\iff x-z=x-y+y-z=q+q'\in\mathbb{Q}\iff x\sim z$ for all $x,y,z\in X$.
		\end{enumerate}
		\item Explicitly find a choice set for the rational equivalence relation on $\mathbb{Q}$.\\
		\\(For any nonempty set $E$ of real numbers, we define two points in $E$ to be \textbf{rationally equivalent} provided their difference belongs to $\mathbb{Q}$. By a \textbf{choice set} for the rational equivalence relation on $E$ we mean a set $\mathcal{C}_E$ consisting of exactly one member of each equivalence class.)
		Therefore for the nonempty set $\mathbb{Q}$, we can choose a choice set $\mathcal{C}_{\mathbb{Q}}=\{q\}$ for any $q\in\mathbb{Q}$.
		\item Define two numbers to be irrationally equivalent provided their difference is irrational or zero. Is this an equivalence relation on $\mathbb{R}$? Is this an equivalence relation on $\mathbb{Q}$?\\
		\begin{enumerate}
			\item $x-x=0\in\{\mathbb{Q}^c,0\}\iff x\sim x$ for all $x\in \mathbb{R}$.
			\item $x\sim y \iff x-y=q \in\{\mathbb{Q}^c,0\}\iff y-x=-q\in\{\mathbb{Q}^c,0\}\iff y\sim x$ for all $x,y\in \mathbb{R}$.
			\item $2-\pi\in\{\mathbb{Q}^c,0\},\pi-0\in\{\mathbb{Q}^c,0\}$ but $2-0\notin\{\mathbb{Q}^c,0\}$
		\end{enumerate}
		Not an equivalence relation on $\mathbb{R}$.\\
		\begin{enumerate}
			\item $x-x=0\in\{\mathbb{Q}^c,0\}\iff x\sim x$ for all $x\in \mathbb{Q}$.
			\item $x\sim y \iff x-y=0 \in\{\mathbb{Q}^c,0\}\iff y-x=0\in\{\mathbb{Q}^c,0\}\iff y\sim x$ for all $x,y\in \mathbb{Q}$.
			\item $x\sim y,y\sim z \iff x-y=0\in\{\mathbb{Q}^c,0\},y-z=0\in\{\mathbb{Q}^c,0\}\iff x-z=x-y+y-z=0+0\in\{\mathbb{Q}^c,0\}\iff x\sim z$ for all $x,y,z\in \mathbb{Q}$.
		\end{enumerate}
		An equivalence relation on $\mathbb{Q}$.
	\end{enumerate}
	\item Show that any choice set for the rational equivalence relation on a set of positive outer measure must be uncountably infinite.\\
	\\Let $E$ be a set of positive outer measure.
	Suppose there exists a choice set $\mathcal{C}_E$ for the rational equivalence relation on $E$ such that $\mathcal{C}_E$ is countable. 
	All countable sets have outer measure zero, so $m^*(\mathcal{C}_E)=0$.
	Because we know $E\subseteq\bigcup_{\lambda\in\mathbb{Q}}(\lambda+\mathcal{C}_E)$, by monotonicity, subadditivity, and translation invariance of outer measure,
	\[
		m^*(E)\le m^*(\bigcup_{\lambda\in\mathbb{Q}}(\lambda+\mathcal{C}_E))\le\sum_{\lambda\in\mathbb{Q}}m^*(\lambda+\mathcal{C}_E)=\sum_{\lambda\in\mathbb{Q}}m^*(\mathcal{C}_E)=\sum_{\lambda\in\mathbb{Q}}0=0,
	\]
	and we have a contradiction to the fact that $m^*(E)>0$.
	\item Justify the assertion in the proof of Vitali's Theorem that it suffices to consider the case that $E$ is bounded.\\
	\\(Vitali: Any set of real numbers with positive outer measure contains a subset that fails to be measurable.)
	By Problem 14, we showed that every set of positive outer measure $E$ contains a bounded subset $A\subseteq E$ of positive outer measure. Therefore if there exists a subset $S\subseteq A$ that fails to be measurable, then $S \subseteq A\subseteq E$ is a subset that fails to be measurable.
	\item Does Lemma 16 remain true if $\Lambda$ is allowed to be finite or to be uncountably infinite? Does it remain true if $\Lambda$ is allowed to be unbounded?\\
	\\(Lemma 16: Let $E$ be a bounded measurable set of real numbers. Suppose there is a bounded, countably infinite set of real numbers $\Lambda$ for which the collection of translates of $E$, $\{\lambda+E\}_{\lambda\in\Lambda}$, is disjoint. Then $m(E)=0$.)\\
	Consider the case $\Lambda=\{1,2\}$ is finite, and $E=(0,1)$. Then $\{\lambda+E\}_{\lambda\in\Lambda=\{1,2\}}=\{1+(0,1),2+(0,1)\}=\{(1,2),(2,3)\}$, which is a disjoint collection. However, $m(E)=1\neq0$.\\
	If $\Lambda$ is uncountably infinite and satisfies that the translates are disjoint, then we can choose a countable subset of $\Lambda$ and thus Lemma 16 remains true.\\
	Consider the case $\Lambda=\{1,2,3,\cdots\}$ is unbounded, and $E=(0,1)$. Then the collection of translates of $E$, $\{(1,2),(2,3),(3,4),\cdots\}$ is disjoint but $m(E)=1\neq0$.
	\item Let $E$ be a nonmeasurable set of finite outer measure. Show that there is a $G_\delta$ set $G$ that contains $E$ for which
	\[
		m^*(E)=m^*(G),\text{ while }m^*(G\setminus E)>0.
	\]
	This is a similar construction for the proof from Theorem 11 (i).\\
	Let $E$ be a nonmeasurable set of finite outer measure.\\
	By definition of outer measure and infimum, for any natural number $n$, there exists a countable collection of intervals $\{(I_n)_k\}_{k=1}^\infty$ such that $E\subseteq\bigcup_{k=1}^\infty (I_n)_k$ and
	\[
		m^*(E)\le\sum_{k=1}^\infty\ell((I_n)_k)<m^*(E)+1/n.	
	\]
	Defining $\mathcal{O}_n=\bigcup_{k=1}^\infty (I_n)_k$, we see that $\mathcal{O}_n$ is an open set containing $E$ for each $n$.
	Further define $G=\bigcap_{n=1}^\infty\mathcal{O}_n$ so that $E\subseteq G\subseteq\mathcal{O}_n$ for any $n$ and $G$ is a $G_\delta$ set that contains $E$.\\
	By subadditivity of outer measure,
	\begin{align*}
		m^*(G)\le m^*(\mathcal{O}_n)=m^*(\bigcup_{k=1}^\infty (I_n)_k)\le\sum_{k=1}^\infty\ell(I_k)<m^*(E)+1/n,
	\end{align*}
	so that $m^*(G)<m^*(E)+1/n\implies m^*(G)\le m^*(E)$.
	By subadditivity, $E\subseteq G$ implies we also have $m^*(E)\le m^*(G)$, and so $m^*(E)=m^*(G)$.\\
	Now, we know that the outer measure is nonnegative by monotonicity, so we have the inequality $m^*(G\setminus E)\ge0$.\\
	By Theorem 11 (ii), $m^*(G\setminus E)=0\iff E$ is measurable, so we must have $m^*(G\setminus E)>0$.
\end{enumerate}

% 2.7
\section{The Cantor Set and the Cantor-Lebesgue Function}
\begin{center}
	\textbf{PROBLEMS}
\end{center}
\begin{enumerate}
	\setcounter{enumi}{33}
	\item Show that there is a continuous, strictly increasing function on the interval $[0,1]$ that maps a set of positive measure onto a set of measure zero.\\
	\\The function $\psi:[0,1]\to[0,2]$ defined by $\psi(x)=\varphi(x)+x$ maps the Cantor set $C\subseteq[0,1]$ onto a measurable set of positive measure.
	That is, $m(C)=0$ and $m(\psi(C))>0$.
	We can consider the inverse function $\psi^{-1}:[0,2]\to[0,1]$ restricted to $[0,1]$: $\psi^{-1}|_{[0,1]}:[0,1]\to[0,1]$.
	Now consider the set $C'=C\cap[0,1]$. This set $C'$ is a measurable subset of $C$, a measurable set of measure zero, so by monotonicity of measure, $m(C')=0$.
	Then the function has $\psi^{-1}|_{[0,1]}(\psi(C'))=C'$, where $m(\psi(C'))>0$ and $m(C')=0$, thus mapping the set $\psi(C')$ of positive measure* onto the set $C'$ of measure zero.\\
	(*We know that  $m(\psi(C))>0$, but not shown that $m(\psi(C'))>0$ where $C'\subseteq C$.)
	\item Let $f$ be an increasing function on the open interval $I$. For $x_0\in I$ show that $f$ is continuous at $x_0$ iff there are sequences $\{a_n\}$ and $\{b_n\}$ in $I$ such that for each $n$, $a_n<x_0<b_n$, and $\lim_{n\to\infty}[f(b_n)-f(a_n)]=0$.\\
	\\Let $f$ be an increasing function on the open interval $I$ and let $x_0\in I$.\\
	$(\implies)$ Suppose that $f$ is continuous at $x_0$.\\
	Because $I$ is open, there exists an index $N$ such that for all $n\ge N$, we have that $(x_0-1/n,x_0+1/n)\subseteq I$.
	Then for each $n\ge N$ we can choose $a_n\in(x_0-1/n,x_0)$ and $b_n\in(x_0,x_0+1/n)$, and for $n<N$ let $a_n=a_N$ and $b_n=b_N$, so that $a_n<x_0<b_n$ for all $n$.
	Now,we have
	\begin{align*}
		x_0-1/n<a_n<x_0&\implies x_0-a_n<1/n,\\
		x_0<b_n<x_0+1/n&\implies b_n-x_0<1/n,
	\end{align*}
	therefore $\lim_{n\to\infty} a_n = x_0$ and $\lim_{n\to\infty} b_n = x_0$.
	Because $f$ is continuous and increasing, for all $\epsilon>0$, there exists the number $1/n>0$ such that
	\begin{align*}
	x_0-a_n<1/n&\implies f(x_0)-f(a_n)<\epsilon,\\
	b_n-x_0<1/n&\implies f(b_n)-f(x_0)<\epsilon.
	\end{align*}
	(therefore $\lim_{n\to\infty} f(a_n) = f(x_0)$ and $\lim_{n\to\infty} f(b_n) = f(x_0)$.) We can write
	\[
		[f(b_n)-f(a_n)]=f(x_0)-f(a_n)+f(b_n)-f(x_0) <\epsilon+\epsilon=\epsilon'
	\]
	and so $\lim_{n\to\infty}[f(b_n)-f(a_n)]=0$.\\
	\\$(\impliedby)$ Suppose that there exist sequences $\{a_n\}$,$\{b_n\}$ such that $a_n<x_0<b_n$ and $\lim_{n\to\infty}[f(b_n)-f(a_n)]=0$.\\
	That is, for any $\epsilon>0$, there exists an index $N$ such that $f(b_n)-f(a_n)<\epsilon$ for all $n\ge N$.
	\\Then $f(b_n)<f(a_n)+\epsilon$ and $f(b_n)-\epsilon<f(a_n)$.\\
	Because $f$ is increasing, we have \[f(b_n)-\epsilon<f(a_n)<f(x_0)<f(b_n)<f(a_n)+\epsilon.\]
	Then $f(x_0)-f(a_n)<\epsilon$ and $f(b_n)-f(x_0)<\epsilon$, which implies $\lim_{n\to\infty} f(a_n) = f(x_0)$ and $\lim_{n\to\infty} f(b_n) = f(x_0)$.\\
	By monotonicity of $f$, we also have \[b_n-\epsilon<a_n<x_0<b_n<a_n+\epsilon.\]
	Then $x_0-a_n<\epsilon$ and $b_n-x_0<\epsilon$, which implies $\lim_{n\to\infty} a_n= x_0$ and $\lim_{n\to\infty} b_n = x_0$.\\
	Now, clearly we see that for any $\epsilon>0$, we have $x_0-a_n<\epsilon\iff f(x_0)-f(a_n)<\epsilon$, and $b_n-x_0<\epsilon\iff f(b_n)-f(x_0)<\epsilon$, and continuity at $x_0$ follows.
	\item Let $f$ be a continuous function defined on $E$. Is it true that $f^{-1}(A)$ is always measurable if $A$ is measurable?\\
	\\No, the function $\psi:[0,1]\to[0,2]$ defined by $\psi(x)=\varphi(x)+x$ maps a measurable set $A$, subset of the Cantor set, onto a nonmeasurable set $\psi(A)$. Define $f=\psi^{-1}$ so that $f^{-1}(A)=(\psi^{-1})^{-1}(A)$ is not measurable but $A$ is measurable.
	\item Let the function $f:[a,b]\to\mathbb{R}$ be Lipschitz; that is, there is a constant $c\ge0$ such that for all $u,v\in[a,b]$, $|f(u)-f(v)|\le c|u-v|$.
	Show that $f$ maps a set of measure zero onto a set of measure zero. Show that $f$ maps a $F_\sigma$ set onto an $F_\sigma$ set. Conclude that $f$ maps a measurable set to a measurable set.\\
	\\Let $f$ be a Lipschitz function on the interval $I$. Clearly $f$ is also continuous.\\
	Let $E\subseteq I$ be a set of measure zero; that is, $m^*(E)=m(E)=0$.
	By definition of infimum, for any $\epsilon>0$, there exists a countable collection of open intervals $\{I_k\}_{k=1}^\infty$, $I_k=(a_k,b_k)$, such that $E\subseteq\bigcup_{k=1}^\infty I_k$ and 
	\[
		0\le \sum_{k=1}^\infty \ell(I_k)<0+\frac{\epsilon}{c}.
	\]
	We also have that $E\subseteq\bigcup_{k=1}^\infty I_k\implies f(E)\subseteq f(\bigcup_{k=1}^\infty I_k)=\bigcup_{k=1}^\infty f(I_k)$.
	\\Also, by Chapter 1 Problem 54, Because $I_k$ is an interval, the continuous real-valued function $f$ on $I_k$ has an interval as its image; that is, $f(I_k)$ is an interval.
	Then there exists some $u_k,v_k\in(a,b)$ such that $f(I_k)=(f(u_k),f(v_k))$ and $m(f(I_k))=f(v_k)-f(u_k)$. Then because $f$ is Lipschitz, $|f(v_k)-f(u_k)|\le c|v_k-u_k|$ for all $k$.
	\begin{align*}
		m(f(E))&\le	m(\bigcup_{k=1}^\infty f(I_k))&&\text{ by monotonicity}\\
		&\le \sum_{k=1}^\infty m(f(I_k))&&\text{ by subadditivity}\\
		&=\sum_{k=1}^\infty m(f(v_k)-f(u_k))\\
		&\le \sum_{k=1}^\infty c|v_k-u_k|&&\text{ because $f$ is Lipschitz}\\
		&\le\sum_{k=1}^\infty c|b_k-a_k|&&\text{ because }(u_k,v_k)\subseteq(a,b)\\
		&= \sum_{k=1}^\infty c\ell(I_k)\\
		&<\epsilon.
	\end{align*}
	Therefore $m(f(E))=0$.\\
	\item Let $F$ be the subset of $[0,1]$ constructed in the same manner as the Cantor set except that each of the intervals removed at the $n$th deletion stage has length $\alpha 3^{-n}$ with $0<\alpha<1$.
	Show that $F$ is a closed set, $[0,1]\setminus F$ is dense in $[0,1]$, and $m(F)=1-\alpha$. Such a set $F$ is called a generalized Cantor set.\\
	\\Define $F$ to be constructed in the same manner as the Cantor set, with 
	\[
		F=\bigcap_{k=1}^\infty F_k,	
	\]
	where $\{F_k\}_{k=1}^\infty$ is a descending sequence of closed sets, and each $F_k$ is a disjoint union of $2^k$ closed intervals, each of length $\alpha/3^k$.
	\\It can clearly be seen that $F$ is a closed set because it is an intersection of closed sets.
	\\Now, for any point $x\in[0,1]$, there exists an index $k$ such that $x\notin F_k$; that is, $x\in F_k^c$, which is an open set. Therefore we can construct sequences in $([0,1]\setminus F)\setminus\{x\}$ that converge to $x$.
	\\Each $F_k$ is the disjoint union of $2^k$ closed intervals each of length $\alpha/3^k$, so at each step we remove $2^{k-1}$ open intervals of length $\alpha/3^k$:
	\begin{align*}
		m(F_1)&=1-\alpha/3\\
		m(F_2)&=1-\alpha/3-2\alpha/3^2\\
		m(F_3)&=1-\alpha/3-2\alpha/3^2-2^2\alpha/3^3\\
		\vdots\\
		m(F_n)&=1-\sum_{k=1}^n2^{k-1}\alpha/3^k
	\end{align*}
	Then by the continuity of measure, we have \[m(\bigcap_{k=1}^\infty F_k)=\lim_{n\to\infty}m(F_n)=\lim_{n\to\infty}(1-\sum_{k=1}^n2^{k-1}\alpha/3^k).\]
	We can see that
	\begin{align*}
		\lim_{n\to\infty}\sum_{k=1}^n2^{k-1}\alpha/3^k&=\alpha/3\lim_{n\to\infty}\sum_{k=1}^n(\frac{2}{3})^{k-1}\\
		&=\alpha/3\lim_{n\to\infty}\sum_{k=0}^{n-1}(\frac{2}{3})^{k}\\
		&=\alpha/3\lim_{n\to\infty}\frac{1-(2/3)^n}{1-(2/3)}\\
		&=\alpha/3\frac{1}{1-(2/3)}\\
		&=\alpha/3(\frac{1}{1/3})\\
		&=\alpha.
	\end{align*}
	Therefore $m(F)=m(\bigcap_{k=1}^\infty F_k)=1-\alpha$.
	\item Show that there is an open set of real numbers that, contrary to intuition, has a boundary of positive measure. (Hint: consider the complement of the generalized Cantor set of the preceding problem.)\\
	\\We have $F\cup(F^c\cap[0,1])=[0,1]$, and $m(F)=1-\alpha$, and $m(F^c\cap[0,1])=\alpha$...
	\item A subset $A$ of $\mathbb{R}$ is said to be \textbf{nowhere dense} in $\mathbb{R}$ provided that every open set $\mathcal{O}$ has a non-empty open subset that is disjoint from $A$. Show that the Cantor set is nowhere dense in $\mathbb{R}$.\\
	\\The Cantor set $C\subseteq[0,1$] is defined to be the countable intersection of sets $C_k$, where $C_k$ is the disjoint union of $2^k$ closed intervals of length $1/3^k$ each.
	From Ch1 Proposition 9, we know that every open set is the countable disjoint union of open intervals. 
	Therefore we need only prove Problem 40 for any open interval.
	\\Consider any open interval $(a,b)\in\mathbb{R}$. 
	\\In the case that there exists an index $k$ such that $(a,b)\in C_k^c$, then the proof is done:
	\\Ex: $(a,b)=(3/18,4/18)$. Then for $k=2$, we have 
	\[
		C_2=[0,1/9]\cup[2/9,1/3]\cup[2/3,7/9]\cup[8/9,1],	
	\]
	so that
	\[
		(3/18,4/18)\subseteq C_2^c=(1/9,2/9)\cup(1/3,2/3)\cup(7/9,8/9)=(2/18,4/18)\cup(1/3,2/3)\cup(7/9,8/9).	
	\] 
	In the case that for all indices $k$ we have that $(a,b)\in C_k$, then simply choose an index far enough so that one of the "open middle third" removal generated from $C_k$ is a subset of $(a,b)$.
	\\Ex: $(a,b)=(6/10,7/10)\ni 2/3$ and $2/3\in C$ so $(a,b)\cap C\neq\emptyset$. Then for $k=1$, we have
	\[
		(6/10,2/3)\subseteq(a,b)\text{ and } (6/10,2/3)\notin C_1=[0,1/3]\cup[2/3,1].
	\]
	Ex: $(a,b)=(2/3,20/27$. Then for $k=3$, we have
	\[
		C_3=[0,\frac{1}{27}]\cup[\frac{2}{27},\frac{1}{9}]\cup[\frac{2}{9},\frac{7}{27}]\cup[\frac{8}{27},\frac{1}{3}]\cup[\frac{2}{3},\frac{19}{27}]\cup[\frac{20}{27},\frac{7}{9}]\cup[\frac{8}{9},\frac{25}{27}]\cup[\frac{26}{27},1].
	\]
	so that
	\[
		(19/27,20/27)\subseteq(a,b)\text{ and } (19/27,20/27)\notin C_3.
	\]
	\item Show that a strictly increasing function that is defined on an interval has a continuous inverse.\\
	\\Let $f$ be a strictly increasing function on the interval $I$. Then for $x,y\in I$ such that $x<y$, we have $f(x)<f(y)$.
	\\Then $f$ is injective because 
	\[
		x\neq y\implies x<y\text{ or }x>y\implies f(x)<f(y)\text{ or }f(x)>f(y)\implies f(x)\neq f(y).
	\]
	Therefore the inverse $f^{-1}:im(f)\to I$ exists:
	\[
		f^{-1}(x)\neq f^{-1}(y)\implies f^{-1}(f(x))\neq f^{-1}(f(y))\implies x\neq y,
	\]
	that is, $f^{-1}$ is a function because $x=y\implies f^{-1}(x)= f^{-1}(y)$ for all $x,y\in im(f)$.\\
	Let $x\in I$ such that $a_n,b_n\in I$ with $a_n < x < b_n$ and $\lim_{n\to\infty}a_n=x$, $\lim_{n\to\infty}b_n=x$. 
	Then clearly, $\lim_{n\to\infty}[b_n-a_n]=0$.
	Then because $f$ is strictly increasing, $f(a_n) < f(x) < f(b_n)$.\\
	Now, we have the sequences $f(a_n)$ and $f(b_n)$ in $im(f)$ such that for each $n$, $f(a_n) < f(x) < f(b_n)$, and $\lim_{n\to\infty}[f^{-1}(f(b_n))-f^{-1}(f(a_n))]=\lim_{n\to\infty}[b_n-a_n]=0$. The results from Problem 35 tells us that $f^{-1}$ is continuous at $f(x)$.
	\item Let $f$ be a continuous function and $B$ be a Borel set. Show that $f^{-1}(B)$ is a Borel set. (Hint: the collection of sets $E$ for which $f^{-1}(E)$ is Borel is a $\sigma$-algebra containing the open sets.)\\
	\\Let $S=\{E\ |\ f^{-1}(E)\text{ is Borel}\}$.\\
	To show that $S$ is a $\sigma$-algebra, know that the Borel sets is a $\sigma$-algebra.\\
	Observe that:
	\begin{enumerate}[label=(\roman*),align=left]
		\item $f^{-1}(\emptyset)=\emptyset\implies\emptyset\in S$.
		\item $E\in S\implies f^{-1}(E)\text{ is Borel }\implies f^{-1}(E)^c=f^{-1}(E^c)\text{ is Borel }\implies E^c\in S$.
		\item $E_k\in S\implies f^{-1}(E_k)\text{ is Borel }\implies\bigcup_{k=1}^\infty f^{-1}(E_k)=f^{-1}(\bigcup_{k=1}^\infty E_k)\text{ is Borel }\implies \bigcup_{k=1}^\infty E_k\in S$.
	\end{enumerate}
	Also, any open set $\mathcal{O}$ is in $S$ because $f^{-1}(\mathcal{O})$ is open and thus Borel.
	Thus $S$ is a $\sigma$-algebra containing the open sets; that is, the Borel $\sigma$-algebra is a subset of $S$.
	Therefore for any Borel set $B$, $B\in S$ and thus $f^{-1}(B)$ is Borel.
	\item Use the preceding two problems to show that a continuous strictly increasing function that is defined on an interval maps Borel sets to Borel sets.\\
	\\Let $I$ be an interval and $f:I\to\mathbb{R}$ be a continuous strictly increasing function.
	\\By Problem 41, we showed that $f^{-1}:im(f)\to I$ exists and is continuous.
	\\Let $B\in I$ be any Borel set. 
	By Problem 42, $(f^{-1})^{-1}(B)=f(B)$ is a Borel set.
\end{enumerate}	
