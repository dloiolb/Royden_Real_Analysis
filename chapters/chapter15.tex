% Chapter 15
\chapter{Compactness Regained: The Weak Topology}

% 15.1
\section{Alaoglu's Extension of Helley's Theorem}
\begin{center}
	\textbf{PROBLEMS}
\end{center}
\begin{enumerate}
	\setcounter{enumi}{0}
    \item For $X$ a normed linear space with closed unit ball $B$, suppose the function $f:B\to[-1,1]$ has the property that whenever $u,v,u+v,\lambda u$ belong to $B$, $f(u+v)=f(u)+f(v)$ and $f(\lambda u)=\lambda f(u)$.
    Show that $f$ is the restriction to $B$ of a linear functional on all of $X$ which belongs to the closed unit ball of $X^*$.
    \item Let $X$ be a normed linear space and $K$ be a bounded convex weak-$*$ closed subset of $X^*$. Show that $K$ possesses an extreme point.
    \item Show that any nonempty weakly open set in an infinite dimensional normed linear spae is unbounded with respect to the norm.
    \item Use the Baire Category Theorem and the preceding problem to show that the weak topology on an infinite dimensional Banach space is not metrizable by a complete metric.
    \item Is every Banach space isomorphic to the dual of a Banach space?
\end{enumerate}

% 15.2
\section{Reflexivity and Weak Compactness: Kakutani's Theorem}
\begin{center}
	\textbf{PROBLEMS}
\end{center}
\begin{enumerate}
	\setcounter{enumi}{5}
    \item Show that every weakly compact subset of a normed linear space is bounded with respect to the norm.
    \item Show that the closed unit ball $B^*$ of the dual $X^*$ of a Banach space $X$ has an extreme point.
    \item Let $\mathcal{T}_1$ and $\mathcal{T}_2$ be two compact, Hausdorff topologies on a set $\mathcal{S}$ for which $\mathcal{T}_1\subseteq\mathcal{T}_2$. Show that $\mathcal{T}_1=\mathcal{T}_2$.
    \item Let $X$ be a normed linear space containing the subspace $Y$. For $A\subseteq Y$, show that the weak topology on $A$ induced by $Y^*$ is the same as the topology $A$ inherits as a subspace of $X$ with its weak topology.
    \item Argue as follows to show that q Banach space $X$ is reflexive iff its dual space $X^*$ is reflexive.
    \begin{enumerate}[label=(\roman*),align=left]
        \item If $X$ is reflexive, show that the weak and weak-$*$ topologies on $B^*$ are the same, and infer from this that $X^*$ is reflexive.
        \item If $X^*$ is reflexive, use part (i) and Proposition 15 of Chapter 14 to show that $X$ is reflexive.
    \end{enumerate}
    \item For $X$ a Banach space, by the preceding problem, if $X$ is reflexive, then so is $X^*$. Conclude that $X$ is not reflexive if there is a closed subspace of $X^*$ that is not reflexive.
    Let $K$ be an infinite compact Hausdorff space and $\{x_n\}$ an enumeration of a countably infinite subset of $K$. Define the operator $T:l^1\to[C(K)]^*$ by
    \[
        [T(\{n_k\})](f)=\sum_{k=1}^\infty \eta_k\cdot f(x_k)\text{ for all }\{\eta_k\}\in l^1\text{ and }f\in C(k).
    \]
    Show that $T$ is an isometry and therefore, since $l^1$ is not reflexive, neither is $T(l^1)$ and therefore neither is $C(K)$. Use a dimension counting argument to show that $C(K)$ is reflexive if $K$ is a finite set.
    \item If $Y$ is a linear subspace of a Banach space $X$, we define the \textit{annihilator} $Y^\perp$ to be the subspace of $X^*$ consisting of those $\psi\in X^*$ for which $\psi=0$ on $Y$.
    If $Y$ is a subspace of $X^*$, we define $Y^0$ to be the subspace of vectors in $X$ for which $\psi(x)=0$ for all $\psi\in Y$.
    \begin{enumerate}[label=(\roman*),align=left]
        \item Show that $Y^\perp$ is a closed linear subspace of $X^*$.
        \item Show that $(Y^\perp)^0=\overline Y$.
        \item If $X$ is reflexive and $Y$ is a subspace of $X^*$, show that $Y^\perp=J(Y^0)$.
    \end{enumerate}
\end{enumerate}

% 15.3
\section{Compactness and Weak Sequential Compactness: The Eberlein-\v Smulian Theorem}
\begin{center}
	\textbf{PROBLEMS}
\end{center}
\begin{enumerate}
	\setcounter{enumi}{12}
    \item In a general topological space that is not metrizable a sequence may converge to more than one point. Show that this cannot occur for the $W$-weak topology on a normed linear space $X$, where $W$ is a subspace of $X^*$ that separates points in $X$.
    \item Show that there is a bounded sequence in $L^\infty[0,1]$ that fails to have a weakly convergent subsequence. Show that the closed unit ball of $C[a,b]$ is not weakly compact.
    \item Let $K$ be a compact metric space with infinitely many points. Show that there is a bounded sequence in $C(K)$ that fails to have a weakly convergent subsequence (see Problem 11), but every bounded sequence of continuous linear functionals on $C(K)$ has a subsequence that converges pointwise to a continuous linear functional on $C(K)$.
\end{enumerate}

% 15.4
\section{Metrizability of Weak Topologies}
\begin{center}
	\textbf{PROBLEMS}
\end{center}
\begin{enumerate}
	\setcounter{enumi}{15}
    \item Show that the dual of an infinite dimensional normed linear space also is infinite dimensional.
    \item Complete the last step of the proof of Theorem 10 by showing that the inequalities (12) imply that the metric $\rho$ induces the $W$-weak topology.
    \item Let $X$ be a Banach space, $W$ a closed subspace of its dual $X^*$, and $\psi_0$ belong to $X^*\setminus W$.
    Show that if either $W$ is finite dimensional or $X$ is reflexive, then there is a vector $x_0$ in $X$ for which $\psi_0(x_0)\neq0$ but $\psi(x_0)=0$ for all $\psi\in W$.
    Exhibit an example of an infinite dimensional closed subspace $W$ of $X^*$ for which this separation property fails.
\end{enumerate}
