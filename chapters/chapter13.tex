% Chapter 13
\authoredby{finished}
\chapter{Continuous Linear Operators Between Banach Spaces}

% 13.1
\authoredby{finished}
\section{Normed Linear Spaces}

\begin{namedthm*}{Definition}
    Two norms $\|\cdot\|_1$ and $\|\cdot\|_2$ on a linear space $X$ are said to be \textbf{equivalent} provided there are constants $c_1,c_2\ge0$ for which
    \[
        c_1\|x\|_1\le\|x\|_2\le c_2\|x\|_1\quad\text{for all }x\in X.
    \]
    We immediately see that two norms are equivalent iff their induced metrics are equivalent.
\end{namedthm*}

Given vectors $x_1,\dots,x_n$ in a linear space $X$ and real numbers $\lambda_1,\dots,\lambda_n$, the vector
\[
    x=\sum_{k=1}^n\lambda_kx_k
\]
is called a \textbf{linear combination} of the $x_i$'s.
A nonempty subset $Y$ of $X$ is called a \textbf{linear subspace}, or simply a subspace, provided every linear combination of vectors in $Y$ also belongs to $Y$.
For a nonempty subset $S$ of $X$, by the \textbf{span} of $S$ we mean the set of all linear combinations of vectors in $S$, we denote this by $\text{span }S$.

For any two nonempty subspaces $Y,Z$ of $X$, then the sum $Y+Z=\{y+z\mid y\in Y,z\in Z\}$ is also a subspace of $X$.
In the case $Y\cap Z=\{0\}$, we denote $Y+Z$ by $Y\oplus Z$ and call this subspace of $X$ the \textbf{direct sum} of $Y$ and $Z$.

Almost all the important theorems for metric space require completeness:
\begin{namedthm*}{Definition}
    A normed linear space is called a \textbf{Banach space} provided it is complete as a metric space with the metric induced by the norm.
\end{namedthm*}
The Riesz-Fischer Theorem tells us that for $E$ a measurable set of real numbers and $1\le p\le\infty$, $L^p(E)$ is a Banach space.
Also, for $X$ a compact topological space, $C(X)$ with the maximum norm is a Banach space (Chapter 7 Problem 31).
From the Completeness Axiom for $\mathbb{R}$, each Euclidean space $\mathbb{R}^n$ is a Banach space.

\begin{center}
	\textbf{PROBLEMS}
\end{center}
\begin{enumerate}
	\setcounter{enumi}{0}
    \item Show that a nonempty subset $S$ of a linear space $X$ is a subspace iff $S+S=S$ and $\lambda \cdot S=S$ for each $\lambda\in\mathbb{R},\lambda\neq0$.
    
    \ \\Let $X$ be a linear space and consider the nonempty subset $S$.

    \ \\$(\implies)$ Suppose that $S$ is a subspace of $X$.
    Then by definition of subspace, every linear combination of vectors in $S$ also belongs to $S$.
    In particular, 
    \begin{align*}
        x&\in S+S=\{y+z\mid y\in S,z\in S\}\implies \text{ $x$ is a linear combination of vectors in $S$ }\implies x\in S\\
        x&\in S\implies x=x+0\in S+S\text{ because }0\in S.
    \end{align*}
    Then the first implies $S+S\subset S$ and the second implies $S\subset S+S$.
    Therefore $S=S+S$.
    \begin{align*}
        x&\in \lambda S=\{\lambda y\mid y\in S\}\implies \text{ $x$ is a linear combination of vectors in $S$ }\implies x\in S\\
        x&\in S\implies x=\lambda \frac{x}{\lambda}\in \lambda S\text{ because }\frac{x}{\lambda}\in S.
    \end{align*}
    Then the first implies $\lambda S\subset S$ and the second implies $S\subset \lambda S$.
    Therefore $\lambda S=S$.

    \ \\$(\impliedby)$ Suppose that $S+S=S$ and $\lambda\cdot S=S$ for each nonzero $\lambda\in\mathbb{R}$.
    Consider any $y,z\in S$ and scalars $\lambda_1,\lambda_2$. 
    Then 
    \[
        \lambda_1 y+\lambda_1z\in \lambda_1 S+\lambda_2 S=S+S=S.
    \]
    Therefore any linear combination of vectors in $S$ also belongs to $S$, and thus $S$ is a subspace of $X$.
    \ \\\item If $Y$ and $Z$ are subspaces of the linear space $X$, show that $Y+Z$ is also a subspace and $Y+Z=\text{span}[Y\cup Z]$. 

    \ \\Let $x_1,x_2\in Y+Z$, and let $\alpha,\beta$ be any scalars.
    Then $x_1=\alpha_1y_1+\beta_1z_1$ and $x_2=\alpha_2y_2+\beta_2z_2$.
    Therefore the linear combination of $x_1$ and $x_2$ are in $Y+Z$:
    \[
        \alpha x_1+\beta x_2
        =\alpha(\alpha_1y_1+\beta_1z_1)+\beta(\alpha_2y_2+\beta_2z_2)
        =(\alpha\alpha_1)y_1+(\beta\alpha_2)y_2+(\alpha\beta_1)z_1+(\beta\beta_2)z_2
        \in Y+Z,
    \]
    because $(\alpha\alpha_1)y_1+(\beta\alpha_2)y_2\in Y$ and $(\alpha\beta_1)z_1+(\beta\beta_2)z_2\in Z$.

    Also, $x\in Y+Z$ trivially implies that $x\in\text{span}[Y\cup Z]$. 
    To show the other side, $x\in\text{span}[Y\cup Z]$ implies that $x$ is a linear combination of vectors in $Y\cup Z$.
    But we showed above that any linear combination of vectors in $Y\cup Z$ is in $Y+Z$, and so $x\in Y+Z$.
    Therefore $Y+Z=\text{span}[Y\cup Z]$.
    \ \\\item Let $S$ be a subset of a normed linear space $X$.
    \begin{enumerate}[label=(\roman*),align=left]
        \item Show that the intersection of a collection of linear subspaces of $X$ is also a linear subspace of $X$.
        
        \ \\Let $\{Y_\alpha\mid \alpha\in \mathcal{A}\}$ be a collection of linear subspaces of $X$.
        Consider $x_1,x_2\in \cap_\alpha Y_\alpha$ and scalars $\alpha,\beta$.
        (Then $x_1,x_2\in Y_\alpha$ for each $\alpha$ by definition of intersection.)
        Now we have that $\alpha x_1+\beta x_2\in Y_\alpha$ for each $\alpha\in\mathcal{A}$ because each $Y_\alpha$ is a subspace.
        Therefore $\alpha x_1+\beta x_2\in Y_\alpha\in\cap_\alpha Y_\alpha$ by definition of intersection.
        Thus we have that $\cap_\alpha Y_\alpha$ is a subspace of $X$ because it contains all linear combinations of its elements.
        \\\item Show that span$[S]$ is the intersection of all the linear subspaces of $X$ that contain $S$ and therefore is a linear subspace of $X$.
        
        \ \\Let $\mathcal{S}=\{Y\text{ subspace of }X,S\subset Y\}$.
        Consider $\cap\mathcal{S}$, the intersection of all elements of $\mathcal{S}$.
        Then by (i), $\cap\mathcal{S}$ is a subspace of $X$.
        Recall that $\text{span}[S]$ is the collection of all linear combinations of elements of $S$.
        It is clear to see that $\text{span}[S]$ is a linear subspace of $X$ such that $S\subset\text{span}[S]$, and therefore $\text{span}[S]\in\mathcal{S}$.
        Then by definition of intersection, 
        \[
            \cap\mathcal{S}\subset \text{span}[S].\tag{a}
        \]
        Also, for any $Y\in\mathcal{S}$, any linear combination of elements of $S$ is in $Y$, which implies
        \[
            \text{span}[S]\subset\cap\mathcal{S}.\tag{b}
        \]
        Therefore by (a) and (b), $\text{span}[S]=\cap\mathcal{S}$.
        \\\item Show that $\overline{\text{span}}[S]$ is the intersection of all the closed linear subspaces of $X$ that contain $S$ and is therefore a closed linear subspace of $X$. 

        \ \\We can simply use (i) and (ii) along with the additional fact that any arbitrary intersection of closed sets (subspaces) is also a closed set (subspace).
    \end{enumerate}
    \ \\\item For a normed linear space $X$, show that the function $\|\cdot\|:X\to\mathbb{R}$ is continuous.\\
    \\Fix $\epsilon>0$.
    \\Then let $\delta:=\epsilon>0$.
    \\Consider any $x,y\in X$ such that $\|x-y\|<\delta$.
    \\Then by the reverse triangle inequality ($\ast$),
    \[
        |\|x\|-\|y\||\le\|x-y\|<\delta=\epsilon,
    \]
    and $\|\cdot\|$ is continuous.\\
    \\($\ast$) Proof of reverse triangle inequality:
    \[
        \|x\|=\|x-y+y\|\le\|x-y\|+\|y\|.
    \]
    \ \\\item For two normed linear spaces $(X,\|\cdot\|_1)$ and $(Y,\|\cdot\|_2)$, define a linear structure on the Cartesian product $X\times Y$ by $\lambda\cdot(x,y)=(\lambda x,\lambda y)$ and $(x_1,y_1)+(x_2,y_2)=(x_1+x_2,y_1+y_2)$.
    Define the product norm $\|\cdot\|$ by $\|(x,y)\|=\|x\|_1+\|y\|_2$, for $x\in X$ and $y\in Y$.
    Show that this is a norm with respect to which a sequence converges if and only if each of the two component sequences converges.
    Furthermore, show that if $X$ and $Y$ are Banach spaces, then so is $X\times Y$.\\
    \\Let $(X\times Y,\|\cdot\|)$ be a normed linear space.\\
    \\$(\implies)$ Let $\{(x_n,y_n)\}$ be a sequence in $X\times Y$, and suppose that it converges to some $(x,y)\in X\times Y$ with respect to the norm $\|\cdot\|$.\\
    \\Fix $\epsilon>0$.
    \\Then there exists an index $N$ such that for all $n\ge N$,
    \[
        0\le\|x_n-x\|_1+\|y_n-y\|_2=\|(x_n-x,y_n-y)\|=\|(x_n,y_n)-(x,y)\|<\epsilon,
    \]
    which implies that $\{x_n\}\to x$ w.r.t. the norm $\|\cdot\|_1$ and $\{y_n\}\to y$ w.r.t. the norm $\|\cdot\|_2$.\\
    \\$(\impliedby)$ Let $\{x_n\}$ be a sequence in $X$, and $\{y_n\}$ be a sequence in $Y$, and suppose that there exist $x\in X$, $y\in Y$ such that $\{x_n\}\to x$ w.r.t. the norm $\|\cdot\|_1$ and $\{y_n\}\to y$ w.r.t. the norm $\|\cdot\|_2$.\\
    \\Fix $\epsilon>0$.
    \\Then there exists an index $N_x$ such that for all $n\ge N_x$,
    \[
        \|x_n-x\|_1<\frac{\epsilon}{2},
    \]
    and there also exists an index $N_y$ such that for all $n\ge N_y$,
    \[
        \|y_n-y\|_1<\frac{\epsilon}{2}.
    \]
    Thus for all $n\ge\max\{N_x,N_y\}$,
    \[
        \|(x_n,y_n)-(x,y)\|=\|(x_n-x,y_n-y)\|=\|x_n-x\|_1+\|y_n-y\|_2<\frac{\epsilon}{2}+\frac{\epsilon}{2}=\epsilon,
    \]
    and therefore the sequence $\{(x_n,y_n)\}$ in $X\times Y$ converges to $(x,y)\in X\times Y$ with respect to the norm $\|\cdot\|$.\\
    \\Finally, suppose that $X$ and $Y$ are Banach spaces.
    \\Let $\{(x_n,y_n)\}$ be any sequence in $X\times Y$ that is Cauchy.
    \\Then for any $\epsilon$, there exists an index $N$ such that for all $n,m\ge N$, then
    \[
        0\le\|x_n-x_m\|_1+\|y_n-y_m\|_2=\|(x_n-x_m,y_n-y_m)\|=\|(x_n,y_n)-(x_m,y_m)\|<\epsilon,
    \]
    which implies that the sequences $\{x_n\}$ and $\{y_n\}$ are also Cauchy.
    \\Then because both $X$ and $Y$ are Banach spaces, then $\{x_n\}\to x$ and $\{y_n\}\to y$ for some $x\in X$ and $y\in Y$, and therefore we proved in $(\impliedby)$ that $\{(x_n,y_n)\}\to(x,y)$, which implies that $X\times Y$ is a Banach space.
    \ \\\item Let $X$ be a normed linear space.
    \begin{enumerate}[(i)]
        \item Let $\{x_n\}$ and $\{y_n\}$ be sequences in $X$ such that $\{x_n\}\to x$ and $\{y_n\}\to y$.
        Show that for any real numbers $\alpha$ and $\beta$, $\{\alpha x_n+\beta y_n\}\to\alpha x+\beta y$.

        \ \\Because $\{x_n\}\to x$ and $\{y_n\}\to y$, choose an index $N$ such that for $n\ge N$, we have $\|x_n-x\|<\frac{\epsilon}{2|\alpha|}$ and $\|y_n-y\|<\frac{\epsilon}{2|\beta|}$ (assuming $|\alpha|,|\beta|>0$).
        Then by subadditivity and absolute homogeneity of the norm,
        \[
            \|\alpha x_n+\beta y_n-(\alpha x+\beta y)\|
            % =\|\alpha (x_n-x)+\beta (y_n-y)\|
            \le|\alpha|\| x_n-x\|+|\beta| \|y_n-y\|
            <|\alpha|\frac{\epsilon}{2|\alpha|}+|\beta|\frac{\epsilon}{2|\beta|}
            =\epsilon,
        \]
        which implies that $\{\alpha x_n+\beta y_n\}\to\alpha x+\beta y$.
        \\\item Use (i) to show that if $Y$ is a subspace of $X$, then its closure $\overline{Y}$ also is a linear subspace of $X$.
    
        \ \\(Recall that for a metric space $X$ and a subset $C\subset X$, then $x\in\overline{C}\iff$ there exists a sequence $\{x_n\}\text{ in }C\text{ such that }\{x_n\}\to x$.)
        
        \ \\Let $x,y\in\overline{Y}$.
        Then there exist sequences $\{x_n\}$ and $\{y_n\}$ in $Y$ such that $\{x_n\}\to x$ and $\{y_n\}\to y$.
        Let $\alpha,\beta$ be any two real numbers.
        Then $\alpha x_n+\beta y_n\in Y$ for each $n$, and by (i), we have $\{\alpha x_n+\beta y_n\}\to\alpha x+\beta y$. 
        Therefore $\alpha x+\beta y\in\overline{Y}$, and $\overline{Y}$ is a linear subspace because it contains all linear combinations of its elements.
        \\\item Use (i) to show that the vector sum is continuous from $X\times X$ to $X$ and scalar multiplication is continuous from $\mathbb{R}\times X$ to $X$.

        \ \\We use the product norm $\|(x,y)\|_P:=\|x\|_X+\|y\|_X$ (see problem 5).
        
        \ \\Define $f:X\times X\to X$ by $f(x,y)=x+y$.
        
        Fix $\epsilon>0$. 
        Then for $(x_1,y_1),(x_2,y_2)\in X\times X$ such that 
        \[
           \epsilon> \|(x_1,y_1)-(x_2,y_2)\|_P=\|(x_1-x_2,y_1-y_2)\|_P=\|x_1-x_2\|_X+\|y_1-y_2\|_X,
        \]
        we have
        \[
            \begin{split}
            \|f(x_1,y_1)-f(x_2,y_2)\|_X
            &=\|x_1+y_1-(x_2+y_2)\|_X\\
            &=\|x_1-x_2+y_1-y_2\|_X\\
            &\le\|x_1-x_2\|_X+\|y_1-y_2\|_X\\
            &<\epsilon,
            \end{split}
        \]
        and thus $f$ is continuous.

        \ \\Define $g:\mathbb{R}\times X\to X$ by $g(\alpha,x)=\alpha x$.
        
        Fix $\epsilon>0$. 
        Then let $\delta:=\min\left\{\frac{\epsilon}{2(\|x_1\|_X+1)},\frac{\epsilon}{2(|\alpha_2|+1)}\right\}>0$.
        Then for $(\alpha_1,x_1),(\alpha_2,x_2)\in \mathbb{R}\times X$ such that
        \[
           \delta> \|(\alpha_1,x_1)-(\alpha_2,x_2)\|_P=\|(\alpha_1-\alpha_2,x_1-x_2)\|_P=|\alpha_1-\alpha_2|+\|x_1-x_2\|_X,
        \]
        we have
        \[
            \begin{split}
            \|g(\alpha_1,x_1)-g(\alpha_2,x_2)\|_X
            &=\|\alpha_1 x_1-\alpha_2 x_2\|_X\\
            &=\|(\alpha_1 -\alpha_2)x_1+\alpha_2(x_1- x_2)\|_X\\
            &\le|\alpha_1 -\alpha_2|\|x_1\|_X+|\alpha_2|\|x_1- x_2\|_X\\
            &<\frac{\epsilon}{2(\|x_1\|_X+1)}\|x_1\|_X+|\alpha_2|\frac{\epsilon}{2(|\alpha_2|+1)}\\
            &<\frac{\epsilon}{2}+\frac{\epsilon}{2},
            \end{split}
        \]
        and thus $g$ is continuous.
    \end{enumerate} 
    \ \\\item Show that the set $\mathcal{P}$ of all polynomials on $[a,b]$ is a linear space.
    For $\mathcal{P}$ considered as a subset of the normed linear space $C[a,b]$, show that $\mathcal{P}$ fails to be closed.
    For $\mathcal{P}$ considered as a subset of the normed linear space $L^1[a,b]$, show that $\mathcal{P}$ fails to be closed.

    \ \\To see that $\mathcal{P}$ is a linear space, look at chapter 7 problem 2.
    
    Let $t\in(a,b)$ and consider the continuous function $f:[a,b]\to\mathbb{R}$ defined by
    \[
        f(x):=
        \begin{cases}
            0 &x\in[a, t]\\
            (\frac{1}{b-t})(x-t)&x\in(t,b]\\
        \end{cases}
    \]
    Then $f$ is not differentiable at $t$, and is thus not a polynomial.
    We can write $f\in C[a,b]$, $f\notin\mathcal{P}$.
    However, by Chapter 12.3 - The Stone-Weierstrass Theorem, there exists a sequence of polynomials $\{p_n\}$ in $\mathcal{P}$ that converges uniformly to $f\notin\mathcal{P}$.
    Therefore $\mathcal{P}$ is not closed.
    Because $\int_a^b|f|=\frac{b-t}{2}<\infty$, then $f\in L^1[a,b]$, and we can use the same argument to say that $\mathcal{P}$ is not closed.
    \ \\\item A nonnegative real-valued function $\|\cdot\|$ defined on a vector space $X$ is called a \textbf{pseudonorm} if $\|x+y\|\le\|x\|+\|y\|$ and $\|\alpha x\|=|\alpha|\|x\|$.
    Define $x\cong y$, provided $\|x-y\|=0$.
    Show that this is an equivalence relation.
    Define $X/\cong$ to be the set of equivalence classes of $X$ under $\cong$ and for $x\in X$ define $[x]$ to be the equivalence class of $x$.
    Show that $X/\cong$ is a normed vector space if we define $\alpha[x]+\beta[y]$ to be the equivalence class of $\alpha x+\beta y$ and define $\|[x]\|=\|x\|$.
    Illustrate this procedure with $X=L^p[a,b],1\le p<\infty$.

    \ \\To see that this is an equivalence relation:
    \begin{enumerate}[(i)]
        \item $x\cong x$ because $\|x-x\|=0$ for all $x\in X$.
        \item $x\cong y\iff0=\|x-y\|=\|y-x\|\iff y\cong x$.
        \item Suppose $x\cong y$ and $y\cong z$. 
        Then $\|x-z\|\le\|x-y\|+\|y-z\|=0$, which implies $x\cong z$.
    \end{enumerate}
\end{enumerate}

% 13.2
\authoredby{inprogress}
\section{Linear Operators}

\begin{namedthm*}{Definition}
    Let $X$ and $Y$ be linear spaces.
    A mapping $T:X\to Y$ is said to be \textbf{linear} provided for each $u,v\in X$, and real numbers $\alpha,\beta$, we have
    \[
        T(\alpha u+\beta v)=\alpha T(u)+\beta T(v).
    \]
\end{namedthm*}
Linear mappings are often called linear operators or linear transformations.
\begin{namedthm*}{Definition}
    Let $X$ and $Y$ be normed linear spaces.
    A linear operator $T:X\to Y$ is said to be \textbf{bounded} provided there is a constant $M\ge0$ for which 
    \[
        \|T(u)\|\le M\|u\|\quad\text{for all }u\in X.
    \]
    The infimum of all such $M$ is called the \textbf{operator norm} of $T$ and denoted by $\|T\|$.
    The collection of bounded linear operators from $X$ to $Y$ is denoted by $\mathcal{L}(X,Y)$.
\end{namedthm*}

\begin{namedthm*}{Theorem 1}
    A linear operator between normed linear spaces is continuous iff it is bounded.    
\end{namedthm*}
\begin{proof}
    Let $T:(X,\|\cdot\|_X)\to(Y,\|\cdot\|_Y)$ be a linear operator.\\
    \\$(\implies)$ Suppose that $T$ is continuous.
    \\Then for $\epsilon=1$, there exists a $\delta>0$ such that, for any $x\in X$ such that $\|x-0\|_X=\|x\|_X\le\delta$, then
    \[
        \|T(x)-T(0)\|_Y=\|T(x)\|_Y<1.
    \]
    (Where $T(0)=0$ by linearity.)
    \\Therefore consider any $u\in X$, $u\neq0$.
    \begin{align*}
        \|T(u)\|_Y&=\left\|T\left(\frac{\delta\cdot\|u\|_X}{\delta\cdot\|u\|_X}u\right)\right\|_Y\\
        &=\frac{\|u\|_X}{\delta}\left\|T\left(\frac{\delta}{\|u\|_X}u\right)\right\|_Y&&\text{by linearity of $T$ and absolute homogeneity of $\|\cdot\|_Y$.}\\
        &<\frac{\|u\|_X}{\delta}\cdot1,&&\text{because }\left\|\frac{\delta}{\|u\|_X}u\right\|_X=\frac{\delta\|u\|_X}{\|u\|_X}=\delta\le\delta.\\
    \end{align*}
    that is, there exists the positive constant $\frac{1}{\delta}$ such that 
    \[
        \|T(u)\|_Y\le\frac{1}{\delta}\|u\|_X\text{ for all }u\in X,
    \]
    which implies that $T$ is bounded.\\
    \\$(\impliedby)$ Suppose that $T$ is bounded.
    \\Then there exists an $M\ge0$ such that 
    $
        \|T(x)\|_Y\le M\|x\|_X\text{ for all }x\in X.
    $
    \\Fix $\epsilon>0$.
    \\Consider any $x,x'\in X$ such that $\|x-x'\|_X<\frac{\epsilon}{M+1}$.
    \\Then by linearity of $T$,
    \[
        \|T(x)-T(x')\|_Y=\|T(x-x')\|_Y\le M\|x-x'\|_X<M\frac{\epsilon}{M+1}<\epsilon,
    \]
    which implies that $T$ is continuous.
\end{proof}
The continuity property of a bounded linear operator $T:X\to Y$ says that:
\[
    \text{if }\{u_n\}\to u\text{ in }X,\text{ then }\{T(u_n)\}\to T(u)\text{ in }Y.
\]
The collection of linear operators between two linear spaces is a linear space. 
\begin{namedthm*}{Proposition 2}
    Let $X$ and $Y$ be normed linear spaces.
    Then the collection of bounded linear operators from $X$ to $Y$, $\mathcal{L}(X,Y)$, is a normed linear space.
\end{namedthm*}
\begin{proof}
    Let $T,S\in\mathcal{L}(X,Y)$, and let $\alpha,\beta$ be real numbers.
    Then using the norm $\|\cdot\|$ on $Y$, we have by subadditivity, absolute homogeneity, and the definition of the operator norm that 
    \[
        \|(\alpha T+\beta S)(u)\|=\|\alpha T(u)+\beta S(u)\|\le|\alpha|\| T(u)\|+|\beta|\| S(u)\|\le
        % |\alpha|\|T\|\|u\|+|\beta|\|S\|\|u\|
        (|\alpha|\|T\|+|\beta|\|S\|)\|u\|\quad\text{for all }u\in X,
    \]
    which implies that $\alpha T+\beta S$ is bounded (and clearly linear) and is thus in $\mathcal{L}(X,Y)$.
    To show that the operator norm is a norm on $\mathcal{L}(X,Y)$:
    \begin{enumerate}[(i)]
        \item $\|T\|=0\iff T(u)=0$ follows from $\|T\|=\inf\{M\ge0\mid \|T(u)\|\le M\|u\|,u\in X\}$
        \item $\|\alpha T\|=|\alpha|\|T\|$ because $\|\alpha T(u)\|=|\alpha|\|T(u)\|$ implies that 
        \[
            \|\alpha T\|=\inf\{M\mid |\alpha|\| T(u)\|\le M\|u\|\}=\inf\{M\mid \| T(u)\|\le \frac{M}{|\alpha|}\|u\|\}=|\alpha|\|T\|.
        \]
        \item $\|T+S\|\le\|T\|+\|S\|$ follows from above:
        \[
            \|(T+S)(u)\|%\le\|T(u)\|+\|S(u)\|
            \le (\|T\|+\|S\|)\|u\|,
        \]
        and therefore $\|T+S\|=\inf\{M\ge0\mid \|(T+S)(u)\|\le M\|u\|,u\in X\}\le \|T\|+\|S\|$.
    \end{enumerate}
\end{proof}
\begin{namedthm*}{Theorem 3}
    Let $X$ and $Y$ be normed linear spaces.
    If $Y$ is a Banach space, then so is $\mathcal{L}(X,Y)$.
\end{namedthm*}
\begin{proof}
    Let $\{T_n\}$ be a Cauchy sequence in $\mathcal{L}(X,Y)$.
    Consider any $u\in X$.
    Then for all indices $n,m$,
    \[
        \|T_n(u)-T_m(u)\|=\|(T_n-T_m)(u)\|\le\|T_n-T_m\|\|u\|.
    \]  
    Thus $\{T_n(u)\}$ is a Cauchy sequence in $Y$.
    Since $Y$ is complete, then the sequence $\{T_n(u)\}$ converges to a member of $Y$, which we can denote $T(u)$.
    This $T(u)$ is defined for any $u\in X$, and so $T:X\to Y$ is well-defined.
    We must show that $T$ belongs to $\mathcal{L}(X,Y)$ and that $\{T_n\}\to T$ in $\mathcal{L}(X,Y)$.
    
    To show linearity of $T$:
    Using the fact that $T_n(u)$ converges to $T(u)$ for any $u$, and that each $T_n$ is linear, for any $u_1,u_2\in X$ and scalars $\lambda_1,\lambda_2$, we have 
    \[
        \lambda_1T(u_1)+\lambda_2T(u_2)=\lambda_1\lim_{n\to\infty}T_n(u_1)+\lambda_2\lim_{n\to\infty}T_n(u_2)=\lim_{n\to\infty}T_n(\lambda_1u_1+\lambda_2u_2)=T(\lambda_1u_1+\lambda_2u_1).
    \]
    
    To show boundedness of $T$ and convergence of $\{T_n\}$ to $T$:
    Fix $\epsilon>0$.
    Because $\{T_n\}$ is a Cauchy sequence, there exists an index $N$ such that for all $n\ge N$ and $k\ge1$, we have $\|T_n-T_{n+k}\|<\epsilon/2$.
    Therefore for all $u\in X$, using the definition of operator norm,
    \[
        \|T_n(u)-T_{n+k}(u)\|=\|(T_n-T_{n+k})(u)\|\le\|T_n-T_{n+k}\|\|u\|<\epsilon/2\|u\|.
    \]  
    Now fix $n\ge N$ and $u\in X$.
    Because $\lim_{k\to\infty}T_{n+k}(u)=T(u)$ and the norm is continuous, we have
    \[
        \|T_n(u)-T(u)\|=\|T_n(u)-\lim_{k\to\infty}T_{n+k}(u)\|\le\epsilon/2\|u\|.
    \]  
    Thus the linear operator $T_N-T$ is bounded, and because $T_N$ is also bounded, so is $T$.
    Moreover, by the definition of the operator norm, we have for $n\ge N$, that 
    $
        \|T_n-T\|\le\epsilon/2<\epsilon,
    $
    and thus $\{T_n\}\to T$ in $\mathcal{L}(X,Y)$.

\end{proof}

For two normed linear spaces $X$ and $Y$, an operator $T\in\mathcal{L}(X,Y)$ is called an \textbf{isomorphism} provided it is one-to-one, onto, and has a continuous inverse.
For $T$ in $\mathcal{L}(X,Y)$, if it is one-to-one and onto, its inverse is linear.
Therefore (because bounded $\iff$ continuous for linear operators,) to be an isomorphism requires the inverse to be bounded; that is, the inverse belong to $\mathcal{L}(X,Y)$.
Two normed linear spaces are said to be \textbf{isomorphic} provided there is an isomorphism between them.
An isomorphism that also preserves the norm is called an \textbf{isometric isomorphism}: it is an isometry of the metric structures associated with the norms.

For a linear operator $T:X\to Y$, the subspace of $X$, $\{x\in X\mid T(x)=0\}$, is called the \textbf{kernel} of $T$ and denoted by $\text{ker }T$.
Observe that $T$ is one-to-one iff $\text{ker }T=\{0\}$.
We denote the \textbf{image} of $T$, $T(X)$, by $\text{Im }T$.

\begin{center}
	\textbf{PROBLEMS}
\end{center}
\begin{enumerate}
	\setcounter{enumi}{8}
    \item Let $X$ and $Y$ be normed linear spaces and $T:X\to Y$ be linear.
    \begin{enumerate}[(i)]
        \item Show that $T$ is continuous iff it is continuous at a single point $u_0$ in $X$.
        
        \ \\$(\implies)$ Suppose that $T$ is continuous.
        Then it is trivial that $T$ is continuous for any point $u_0$ in $X$.

        \ \\$(\impliedby)$ Suppose that $T$ is continuous at the point $u_0$ in $X$.
        Then there exists a $\delta>0$ such that for any $x\in X$ such that $\|u_0-x\|_X<\delta$, then $\|Tu_0-Tx\|_X<\epsilon$. 
        Let $y:=u_0-x$, and by linearity, we have
        \[
            \|y-0\|_X=\|y\|_X<\delta\implies \|Ty-T0\|_Y=\|Ty\|_Y<\epsilon,
        \]
        which tells us that $T$ is continuous at $0$.
        Now consider any point $x'\in X$.
        Then by continuity at zero, there exists a $\delta>0$ such that 
        \[
            \text{ for }x\in X\text{ s.t. }\|x'-x\|_X=\|(x'-x)-0\|_X<\delta\implies\|T(x'-x)-T0\|_Y=\|Tx'-Tx\|_Y<\epsilon.
        \]
        Therefore $T$ is continuous at every point.
        \ \\\item Show that $T$ is Lipschitz iff it is continuous.
        \\$(\implies)$ Suppose that $T$ is Lipschitz with Lipschitz constant $L\ge0$.
        Then for any $x\in X$, by linearity of $T$, we have
        \[  
            \|Tx\|=\|Tx-T0\|\le L\|x-0\|=L\|x\|,
        \]
        and $T$ is bounded and thus continuous.
        \\$(\impliedby)$ Suppose that $T$ is continuous.
        Then $T$ is bounded and there exists an $M\ge0$ for which 
        \[  
            \|Tx-Ty\|=\|T(x-y)\|\le M\|x-y\|,
        \]
        and thus $T$ is Lipschitz.
        \\\item Show that neither $(i)$ nor $(ii)$ hold in the absence of the linearity assumption on $T$. 

        \ \\$(i)$ The function $f(x):=1$ for $x\ge0$ and $f(x):=0$ for $x<0$ is continuous at the point $2$ but not continuous on all of $X$ (namely, $0$).
        \ \\$(ii)$ The function $g(x):=\sqrt{|x|}$ is continuous but it is not Lipschitz.
    \end{enumerate}
    \ \\\item For $X$ and $Y$ normed linear spaces and $T\in\mathcal{L}(X,Y)$, show that $\|T\|$ is the smallest Lipschitz constant for the mapping $T$; that is, the smallest number $c\ge0$ for which
    \[
        \|T(u)-T(v)\|\le c\cdot\|u-v\|\text{ for all }u,v\in X.
    \]
    \\Consider any $x,y\in X$, and consider the vector $(x-y)\in X$.
    \\Suppose that $T$ is Lipschitz; that is, there exists a $c\ge0$ such that 
    \[
        \|T(u-v)\|_Y=\|T(u)-T(v)\|_Y\le L\cdot\|u-v\|_X.
    \]
    Because $T$ is a Lipschitz function it is thus continuous (previous Problem 9(ii)), and because $T$ is linear, it is also thus bounded (Theorem 1), and so the operator norm of $T$ is well-defined.
    In particular, $\|T\|$ is the infimum of all such $c\ge0$.
    \ \\\item For $X$ and $Y$ normed linear spaces and $Y\in\mathcal{L}(X,Y)$, show that 
    \[
        \|T\|=\sup\{\|T(u)\|\mid u\in X, \|u\|\le1\}.
    \]

    \ \\Define 
    \[
        \begin{split}
        M'&:=\inf\{M\ge0\mid\|T(u)\|\le M\|u\|\text{ for all }u\in X\},\\
        N'&:=\sup\{\|T(u)\|\mid u\in X, \|u\|\le1\}.
        \end{split}
    \]
    We aim to show that they are equal.
    
    First, for $u\neq0$, then $\|\frac{u}{\|u\|}\|\le1$ so that by linearity of $T$,
    \[\|T(\frac{u}{\|u\|})\|\le N'\implies\|Tu\|\le N'\|u\|\implies M'\le N'.\]
    On the other hand, for $u$ such that $\|u\|\le1$, then
    \[\|Tu\|\le M'\|u\|\le M'\implies N'\le M.\]
    Therefore $M'=N'$.
    \ \\\item For $X$ and $Y$ normed linear spaces, let $\{T_n\}\to T$ in $\mathcal{L}(X,Y)$ and $\{u_n\}\to u$ in $X$.
    Show that $\{T_n(u_n)\}\to T(u)$ in $Y$.

    \ \\
    \item Let $X$ be a Banach space and $T\in\mathcal{L}(X,Y)$ have $\|T\|<1$.
    \begin{enumerate}[(i)]
        \item Use the Contraction Mapping Principle to show that $I-T\in\mathcal{L}(X,Y)$ is one-to-one and onto.
        \item Show that $I-T$ is an isomorphism.
    \end{enumerate}
    \item (Neumann Series) Let $X$ be a Banach space and $Y\in\mathcal{L}(X,Y)$ have $\|T\|<1$.
    Define $T^0=Id$.
    \begin{enumerate}[(i)]
        \item Use the completeness of $\mathcal{L}(X,X)$ to show that $\sum_{n=1}^\infty T^n$ converges in $\mathcal{L}(X,X)$.
        \item Show that $(I-T)^{-1}=\sum_{n=0}^\infty T^n$
    \end{enumerate}
    \item For $X$ and $Y$ normed linear spaces and $T\in\mathcal{L}(X,Y)$, show that $T$ is an isomorphism iff there is an operator $S\in\mathcal{L}(Y,X)$ such that for each $u\in X$ and $v\in Y$,
    \[
        S(T(u))=u\text{ and }T(S(v))=v.
    \]
    \item For $X$ and $Y$ normed linear spaces and $T\in\mathcal{L}(X,Y)$, show that $\text{ker }T$ is a closed subspace of $X$ and that $T$ is one-to-one iff $\text{ker }T=\{0\}$.
    \item Let $(X,\rho)$ be a metric space containing the point $x_0$.
    Define $\text{Lip}_0(X)$ to be the set of real-valued Lipschitz functions $f$ on $X$ that vanish at $x_0$.
    Show that $\text{Lip}_0(X)$ is a linear space that is normed by defining, for $f\in\text{Lip}_0(X)$,
    \[
        \|f\|=\sup_{x\neq y}\frac{|f(x)-f(y)|}{\rho(x,y)}.
    \]
    Show that $\text{Lip}_0(X)$ is a Banach space.
    For each $x\in X$, define the linear functional $F_x$ on $\text{Lip}_0(X)$ by setting $F_x(f)=f(x)$.
    Show that $F_x$ belongs to $\mathcal{L}(\text{Lip}_0(X),\mathbb{R})$ and that for $x,y\in X$, $\|F_x-F_y\|=\rho(x,y)$.
    Thus $X$ is isometric to a subset of the Banach space $\mathcal{L}(\text{Lip}_0(X),\mathbb{R})$.
    Since any closed subset of a complete metric space is complete, this provides another proof of the existence of a completion for any metric space $X$.
    It also shows that any metric space is isometric to a subset of a normed linear space.
    \item Use the preceding problem to show that every normed linear space is a dense subspace of a Banach space.
    \item For $X$ a normed linear space and $T,S\in\mathcal{L}(X,X)$, show that the composition $S\circ T$ also belongs to $\mathcal{L}(X,X)$ and $\|S\circ T\|\le\|S\|\cdot\|T\|$.
    \item Let $X$ be a normed linear space and $Y$ a closed linear subspace of $X$.
    Show that $\|x\|_1=\inf_{y\in Y}\|x-y\|$ defines a pseudonorm on $X$.
    The normed linear space induced by the pseudonorm $\|\cdot\|_1$ (see Problem 8) is denoted by $X/Y$ and called the \textbf{quotient space} of $X$ modulo $Y$.
    Show that the natural map $\varphi$ of $X$ onto $X/Y$ takes open sets into open sets.
    \item Show that if $X$ is a Banach space and $Y$ a closed linear proper subspace of $X$, then the quotient $X/Y$ also is a Banach space and the natural map $\varphi:X\to X/Y$ has norm 1.
    \item Let $X$ and $Y$ be normed linear spaces, $T\in\mathcal{L}(X,Y)$ and $\text{ker }T=Z$.
    Show that there is a unique bounded linear operator $S$ from $X/Z$ onto $Y$ such that $T=S\circ\varphi$ where $\varphi:X\to X/Z$ is the natural map.
    Moreover, show that $\|T\|=\|S\|$.
\end{enumerate}

% 13.3
\authoredby{inprogress}
\section{Compactness Lost: Infinite Dimensional Normed Linear Spaces}

A linear space $X$ is said to be finite dimensional provided there is a subset $\{e_1,\dots,e_n\}$ of $X$ that spans $X$.
If no proper subset also spans $X$, we call the set $\{e_1,\dots,e_n\}$ a basis for $X$ and call $n$ the dimension of $X$.
If $X$ is not spanned by a finite collection of vectors it is said to be finite dimensional.
Observe that a basis $\{e_1,\dots,e_n\}$ for $X$ is linearly independent in the sense that 
\[
    \text{if }\sum_{i=1}^nx_ie_i=0,\ \text{then }x_i=0\ \text{for all }1\le i\le n,
\]
for otherwise a proper subset of $\{e_1,\dots,e_n\}$ would span $X$.

For example, if there exists a nontrivial solution to the above homogenous equation, then there exists an index $j$ such that
\[
    \sum_{i=1,i\neq j}^n \frac{x_i}{-x_j}e_i=e_j,
\]
and so for any $y\in X$, we can essentially remove the $e_j$: 
\[
    y=\sum_{i=1}^ny_ie_i= \sum_{i=1,i\neq j}^ny_ie_i+y_je_j=\sum_{i=1,i\neq j}^ny_ie_i+\sum_{i=1,i\neq j}^n y_j\frac{x_i}{-x_j}e_i=\sum_{i=1,i\neq j}^n(y_i-y_j\frac{x_i}{x_j})e_i
\]
That is, the proper subset $\{e_1,\dots,e_n\}\setminus \{e_j\}$ spans $X$.
\begin{namedthm*}{Theorem 4}
    Any two norms on a finite dimensional linear space are equivalent.
\end{namedthm*}
\begin{proof}
    Since equivalnece of norms is an equivalence relation on the set of norms on $X$, it suffices to select a particular norm $\|\cdot\|_*$ on $X$ and show that any norm on $X$ is equivalent to $\|\cdot\|_*$.
    Let $\text{dim} X=n$ and $\{e_1,\dots,e_n\}$ be a basis for $X$.
    For any $x=x_1e_1+\dots x_ne_n\in X$, define
    \[
        \|x\|_*=\sqrt{x_1^2+\dots+x_n^2}.
    \]
    This is a norm on $X$ because the Euclidean norm is a norm on $\mathbb{R}^n$.

    Now let $\|\cdot\|$ be any norm on $X$.
    We must show the existance of $c_1,c_2\ge0$ such that (1) $ \|x\|\le c_1\|x\|_*$ and (2) $ c_2\|x\|_*\le \|x\|$ for all $x\in X$.
    
    (1) Consider any $x=\sum_{i=1}^n x_ie_i\in X$, where $\{e_i\}$ is a basis for $X$.
    Then by subadditivity and absolute homogeneity of $\|\cdot\|$, and by Cauchy-Schwarz, we get
    \[
      \|x\|
      \le\sum_{i=1}^n|x_i|\|e_i\|
    %   =\sum_{i=1}^n|x_i|\sqrt{\|e_i\|^2}
    %   \le\sum_{i=1}^n|x_i|\sqrt{\sum_{j=1}^n\|e_j\|^2}
      \le\left(\sqrt{\sum_{j=1}^n\|e_j\|^2}\right)\sqrt{\sum_{j=1}^n|x_i|^2}
      =\left(\sqrt{\sum_{j=1}^n\|e_j\|^2}\right)\|x\|_*
      :=c_1\|x\|_*\tag{1}
    \]
    (2) Now define the real-valued function $f:\mathbb{R}^n\to\mathbb{R}$ by
    \[
        f(x_1,\dots,x_n):=\|\sum_{i=1}^n x_ie_i\|
    \]
    Then by (1), we see that $f$ is Lipschitz continuous with respect to the topology induced by the $\|\cdot\|_*$ norm on $\mathbb{R}^n$:
    \[
      |f(x)-f(x')|=|\sum_{i=1}^n (x_i-x_i')e_i|=\|x-x'\|\le c_1\|x-x'\|_*
    \]
    Because $\{e_1,\dots,e_n\}$ is linearly independent, then $f(x)=\|\sum_{i=1}^n x_ie_i\|=\|0\|=0\iff x_i=0$ for each $i$, which tells us that $f$ takes positive values on the boundary of the unit ball $S=\{x\in\mathbb{R}^n\mid \|x\|_*=(\sum_{i=1}^n x_i^2)^{1/2}=1\}$, which is compact because it is closed and bounded.
    A continuous real-valued function on a compact topological space takes a minimum value.
    Therefore there exists the value $m>0$ and the point $x_m\in S$ such that $f(x_m)=m$, and 
    \[
        0<m=f(x_m)\le f(x)\quad\text{for all }x\in S.
    \]
    Now, for any $x\neq0$, we have $\frac{x}{\|x\|_*}\in S$, so that 
    $
        m\le f(\frac{x}{\|x\|_*})=\|\frac{x}{\|x\|_*}\|,
    $
    and by absolute homogeneity of the norm $\|\cdot\|$, we have
    \[
        m\|x\|_*\le \|x\|\quad\text{for all }x\in X.\tag{2}
    \]  
    Therefore by (1) and (2), there exists $c_1,m\ge0$ such that 
    \[
        m\|x\|_*\le \|x\|\le c_1\|x\|_*\quad\text{for all }x\in X.
    \] 

\end{proof}
\begin{namedthm*}{Corollary 5}
    Any two normed space of the same finite dimension are isomorphic.
\end{namedthm*}
\begin{namedthm*}{Corollary 6}
    Any finite dimensional normed linear space is complete and therefore any finite dimensional subspace of a normed linear space is closed.
\end{namedthm*}
``completeness is preserved under isomorphisms''
\begin{namedthm*}{Corollary 7}
    The closed unit ball in a finite dimensional normed linear space is compact.
\end{namedthm*}
\begin{namedthm*}{Riesz's Theorem}
    The closed unit ball of a normed linear space $X$ is compact iff $X$ is finite dimensional.
\end{namedthm*}
\begin{namedthm*}{Riesz's Lemma}
    Let $Y$ be a closed proper linear subspace of a normed linear space $X$.
    Then for each $\epsilon>0$, there is a unit vector $x_0\in X$ for which 
    \[
        \|x_0-y\|>1-\epsilon\text{ for all }y\in Y.
    \]
\end{namedthm*}
\begin{proof}[Proof of Riesz's Theorem]
    By Corollary 7, the closed unit ball in a finite dimensional normed linear space is compact.
    It remains to show that $B$ fails to be compact if $X$ is finite dimensional.
    Assume $X$ is finite dimensional.
    We will inductively choose a sequence $\{x_n\}$ in $B$ such that $\|x_n-x_m\|>1/2$ for $n\neq m$.
    This sequence has no Cauchy subsequence and therefore no convergent subsequence.
    Thus $B$ is not sequentially compact (see chapter 9.5), and therefore, since $B$ is a metric space, not compact.

    It remains to choose this subsequence.
    Choose any vector $x_1\in B$.
    For a natural number $n$, suppose we have chosen $n$ vectors in $B$, $\{x_1,\dots,x_n\}$, each pair of which are more than a distance $1/2$ apart.
    Let $X_n$ be the linear space spanned by these $n$ vectors.
    Then $X_n$ is a finite dimensional subspace of $X$, and so by Corollary 6, it is closed.
    Moreover, $X_n$ is a proper subspace of $X$ since $\text{dim }X=\infty$.
    By the preceding lemma we may choose $x_{n+1}$ in $B$ such that $\|x_{n+1}-x_i\|>1/2$ for all $1\le i\le n$.
    Thus we have inductively chosen a sequence in $B$, any two terms of which are more than a distance $1/2$ apart.
\end{proof}

\begin{center}
	\textbf{PROBLEMS}
\end{center}
\begin{enumerate}
	\setcounter{enumi}{22}
    \item Show that a subset of a finite dimensional normed linear space $X$ is compact iff it is closed and bounded.
    \\\item Complete the proof of Riesz's Lemma for $\epsilon\neq1/2$.
    
    \ \\The case where $\epsilon\ge1$ holds trivially, so consider $0<\epsilon<1$.
    Since $Y$ is a proper subset of $X$, there exists $x\in X\setminus Y$.
    Then, because $Y$ is closed, then $X\setminus Y$ is open, so that there exists an open ball centered at $x$ that is disjoint from $Y$.
    That is, 
    % That is, $y\notin\mathbb{B}(x,\delta)=\{z\in X\mid \|x-z\|<\delta\}\implies$
    \[
        0<d:=\inf\{\|x-y\|, y\in Y\}.\tag{1}
    \]
    Because $1-\epsilon<1$, by definition of infimum, there exists a vector $y_1\in Y$ for which 
    \[
        d\le\|x-y_1\|<\frac{d}{1-\epsilon}.\tag{2}
    \]
    Now define 
    \[
        x_0:=\frac{x-y_1}{\|x-y_1\|},
    \]
    so that $x_0$ is a unit vector.
    Moreover, for any $y\in Y$, we have
    \[
        x_0-y=\frac{x-(y_1+y\|x-y_1\|)}{\|x-y_1\|}:=\frac{1}{\|x-y_1\|}(x-y'),
    \] 
    where $y'=y_1+y\|x-y_1\|$ is a linear combination of elements of $Y$, and so is in $Y$.
    
    Therefore by (1) and (2),
    \[
        \|x_0-y\|=\frac{1}{\|x-y_1\|}\|x-y'\|>\frac{1-\epsilon}{d}\|x-y'\|\ge1-\epsilon.
    \]
    \\\item Exhibit an open cover of the closed unit ball of $X=\ell^2$ that has no finite subcover.
    Then do the same for $X=C[0,1]$ and $X=L^2[0,1]$.
    \\\item For normed linear spaces $X$ and $Y$, let $T:X\to Y$ be linear.
    If $X$ is finite dimensional, show that $T$ is continuous. 
    If $Y$ is finite dimensional, show that $T$ is continuous iff $\text{ker }T$ is closed.

    From Chapter 13 Theorem 1, we showed that continuous is equivalent to bounded for the linear operator $T$.
    Therefore it is sufficient to show that $T$ is bounded.
    Because $X$ is finite dimensional of some dimension $n$, choose a normalized basis $\{e_i\}_{i\in[n]}$ of $X$.
    Thus for any $x\in X$, then $x=\sum_{i=1}^nx_ie_i$, and then by linearity, $T(x)=\sum_{i=1}^nx_if(e_i)$.
    By subadditivity and absolute homogeneity,
    \[
        \|f(x)\|_Y\le\sum_{i=1}^n|x_i|\|f(e_i)\|_Y.
    \]
    \begin{enumerate}
        \item Let $M:=\sup_{i\in[n]}\{\|f(e_i)\|_Y\}=\max_{i\in[n]}\{\|f(e_i)\|_Y\}$.
        \item Note that any two norms on a finite-dimensional space are equivalent, so that 
        \[
            \sum_{i=1}^n|x_i|=\|x\|_1\le C\|x\|_X\quad\text{for some }C>0.
        \]
    \end{enumerate}
    Therefore by (i) and (ii),
    \begin{align*}
        \|f(x)\|_Y
        &\le\sum_{i=1}^n|x_i|\|f(e_i)\|_Y\\
        &\le C\|x\|_XM,
    \end{align*}
    which implies $f$ is bounded.
    \\\item (Another proof of Riesz's Theorem) Let $X$ be an infinite dimensional normed linear space, $B$ the closed unit ball in $X$, and $B_0$ the unit open ball in $X$.
    Suppose $B$ is compact.
    Then the open cover $\{x+(1/3)B_0\}_{x\in B}$ of $B$ has a finite subcover $\{x_i+(1/3)B_0\}_{1\le i\le n}$.
    Use Riesz's Lemma with $Y=\text{span}[\{x_1,\dots,x_n\}]$ to derive a contradiction.
    \\\item Let $X$ be a normed linear space.
    Show that $X$ is separable iff there is a compact subset $K$ of $X$ for which $\overline{\text{span}}[K]=X$.
\end{enumerate}

% 13.4
\authoredby{inprogress}
\section{The Open Mapping and Closed Graph Theorems}\

\begin{center}
	\textbf{PROBLEMS}
\end{center}
\begin{enumerate}
	\setcounter{enumi}{28}
    \item Let $X$ be a finite dimensional normed linear space and $Y$ a normed linear space.
    Show that every linear operator $T:X\to Y$ is continuous and open.\\
    \\Let $\|\cdot\|_X$ and $\|\cdot\|_Y$ be the norms on $X$ and $Y$ respectively.
    \\Because $X$ is finite dimensional, we can choose an orthonormal basis $\{e_1,\dots,e_n\}$ of $X$.
    \\Define 
    \[
        M:=\max\{\|T(e_1)\|_Y,\dots,\|T(e_n)\|_Y\}\ge0.
    \]
    By Chapter 13 Theorem 4, any two norms on the finite dimensional linear space $X$ are equivalent; 
    therefore in particular there exists $c\ge0$ such that
    \[
        \sum_{i=1}^n|x_i|=\|x\|_1\le c\|x\|_X\text{ for all }x\in X\tag{1}
    \] 
    Now consider any $x\in X$.
    \begin{align*}
        \|T(x)\|_Y&=\|T(\sum_{i=1}^nx_ie_i)\|_Y&&\text{using the orthonormal basis}\\
        &=\|\sum_{i=1}^nx_iT(e_i)\|_Y&&\text{by linearity of }T\\
        &\le\sum_{i=1}^n\|x_iT(e_i)\|_Y&&\text{by subadditivity of the norm }\|\cdot\|_Y\\
        &=\sum_{i=1}^n|x_i|\|T(e_i)\|_Y&&\text{by absolute homogeneity of the norm }\|\cdot\|_Y\\
        &\le\sum_{i=1}^n|x_i|M&&\text{by definition of }M\\
        &=\|x\|_1M&&\text{by definition of the 1-norm}\\
        &\le c\|x\|_XM,&&\text{by equivalence of norms (1)}
    \end{align*}
    and therefore there exists the constant $c\cdot M\ge0$ such that 
    \[
        \|T(x)\|_Y\le(c\cdot M)\|x\|_X\text{ for all }x\in X,
    \]
    which implies that $T$ is bounded, and thus by Chapter 13 Theorem 1, it is continuous.
    \\(it remains to show that $T$ is open)
    \item Let $X$ be a Banach space and $P\in\mathcal{L}(X,X)$ be a projection.
    Show that $P$ is open.
    \item Let $T:X\to Y$ be a continuous linear operator between the Banach spaces $X$ and $Y$.
    Show that $T$ is open if the image under $T$ of the open unit ball in $X$ is dense in a neighborhood of the origin in $Y$.
    \item Let $\{u_n\}$ be a sequence in a Banach space $X$.
    Suppose that $\sum_{k=1}^\infty\|u_k\|<\infty$.
    Show that there is an $x\in X$ for which
    \[
        \lim_{n\to\infty}\sum_{k=1}^\infty u_k=x.
    \]
    \item Let $T$ be a linear operator from a normed linear space $X$ to a finite-dimensional normed linear space $Y$.
    Show that $T$ is continuous iff $\text{ker }T$ is a closed subspace of $X$.
    \item Let $X$ be a Banach space, the operator $T\in\mathcal{L}(X,X)$ be open and $X_0$ be a closed subspace of $X$.
    The restriction $T_0$ of $T$ to $X_0$ is continuous.
    Is $T_0$ necessarily open?
    \item Let $V$ be a linear subspace of a linear space $X$.
    Argue as follows to show that $V$ has a linear complement in $X$.
    \begin{enumerate}[(i)]
        \item If $\text{dim }X<\infty$, let $\{e_i\}_{i=1}^n$ be a basis for $V$.
        Extend this basis for $V$ to a basis $\{e_i\}_{i=1}^{n+k}$ for $X$.
        Then define $W=\text{span}[\{e_{n+1},\dots,e_{n+k}\}]$.
        \item If $\text{dim }X=\infty$, apply Zorn's Lemma to the collection $\mathcal{F}$ of all subspaces $Z$ of $X$ for which $V\cap Z=\{0\}$, ordered by set inclusion.
        \item Verify (15) and (16).
        \item Let $Y$ be a normed linear space.
        Show that $Y$ is a Banach space iff there is a Banach space $X$ and a continuous, open mapping of $X$ onto $Y$.
    \end{enumerate}
\end{enumerate}

% 13.5
\authoredby{untouched}
\section{The Uniform Boundedness Principle}
\begin{center}
	\textbf{PROBLEMS}
\end{center}
\begin{enumerate}
	\setcounter{enumi}{37}
    \item As a consequence of the Baire Category Theorem we showed that a real-valued mapping that is the pointwise limit of a sequence of continuous mapping on a complete metric space must be continuous at all points of a dense subset of its domain.
    Adapt that proof so that it applies to mapping into any metric space.
    Use this to prove that the pointwise limit of a sequence of continuous linear operators on a Banach space has a limit that is continuous at some point and hence, by linearity, is continuous.
    \item Let $\{f_n\}$ be a sequence in $L^\infty[a,b]$.
    Suppose that for each $g\in L^1[a,b]$, $\lim_{n\to\infty}\int_a^bg\cdot f_n$ exists.
    Show that there is a function $f\in L^\infty[a,b]$ such that $\lim_{n\to\infty}\int_a^bg\cdot f_n=\int_a^bg\cdot f$ for all $g\in L^1[a,b]$.
    \item Let $X$ be the linear space of all polynomials defined on $\mathbb{R}$.
    For $p\in X$, define $\|p\|$ to be the sum of the absolute values of the coefficients of $p$.
    Show that this is a norm on $X$.
    For each $n$, define $\psi_n:X\to\mathbb{R}$ by $\psi_n(p)=p^{(n)}(0)$.
    Use the properties of the sequence $\{\psi_n\}$ in $\mathcal{L}(X,\mathbb{R})$ to show that $X$ is not a Banach space.
\end{enumerate}
