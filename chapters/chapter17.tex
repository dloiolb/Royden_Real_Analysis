% Chapter 17
\authoredby{finished}
\chapter{General Measure Spaces: Their Properties and Construction}

% 17.1
\authoredby{finished}
\section{Measures and Measurable Sets}
\begin{flushleft}
	\begin{namedthm*}{Definition}
		By a \textbf{measurable space} we mean a couple $(X,\mathcal{M})$ consisting of a set $X$ and a $\sigma$-algebra $\mathcal{M}$ of subsets of $X$.
		A subset $E$ of $X$ is called \textbf{measurable} (or measurable with respect to $\mathcal{M}$) provided $E$ belongs to $\mathcal{M}$.
	\end{namedthm*}
	\begin{namedthm*}{Definition}
		By a \textbf{measure} $\mu$ on a measurable space $(X,\mathcal{M})$ we mean an extended real-valued nonnegative set function $\mu:\mathcal{M}\to[0,\infty]$ for which $\mu(\emptyset)=0$ and which is \textbf{countably additive} in the sense that for any countable disjoint collection $\{E_k\}_{k=1}^\infty$ of measurable sets,
		\[
			\mu\biggl(\bigcup_{k=1}^\infty E_k\biggr)=\sum_{k=1}^\infty\mu(E_k).	
		\]
	\end{namedthm*}
	\begin{namedthm*}{Definition}
		By a \textbf{measure space} $(X,\mathcal{M},\mu)$ we mean a measurable space $(X,\mathcal{M})$ together with a measure $\mu$ defined on $\mathcal{M}$.
	\end{namedthm*}
	\begin{namedthm*}{Proposition 1}
		Let $(X,\mathcal{M},\mu)$ be a measure space.
		\begin{enumerate}[\indent {}]
			\item (Finite Additivity) For any finite disjoint collection $\{E_k\}_{k=1}^n$ of measurable sets,
			\[
				\mu\biggl(\bigcup_{k=1}^n E_k\biggr)=\sum_{k=1}^n\mu(E_k).	
			\]
			\item (Monotonicity) If $A$ and $B$ are measurable sets and $A\subseteq B$, then
			\[
				\mu(A)\le\mu(B).
			\]
			\item (Excision) If, moreover, $A\subseteq B$ and $\mu(A)<\infty$, then
			\[
				\mu(B\setminus A)=\mu(B)-\mu(A),
			\]
			so that if $\mu(A)=0$, then
			\[
				\mu(B\setminus A)=\mu(B).
			\]
			\item (Countable Monotonicity) For any countable collection $\{E_k\}_{k=1}^n$ of measurable sets that covers a measurable set $E$,
			\[
				\mu(E)\le\sum_{k=1}^\infty\mu(E_k).	
			\]
		\end{enumerate}
	\end{namedthm*}
	\begin{namedthm*}{Definition}
		Let $(X,\mathcal{M},\mu)$ be a measure space. 
		The measure $\mu$ is called \textbf{finite} provided $\mu(X)<\infty$. 
		It is called \textbf{$\sigma$-finite} provided $X$ is the union of a countable collection of measurable sets, each of which has finite measure.
		A measurable set $E$ is said to be of \textbf{finite measure} provided $\mu(E)<\infty$, and is said to be \textbf{$\sigma$-finite} provided $E$ is the union of a countable collection of measurable sets, each of which has finite measure.
	\end{namedthm*}
	\begin{namedthm*}{Definition}
		A measure space $(X,\mathcal{M},\mu)$ is said to be \textbf{complete} provided $\mathcal{M}$ contains all subsets of sets of measure zero, that is, if $E$ belongs to $\mathcal{M}$ and $\mu(E)=0$, then every subset of $E$ also belongs to $\mathcal{M}$.
	\end{namedthm*}
	For example, the Lebesgue measure $m$ on the real line is complete. 
	Moreover, in Chapter 2 Proposition 22, we showed that the Cantor set $C$, a Borel set that has Lebesgue measure zero, contains a Lebesgue measurable set that is not a Borel set.
	Therefore the Lebesgue measure restricted to the Borel $\sigma$-algebra $\mathcal{B}$ is not complete because $C$ belongs to $\mathcal{B}$ and $m(C)=0$ but there exists a subset $A\subseteq C$ such that $A\notin\mathcal{B}$.\\
	\medskip
	The following proposition tells us that each measure space can be completed.
	\begin{namedthm*}{Proposition 3}
		Let $(X,\mathcal{M},\mu)$ be a measure space.
		Define $\mathcal{M}_0$ to be the collection of subsets $E$ of $X$ of the form $E=A\cup B$ where $B\in\mathcal{M}$ and $A\subseteq C$ for some $C\in\mathcal{M}$ for which $\mu(C)=0$.
		For such a set $E$ define $\mu_0(E)=\mu(B)$. 
		Then $\mathcal{M}_0$ is a $\sigma$-algebra that contains $\mathcal{M}$, $\mu_0$ is a measure that extends $\mu$, and $(X,\mathcal{M}_0,\mu_0)$ (the \textbf{completion} of $(X,\mathcal{M},\mu)$) is a complete measure space.
	\end{namedthm*}
\end{flushleft}
\begin{center}
	\textbf{PROBLEMS}
\end{center}
\begin{enumerate}
	\setcounter{enumi}{0}
	\item Let $f$ be a nonnegative Lebesgue measurable function on $\mathbb{R}$. 
	For each Lebesgue measurable subset $E$ of $\mathbb{R}$, define $\mu(E) = \smallint_E f$, the Lebesgue integral of $f$ over $E$.
	Show that $\mu$ is a measure on the $\sigma$-algebra of Lebesgue measurable subsets of $\mathbb{R}$.\\
	\\Because $f$ is nonnegative, by monotonicity of integration, for any Lebesgue measurable set $E$, 
	\[
		0\le f\implies 0=\int_E 0\le \int_E f=\mu(E).
	\]
	Check Chapter 4 Problem 28 to see that for $f$ Lebesgue integrable over $\mathbb{R}$ and $\emptyset$ a Lebesgue measurable subset of $\mathbb{R}$, we have that
	\[
		\mu(\emptyset)=\int_\emptyset f=\int_\mathbb{R} f\cdot\chi_\emptyset=\int_\mathbb{R} f\cdot\chi_\emptyset=\int_\mathbb{R} 0 = 0.
	\]
	Let $\{E_n\}_{n=1}^\infty$ be a disjoint countable collection of Lebesgue measurable sets so that each $\mu(E_n) = \smallint_{E_n} f$ is defined.
	Then $E=\bigcup_{n=1}^\infty E_n$ is Lebesgue measurable, and $\mu(E) = \smallint_E f$ is defined.
	\\Then by Chapter 4 Theorem 20,
	\[
		\mu(\bigcup_{n=1}^\infty E_n)=\mu(E)=\int_E f =\sum_{n=1}^\infty\int_{E_n}f=\sum_{n=1}^\infty\mu(E_n).
	\]
	Therefore $\mu$ is a measure on the $\sigma$-algebra of Lebesgue measurable sets. 
	\item Let $\mathcal{M}$ be a $\sigma$-algebra of subsets of a set $X$ and the set function $\mu : \mathcal{M} \to [0,\infty)$ be finitely additive.
	Prove that $\mu$ is a measure iff whenever $\{A_k\}_{k=1}^\infty$ is an ascending sequence of sets in $\mathcal{M}$, then
	\[
	\mu \biggl ( \bigcup_{k=1}^\infty A_k \biggr ) = \lim_{k \to \infty} \mu(A_k).	
	\]
	\\$(\implies)$ Suppose that $\mu$ is a measure.\\
	Then by Continuity of Measure, the conclusion follows.\\
	\\$(\impliedby)$ Suppose that whenever $\{A_k\}_{k=1}^\infty$ is an ascending sequence of sets in $\mathcal{M}$, then $\mu ( \bigcup_{k=1}^\infty A_k ) = \lim_{k \to \infty} \mu(A_k)$.
	(See Chapter 2 Problem 28.)\\
	Finite additivity of $\mu$ means that for any finite disjoint collection $\{E_k\}_{k=1}^n$ of measurable sets, we have $\mu(\bigcup_{k=1}^n E_k)=\sum_{k=1}^n\mu(E_k)$.
	\\We define $F_n=\bigcup_{k=1}^n E_k$ so that $\{F_n\}_{n=1}^\infty$ is an ascending sequence of sets in $\mathcal{M}$, and thus $\mu ( \bigcup_{n=1}^\infty F_n ) = \lim_{n \to \infty} \mu(F_n)$.
	\\Thus we see
	\[
		\mu(\bigcup_{n=1}^\infty E_n)=\mu(\bigcup_{n=1}^\infty F_n)=\lim_{n \to \infty} \mu(F_n)=\lim_{n \to \infty} \mu(\bigcup_{k=1}^n E_k)=\lim_{n \to \infty}\sum_{k=1}^n\mu(E_k)=\sum_{k=1}^\infty\mu(E_k),
	\]
	that is, $\mu$ satisfies countable additivity, and thus $\mu$ is a measure.
	\item Let $\mathcal{M}$ be a $\sigma$-algebra of subsets of a set $X$. Formulate and establish a correspondent of the preceding problem for descending sequences of sets in $\mathcal{M}$.\\
	\\Let $\mathcal{M}$ be a $\sigma$-algebra of subsets of a set $X$ and the set function $\mu : \mathcal{M} \to [0,\infty)$ be finitely additive.
	Prove that $\mu$ is a measure iff whenever $\{A_k\}_{k=1}^\infty$ is a descending sequence of sets in $\mathcal{M}$ with $m(A_1)<\infty$, then
	\[
	\mu \biggl ( \bigcap_{k=1}^\infty A_k \biggr ) = \lim_{k \to \infty} \mu(A_k).	
	\]
	\\$(\implies)$ Suppose that $\mu$ is a measure.\\
	Then by Continuity of Measure, the conclusion follows.\\
	\\$(\impliedby)$ Suppose that whenever $\{A_k\}_{k=1}^\infty$ is a descending sequence of sets in $\mathcal{M}$ with $\mu(A_1)<\infty$, then $\mu ( \bigcap_{k=1}^\infty A_k ) = \lim_{k \to \infty} \mu(A_k)$.\\
	Finite additivity of $\mu$ means that for any finite disjoint collection $\{E_k\}_{k=1}^n$ of measurable sets, we have $\mu(\bigcup_{k=1}^n E_k)=\sum_{k=1}^n\mu(E_k)$.
	\\We can consider $\{\bigcup_{k={n+1}}^\infty E_k\}_{n=1}^\infty$, a descending sequence of sets in $\mathcal{M}$ with $\mu(\bigcup_{k=2}^\infty E_k)<\infty$, and then because $\{E_k\}_{k=1}^n$ is disjoint,
	\[
		\mu ( \bigcap_{n=1}^\infty [\bigcup_{k={n+1}}^\infty E_k]) = \lim_{n \to \infty} \mu(\bigcup_{k={n+1}}^\infty E_k)=0.
	\]
	\\Thus we see
	\begin{align*}
		\mu(\bigcup_{k=1}^\infty E_k)&=\mu([\bigcup_{k=1}^n E_k]\cup[\bigcup_{k={n+1}}^\infty E_k])\\
		&=\mu(\bigcup_{k=1}^n E_k)+\mu(\bigcup_{k={n+1}}^\infty E_k)&&\text{ by disjoint additivity}\\
		&=\sum_{k=1}^n\mu( E_k)+\mu(\bigcup_{k={n+1}}^\infty E_k)&&\text{ by disjoint additivity}
	\end{align*}
	The left hand side is independent of $n$, so taking the limit, we have
	\begin{align*}
		\mu(\bigcup_{k=1}^\infty E_k)&=\lim_{n\to\infty}\sum_{k=1}^n\mu( E_k)+\lim_{n\to\infty}\mu(\bigcup_{k={n+1}}^\infty E_k)\\
		&=\sum_{k=1}^\infty\mu( E_k)+0\\
		&=\sum_{k=1}^\infty\mu( E_k),
	\end{align*}
	that is, $\mu$ satisfies countable additivity, and thus $\mu$ is a measure.
	\item Let $\{(X_\lambda,\mathcal{M}_\lambda,\mu_\lambda)\}_{\lambda\in\Lambda}$ be a collection of measure spaces parametrized by the set $\Lambda$.
	Assume the collection of sets $\{X_\lambda\}_{\lambda\in\Lambda}$ is disjoint.
	Then we can form a new measure space (called their union) $(X,\mathcal{B},\mu)$ by letting $X=\bigcup_{\lambda\in\Lambda}X_\lambda$, letting $\mathcal{B}$ be the collection of subsets $B$ of $X$ such that $B\cap X_\lambda\in\mathcal{M}_\lambda$ for all $\lambda\in\Lambda$, and defining $\mu(B)=\sum_{\lambda\in\Lambda}\mu_\lambda[B\cap X_\lambda]$ for $B\in\mathcal{B}$.
	\begin{enumerate}[label=(\roman*),align=left]   
		\item Show that $\mathcal{B}$ is a $\sigma$-algebra.\\
		\\We have:
		\begin{enumerate}[label=(\roman*),align=left]   
			\item $X\in\mathcal{B}$ because $X\subseteq X$ such that for any $\lambda'\in\Lambda$,
			\begin{align*}
				X\cap X_{\lambda'}&= \bigcup_{\lambda\in\Lambda}X_\lambda\cap X_{\lambda'}= X_{\lambda'},
			\end{align*}
			where $X_{\lambda'}\in\mathcal{M}_{\lambda'}$ because $(X_{\lambda'},\mathcal{M}_{\lambda'},\mu_{\lambda'})$ is a measure space.
			\item if $B\in\mathcal{B}$, then $B\subseteq X$ such that for any $\lambda'\in\Lambda$, $B\cap X_{\lambda'}\in\mathcal{M}_{\lambda'}$.\\
			Then $B^c\subseteq X$ and because $ B\cap X_{\lambda'}\in\mathcal{M}_{\lambda'}$ and $X_{\lambda'}\in\mathcal{M}_{\lambda'}$,
			\[
				\mathcal{M}_{\lambda'}\ni[ B\cap X_{\lambda'}]^c\cap X_{\lambda'}=[B^c\cup X_{\lambda'}^c]\cap X_{\lambda'}=[B^c\cap X_{\lambda'}]\cup[X_{\lambda'}^c\cap X_{\lambda'}]=B^c\cap X_{\lambda'}.
			\]
			Therefore $B^c\in\mathcal{B}$.
			\item if $B_i\in\mathcal{B}$, then $B_i\in X$ such that for any $\lambda'\in\Lambda$, $B_i\cap X_{\lambda'}\in\mathcal{M}_{\lambda'}$ for all $i$.\\
			Then $\bigcup_{i=1}^\infty B_i\in X$ and $[\bigcup_{i=1}^\infty B_i]\cap X_{\lambda'} =\bigcup_{i=1}^\infty [B_i\cap X_{\lambda'}]\in\mathcal{M}_{\lambda'}$.
			\\Therefore $\bigcup_{i=1}^\infty B_i\in\mathcal{B}$.
		\end{enumerate}
		\item Show that $\mu$ is a measure.\\
		\\For $B\in\mathcal{B}$, we have $B\cap X_{\lambda'}\in\mathcal{M}_{\lambda'}$ for all $\lambda'\in\Lambda$, and so $\mu_{\lambda'}[B\cap X_{\lambda'}]$ is defined.
		\\Then $\mu_{\lambda'}[B\cap X_{\lambda'}]\ge0$ for all $\lambda'\in\Lambda$, which implies $\mu(B)=\sum_{\lambda\in\Lambda}\mu_\lambda[B\cap X_\lambda]\ge0$.
		\\Then because $\emptyset=X^c$ is in $\mathcal{B}$, then $\emptyset=\emptyset\cap X_{\lambda'}\in\mathcal{M}_{\lambda'}$ for all $\lambda'\in\Lambda$, and then $\mu(\emptyset)=\sum_{\lambda\in\Lambda}\mu_\lambda[\emptyset\cap X_\lambda]=\sum_{\lambda\in\Lambda}0=0$.
		\\Finally, consider any countable disjoint collection $\{B_k\}_{k=1}^\infty$ in $\mathcal{B}$.
		Then for any $\lambda'\in\Lambda$, the collection $\{B_k\cap X_{\lambda'}\}_{k=1}^\infty$ is disjoint so that 
		\begin{align*}
			\mu(\bigcup_{k=1}^\infty B_k)&= \sum_{\lambda\in\Lambda}\mu_\lambda[(\bigcup_{k=1}^\infty B_k)\cap X_\lambda]\\
			&=\sum_{\lambda\in\Lambda}\mu_\lambda[\bigcup_{k=1}^\infty (B_k\cap X_\lambda)]\\
			&=\sum_{\lambda\in\Lambda}\sum_{k=1}^\infty \mu_\lambda(B_k\cap X_\lambda)\\
			&=\sum_{k=1}^\infty \sum_{\lambda\in\Lambda} \mu_\lambda(B_k\cap X_\lambda)\\
			&=\sum_{k=1}^\infty \mu(B_k).
		\end{align*}
		Therefore $\mu$ is a measure.
		\item Show that $\mu$ is $\sigma$-finite iff all but a countable number of the measures $\mu_\lambda$ have $\mu(X_\lambda)=0$ and the remainder are $\sigma$-finite.\\
		\\$(\implies)$ Suppose $\mu$ is $\sigma$-finite.\\
		Then $X$ can be written as the countable union of disjoint measurable sets, each of which has finite measure under $\mu$.
		\\That is, we have $X=\bigcup_{k=1}^\infty A_k$, with $A_k\in\mathcal{B}$ s.t. $\mu(A_k)<\infty$ for each $k$.
		\\So $A_k\in\mathcal{B}\implies A_k\cap X_\lambda\in\mathcal{M}_\lambda$ for each $\lambda$, and $\sum_{\lambda\in\Lambda} \mu_\lambda(A_k\cap X_\lambda)=\mu(A_k)<\infty\implies\mu_\lambda(A_k\cap X_\lambda)<\infty$ for each $\lambda$.
		\\Then 
		\[
			\mu(X)=\mu(\bigcup_{k=1}^\infty A_k)=\sum_{k=1}^\infty\mu(A_k)=\sum_{k=1}^\infty\sum_{\lambda\in\Lambda} \mu_\lambda(A_k\cap X_\lambda)
		\]
		\\Then all but a countable number of the measures $\mu_\lambda$ can be nonzero, and the remainder must be $\sigma$-finite.\\
		\\$(\impliedby)$ Suppose all but a countable number of the measures $\mu_\lambda$ have $\mu(X_\lambda)=0$ and the remainder are $\sigma$-finite.\\
		Let $\Lambda^*\subseteq\Lambda$ be the set of measures $\mu_\lambda$ such that $\mu_\lambda(X_\lambda)=0$.
		\\Let $\Lambda^{*c}=\{\lambda_k\}_{k=1}^\infty\subseteq\Lambda$ be a countable collection such that each $\mu_{\lambda_k}$ is $\sigma$-finite.
		\\By definition of $\sigma$-finite, for each $k$, we have $X_{\lambda_k}=\bigcup_{i=1}^\infty [A_{\lambda_k}]_i$, with $[A_{\lambda_k}]_i\in\mathcal{M}_{\lambda_k}$ s.t. $\mu_{\lambda_k}([A_{\lambda_k}]_i)<\infty$ for each $i$.
		\\Then because the collection $\{X_\lambda\}_{\lambda\in\Lambda}$ is disjoint, then $[A_{\lambda_k}]_i\cap X_\lambda=([A_{\lambda_k}]_i\cap X_{\lambda_k})\cap X_\lambda=\emptyset$ for $\lambda\neq\lambda_k$.
		\\Also $[A_{\lambda_k}]_i=[A_{\lambda_k}]_i\cap X_{\lambda_k}\in\mathcal{M}_{\lambda_k}$ so that $[A_{\lambda_k}]_i\in\mathcal{B}$ and $\mu([A_{\lambda_k}]_i)$ is defined.
		\\Then we have for each $i$,
		\begin{align*}
			\mu([A_{\lambda_k}]_i)&=\sum_{\lambda\in\Lambda} \mu_\lambda([A_{\lambda_k}]_i\cap X_\lambda)\\
			&=\sum_{\lambda\neq\lambda_k} \mu_\lambda([A_{\lambda_k}]_i\cap X_\lambda)+\mu_{\lambda_k}([A_{\lambda_k}]_i\cap X_\lambda)\\
			&=\sum_{\lambda\neq\lambda_k} 0+\mu_{\lambda_k}([A_{\lambda_k}]_i)\\
			&=\mu_{\lambda_k}([A_{\lambda_k}]_i).
		\end{align*}
		Therefore $\mu_{\lambda_k}([A_{\lambda_k}]_i)=\mu([A_{\lambda_k}]_i)<\infty$.
		\\Then we can write
		\begin{align*}
			\mu(X)&=\sum_{\lambda\in\Lambda} \mu_\lambda(X_\lambda)\\		
			&=\sum_{\lambda\in\Lambda*} \mu_\lambda(X_\lambda)+\sum_{\lambda\notin\Lambda^*} \mu_\lambda(X_\lambda)\\	
			&=\sum_{\lambda\in\Lambda*} 0+\sum_{k=1}^\infty \mu_{\lambda_{k}}(X_{\lambda_{k}})\\
			&=\sum_{k=1}^\infty \mu_{\lambda_{k}}(\bigcup_{i=1}^\infty [A_{\lambda_k}]_i)\\
			&=\sum_{k=1}^\infty \sum_{i=1}^\infty \mu_{\lambda_{k}}( [A_{\lambda_k}]_i)\\
			&=\sum_{k=1}^\infty\sum_{i=1}^\infty \mu([A_{\lambda_k}]_i),
		\end{align*}
		and thus $X$ can be written as a countable disjoint union of measurable sets $[A_{\lambda_k}]_i$, each of which has finite measure under $\mu$.
		\\Therefore $\mu$ is $\sigma$-finite.
	\end{enumerate}
	\item Let $(X,\mathcal{M},\mu)$ be a measure space. The symmetric difference, $E_1\Delta E_2$, of two subsets $E_1$ and $E_2$ of $X$ is defined by
	\[
		E_1\Delta E_2=[E_1\setminus E_2]\cup[E_2\setminus E_1].
	\]
	\begin{enumerate}[label=(\roman*),align=left]   
		\item Show that if $E_1$ and $E_2$ are measurable and $\mu(E_1\Delta E_2)=0$, then $\mu(E_1)=\mu(E_2)$.\\
		\\We can see that
		\begin{align*}
			\mu(E_1\cup E_2)=\mu([E_1\Delta E_2]\cup[E_1\cap E_2])=\mu(E_1\Delta E_2)+\mu(E_1\cap E_2)=\mu(E_1\cap E_2).
		\end{align*}
		Then we also know that by monotonicity we have
		\begin{align*}
			E_1\cap E_2\subseteq E_1,E_2\subseteq E_1\cup E_2\implies\mu(E_1\cap E_2)\le\mu(E_1),\mu(E_2)\le\mu(E_1\cup E_2),
		\end{align*}
		and therefore $\mu(E_1)=\mu(E_2)$.
		\item Show that if $\mu$ is complete and $E_1\in\mathcal{M}$, then $E_2\in\mathcal{M}$ if $\mu(E_1\Delta E_2)=0$.\\
		\\Because $\mu(E_1\Delta E_2)=0$, then because $\mu$ is complete, the subsets $[E_1\setminus E_2]\subseteq E_1\Delta E_2$ and $[E_2\setminus E_1]\subseteq E_1\Delta E_2$ are measurable.
		Therefore the set $[E_2\setminus E_1]\cup[E_1]\cap[E_1\setminus E_2]^c$ is also measurable, and
		\begin{align*}
			[E_2\setminus E_1]\cup[E_1]\cap[E_1\setminus E_2]^c&=[E_2\cup E_1]\cap[E_1^c\cup E_1]\cap[E_1^c\cup E_2]\\
			&=[E_2\cup E_1]\cap[E_1^c\cup E_2]\\
			&=([E_2\cup E_1]\cap E_1^c)\cup ([E_2\cup E_1]\cap E_2)\\
			&=([E_2\cap E_1^c]\cup [E_1\cap E_1^c])\cup E_2\\
			&=(E_2\cap E_1^c)\cup E_2\\
			&=E_2,
		\end{align*}
		therefore $E_2=[E_2\setminus E_1]\cup[E_1]\cap[E_1\setminus E_2]^c$ is measurable.
	\end{enumerate}
	\item Let $(X,\mathcal{M},\mu)$ be a measure space and $X_0$ belong to $\mathcal{M}$.
	Define $\mathcal{M}_0$ to be the collection of sets in $\mathcal{M}$ that are subsets of $X_0$ and $\mu_0$ the restriction of $\mu$ to $\mathcal{M}_0$.
	Show that $(X_0,\mathcal{M}_0,\mu_0)$ is a measure space.\\
	\\We want to show that $(X_0,\mathcal{M}_0)$ is a measurable space (i.e., that $\mathcal{M}_0$ is a $\sigma$-algebra of subsets of $X_0$) and that $\mu_0$ is a measure on $\mathcal{M}_0$.
	\\To see that $\mathcal{M}_0$ is a $\sigma$-algebra:
	\begin{enumerate}[label=(\roman*),align=left]   
		\item $X_0\in\mathcal{M}_0$ because $X_0\in\mathcal{M}$ and $X_0\subseteq X_0$.
		\item if $A\in\mathcal{M}_0$, then $A\in\mathcal{M}$ and $A\subseteq X_0$.\\
		Then $X_0\cap A^c\in\mathcal{M}$ and $X_0\cap A^c\subseteq X_0$ imply that $X_0\cap A^c\in\mathcal{M}_0$.
		\item if $A_i\in\mathcal{M}_0$, then $A_i\in\mathcal{M}$ and $A_i\subseteq X_0$ for all $i$.\\
		Then $\bigcup_{i=1}^\infty A_i\in\mathcal{M}$ and $\bigcup_{i=1}^\infty A_i\subseteq X_0$ imply that $\bigcup_{i=1}^\infty A_i\in\mathcal{M}_0$.
	\end{enumerate}
	Therefore $(X_0,\mathcal{M}_0)$ is a measurable space.\\
	Clearly $\mu_0$ is a measure on $\mathcal{M}_0$, because it inherits the properties of a measure from $\mu$.\\
	Thus $(X_0,\mathcal{M}_0,\mu_0)$ is a measure space.
	\item Let $(X,\mathcal{M})$ be a measurable space. Verify the following:
	\begin{enumerate}[label=(\roman*),align=left]  
		\item If $\mu$ and $\nu$ are measures defined on $\mathcal{M}$, then set set function $\lambda$ defined on $\mathcal{M}$ by $\lambda(E)=\mu(E)+\nu(E)$ also is a measure. We denote $\lambda$ by $\mu+\nu$.\\
		\\Because $\mu(E)\ge 0$ and $\nu(E)\ge 0$ for any $E\in\mathcal{M}$, then $\lambda(E)=\mu(E)+\nu(E)\ge 0$.
		\\Also, $\mu(\emptyset)= 0$ and $\nu(\emptyset)= 0$ imply that $\lambda(\emptyset)=\mu(\emptyset)+\nu(\emptyset)= 0$.
		\\Finally, for any countable disjoint collection $\{E_k\}_{k=1}^\infty$ of measurable sets,
		\begin{align*}
			\lambda\left(\bigcup_{k=1}^\infty E_k\right)&=\mu\left(\bigcup_{k=1}^\infty E_k\right)+\nu\left(\bigcup_{k=1}^\infty E_k\right)\\
			&=\sum_{k=1}^\infty\mu(E_k)+\sum_{k=1}^\infty\nu(E_k)\\
			&=\sum_{k=1}^\infty[\mu(E_k)+\nu(E_k)]\\
			&=\sum_{k=1}^\infty\lambda(E_k).
		\end{align*}
		Therefore $\lambda$ is a measure.
		\item If $\mu$ and $\nu$ are measures on $\mathcal{M}$ and $\mu\ge\nu$, then there is a measure $\lambda$ on $\mathcal{M}$ for which $\mu=\nu+\lambda$.\\
		\\In the case $\mu(E)<\infty$, then we also have $\nu(E)\le\mu(E)<\infty$, and we can let $\lambda = \mu-\nu$.
		\\We clearly see that $\mu\ge\nu\implies\mu-\nu\ge0$ so that $\lambda(E)=\mu(E)-\nu(E)\ge 0$ for any $E\in\mathcal{M}$ (of finite measure under $\mu$).
		\\Also, $\mu(\emptyset)= 0$ and $\nu(\emptyset)= 0$ imply that $\lambda(\emptyset)=\mu(\emptyset)-\nu(\emptyset)= 0$.
		\\Finally, for any countable disjoint collection $\{E_k\}_{k=1}^\infty$ of measurable sets each of finite measure,
		\begin{align*}
			\lambda\left(\bigcup_{k=1}^\infty E_k\right)&=\mu\left(\bigcup_{k=1}^\infty E_k\right)-\nu\left(\bigcup_{k=1}^\infty E_k\right)\\
			&=\sum_{k=1}^\infty\mu(E_k)-\sum_{k=1}^\infty\nu(E_k)\\
			&=\sum_{k=1}^\infty[\mu(E_k)-\nu(E_k)]\\
			&=\sum_{k=1}^\infty\lambda(E_k).
		\end{align*}
		\\In the case $\mu(E)=\infty$, we can let $\lambda(E) = \infty$ so that $\nu(E)+\lambda(E)=\mu(E)$.
		\\Then $\lambda(E)=\infty\ge0$.
		\\For any countable disjoint collection $\{E_k\}_{k=1}^\infty$ of measurable sets, supposing there exists an index $j$ such that $\mu(E_j)=\infty$, then we defined $\lambda(E_j) = \infty$ so that by monotonicity, we have
		\begin{align*}
			\infty=\lambda(E_j)\le\lambda\left(\bigcup_{k=1}^\infty E_k\right),
		\end{align*}
		so $\lambda\left(\bigcup_{k=1}^\infty E_k\right)=\infty=\sum_{k=1}^\infty\lambda(E_k)$.
		\\Then we also have $\sum_{k=1}^\infty\mu(E_k)=\infty$ and 
		\begin{align*}
			\mu\left(\bigcup_{k=1}^\infty E_k\right)&=\mu\left(\bigcup_{k=1}^\infty E_k\right)+\lambda\left(\bigcup_{k=1}^\infty E_k\right)\\
			&=\mu\left(\bigcup_{k=1}^\infty E_k\right)+\infty\\
			&=\infty.
		\end{align*}
		In conclusion, we have defined
		\[
		\lambda(E)=
		\begin{cases}
			\mu(E)-\nu(E)&\text{if }\mu(E)<\infty\\\
			\infty&\text{if }\mu(E)=\infty,
		\end{cases}
		\]
		and we have proved that $\lambda$ is a measure.
		\item If $\nu$ is $\sigma$-finite, the measure $\lambda$ in (ii) is unique.\\
		\\Because $\nu$ is $\sigma$-finite, then $X$ is the union of a countable collection of measurable sets (may be taken to be disjoint), each of which has finite measure under $\nu$.
		That is, $X=\bigcup_{k=1}^\infty X_k$, where $\nu(X_k)<\infty$.
		Then for any $E\in\mathcal{M}$, we have 
		\[
			E=E\cap X = E\cap \bigcup_{k=1}^\infty X_k = \bigcup_{k=1}^\infty[E\cap X_k],
		\]
		where by monotonicity of measure we have $\nu(E\cap X_k)\le\nu(X_k)<\infty$, and thus any measurable set $E$ is also $\sigma$-finite when $\nu$ is $\sigma$-finite.
		\\Now, suppose there exist measures $\lambda_1$ and $\lambda_2$ such that $\mu=\nu+\lambda_1$ and $\mu=\nu+\lambda_2$.
		Then $\nu+\lambda_1=\nu+\lambda_2$ and thus $\nu-\nu=\lambda_2-\lambda_1$.
		\\For any $E\in\mathcal{M}$ such that $\nu(E)<\infty$, then clearly $\lambda_1(E)=\lambda_2(E)$.
		\\For any $E\in\mathcal{M}$ such that $\nu(E)=\infty$, $\nu(E)-\nu(E)=\infty-\infty$ is not defined. 
		\\However, because $\nu$ is $\sigma$-finite, there exists a countable disjoint collection $\{E_k\}_{k=1}^\infty$ such that $E=\bigcup_{k=1}^\infty E_k$ and $\nu(E_k)<\infty$ for each $k$. 
		Then we see that $\nu(E_k)-\nu(E_k)$ is defined for all $k$, and
		\[
			\nu(E)=\nu(\bigcup_{k=1}^\infty E_k)=\sum_{k=1}^\infty \nu(E_k)=\lim_{n\to\infty}\sum_{k=1}^n \nu(E_k).
		\] 
		Then we can write 
		\begin{align*}
			\lim_{n\to\infty}\sum_{k=1}^n \nu(E_k)-\lim_{n\to\infty}\sum_{k=1}^n \nu(E_k)&=\lambda_2(E)-\lambda_1(E)\\
			\lim_{n\to\infty}\sum_{k=1}^n [\nu(E_k)-\nu(E_k)]&=\lambda_2(E)-\lambda_1(E)\\
			\lim_{n\to\infty}\sum_{k=1}^n 0&=\lambda_2(E)-\lambda_1(E)\\
			0&=\lambda_2(E)-\lambda_1(E).
		\end{align*}
		Therefore $\lambda_1(E)=\lambda_2(E)$, and the measure $\lambda$ is unique.
		\item Show that in general the measure $\lambda$ need not be unique but that there is always a smallest such $\lambda$.\\
		\\Suppose there exists a set $E\in\mathcal{M}$ such that $\mu(E)=\infty$ and $\nu(E)=\infty$. 
		Then regardless of the number $\lambda(E)\in[0,\infty]$ we define $\lambda$ to be, we always have $\infty=\mu(E)=\nu(E)+\lambda(E)$.
		Then $\lambda(E)=0$ is the smallest value that we can set $\lambda$ to be, and we can define the smallest $\lambda$ in the following way:
		\[
		\lambda(E)=
		\begin{cases}
			\mu(E)-\nu(E)&\text{if }\mu(E)<\infty\ (\text{ forces }\nu(E)<\infty)\\
			\infty&\text{if }\mu(E)=\infty,\nu(E)<\infty\\
			0&\text{if }\mu(E)=\infty,\nu(E)=\infty
		\end{cases}
		\]
	\end{enumerate}
	\item Let $(X,\mathcal{M},\mu)$ be a measure space.
	The measure $\mu$ is said to be \textbf{semifinite} provided each measurable set of infinite measure contains measurable sets of arbitrarily large finite measure.
	\begin{enumerate}[label=(\roman*),align=left]  
		\item Show that each $\sigma$-finite measure is semifinite.\\
		\\If we suppose $\mu$ is $\sigma$-finite, then we can write any $E\in\mathcal{M}$ as the countable disjoint union of measurable sets of finite measure under $\mu$:
		$E=\bigcup_{k=1}^\infty E_k$ with $\mu(E_k)<\infty$.
		\\Consider any measurable set $E$ such that $\mu(E)=\infty$.
		Then 
		\[
			\mu(E)=\mu(\bigcup_{k=1}^\infty E_k)=\sum_{k=1}^\infty \mu(E_k)=\infty.
		\]
		Then the sequence of partial sums $\sum_{k=1}^n \mu(E_k)$ converges to infinity.
		That is, for any real number $x$, there exists an index $j$ such that $\sum_{k=1}^j \mu(E_k)>x$.
		Because each $E_k$ is disjoint and measurable, we have that $E_x:=\bigcup_{k=1}^j E_k\in\mathcal{M}$, and we can write
		\[
			x<\sum_{k=1}^j \mu(E_k)=\mu(\bigcup_{k=1}^j E_k)=\mu(E_x)<\infty.
		\]
		That is, for any real number we choose, there exists a measurable set $E_x\subseteq E$ of finite measure that is larger than $x$.
		\\Therefore $\mu$ is semifinite.
		\item For $E\in\mathcal{M}$, define $\mu_1(E)=\sup\{\mu(F)\ |\ F\subseteq E,\mu(F)<\infty\}$. 
		Show that $\mu_1$ is a semifinite measure: it is called the semifinite part of $\mu$.\\
		\\Consider any measurable set $E$ such that $\mu_1(E)=\infty$.
		Then for any subset $F$ of $E$ such that $\mu(F)<\infty$, we have that $\mu_1(E)\ge\mu(F)=\mu_1(F)$ by definition of supremum.
		(We have $\mu_1(F)=\mu(F)$ because $F$ is the largest subset of itself).
		However, because $\mu_1$ is the least upper bound, for any real number $x$, there exists a subset $F_x$ of $E$ such that $x<\mu_1(F_x)\le\mu_1(E)$, else we reach a contradiction to the supremum.
		Therefore for any real number $x$ we choose, there exists a measurable set $F_x\subseteq E$ of finite measure that is larger than $x$.  
		\item Find a measure $\mu_2$ on $\mathcal{M}$ that only takes the values $0$ and $\infty$ and $\mu=\mu_1+\mu_2$.\\
		\\We can define, for any $E\in\mathcal{M}$, 
		\[
			\mu_2(E)=
			\begin{cases}
				0&\text{if }\mu_1(E)<\infty\\
				\infty&\text{if }\mu_1(E)=\infty
			\end{cases}
		\]
		So that we have
		\[
			\mu(E)=
			\begin{cases}
				\mu_1(E)+\mu_2(E)=\mu(E)+0&\text{if }\mu_1(E)<\infty\\
				\mu_1(E)+\mu_2(E)=\mu(E)+\infty&\text{if }\mu_1(E)=\infty
			\end{cases}
		\]
	\end{enumerate}
	\item Prove Proposition 3; that is, show that $\mathcal{M}_0$ is a $\sigma$-algebra, $\mu_0$ is properly defined, and $(X,\mathcal{M}_0,\mu_0)$ is complete. In what sense is $\mathcal{M}_0$ minimal?\\
	\\We can see
	\begin{enumerate}[label=(\roman*),align=left]
		\item $X\in\mathcal{M}_0$ because $X\subseteq X$, and $X=\emptyset\cup X$ with $X\in\mathcal{M}$ and $\emptyset\subseteq\emptyset$ for $\emptyset\in\mathcal{M}$ where $\mu(\emptyset)=0$.
		\item If $E\in\mathcal{M}_0$, then $E\subseteq X$, and $E=A\cup B$ with $B\in\mathcal{M}$ and $A\subseteq C$ for $C\in\mathcal{M}$ where $\mu(C)=0$.\\
		Then $A\subseteq C\implies A^c\supseteq C^c$, and $A^c=[A^c\cap C]\cup[A^c\cap C^c]=[A^c\cap C]\cup C^c$.
		\\Now, $X\cap E^c\subseteq X$.
		\\We can write
		\begin{align*}
			E^c&=A^c\cap B^c\\
			&=([A^c\cap C]\cup C^c)\cap B^c\\
			&=([A^c\cap C]\cap B^c)\cup (C^c\cap B^c),
		\end{align*}
		Where $C^c\cap B^c\in\mathcal{M}$ and $[A^c\cap C]\cap B^c\subseteq C$ for $C\in\mathcal{M}$ where $\mu(C)=0$.
		\\Therefore $E^c\in\mathcal{M}_0$.
		\item If $E_k\in\mathcal{M}_0$, then $E_k\subseteq X$, and $E_k=A_k\cup B_k$ with $B_k\in\mathcal{M}$ and $A_k\subseteq C_k$ for $C_k\in\mathcal{M}$ where $\mu(C_k)=0$ for all $k$.\\
		Then $\bigcup_{k=1}^\infty E_k\subseteq X$, and 
		\[
			\bigcup_{k=1}^\infty E_k=\bigcup_{k=1}^\infty [A_k\cup B_k]=[\bigcup_{k=1}^\infty A_k]\cup[\bigcup_{k=1}^\infty B_k],
		\]
		Where $\bigcup_{k=1}^\infty B_k\in\mathcal{M}$ and $A_k\subseteq C_k\implies\bigcup_{k=1}^\infty A_k\subseteq \bigcup_{k=1}^\infty C_k$ for $\bigcup_{k=1}^\infty C_k\in\mathcal{M}$ with $\mu(\bigcup_{k=1}^\infty C_k)\le\sum_{k=1}^\infty\mu(C_k)=\sum_{k=1}^\infty0=0$.
		\\Thus $\bigcup_{k=1}^\infty E_k\in\mathcal{M}_0$.
	\end{enumerate}
	Thus $\mathcal{M}_0$ is a $\sigma$-algebra of subsets of $X$.\\
	\\To see that $\mu_0$ is a measure on the measurable space $(X,\mathcal{M}_0)$:
	\\For any $E\in\mathcal{M}_0$, we have $E=A\cup B,B\in\mathcal{M}$, so that $\mu_0(E)=\mu(B)\ge0$.
	\\Then for $\emptyset\in\mathcal{M}_0$, we have $\emptyset=\emptyset\cup \emptyset,\emptyset\in\mathcal{M}$, so that $\mu_0(\emptyset)=\mu(\emptyset)=0$.
	\\Finally, consider a disjoint collection $\{E_k\}_{k=1}^\infty$ of sets in $\mathcal{M}_0$.
	\\See (iii) to see that $\bigcup_{k=1}^\infty E_k=[\bigcup_{k=1}^\infty A_k]\cup[\bigcup_{k=1}^\infty B_k]$, where $\{E_k\}_{k=1}^\infty$ disjoint implies $\{B_k\}_{k=1}^\infty$ disjoint and we have $\bigcup_{k=1}^\infty B_k\in\mathcal{M}$.
	\\Then 
	\[
		\mu_0(\bigcup_{k=1}^\infty E_k)=\mu(\bigcup_{k=1}^\infty B_k)=\sum_{k=1}^\infty\mu(B_k)=\sum_{k=1}^\infty\mu_0(E_k).
	\]
	Therefore $(X,\mathcal{M}_0,\mu_0)$ is a measure space.\\
	\\To see that $(X,\mathcal{M}_0,\mu_0)$ is complete, consider any set $E\in\mathcal{M}_0$ such that $\mu_0(E)=0$.
	\[
		E\in\mathcal{M}_0\implies E\subseteq X,E=A\cup B, B\in\mathcal{M}\text{ and }A\subseteq C\text{ with }C\in\mathcal{M}, \mu(C)=0.
	\]
	Then $A\subseteq C\implies A\cup B\subseteq C\cup B$, and $C,B\in\mathcal{M}\implies C\cup B\in\mathcal{M}$.
	Thus $\mu(C\cup B)\le \mu(C)+\mu(B)=0$ is well-defined.
	\\Consider any $D\subseteq E$.
	\[
		D\subseteq E\subseteq X,D=D\cup \emptyset, \emptyset\in\mathcal{M}\text{ and }D\subseteq A\cup B\subseteq C\cup B\text{ with }C\cup B\in\mathcal{M}, \mu(C\cup B)=0.
	\]
	Therefore $D\in \mathcal{M}_0$ and $(X,\mathcal{M}_0,\mu_0)$ is complete.
	\item If $(X,\mathcal{M},\mu)$ is a measure space, we say that a subset $E$ of $X$ is \textbf{locally measurable} provided for each $B\in\mathcal{M}$ with $\mu(B)<\infty$, the intersection $E\cap B$ belongs to $\mathcal{M}$.
	The measure $\mu$ is called \textbf{saturated} provided every locally measurable set is measurable.
	\begin{enumerate}[label=(\roman*),align=left]  
		\item Show that each $\sigma$-finite measure is saturated.\\
		\\Suppose $\mu$ is $\sigma$-finite, then $X$ can be taken to be the union of a countable collection of measurable sets, each of which has finite measure under $\mu$.
		\\That is, $X=\bigcup_{k=1}^\infty X_k$, where $\mu(X_k)<\infty$.
		\\Then for any $E\in X$, we have 
		\[
			E=E\cap X = E\cap \bigcup_{k=1}^\infty X_k = \bigcup_{k=1}^\infty[E\cap X_k],
		\]
		In the case that $E$ is locally measurable, then each intersection $E\cap X_k$ is measurable.
		Then the countable intersection of measurable sets $\bigcup_{k=1}^\infty[E\cap X_k]=E$ is measurable.
		\\Thus when $\mu$ is $\sigma$-finite, every locally measurable set is measurable, and thus $\mu$ is saturated.
		\item Show that the collection $\mathcal{C}$ of locally measurable sets is a $\sigma$-algebra.\\
		\\We have
		\begin{enumerate}[label=(\roman*),align=left]   
			\item $X\in\mathcal{C}$ because for all $B\in\mathcal{M}$ with $\mu(B)<\infty$, then $X\cap B=B\in\mathcal{M}$.
			\item if $E\in\mathcal{C}$, then for all $B\in\mathcal{M}$ with $\mu(B)<\infty$, then $E\cap B\in\mathcal{M}$.\\
			Then we have the two measurable sets $E\cap B$ and $B$ so that $[E\cap B]^c\cap B$ is also measurable, and
			\[
				\mathcal{M}\ni[E\cap B]^c\cap B=[E^c\cup B^c]\cap B=[E^c\cap B]\cup[B^c\cap B]=E^c\cap B,
			\]
			and thus $E^c\in\mathcal{C}$. 
			\item if $E_i\in\mathcal{C}$, then for all $B\in\mathcal{M}$ with $\mu(B)<\infty$, then $E_i\cap B\in\mathcal{M}$ for all $i$.\\
			Then $\left[\bigcup_{i=1}^\infty E_i\right]\cap B=\bigcup_{i=1}^\infty [E_i\cap B]\in\mathcal{M}$ and thus $\bigcup_{i=1}^\infty E_i\in\mathcal{C}$.
		\end{enumerate}
		\item Let $(X,\mathcal{M},\mu)$ be a measure space and $\mathcal{C}$ the $\sigma$-algebra of locally measurable sets.
		For $E\in\mathcal{C}$, define $\overline\mu(E)=\mu(E)$ if $E\in\mathcal{M}$ and $\overline\mu(E)=\infty$ if $E\notin\mathcal{M}$.
		Show that $(X,\mathcal{C},\overline\mu)$ is a saturated measure space.\\
		\\In (ii) we showed that $\mathcal{C}$ is a $\sigma$-algebra of subsets of $X$.\\
		Therefore $(X,\mathcal{C})$ is a measurable space.\\
		\\We have defined
		\[
			\overline\mu(E)=
			\begin{cases}
				\mu(E)&\text{ if }E\in\mathcal{M}\\
				\infty&\text{ if }E\notin\mathcal{M}
			\end{cases}	
		\]
		\\We have $\overline\mu(E)\in\{\mu(E),\infty\}\ge0$ for all $E\in\mathcal{C}$.
		We have $\overline\mu(\emptyset)=\mu(\emptyset)=0$ because $\emptyset\in\mathcal{M}$ and $\emptyset\in\mathcal{C}$.
		Finally, consider a countable disjoint collection of sets $\{E_k\}_{k=1}^\infty$ in $\mathcal{C}$.
		\begin{enumerate}[label=(\roman*),align=left]  
			\item If for all $k$ we have $E_k\in\mathcal{M}$, then $\mu(\bigcup_{k=1}^\infty E_k)$ is measurable, $\overline\mu(E_k)=\mu(E_k)$, and
			\[
				\overline\mu(\bigcup_{k=1}^\infty E_k)=\mu(\bigcup_{k=1}^\infty E_k)=\sum_{k=1}^\infty \mu(E_k)=\sum_{k=1}^\infty \overline\mu(E_k).
			\]
			\item If there exists an index $j$ such that $E_j\notin\mathcal{M}$, then $\overline\mu(E_j)=\infty$ and $\sum_{k=1}^\infty \overline\mu(E_k)=\infty$.
		\end{enumerate}
		Consider the case that $\bigcup_{k=1}^\infty E_k\in\mathcal{C}$ is measurable and $\mu(\bigcup_{k=1}^\infty E_k)<\infty$. 
		\\Then because for any $j$, we have $E_j\in\mathcal{C}$, then $E_j=E_j\cap\bigcup_{k=1}^\infty E_k\in\mathcal{M}$.\\
		Then (i) must hold.\\
		\\Consider the case that $\bigcup_{k=1}^\infty E_k\in\mathcal{C}$ is measurable and $\mu(\bigcup_{k=1}^\infty E_k)=\infty$. \\
		If (i) holds, then $\overline\mu(\bigcup_{k=1}^\infty E_k)=\sum_{k=1}^\infty\overline\mu(E_k)$.\\
		If (ii) holds, then $\overline\mu(\bigcup_{k=1}^\infty E_k)=\mu(\bigcup_{k=1}^\infty E_k)=\infty=\sum_{k=1}^\infty \overline\mu(E_k)$.\\\bigskip
		Consider the case that $\bigcup_{k=1}^\infty E_k\in\mathcal{C}$ is not measurable. Then $\mu(\bigcup_{k=1}^\infty E_k)=\infty$. \\
		Then (ii) must hold else we reach a contradiction to $\bigcup_{k=1}^\infty E_k\notin\mathcal{M}$.\\\bigskip
		Therefore $(X,\mathcal{C},\overline\mu)$ is a measure space.\\
		\\We can use the definition of $\overline\mu$ to see that 
		\[
			B\in\mathcal{C}\text{ with }\overline\mu(B)<\infty\iff B\in\mathcal{M}\text{ with }\mu(B)<\infty.
		\]
		Consider a set $E\subseteq X$ such that $E\cap B\in\mathcal{C}$ for any such $B$.
		\\Then by monotonicity, $\overline\mu(E\cap B)\le \overline\mu(B)<\infty$.
		\\Because $\overline\mu(E\cap B)<\infty$, see the definition of $\overline\mu$ to see that $E\cap B\in\mathcal{M}$.
		\\Then we see that $E\in\mathcal{C}$ because for each $B\in\mathcal{M}$ with $\mu(B)<\infty$, we have $E\cap B\in\mathcal{M}$.\\\bigskip
		Therefore $(X,\mathcal{C},\overline\mu)$ is a saturated measure space.
		\item If $\mu$ is semifinite and $E\in\mathcal{C}$, the set $\underline\mu(E)=\sup\{\mu(B)\ |\ B\in\mathcal{M},B\subseteq E\}$.
		Show that $(X,\mathcal{C},\underline\mu)$ is a saturated measure space and that $\underline\mu$ is an extension of $\mu$.
		Give an example to show that $\overline\mu$ and $\underline\mu$ may be different.\\
		\\We first want to show that $\underline\mu$ is a measure on the measurable space $(X,\mathcal{C})$:
		\\For any $E\in\mathcal{C}$, we have $\underline\mu(E)\ge\mu(B)\ge0$ for $B\in\mathcal{M},B\subseteq E$.
		\\For $\emptyset\in\mathcal{C}$, we have $\underline\mu(\emptyset)=\mu(\emptyset)=0$ because $\{\emptyset\}=\{B\in\mathcal{M}\ |\ B\subseteq\emptyset\}$.
		\\Finally, for any disjoint collection $\{E_k\}_{k=1}^\infty$ in $\mathcal{C}$,\\
		\\Therefore $(X,\mathcal{C},\underline\mu)$ is a measure space.\\
		\\Consider any $E\subseteq X$ such that $E\cap B\in\mathcal{C}$ whenever $B\in\mathcal{C}$ with $\underline\mu(B)<\infty$.
		\\Then $E\cap B\in\mathcal{C}$ implies that $[E\cap B]\cap B'\in\mathcal{M}$ whenever $B'\in\mathcal{M}$ with $\mu(B')<\infty$.
	\end{enumerate}
	\item Let $\mu$ and $\eta$ be measures on the measurable space $(X,\mathcal{M})$.
	For $E\in\mathcal{M}$, define $\nu(E)=\max\{\mu(E),\eta(E)\}$. Is $\nu$ a measure on $(X,\mathcal{M})$?\\
	\\We have $0\le\mu(E),\eta(E)\le\max\{\mu(E),\eta(E)\}$ for any $E\in\mathcal{M}$.
	\\We have $\max\{\mu(E),\eta(E)\}\in[0,\infty]$ for any $E\in\mathcal{M}$.
	\\Counterexample:
	Let $E_1,E_2$ be nonempty disjoint measurable (singleton) sets such that
	\[
		\mu(E)=
		\begin{cases}
			1&E\supseteq E_1\\
			0&E\not\supseteq E_1\\
		\end{cases}
		\ \text{ and }\ \eta(E)=
		\begin{cases}
			1&E\supseteq E_2\\
			0&E\not\supseteq E_2\\
		\end{cases}
	\]
	Then $\mu(E)\in\{0,1\}\ge0$, $\mu(\emptyset)=0$ because $\emptyset\not\supseteq E_1$, and for any countable disjoint collection $\{A_k\}_{k=1}^\infty$ of measurable sets,
	in the case that for all $k$, $A_k\not\supseteq E_1$, then $\bigcup_{k=1}^\infty A_k\not\supseteq E_1$ and 
	\[
		\mu\biggl(\bigcup_{k=1}^\infty A_k\biggr)=\sum_{k=1}^\infty\mu(A_k)=0.
	\]
	In the case that there exists an index $j$ such that $A_j\supseteq E_1$, then $\bigcup_{k=1}^\infty A_k\supseteq E_1$, and because the sets are disjoint, $A_i\not\supseteq E_1$ for $i\neq j$.
	\[
		\mu\biggl(\bigcup_{k=1}^\infty A_k\biggr)=\sum_{k=1}^\infty\mu(A_k)=\sum_{k\in\mathbb{N}\setminus\{j\}}\mu(A_k)+\mu(A_j)=0+\mu(A_j)=1.
	\]
	Then $\eta$ can also be shown to be a measure in the exact same way.
	\\Then we see that
	\begin{align*}
		\nu(E_1\cup E_2)&=\max\{\mu(E_1\cup E_2),\eta(E_1\cup E_2)\}=\max\{1,1\}=1,\\
		\nu(E_1)+\nu(E_2)&=\max\{\mu(E_1),\eta(E_1)\}+\max\{\mu(E_2),\eta(E_2)\}=\max\{1,0\}+\max\{0,1\}=2.
	\end{align*}
	Thus $\nu$ is not a measure because it does not satisfy countable additivity.
\end{enumerate}

% 17.2
\authoredby{finished}
\section{Signed Measures: The Hahn and Jordan Decompositions}
\begin{flushleft}
	\begin{namedthm*}{Definition}
		By a \textbf{signed measure} $\nu$ on the measurable space $(X,\mathcal{M})$ we mean an extended real-valued set function $\nu:\mathcal{M}\to[-\infty,\infty]$ that possesses the following properties:
		\begin{enumerate}[label=(\roman*),align=left]  
			\item $\nu$ assumes at most one of the values $+\infty,-\infty$.
			\item $\nu(\emptyset)=0$.
			\item For any countable collection $\{E_k\}_{k=1}^\infty$ of disjoint measurable sets,
			\[
				\nu\left(\bigcup_{k=1}^\infty E_k\right)=\sum_{k=1}^\infty\nu(E_k),
			\]
			where the series $\sum_{k=1}^\infty\nu(E_k)$ converges absolutely if $\nu(\bigcup_{k=1}^\infty E_k)$ is finite (convergence must hold for any rearrangement).
		\end{enumerate}
	\end{namedthm*}
	A set $A$ is \textbf{positive} (with respect to $\nu$) provided that $A$ is measurable and for every measurable subset $E$ of $A$ we have $\nu(E)\ge0$.
	The restriction of $\nu$ to the measurable subsets of a positive set is a measure.
	(See Problem 6).
	Similarly, a set $B$ is called \textbf{negative} (with respect to $\nu$) provided that $B$ is measurable and for every measurable subset $E$ of $B$ we have $\nu(E)\le0$.
	The restriction of $-\nu$ to the measurable subsets of a negative set is a measure.
	A measurable set is called \textbf{null} with respect to $\nu$ provided every measurable subset of it also has measure zero.
	(Clearly a null set is both positive and negative.)
	\\Monotonicity for signed measures:
	\[
	A\subseteq B\text{ and }|\nu(B)|<\infty\implies|\nu(A)|<\infty.
	\]
	It is not possible for a signed measure to take on both $\pm\infty$ at the same time.
		\\To see this, suppose there exist two subsets $E_1,E_2$ of $X$ such that $\nu(E_1)=\infty$ and $\nu(E_2)=-\infty$.
		\\If $-\infty<\nu(E_1\cap E_2)<\infty$, then 
		\begin{align*}
			\nu(E_1)&=\nu(E_1\cap E_2)+\nu(E_1\setminus E_2)=\infty\text{ so that }\nu(E_1\setminus E_2)=\infty,\\
			\nu(E_2)&=\nu(E_2\cap E_1)+\nu(E_2\setminus E_1)=-\infty\text{ so that }\nu(E_2\setminus E_1)=-\infty,
		\end{align*}
		and $E_1\setminus E_2$ and $E_2\setminus E_1$ are disjoint but $\infty-\infty$ is not defined.
		\\If $\nu(E_1\cap E_2)=\infty$, then 
		\begin{align*}
			-\infty=(E_2)=\nu(E_2\cap E_1)+\nu(E_2\setminus E_1)=\infty+\nu(E_2\setminus E_1),
		\end{align*}
		and we cannot find an $E_2\setminus E_1$ that satisfies this because $-\infty+\infty$ is not defined.
		\\If $\nu(E_1\cap E_2)=-\infty$, then 
		\begin{align*}
			\infty=(E_1)=\nu(E_1\cap E_2)+\nu(E_1\setminus E_2)=-\infty+\nu(E_1\setminus E_2),
		\end{align*}
		and we cannot find an $E_1\setminus E_2$ that satisfies this because $\infty-\infty$ is not defined.
	\begin{namedthm*}{The Hahn Decomposition Theorem}
		Let $\nu$ be a signed measure on the measurable space $(X,\mathcal{M})$.
		Then there is a positive set $A$ for $\nu$ and a negative set $B$ for $\nu$ for which
		\[
			X=A\cup B\text{ and }A\cap B=\emptyset.	
		\]
	\end{namedthm*}
	\begin{proof}
		Without loss of generality assume $+\infty$ is the infinite value omitted by $\nu$.
		(Otherwise let $A$ be the negative set and $B$ be the positive set. We need this to use Hahn's Lemma: $0<\nu(E)<\infty$.)
		\\Let $\mathcal{P}$ be the collection of positive subsets of $X$.
		Define $\lambda=\sup\{\nu(E)\ |\ E\in\mathcal{P}\}$, and $\lambda\ge0$ and it is nonempty because $\emptyset\in\mathcal{P}$.
		Then by definition of supremum, for each natural number $k$, there exists an element $A_k\in\lambda$ such that
		\[
			\lambda-\frac{1}{k}<\nu(A_k)\le\lambda.
		\]
		Then $\{A_k\}_{k=1}^\infty$ is a countable collection of positive subsets of $X$ for which $\lambda=\lim_{k\to\infty}\nu(A_k)$.
		We can let $A=\cup_{k=1}^\infty A_k$, and by Proposition 4, the countable union of a positive collection of positive sets is positive, and thus $A$ is a positive subset as well; then $\nu(A)\le\lambda$.
		Then for each $k$, we have $A\cap A_k^c\subseteq A$, and because $A$ is positive, all of its subsets including $A\setminus A_k$ is positive, so $\nu(A\setminus A_k)\ge0$.
		Then by countable disjoint additivity,
		\[
			\nu(A)=\nu(A_k)+\nu(A\setminus A_k)\ge\nu(A_k).
		\]
		Therefore $\nu(A)\ge\lim_{k\to\infty}\nu(A_k)=\lambda$ and $\nu(A)\le\lambda$ implies $\nu(A)=\lambda$.
		Also, $\lambda<\infty$ because $\nu$ does not take on the value $\infty$.
		\\Let $B=X\setminus A$.
		Supposing by contradiction that $B$ is not negative, then there exists a subset $E$ of $B$ of nonnegative measure: $\nu(E)\not\le0\text{ and }\nu(E)\neq\infty\implies 0<\nu(E)<\infty$, and therefore, by Hahn's Lemma, there exists a measurable subset $E_0$ of $E$ that is both positive and of positive measure.
		Then $A\cup E_0$ is a positive set and 
		\[
			\nu(A\cup E_0)=\nu(A)+\nu(E_0)>\lambda,
		\]
		a contradiction to the choice of $\lambda$ as the supremum.
	\end{proof}
	If $\{A,B\}$ is a Hahn decomposition for $\nu$, then we define two measures $\nu^+$ and $\nu^-$ (the positive and negative variations of $\nu$) with $\nu=\nu^+-\nu^-$ by setting
	\[
		\nu^+(E)=\nu(E\cap A)\text{ and }\nu^-(E)=-\nu(E\cap B).	
	\]
	(The disjoint sets $E\cap A$ and $E\cap B$ have positive and negative measure, respectively, because $A$ is positive (and thus all subsets have positive measure), and $B$ is negative (and thus all subsets have negative measure).) 
	\\\bigskip Two measures $\nu_1$ and $\nu_2$ on $(X,\mathcal{M})$ are said to be \textbf{mutually singular} ($\nu_1\perp\nu_2$) if there are disjoint measurable sets $A,B$ with $X=A\cup B$ for which $\nu_1(A)=\nu_2(B)=0$.
	\begin{namedthm*}{The Jordan Decomposition Theorem}
		Let $\nu$ be a signed measure on the measurable space $(X,\mathcal{M})$. 
		Then there are two mutually singular measures $\nu^+$ and $\nu^-$ on $(X,\mathcal{M})$ for which $\nu=\nu^+-\nu^-$.
		Moreover, there is only one such pair of mutually singular measures.
	\end{namedthm*}
	\begin{proof}
		Existence:
		\\By the Hahn Decomposition Theorem, there exists a positive set $A$ and a negative set $B$ for which $X=A\cup B$ and $A\cap B=\emptyset$.
		Then we can define $\nu^+$ and $\nu^-$ on $(X,\mathcal{M})$ such that
		\begin{align*}
			\nu^+(E)&=\nu(A\cap E)\text{ for all }E\in\mathcal{M}\\
			\nu^-(E)&=-\nu(B\cap E)\text{ for all }E\in\mathcal{M}
		\end{align*}
		Clearly  $\nu^+$ and $\nu^-$ are measures because 
			\begin{align*}
				\nu^+(E)&\ge0&&\text{because }A\cap E\subseteq A,\text{ and }A\text{ is positive, so for any }C\subseteq A,\text{ then }\nu(C)\ge0\\
				\nu^-(E)&\ge0&&\text{because }B\cap E\subseteq B,\text{ and }B\text{ is negative, so for any }C\subseteq B,\text{ then }\nu(C)\le0\implies-\nu(C)\ge0
			\end{align*}
			\begin{align*}
				\nu^+(\emptyset)&=\nu(A\cap\emptyset)=\nu(\emptyset)=0\\
				\nu^-(\emptyset)&=-\nu(B\cap\emptyset)=-\nu(\emptyset)=0
			\end{align*}
			For disjoint measurable collection $\{E_k\}_{k=1}^\infty$,
			\begin{align*}
				\nu^+(\bigcup_{k=1}^\infty E_k)&=\nu(A\cap\bigcup_{k=1}^\infty E_k)=\nu(\bigcup_{k=1}^\infty [A\cap E_k])=\sum_{k=1}^\infty\nu(A\cap E_k)=\sum_{k=1}^\infty\nu^+(E_k)\\
				\nu^-(\bigcup_{k=1}^\infty E_k)&=-\nu(B\cap\bigcup_{k=1}^\infty E_k)=-\nu(\bigcup_{k=1}^\infty [B\cap E_k])=\sum_{k=1}^\infty-\nu(B\cap E_k)=\sum_{k=1}^\infty\nu^-(E_k)
			\end{align*}
		The measures $\nu^+$ and $\nu^-$ are mutually singular because $X=A\cup B$, $A\cap B=\emptyset$, and 
		\begin{align*}
			\nu^+(B)&=\nu(A\cap B)=\nu(\emptyset)=0\\
			\nu^-(A)&=-\nu(B\cap A)=-\nu(\emptyset)=0
		\end{align*}
		\\Then for any measurable set $E$, we have $E=[E\cap A]\cup[E\cap B]$ so that 
		\[
			\nu(E)=\nu([E\cap A]\cup[E\cap B])=\nu(E\cap A)+\nu(E\cap B)=\nu^+(E)-\nu^-(E).
		\]
	\end{proof}
	We define $|\nu|$ on $\mathcal{M}$ by
	\[
		|\nu|(E)=\nu^+(E)+\nu^-(E)\text{ for all }E\in\mathcal{M}.
	\]
	In Problem 16, prove that 
	\begin{equation*}
		|\nu|(X)=\sup\sum_{k=1}^n|\nu(E_k)|,\tag{4}
	\end{equation*}
	where the supremum is taken over all finite disjoint collections $\{E_k\}_{k=1}^n$ of measurable subsets of $X$.
	For this reason $|\nu|(X)$ is called the \textbf{total variation} of $\nu$ and denoted by $\|\nu\|_{var}$.
	\\\bigskip
	\textbf{Example}
	Let $f:\mathbb{R}\to\mathbb{R}$ be a function that is Lebesgue integrable over $\mathbb{R}$.
	For a Lebesgue measurable set $E$, define $\nu(E)=\int_E f dm$.
	\\Then $\nu$ is a signed measure on the measurable space $(\mathbb{R},\mathcal{L})$:
	\begin{enumerate}[(i)]
		\item $f$ is integrable means that we have $\int_\mathbb{R}|f|<\infty$.
		Then because $|f\cdot\chi_E|\le|f|$ on $\mathbb{R}$, by the integral comparison test, $f\cdot\chi_E$ is integrable over $\mathbb{R}$.
		That is,
		\[
			|\nu(E)|=\left|\int_Efdm\right|=\left|\int_\mathbb{R}f\cdot\chi_Edm\right|<\infty.
		\]
		\item $\nu(\emptyset)=\int_\emptyset f dm=0$ (See Chapter 4 Problem 17).
		\item Let $E=\{E_k\}_{k=1}^\infty$ be a countable collection of disjoint measurable sets.
		Then by Chapter 4 Theorem 20,
		\[
			\nu\left(\bigcup_{k=1}^\infty E_k\right)=\nu(E)=\int_{E}fdm=\sum_{k=1}^\infty\int_{E_k}fdm=\sum_{k=1}^\infty\nu(E_k).
		\]
	\end{enumerate}
	Thus $\nu$ is a signed measure.
	\\Define 
	\begin{align*}
		A&:=\{x\in\mathbb{R}\mid f(x)\ge0\}\\
		B&:=\{x\in\mathbb{R}\mid f(x)<0\}
	\end{align*}
	and define for each Lebesgue measurable set $E$,
	\begin{align*}
		\nu^+(E)&:=\nu(A\cap E)=\int_{A\cap E} f dm\\
		\nu^-(E)&:=-\nu(B\cap E)=-\int_{B\cap E} f dm
	\end{align*}
	Then $\{A,B\}$ is a Hahn decomposition of $\mathbb{R}$ w.r.t. the signed measure $\nu$.
	Moreover, $\nu=\nu^+-\nu^-$ is a Jordan decomposition of $\nu$.
\end{flushleft}
\begin{center}
	\textbf{PROBLEMS}
\end{center}
\begin{enumerate}
	\setcounter{enumi}{11}
	\item In the above example, let $E$ be a Lebesgue measurable set such that $0<\nu(E)<\infty$.
	Find a positive set $A$ contained in $E$ for which $\nu(A)>0$.\\
	\\Consider the set $A'=A\cap E$, which is a positive set because it is a measurable subset of the positive set $A$.
	\\Then
	\[
		\nu(A')=\nu(A\cap E)=\int_{A\cap A'} f dm-\int_{B\cap A'} f dm=\int_{A\cap E} f dm\ge0.
	\]
	If we suppose that $\nu(A')=0$, then we have
	\[
		\nu(E)=\int_{A\cap E} f dm-\int_{B\cap E} f dm=-\int_{B\cap E} f dm\le0,
	\]
	which is a contradiction to $\nu(E)>0$.
	\\Therefore $\nu(A')>0$.
	\item Let $\mu$ be a measure and $\mu_1$ and $\mu_2$ be mutually singular measures on a measurable space $(X,\mathcal{M})$ for which $\mu=\mu_1-\mu_2$.
	Show that $\mu_2=0$.
	Use this to establish the uniqueness assertion of the Jordan Decomposition Theorem.\\
	\\Because $\mu_1$ and $\mu_2$ are mutually singular, then there exist disjoint measurable sets $A,B$ with $X=A\cup B$ for which $\mu_1(A)=\mu_2(B)=0$.
	\\Consider $E\in\mathcal{M}$.
	\\In the case $E\subseteq B$, we have $\mu_2(E)=0$ by monotonicity of measure.
	\\In the case $E\subseteq A$, we have $\mu_1(E)=0$, so that
	\[
		\mu(E)=\mu_1(E)-\mu_2(E)=-\mu_2(E)\le0,
	\]
	where $\mu(E)\ge0$ implies $\mu_2(E)=0$.
	\\Therefore for any measurable set $E$,
	\[
		\mu_2(E)=\mu_2(E\cap A)+\mu_2(E\cap B)=0+0=0.
	\]
	\\Suppose $\nu$ is a signed measure on the measurable space $(X,\mathcal{M})$, and suppose we have any two pairs of mutually singular measures $(\nu_1^+,\nu_1^-)$ and $(\nu_2^+,\nu_2^-)$ so that $\nu=\nu_1^+-\nu_1^-=\nu_2^+-\nu_2^-$.
	\\By definition of mutually singular, there exist disjoint pairs $(A_1,B_1)$ and $(A_2,B_2)$ with $X=A_1\cup B_1=A_2\cup B_2$ for which $\nu_1^+(A_1)=\nu_1^-(B_1)=0$ and $\nu_2^+(A_2)=\nu_2^-(B_2)=0$.
	\\Let $E\in\mathcal{M}$.
	\\In the case $E\subseteq B_1$,
	\[
		\nu_1^+(E)=\nu_2^+(E)-\nu_2^-(E).
	\]
	Restricting $\nu_1^+$ to all measurable subsets of $B_1$ (see Problem 6), we have that $\nu_2^-=0$ so that $\nu_1^+=\nu_2^+$.
	\\In the case $E\subseteq A_1$,
	\[
		\nu_1^-(E)=-\nu_2^+(E)+\nu_2^-(E).
	\]
	Restricting $\nu_1^-$ to all measurable subsets of $A_1$, we have that $\nu_2^+=0$ so that $\nu_1^-=\nu_2^-$.\\
	\\Therefore for any measurable set $E$,
	\[
		\nu_1^+(E)=\nu_1^+(E\cap B_1)=\nu_2^+(E\cap B_1)=\nu_2^+(E),
	\]
	and Similarly,
	\[
		\nu_1^-(E)=\nu_1^-(E\cap A_1)=\nu_2^-(E\cap A_1)=\nu_2^-(E).
	\]
	\item Show that if $E$ is any measurable set, then
	\[
		-\nu^-(E)\le\nu(E)\le\nu^+(E)\text{ and }|\nu(E)|\le|\nu|(E).	
	\]
	\\We have the Jordan decomposition
	\[
		\nu(E)=\nu^+(E)-\nu^-(E),
	\]
	where $\nu^+$ and $\nu^-$ are measures so that they are nonnegative; therefore
	\[
		0\le\nu^-(E)\implies \nu(E)\le\nu^+(E),
	\]
	and
	\[
		\nu^+(E)\ge0\implies \nu(E)\ge-\nu^-(E).
	\]
	By the triangle inequality,
	\[
		|\nu(E)|=|\nu^+(E)-\nu^-(E)|\le\nu^+(E)+\nu^-(E)=|\nu|(E).
	\]
	\item Show that if $\nu_1$ and $\nu_2$ are any two finite signed measures, then so is $\alpha\nu_1+\beta\nu_2$, where $\alpha$ and $\beta$ are real numbers. Show that
	\[
		|\alpha\nu|=|\alpha||\nu|\text{ and }|\nu_1+\nu_2|\le|\nu_1|+|\nu_2|,
	\]
	where $\nu\le\mu$ means $\nu(E)\le\mu(E)$ for all measurable sets $E$.\\
	\\Let $E$ be any measurable set.
	\begin{enumerate}[(i)]
		\item $|\alpha\nu_1(E)+\beta\nu_2(E)|\le|\alpha||\nu_1(E)|+|\beta||\nu_2(E)|<|\alpha|\cdot\infty+|\beta|\cdot\infty=\infty$
		\item $\alpha\nu_1(\emptyset)+\beta\nu_2(\emptyset)=\alpha\cdot0+\beta\cdot0=0$
		\item Let $\{E_k\}_{k=1}^\infty$ be a countable collection of disjoint measurable sets.
		\[
			\alpha\nu_1(\bigcup_{k=1}^\infty E_k)+\beta\nu_2(\bigcup_{k=1}^\infty E_k)=\alpha\sum_{k=1}^\infty\nu_1(E_k)+\beta\sum_{k=1}^\infty\nu_2(E_k)=\sum_{k=1}^\infty\left[\alpha\nu_1(E_k)+\beta\nu_2(E_k)\right].
		\]
	\end{enumerate}
	Therefore $\alpha\nu_1+\beta\nu_2$ is itself a finite signed measure.
	\\Homogeneity and subadditivity of the absolute value come from the properties of the real numbers. 
	\item Prove (4).\\
	\\Let $\{E_k\}_{k=1}^n$ be a finite, measurable partition of $X$.
	\\Recall the definition $|\nu|=\nu^++\nu^-$.
	\\From problem 14, we have $|\nu(E_k)|\le|\nu|(E_k)$ for each $k$.
	\\Then 
	\begin{align*}
		\sum_{k=1}^n|\nu(E_k)|
		&\le\sum_{k=1}^n|\nu|(E_k)\\
		&=\sum_{k=1}^n\nu^+(E_k)+\sum_{k=1}^n\nu^-(E_k)\\
		&=\nu^+(\bigcup_{k=1}^n E_k)+\nu^-(\bigcup_{k=1}^n E_k)\\
		&=\nu^+(X)+\nu^-(X)\\
		&=|\nu|(X),
	\end{align*}
	so that
	\[
		\sup\sum_{k=1}^n|\nu(E_k)|\le|\nu|(X).\tag{a}
	\]
	\\Because $\nu^+$ and $\nu^-$ are mutually singular measures, consider the disjoint sets $E_1,E_2$ such that $X=E_1\cup E_2$ and $\nu^+(E_1)=\nu^-(E_2)=0$.
	\\Therefore $\nu(E_1)=\nu^+(E_1)-\nu^-(E_1)=-\nu^-(E_1)$ and $\nu(E_2)=\nu^+(E_2)-\nu^-(E_2)=\nu^+(E_2)$ so that 
	\begin{align*}
		|\nu|(X)&=|\nu|(E_1\cup E_2)\\
		&=\nu^+(E_1\cup E_2)+\nu^-(E_1\cup E_2)\\
		%&=\nu^+(E_1)+\nu^+(E_2)+\nu^-(E_1)+\nu^-(E_2)&&\text{disjoint additivity}\\
		&=\nu^-(E_1)+\nu^+(E_2)\\
		&=|\nu(E_1)|+|\nu(E_2)|\\
		&=\sum_{k=1}^2|\nu(E_k)|\\
		&\le\sup\sum_{k=1}^n|\nu(E_k)|.\tag{b}
	\end{align*}
	Then (a) and (b) imply equality:
	\[
		|\nu|(X)=\sup\sum_{k=1}^n|\nu(E_k)|.
	\]
	\item Let $\mu$ and $\nu$ be finite signed measures.
	Define $\mu\land\nu=\frac{1}{2}(\mu+\nu-|\mu-\nu|)$ and $\mu\lor\nu=\mu+\nu-\mu\land\nu$.
	\begin{enumerate}[label=(\roman*),align=left]  
		\item Show that the signed measure $\mu\land\nu$ is smaller than $\mu$ and $\nu$ but larger than any other signed measure that is smaller than $\mu$ and $\nu$.
		\item Show that the signed measure $\mu\lor\nu$ is larger than $\mu$ and $\nu$ but smaller than any other signed measure that is larger than $\mu$ and $\nu$.
		\item If $\mu$ and $\nu$ are positive measures, show that they are mutually singular iff $\mu\land\nu=0$.\\
	\end{enumerate}
	We have the identities
	\begin{align*}
		\max(\mu,\nu)+\min(\mu,\nu)&=\mu+\nu\\
		\max(\mu,\nu)-\min(\mu,\nu)&=|\mu-\nu|\\
		\max(\mu,\nu)&=\frac{1}{2}(\mu+\nu+|\mu-\nu|)\\
		\min(\mu,\nu)&=\frac{1}{2}(\mu+\nu-|\mu-\nu|)
	\end{align*}
	\begin{enumerate}[label=(\roman*),align=left]  
		\item $\mu(E)\land\nu(E)=\min(\mu(E),\nu(E))\le\mu(E),\nu(E)$.
		\\If $\lambda(E)\le\mu(E),\nu(E)$, then $\lambda(E)\le\min\{\mu(E),\nu(E)\}=\mu(E)\land\nu(E)$.
		\item We can see
		\[
			\mu\lor\nu=\frac{1}{2}(2\mu+2\nu)-\frac{1}{2}(\mu+\nu-|\mu-\nu|)=\frac{1}{2}(\mu+\nu+|\mu-\nu|),
		\]
		So that $\mu(E)\lor\nu(E)=\max(\mu(E),\nu(E))\ge\mu(E),\nu(E)$.
		\\If $\lambda(E)\ge\mu(E),\nu(E)$, then $\lambda(E)\ge\max\{\mu(E),\nu(E)\}=\mu(E)\lor\nu(E)$.
		\item Let $\mu$ and $\nu$ be positive measures.
		\\$(\implies)$ Suppose that $\mu$ and $\nu$ are mutually singular.
		\\Then there exist disjoint measurable sets $A,B$ with $X=A\cup B$ s.t. $\mu(A)=\nu(B)=0$.
		\\Let $E$ be a measurable set.
		\\In the case $E\subseteq A$,
		\[
			(\mu\land\nu)(E)=\frac{1}{2}(0+\nu(E)-|0-\nu(E)|)=0.
		\]
		In the case $E\subseteq B$,
		\[
			(\mu\land\nu)(E)=\frac{1}{2}(\mu(E)+0-|\mu(E)-0|)=0.
		\]
		Therefore, for any measurable set $E$,
		\[
			(\mu\land\nu)(E)=(\mu\land\nu)(E\cap A)+(\mu\land\nu)(E\cap B)=0+0=0.
		\]
		\\$(\impliedby)$ Suppose that $\mu\land\nu=0$.\\
		\\This implies that for any measurable set $E$, at least one of $\mu(E)$ and $\nu(E)$ must equal zero. 
		\\Consider the finite signed measure $\lambda=\mu-\nu$. 
		% By the Jordan Decomposition Theorem, there exists a pair of mutually singular measures $\lambda^+$ and $\lambda^-$ for which $\lambda=\lambda^+-\lambda^-$.
		% By definition of mutually singular, there exist disjoint measurable sets $P,N$ with $X=P\cup N$ s.t. $\lambda^+(N)=\lambda^-(P)=0$.
		\\By the Hahn Decomposition Theorem, there is a positive set $P$ for $\lambda$ and a negative set $N$ for $\lambda$ for which
		\[
			X=P\cup N\text{ and }P\cap N=\emptyset.	
		\]
		%By the Jordan Decomposition Theorem, we have $\lambda=\lambda^+-\lambda^-$.
		Let $E$ be a measurable set.
		\\In the case $E\subseteq P$,
		\[
			\mu(E)-\nu(E)\ge0.
		\]
		\begin{itemize}
			\item If $\mu(E)-\nu(E)>0$, then $\mu(E)>0$ and $\nu(E)=0$.
			\item If $\mu(E)-\nu(E)=0$, then $\mu(E)=\nu(E)=0$.
		\end{itemize}
		In the case $E\subseteq N$,
		\[
			\mu(E)-\nu(E)\le0.
		\]
		\begin{itemize}
			\item If $\mu(E)-\nu(E)<0$, then $\mu(E)=0$ and $-\nu(E)<0$.
			\item If $\mu(E)-\nu(E)=0$, then $\mu(E)=\nu(E)=0$.
		\end{itemize}
		Then $\mu(N)=\nu(P)=0$ so that $\mu$ and $\nu$ are mutually singular.
	\end{enumerate}
\end{enumerate}

% 17.3
\authoredby{inprogress}
\section{The Cath\'eodory Measure Induced by an Outer Measure}


% no problems

% 17.4
\authoredby{untouched}
\section{The Construction of Outer Measures}
\begin{center}
	\textbf{PROBLEMS}
\end{center}
\begin{enumerate}
	\setcounter{enumi}{17}
	\item Let $\mu^*:2^X\to[0,\infty]$ be an outer measure.
	Let $A\subseteq X, \{E_k\}_{k=1}^\infty$ be a disjoint countable collection of measurable sets and $E=\bigcup_{k=1}^\infty E_k$. Show that
	\[
		\mu^*(A\cap E)=\sum_{k=1}^\infty \mu^*(A\cap E_k).
	\]
	\item Show that any measure that is induced by an outer measure is complete.
	\item Let $X$ be any set.
	Define $\eta^*:2^X\to[0,\infty]$ by defining $\eta(\emptyset)=0$ and for $E\subseteq X,E\neq\emptyset$, defining $\eta(E)=\infty$.
	Show that $\eta$ is an outer measure.
	Also show that the set function that assigns $0$ to every subset of $X$ is an outer measure.
	\item Let $X$ be a set, $\mathcal{S}=\{\emptyset,X\}$, and define $\mu(\emptyset)=0,\mu(X)=1$. 
	Determine the outer measure $\mu^*$ induced by the set function $\mu:\mathcal{S}\to[0,\infty)$ and the $\sigma$-algebra of measurable sets.
	\item On the collection $\mathcal{S}=\{\emptyset,[1,2]\}$ of subsets of $\mathbb{R}$, define the set function $\mu:\mathcal{S}\to[0,\infty)$ as follows: $\mu(\emptyset)=0,\mu([1,2])=1$. 
	Determine the outer measure $\mu^*$ induced by $\mu$ and the $\sigma$-algebra of measurable sets.
	\item On the collection $\mathcal{S}$ of all subsets of $\mathbb{R}$, define the set function $\mu:\mathcal{S}\to\mathbb{R}$ by setting $\mu(A)$ to be the number of integers in $A$.
	Determine the outer measure $\mu^*$ induced by $\mu$ and the $\sigma$-algebra of measurable sets.
	\item Let $\mathcal{S}$ be a collection of subsets of $X$ and $\mu:\mathcal{S}\to[0,\infty]$ a set function.
	Is every set in $\mathcal{S}$ measurable with respect to the outer measure induced by $\mu$?
\end{enumerate}

% 17.5
\section{The Cath\'eodory-Hahn Theorem: The Extension of a Premeasure to a Measure}
\begin{center}
	\textbf{PROBLEMS}
\end{center}
\begin{enumerate}
	\setcounter{enumi}{24}
	\item Let $X$ be any set containing more than one point and $A$ a proper nonempty subset of $X$.
	Define $\mathcal{S}=\{A,X\}$ and the set function $\mu:\mathcal{S}\to[0,\infty]$ by $\mu(A)=1$ and $\mu(X)=2$.
	Show that $\mu:\mathcal{S}\to[0,\infty]$ is a premeasure.
	Can $\mu$ be extended to a measure?
	What are the subsets of $X$ that are measurable with respect to the outer measure $\mu^*$ induced by $\mu$?
	\item Consider the collection $\mathcal{S}=\{\emptyset,[0,1],[0,3],[2,3]\}$ of subsets of $\mathbb{R}$ and define $\mu(\emptyset)=0,\mu([0,1])=1,\mu([0,3])=1,\mu([2,3])=1$.
	Show that $\mu:\mathcal{S}\to[0,\infty]$ is a premeasure.
	Can $\mu$ be extended to a measure?
	What are the subsets of $\mathbb{R}$ that are measurable with respect to the outer measure $\mu^*$ induced by $\mu$?
	\item Let $\mathbb{S}$ be a collection of subsets of a set $X$ and $\mu:\mathcal{S}\to[0,\infty]$ a set function.
	Show that $\mu$ is countably monotone iff $\mu^*$ is an extension of $\mu$.
	\item Show that a set function is a premeasure if it has an extension that is a measure.
	\item Show that a set function on a $\sigma$-algebra is a measure iff it is a premeasure.
	\item Let $\mathcal{S}$ be a collection of sets that is closed with respect to the formation of finite unions and finite intersections.
	\begin{enumerate}[label=(\roman*),align=left]  
		\item Show that $\mathcal{S}_\sigma$ is closed with respect to the formation of countable unions and finite intersections.
		\item Show that each set in $\mathcal{S}_{\sigma\delta}$ is the intersection of a decreasing sequence of $\mathcal{S}_\sigma$ sets.
	\end{enumerate}
	\item Let $\mathcal{S}$ be a semialgebra of subsets of a set $X$ and $\mathcal{S}'$ the collection of unions of finite disjoint collections of sets in $\mathcal{S}$.
	\begin{enumerate}[label=(\roman*),align=left]  
		\item Show that $\mathcal{S}'$ is an algebra.
		\item Show that $\mathcal{S}_\sigma=\mathcal{S}_\sigma'$ and therefore $\mathcal{S}_{\sigma\delta}=\mathcal{S}_{\sigma\delta}'$.
		\item Let $\{E_k\}_{k=1}^\infty$ be a collection of sets in $\mathcal{S}'$. Show that we can express 
		\[
			\sum_{k=1}^\infty \mu'(E_k')\ge\sum_{k=1}^\infty \mu(E_k).
		\]
		\item Let $A$ belong to $\mathcal{S}_{\sigma\delta}'$. Show that $A$ is the intersection of a descending sequence $\{A_k\}_{k=1}^\infty$ of sets in $\mathcal{S}_\sigma$.
	\end{enumerate}
	\item Let $\mathbb{Q}$ be the set of rational numbers and and $\mathcal{S}$ the collection of all finite unions of intervals of the form $(a,b]\cap\mathbb{Q}$, where $a,b\in\mathbb{Q}$ and $a\le b$. 
	Define $\mu((a,b])=\infty$ if $a<b$ and $\mu(\emptyset)=0$.
	Show that $\mathcal{S}$ is closed with respect to the formation of relative complements and $\mu:\mathcal{S}\to[0,\infty]$ is a premeasure.
	Then show that the extension of $\mu$ to the smallest $\sigma$-algebra containing $\mathcal{S}$ is not unique.
	\item By a bounded interval of real numbers we mean a set of the form $[a,b],[a,b),(a,b]$, or $(a,b)$ for real numbers $a\le b$.
	Thus we consider the empty-set and a set consisting of a single point to be a bounded interval.
	Show that each of the following three collections of sets $\mathcal{S}$ is a semiring.
	\begin{enumerate}[label=(\roman*),align=left]  
		\item Let $\mathcal{S}$ be the collection of all bounded intervals of real numbers.
		\item Let $\mathcal{S}$ be the collection of all subsets of $\mathbb{R}\times\mathbb{R}$ that are products of bounded intervals of real numbers.
		\item Let $n$ be a natural number an $X$ be the $n$-fold Cartesian product of $\mathbb{R}$:
		\[
			X=\mathbb{R}\times\cdots\times\mathbb{R}=\mathbb{R}^n.
		\]
		Let $\mathcal{S}$ be the collection of all subsets of $X$ that are $n$-fold Cartesian products of bounded intervals of real numbers.
	\end{enumerate}
	\item If we start with an outer measure $\mu^*$ on $2^X$ and form the induced measure $\overline\mu$ on the $\mu^*$-measurable sets, we can view $\overline\mu$ as a set function and denote by $\mu^+$ the outer measure induced by $\overline\mu$.
	\begin{enumerate}[label=(\roman*),align=left]  
		\item Show that for each set $E\subset X$ we have $\mu^+(E)\ge\mu^*(E)$.
		\item For a given set $E$, show that $\mu^+(E)=\mu^*(E)$ iff there is a $\mu^*$-measurable set $A\supseteq E$ with $\mu^*(A)=\mu^*(E)$.
	\end{enumerate}
	\item Let $\mathcal{S}$ be a $\sigma$-algebra of subsets of $X$ and $\mu:\mathcal{S}\to[0,\infty]$ a measure.
	Let $\overline\mu:\mathcal{M}\to[0,\infty]$ be the measure induced by $\mu$ via the Carath\'eodory construction.
	Show that $\mathcal{S}$ is a subcollection of $\mathcal{M}$ and it may be a proper subcollection.
	\item Let $\mu$ be a finite premeasure on an algebra $\mathcal{S}$, and $\mu^*$ the induced outer measure.
	Show that a subset $E$ of $X$ is $\mu^*$-measurable iff for each $\epsilon>0$ there is a set $A\in\mathcal{S}_\delta,A\subseteq E$, such that $\mu^*(E\setminus A)<\epsilon$.
\end{enumerate}
