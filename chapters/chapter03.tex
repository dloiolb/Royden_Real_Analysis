% Chapter 3
\chapter{Lebesgue Measurable Functions}

% 3.1
\section{Sums, Products, and Compositions}
\begin{center}
	\textbf{PROBLEMS}
\end{center}
\begin{enumerate}
	\setcounter{enumi}{0}
	\item Suppose $f$ and $g$ are continuous functions on $[a,b]$. Show that if $f=g$ a.e. on $[a,b]$, then, in fact, $f=g$ on $[a,b]$.
    Is a similar assertion true if $[a,b]$ is replaced by a general measurable set $E$?\\
    \\hi
    \item Let $D$ and $E$ be measurable sets and $f$ a function with domain $D\cup E$. We proved that $f$ is measurable on $D\cup E$ iff its restrictions to $D$ and $E$ are measurable.
    Is the same true if "measurable" is replaced by "continuous"?\\
    \\No; consider the function $f:[-1,1]\to\mathbb{R}$, where $[-1,1]=[-1,0)\cup[0,1]$, and we define
    \[
    f(x)=
    \begin{cases}
        0&x\in[-1,0),\\
        1&x\in[0,1].
    \end{cases}    
    \]
    Clearly we have a point of discontinuity at $x=0$, so $f$ is not continuous even though $f|_{[-1,0)}$ and $f|_{[0,1]}$ are continuous.
    \item Suppose a function $f$ has a measurable domain and is continuous except at a finite number of points.
    Is $f$ necessarily measurable?
    \item Suppose $f$ is a real-valued function on $\mathbb{R}$ such that $f^{-1}(c)$ is measurable for each number $c$. Is $f$ necessarily measurable?
    \item Suppose the function $f$ is defined on a measurable set $E$ and $\{x\in E\ |\ f(x)>c\}$ is a measurable set for each rational number $c$. Is $f$ necessarily a measurable function?
    \item Let $f$ be a function with measurable domain $D$. Show that $f$ is measurable iff the function $g$ defined on $\mathbb{R}$ by $g(x)=f(x)$ for $x\in D$ and $g(x)=0$ for $x\notin D$ is measurable.
    \item Let the function $f$ be defined on a measurable set $E$. Show that $f$ is measurable iff for each borel set $A$, $f^{-1}(A)$ is measurable. (Hint: the collection of sets $A$ that have the property that $f^{-1}(A)$ is measurable is a $\sigma$-algebra.)
    \item (Borel measurability) A function $f$ is said to be \textbf{Borel measurable} provided its domain $E$ is a Borel set and for each $c$, the set $\{x\in E\ |\ f(x)>c\}$ is a Borel set.
    Verify that Proposition 1 and Theorem 6 remain valid if we replace "(Lebesgue) measurable set" by "Borel set".
    Show that:
    \begin{enumerate}[label=(\roman*),align=left]
        \item every Borel measurable function is Lebesgue measurable
        \item if $f$ is Borel measurable and $B$ is a Borel set, then $f^{-1}(B)$ is a Borel set
        \item if $f$ and $g$ are Borel measurable, so is $f\circ g$, 
        \item if $f$ is Borel measurable and $g$ is Lebesgue measurable, then $f\circ g$ is Lebesgue measurable.
    \end{enumerate}
    \item Let $\{f_n\}$ be a sequence of measurable functions defined on a measurable set $E$.
    Define $E_0$ to be the set of points of $x$ in $E$ at which $\{f_n(x)\}$ converges. Is the set $E_0$ measurable?
    \item Suppose $f$ and $g$ are real-valued functions defined on all of $\mathbb{R}$, $f$ is measurable, and $g$ is continuous.
    Is the composition $f\circ g$ necessarily measurable?
    \item Let $f$ be a measurable function and $g$ be a one-to-one function from $\mathbb{R}$ onto $\mathbb{R}$ which has a Lipschitz inverse. Show that the composition $f\circ g$ is measurable. (Hint: examine Problem 37 in Chapter 2.)
\end{enumerate}

% 3.2
\section{Sequential Pointwise Limits and Simple Approximation}
\begin{center}
	\textbf{PROBLEMS}
\end{center}
\begin{enumerate}
	\setcounter{enumi}{11}
    \item Let $f$ be a bounded measurable function on $E$. Show that there are sequences of simple functions on $E$, $\{\varphi_n\}$ and $\{\psi_n\}$, such that $\{\varphi_n\}$ is increasing and $\{\psi_n\}$ is decreasing and each of these sequences converges to $f$ uniformly on $E$.
    \item A real-valued measurable function is said to be \textit{semisimple} provided it takes only a countable number of values. Let $f$ be any measurable function on $E$.
    Show that there is a sequence of semisimple functions $\{f_n\}$ on $E$ that converges to $f$ uniformly on $E$.
    \item Let $f$ be a measurable function on $E$ that is finite a.e. on $E$ and $m(E)<\infty$.
    For each $\epsilon>0$, show that there is a measurable set $F$ contained in $E$ such that $f$ is bounded on $F$ and $m(E\setminus F)<\epsilon$.
    \item Let $f$ be a measurable function on $E$ that if finite a.e. on $E$ and $m(E)<\infty$. Show that for each $\epsilon>0$, there is a measurable set $F$ contained in $E$ and a sequence $\{\varphi_n\}$ of simple functions on $E$ such that $\{\varphi_n\}\to f$ uniformly on $F$ and $m(E\setminus F)<\epsilon$. (Hint: see the preceding problem.)
    \item Let $I$ be a closed, bounded interval and $E$ a measurable subset of $I$. Let $\epsilon>0$.
    Show that there is a step function $h$ on $I$ and a measurable subset $F$ of $I$ for which 
    \[
        h=\chi_E\text{ on }F\text{ and }m(I\setminus F)<\epsilon.    
    \]
    (Hint: use Theorem 12 of Chapter 2.)
    \item Let $I$ be a closed, bounded interval and $\psi$ a simple function defined on $I$. Let $\epsilon>0$.
    Show that there is a step function $h$ on $I$ and a measurable subset $F$ of $I$ for which 
    \[
        h=\psi\text{ on }F\text{ and }m(I\setminus F)<\epsilon.    
    \] 
    (Hint: use the fact that a simple function is a linear combination of characteristic functions and the preceding problem.)
    \item Let $I$ be a closed, bounded interval and $f$ a bounded measurable function defined on $I$. Let $\epsilon>0$.
    Show that there is a step function $h$ on $I$ and a measurable subset $F$ of $I$ for which 
    \[
        |h-f|<\epsilon\text{ on }F\text{ and }m(I\setminus F)<\epsilon.    
    \]
    \item Show that the sum and product of two simple functions are simple as are the max and the min.
    \item Let $A,B$ be any sets. Show that
    \begin{align*}
        \chi_{A\cap B}&=\chi_A\cdot\chi_B\\
        \chi_{A\cup B}&=\chi_A+\chi_B-\chi_A\cdot\chi_B\\
        \chi_{A^c}&=1-\chi_A\\
    \end{align*}
    \item For a sequence $\{f_n\}$ of measurable functions with common domain $E$, show that each of the following functions is measurable:
    \begin{itemize}
        \item $\inf\{f_n\}$
        \item $\sup\{f_n\}$
        \item $\lim\inf\{f_n\}$
        \item $\lim\sup\{f_n\}$
    \end{itemize}
    \item (Dini's Theorem) Let $\{f_n\}$ be an increasing sequence of continuous functions on $[a,b]$ which converges pointwise on $[a,b]$ to the continuous function $f$ on $[a,b]$.
    Show that the convergence is uniform on $[a,b]$. (Hint: let $\epsilon>0$. For each natural number $n$, define $E_n=\{x\in[a,b]\ |\ f(x)-f_n(x)<\epsilon\}$. Show that $\{E_n\}$ is an open cover of $[a,b]$ and use the Heine-Borel Theorem.)
    \item Express a measurable function as the difference of nonnegative measurable functions and thereby prove the general Simple Approximation Theorem based on the special case of a nonnegative measurable function.
    \item Let $I$ be an interval and $f:I\to\mathbb{R}$ be increasing. Show that $f$ is measurable by first showing that, for each natural number $n$, the strictly increasing function $x\mapsto f(x)+x/n$ is measurable, and then taking pointwise limits.
\end{enumerate}

% 3.3
\section{Littlewood's Three Principles, Ergoff's Theorem, and Lusin's Theorem`'}
\begin{center}
	\textbf{PROBLEMS}
\end{center}
\begin{enumerate}
	\setcounter{enumi}{24}
    \item Suppose $f$ is a function that is continuous on a closed set $F$ of real numbers. Show that $f$ has a continuous extension to all of $\mathbb{R}$. This is a special case of the forthcoming Tietze Extension Theorem.
    (Hint: express $\mathbb{R}\setminus F$ as the union of a countable disjoint collection of open intervals and define $f$ to be linear on the closure of each of these intervals.)
    \item For the function $f$ and the set $F$ in the statement of Lusin's Theorem, show that the restriction of $f$ to $F$ is a continuous function.
    Must there be any points at which $f$, considered as a function of $E$, is continuous?
    \item Show that the conclusion of Egoroff's Theorem can fail if we drop the assumption that the domain has finite measure.
    \item Show that Egoroff's Theorem continues to hold if the convergence is pointwise a.e. and $f$ is finite a.e.
    \item Prove the extension of Lusin's Theorem to the case that $E$ has finite measure.
    \item Prove the extension of Lusin's Theorem to the case that $f$ is not necessarily real-valued, but is finite a.e.
    \item Let $\{f_n\}$ be a sequence of measurable functions on $E$ that converges to the real-valued $f$ pointwise on $E$. Show that $E=\bigcup_{k=1}^\infty E_k$, where for each index $k$, $E_k$ is measurable, and $\{f_n\}$ converges uniformly to $f$ on each $E_k$ if $k>1$, and $m(E_1)=0$.  
\end{enumerate}