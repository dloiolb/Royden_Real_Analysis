% Chapter 3
\chapter{Lebesgue Measurable Functions}

% 3.1
\section{Sums, Products, and Compositions}
\begin{flushleft}
    \begin{namedthm*}{Proposition 1}
        Let the function $f$ have a measurable domain $E$. Then the following statements are equivalent:
        \begin{enumerate}[label=(\roman*),align=left]
            \item For each real number $c$, the set $\{x\in E\ |\ f(x)>c\}$ is measurable.
            \item For each real number $c$, the set $\{x\in E\ |\ f(x)\ge c\}$ is measurable.
            \item For each real number $c$, the set $\{x\in E\ |\ f(x)<c\}$ is measurable.
            \item For each real number $c$, the set $\{x\in E\ |\ f(x)\le c\}$ is measurable.
        \end{enumerate}
        Each of these properties implies that for each extended real number $c$,
        \begin{center}
            the set $\{x\in E\ |\ f(x)= c\}$ is measurable.
        \end{center}
    \end{namedthm*}

    \begin{namedthm*}{Definition}
        An extended real-valued function $f$ defined on $E$ is said to be \textbf{Lebesgue measurable}, or simply \textbf{measurable}, provided its domain $E$ is measurable and it satisfies one of the four statements of Proposition 1.        
    \end{namedthm*}

\begin{namedthm*}{Proposition 2}
    Let the real-valued function $f$ be defined on a measurable set $E$.
    Then the function $f$ is measurable iff for each open set $\mathcal{O}$, the inverse image of $\mathcal{O}$ under f, $f^{-1}(\mathcal{O})=\{x\in E\ |\ f(x)\in\mathcal{O}\}$, is a measurable set.
\end{namedthm*}
\begin{proof}
    Let $f:E\to\mathbb{R}$, where $E$ is a measurable set.\\
    $(\implies)$ Suppose that $f$ is measurable.\\
    Let $\mathcal{O}$ be open. Then by Chapter 1, Proposition 9, $\mathcal{O}$ can be written as the countable disjoint union of open intervals: $\mathcal{O}=\bigcup_{k=1}^\infty I_k$.
    We can construct these intervals in the following form:
    \[
        I_k=(a_k,b_k)=(-\infty,b_k)\cap(a_k,\infty)  
    \] 
    Therefore we see that
    \begin{align*}
        f^{-1}(\mathcal{O})&=f^{-1}(\bigcup_{k=1}^\infty I_k)\\
        &=f^{-1}(\bigcup_{k=1}^\infty (-\infty,b_k)\cap(a_k,\infty))\\
        &=\bigcup_{k=1}^\infty f^{-1}((-\infty,b_k)\cap(a_k,\infty))\\
        &=\bigcup_{k=1}^\infty f^{-1}(-\infty,b_k)\cap f^{-1}(a_k,\infty).
    \end{align*}
    Because $f$ is measurable, we see that $f^{-1}((-\infty,b_k))$ and $f^{-1}((a_k,\infty))$ are measurable sets, and countable union and intersection of measurable sets is also a measurable set, so $f^{-1}(\mathcal{O})$ is a measurable set.
    \\$(\impliedby)$ Suppose that for each open set $\mathcal{O}$, $f^{-1}(\mathcal{O})$ is a measurable set.\\
    Because for any real number $c$, the interval of the form $(c,\infty)$ is an open set, and therefore we have that the set $f^{-1}((c,\infty))=\{x\in E\ |\ f(x)\in(c,\infty)\}=\{x\in E\ |\ f(x)>c\}$ is measurable, which implies that $f$ is measurable.
\end{proof}

\begin{namedthm*}{Proposition 5}
    Let $f$ be an extended real-valued function on the measurable set $E$.
    \begin{enumerate}[label=(\roman*),align=right]
        \item If $f$ is measurable on $E$ and $f=g$ a.e. on $E$, then $g$ is measurable on $E$.
        \item For a measurable subset $D$ of $E$, $f$ is measurable on $E$ iff the restrictions of $f$ to $D$ and $E\setminus D$ are measurable.
    \end{enumerate}
\end{namedthm*}
\begin{proof}
    Let $f$ be an extended real-valued function on the measurable set $E$.
    \begin{enumerate}[label=(\roman*),align=right]
        \item Let $f$ be measurable on $E$ and $f=g$ a.e. on $E$.\\
        Define $A=\{x\in E\ |\ f(x)\neq g(x)\}\subseteq E$, so that $f=g$ on $E\setminus A$, and $m(A)=0$.
        \begin{align*}
            \{x\in E\ |\ g(x)>c\}&=(\{x\in E\ |\ g(x)>c\}\cap[E\cap A])\cup(\{x\in E\ |\ g(x)>c\}\cap[E\cap A^c])\\
            &=\{x\in A\ |\ g(x)>c\}\cup(\{x\in E\ |\ f(x)>c\}\cap[E\setminus A]).
        \end{align*}
        Now, $\{x\in A\ |\ g(x)>c\}\subseteq A$, and because $m(A)=0$, $\{x\in A\ |\ g(x)>c\}$ is measurable and has measure zero.
        The set $\{x\in E\ |\ f(x)>c\}$ is measurable because $f$ is measurable, and $E\cap A^c$ is measurable because $E$ and $A$ (and thus $A^c$) are measurable.
        Thus $\{x\in E\ |\ g(x)>c\}$ is measurable because it is the finite union and intersection of measurable sets; therefore $g$ is measurable on $E$.
        \item Let $D$ be a measurable subset of $E$.\\
        $(\implies)$ Suppose $f$ is measurable on $E$.\\
        Then for any real number $c$, we see that 
        \begin{align*}
            \{x\in D\ |\ f|_{D}(x)>c\}&=\{x\in E\ |\ f(x)>c\}\cap[E\cap D],\\
            \{x\in E\setminus D\ |\ f|_{E\setminus D}(x)>c\}&=\{x\in E\ |\ f(x)>c\}\cap[E\setminus D],
        \end{align*}
        where both are measurable because they are each intersections of measurable sets. Therefore the restrictions $f|_{D}$ and $f|_{E\setminus D}$ are measurable.\\
        $(\impliedby)$ Suppose the restrictions of $f$ to $D$ and $E\setminus D$ are measurable.\\
        Then for any real number $c$,
        \begin{align*}
            \{x\in E\ |\ f(x)>c\}&=\{x\in D\ |\ f|_{D}(x)>c\}\cup\{x\in E\setminus D\ |\ f|_{E\setminus D}(x)>c\},
        \end{align*}
        which is measurable because it is a union of measurable sets.
    \end{enumerate}
\end{proof}

\end{flushleft}
\begin{center}
	\textbf{PROBLEMS}
\end{center}
\begin{enumerate}
	\setcounter{enumi}{0}
	\item Suppose $f$ and $g$ are continuous functions on $[a,b]$. Show that if $f=g$ a.e. on $[a,b]$, then, in fact, $f=g$ on $[a,b]$.
    Is a similar assertion true if $[a,b]$ is replaced by a general measurable set $E$?\\
    \\Let $f,g$ be continuous functions on $[a,b]$, where $f=g$ on $[a,b]\setminus E_0$, where $E_0$ is a subset of $[a,b]$ and $m(E_0)=0$.
    \\Suppose that $E_0$ is nonempty.\\
    Consider any point $x_0\in E_0\subseteq[a,b]$. For any $\epsilon>0$, there exists a $c\in(x_0-\epsilon,x_0+\epsilon)\cap[a,b]$ such that $f(c)=g(c)$, else we reach a contradiction because $m((x_0-\epsilon,x_0+\epsilon)\cap[a,b])\neq0$.
    This means that we can construct a sequence $\{c_i\}_{i=1}^\infty$ that converges to $x_0$, where $f(c_i)=g(c_i)$ is defined for all $i$. 
    However, because $\{c_i\}\to x_0$, by continuity of $f,g$, we have $\{f(c_i)\}\to f(x_0)$ and $\{g(c_i)\}\to g(x_0)$, and because $f(c_i)=g(c_i)$ for all $i$, the limit is unique; that is,
    \[
        |f(x_0)-g(x_0)|\le|f(x_0)-f(c_i)|+|f(c_i)-g(x_0)|<\epsilon,
    \]
    and $f(x_0)=g(x_0)$.
    \\However, this is a contradiction to $f(x)\neq g(x)$ for all $x\in E_0$, and so $E_0=\emptyset$.\\
    \\In the case that $[a,b]$ is replaced by a general measurable set $E$, the assertion is not true.
    Consider the case where $E=\{a\}$, so that $f,g$ are continuous on $\{a\}$, and $f=g$ a.e. on $\{a\}$.
    This only implies that $f=g$ on $E$ except for a set of measure zero. But $E$ is already of measure zero, so $f(a)\neq g(a)$ is possible, and $f\neq g$ on $E$.
    \item Let $D$ and $E$ be measurable sets and $f$ a function with domain $D\cup E$. We proved that $f$ is measurable on $D\cup E$ iff its restrictions to $D$ and $E$ are measurable.
    Is the same true if "measurable" is replaced by "continuous"?\\
    \\No; consider the function $f:[-1,1]\to\mathbb{R}$, where $[-1,1]=[-1,0)\cup[0,1]$, and we define
    \[
    f(x)=
    \begin{cases}
        0&x\in[-1,0),\\
        1&x\in[0,1].
    \end{cases}    
    \]
    Clearly we have a point of discontinuity at $x=0$, so $f$ is not continuous even though $f|_{[-1,0)}$ and $f|_{[0,1]}$ are continuous.
    \item Suppose a function $f$ has a measurable domain and is continuous except at a finite number of points.
    Is $f$ necessarily measurable?\\
    \\Yes; let $f$ be a function on the measurable domain $E$, and suppose $f$ is continuous on $E\setminus E_0$, where $E_0=\{x_i\}_{i=1}^n \subseteq E$. Then $m(E_0)=0$ because countable sets are measurable and have measure zero.
    \\Now, $f|_{E\setminus E_0}$ is continuous and therefore measurable (Proposition 3), and $f|_{E_0}$ is defined on a set of measure zero, so any subset $\{x\in E_0\ |\ f|_{E_0}(x)>c\}\subseteq E_0$ has measure zero and is thus measurable, and therefore $f|_{E_0}$ is a measurable function.
    \\Recall Proposition 5 to see that for the measurable subset $E_0$ of $E$, $f$ is measurable because $f|_{E_0}$ and $f|_{E\setminus E_0}$ are both measurable functions.
    \item Suppose $f$ is a real-valued function on $\mathbb{R}$ such that $f^{-1}(c)$ is measurable for each number $c$. Is $f$ necessarily measurable?\\
    \\No; let $V\subseteq[0,1]$ be a Vitali set. Therefore $V$ is nonmeasurable (see Ch 2.6).
    Consider the function $f:\mathbb{R}\to\mathbb{R}$, defined as
    \[
        f(x)=
        \begin{cases}
            -e^x&x\in V\\
            e^x&x\notin V\\
        \end{cases}    
    \]
    For any real number $c$, we have
    \[
        f^{-1}(c)=
        \begin{cases}
            \ln(-c)& c<0\\
            \ln(c)& c>0\\
            \emptyset& c=0
        \end{cases}
    \]
    and so $f^{-1}(c)$ is a singleton set or is the empty set, which are measurable, so $f^{-1}(c)$ is measurable.\\
    Now, we know that $e^x:\mathbb{R}\to\mathbb{R}_{>0}$ and so $e^x>0$ for any real number $x$.\\
    Therefore $f(x)=-e^x<0$ only when $x\in V$.
    However, the set $\{x\in\mathbb{R}\ |\ f(x)<0\}=V$ is not measurable, and so $f$ is not a measurable function.
    \item Suppose the function $f$ is defined on a measurable set $E$ and $\{x\in E\ |\ f(x)>c\}$ is a measurable set for each rational number $c$. Is $f$ necessarily a measurable function?\\
    \\Yes. Let $f:E\to\mathbb{R}$ with $E$ a measurable set, and let $\{x\in E\ |\ f(x)>c\}=\{x\in E\ |\ f(x)\in(c,\infty)\}$ be measurable for each $c\in\mathbb{Q}$.\\
    Let $a$ be any real number. Then for any natural number $n$, there exists a rational number $c_n$ such that $a<c_n<a+\frac{1}{n}$, and therefore $\bigcup_{n=1}^\infty(c_n,\infty)=(a,\infty)$.
    Therefore we have
    \begin{align*}
        \{x\in E\ |\ f(x)>a\} &= f^{-1}((a,\infty))\\
        &=f^{-1}(\bigcup_{n=1}^\infty(c_n,\infty))\\
        &=\bigcup_{n=1}^\infty f^{-1}((c_n,\infty))\\
        &=\bigcup_{n=1}^\infty \{x\in E\ |\ f(x)>c_n\},
    \end{align*}
    which is a countable union of measurable sets, and therefore is also measurable.
    \item Let $f$ be a function with measurable domain $D$. Show that $f$ is measurable iff the function $g$ defined on $\mathbb{R}$ by $g(x)=f(x)$ for $x\in D$ and $g(x)=0$ for $x\notin D$ is measurable.\\
    \\Let $D\subseteq\mathbb{R}$ be a measurable set, let $f:D\to\mathbb{R}$, and let $g:\mathbb{R}\to\mathbb{R}$ be defined by
    \[
        g(x)=
        \begin{cases}
            f(x)&x\in D\\
            0&x\notin D
        \end{cases}
    \]
    $(\implies)$ Suppose that $f$ is measurable.\\
    For any real number $c$,
    \[
        \{x\in \mathbb{R}\ |\ g(x)>c\}=
        \begin{cases}
            \{x\in D\ |\ f(x)>c\}&c\ge0\\
            \{x\in D\ |\ f(x)>c\}\cup D^c&c<0\\
        \end{cases}
    \]
    Both of the sets $\{x\in D\ |\ f(x)>c\}$ and $\{x\in D\ |\ f(x)>c\}\cap D^c$ are measurable, so $ \{x\in \mathbb{R}\ |\ g(x)>c\}$ is measurable and thus $g$ is a measurable function.\\
    \\$(\impliedby)$ Suppose that $g$ is measurable.\\
    Recall Proposition 5 (ii) to see that for the measurable subset $D$ of $\mathbb{R}$, $g$ is measurable on $\mathbb{R}$, which implies that the restrictions $g|_{D}$ and $g|_{\mathbb{R}\setminus D}$ are measurable.
    Therefore for any real number $c$,
    \begin{align*}
        \{x\in D\ |\ f(x)>c\}&=\{x\in \mathbb{R}\ |\ g|_{D}(x)>c\}\cap D\text{ is measurable},
    \end{align*}
    and $f$ is measurable.
    \item Let the function $f$ be defined on a measurable set $E$. Show that $f$ is measurable iff for each borel set $A$, $f^{-1}(A)$ is measurable. (Hint: the collection of sets $A$ that have the property that $f^{-1}(A)$ is measurable is a $\sigma$-algebra.)\\
    \\Let $f:E\to\mathbb{R}$, where $E$ is a measurable set.\\
    $(\implies$) Suppose that $f$ is measurable.\\
    Let $\mathcal{M}=\{A\ |\ f^{-1}(A)\text{ is measurable}\}$.\\
    To show that $\mathcal{M}$ is a $\sigma$-algebra, know that the measurable sets is a $\sigma$-algebra.\\
	Observe that:
	\begin{enumerate}[label=(\roman*),align=left]
		\item $f^{-1}(\emptyset)=\emptyset\implies\emptyset\in \mathcal{M}$.
		\item $A\in \mathcal{M}\implies f^{-1}(A)\text{ is measurable }\implies f^{-1}(A)^c=f^{-1}(A^c)\text{ is measurable }\implies A^c\in \mathcal{M}$.
		\item $A_k\in \mathcal{M}\implies f^{-1}(A_k)\text{ is measurable }\implies\bigcup_{k=1}^\infty f^{-1}(A_k)=f^{-1}(\bigcup_{k=1}^\infty A_k)\text{ is measurable }\implies \bigcup_{k=1}^\infty A_k\in \mathcal{M}$.
	\end{enumerate}
    Then because $f$ is measurable, for any real number $a$, the set $f^{-1}((a,\infty))=\{x\in E\ |\ f(x)>a\}$ is measurable.
    Now, $(a,\infty)\in\mathcal{M}$ because $f^{-1}((a,\infty))$ is measurable.
    Because $(a,\infty)$ is a Borel set, all other Borel sets are in $\mathcal{M}$ because the Borel sets are a $\sigma$-algebra.\\
    \\$(\impliedby$) Suppose that for each borel set $A$, the set $f^{-1}(A)=\{x\in E\ |\ f(x)\in A\}$ is measurable.\\
    Every interval of the form $(a,\infty)$ is a borel set, so we have that for any real number $a$, the set $f^{-1}((a,\infty))=\{x\in E\ |\ f(x)>a\}$ is measurable. This is equivalent to the measurability of $f$.
    \item (Borel measurability) A function $f$ is said to be \textbf{Borel measurable} provided its domain $E$ is a Borel set and for each $c$, the set $\{x\in E\ |\ f(x)>c\}$ is a Borel set.
    Verify that Proposition 1 and Theorem 6 remain valid if we replace "(Lebesgue) measurable set" by "Borel set".
    Show that:
    \begin{enumerate}[label=(\roman*),align=left]
        \item every Borel measurable function is Lebesgue measurable,\\
        \\The Borel sets are a subset of the measurable sets.
        Therefore for a Borel measurable function $f$, its domain $E$ is a Borel set (and thus a measurable set), and for each $c$, the set $\{x\in E\ |\ f(x)>c\}$ is a Borel set (and thus a measurable set).
        Thus $f$ is a measurable function.
        \item if $f$ is Borel measurable and $B$ is a Borel set, then $f^{-1}(B)$ is a Borel set,\\
        \\Let $S=\{B\ |\ f^{-1}(B)\text{ is Borel}\}$.\\
        To show that $S$ is a $\sigma$-algebra, know that the Borel sets is a $\sigma$-algebra.\\
        Observe that:
        \begin{enumerate}[label=(\roman*),align=left]
            \item $f^{-1}(\emptyset)=\emptyset\implies\emptyset\in S$.
            \item $B\in S\implies f^{-1}(B)\text{ is Borel }\implies f^{-1}(B)^c=f^{-1}(B^c)\text{ is Borel }\implies B^c\in S$.
            \item $B_k\in S\implies f^{-1}(B_k)\text{ is Borel }\implies\bigcup_{k=1}^\infty f^{-1}(B_k)=f^{-1}(\bigcup_{k=1}^\infty B_k)\text{ is Borel }\implies \bigcup_{k=1}^\infty B_k\in S$.
        \end{enumerate}
        Now, because $f$ is Borel measurable, for any real number $a$, the set $f^{-1}((a,\infty))$ is a Borel set.
        This implies $(a,\infty)\in S$.
        Because $(a,\infty)$ is a Borel set, all other Borel sets are in $S$ because the Borel sets is a $\sigma$-algebra.
        \item if $f$ and $g$ are Borel measurable, so is $f\circ g$,\\
        \\Let $f,g$ be Borel measurable, with $g:E\to F$, and $f:F\to\mathbb{R}$, where $E,F$ are Borel sets. 
        Then $f\circ g:E\to\mathbb{R}$ has a Borel set as its domain.\\
        Recall that $\{x\ |\ f(x)>a\}=\{x\ |\ f(x)\in(a,\infty)\}=f^{-1}((a,\infty))$.
        \begin{align*}
            \{x\ |\ (f\circ g )(x)>a\}&=\{x\ |\ (f\circ g )(x)\in(a,\infty)\}\\
            &=(f\circ g )^{-1}((a,\infty))\\
            &=(g^{-1}\circ f^{-1})((a,\infty))\\
            &=g^{-1}(f^{-1}((a,\infty)))\\
            &=g^{-1}(B)&&\text{ where }B\text{ is Borel}\\
            &=B'.&&\text{ where }B'\text{ is Borel}
        \end{align*}
        Therefore $f\circ g$ is Borel measurable.
        \item if $f$ is Borel measurable and $g$ is Lebesgue measurable, then $f\circ g$ is Lebesgue measurable.\\
        \\Let $f$ be Borel measurable, with $f:F\to\mathbb{R}$, and let $g$ be Lebesgue measurable, with $g:E\to F$, where $F$ is a Borel set and $E$ is a measurable set. 
        Then $f\circ g:E\to\mathbb{R}$ has a measurable set as its domain.
        \begin{align*}
            \{x\ |\ (f\circ g )(x)>a\}&=\{x\ |\ (f\circ g )(x)\in(a,\infty)\}\\
            &=(f\circ g )^{-1}((a,\infty))\\
            &=(g^{-1}\circ f^{-1})((a,\infty))\\
            &=g^{-1}(f^{-1}((a,\infty)))\\
            &=g^{-1}(B)&&\text{ where }B\text{ is Borel}\\
            &=B'.&&\text{ where }B'\text{ is measurable (Problem 7)}
        \end{align*}
        Therefore $f\circ g$ is Lebesgue measurable.
    \end{enumerate}
    \item Let $\{f_n\}$ be a sequence of measurable functions defined on a measurable set $E$.
    Define $E_0$ to be the set of points of $x$ in $E$ at which $\{f_n(x)\}$ converges. Is the set $E_0$ measurable?\\
    \\Let $E_0=\{x\in E\ |\ \{f_n(x)\}\text{ converges}\}\subseteq \{x\in E\ |\ \{f_n(x)\}\text{ is Cauchy}\}$, because all convergent sequences are Cauchy.
    \\Therefore $E_0=\{x\in E\ |\ \forall k\in\mathbb{N},\exists N\in\mathbb{N}\text{ such that }|f_n(x)-f_m(x)|<\frac{1}{k}\text{ for all }n,m\ge N\}$.
    \\This is equivalent to writing 
    \[
        E_0=\bigcap_{k\in\mathbb{N}}\bigcup_{N\in\mathbb{N}}\bigcap_{n,m\ge N}\{x\in E\ |\ |f_n(x)-f_m(x)|<\frac{1}{k}\}.
    \]
    The functions $f_n$ and $f_m$ are measurable, so by Theorem 6, $f_n-f_m$ is also measurable.
    The absolute value function $|\cdot|$ is continuous, so by Proposition 7, the composition $|\cdot|\circ(f_n-f_m)=|f_n-f_m|$ is a measurable function.
    Therefore for the real number $\frac{1}{k}$, the set $\{x\in E\ |\ |f_n(x)-f_m(x)|<\frac{1}{k}\}$ is a measurable set.\\
    Then $E_0$ is a countable union and intersection of measurable sets, so $E_0$ is measurable.
    \item Suppose $f$ and $g$ are real-valued functions defined on all of $\mathbb{R}$, $f$ is measurable, and $g$ is continuous.
    Is the composition $f\circ g$ necessarily measurable?\\
    \\No; let $f$ be measurable, with $f:\mathbb{R}\to\mathbb{R}$, and let $g$ be continuous, with $g:\mathbb{R}\to \mathbb{R}$.
        Then $f\circ g:\mathbb{R}\to\mathbb{R}$.
        \begin{align*}
            \{x\ |\ (f\circ g )(x)>a\}&=\{x\ |\ (f\circ g )(x)\in(a,\infty)\}\\
            &=(f\circ g )^{-1}((a,\infty))\\
            &=(g^{-1}\circ f^{-1})((a,\infty))\\
            &=g^{-1}(f^{-1}((a,\infty)))\\
            &=g^{-1}(A)&&\text{ where }A\text{ is measurable}
        \end{align*}
        Recall Chapter 2 Problem 36 to see that for a continuous function $g$, the set $g^{-1}(A)$ is not always measurable when $A$ is measurable.
    \item Let $f$ be a measurable function and $g$ be a one-to-one function from $\mathbb{R}$ onto $\mathbb{R}$ which has a Lipschitz inverse. Show that the composition $f\circ g$ is measurable. (Hint: examine Problem 37 in Chapter 2.)\\
    \\Let $f$ be measurable, with $f:\mathbb{R}\to\mathbb{R}$, and let $g$ be a bijection from $\mathbb{R}$ to $\mathbb{R}$, where $g^{-1}$ is Lipschitz.
    From Chapter 2 Problem 37, we have that $g^{-1}$ maps a measurable set to a measurable set; that is, for the measurable set $A$, the set $g^{-1}(A)$ is measurable.
    \\We have $f\circ g:\mathbb{R}\to\mathbb{R}$.
    \begin{align*}
        \{x\ |\ (f\circ g )(x)>a\}&=\{x\ |\ (f\circ g )(x)\in(a,\infty)\}\\
        &=(f\circ g )^{-1}((a,\infty))\\
        &=(g^{-1}\circ f^{-1})((a,\infty))\\
        &=g^{-1}(f^{-1}((a,\infty)))\\
        &=g^{-1}(A)&&\text{ where }A\text{ is measurable}\\
        &=A'.&&\text{ where }A'\text{ is measurable (Chapter 2 Problem 37)}
    \end{align*}
    Therefore $f\circ g$ is measurable.
\end{enumerate}

% 3.2
\section{Sequential Pointwise Limits and Simple Approximation}
\begin{center}
	\textbf{PROBLEMS}
\end{center}
\begin{enumerate}
	\setcounter{enumi}{11}
    \item Let $f$ be a bounded measurable function on $E$. Show that there are sequences of simple functions on $E$, $\{\varphi_n\}$ and $\{\psi_n\}$, such that $\{\varphi_n\}$ is increasing and $\{\psi_n\}$ is decreasing and each of these sequences converges to $f$ uniformly on $E$.
    \item A real-valued measurable function is said to be \textit{semisimple} provided it takes only a countable number of values. Let $f$ be any measurable function on $E$.
    Show that there is a sequence of semisimple functions $\{f_n\}$ on $E$ that converges to $f$ uniformly on $E$.
    \item Let $f$ be a measurable function on $E$ that is finite a.e. on $E$ and $m(E)<\infty$.
    For each $\epsilon>0$, show that there is a measurable set $F$ contained in $E$ such that $f$ is bounded on $F$ and $m(E\setminus F)<\epsilon$.
    \item Let $f$ be a measurable function on $E$ that if finite a.e. on $E$ and $m(E)<\infty$. Show that for each $\epsilon>0$, there is a measurable set $F$ contained in $E$ and a sequence $\{\varphi_n\}$ of simple functions on $E$ such that $\{\varphi_n\}\to f$ uniformly on $F$ and $m(E\setminus F)<\epsilon$. (Hint: see the preceding problem.)
    \item Let $I$ be a closed, bounded interval and $E$ a measurable subset of $I$. Let $\epsilon>0$.
    Show that there is a step function $h$ on $I$ and a measurable subset $F$ of $I$ for which 
    \[
        h=\chi_E\text{ on }F\text{ and }m(I\setminus F)<\epsilon.    
    \]
    (Hint: use Theorem 12 of Chapter 2.)
    \item Let $I$ be a closed, bounded interval and $\psi$ a simple function defined on $I$. Let $\epsilon>0$.
    Show that there is a step function $h$ on $I$ and a measurable subset $F$ of $I$ for which 
    \[
        h=\psi\text{ on }F\text{ and }m(I\setminus F)<\epsilon.    
    \] 
    (Hint: use the fact that a simple function is a linear combination of characteristic functions and the preceding problem.)
    \item Let $I$ be a closed, bounded interval and $f$ a bounded measurable function defined on $I$. Let $\epsilon>0$.
    Show that there is a step function $h$ on $I$ and a measurable subset $F$ of $I$ for which 
    \[
        |h-f|<\epsilon\text{ on }F\text{ and }m(I\setminus F)<\epsilon.    
    \]
    \item Show that the sum and product of two simple functions are simple as are the max and the min.
    \item Let $A,B$ be any sets. Show that
    \begin{align*}
        \chi_{A\cap B}&=\chi_A\cdot\chi_B\\
        \chi_{A\cup B}&=\chi_A+\chi_B-\chi_A\cdot\chi_B\\
        \chi_{A^c}&=1-\chi_A\\
    \end{align*}
    \item For a sequence $\{f_n\}$ of measurable functions with common domain $E$, show that each of the following functions is measurable:
    \begin{itemize}
        \item $\inf\{f_n\}$
        \item $\sup\{f_n\}$
        \item $\lim\inf\{f_n\}$
        \item $\lim\sup\{f_n\}$
    \end{itemize}
    \item (Dini's Theorem) Let $\{f_n\}$ be an increasing sequence of continuous functions on $[a,b]$ which converges pointwise on $[a,b]$ to the continuous function $f$ on $[a,b]$.
    Show that the convergence is uniform on $[a,b]$. (Hint: let $\epsilon>0$. For each natural number $n$, define $E_n=\{x\in[a,b]\ |\ f(x)-f_n(x)<\epsilon\}$. Show that $\{E_n\}$ is an open cover of $[a,b]$ and use the Heine-Borel Theorem.)
    \item Express a measurable function as the difference of nonnegative measurable functions and thereby prove the general Simple Approximation Theorem based on the special case of a nonnegative measurable function.
    \item Let $I$ be an interval and $f:I\to\mathbb{R}$ be increasing. Show that $f$ is measurable by first showing that, for each natural number $n$, the strictly increasing function $x\mapsto f(x)+x/n$ is measurable, and then taking pointwise limits.
\end{enumerate}

% 3.3
\section{Littlewood's Three Principles, Ergoff's Theorem, and Lusin's Theorem`'}
\begin{center}
	\textbf{PROBLEMS}
\end{center}
\begin{enumerate}
	\setcounter{enumi}{24}
    \item Suppose $f$ is a function that is continuous on a closed set $F$ of real numbers. Show that $f$ has a continuous extension to all of $\mathbb{R}$. This is a special case of the forthcoming Tietze Extension Theorem.
    (Hint: express $\mathbb{R}\setminus F$ as the union of a countable disjoint collection of open intervals and define $f$ to be linear on the closure of each of these intervals.)
    \item For the function $f$ and the set $F$ in the statement of Lusin's Theorem, show that the restriction of $f$ to $F$ is a continuous function.
    Must there be any points at which $f$, considered as a function of $E$, is continuous?
    \item Show that the conclusion of Egoroff's Theorem can fail if we drop the assumption that the domain has finite measure.
    \item Show that Egoroff's Theorem continues to hold if the convergence is pointwise a.e. and $f$ is finite a.e.
    \item Prove the extension of Lusin's Theorem to the case that $E$ has finite measure.
    \item Prove the extension of Lusin's Theorem to the case that $f$ is not necessarily real-valued, but is finite a.e.
    \item Let $\{f_n\}$ be a sequence of measurable functions on $E$ that converges to the real-valued $f$ pointwise on $E$. Show that $E=\bigcup_{k=1}^\infty E_k$, where for each index $k$, $E_k$ is measurable, and $\{f_n\}$ converges uniformly to $f$ on each $E_k$ if $k>1$, and $m(E_1)=0$.  
\end{enumerate}