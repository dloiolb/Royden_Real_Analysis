% Chapter 9
\authoredby{inprogress}
\chapter{Metric Spaces: General Properties}

% 9.1
\authoredby{inprogress}
\section{Examples of Metric Spaces}
\begin{flushleft}

"The object of the present chapter is to study general spaces called metric spaces for which the notion of distance between two points is fundamental."

\begin{namedthm*}{Definition}
    Let $X$ be a nonempty set. A function $\rho:X\times X\to\mathbb{R}$ is called a \textbf{metric} provided for all $x,y,z\in X$,
    \begin{enumerate}[label=(\roman*),align=right]
        \item $\rho(x,y)\ge0$,
        \item $\rho(x,y)=0$ iff $x=y$,
        \item $\rho(x,y)=\rho(y,x)$,
        \item $\rho(x,y)\le\rho(x,z)+\rho(z,y)$.
    \end{enumerate}
    A nonempty set together with a metric on the set is called a \textbf{metric space}, often denoted by $(X,\rho)$.
\end{namedthm*}

An example of a metric space is the set $\mathbb{R}$ of all real numbers with the metric $\rho(x,y) = |x-y|$. \par\medskip

A linear space with a norm is called a normed linear space.
A norm $\|\cdot\|$ on a linear space $X$ induces a metric $\rho$ on $X$ by defining
\[
    \rho(x,y)=\|x-y\|\text{ for all }x,y\in X.  
\]
To show this, let $x,y\in X$. Because $X$ is a linear space, $x-y \in X$, and $\|x-y\|$ is defined.
\begin{enumerate}[label=(\roman*),align=right]
    \item $\|x-y\|\ge0$ by positive definiteness of norm
    \item $\|x-y\|=0$ iff $x-y=0 \implies x=y$ by positive definiteness of norm
    \item $\|x-y\|=\|-1(y-x)\|=|-1|\|y-x\|=\|y-x\|$ by absolute homogeneity of norm
    \item $\|x-y\|= \|x-z+z-y\|\le\|x-z\|+\|z-y\|$ by subadditivity of norm
\end{enumerate}

Three prominent examples of normed linear spaces:
the Euclidean spaces $\mathbb{R}^n$, 
the $L^p(E)$ spaces,
$C[a,b]$.\\

For a natural number $n$, consider the linear space $\mathbb{R}^n$, whose points are $n$-tuples of real numbers.
For $x=(x_1,\cdots,c_n)\in\mathbb{R}^n$, the Euclidean norm $\|x\|$ is defined by
\[
    \|x\|=[x_1^2+\cdots+x_n^2]^{1/2}.    
\]
\par\medskip

\textit{The Discrete Metric}
For any nonempty set $X$, the discrete metric $\rho$ is defined by setting $\rho(x,y)=0$ if $x=y$ and $\rho(x,y)=1$ if $x\neq y$.
\begin{enumerate}[label=(\roman*),align=right]
    \item $\rho(x,y)\in\{0,1\}\implies\rho(x,y)\ge0$.
    \item $\rho(x,y)=0\iff x=y$ by definition.
    \item By symmetry of the equality relation,
    \[
    \rho(x,y)=
    \begin{cases}
        1 &x=y\\
        0 &x\neq y
    \end{cases}    
    =
    \begin{cases}
        1 &y=x\\
        0 &y\neq x
    \end{cases}  
    = \rho(y,x)
    \]
    \item In the case $x\neq y$, the triangle inequality is trivial.\\
    In the case $x=y$,\\
    if $x=z$, then by transitivity, $z=y$ and\\$\rho(x,y)=0,\rho(x,z)=0,\rho(z,y)=0$, and the triangle inequality is obvious: $0\le0$.\\
    if $x\neq z$, then by transitivity, $z\neq y$ and\\$\rho(x,y)=0,\rho(x,z)=1,\rho(z,y)=1 \implies \rho(x,y)=0\le2=\rho(x,z)+\rho(z,y)$.
\end{enumerate}
\par\medskip

\textit{Metric Subspaces}
For a metric space $(X,\rho)$, let $Y$ be a nonempty subset of $X$. 
Then the restriction of $\rho$ to $Y \times Y$ defines a metric on $Y$ and we call such a metric space a metric \textbf{subspace}.
Therefore every nonempty subset of Euclidean space, of and $L^p(E)$ space, $1\le p\le\infty$, and of $C[a,b]$ is a metric space.
\par\medskip

\textit{Metric Products}
For metric spaces $(X_1,\rho_1)$ and $(X_2,\rho_2)$, we define the \textbf{product metric} $\tau$ on the Cartesian product $X_1 \times X_2$ by setting, for $(x_1,x_2),(y_1,y_2)\in X_1 \times X_2$,
\[
    \tau((x_1,x_2),(y_1,y_2))= \{[\rho(x_1,y_1)]^2+[\rho(x_2,y_2)]^2\}^{1/2}.
\]
To see that this is a metric:
\begin{enumerate}[label=(\roman*),align=right]
    \item By metric property (i) of $\rho_1,\rho_2$, we have
    \begin{align*}
        \lbrack\rho_1(x_1,y_1)\rbrack^2\ge0\text{ and }\lbrack\rho_2(x_2,y_2)\rbrack^2\ge0
        &\iff \{\lbrack\rho_1(x_1,y_1)\rbrack^2+\lbrack\rho_2(x_2,y_2)\rbrack^2\}^{1/2}\ge0\\
        &\iff \tau((x_1,x_2),(y_1,y_2))\ge0
    \end{align*}
    \item By metric property (ii) of $\rho_1,\rho_2$, we have
    \begin{align*} 
        \tau((x_1,x_2),(y_1,y_2))&=\{\lbrack\rho_1 (x_1,y_1)\rbrack^2+\lbrack\rho_2 (x_2,y_2)\rbrack^2\}^{1/2}=0\\
        &\iff \lbrack\rho_1 (x_1,y_1)\rbrack^2=0\text{ and }\lbrack\rho_2 (x_2,y_2)\rbrack^2=0\\
        &\iff x_1=y_1\text{ and }x_2=y_2\\
        &\iff (x_1,x_2)=(y_1,y_2).
    \end{align*}
    \item By metric property (iii) (symmetry) of $\rho_1,\rho_2$, we have
    \begin{align*}
        \tau((x_1,x_2),(y_1,y_2)) &=\{\lbrack\rho_1 (x_1,y_1)\rbrack^2+\lbrack\rho_2 (x_2,y_2)\rbrack^2\}^{1/2}\\
        &=\{\lbrack\rho_1 (y_1,x_1)\rbrack^2+\lbrack\rho_2 (y_2,x_2)\rbrack^2\}^{1/2}\\
        &= \tau((y_1,y_2),(x_1,x_2)).
    \end{align*}
    \item We must first prove an inequality $\star$. Because $x^2\ge0$ for any real number $x$, we have
    \begin{align*}
        0&\le\lbrack\rho_1(x_1,z_1)\rho_2(z_2,y_2)-\rho_2(x_2,z_2)\rho_1(z_1,y_1)\rbrack^2\\
        0&\le\rho_1(x_1,z_1)^2\rho_2(z_2,y_2)^2+\rho_2(x_2,z_2)^2\rho_1(z_1,y_1)^2\\&\phantom{\le+{}}-2\rho_1(x_1,z_1)\rho_1(z_1,y_1)\rho_2(x_2,z_2)\rho_2(z_2,y_2)\\
        2\rho_1(x_1,z_1)\rho_1(z_1,y_1)\rho_2(x_2,z_2)\rho_2(z_2,y_2)&\le\rho_1(x_1,z_1)^2\rho_2(z_2,y_2)^2+\rho_2(x_2,z_2)^2\rho_1(z_1,y_1)^2\\
    \end{align*}
    Adding $\rho_1(x_1,z_1)^2\rho_1(z_1,y_1)^2 + \rho_2(x_2,z_2)^2\rho_2(z_2,y_2)^2$ to both sides, we have
    \begin{align*}
        \rho_1(x_1,z_1)^2\rho_1(z_1,y_1)^2 + \rho_2(x_2,z_2)^2\rho_2(z_2,y_2)^2+ 2\rho_1(x_1,z_1)\rho_1(z_1,y_1)\rho_2(x_2,z_2)\rho_2(z_2,y_2)\\
        \le\rho_1(x_1,z_1)^2\rho_1(z_1,y_1)^2 + \rho_2(x_2,z_2)^2\rho_2(z_2,y_2)^2+\rho_1(x_1,z_1)^2\rho_2(z_2,y_2)^2+\rho_2(x_2,z_2)^2\rho_1(z_1,y_1)^2\\
    \end{align*}
    Therefore we end up with the inequality: $\star$
    \begin{align*}
        \lbrack\rho_1(x_1,z_1)\rho_1(z_1,y_1) + \rho_2(x_2,z_2)\rho_2(z_2,y_2)\rbrack^2&\le\lbrack\rho_1(x_1,z_1)^2+\rho_2(x_2,z_2)^2\rbrack\lbrack\rho_1(z_1,y_1)^2+\rho_2(z_2,y_2)^2\rbrack\\
        \rho_1(x_1,z_1)\rho_1(z_1,y_1) + \rho_2(x_2,z_2)\rho_2(z_2,y_2)&\le\sqrt{\lbrack\rho_1(x_1,z_1)^2+\rho_2(x_2,z_2)^2\rbrack\lbrack\rho_1(z_1,y_1)^2+\rho_2(z_2,y_2)^2\rbrack}\\
        2\rho_1(x_1,z_1)\rho_1(z_1,y_1) + 2\rho_2(x_2,z_2)\rho_2(z_2,y_2)&\le2\sqrt{\rho_1(x_1,z_1)^2+\rho_2(x_2,z_2)^2}\sqrt{\rho_1(z_1,y_1)^2+\rho_2(z_2,y_2)^2}&&\star\\
    \end{align*}
    Now, to prove the triangle inequality of the product metric, we use the metric property (iv) (triangle inequality) of $\rho_1,\rho_2$:
    \begin{align*}
        &\rho(x_1,y_1)^2+\rho(x_2,y_2)^2\\
        &\le\lbrack\rho(x_1,z_1)+\rho(z_1,y_1)\rbrack^2+\lbrack\rho(x_2,z_2)+\rho(z_2,y_2)\rbrack^2\\
        &=\rho(x_1,z_1)^2+\rho(z_1,y_1)^2+2\rho(x_1,z_1)\rho(z_1,y_1)+\rho(x_2,z_2)^2+\rho(z_2,y_2)^2+2\rho(x_2,z_2)\rho(z_2,y_2)\\
        &\le\rho(x_1,z_1)^2+\rho(z_1,y_1)^2+\rho(x_2,z_2)^2+\rho(z_2,y_2)^2+2\sqrt{\rho_1(x_1,z_1)^2+\rho_2(x_2,z_2)^2}\sqrt{\rho_1(z_1,y_1)^2+\rho_2(z_2,y_2)^2}&&\star\\
        &=\biggl\lbrack\sqrt{\rho_1(x_1,z_1)^2+\rho_2(x_2,z_2)^2}+\sqrt{\rho_1(z_1,y_1)^2+\rho_2(z_2,y_2)^2}\biggr\rbrack^2
    \end{align*}
    Therefore we have
    \[
        \sqrt{\rho(x_1,y_1)^2+\rho(x_2,y_2)^2}\le\sqrt{\rho_1(x_1,z_1)^2+\rho_2(x_2,z_2)^2}+\sqrt{\rho_1(z_1,y_1)^2+\rho_2(z_2,y_2)^2}
    \]
    and thus
    \[
        \tau((x_1,x_2),(y_1,y_2)) \le \tau((x_1,x_2),(z_1,z_2))+\tau((z_1,z_2),(y_1,y_2)).
    \]
\end{enumerate}
This construction extends to countable products (problem 10).
\par\medskip

\begin{namedthm*}{Definition}
    Two metrics $\rho$ and $\sigma$ on a set $X$ are said to be \textbf{equivalent} provided there are positive numbers $c_1,c_2$ such that for all $x_1,x_2\in X$,
    \[
    c_1\cdot \sigma(x_1,x_2)\le\rho(x_1,x_2)\le c_2\cdot\sigma(x_1,x_2).    
    \]    
\end{namedthm*}

\begin{namedthm*}{Definition}
    A mapping $f$ from a metric space $(X,\rho)$ to a metric space $(Y,\sigma)$ is said to be an \textbf{isometry} provided it maps $X$ onto $Y$ and for all $x_1,x_2\in X$,
    \[
    \sigma(f(x_1),f(x_2))=\rho(x_1,x_2).    
    \]
\end{namedthm*}

Two metric spaces are called \textbf{isometric} provided there is an isometry from one onto the other.
To be isometric is an equivalence relation among metric spaces.
From the viewpoint of metric spaces, two isometric metric spaces are exactly the same, an isometry amounting merely to a relabeling of the points.
\par\medskip
In the definition of a metric $\rho$ on a set $X$ it is often convenient to relax the condition that $\rho(x,y)=0$ if and only if $x=y$.
When we allow the possibility that $\rho(x,y)=0$ for some $x\neq y$, we call $\rho$ a \textbf{pseudometric} and $(X,\rho)$ a pseudometric space.
On such a space, define the relation $x\cong y$ provided $\rho(x,y)=0$.
This is an equivalence relation that separates $X$ into a disjoint collection of equivalence classes $X/\cong$.  

\end{flushleft}
\begin{center}
	\textbf{PROBLEMS}
\end{center}
\begin{enumerate}
	\setcounter{enumi}{0}
	\item Show that two metrics $\rho$ and $\tau$ on the same set $X$ are equivalent iff there is a $c>0$ such that for all $u,v\in X$,
	\[
        \frac{1}{c}\tau(u,v)\le\rho(u,v)\le c\tau(u,v).    
    \]
    Let $\rho$ and $\tau$ be two metrics on the same set $X$.\\
    $(\implies)$ Suppose $\rho$ and $\tau$ are equivalent.\\
    Then there exist $c_1,c_2>0$ such that for all $u,v\in X$,
    \[
    c_1\cdot \tau(u,v)\le\rho(u,v)\le c_2\cdot\tau(u,v).    
    \]
    By the Archimedean Property of $\mathbb{R}$, for the positive real number $c_1$, there exists a natural number $n$ for which $\frac{1}{n}<c_1$.\\
    Let $c=\max\{n,c_2\}$ so that $n\le c \implies \frac{1}{c}\le\frac{1}{n}<c_1$ and also $c_2\le c$, so we have
    \[
        \frac{1}{c}\cdot \tau(u,v)<c_1\cdot \tau(u,v)\le\rho(u,v)\le c_2\cdot\tau(u,v) \le c\tau(u,v).
    \]
    \\
    $(\impliedby)$ Suppose that there is a $c>0$ such that for all $u,v\in X$,
	\[
        \frac{1}{c}\tau(u,v)\le\rho(u,v)\le c\tau(u,v).    
    \]
    We showed that for a positive number $c$, its multiplicative inverse $\frac{1}{c}$ is also positive [ch1, 2(ii)], and so we have $\frac{1}{c},c>0$ such that for all $u,v\in X$,
    \[
        (\frac{1}{c})\tau(u,v)\le\rho(u,v)\le c\tau(u,v).    
    \]
    Therefore $\rho$ and $\tau$ are equivalent.
    \item Show that the following define equivalent metrics on $\mathbb{R}^n$:
    \begin{align*}
        \rho^*(x,y) &= |x_1-y_1| + \cdots + |x_n-y_n|;\\
        \rho^+(x,y) &= \max\{|x_1-y_1|, \cdots,|x_n-y_n|\}.
    \end{align*}
    For some $j\in\{1,\cdots,n\}$, we have $|x_j-y_j|=\max\{|x_1-y_1|, \cdots,|x_n-y_n|\}$.\\
    Then 
    \begin{align*}
        \rho^+(x,y) &= |x_j-y_j|\\
        &= 0+\cdots +|x_j-y_j|+\cdots+0\\
        &\le |x_1-y_1| + \cdots|x_j-y_j|+\cdots + |x_n-y_n|\\
        &= \rho^*(x,y).
    \end{align*}
    Also,
    \begin{align*}
        \rho^*(x,y) &= |x_1-y_1| + \cdots|x_j-y_j|+\cdots + |x_n-y_n|\\
        &\le |x_j-y_j| + \cdots|x_j-y_j|+\cdots + |x_j-y_j|\\
        &= n|x_j-y_j|\\
        &=n\rho^+(x,y).
    \end{align*}
    Therefore we have $1,n>0$ such that for all $x,y\in \mathbb{R}^n$,
    \[
        \rho^+(x,y)\le\rho^*(x,y)\le n\rho^+(x,y).
    \]
    \item Find a metric on $\mathbb{R}^n$ that fails to be equivalent to either of those defined in the preceding problem.\\
    Let 
    \[
    \rho(x,y)=
    \begin{cases}
        \min\{|x-y|,1\}&\text{if $x,y$ are both rational or $x,y$ are both irrational}\\
        1&\text{else}
    \end{cases}    
    \]
    \item For a closed, bounded interval $[a,b]$, consider the set $X=C[a,b]$ of continuous real-valued functions on $[a,b]$.
    Show that the metric induced by the maximum norm and that induced by the $L^1[a,b]$ norm are not equivalent.\\
    From Chapter 7 Problem 1, we proved that there is no number $c \ge 0$ for which
	\[
	\| f \|_{\max}	\le c \| f \|_1 \text{ for all $f$ in $C[a,b]$}.
	\]
    Therefore there exists no $c_1,c_2>0$ such that for all $f,g$ in $C[a,b]$,
    \[
        c_1\| f-g \|_1 \le \| f-g \|_{\max}	\le c_2 \| f-g \|_1,
	\]
    and the metrics induced by the norms $\| \cdot \|_{\max}$ and $\| \cdot \|_1$ are not equivalent.
    \item \textit{The Nikodym Metric}. Let $E$ be a Lebesgue measurable set of real numbers of finite measure, $X$ the set of measurable subsets of $E$, and $m$ Lebesgue measure.
    For $A,B\in X$, define $\rho(A,B)=m(A\Delta B)$, where $A\Delta B = [A\setminus B]\cup[B\setminus A]$, the symmetric difference of $A$ and $B$.
    Show that this is a pseudometric on $X$.
    Define two measurable sets to be equivalent provided their symmetric difference has measure zero.
    Show that $\rho$ induces a metric on the collection of equivalence classes.
    Finally, show that for $A,B\in X$, 
    \[
    \rho(A,B)=\int_E|\chi_A-\chi_B|,    
    \]
    where $\chi_A$ and $\chi_B$ are the characteristic functions of $A$ and $B$, respectively.
    \item Show that for $a,b,c\ge0$,
    \[
    \text{if }a\le b+c,\text{ then }\frac{a}{1+a}\le\frac{b}{1+b}+\frac{c}{1+c}.
    \]
    Because $a\le b+c$, we see that $\frac{1}{1+b+c}\le\frac{1}{1+a}$, and $-\frac{1}{1+a}\le-\frac{1}{1+b+c}$ so
    \begin{align*}
        \frac{a}{1+a}&=1-\frac{1}{1+a}\le1-\frac{1}{1+b+c}=\frac{b+c}{1+b+c}=\frac{b}{1+b+c}+\frac{c}{1+b+c}\le\frac{b}{1+b}+\frac{c}{1+c}.
    \end{align*}
    \item Let $E$ be a Lebesgue measurable set of real numbers that has finite measure and $X$ the set of Lebesgue measurable real-valued functions on $E$.
    For $f,g\in X$, define
    \[
    \rho(f,g)=\int_E\frac{|f-g|}{1+|f-g|}.    
    \]
    Use the preceding problem to show that this is a pseudometric on $X$. 
    Define two measurable functions to be equivalent provided they are equal a.e. on $E$.
    Show that $\rho$ induces a metric on the collection of equivalence classes.\\
    \\Let $f,g,h\in X$. 
    \\Then for all $x\in E$, we have the triangle inequality $|f(x)-g(x)|\le|f(x)-h(x)|+|h(x)-g(x)|$.
    By Problem 6 and monotonicity and linearity of integration, we have
    \[
        \int_E\frac{|f-g|}{1+|f-g|}\le\int_E\frac{|f-h|}{1+|f-h|}+\int_E\frac{|h-g|}{1+|h-g|},
    \]
    Therefore $\rho(f,g)\le\rho(f,h)+\rho(h,g)$ and (iv) holds.\\
    Clearly $\rho(f,g)\ge0$ is true so (i) holds.\\
    We also have $\rho(f,g)=\int_E\frac{|f-g|}{1+|f-g|}=\int_E\frac{|g-h|}{1+|g-h|}=\rho(g,f)$ so (iii) holds.\\
    However, (ii) does not hold. We can consider two functions $f,g\in X$ such that $f=g$ on $E\setminus E_0$, where $m(E_0)=0$.
    Then $ \rho(f,g)=\int_E\frac{|f-g|}{1+|f-g|}= \int_{E\setminus E_0}\frac{|f-f|}{1+|f-f|}=0$ but $f\neq g$.\\
    If we consider defining the equivalence 
    \[
        f\cong g\text{ when }f=g \text{ on }E\setminus E_0\text{ where }m(E_0)=0,
    \]
    Then $\rho$ now induces a metric on the collection of equivalence classes $X/\cong$, because we now have $ \rho(f,g)=0\iff f\cong g$.
    \item For $0<p<1$, show that 
    \[
    (a+b)^p\le a^p+b^p\text{ for all }a,b\ge0.    
    \]
    
    \item For $E$ a Lebesgue measurable set of real numbers, $0<p<1$, and $g,h$ Lebesgue measurable functions on $E$ that have integrable $p^{th}$ powers, define
    \[
    \rho_p(h,g)=\int_E|g-h|^p.    
    \]  
    Use the preceding problem to show that this is a pseudometric on the collection of Lebesgue measurable functions on $E$ that have integrable $p^{th}$ powers.
    Define two such functions to be equivalent provided they are equal a.e. on $E$. 
    Show that $\rho_p(\cdot,\cdot)$ induces a metric on the collection of equivalence classes.
    \\Let $f,g,h:E\to\mathbb{R}$ such that $f,g,h$ are Lebesgue measurable and have integrable $p^{th}$ powers. 
    \\For all $x\in E$, Problem 8 tells us that $|g(x)-h(x)|^p\le|g(x)-f(x)|^p+|f(x)-h(x)|^p$.
    By monotonicity and linearity of integration, we have
    \[
        \int_E|g-h|^p\le\int_E|g-f|^p+\int_E|f-h|^p.
    \]
    Therefore $\rho_p(h,g)\le\rho_p(f,g)+\rho_p(h,f)$ and (iv) holds.\\
    Clearly $\rho_p(h,g)\ge0$ is true so (i) holds.\\
    We also have $\rho_p(h,g)=\int_E|g-h|^p=\int_E|h-g|^p=\rho_p(g,h)$ so (iii) holds.\\
    However, (ii) does not hold. We can consider two functions $h,g$ that are equivalent a.e. on $E$ but not equal.
    Then $ \rho_p(h,g)=\int_E|g-h|^p= \int_{E\setminus E_0}|g-h|^p=0$ with $m(E_0)=0$ but $h\neq g$.\\
    If we define an equivalence when two functions are equal a.e. on $E$, then $\rho_p$ is a metric on the collection of such equivalence classes. 
    \item Let $\{(X_n,\rho_n)\}_{n=1}^\infty$ be a countable collection of metric spaces.
    Use problem 6 to show that $\rho_*$ defines a metric on the Cartesian product $\prod_{n=1}^\infty X_n$, where for points $x = \{x_n\}$ and $y= \{y_n\}$ in $\prod_{n=1}^\infty X_n$,
    \[
    \rho_*(x,y)=\sum_{n=1}^\infty\frac{1}{2^n}\cdot\frac{\rho_n(x_n,y_n)}{1+\rho_n(x_n,y_n)}.    
    \]
    \item Let $(X,\rho)$ be a metric space and $A$ any set for which there is a one-to-one (injective) mapping $f$ of $A$ onto (surjective?) the set $X$ (bijection?).
    Show that there is a unique metric on $A$ for which $f$ is an isometry of metric spaces.
    (This is the sense in which an isometry amounts merely to a relabeling of the points in a space.)
    \item Show that the triangle inequality for Euclidean space $\mathbb{R}^n$ follows from the triangle inequality for $L^2[0,1]$.
\end{enumerate}

% 9.2
\authoredby{untouched}
\section{Open Sets, Closed Sets, and Convergent Sequences}

% 9.3
\section{Continuous Mappings Between Metric Spaces}

% 9.4
\section{Complete Metric Spaces}

% 9.5
\section{Compact Metric Spaces}

% 9.6
\section{Separable Metric Spaces}
