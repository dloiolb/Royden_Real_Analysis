% Chapter 6
\authoredby{inprogress}
\chapter{Differentiation and Integration}
For a function $f$ on the closed, bounded interval $[a,b]$, when is
\[
    \int_a^bf'=f(b)-f(a)?\tag{i}
\] 
Assume $f$ is continuous.
Extend $f$ to take the value $f(b)$ on $(b,b+1]$, and for $0<h\le1$, define the divided difference function $\text{Diff}_hf$ and average value function $\text{Av}_hf$ on $[a,b]$ by
\[
    \text{Diff}_hf(x):=\frac{f(x+h)-f(x)}{h}\ \text{ and }\ Av_hf(x):=\frac{1}{h}\int_x^{x+h}f(t)dt\ \text{ for all }x\in[a,b].
\]
A change of variables by translation, and cancellation, provides the discrete formulation of (i) fo the Riemann integral:
\[
    \int_a^b\text{Diff}_hf=Av_hf(b)-Av_hf(a).
\]
The limit of the RHS as $h\to0^+$ equals $f(b)-f(a)$.

% 6.1
\authoredby{inprogress}
\section{Continuity of Monotone Functions}
Recall from Chapter 1.6 that a real-valued function $f$ defined on a set $E$ of real numbers is said to be increasing provided $f(x)\le f(x')$ whenever $x,x'\in E$ and $x\le x'$, and decreasing provided $f(x)\ge f(x')$ whenever $x,x'\in E$ and $x\le x'$.
It is called monotone if it is either increasing or decreasing.
\begin{namedthm*}{Theorem 1}
    Let $f$ be a monotone function on the open interval $(a,b)$.
    Then $f$ is continuous except possibly at a countable number of points in $(a,b)$.
\end{namedthm*}
\begin{proof}
    Assume $f$ is increasing.\\
    \\Case 1: $(a,b)$ is bounded and $f:[a,b]\to\mathbb{R}$:\\
    \\For each $x_0\in(a,b)$, $f$ has a limit from the left and from the right:
    \\Because $f$ is increasing, we can define
    \begin{align*}
        f(x_0^-)&=\lim_{x\to x_0^-}f(x)=\sup\{f(x)\mid a<x<x_0\},\\
        f(x_0^+)&=\lim_{x\to x_0^+}f(x)=\sup\{f(x)\mid x_0<x<b\},
    \end{align*}
    and $f(x_0^-)\le f(x_0^+)$.
    The function $f$ fails to be continuous at $x_0$ iff $f(x_0^-)<f(x_0^+)$, in which case we define the open ``jump'' interval $J(x_0)$ by
    \[
        J(x_0):=\{y\mid f(x_0^-)<y<f(x_0^+)\}.
    \]
    Each jump interval is contained in the bounded interval $[f(a),f(b)]$ and the collection of jump intervals is disjoint.
    Therefore, for each natural number $n$, the number of jump intervals of length greater than $1/n$ is finite because the number is bounded by $\frac{f(b)-f(a)}{1/n}=n(f(b)-f(a))<\infty$.
    Thus the set of points of discontinuity of $f$ is the union of a countable collection of finite sets and therefore is countable.\\
    \\Case 2: $(a,b)$ is not bounded or $f(a^+)$ or $f(b^-)$ is not finite:\\
    \\If $(a,b)$ is not bounded, express
    \[
        (a,b)=\bigcup_{n=1}^\infty\left(-n,n\right),
    \]
    where $\left(-n,n\right)$ is bounded and $f:\left[-n,n\right]\to\mathbb{R}$ for each $n$.\\
    \\If $f(a^+)$ or $f(b^-)$ is not finite, express
    \[
        (a,b)=\bigcup_{n=1}^\infty\left(a+\frac{1}{n},b-\frac{1}{n}\right),
    \]
    where $\left(a+\frac{1}{n},b-\frac{1}{n}\right)$ is bounded and $f:\left[a+\frac{1}{n},b-\frac{1}{n}\right]\to\mathbb{R}$ for each $n$.\\
    \\Then we can use Case 1 to see that each interval has a countable set of points of discontinuity.
    \\Because $(a,b)$ is the union of an ascending sequence of such intervals, then $(a,b)$ also has a countable set of points of discontinuity.
\end{proof}
\begin{namedthm*}{Proposition 2}
    Let $C$ be a countable subset of the open interval $(a,b)$.
    Then there is an increasing function on $(a,b)$ that is continuous only at points in $(a,b)\setminus C$.
\end{namedthm*}
\begin{proof}
    If $C$ is finite the proof is clear.
    Assume $C$ is countably infinite.
    Let $\{q_n\}_{n=1}^\infty$ be an enumeration of $C$.
    Define the function $f$ on $(a,b)$ by setting
    \[
        f(x)=\sum_{\{n\mid q_n\le x\}}\frac{1}{2^n}\text{ for all }a<x<b.
    \]
    Then because $\{n\mid q_n\le x\}\subseteq\{n\mid n\in\mathbb{N}\}$, we have
    \[
        \sum_{\{n\mid q_n\le x\}}\frac{1}{2^n}\le \sum_{n=1}^\infty\frac{1}{2^n}=1.
    \]
    We see that the geometric series converges so that $f$ is properly defined.
    Moreover,
    \begin{align*}
        &\text{if }a<u<v<b,\\
        &\text{then }f(v)-f(u)=\sum_{\{n\mid q_n\le v\}}\frac{1}{2^n}-\sum_{\{n\mid q_n\le u\}}\frac{1}{2^n}=\sum_{\{n\mid u< q_n\le v\}}\frac{1}{2^n}\ge0,\tag{1}
    \end{align*}
    so that $f$ is increasing.\\
    \\Let $x_0=q_k$ belong to $C$.
    \\Then by (1),
    \[
        f(x_0)-f(x)=\sum_{\{n\mid x< q_n\le x_0=q_k\}}\frac{1}{2^n}\ge\frac{1}{2^k}\text{ for all }x<x_0.
    \]
    Therefore $f$ fails to be continuous at $x_0$:
    \\For $\epsilon=\frac{1}{2^{k+1}}>0$, for all $\delta>0$, we have $|x_0-x|<\delta$ but $|f(x_0)-f(x)|\ge\frac{1}{2^k}>\frac{1}{2^{k+1}}$.\\
    \\Let $x_0$ belong to $(a,b)\setminus C$.
    \\Fix $\epsilon>0$.
    \\Then there exists $k\in\mathbb{N}$ for which $\frac{1}{2^k}<\epsilon$, and there is an open interval $I$ containing $x_0$ for which $q_n$ does not belong to $I$ for $n\in\{1,\dots,k\}$.
    \\Then from (1), for all $x\in I$,
    \begin{align*}
        |f(x_0)-f(x)|&=\sum_{\{n\mid x< q_n\le x_0,n>k\}}\frac{1}{2^n}\\
        &\le\sum_{n={k+1}}^\infty\frac{1}{2^n}\\
        &=\sum_{n=1}^\infty\frac{1}{2^n}-\sum_{n=1}^k\frac{1}{2^n}\\
        &=1-\frac{1}{2}\sum_{n=1}^k\left(\frac{1}{2}\right)^{n-1}\\
        &=1-\frac{1}{2}\frac{1-(1/2)^k}{1-(1/2)}\\
        &=\frac{1}{2^k}\\
        &<\epsilon.
    \end{align*}
    Therefore $f$ is continuous at $x_0$.
\end{proof}
\begin{center}
	\textbf{PROBLEMS}
\end{center}
\begin{enumerate}
	\setcounter{enumi}{0}
    \item Let $C$ be a countable subset of the nondegenerate closed, bounded interval $[a,b]$. Show that there is an increasing function on $[a,b]$ that is continuous only at points in $[a,b]\setminus C$.\\
    \\Simply take the function $f$ from Proposition 2, where we have
    \begin{align*}
        \lim_{x\to a^+}f(x)&=\lim_{x\to a^+}\sum_{\{n\mid q_n\le a\}}\frac{1}{2^n}=0,\\
        \lim_{x\to b^-}f(x)&=\lim_{x\to b^-}\sum_{\{n\mid q_n\le b\}}\frac{1}{2^n}=1.
    \end{align*}
    If the points $a$ or $b$ are not in $C$, we can extend $f$ to $[a,b]$ by defining
    \begin{align*}
        f(a)&:=0,\\
        f(b)&:=1,
    \end{align*}
    which makes $f$ continuous at $a$ and $b$.
    \\Else if $a$ or $b$ are in $C$, then we can extend $f$ to $[a,b]$ by defining
    \begin{align*}
        f(a)&:=y,\text{ for any real number }y\neq0,\\
        f(b)&:=y,\text{ for any real number }y\neq1,
    \end{align*}
    which makes $f$ discontinuous at $a$ and $b$.
    \item Show that there is a strictly increasing function on $[0,1]$ that is continuous only at the irrational numbers in $[0,1]$.\\
    \\Enumerate $\mathbb{Q}\cap[0,1]$ by $\{q_n\}_{n=1}^\infty$, and define the function $f$ on $[0,1]$ by setting
    \[
        f(x)=\sum_{\{n\mid q_n\le x\}}\frac{1}{n^2}\text{ for }0<x\le1.
    \]
    We can see that for any $0\le a< b\le1$, by density of the rationals there exists a rational number $q_k\in(a,b)$ so that we have
    \[
        f(b)-f(a)=\sum_{\{n\mid q_n\le b\}}\frac{1}{n^2}-\sum_{\{n\mid q_n\le a\}}\frac{1}{n^2}=\sum_{\{n\mid a<q_n\le b\}}\frac{1}{n^2}\ge\frac{1}{q_k^2}>0,
    \]
    and therefore $f$ is strictly increasing.\\
    \\In the case $x_0\in\mathbb{Q}\cap[0,1]$, $x_0\neq0$:
    \\Then there exists an index $k$ such that $x_0=q_k$.
    \\We can write, for all $x\in[0,1]$ such that $x<x_0$,
    \[
        f(x_0)-f(x)=\sum_{\{n\mid x<q_n\le x_0=q_k\}}\frac{1}{n^2}\ge\frac{1}{q_k^2}.
    \]
    Then $f$ is discontinuous at $x_0$:
    \\For $\epsilon=\frac{1}{2\cdot q_k^2}>0$, for all $\delta>0$, we have $|x_0-x|<\delta$ but $|f(x_0)-f(x)|\ge\frac{1}{q_k^2}>\frac{1}{2\cdot q_k^2}$.\\
    \\In the case $x_0\in\mathbb{Q}^c\cap[0,1]$:
    \\Fix $\epsilon>0$.
    \\We have $\sum_{n=1}^\infty\frac{1}{n^2}=\frac{\pi^2}{6}$, so that the sequence of partial sums $\sum_{n=1}^k\frac{1}{n^2}$ converges.
    \\That is, there exists $k\in\mathbb{N}$ for which $\sum_{n=1}^\infty\frac{1}{n^2}-\sum_{n=1}^k\frac{1}{n^2}=\sum_{n=k}^\infty\frac{1}{n^2}<\epsilon$, and there is an open interval $I$ containing $x_0$ for which $q_n$ does not belong to $I$ for $n\in\{1,\dots,k\}$.
    \\Then for all $x\in I$,
    \begin{align*}
        |f(x_0)-f(x)|&=\sum_{\{n\mid x< q_n\le x_0,n>k\}}\frac{1}{n^2}
        \le\sum_{n={k+1}}^\infty\frac{1}{n^2}
        <\epsilon.
    \end{align*}
    Therefore $f$ is continuous at $x_0$.
    \item Let $f$ be a monotone function on a subset $E$ of $\mathbb{R}$. Show that $f$ is continuous except possibly at a countable number of points in $E$.\\
    \\By the $\sigma$-finite property of the Lebesgue measure on the reals, we can write $E$ as a disjoint union of sets of finite measure:
    \[
        a
    \]
    \item Let $E$ be a subset of $\mathbb{R}$ and let $C$ be a countable subset of $E$. Is there a monotone function on $E$ that is continuous only at points in $E\setminus C$?\\
    \\(generalization of Proposition 2)
    \\If $C$ is finite the proof is clear.
    Assume $C$ is countably infinite.
    Let $\{q_n\}_{n=1}^\infty$ be an enumeration of $C$.
    Define the function $f$ on $E$ by setting
    \[
        f(x)=\sum_{\{n\mid q_n\le x\}}\frac{1}{2^n}\text{ for all }x\in E.
    \]
    Then because $\{n\mid q_n\le x\}\subseteq\{n\mid n\in\mathbb{N}\}$, we have
    \[
        \sum_{\{n\mid q_n\le x\}}\frac{1}{2^n}\le \sum_{n=1}^\infty\frac{1}{2^n}=1.
    \]
    We see that the geometric series converges so that $f$ is properly defined.
    Moreover,
    \begin{align*}
        &\text{if }u,v\in E\text{ with }u<v,\\
        &\text{then }f(v)-f(u)=\sum_{\{n\mid q_n\le v\}}\frac{1}{2^n}-\sum_{\{n\mid q_n\le u\}}\frac{1}{2^n}=\sum_{\{n\mid u< q_n\le v\}}\frac{1}{2^n}\ge0,\tag{1}
    \end{align*}
    so that $f$ is increasing.\\
    \\Let $x_0=q_k$ belong to $C$.
    \\Then by (1),
    \[
        f(x_0)-f(x)=\sum_{\{n\mid x< q_n\le x_0=q_k\}}\frac{1}{2^n}\ge\frac{1}{2^k}\text{ for all }x<x_0.
    \]
    Therefore $f$ fails to be continuous at $x_0$:
    \\For $\epsilon=\frac{1}{2^{k+1}}>0$, for all $\delta>0$, we have $|x_0-x|<\delta$ but $|f(x_0)-f(x)|\ge\frac{1}{2^k}>\frac{1}{2^{k+1}}$.\\
    \\Let $x_0$ belong to $(a,b)\setminus C$.
    \\Fix $\epsilon>0$.
    \\Then there exists $k\in\mathbb{N}$ for which $\frac{1}{2^k}<\epsilon$, and there is an open interval of real numbers $I$ containing $x_0$ for which $q_n$ does not belong to $I$ for $n\in\{1,\dots,k\}$.
    \\Then from (1), for all $x\in I\cap E$,
    \begin{align*}
        |f(x_0)-f(x)|&=\sum_{\{n\mid x< q_n\le x_0,n>k\}}\frac{1}{2^n}\\
        &\le\sum_{n={k+1}}^\infty\frac{1}{2^n}\\
        &=\sum_{n=1}^\infty\frac{1}{2^n}-\sum_{n=1}^k\frac{1}{2^n}\\
        &=1-\frac{1}{2}\sum_{n=1}^k\left(\frac{1}{2}\right)^{n-1}\\
        &=1-\frac{1}{2}\frac{1-(1/2)^k}{1-(1/2)}\\
        &=\frac{1}{2^k}\\
        &<\epsilon.
    \end{align*}
    Therefore $f$ is continuous at $x_0$.
\end{enumerate}

% 6.2
\authoredby{inprogress}
\section{Differentiability of Monotone Functions: Lebesgue's Theorem}
A closed, bounded interval $[c,d]$ is said to be nondegenerate provided $c<d$.
\begin{center}
	\textbf{PROBLEMS}
\end{center}
\begin{enumerate}
	\setcounter{enumi}{4}
    \item Show that the Vitali Covering Lemma does not extend to the case in which the covering collection has degenerate closed intervals.
    \item Show that the Vitali Covering Lemma does extend to the case in which the covering collection consists of nondegenerate general intervals.
    \item let $f$ be continuous on $\mathbb{R}$. Is there an open interval on which $f$ is monotone?
    \item Let $I$ and $J$ be closed, bounded intervals and $\gamma>0$ be such that $\ell(I)>\gamma\cdot\ell(J)$.
    Assume $I\cap J\neq\emptyset$. Show that if $\gamma\ge 1/2$, then $J\subseteq 5*I$, where $5*I$ denotes the interval with the same center as $I$ and five times its length.
    Is the same true if $0<\gamma<1/2$?
    \item Show that a set $E$ of real numbers has measure zero iff there is a countable collection of open intervals $\{I_k\}_{k=1}^\infty$ for which each point in $E$ belongs to infinitely many of the $I_k's$ and $\sum_{k=1}^\infty\ell(I_k)<\infty$.
    \item (Riesz-Nagy) Let $E$ be a set of measure zero contained in the open interval $(a,b)$.
    According to the preceding problem, there is a countable collection of open intervals contained in $(a,b)$, $\{(c_k,d_k)\}_{k=1}^\infty$, for which each point in $E$ belongs to infinitely many intervals in the collection and $\sum_{k=1}^\infty(d_k-c_k)<\infty$.
    Define
    \[
        f(x)=\sum_{k=1}^\infty\ell((c_k,d_k)\cap(-\infty,x))\text{ for all }x\text{ in }(a,b).
    \]
    Show that $f$ is increasing and fails to be differentiable at each point in $E$.
    \item For real numbers $\alpha<\beta$ and $\gamma>0$, show that if $g$ is integrable over $[\alpha+\gamma,\beta+\gamma]$, Then
    \[
        \int_{\alpha}^{\beta}g(t+\gamma)dt=\int_{\alpha+\gamma}^{\beta+\gamma}g(t)dt.  
    \]
    Prove this change of variables formula by successively considering simple functions, bounded measurable functions, nonnegative integrable functions, and general integrable functions.
    Use it to prove (14).
    \item Compute the upper and lower derivatives of the characteristic function of the rationals.
    \item Let $E$ be a set of finite outer measure and $\mathcal{F}$ a collection of closed, bounded intervals that cover $E$ in the sense of Vitali.
    Show that there is a countable disjoint collection $\{I_k\}_{k=1}^\infty$ of intervals in $\mathcal{F}$ for which
    \[
        m^*\biggl[E\setminus\bigcup_{k=1}^\infty I_k\biggr]=0.    
    \]
    \item Use the Vitali Covering Lemma to show that the union of any collection (countable or uncountable) of closed, bounded, nondegenerate intervals is measurable.
    \item Define $f$ on $\mathbb{R}$ by
    \[
        f(x)=
        \begin{cases}
            x\sin(1/x)&x\neq0\\
            0&x=0
        \end{cases}    
    \]
    Find the upper and lower derivatives of $f$ at $x=0$.
    \item Let $g$ be integrable over $[a,b]$. Define the antiderivative of $g$ of $g$ to be the function $f$ defined on $[a,b]$ by
    \[
        f(x)=\int_a^xg\text{ for all }x\in[a,b].    
    \]
    Show that $f$ is differentiable almost everywhere on $(a,b)$.
    \item Let $f$ be an increasing bounded function on the open, bounded interval $(a,b)$. Verify (18).
    \item Show that if $f$ is defined on $(a,b)$ and $c\in(a,b)$ is a local minimizer for $f$, then $\underline{D}f(c)\le0\le\overline{D}f(c)$.
    \item Let $f$ be continuous on $[a,b]$ with $\underline{D}f\ge0$ on $(a,b)$. Show that $f$ is increasing on $[a,b]$.
    (Hint: first show this for a function $g$ for which $\underline{D}g\ge\epsilon>0$ on $(a,b)$. Apply this to the function $g(x)=f(x)+\epsilon x$.)
    \item Let $f$ and $g$ be real-valued functions on $(a,b)$. Show That
    \[
        \underline{D}f+\underline{D}g\le\underline{D}(f+g)\le\overline{D}(f+g)\le \overline{D}(f)+\overline{D}(g)\text{ on }(a,b).
    \]
    \item Let $f$ be defined on $[a,b]$ and $g$ a continuous function on $[\alpha,\beta]$ that is differentiable at $\gamma\in(\alpha,\beta)$ with $g(\gamma)=c\in(a,b)$. Verify the following.
    \begin{enumerate}[label=(\roman*),align=left]
        \item If $g'(\gamma)>0$, then $\overline{D}(f\circ g)(\gamma)=\overline{D}f(c)\cdot g'(\gamma)$.
        \item If $g'(\gamma)=0$ and the upper and lower derivatives of $f$ at $c$ are finite, then $\overline{D}(f\circ g)(\gamma)=0$.
    \end{enumerate}
    \item Show that a strictly increasing function that is defined on an interval is measurable and then use this to show that a monotone function that is defined on an interval is measurable.
    \item Show that a continuous function $f$ on $[a,b]$ is Lipschitz if its upper and lower derivatives are bounded on $(a,b)$.
    \item Show that for $f$ defined in the last remark of this section, $f'$ is not integrable over $[0,1]$.
\end{enumerate}

% 6.3
\authoredby{untouched}
\section{Functions of Bounded Variation: Jordan's Theorem}
\begin{center}
	\textbf{PROBLEMS}
\end{center}
\begin{enumerate}
	\setcounter{enumi}{24}
    \item Suppose $f$ is continuous on $[0,1]$. Must there be a nondegenerate closed subinterval $[a,b]$ of $[0,1]$ for which the restriction of $f$ to $[a,b]$ is of bounded variation?
    \item Let $f$ be the Dirichlet function, the characteristic function of the rationals in $[0,1]$. Is $f$ of bounded variation on $[0,1]$?
    \item Define $f(x)=\sin x$ on $[0,2\pi]$. Find two increasing functions $h$ and $g$ for which $f=h-g$ on $[0,2\pi]$.
    \item Let $f$ be a step function on $[a,b]$. Find a formula for its total variation.
    \item
    \begin{enumerate}[label=(\roman*),align=left]
        \item Define
        \[
            f(x)=
            \begin{cases}
                x^2\cos(1/x^2)&x\neq0,x\in[-1,1]\\
                0&x=0
            \end{cases}  
        \]
        Is $f$ of bounded variation on $[-1,1]$?
        \item Define
        \[
            g(x)=
            \begin{cases}
                x^2\cos(1/x)&x\neq0,x\in[-1,1]\\
                0&x=0
            \end{cases}  
        \]
        Is $g$ of bounded variation on $[-1,1]$?
    \end{enumerate}
    \item Show that the linear combination of two functions of bounded variation is also of bounded variation.
    Is the product of two such functions also of bounded variation?
    \item Let $P$ be a partition of $[a,b]$ that is a refinement of the partition $P'$. For a real-valued function $f$ on $[a,b]$, show that $V(f,P')\le V(f,P)$.
    \item Assume $f$ is of bounded variation on $[a,b]$. Show that there is a sequence of partitions $\{P_n\}$ of $[a,b]$ for which the sequence $\{V(f,P_n)\}$ is increasing and converges to $TV(f)$.
    \item Let $\{f_n\}$ be a sequence of real-valued functions on $[a,b]$ that converges pointwise on $[a,b]$ to the real-valued function $f$. Show that 
    \[
        TV(f)\le\lim\inf TV(f_n).  
    \]
    \item Let $f$ and $g$ be of bounded variation on $[a,b]$. Show that 
    \[
        TV(f+g)\le TV(f)+TV(g)\text{ and }TV(\alpha f)=|\alpha|TV(f).
    \]
    \item For $\alpha$ and $\beta$ positive numbers, define the function $f$ on $[0,1]$ by
    \[
        f(x)=
        \begin{cases}
            x^\alpha\sin(1/x^\beta)&\text{for }0<x\le1\\
            0&\text{for }x=0
        \end{cases}    
    \]
    Show that if $\alpha>\beta$, then $f$ is of bounded variation on $[0,1]$, by showing that $f'$ is integrable over $[0,1]$. Then show that if $\alpha\le\beta$, then $f$ is not of bounded variation on $[0,1]$.
    \item Let $f$ fail to be of bounded variation on $[0,1]$. Show that there is a point $x_0$ in $[0,1]$ such that there are subintervals of $[0,1]$ that contain $x_0$ and have arbitrarily small length on which $f$ fails to be of bounded variation.
\end{enumerate}

% 6.4
\section{Absolutely Continuous Functions}
\begin{center}
	\textbf{PROBLEMS}
\end{center}
\begin{enumerate}
	\setcounter{enumi}{36}
    \item Let $f$ be a continuous function on $[0,1]$ that is absolutely continuous on $[\epsilon,1]$ for each $0<\epsilon<1$.
    \begin{enumerate}[label=(\roman*),align=left]
        \item Show that $f$ may not be absolutely continuous on $[0,1]$.
        \item Show that $f$ is absolutely continuous on $[0,1]$ if it is increasing.
        \item Show that the function $f$ on $[0,1]$, defined by $f(x)=\sqrt(x)$ for $0\le x\le1$, is absolutely continuous, but not Lipschitz, on $[0,1]$.
    \end{enumerate}
    \item Show that $f$ is absolutely continuous on $[a,b]$ iff for each $\epsilon>0$, there is a $\delta>0$ such that for every countable disjoint collection $\{(a_k,b_k)\}_{k=1}^\infty$ of open intervals in $(a,b)$,
    \[
        \sum_{k=1}^\infty|f(b_k)-f(a_k)|<\epsilon\text{ if }\sum_{k=1}^\infty[b_k-a_k]<\delta.  
    \]
    \item Use the preceding problem to show that if $f$ is continuous and increasing on $[a,b]$, then $f$ is absolutely continuous on $[a,b]$ iff for each $\epsilon$, there is a $\delta>0$ such that for a measurable subset $E$ of $[a,b]$,
    \[
        m^*(f(E))<\epsilon\text{ if }m(E)<\delta.  
    \]
    \item Use the preceding problem to show that an increasing absolutely continuous function $f$ on $[a,b]$ maps sets of measure zero onto sets of measure zero.
    Conclude that the Cantor-Lebesgue function $\varphi$ is not absolutely continuous on $[0,1]$ since the function $\psi$, defined by $\psi(x)=x+\varphi(x)$ for $0\le x\le1$, maps the Cantor set to a set of measure 1 (page 52).
    \item Let $f$ be an increasing absolutely continuous function on $[a,b]$. Use (i) and (ii) below to conclude that $f$ maps measurable sets to measurable sets.
    \begin{enumerate}[label=(\roman*),align=left]
        \item Infer from the continuity of $f$ and the compactness of $[a,b]$ that $f$ maps closed sets to closed sets and therefore maps $F_\sigma$ sets to $F_\sigma$ sets.
        \item The preceding problem tells us that $f$ maps sets of measure zero to sets of measure zero. 
    \end{enumerate}
    \item Show that both the sum and product of absolutely continuous functions are absolutely continuous.
    \item Define the functions $f$ and $g$ on $[-1,1]$ by $f(x)=x^{\frac{1}{3}}$ for $-1\le x\le 1$ and
    \[
        g(x)=
        \begin{cases}
            x^2\cos(\pi/2x)&\text{if }x\neq0,x\in[-1,1]\\
            0&\text{if }x=0
        \end{cases}  
    \]
    \begin{enumerate}[label=(\roman*),align=left]
        \item Show that both $f$ and $g$ are absolutely continuous on $[-1,1]$.
        \item For the partition $P_n=\{-1,0,1/2n,1/[2n-1],\cdots,1/3,1/2,1\}$ of $[-1,1]$, examine $V(f\circ g,P_n)$.
        \item Show that $f\circ g$ fails to be of bounded variation, and hence also fails to be absolutely continuous, on $[-1,1]$. 
    \end{enumerate}
    \item Let $f$ be Lipschitz on $\mathbb{R}$ and $g$ be absolutely continuous on $[a,b]$. Show that the composition $f \circ g$ is absolutely continuous on $[a,b]$.
    \item Let $f$ be absolutely continuous on $\mathbb{R}$ and $g$ be absolutely continuous and strictly monotone on $[a,b]$. Show that the composition $f\circ g$ is absolutely continuous on $[a,b]$.
    \item Verify the assertions made in the final remark of this section.
    \item Show that a function $f$ is absolutely continuous on $[a,b]$ iff for each $\epsilon>0$, there is a $\delta>0$ such that for every finite disjoint collection $\{(a_k,b_k)\}_{k=1}^n$ of open intervals in $(a,b)$,
    \[
        \biggl|\sum_{k=1}^n[f(b_k)-f(a_k)]\biggr|<\epsilon\text{ if }\sum_{k=1}^n[b_k-a_k]<\delta.  
    \]
\end{enumerate}

% 6.5
\section{Integrating Derivatives: Differentiating Indefinite Integrals}

\begin{center}
	\textbf{PROBLEMS}
\end{center}
\begin{enumerate}
	\setcounter{enumi}{47}
    \item The Cantor-Lebesgue function $\varphi$ is continuous and increasing on $[0,1]$. Conclude from Theorem 10 that $\varphi$ is not absolutely continuous on $[0,1]$.
    Compare this reasoning with that proposed in Problem 40.
    \item Let $f$ be continuous on $[a,b]$ and differentiable almost everywhere on $(a,b)$. Show that 
    \[
        \int_a^bf'=f(b)-f(a)  
    \]
    if and only if 
    \[
        \int_a^b[\lim_{n\to\infty}\text{Diff}_{1/n}f]=\lim_{n\to\infty}[\int_a^b\text{Diff}_{1/n}f].
    \]
    \item Let $f$ be continuous on $[a,b]$ and differentiable almost everywhere on $(a,b)$. Show that if $\{\text{Diff}_{1/n}f\}$ is uniformly integrable over $[a,b]$, then
    \[
        \int_a^bf'=f(b)-f(a).  
    \]
    \item Let $f$ be continuous on $[a,b]$ and differentiable almost everywhere on $(a,b)$. Suppose there is a nonnegative function $g$ that is integrable over $[a,b]$ and 
    \[
        |\text{Diff}_{1/n}f|\le g\text{ a.e. on }[a,b]\text{ for all }n.  
    \]
    Show that 
    \[
        \int_a^bf'=f(b)-f(a).  
    \]
    \item Let $f$ and $g$ be absolutely continuous on $[a,b]$. Show that
    \[
        \int_a^bf\cdot g'=f(b)g(b)-f(a)g(a)-\int_a^bf'\cdot g.  
    \] 
    \item Let the function $f$ be absolutely continuous on $[a,b]$. Show that $f$ is Lipschitz on $[a,b]$ iff there is a $c>0$ for which $|f'|\le c$ a.e. on $[a,b]$.
    \item 
    \begin{enumerate}[label=(\roman*),align=left]
        \item Let $f$ be a singular increasing function on $[a,b]$. Use the Vitali Covering Lemma to show that $f$ has the following property: Given $\epsilon>0,\delta>0$, there is a finite disjoint collection $\{(a_k,b_k)\}_{k=1}^n$ of open intervals in $(a,b)$ for which
        \[
            \sum_{k=1}^n[b_k-a_k]<\delta\text{ and }\sum_{k=1}^n[f(b_k)-f(a_k)]>f(b)-f(a)-\epsilon.  
        \]
        \item Let $f$ be an increasing function on $[a,b]$ with the property described in part (i). Show that $f$ is singular.
        \item Let $\{f_n\}$ be a sequence of singular increasing functions on $[a,b]$ for which the series $\sum_{n=1}^\infty f_n(x)$ converges to a finite value for each $x\in[a,b]$. Define 
        \[
            f(c)=\sum_{n=1}^\infty f_n(x)\text{ for }x\in[a,b].  
        \]
        Show that $f$ is also singular.
    \end{enumerate}
    \item Let $f$ be of bounded variation on $[a,b]$, and define $v(x)=TV(f_{[a,x]})$ for all $x\in[a,b]$.
    \begin{enumerate}[label=(\roman*),align=left]
        \item Show that $|f'|\le v'$ a.e. on $[a,b]$, and infer from this that 
        \[
            \int_a^b|f'|\le TV(f).  
        \]
        \item Show that the above is an equality iff $f$ is absolutely continuous on $[a,b]$.
        \item Compare parts (i) and (ii) with Corollaries 4 and 12, respectively.
    \end{enumerate}
    \item Let $g$ be strictly increasing and absolutely continuous on $[a,b]$.
    \begin{enumerate}[label=(\roman*),align=left]
        \item Show that for any open subset $\mathcal{O}$ of $(a,b)$,
        \[
            m(g(\mathcal{O}))=\int_{\mathcal{O}}g'(x)dx.  
        \]
        \item Show that for any $G_\delta$ subset $E$ of $(a,b)$,
        \[
            m(g(E))=\int_Eg'(x)dx.  
        \]
        \item Show that for any subset $E$ of $[a,b]$ that has measure 0, its image $g(E)$ also has measure 0, so that
        \[
            m(g(E))=0=\int_{E}g'(x)dx.  
        \]
        \item Show that for any measurable subset $A$ of $[a,b]$,
        \[
            m(g(A))=\int_{A}g'(x)dx.  
        \]
        \item Let $c=g(a)$ and $d=g(b)$. Show that for any simple function $\varphi$ on $[c,d]$,
        \[
            \int_c^d\varphi(y)dy=\int_a^b\varphi(g(x))g'(x)dx.
        \]
        \item Show that for any nonnegative integrable function $f$ over $[c,d]$,
        \[
            \int_c^df(y)dy=\int_a^bf(g(x))g'(x)dx.
        \]
        \item Show that part (i) follows from (vi) in the case that $f$ is the characteristic function of $g(\mathcal{O})$ and the composition is defined.
    \end{enumerate}
    \item Is the change of variables formula in part (vi) of the preceding problem true if we just assume $g$ is increasing, not necessarily strictly?
    \item Construct an absolutely continuous strictly increasing function $f$ on $[0,1]$ for which $f'=0$ on a set of positive measure.
    (Hint: Let $E$ be the relative complement in $[0,1]$ of a generalized Cantor set of positive measure and $f$ the indefinite integral of $\chi_E$. See Problem 39 of Chapter 2 for the construction of such a Cantor set.)
    \item For a nonnegative integrable function $f$ over $[c,d]$, and a strictly increasing absolutely continuous function $g$ on $[a,b]$ such that $g([a,b])\subseteq [c,d]$, is it possible to justify the change of variables formula
    \[
        \int_{g(a)}^{g(b)}f(y)dy=\int_a^bf(g(x))g'(x)dx  
    \]
    by showing that
    \[
        \frac{d}{dx}\biggl[\int_{g(a)}^{g(x)}f(s)ds-\int_a^xf(g(t))g'(t)dt\biggr]=0\text{ for almost all }x\in(a,b)?  
    \]\
    \item Let $f$ be absolutely continuous and singular on $[a,b]$. Show that $f$ is constant. Also show that the Lebesgue decomposition of a function of bounded variation is unique if the singular function is required to vanish a t $x=a$.
\end{enumerate}

% 6.6
\section{Convex Functions}

\begin{center}
	\textbf{PROBLEMS}
\end{center}
\begin{enumerate}
	\setcounter{enumi}{60}
	\item Show that a real-valued function $\varphi$ on $(a,b)$ is convex iff for points $x_1,\cdots,x_n$ in $(a,b)$ and nonnegative numbers $\lambda_1,\cdots,\lambda_n$ such that $\sum_{k=1}^n \lambda_k=1$,
	\[
        \varphi \biggr ( \sum_{k=1}^n \lambda_k x_k \biggl ) \le \sum_{k=1}^n \lambda_k \varphi(x_k).   
    \]
    Use this to directly prove Jensen's Inequality for $f$ a simple function.
    \item Show that a continuous function on $(a,b)$ is convex iff
    \[
        \varphi(\frac{x_1+x_2}{2})\le\frac{\varphi(x_1)+\varphi(x_2)}{2}\text{ for all }x_1,x_2\in (a,b).  
    \]
    \item A function on a general interval $I$ is said to be convex provided it is continuous on $I$ and (38) holds for all $x_1,x_2\in I$.
    Is a convex function on a closed, bounded interval $[a,b]$ necessarily Lipschitz on $[a,b]$?
    \item Let $\varphi$ have a second derivative at each point in $(a,b)$.
    Show that $\varphi$ is convex iff $\varphi''$ is nonnegative.
    \item Suppose $a\ge 0$ and $b\ge 0$.
    Show that the function $\varphi(t)=(a+bt)^p$ is convex on $[0,\infty)$ for $1\le p < \infty$.
    \item For what functions $\varphi$ is Jensen's Inequality always an equality?
    \item State and prove a version of Jensen's Inequality on a general closed, bounded interval $[a,b]$.
    \item Let $f$ be integrable over $[0,1]$. Show that 
    \[
    \exp\biggl [\int_0^1f(x)dx\biggr ] \le \int_0^1 \exp(f(x))dx.   
    \]
    \item Let $\{\alpha_n\}$ be a sequence of nonnegative numbers whose sum is $1$ and $\{\zeta_n\}$ is a sequence of positive numbers. Show that
    \[
    \prod_{n=1}^\infty \zeta_n^{\alpha_n} \le \sum_{n=1}^\infty \alpha_n \zeta_n. 
    \]
    \item Let $g$ be a positive measurable function on $[0,1]$. Show that $\log(\int_0^1g(x)dx) \ge\int_0^1\log(g(x))dx$ whenever each side is defined.
    \item (Nemytskii) Let $\varphi$ be a continuous function on $\mathbb{R}$.
    Show that if there are constants for which (43) holds, then $\varphi\circ f$ is integrable over $[0,1]$ whenever $f$ is.
    Then show that if $\varphi\circ f$ is integrable over $[0,1]$ whenever $f$ is, then there are constants $c_1$ and $c_2$ for which (43) holds.
\end{enumerate}