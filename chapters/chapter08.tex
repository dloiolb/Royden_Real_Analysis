% Chapter 8
\authoredby{inprogress}
\chapter{The $L^p$ Spaces: Duality and Weak Convergence}

% 8.1
\authoredby{inprogress}
\section{The Riesz Representation for the Dual of $L^p,a\le p\le \infty$}

\textbf{Example}
Let $E$ be a measurable set, $1\le p<\infty$, $q$ the conjugate of $p$, and $g$ belong to $L^q(E)$.
Define the functional $T$ on $L^p(E)$ by
\[
    T(f)=\int_Eg\cdot f\text{ for all }f\in L^p(E).
\]
H\"older's Inequality tells us that for $f\in L^p(E)$, the product $g\cdot f$ is integrable over $E$ so the functional $T$ is properly defined.
By the linearity of integration, $T$ is linear.
Observe that H\"older's inequality is the statement that 
\[
    |T(f)|\le\|g\|_q\cdot\|f\|_p\text{ for all }f\in L^p(E).
\]
\begin{flushleft}

\textbf{Example}
Let $[a,b]$ be a closed, bounded interval and the function $g$ be of bounded variation on $[a,b]$.
Define the functional $T$ on $C[a,b]$ by
\[
    T(f)=\int_a^bf(x)dg(x)\text{ for all }f\in C[a,b],
\]  
where the integral is in the sense of Riemann-Stieltjes.
The functional $T$ is properly defined and linear.
Moreover, it follows immediately from the definition of this integral that 
\[
    |T(f)|\le TV(g)\cdot\|f\|_{\max}\text{ for all }f\in C[a,b],
\]  
where $TV(g)$ is the total variation of $g$ over $[a,b]$.
\end{flushleft}
\begin{namedthm*}{Definition}
    For a normed linear space $X$, a linear functional $T$ on $X$ is said to be \textbf{bounded} provided there is an $M\ge0$ for which 
\[
    |T(f)|\le M\|f\|\text{ for all }f\in X.
\]
The infimum of all such $M$ is called the \textbf{norm} of $T$ and denoted by $\|T\|_*$.
\end{namedthm*}
The inequalities in the first and second example above tell us that the linear functionals are bounded.

Let $T$ be a bounded linear functional on the normed linear space $X$.
It is easy to see that the above equation holds for $M=\|T\|_*$.
Hence, by the linearity of $T$,
\[
    |T(f)-T(h)|\le \|T\|_*\|f-h\|\text{ for all }f,h\in X.
\]
From this we infer the following continuity property of a bounded linear functional $T$:
\[
    \text{if }\{f_n\}\to f\text{ in }X,\text{ then }\{T(f_n)\}\to T(f).\tag{7}
\]
We leave it as an exercise (see chapter 13 problem 11) to show that 
\[
    \|T\|_*=\sup\{T(f)\mid f\in X,\|f\|\le1\},\tag{8}
\]
and use this characterization of $\|\cdot\|_*$ to prove the following proposition.
\begin{namedthm*}{Proposition 1}
    Let $X$ be a normed linear space.
    Then the collection of bounded linear functionals on $X$ is a linear space on which $\|\cdot\|_*$ is a norm.
    This normed linear space is called the \textbf{dual space} of $X$ and denoted by $X^*$.
\end{namedthm*}
\begin{namedthm*}{Proposition 2}
    Let $E$ be a measurable set, $1\le p<\infty$, $q$ the conjugate of $p$, and $g$ belong to $L^q(E)$.
    Define the functional $T$ on $L^p(E)$ by
    \[
        T(f)=\int_Eg\cdot f\text{ for all }f\in L^p(E).
    \]
    Then $T$ is a bounded linear functional on $L^p(E)$ and $\|T\|_*=\|g\|_q$.
\end{namedthm*}
\begin{proof}
    By linearity of integration, $T$ is linear, and from H\"older's inequality, we infer that $T$ is bounded and $\|T\|_*\le\|g\|_q$.
    For $p>1$, according to Chapter 7 Theorem 1, the conjugate function of $g$, $g^*=\|g\|_q^{1-q}\text{sgn}(g)|g|^{q-1}$, belongs to $L^p(E)$, and 
    \[
        T(g^*)=\int_Eg\cdot g^*=\|g\|_q\quad\text{and}\quad\|g^*\|_p=1.
    \] 
    Then by definition of the operator norm as a supremum (8), we have $\|T\|_*\ge T(g^*)=\|g\|_q$.
    Thus from the two inequalities we obtain $\|T\|_*=\|g\|_q$.
    For $p=1$, we argue by contradiction.
    If we have strict inequality, i.e., $\|g\|_\infty>\|T\|_*$, then by definition of supremum norm there is a set $A$ of finite positive measure on which $|g|>\|T\|_*$.
    Define $f:=[1/m(A)][\text{sgn}(g)]\chi_A$. 
    Then $\|f\|_1=1/m(A)\int_E\chi_A=1$ and yet, using the fact that $g\cdot\text{sgn}(g)=|g|$ and monotonicity of integration,
    \[
        T(f)=1/m(A)\int_E|g|\chi_A>1/m(A)\int_E\|T\|_*\chi_A=\|T\|_*,
    \]
    which is a contradiction to the fact that $\|T\|_*$ is the supremum.

\end{proof}
Our goal now is to prove that for $1\le p<\infty$, every bounded linear functional on $L^p(E)$ is given by integration against a function in $L^q(E)$, where $q$ is the conjugate of $p$.

\begin{center}
	\textbf{PROBLEMS}
\end{center}
\begin{enumerate}
	\setcounter{enumi}{0}
    \item Verify (8).
    
    \ \\(See Chapter 13 Problem 11) 
    Define 
    \[
        \begin{split}
        M'&:=\inf\{M\ge0\mid\|T(f)\|\le M\|f\|\text{ for all }f\in X\},\\
        N'&:=\sup\{\|T(f)\|\mid f\in X, \|f\|\le1\}.
        \end{split}
    \]
    We aim to show that they are equal.
    
    First, for $f\neq0$, then $\|\frac{f}{\|f\|}\|\le1$ so that by linearity of $T$,
    \[\|T(\frac{f}{\|f\|})\|\le N'\implies\|Tf\|\le N'\|f\|\implies M'\le N'.\]
    On the other hand, for $f$ such that $\|f\|\le1$, then
    \[\|Tf\|\le M'\|f\|\le M'\implies N'\le M.\]
    Therefore $M'=N'$.
    \\\item Prove Proposition 1.

    This was proved in Chapter 13.2 Proposition 2.
    \item Let $T$ be a linear functional on a normed linear space $X$. Show that $T$ is bounded iff the continuity property (7) holds.
    
    This was proved in Chapter 13.2 Theorem 1.
    \item A functional $T$ on a normed linear space $X$ is said to be Lipschitz provided there is a $c\ge0$ such that
    \[
        |T(g)-T(h)|\le c\|g-h\|\text{ for all }g,h\in X.  
    \]
    The infimum of such $c$'s is called the Lipschitz constant for $T$. Show that a linear functional is bounded iff it is Lipschitz, in which case its Lipschitz constant is $\|T\|_*$.
    
    Suppose a linear functional $T$ is Lipschitz. We know that Lipschitz implies continuity, and so the linear functional is bounded by Problem 3 above.
    On the other hand, suppose the linear functional is bounded.
    Then we may consider the norm defined by $\|T\|_*=\inf\{C>0:|T(x)|\le C\|x\|\text{ for all }x\in X\}$.
    But simply by linearity, $|T(g)-T(h)|=\|T(g-h)\|\le \|T\|_*\|g-h\|\text{ for all }g,h\in X$.
    \item Let $E$ be a measurable set and $1\le 0<\infty$. Show that the functions in $L^p(E)$ that vanish outside a bounded set are dense in $L^p(E)$. Show that this is false for $L^\infty(\mathbb{R})$.
    \item Establish the Riesz Representation Theorem in the case $p=1$ by first showing, in the notation of the proof of the theorem, that the function $\Phi$ is Lipschitz and therefore it is absolutely continuous. Then follow the $p>1$ proof.
    \item State and prove a Riesz Representation Theorem for the bounded linear functionals on $\ell^p$, $1\le p<\infty$.
    \item Let $c$ be the linear space of real sequences that converge to a real number and $c_0$ the subspace of $c$ comprising sequences that converge to $0$. Norm each of these linear spaces with the $\ell^\infty$ norm. Determine the dual space of $c$ and $c_0$.
    \item Let $[a,b]$ be a closed, bounded interval and $C[a,b]$ be normed by the maximum norm. Let $x_0$ belong to $[a,b]$. Define the linear functional $T$ on $C[a,b]$ by $T(f)=f(x_0)$. Show that $T$ is bounded and is given by Riemann-Stieltjes integration against a function of bounded variation.
    \item Let $f$ belong to $C[a,b]$. Show that there is a function $g$ that is of bounded variation on $[a,b]$ for which 
    \[
        \int_a^bfdg=\|f\|_{\max}\text{ and }TV(g)=1.  
    \]
    \item Let $[a,b]$ be a closed, bounded interval and $C[a,b]$ be normed by the maximum norm. Let $T$ be a bounded linear functional on $C[a,b]$.
    For $x\in[a,b]$, let $g_x$ be the member of $C[a,b]$ that is linear on $[a,x]$ and on $[x,b]$ with $g_x(a)=0,g_x(x)=x-a$ and $g_x(b)=x-a$. Define $\Phi(x)=T(g_x)$ for $x\in[a,b]$. Show that $\Phi$ is Lipschitz on $[a,b]$.
\end{enumerate}

% 8.2
\authoredby{inprogress}
\section{Weak Sequential Convergence in $L^p$}

The Bolzano-Weierstrass Theorem for the real numbers is the assertion that every bounded sequence of real numbers has a convergent subsequence.
This property immediately extends to bounded sequences in each Euclidean space $\mathbb{R}^n$.
This property fails in an infinite dimensional normed linear space (See Riesz's Theorem, Chapter 13.3).
The following example shows that for $1\le p\le\infty$, there are bounded sequences in $L^p[0,1]$ that fail to have any subsequences that converge in $L^p[0,1]$ (Radamacher functions).

\textbf{Example.}
For $I=[0,1]$ and a natural number $n$, consider the step function $f_n$ defined on $I$ by
\[
    f_n(x):=(-1)^k\quad\text{for}\quad k/2^n\le x<(k+1)/2^n\quad\text{where}\quad0\le k<2^n-1.
\]
Fix $1\le p\le\infty$.
It is clear that $\|f_n\|_p=1$ for each index $n$, and so the sequence is bounded in $L^p(I)$.
However, for $n\neq m$, then $|f_n-f_m|$ takes the value 2 on a set of measure 1/2.
Therefore for any index $N$, we may always find $m,n\ge N$ with $\|f_n-f_m\|_p\ge(2)^{1-1/p}$, which implies that no subsequence of $\{f_n\}$ is Cauchy in $L^p(I)$, and thus no subsequence can converge in $L^p(I)$.
Further, no subsequence can converge pointwise a.e. on $I$ since, for $1\le p<\infty$, if there were such a sequence, then by the Bounded Convergence Theorem it would converge in $L^p(I)$.

\begin{namedthm*}{Definition}
    Let $X$ be a normed linear space.
    A sequence $\{f_n\}$ in $X$ is said to \textbf{converge weakly} in $X$ to $f$ in $X$ provided
    \[
        \lim_{n\to\infty}T(f_n)=T(f)\quad\text{for all }T\in X^*.
    \]
    We write 
    \[
        \{f_n\}\rightharpoonup f\quad\text{in }X
    \]
    to mean that $f$ and each $f_n$ belong to $X$ and $\{f_n\}$ converges weakly in $X$ to $f$.
\end{namedthm*}
We continue to write $\{f_n\}\to f$ in $X$ to mean that $\lim_{n\to\infty}\|f_n-f\|=0$, and we refer to this as \textbf{strong convergence}.
Strong convergence is a stronger condition as weak convergence as
\[
    |T(f_n)-T(f)|=|T(f_n-f)|\le\|T\|_*\|f_n-f\|\quad\text{for all }T\in X^*.
\]
\begin{namedthm*}{Proposition 6}
    Let $E$ be a measurable set, $1\le p<\infty$, and $q$ the conjugate of $p$.
    Then $\{f_n\}\rightharpoonup f$ in $L^p(E)$ if and only if 
    \[
        \lim_{n\to\infty}\int_Eg\cdot f_n=\int_Eg\cdot f\quad\text{for all }g\in L^q(E).
    \]
\end{namedthm*}
\begin{proof}
    The Riesz Representation Theorem tells us that every bounded linear functional on $L^p(E)$ is given by integration against a function in $L^q(E)$.
\end{proof}

By the \textbf{linear span} of a subset $\mathcal{S}$ if a linear space $X$ we mean the linear space consisting of all linear combinations of functions in $\mathcal{S}$;
that is, the linear space of functions of the form 
\[
    f=\sum_{k=1}^n\alpha_k\cdot f_k,
\]
where each $\alpha_k$ is a real number and each $f_k$ belongs to $\mathcal{S}$.

\begin{center}
	\textbf{PROBLEMS}
\end{center}
\begin{enumerate}
	\setcounter{enumi}{11}
    \item f
\end{enumerate}

% 8.3
\section{Weak Sequential Compactness}
\begin{center}
	\textbf{PROBLEMS}
\end{center}
\begin{enumerate}
	\setcounter{enumi}{36}
    \item f
\end{enumerate}

% 8.4
\section{The Minimization of Convex Functionals}

\begin{namedthm*}{Definition}
    A subset $C$ of a linear space $X$ is said to be \textbf{convex} provided whenever $f$ and $g$ belong to $C$ and $\lambda\in[0,1]$, then $\lambda f+(1-\lambda)g$ also belongs to $C$.
\end{namedthm*}

\begin{namedthm*}{Definition}
    A subset $C$ of a normed linear space $X$ is said to be \textbf{closed} provided whenever $\{f_n\}$ is a sequence in $X$ that converges strongly in $X$ to $f$, then if each $f_n$ belongs to $C$, the limit $f$ also belongs to $C$.
\end{namedthm*}

\textbf{Example}
Let $E$ be a measurable set, $1\le p<\infty$, and $g$ nonnegative function in $L^p(E)$.
Define
\[
    C:=\{f\text{ measurable on }E: |f|\le g\text{ a.e. on }E\}.
\]
Then $C$ is a closed, convex subset of $L^p(E)$.

\textbf{Example}
Let $E$ be a measurable set and $1\le p<\infty$.
Then $B:=\{f\in L^p(E):\|f\|_p\le1\}$ is closed and convex.
Convexity follows from the minkowski (triangle) inequality and absolute homogeneity.
To see the closed property, let $\{f_n\}$ be a sequence in $B$ that converges in $L^p(E)$ to $f\in L^p(E)$.
The reverse triangle inequality gives us that $|\|f_n\|_p-\|f\|_p|\le\|f_n-f\|_p$, and so $\{\|f_n\|_p\}$ converges to $\|f\|_p$.
Therefore $\|f\|_p\le1$.

\begin{namedthm*}{Definition}
    A real-valued functional $T$ defined on a subset $C$ of a normed linear space $X$ is said to be \textbf{continuous} provided whenever a sequence $\{f_n\}$ in $C$ converges strongly to $f\in C$, then $\{T(f_n)\}\to T(f)$.    
\end{namedthm*}

\begin{center}
	\textbf{PROBLEMS}
\end{center}
\begin{enumerate}
	\setcounter{enumi}{37}
    \item For $1<p<\infty$ and each index $n$, let $e_n\in\ell^p$ have $n$th component 1 and the other components 0.
    Show that $\{e_n\}$ converges weakly to 0 in $\ell^p$, but no subsequence converges strongly to 0.
    Find a subsequence whose arithmetic means converge strongly to 0 in $\ell^p$.

    By the Riesz Representation Theorem for $\ell^p$, for any $\varphi\in(\ell^p)^*$, there exists $g\in\ell^q$ such that 
    \[
        \varphi(x)=\sum_{i=1}^\infty x(i)g(i)\quad\text{for all }x\in\ell^p.
    \]
    Consider any linear functional $\varphi_g$ represented by integration against $g\in\ell^q$.
    Because $g\in\ell^p$, then we have that $\sum_{i=1}^\infty|g(i)|^q=\|g\|_q^q<\infty$; that is, the series is summable.
    From Chapter 1.5 Proposition 20(i), then for each $\epsilon>0$, there is an index $N$ for which 
    \[
        |g(n)|^q<\epsilon^q\quad\text{for }n\ge N.
    \]
    Then because $\varphi_g(0)=0$, we have
    \[
        |\varphi_g(e_n)-\varphi_g(0)|=|\varphi_g(e_n)|=\left|\sum_{i=1}^\infty e_n(i)g(i)\right|=|g(n)|<\epsilon\quad\text{for }n\ge N,
    \]
    and so $\{e_n\}$ converges weakly to $0$.
    No subsequence of $\{e_n\}$ converges strongly to zero because for any $e_n$, we have that 
    \[
        \|e_n-0\|_p^p=\sum_{i=1}^\infty |e_n(i)|^p=|e_n(n)|^p=1.
    \]

\end{enumerate}