% Chapter 8
\authoredby{inprogress}
\chapter{The $L^p$ Spaces: Duality and Weak Convergence}

% 8.1
\authoredby{inprogress}
\section{The Riesz Representation for the Dual of $L^p,a\le p\le \infty$}

\textbf{Example}
Let $E$ be a measurable set, $1\le p<\infty$, $q$ the conjugate of $p$, and $g$ belong to $L^q(E)$.
Define the functional $T$ on $L^p(E)$ by
\[
    T(f)=\int_Eg\cdot f\text{ for all }f\in L^p(E).
\]
H\"older's Inequality tells us that for $f\in L^p(E)$, the product $g\cdot f$ is integrable over $E$ so the functional $T$ is properly defined.
By the linearity of integration, $T$ is linear.
Observe that H\"older's inequality is the statement that 
\[
    |T(f)|\le\|g\|_q\cdot\|f\|_p\text{ for all }f\in L^p(E).
\]
\begin{flushleft}

\textbf{Example}
Let $[a,b]$ be a closed, bounded interval and the function $g$ be of bounded variation on $[a,b]$.
Define the functional $T$ on $C[a,b]$ by
\[
    T(f)=\int_a^bf(x)dg(x)\text{ for all }f\in C[a,b],
\]  
where the integral is in the sense of Riemann-Stieltjes.
The functional $T$ is properly defined and linear.
Moreover, it follows immediately from the definition of this integral that 
\[
    |T(f)|\le TV(g)\cdot\|f\|_{\max}\text{ for all }f\in C[a,b],
\]  
where $TV(g)$ is the total variation of $g$ over $[a,b]$.
\end{flushleft}
\begin{namedthm*}{Definition}
    For a normed linear space $X$, a linear functional $T$ on $X$ is said to be \textbf{bounded} provided there is an $M\ge0$ for which 
\[
    |T(f)|\le M\|f\|\text{ for all }f\in X.
\]
The infimum of all such $M$ is called the \textbf{norm} of $T$ and denoted by $\|T\|_*$.
\end{namedthm*}
The inequalities in the first and second example above tell us that the linear functionals are bounded.

Let $T$ be a bounded linear functional on the normed linear space $X$.
It is easy to see that the above equation holds for $M=\|T\|_*$.
Hence, by the linearity of $T$,
\[
    |T(f)-T(h)|\le \|T\|_*\|f-h\|\text{ for all }f,h\in X.
\]
From this we infer the following continuity property of a bounded linear functional $T$:
\[
    \text{if }\{f_n\}\to f\text{ in }X,\text{ then }\{T(f_n)\}\to T(f).\tag{7}
\]
We leave it as an exercise (see chapter 13 problem 11) to show that 
\[
    \|T\|_*=\sup\{T(f)\mid f\in X,\|f\|\le1\},\tag{8}
\]
and use this characterization of $\|\cdot\|_*$ to prove the following proposition.
\begin{namedthm*}{Proposition 1}
    Let $X$ be a normed linear space.
    Then the collection of bounded linear functionals on $X$ is a linear space on which $\|\cdot\|_*$ is a norm.
    This normed linear space is called the \textbf{dual space} of $X$ and denoted by $X^*$.
\end{namedthm*}
\begin{namedthm*}{Proposition 2}
    Let $E$ be a measurable set, $1\le p<\infty$, $q$ the conjugate of $p$, and $g$ belong to $L^q(E)$.
    Define the functional $T$ on $L^p(E)$ by
    \[
        T(f)=\int_Eg\cdot f\text{ for all }f\in L^p(E).
    \]
    Then $T$ is a bounded linear functional on $L^p(E)$ and $\|T\|_*=\|g\|_q$.
\end{namedthm*}
\begin{proof}
    By linearity of integration, $T$ is linear, and from H\"older's inequality, we infer that $T$ is bounded and $\|T\|_*\le\|g\|_q$.
    For $p>1$, according to Chapter 7 Theorem 1, the conjugate function of $g$, $g^*=\|g\|_q^{1-q}\text{sgn}(g)|g|^{q-1}$, belongs to $L^p(E)$, and 
    \[
        T(g^*)=\int_Eg\cdot g^*=\|g\|_q\quad\text{and}\quad\|g^*\|_p=1.
    \] 
    Then by definition of the operator norm as a supremum (8), we have $\|T\|_*\ge T(g^*)=\|g\|_q$.
    Thus from the two inequalities we obtain $\|T\|_*=\|g\|_q$.
    For $p=1$, we argue by contradiction.

\end{proof}

\begin{center}
	\textbf{PROBLEMS}
\end{center}
\begin{enumerate}
	\setcounter{enumi}{0}
    \item Verify (8).
    
    \ \\(See Chapter 13 Problem 11) 
    Define 
    \[
        \begin{split}
        M'&:=\inf\{M\ge0\mid\|T(f)\|\le M\|f\|\text{ for all }f\in X\},\\
        N'&:=\sup\{\|T(f)\|\mid f\in X, \|f\|\le1\}.
        \end{split}
    \]
    We aim to show that they are equal.
    
    First, for $f\neq0$, then $\|\frac{f}{\|f\|}\|\le1$ so that by linearity of $T$,
    \[\|T(\frac{f}{\|f\|})\|\le N'\implies\|Tf\|\le N'\|f\|\implies M'\le N'.\]
    On the other hand, for $f$ such that $\|f\|\le1$, then
    \[\|Tf\|\le M'\|f\|\le M'\implies N'\le M.\]
    Therefore $M'=N'$.
    \\\item Prove Proposition 1.
    \item Let $T$ be a linear functional on a normed linear space $X$. Show that $T$ is bounded iff the continuity property (7) holds.
    \item A functional $T$ on a normed linear space $X$ is said to be Lipschitz provided there is a $c\ge0$ such that
    \[
        |T(g)-T(h)|\le c\|g-h\|\text{ for all }g,h\in X.  
    \]
    The infimum of such $c$'s is called the Lipschitz constant for $T$. Show that a linear functional is bounded iff it is Lipschitz, in which case its Lipschitz constant is $\|T\|_*$.
    \item Let $E$ be a measurable set and $1\le 0<\infty$. Show that the functions in $L^p(E)$ that vanish outside a bounded set are dense in $L^p(E)$. Show that this is false for $L^\infty(\mathbb{R})$.
    \item Establish the Riesz Representation Theorem in the case $p=1$ by first showing, in the notation of the proof of the theorem, that the function $\Phi$ is Lipschitz and therefore it is absolutely continuous. Then follow the $p>1$ proof.
    \item State and prove a Riesz Representation Theorem for the bounded linear functionals on $\ell^p$, $1\le p<\infty$.
    \item Let $c$ be the linear space of real sequences that converge to a real number and $c_0$ the subspace of $c$ comprising sequences that converge to $0$. Norm each of these linear spaces with the $\ell^\infty$ norm. Determine the dual space of $c$ and $c_0$.
    \item Let $[a,b]$ be a closed, bounded interval and $C[a,b]$ be normed by the maximum norm. Let $x_0$ belong to $[a,b]$. Define the linear functional $T$ on $C[a,b]$ by $T(f)=f(x_0)$. Show that $T$ is bounded and is given by Riemann-Stieltjes integration against a function of bounded variation.
    \item Let $f$ belong to $C[a,b]$. Show that there is a function $g$ that is of bounded variation on $[a,b]$ for which 
    \[
        \int_a^bfdg=\|f\|_{\max}\text{ and }TV(g)=1.  
    \]
    \item Let $[a,b]$ be a closed, bounded interval and $C[a,b]$ be normed by the maximum norm. Let $T$ be a bounded linear functional on $C[a,b]$.
    For $x\in[a,b]$, let $g_x$ be the member of $C[a,b]$ that is linear on $[a,x]$ and on $[x,b]$ with $g_x(a)=0,g_x(x)=x-a$ and $g_x(b)=x-a$. Define $\Phi(x)=T(g_x)$ for $x\in[a,b]$. Show that $\Phi$ is Lipschitz on $[a,b]$.
\end{enumerate}

% 8.2
\authoredby{untouched}
\section{Weak Sequential Convergence in $L^p$}
\begin{center}
	\textbf{PROBLEMS}
\end{center}
\begin{enumerate}
	\setcounter{enumi}{11}
    \item f
\end{enumerate}

% 8.3
\section{Weak Sequential Compactness}
\begin{center}
	\textbf{PROBLEMS}
\end{center}
\begin{enumerate}
	\setcounter{enumi}{36}
    \item f
\end{enumerate}

% 8.4
\section{The Minimization of Convex Functionals}
\begin{center}
	\textbf{PROBLEMS}
\end{center}
\begin{enumerate}
	\setcounter{enumi}{36}
    \item f
\end{enumerate}