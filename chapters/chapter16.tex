% Chapter 16
\chapter{Continuous Linear Operators on Hilbert Spaces}

% 16.1
\section{The Inner Product and Orthogonality}

\begin{namedthm*}{Definition}
    Let $H$ be a linear space.
    A function $\langle\cdot,\cdot\rangle:H\times H\to\mathbb{R}$ is called an \textbf{inner product} on $H$ provided for all $x_1,x_2,x,y\in X$ and real numbers $\alpha,\beta$, it satisfies
    \begin{enumerate}[(i)]
        \item linearity in first argument: $\langle \alpha x_1+\beta x_2,y\rangle=\alpha\langle x_1,y\rangle+\beta\langle x_2,y\rangle$.
        \item symmetry: $\langle x,y\rangle=\langle y,x\rangle$
        \item positive definiteness: $\langle x,x\rangle>0$ if $x\neq0$
    \end{enumerate}
    A linear space $H$ together with an inner product on $H$ is called an \textbf{inner product space}.
\end{namedthm*}
Property (i) along with property (ii) reveals that the real inner product in linear in both arguments: this is called \textbf{bilinearity}.

For two vectors $u=(u_1,\dots u_n)$ ad $u=(v_1,\dots v_n)$ in Euclidean space $\mathbb{R}^n$, the Euclidean inner product, $\langle u,v\rangle$, is defined by
\[
    \langle u,v\rangle=\sum_{k=1}^nu_kv_k.
\]
For two sequences $x=\{x_k\}$ ad $y=\{y_k\}$ in $\ell^2$, the $\ell^2$ inner product, $\langle x,y \rangle$, is defined by
\[
    \langle x,y\rangle=\sum_{k=1}^\infty x_ky_k.
\]
For a measurable set $E\subseteq\mathbb{R}$ and two functions $f$ and $g$ in $L^2(E)$, the $L^2$ inner product, $\langle f,g \rangle$, is defined by
\[
    \langle f,g\rangle=\int_Ef\cdot g.
\]
From the Cauchy-Schwarz (H\"older's) inequality, we infer that these inner products are properly defined (finite).
\begin{namedthm*}{The Cauchy-Schwarz Inequality}
    For any two vectors $u,v$ in an inner product space $H$,
    \[
        |\langle u,v\rangle|\le\|u\|\cdot\|v\|.
    \]
\end{namedthm*}
\begin{namedthm*}{Proposition 1}
    For a vector $h$ in an inner product space $H$, define
    \[
        \|h\|=\sqrt{\langle h,h\rangle}.
    \]
    Then $\|\cdot\|$ is a norm on $H$ called that norm induced by the inner product $\langle\cdot,\cdot\rangle$.
\end{namedthm*}
\begin{proof}
    Let $h\in H$.
    \begin{enumerate}[(i)]
        \item By positive definiteness of the inner product, $\langle h,h\rangle>0\implies \sqrt{\langle h,h\rangle}>0$.
        \\If $h=0$ then $\langle h,h\rangle=0\implies\sqrt{\langle h,h\rangle}=0$.
        \item By bilinearity of the inner product, $\|\alpha h\|=\sqrt{\langle \alpha h,\alpha h\rangle}=\sqrt{\alpha^2\langle h,h\rangle}=|\alpha|\sqrt{\langle h,h\rangle} =|\alpha|\|h\|$.
        \item Use the Cauchy-Schwarz inequality to see that 
        \[
            \|u+v\|^2=\langle u+v,u+v\rangle=\langle u,u\rangle+2\langle u,v\rangle+\langle v,v\rangle\le \|u\|^2+2\|u\|\|v\|+\|v\|^2=(\|u\|+\|v\|)^2.
        \]
    \end{enumerate}
\end{proof}
\begin{namedthm*}{The Parallelogram Identity}
    For any two vectors $u,v$ in an inner product space $H$,
    \[
        \|u-v\|^2+\|u+v\|^2=2\|u\|^2+2\|v\|^2.
    \]
\end{namedthm*}
To verify this we can simply add the following two equalities:
\begin{align*}
    \|u-v\|^2&=\|u\|^2-2\langle u,v\rangle+\|v\|^2,\\
    \|u+v\|^2&=\|u\|^2+2\langle u,v\rangle+\|v\|^2.
\end{align*}

\begin{namedthm*}{Definition}
    An inner product space $H$ is called a \textbf{Hilbert space} provided it is a Banach space with respect to the norm induced by the inner product.
\end{namedthm*}
The Riesz-Fischer Theorem tells us that for $E$ a measurable set of real numbers, $L^2(E)$ is a Hilbert space and, as a consequence, so is $\ell^2$.
\begin{namedthm*}{Proposition 2}
    Let $K$ be a nonempty, closed, convex subset of a Hilbert space $H$ and $h_0$ belong to $H\setminus K$.
    Then there is exactly on vector $h_*\in K$ that is closest to $h_0$ in the sense that
    \[
        \|h_0-h_*\|=\text{dist}(h_0,K)=\underset{h\in K}{\inf}\|h_0-h\|.
    \]
\end{namedthm*}
% \begin{proof}
%     d
% \end{proof}

\begin{center}
	\textbf{PROBLEMS}
\end{center}
In the following problems, $H$ is a Hilbert space.
\begin{enumerate}
	\setcounter{enumi}{0}
    \item Let $[a,b]$ be a closed, bounded interval of real numbers. Show that the $L^2[a,b]$ inner product is also an inner product on $C[a,b]$. Is $C[a,b]$ considered as an inner product space with the $L^2[a,b]$ inner product, a Hilbert space?\\
    \\We see that it is true that the $L^2[a,b]$ inner product is also an inner product on $C[a,b]$:
    \begin{enumerate}[(i)]
        \item 
        $
            \langle \alpha f_1+\beta f_2,g\rangle_{L^2}
            =\int_{[a,b]}( \alpha f_1+\beta f_2)
            %=\int_{[a,b]} \alpha f_1\cdot g+\beta f_2\cdot g
            %&=\int_{[a,b]} \alpha f_1\cdot g+\int_{[a,b]}\beta f_2\cdot g\\
            =\alpha\int_{[a,b]} f_1\cdot g+\beta\int_{[a,b]} f_2\cdot g
            =\alpha\langle f_1,g\rangle_{L^2}+\beta\langle f_2,g\rangle_{L^2}
        $
        \item $\langle f,g\rangle_{L^2}=\int_{[a,b]}f\cdot g=\int_{[a,b]}g\cdot f=\langle g,f\rangle_{L^2}$
        \item $\langle f,f\rangle_{L^2}=\int_{[a,b]}f^2\ge0$, with $\langle f,f\rangle_{L^2}=0\iff f=0$
    \end{enumerate}
    However, the inner product space $(C[a,b],\langle\cdot,\cdot\rangle_{L^2})$ is not a Hilbert space because it is not complete (not a Banach space) with respect to the norm defined by $\|\cdot\|_{L^2}:=\sqrt{\langle\cdot,\cdot\rangle_{L^2}}$.\\
    \\To see this, let $t\in(a,b)$ and consider a sequence of continuous functions $\{f_n\}$ defined on $[a,b]$ such that for each $n$,
    \[
        f_n(x):=
        \begin{cases}
            0 &x\le t\\
            n(x-t)&t<x<t+\frac{1}{n}\\
            1 &x\ge t+\frac{1}{n}
        \end{cases}
    \]
    We aim to show that this sequence is Cauchy but does not converge to a continuous function, and therefore the space is not complete.\\
    \\First we prove that this sequence is Cauchy:
    \\Fix $\epsilon>0$.
    \\Consider any natural numbers $m,n>\frac{1}{\epsilon^2}$, with $m\ge n$.
    \[
        (f_m-f_n)(x)=
        \begin{cases}
            0-0 &x\le t\\
            m(x-t)-n(x-t)&x\in(t,t+\frac{1}{m})\\
            1-n(x-t)&x\in[t+\frac{1}{m},t+\frac{1}{n})\\
            1-1 &x\ge t+\frac{1}{n}
        \end{cases}
        \le
        \begin{cases}
            0 &x\le t\\
            1&x\in(t,t+\frac{1}{m})\\
            1&x\in[t+\frac{1}{m},t+\frac{1}{n})\\
            0 &x\ge t+\frac{1}{n}
        \end{cases}
        :=g(x)
    \]
    By monotonicity of integration, we get
    \[
        \int_{[a,b]}|f_n-f_m|^2\le\int_{[a,b]}|g|^2=\int_{(t,t+\frac{1}{n})}1=m((t,t+\frac{1}{n}))=\frac{1}{n}.
    \]
    Therefore we see that
    \[
        \|f_n-f_m\|_{L^2}=\left(\int_{[a,b]}|f_n-f_m|^2\right)^{1/2}\le\left(\frac{1}{n}\right)^{1/2}<\epsilon,
    \]
    which implies that $\{f_n\}$ is Cauchy.\\
    \\Very similarly, we prove that this sequence converges to a discontinuous function:
    \\Define the discontinuous function $f:[a,b]\to\mathbb{R}$ by
    \[
        f(x):=
        \begin{cases}
            0 &x\le t\\
            1 &x> t
        \end{cases}
    \]
    \\Fix $\epsilon>0$.
    \\Consider any natural number $n>\frac{1}{\epsilon^2}$.
    \[
        (f-f_n)(x)=
        \begin{cases}
            0-0 &x\le t\\
            1-n(x-t)&x\in(t,t+\frac{1}{n})\\
            1-1 &x\ge t+\frac{1}{n}
        \end{cases}
        \le
        \begin{cases}
            0 &x\le t\\
            1&x\in(t,t+\frac{1}{n})\\
            0 &x\ge t+\frac{1}{n}
        \end{cases}
        :=g(x)
    \]
    By monotonicity of integration, we get
    \[
        \int_{[a,b]}|f-f_n|^2\le\int_{[a,b]}|g|^2=\int_{(t,t+\frac{1}{n})}1=m((t,t+\frac{1}{n}))=\frac{1}{n}.
    \]
    Therefore we see that
    \[
        \|f-f_n\|_{L^2}=\left(\int_{[a,b]}|f-f_n|^2\right)^{1/2}\le\left(\frac{1}{n}\right)^{1/2}<\epsilon,
    \]
    which implies that $\{f_n\}$ converges to $f$.
    \item Show that the maximum norm on $C[a,b]$ is not induced by an inner product and neither is the usual norm on $\ell^1$.\\
    \\See Problem 7 to see that for a normed linear space $(X,\|\cdot\|)$, the norm is induced by an inner product iff the parallelogram identity holds.
    Therefore it is sufficient to show that the parallelogram identity does not hold for the spaces $(C[a,b],\|\cdot\|_{\max})$ and $(\ell^1,\|\cdot\|_1)$.\\
    \\$(C[a,b],\|\cdot\|_{\max})$\\
    \\Let $f,g\in C[0,1]$ defined by $f(x)=x^2$, $g(x)=2$.
    \\Recall that $\|f\|_{\max}:=\max_{x\in[a,b]}|f(x)|$.
    \\Then $\|f-g\|_{\max}=2$, $\|f+g\|_{\max}=3$, $\|f\|_{\max}=1$, $\|g\|_{\max}=2$, so that 
    \[
        \|f-g\|_{\max}^2+\|f+g\|_{\max}^2=13\neq 6=2\|f\|_{\max}+2\|g\|_{\max}.
    \]
    $(\ell^2,\|\cdot\|_1)$\\
    \\Let $x=(1,0,0,\dots)\in\ell^1$ and $y=(0,1,0,\dots)\in\ell^1$.
    \\Recall that $\|x\|_1:=\sum_{i=1}^\infty|x_i|$.
    \\Then
    % \begin{align*}
    %     \|x-y\|_1=|1-0|+|0-1|=2\\
    %     \|x+y\|_1=|1+0|+|0+1|=2\\
    %     \|x\|_1=1\\
    %     \|y\|_1=1\\
    % \end{align*}
    $\|x-y\|_1=2,\ \|x+y\|_1=2,\ \|x\|_1=1,\ \|y\|_1=1$, so that
    \[
        \|x-y\|_2^2+\|x+y\|_2^2=8\neq 4=2\|x\|_2+2\|y\|_2.
    \]
    \item Let $H_1$ and $H_2$ be Hilbert spaces. Show that the Cartesian product $H_1\times H_2$ is also a Hilbert space with an inner product with respect to which $H_1\times\{0\}=[\{0\}\times H_2]^\perp$.
    \item Show that if $S$ is a subset of an inner product space $H$, then $S^\perp$ is a closed subspace of $H$.
    \item Let $S$ be a subset of $H$. Show that $S=(S^\perp)^\perp$ iff $S$ is a closed subspace of $H$.
    \item (Polarization Identity) Show that for any two vectors $u,v\in H$,
    \[
        \langle u,v \rangle = \frac{1}{4}[\|u+v\|^2-\|u-v\|^2].
    \]
    \\Recall the Parallelogram inequality and instead of adding, simply subtract the first inequality:
    \begin{align*}
        -\|u-v\|^2&=-\|u\|^2+2\langle u,v\rangle-\|v\|^2,\\
        \|u+v\|^2&=\|u\|^2+2\langle u,v\rangle+\|v\|^2,
    \end{align*}
    so that
    \[
        \|u+v\|^2-\|u-v\|^2=4\langle u,v\rangle.
    \]
    \item (Jordan-von Neumann) Let $X$ be a linear space normed by $\|\cdot\|$. Use the polarization identity to show that a norm $\|\cdot\|$ is induced by an inner product iff the parallelogram identity holds.\\
    \\Let $(X,\|\cdot\|)$ be a normed linear space.\\
    \\$(\implies)$ Suppose that the norm is induced by some inner product $\langle \cdot,\cdot\rangle$ on $X$.
    \\Then for any two vectors $u,v\in X$,
    \begin{align*}
        \|u-v\|^2&=\langle u-v,u-v\rangle=\langle u,u\rangle+\langle u,-v\rangle+\langle -v,u\rangle+\langle -v,-v\rangle=\|u\|^2-2\langle u,v\rangle+\|v\|^2,\\
        \|u+v\|^2&=\langle u+v,u+v\rangle=\langle u,u\rangle+\langle u,v\rangle+\langle v,u\rangle+\langle v,v\rangle=\|u\|^2+2\langle u,v\rangle+\|v\|^2,
    \end{align*}
    and adding the two equalities shows that the parallelogram identity holds.\\
    \\$(\impliedby)$ Suppose that the parallelogram identity holds.
    \\Define the function $\langle\cdot,\cdot\rangle:X\times X\to\mathbb{R}$ by
    \[
        \langle u,v \rangle = \frac{1}{4}[\|u+v\|^2-\|u-v\|^2].
    \]
    \\The aim is to show that $\sqrt{\langle v,v\rangle}=\|v\|$ for $v\in X$, and that $\langle\cdot,\cdot\rangle$ is an inner product.
    \\First see that
    \[
        \sqrt{\langle v,v \rangle} = \sqrt{\frac{1}{4}[\|v+v\|^2-\|v-v\|^2]}=\sqrt{\frac{2^2}{4}\|v\|^2}=\|v\|.\tag{1}
    \]
    (i) Bilinearity (additivity):
    % \\For any $x,y,v\in X$, by the parallelogram identity,
    % \begin{align*}
    %     2\|x+v\|^2+2\|y\|^2&=\|(x+v)-y\|^2+\|(x+v)+y\|^2\tag{a}\\
    %     2\|x-v\|^2+2\|y\|^2&=\|(x-v)-y\|^2+\|(x-v)+y\|^2\tag{b}
    % \end{align*}
    % Then, subtracting (b) from (a), we get
    % \begin{align*}
    %     2\|x+v\|^2-2\|x-v\|^2&=\|(x+v)+y\|^2-\|(x-v)+y\|^2+\|(x+v)-y\|^2-\|(x-v)-y\|^2\\
    %     2[\|x+v\|^2-\|x-v\|^2]&=[\|(x+y)+v\|^2-\|(x+y)-v\|^2]+[\|(x-y)+v\|^2-\|(x-y)-v\|^2]\\
    % %4\frac{1}{4}\left[\|(x-y)+v\|^2-\|(x-y)-v\|^2\right]+4\frac{1}{4}\left[\|(x+y)+v\|^2-\|(x+y)-v\|^2\right]&=2\cdot4\frac{1}{4}\left[\|x+v\|^2-\|x-v\|^2\right]\\
    %     8\langle x,v\rangle&=4\langle x+y,v\rangle+4\langle x-y,v\rangle\\
    %     2\langle x,v\rangle&=\langle x+y,v\rangle+\langle x-y,v\rangle
    % \end{align*}
    % Therefore, when we let $x=\frac{u_1+u_2}{2},\ y=\frac{u_1-u_2}{2},$
    % we get
    % \begin{align*}
    %     \langle u_1+u_2,v\rangle
    %     =2\langle \frac{u_1+u_2}{2},v\rangle
    %     =\langle \frac{u_1+u_2}{2}+\frac{u_1-u_2}{2},v\rangle+\langle \frac{u_1+u_2}{2}-\frac{u_1-u_2}{2},v\rangle
    %     =\langle u_1,v\rangle+\langle u_2,v\rangle.
    % \end{align*}
    \\For any $x,y,v\in X$, by the parallelogram identity,
    \begin{align*}
        \|(u_1+v)+u_2\|^2&=2\|u_1+v\|^2+2\|u_2\|^2-\|(u_1+v)-u_2\|^2\\
        \|(u_2+v)+u_1\|^2&=2\|u_2+v\|^2+2\|u_1\|^2-\|(u_2+v)-u_1\|^2
    \end{align*}
    Then
    \begin{align*}
        \|u_1+u_2+v\|^2&=\frac{1}{2}(\|(u_1+v)+u_2\|^2+\|(u_2+v)+u_1\|^2)\\
        &=\|u_1+v\|^2+\|u_2+v\|^2+\|u_1\|^2+\|u_2\|^2-\frac{1}{2}\|(u_1+v)-u_2\|^2-\frac{1}{2}\|(u_2+v)-u_1\|^2
    \end{align*}
    And so this also holds for $v=-v$:
    \begin{align*}
        \|u_1+u_2+(-v)\|^2&=\|u_1-v\|^2+\|u_2-v\|^2+\|u_1\|^2+\|u_2\|^2-\frac{1}{2}\|(u_1-v)-u_2\|^2-\frac{1}{2}\|(u_2-v)-u_1\|^2\\
        &=\|u_1-v\|^2+\|u_2-v\|^2+\|u_1\|^2+\|u_2\|^2-\frac{1}{2}\|(u_2+v)-u_1\|^2-\frac{1}{2}\|(u_1+v)-u_2\|^2
    \end{align*}
    Therefore we can write
    \begin{align*}
        \langle u_1+u_2,v\rangle&=\frac{1}{4}[\|u_1+u_2+v\|^2-\|u_1+u_2-v\|^2]\\
        &=\frac{1}{4}[\|u_1+v\|^2-\|u_1-v\|^2]+\frac{1}{4}[\|u_2+v\|^2-\|u_2-v\|^2]\\
        &=\langle u_1,v\rangle+\langle u_2,v\rangle.
    \end{align*}
    (i) Bilinearity (homogeneity):
    \\Let $S=\{\alpha\in\mathbb{R}\mid\alpha\langle u,v\rangle=\langle \alpha u,v\rangle\}$.
    \\Clearly $1,0\in S$, and $-1\in S$ because
    \[
        -1\langle u,v\rangle=\frac{1}{4}[\|u-v\|^2-\|u+v\|^2]=\frac{1}{4}[\|-u+v\|^2-\|-u-v\|^2]=\langle -1u,v\rangle.
    \]
    \\Suppose $\alpha,\beta\in S$.
    Then 
    \begin{align*}
        (\alpha+\beta)\langle u,v\rangle
        &=\alpha\langle u,v\rangle+\beta\langle u,v\rangle\\
        &=\langle \alpha u,v\rangle+\langle \beta u,v\rangle\\
        &=\langle \alpha u+\beta u,v\rangle&&\text{by bilinearity (additivity)}\\
        &=\langle (\alpha+\beta)u,v\rangle,
    \end{align*}
    so that $(\alpha+\beta)\in S$ and $S$ contains all integers.\\
    \\Suppose $\alpha,\beta\in S$, $\beta\neq0$.
    Then 
    \[
        \alpha\langle u,v\rangle=\langle \alpha u,v\rangle=\langle \frac{\beta}{\beta}\alpha u,v\rangle=\beta\langle \frac{\alpha}{\beta}u,v\rangle\implies \frac{\alpha}{\beta}\langle u,v\rangle=\langle\frac{\alpha}{\beta}u,v\rangle,
    \]
    so that $\frac{\alpha}{\beta}\in S$ and $S$ contains all rational numbers.\\
    \\Fix any $x,y\in X$.
    Consider the functions $f,g:\mathbb{R}\to\mathbb{R}$ defined by $f(\alpha)=\alpha\langle u,v\rangle$ and $g(\alpha)=\langle \alpha u,v\rangle$.
    The function $f$ is linear and thus continuous ($\ast$), and the function $g$ is a composition of $\alpha\mapsto\alpha x$ and $t\mapsto\langle t,y\rangle$, which are both continuous ($\ast\ast$).
    Then we have that $f,g$ are continuous with $f(\alpha)=g(\alpha)$ for all $\alpha\in\mathbb{Q}$, which implies that $f(\alpha)=g(\alpha)$ for all $\alpha\in\mathbb{R}$.
    \\That is, $\alpha\langle u,v\rangle=\langle \alpha u,v\rangle$ for all scalars $\alpha$.
    \\(ii) Symmetry:
    \[\langle u,v \rangle = \frac{1}{4}[\|u+v\|^2-\|u-v\|^2]=\frac{1}{4}[\|v+u\|^2-\|v-u\|^2]=\langle v,u\rangle.\]
    (iii) Positive Definiteness:
    \\For $v\neq0$: $\langle v,v \rangle = \|v\|^2>0$ by (1) and positive definiteness of norm.
    \\For $v=0$: $\langle v,v \rangle = \|v\|^2=0$ by (1) and positive definiteness of norm.\\
    \\Therefore $\langle\cdot,\cdot\rangle$ is an inner product.\\
    \\($\ast$) Linear functions are continuous:
    \\Let $H$ be a hilbert space, let $a\in H$, and let $f:H\to\mathbb{R}$ be a linear function:
    \[
        f(x):=\langle a,x\rangle\text{ for any }x\in X.
    \]
    Fix $\varepsilon>0$.
    \\Then there exists $\delta=\frac{\varepsilon}{\|a\|}>0$.
    \\For any $x,y\in H$ such that $\|x-y\|<\delta$, we use Cauchy-Schwarz inequality to see that
    \[
        |f(x)-f(y)|=|\langle a,x\rangle-\langle a,y\rangle|=|\langle a,x-y\rangle|\le\|a\|\cdot\|x-y\|<\varepsilon.
    \]
    \\($\ast\ast$) For a normed linear space $X$ and $y\in X$, the function $x\mapsto\|x+y\|$ for $x\in X$ is continuous.
    \\(Use this as a simplified version of $x\mapsto\langle x,y\rangle=\frac{1}{4}[\|x+y\|^2-\|x-y\|^2]$, which will also be continuous as it is the product and sum of continuous functions.)
    \\Fix $\varepsilon>0$.
    \\Let $\delta=\varepsilon>0$.
    \\For $x_1,x_2\in X$ such that $\|x_1-x_2\|<\delta=\varepsilon$, we use the reverse triangle inequality to see
    \[
        |f(x_1)-f(x_2)|=|\|x_1+y\|-\|x_2+y\||\le\|x_1+y-(x_2+y)\|<\varepsilon.
    \]
    \item Let $V$ be a closed subspace of $H$ and $P$ a projection of $H$ onto $V$. Show that $P$ is the orthogonal projection of $H$ onto $V$ iff (4) holds.
    \item Let $T$ belong to $\mathcal{L}(H)$. Show that $T$ is an isometry iff
    \[
        \langle T(u),T(v)\rangle=\langle u,v\rangle\text{ for all }u,v\in H.
    \]
    \\$(\implies)$ Suppose that $T$ is an isometry.
    \\Then $\|Tu\|=\|u\|$ for any $u\in H$.
    \\Let $u,v\in H$ so that we can derive:
    \begin{align*}
        \langle T(u),T(v)\rangle&=\frac{1}{4}[\|Tu+Tv\|^2-\|Tu-Tv\|^2]&&\text{Polarization identity}\\
        &=\frac{1}{4}[\|T(u+v)\|^2-\|T(u-v)\|^2]&&T\text{ is linear}\\
        &=\frac{1}{4}[\|u+v\|^2-\|u-v\|^2]&&T\text{ is an isometry}\\
        &=\frac{1}{4}[\langle u+v,u+v\rangle-\langle u-v,u-v\rangle]&&\text{norm induced by inner product}\\
        &=\frac{1}{4}[\|u\|^2+2\langle u,v\rangle+\|v\|^2-\|u\|^2+2\langle u,v\rangle-\|v\|^2]&&\text{norm induced by inner product}\\
        &=\langle u,v\rangle.
    \end{align*}
    \\$(\impliedby)$ Suppose that $\langle T(u),T(v)\rangle=\langle u,v\rangle\text{ for all }u,v\in H$.
    \\Therefore for any $u\in H$, we have 
    \[
        \|Tu\|=\sqrt{\langle Tu,Tu\rangle}=\sqrt{\langle u,u\rangle}=\|u\|.
    \]
    \item Let $V$ be a finite dimensional subspace of $H$ and $\varphi_1,\cdots,\varphi_n$ a basis for $V$ consisting of unit vectors, each pair of which is orthogonal. Show that the orthogonal projection $P$ of $H$ onto $V$ is given by 
    \[
        P(h)=\sum_{k=1}^n\langle h,\varphi_k\rangle\varphi_k\text{ for all }h\in V.
    \]
    \item For $h$ a vector in $H$, show that the function $u\mapsto\langle h,u\rangle$ belongs to $H^*$.\\
    \\The aim is to show that the function $\varphi:H\to\mathbb{R}$ defined by $\varphi(u)=\langle h,u\rangle$ is bounded and linear (see Chapter 8, Proposition 1).
    \\To see boundedness (continuity), use the Cauchy-Schwarz Inequality so that for any $u\in H$,
    \[
        \varphi(u)=\langle h,u\rangle\le\|h\|\cdot\|u\|<\infty.
    \]
    (Norms are defined to be \textbf{real-valued}; therefore the norm of any element in a linear space is a real number and thus finite: $\|h\|,\|u\|<\infty$)
    \\To see linearity, simply use bilinearity of the inner product:
    \[
        \varphi(\alpha u+\beta v)=\langle h,\alpha u+\beta v\rangle=\alpha\langle h,u\rangle+\beta\langle h,v\rangle=\alpha\varphi(u)+\beta\varphi(v).
    \]
    \item For any vector $h\in H$, show that there is a bounded linear functional $\psi\in H^*$ for which 
    \[
        \|\psi\|=1\text{ and }\psi(h)=\|h\|.  
    \]
    \\For $h\in H$ $(h\neq0)$, let $\psi:H\to\mathbb{R}$ be defined by 
    \[
        \psi(u)=\langle \frac{h}{\|h\|},u\rangle\text{ for any }u\in H.
    \]
    By the previous Problem 11, we proved that $\psi\in H^*$.
    \\Then 
    \[
        \psi(h)=\langle \frac{h}{\|h\|},h\rangle=\frac{1}{\|h\|}\langle h,h\rangle=\frac{1}{\|h\|}\|h\|^2=\|h\|.
    \]
    It remains to show $\|\psi\|_*=1$.
    \\Recall Chapter 8.1 for the following:
    \\We define the operator norm:
    \[
        \|\psi\|_*:=\inf\{M\mid|\psi(u)|\le M\|u\|\text{ for all }u\in H\},
    \]
    which implies the two: 
    \begin{itemize}
        \item $|\psi(u)|\le \|\psi\|_*\|u\|\text{ for all }u\in H$
        \item $\|\psi\|_*=\sup\{\psi(u)\mid u\in H, \|u\|\le1\}$
    \end{itemize}
    Therefore
    \[
        \|h\|=|\psi(h)|\le \|\psi\|_*\|h\|\implies 1\le \|\psi\|_*\tag{\text{a}}
    \]
    \[
        \|\psi\|_*=\underset{\|u\|\le1}{\underset{u\in H}{\sup}}\langle \frac{h}{\|h\|},u\rangle\le \|\frac{h}{\|h\|}\|\cdot\|u\|\le1\tag{\text{b}}
    \]
    Then (a) and (b) imply $\|\psi\|_*=1$.
    \item Let $V$ be a closed subspace of $H$ and $P$ the orthogonal projection of $H$ onto $V$.
    For any normed linear space $X$ and $T\in\mathcal{L}(V,X)$, show that $T\circ P$ belongs to $\mathcal{L}(H,X)$, and is an extension of $T:V\to X$ for which $\|T\circ P\|=\|T\|$.
    \item Prove the Hyperplane Separation Theorem for $H$, considered as a locally convex topological vector space with respect to the strong topology, by directly using Proposition 2.
    \item Use Proposition 2 to prove the Krein-Milman Lemma in a Hilbert space.
\end{enumerate}

% 16.2
\section{The Dual Space and Weak Sequential Convergence}
\begin{center}
	\textbf{PROBLEMS}
\end{center}
In the following problems, $H$ is a Hilbert space.
\begin{enumerate}
	\setcounter{enumi}{15}
    \item Show that neither $\ell^1,\ell^\infty,L^1[a,b]$ nor $L^\infty[a,b]$ is Hilbertable.
    \item Prove Proposition 7.
    \item Let $H$ be an inner product space. Show that since $H$ is a dense subset of a Banach space $X$ whose norm restricts to the norm induced by the inner product on $H$, the inner product on $H$ extends to $X$ and induces the norm on $X$.
    Thus inner product spaces have Hilbert space completions.
\end{enumerate}

% 16.3
\section{Bessel's Inequality and Orthonormal Bases}
\begin{center}
	\textbf{PROBLEMS}
\end{center}
In the following problems, $H$ is a Hilbert space.
\begin{enumerate}
	\setcounter{enumi}{18}
    \item Show that an orthonormal subset of a separable Hilbert space $H$ must be countable.
    \item Let $\{\varphi_k\}$ be an orthonormal sequence in a Hilbert space $H$. Show that $\{\varphi_k\}$ converges weakly to $0$ in $H$.
    \item Let $\{\varphi_k\}$ be an orthonormal basis for the separable Hilbert space $H$. Show that $\{u_n\}\to u$ in $H$ iff $\{u_n\}$ is bounded and, for each $k$, $\lim_{n\to\infty}\langle u_n,\varphi_k\rangle=\langle u,\varphi_k\rangle$.
    \item Show that any two infinite dimensional separable Hilbert spaces are isometrically isomorphic and that any such isomorphism preserves the inner product.
    \item Let $H$ be a Hilbert space and $V$ a closed separable subspace of $H$ for which $\{\varphi_k\}$ is an orthonormal basis.
    Show that the orthogonal projection of $H$ onto $V$, $P$, is given By
    \[
        P(h)=\sum_{k=1}^\infty \langle \varphi_k,h\rangle\varphi_k\text{ for all }h\in H.  
    \] 
    \item (Parseval's Indentities) Let $\{\varphi_k\}$ be an orthonormal basis for a Hilbert space $H$. Verify that
    \[
        \|h\|^2=\sum_{k=1}^\infty\langle \varphi_k,h\rangle^2\text{ for all }h\in H.
    \]
    Also verify that
    \[
        \langle u,v\rangle =  \sum_{k=1}^\infty a_k \cdot b_k\text{ for all }u,v\in H,
    \]
    where, for each natural number $k$, $a_k=\langle u,\varphi_k\rangle$ and $b_k=\langle v,\varphi_k\rangle$.
    \item Verify the assertions in the example of the orthonormal basis for $L^2[0,2\pi]$.
    \item Use Proposition 10 and the Stone-Weierstrass Theorem to show that for each $f\in L^2[-\pi,\pi]$,
    \[
        f(x)=a_0/2+\sum_{k=1}^\infty[a_k\cdot\cos kx+b_k\cdot\sin kx],  
    \]
    where the convergence is in $L^2[-\pi,\pi]$ and each 
    \[
        a_k=\frac{1}{\pi}\int_{-\pi}^\pi f(x)\cos kxdx\text{ and }b_k=\frac{1}{\pi}\int_{-\pi}^\pi f(x)\sin kxdx.
    \]
\end{enumerate}

% 16.4
\section{Adjoints and Symmetry for Linear Operators}
\begin{center}
	\textbf{PROBLEMS}
\end{center}
In the following problems, $H$ is a Hilbert space.
\begin{enumerate}
	\setcounter{enumi}{26}
    \item Verify (12).
    \item Let $T$ and $S$ belong to $\mathcal{L}(H)$ and be symmetric. 
    Show that $T=S$ iff $Q_T=Q_S$.
    \item Show the symmetric operators are a closed subspace of $\mathcal{L}(H)$.
    Also show that if $T$ and $S$ are symmetric, then so is the composition $S\circ T$ iff $T$ commutes with $S$ with respect to composition; that is, $S\circ T=T\circ S$.
    \item (Helliger-Toplitz) Let $H$ be a Hilbert space and the linear operator $T:H\to H$ have the property that $\langle T(u),v\rangle=\langle u,T(v)\rangle$ for all $u,v\in H$.
    Show that $T$ belongs to $\mathcal{L}(H)$.
    \item Exhibit an operator $T\in\mathcal{L}(\mathbb{R}^2)$ for which $\|T\|>\sup_{\|u\|=1}|\langle T(u),u\rangle|$.
    \item Let $S$ and $T$ in $\mathcal{L}(H)$ be symmetric.
    Assume $S\ge T$ and $T\ge S$.
    Prove that $T=S$.
    \item Let $V$ be a closed nontrivial subspace of a Hilbert space $H$ and $P$ the orthogonal projection of $H$ onto $V$.
    Show that $P=P^*$, $P\ge0$, and $\|P\|=1$.
    \item Let $P\in\mathcal{L}(H)$ be a projection.
    Show that $P$ is the orthogonal projection of $H$ onto $P(H)$ iff $P=P^*$.
    \item Let $\{\varphi_k\}$ be an orthonormal basis for a Hilbert space $H$ and for each natural number $n$, define $P_n$ to be the orthogonal projection of $H$ onto the linear span of $\{\varphi_1,\dots,\varphi_n\}$.
    Show that $P_n$ is symmetric and 
    \[
        0\le P_n\le P_{n+1}\le Id\text{ for all }n.
    \]
    Show that $\{P_n\}$ converges pointwise on $H$ to $Id$ but does not converge uniformly on the unit ball.
    \item Show that if $T\in\mathcal{L}(H)$ is invertible, so is $T^*\circ T$ and therefore so is $T$.
    \item (A General Cauchy-Schwarz Inequality) Let $T\in\mathcal{L}(H)$ be symmetric and nonnegative.
    Show that for all $u,v\in H$,
    \[
        |\langle T(u),v\rangle|^2\le\langle T(u),u\rangle\cdot\langle T(v),v\rangle.
    \] 
    \item Use the preceding problem to show that if $S,T\in\mathcal{L}(H)$ are symmetric and $S\ge T$, then for each $u\in H$,
    \[
        \|S(u)-T(u)\|^4=\langle (S-T)(u),(S-T)(u)\rangle^2\le|\langle(S-T)(u),u\rangle||\langle(S-T)^2(u),(S-T)(u)\rangle|
    \]
    and thereby conclude that
    \[
        \|S(u)-T(u)\|^4\le|\langle S(u),u\rangle-\langle T(u),u\rangle|\cdot\|S-T\|^3\cdot\|u\|^2.
    \]
    \item (a Monotone Convergence Theorem for Symmetric Operators) A sequence $\{T_n\}$ of symmetric operators in $\mathcal{L}(H)$ is said to be monotone increasing provided $T_{n+1}\ge T_n$ for each $n$, and said to be bounded above provided there is a symmetric operator $S$ in $\mathcal{L}(H)$ such that $T_n\le S$ for all $n$.
    \begin{enumerate}[(i)]
        \item Use the preceding problem to show that a monotone increasing sequence $\{T_n\}$ of symmetric operators in $\mathcal{L}(H)$ converges pointwise to a symmetric operator in $\mathcal{L}(H)$ iff it is bounded above.
        \item Show that a monotone increasing sequence $\{T_n\}$ of symmetric operators in $\mathcal{L}(H)$ is bounded above iff it is pointwise bounded; that is, for each $h\in H$, the sequence $\{T_n(h)\}$ is bounded.
    \end{enumerate}
    \item Let $S\in\mathcal{L}(H)$ be a symmetric operator for which $0\le S\le Id$.
    Define a sequence $\{T_n\}$ in $\mathcal{L}(H)$ by letting $T_1=1/2(Id-S)$ and if $n$ is a natural number for which $T_n\in\mathcal{L}(H)$ has been defined, defining $T_{n+1}=1/2(Id-S+T_n^2)$.
    \begin{enumerate}[(i)]
        \item Show that for each natural umber $n$, $T_n$ and $T_{n+1}-T_n$ are polynomials in $Id-S$ with nonnegative coefficients.
        \item Show that $\{T_n\}$ is a monotone increasing sequence of symmetric operators that is bounded above by $Id$.
        \item Use the preceding problem to show that $\{T_n\}$ converges pointwise to a symmetric operator $T$ for which $0\le T\le Id$ and $T=1/2(Id-S+T^2)$.
        \item Define $A=(Id-T)$. Show that $A^2=S$.
    \end{enumerate}
    \item (Square Roots of Nonnegative Symmetric Operators) Let $T\in\mathcal{L}(H)$ be a nonnegative symmetric operator.
    A nonnegative symmetric operator $A\in\mathcal{L}(H)$ is called a square root of $T$ provided $A^2=T$.
    Use the inductive construction in the preceding problem to show that $T$ has a square root $A$ which commutes with each operator in $\mathcal{L}(H)$ that commutes with $T$.
    Show that the square root is unique: it is denoted by $\sqrt{T}$.
    Finally, show that $T$ is invertible iff $\sqrt{T}$ is invertible.
    \item An invertible operator $T\in\mathcal{L}(H)$ is said to be \textbf{orthogonal} provided $T^{-1}=T^*$.
    Show that an invertible operator is orthogonal iff it is an isometry.
    \item (Polar Decompositions) Let $T\in\mathcal{L}(H)$ be invertible.
    Show that there is an orthogonal invertible operator $A\in\mathcal{L}(H)$ and a nonnegative symmetric invertible operator $B\in\mathcal{L}(H)$ such that $T=B\circ A$. 
    (Hint: show that $TT^*$ is invertible and symmetric and let $B=\sqrt{T\circ T^*}$.)
\end{enumerate}

% 16.5
\section{Compact Operators}
\begin{center}
	\textbf{PROBLEMS}
\end{center}
\begin{enumerate}
	\setcounter{enumi}{43}
    \item Show that if $H$ is infinite dimensional and $T\in\mathcal{L}(H)$ is invertible, then $T$ is not compact.
    \item Prove Proposition 18.
    \item Let $\mathcal{K}(H)$ denote the set of compact operators in $\mathcal{L}(H)$.
    Show that $\mathcal{K}(H)$ is a linear subspace of $\mathcal{L}(H)$.
    Moreover, show that for $K\in\mathcal{K}(H)$ and $T\in\mathcal{L}(H)$, both $K\circ T$ and $T\circ K$ belong to $\mathcal{K}(H)$.
    \item Show that a linear operator $T:H\to H$ is continuous iff it maps weakly convergent sequences to weakly convergent sequences.
    \item Show that $K\in\mathcal{L}(H)$ is compact iff whenever $\{u_n\}\to u$ in $H$ and $\{v_n\}\to v$ in $H$, then $\langle K(u_n),v_n\rangle\to\langle K(u),v\rangle$.
    \item Let $\{P_n\}$ be a sequence of orthogonal projections in $\mathcal{L}(H)$ with the property that for natural numbers $n$ and $m$, $P_n(H)$ and $P_m(H)$ are orthogonal finite dimensional subspaces of $H$.
    Let $\{\lambda_n\}$ be a bounded sequence of real numbers.
    Show that 
    \[
        K=\sum_{n=1}^\infty\lambda_n\cdot P_n
    \]
    is a properly defined symmetric operator in $\mathcal{L}(H)$ that is compact iff $\{\lambda_n\}$ converges to $0$.
    \item For $X$ a Banach space, define an operator $T\in\mathcal{L}(X)$ to be compact provided $T(B)$ has compact closure.
    Show that Proposition 18 holds for a general Banach space and Proposition 19 holds for a reflexive Banach space.
\end{enumerate}

% 16.6
\section{The Hilbert-Schmidt Theorem}
\begin{center}
	\textbf{PROBLEMS}
\end{center}
\begin{enumerate}
	\setcounter{enumi}{50}
    \item Let $H$ be a Hilbert space and $T\in\mathcal{L}(H)$ be compact and symmetric. Define
    \[
        \alpha=\underset{\|h\|=1}{\inf}\langle T(h),h\rangle\ \text{ and }\ \beta=\underset{\|h\|=1}{\sup}\langle T(h),h\rangle.
    \]
    Show that if $\alpha<0$, then $\alpha$ is an eigenvalue of $T$ and if $\beta>0$, then $\beta$ is an eigenvalue of $T$.
    Exhibit an example where $\alpha=0$ and yet $\alpha$ is not an eigenvalue of $T$; that is, $T$ is one-to-one (injective).
    \item Let $H$ be a Hilbert space and $K\in\mathcal{L}(H)$ be compact and symmetric.
    Suppose
    \[
        \underset{\|h\|=1}{\sup}\langle K(h),h\rangle=\beta>0.
    \] 
    Let $\{h_n\}$ be a sequence of unit vectors for which $\lim_{n\to\infty}\langle K(h_n),h_n\rangle=\beta$.
    Show that a subsequence of $\{h_n\}$ converges strongly to an eigenvector of $T$ with corresponding eigenvalue $\beta$.
\end{enumerate}

% 16.7
\section{The Riesz-Schauder Theorem: Characterization of Fredholm Operators}
\begin{center}
	\textbf{PROBLEMS}
\end{center}
\begin{enumerate}
	\setcounter{enumi}{52}
    \item Let $K\in\mathcal{L}(H)$ be compact.
    Show that $T=K^*K$ is compact and symmetric.
    Then use the Hilbert-Schmidt Theorem to show that there is an orthonormal sequence $\{\varphi_k\}$ of $H$ such that $T(\varphi_k)-\lambda_k\varphi_k$ for all $k$ and $T(h)=0$ if $h$ is orthogonal to $\{\varphi_k\}_{k=0}^\infty$.
    Conclude that if $h$ is orthogonal to $\{\varphi_k\}_{k=0}^\infty$, then
    \[
        \|K(h)\|^2\langle K(h),K(h)\rangle=\langle T(h),h\rangle=0.
    \]
    Define $H_0$ to be the closed linear span of $\{K^m(\varphi_k)\mid m\ge0,k\ge1\}$.
    Show that $H_0$ is closed and separable, $K(H_0)\subseteq H_0$ and $K=0$ on $H_0^\perp$.
    \item Let $\mathcal{K}(H)$ denote the set of compact operators in $\mathcal{L}(H)$.
    Show that $\mathcal{K}(H)$ is a closed subspace of $\mathcal{L}(H)$ that has the set of operators of finite rank as a dense subspace.
    Is $\mathcal{K}(H)$ an open subset of $\mathcal{L}(H)$?
    \item Show that the composition in either order of a Fredholm operator of index $0$ with an invertible operator is also Fredholm of index $0$.
    \item Show that the composition of two Fredholm operators of index $0$ is also Fredholm of index $0$.
    \item Show that an operator $T\in\mathcal{L}(H)$ is Fredholm of index $0$ iff it is the perturbation of an invertible operator by an operator of finite rank.
    \item Argue as follows to show that the collection of invertible operators in $\mathcal{L}(H)$ is an open subset of $\mathcal{L}(H)$.
    \begin{enumerate}[(i)]
        \item For $A\in\mathcal{L}(H)$ with $\|A\|<1$, use the completeness of $\mathcal{L}(H)$ to show that the so-called Neumann series $\sum_{n=0}^\infty A^n$ converges to an operator in $\mathcal{L}(H)$ that is the inverse of $Id-A$.
        \item For a invertible operator $S\in\mathcal{L}(H)$ show that for any $T\in\mathcal{L}(H)$, $T=S[Id+S^{-1}(T-S)]$.
        \item Use (i) and (ii) to show that if $S\in\mathcal{L}(H)$ is invertible then so is any $T\in\mathcal{L}(H)$ of which $\|S-T\|<1/\|S^{-1}\|$.
    \end{enumerate}
    \item Show that the set of operators in $\mathcal{L}(H)$ that are Fredholm of index $0$ is an open subset of $\mathcal{L}(H)$.
    \item By following the orthogonal approximation sequence method used in the proof of Proposition 22, provide another proof of Proposition 14 in case $H$ is separable.
    \item For $T\in\mathcal{L}(H)$, suppose that $\langle T(h),h\rangle\ge\|h\|^2$ for all $h\in H$.
    Assume that $K\in\mathcal{L}(H)$ is compact and $T+K$ is one-to-one.
    Show that $T+K$ is onto.
    \item Let $K\in\mathcal{L}(H)$ be compact an $\mu\in\mathbb{R}$ have $|\mu|>\|K\|$.
    Show that $\mu- K$ is invertible.
    \item Let $S\in\mathcal{L}(H)$ have $\|S\|<1$,$K\in\mathcal{L}(H)$ be compact and $(Id+S+K)(H)=H$.
    Show that $Id+S+K$ is one-to-one.
    \item Let $\mathcal{G}L(H)$ denote the set of invertible operators in $\mathcal{L}(H)$.
    \begin{enumerate}[(i)]
        \item Show that under the operation of composition of operators, $\mathcal{G}L(H)$ is a group: it is called the general linear group of $H$.
        \item An operator $T$ in $\mathcal{G}L(H)$ is said to be orthogonal, provided that $T^*=T^{-1}$.
        Show that the set of orthogonal operators is a subgroup of $\mathcal{G}L(H)$: it is called the orthogonal group.
    \end{enumerate} 
    \item Let $H$ be a Hilbert space, $T\in\mathcal{L}(H)$ be Fredholm of index zero, and $K\in\mathcal{L}(H)$ be compact.
    Show that $T+K$ is Fredholm of index zero.
    \item Let $X_0$ be a finite codimensional subspace of a Banach space $X$.
    Show that all finite dimensional linear complements of $X_0$ in $X$ have the same dimension.
\end{enumerate}
