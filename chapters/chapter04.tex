% Chapter 4
\chapter{Lebesgue Integration}
% 4.1
\section{The Riemann Integral}

\begin{center}
	\textbf{PROBLEMS}
\end{center}
\begin{enumerate}
	\setcounter{enumi}{0}
    \item Show that, in the above Dirichlet function example, $\{f_n\}$ fails to converge to $f$ uniformly on $[0,1]$.
    \item A partition $P'$ of $[a,b]$ is called a refinement of a partition $P$ provided each partition point of $P$ is also a partition point of $P'$.
    For a bounded function $f$ on $[a,b]$, show that under refinement lower Darboux sums increase and upper Darboux sums decrease.
    \item Use the preceding problem to show that for a bounded function on a closed, bounded interval, each lower Darboux sum is no greater than each upper Darboux sum.
    From this conclude that the lower Riemann integral is no greater than the upper Riemann integral.
    \item Suppose the bounded function $f$ on $[a,b]$ is Riemann integrable over $[a,b]$.
    Show that there is a sequence $\{P_n\}$ of partitions of $[a,b]$ for which $\lim_{n\to\infty}[U(f,P_n)-L(f,P_n)]=0$.
    \item Let $f$ be a bounded function on $[a,b]$. Suppose there is a sequence $\{P_n\}$ of partitions of $[a,b]$ for which $\lim_{n\to\infty}[U(f,P_n)-L(f,P_n)]=0$. Show that $f$ is Riemann integrable over $[a,b]$.
    \item Use the preceding problem to show that since a continuous function $f$ on a closed, bounded interval $[a,b]$ is uniformly continuous on $[a,b]$, it is Riemann integrable over $[a,b]$.
    \item Let $f$ be an increasing real-valued function on $[0,1]$. For a natural number $n$, define $P_n$ to be the partition of $[0,1]$ into $n$ subintervals of length $1/n$. 
    Show that $U(f,P_n)-L(f,P_n)\le 1/n[f(1)-f(0)]$. Use Problem 5 to show that $f$ is Riemann integrable over $[0,1]$.
    \item Let $\{f_n\}$ be a sequence of bounded functions that converges uniformly to $f$ on the closed, bounded interval $[a,b]$. 
    If each $f_n$ is Riemann integrable over $[a,b]$, show that $f$ also is Riemann integrable over $[a,b]$. Is it true that
    \[
        \lim_{n\to\infty}\int_a^bf_n=\int_a^bf?  
    \]
\end{enumerate}

% 4.2
\section{The Lebesgue Integral of a Bounded Measurable Function over a Set of Finite Measure}
\begin{flushleft}
    
\begin{namedthm*}{Remark}
    Prior to the proof of the Bounded Convergence Theorem, no use was made of the countable additivity of the Lebesgue measure on the real line.
    Only finite additivity was used, and it was used just once, in the proof of Lemma 1. Bur for the proof of the Bounded Convergence Theorem we used Egoroff's Theorem.
    Egoroff's Theorem needed the continuity of Lebesgue measure, a consequence of countable additivity of Lebesgue measure.    
\end{namedthm*}

\end{flushleft}
\begin{center}
	\textbf{PROBLEMS}
\end{center}
\begin{enumerate}
	\setcounter{enumi}{8}
    \item Let $E$ have measure zero. Show that if $f$ is a bounded function on $E$, then $f$ is measurable and $\int_Ef=0$.
    \item Let $f$ be a bounded measurable function on a set of finite measure $E$. For a measurable subset $A$ of $E$, show that $\int_Af=\int_Ef\cdot\chi_A$.
    \item Does the Bounded Convergence Theorem hold for the Riemann integral?
    \item Let $f$ be a bounded measurable function on a set of finite measure $E$. Assume $g$ is bounded and $f=g$ a.e. on $E$. Show that $\int_Ef=\int_Eg$.
    \item Does the Bounded Convergence Theorem hold if $m(E)<\infty$ but we drop the assumption that the sequence $\{|f_n|\}$ is uniformly bounded on $E$?
    \item Show that Proposition 8 is a special case of the Bounded Convergence Theorem.
    \item Verify the assertions in the last Remark of this section.
    \item Let $f$ be a nonnegative bounded measurable function on a set of finite measure $E$. Assume $\int_Ef=0$. Show that $f=0$ a.e. on $E$.
\end{enumerate}

% 4.3
\section{The Lebesgue Integral of a Measurable Nonnegative Function}

\begin{center}
	\textbf{PROBLEMS}
\end{center}
\begin{enumerate}
	\setcounter{enumi}{16}
    \item Let $E$ be a set of measure zero and define $f\equiv\infty$ on $E$. Show that $\int_Ef=0$.
    \item Show that the integral of a bounded measurable function of finite support is properly defined.
    \item For a number $\alpha$, define $f(x)=x^\alpha$ for $0<x\le1$, and $f(0)=0$. Compute $\int_0^1f$.
    \item Let $\{f_n\}$ be a sequence of nonnegative measurable functions that converges to $f$ pointwise on $E$.
    Let $M\ge0$ be such that $\int_Ef_n\le M$ for all $n$. Show that $\int_Ef\le M$. Verify that this property is equivalent to the statement of Fatou's Lemma.
    \item Let the function $f$ be nonnegative and integrable over $E$ and $\epsilon>0$. Show there is a simple function $\eta$ on $E$ that has finite support, $0\le\eta\le f$ on $E$ and $\int_E|f-\eta|<\epsilon$.
    If $E$ is a closed, bounded interval, show there is a step function $h$ on $E$ that has finite support and $\int_E|f-h|<\epsilon$.
    \item Let $\{f_n\}$ be a sequence of nonnegative measurable functions on $\mathbb{R}$ that converges pointwise on $\mathbb{R}$ to $f$ and $f$ be integrable over $\mathbb{R}$. Show that
    \[
        \text{if }\int_{\mathbb{R}}f=\lim_{n\to\infty}\int_{\mathbb{R}}f_n,\text{ then }\int_Ef=\lim_{n\to\infty}\int_Ef_n\text{ for any measurable set }E.    
    \]
    \item Let $\{a_n\}$ be a sequence of nonnegative real numbers. Define the function $f$ on $E=[1,\infty)$ by setting $f(x)=a_n$ if $n\le x<n+1$. Show that $\int_Ef=\sum_{n=1}^\infty a_n$.
    \item Let $f$ be a nonnegative measurable function on $E$.
    \begin{enumerate}[label=(\roman*),align=left]
        \item Show there is an increasing sequence $\{\varphi_n\}$ of nonnegative simple functions on $E$, each of finite support, which converges pointwise on $E$ to $f$.
        \item Show that $\int_Ef=\sup\{\int_E\varphi\ |\ \varphi\text{ simple, of finite support and }0\le\varphi\le f\text{ on }E\}$.
    \end{enumerate}
    \item Let $\{f_n\}$ be a sequence of nonnegative measurable functions on $E$ that converges pointwise on $E$ to $f$. Suppose $f_n\le f$ on $E$ for each $n$. Show that
    \[
        \lim_{n\to\infty}\int_Ef_n=\int_Ef.
    \]
    \item Show that the Monotone Convergence Theorem may not hold for decreasing sequences of functions.
    \item Prove the following generalization of Fatou's Lemma: If $\{f_n\}$ is a sequence of nonnegative measurable functions on $E$, then 
    \[
    \int_E\lim\inf f_n\le\lim\inf\int_Ef_n.     
    \]
\end{enumerate}
    
% 4.4
\section{The General Lebesgue Integral}
\begin{center}
	\textbf{PROBLEMS}
\end{center}
\begin{enumerate}
	\setcounter{enumi}{27}
    \item Let $f$ be integrable over $E$ and let $C$ be a measurable subset of $E$. Show that $\int_Cf=\int_Ef\cdot\chi_C$.
    \item For a measurable function $f$ on $[1,\infty)$ which is bounded on bounded sets, define $a_n=\int_n^{n+1}f$ for each natural number $n$.
    Is it true that $f$ is integrable over $[1,\infty)$ iff the series $\sum_{n=1}^\infty a_n$ converges?
    Is it true that $f$ is integrable over $[1,\infty)$ iff the series $\sum_{n=1}^\infty a_n$ converges absolutely?
    \item Let $g$ be a nonnegative integrable function over $E$ and suppose $\{f_n\}$ is a sequence of measurable functions on $E$ such that for each $n$, $|f_n|\le g$ a.e. on $E$. Show that
    \[
        \int_E\lim\inf f_n \le \lim\inf\int_E f_n \le \lim\sup\int_E f_n \le \int_E\lim\sup f_n.
    \]
    \item Let $f$ be a measurable function on $E$ which can be expressed as $f=g+h$ on $E$, where $g$ is finite and integrable over $E$ and $h$ is nonnegative on $E$.
    Define $\int_Ef=\int_Eg+\int_Eh$. Show that this is properly defined in the sense that it is independent of the particular choice of finite integrable function $g$ and nonnegative function $h$ whose sum is $f$.
    \item Prove the General Lebesgue Dominated Convergence Theorem by following the proof of the Lebesgue Dominated Convergence Theorem, but replacing the sequences $\{g-f_n\}$ and $\{g+f_n\}$, respectively, by $\{g_n-f_n\}$ and $\{g_n+f_n\}$.
    \item Let $\{f_n\}$ be a sequence of integrable functions on $E$ for which $f_n\to f$ a.e. on $E$ and $f$ is integrable over $E$. Show that $\int_E|f-f_n|\to0$ iff $\lim_{n\to\infty}\int_E|f_n|=\int_E|f|$.
    (Hint: use the General Lebesgue Dominated Convergence Theorem.)
    \item Let $f$ be a nonnegative measurable function on $\mathbb{R}$. Show that 
    \[
        \lim_{n\to\infty}\int_{-n}^nf=\int_{\mathbb{R}}f.
    \]   
    \item Let $f$ be a real-valued function of two variables $(x,y)$ that is defined on the square $Q=\{(x,y)\ |\ 0\le x\le 1,0\le y\le 1\}$ and is a measurable function of $x$ for each fixed value of $y$.
    Suppose for each fixed value of $x$, $\lim_{y\to0}f(x,y)=f(x)$ and that for all $y$, we have $|f(x,y)|\le g(x)$, where $g$ is integrable over $[0,1]$. Show that
    \[
    \lim_{y\to0}\int_0^1f(x,y)dx=\int_0^1f(x)dx.    
    \] 
    Also show that if the function $f(x,y)$ is continuous in $y$ for each $x$, then 
    \[
        h(y)=\int_0^1f(x,y)dx  
    \]
    is a continuous function of $y$.
    \item Let $f$ be a real-valued function of two variables $(x,y)$ that is defined on the square $Q=\{(x,y)\ |\ 0\le x\le 1,0\le y\le 1\}$ and is a measurable function of $x$ for each fixed value of $y$.
    For each $(x,y)\in Q$ let the partial derivative $\partial f/\partial y$ exist. Suppose there is a function $g$ that is integrable over $[0,1]$ and such that 
    \[
        \biggl | \frac{\partial f}{\partial y}(x,y) \biggr | \le g(x)\text{ for all }(x,y)\in Q.
    \]
    Prove that 
    \[
        \frac{d}{dy}\biggl[\int_0^1f(x,y)dx\biggr]=\int_0^1\frac{\partial f}{\partial y}(x,y)dx\text{ for all }y\in [0,1].
    \]
\end{enumerate}

% 4.5
\section{Countable Additivity and Continuity of Integration}
\begin{center}
	\textbf{PROBLEMS}
\end{center}
\begin{enumerate}
	\setcounter{enumi}{36}
    \item Let $f$ be an integrable function on $E$. Show that for each $\epsilon>0$, there is a natural number $N$ for which if $n\ge N$, then $|int_{E_n}f|<\epsilon$ where $E_n=\{x\in E\ |\ |x|\ge n\}$.
    \item For each of the two functions $f$ on $[1,\infty)$ defined below, show that $\lim_{n\to\infty}\int_1^nf$ exists while $f$ is not integrable over $[1,\infty)$. Does this contradict the continuity of integration?
    \begin{enumerate}[label=(\roman*),align=left]
        \item Define $f(x)=\frac{(-1)^n}{n}$, for $n\le x < n+1$.
        \item Define $f(x) = \frac{(\sin x)}{x}$ for $1\le x<\infty$.
    \end{enumerate}
    \item Prove the theorem regarding the continuity of integration.
\end{enumerate}

% 4.6
\section{Uniform Integrability: The Vitali Convergence Theorem}
\begin{center}
	\textbf{PROBLEMS}
\end{center}
\begin{enumerate}
	\setcounter{enumi}{39}
    \item Let $f$ be integrable over $\mathbb{R}$. Show that the function $F$ defined by 
    \[
        F(x) = \int_{-\infty}^xf\text{ for all }x\in\mathbb{R}
    \]
    is properly defined and continuous. Is it necessarily Lipschitz?
    \item Show that Proposition 25 is false if $E=\mathbb{R}$.
    \item Show that Theorem 26 is false without the assumption that the $h_n$'s are nonnegative.
    \item Let the sequences of functions $\{h_n\}$ and $\{g_n\}$ be uniformly integrable over $E$. Show that for any $\alpha$ and $\beta$, the sequence of linear combinations $\{\alpha f_n + \beta g_n\}$ also is uniformly integrable over $E$.
    \item Let $f$ be integrable over $\mathbb{R}$ and let $\epsilon>0$. Establish the following three approximation properties. 
    \begin{enumerate}[label=(\roman*),align=left]
        \item There is a simple function $\eta$ on $\mathbb{R}$ which has finite support and $\int_{\mathbb{R}}|f-\eta|<\epsilon$. (Hint: first verify this if $f$ is nonnegative.)
        \item There is a step function $s$ on $\mathbb{R}$ which vanishes outside a closed, bounded interval and $\int_{\mathbb{R}}|f-s|<\epsilon$. (Hint: apply part (i) and Problem 18 of Chapter 3.)
        \item There is a continuous function $g$ on $\mathbb{R}$ which vanishes outside a bounded set and $\int_{\mathbb{R}}|f-g|<\epsilon$.
    \end{enumerate}
    \item Let $f$ be integrable over $E$. Define $\hat f$ to be the extension of $f$ to all of $\mathbb{R}$ obtained by setting $\hat f \equiv 0$ outside of $E$. 
    Show that $\hat f$ is integrable over $\mathbb{R}$ and 
\end{enumerate}