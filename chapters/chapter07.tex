% Chapter 7
\chapter{The $L^p$ Spaces: Completeness and Approximation}

% 7.1
\section{Normed Linear Spaces}
\begin{flushleft}
	
	Throughout this chapter $E$ denotes a measurable set of real numbers.
	Define $\mathcal{F}$ to be the collection of all measurable extended real-valued functions on $E$ that are finite a.e. on $E$.
	We can say that two functions $f,g\in\mathcal{F}$ are equivalent, denoted by $f\cong g$, provided
	\[
	f(x)=g(x)\text{ for almost all }x\in E.	
	\]
	This is an equivalence relation and induces a partition of $\mathcal{F}$ into a disjoint collection of equivalence classes, denoted by $\mathcal{F}/\cong$, which is a linear space.
	There is a natural family $\{L^p(E)\}_{1\le p\le\infty}$ of subspaces of $\mathcal{F}/\cong$.
	\\
	For $1\le p<\infty$, define $L^p(E)$ to be the collection of equivalence class $[f]$ for which 
	\[
	\int_E|f|^p<\infty.	
	\]
	Then if $f\cong g$, then $\int_E|f|^p=\int_E|g|^p$.
	Showing that $L^p(E)$ is closed under linear combinations will prove that $L^p(E)$ is a linear subspace.
	To do this, let $c=\max\{|a|,|b|\}$ so that
	\[
	|a+b|\le|a|+|b|\le 2c,
	\]
	which implies
	\[
	|a+b|^p\le 2^pc^p\le2^p(|a|^p+|b|^p).	
	\]
	This inequality, together with the linearity and monotonicity of integration tells us that 
	\[
		\int_E|\alpha f+\beta g|^p \le 2^p(|\alpha|^p\int_E|f|^p+|\beta|^p\int_E| g|^p)<\infty.
	\]
	That is, for $[f],[g]\in L^p(E)$, then $\alpha[f]+\beta[g]\in L^p(E)$.\\
	We call a function $f\in\mathcal{F}$ \textbf{essentially bounded} provided there is some $M\ge 0$, called an \textbf{essential upper bound} for $f$, for which
	\[
	|f(x)|\le M\text{ for almost all }x\in E.	
	\]
	Then we can define $L^\infty(E)$ to be the collection of equivalence classes $[f]$ for which $f$ is essentially bounded. 
	Clearly $L^\infty(E)$ is a linear subspace because
	\[
	|\alpha f(x)+\beta g(x)|\le|\alpha| |f(x)|+|\beta|| g(x)|\le |\alpha| M + |\beta|M' = M'' \text{ a.e. on $E$}
	\]
	To state that a function $f$ in $L^p[a,b]$ is continuous means that there is a continuous function that agrees with $f$ a.e. on $[a,b]$.
	There is only one such continuous function and it is often convenient to consider this unique continuous function as the representative of $[f]$.\\
	It is useful to consider real-valued functions that have as their domain linear spaces of functions: such functions are called \textbf{functionals}.

	\begin{namedthm*}{Definition}
		Let $X$ be a linear space.
		A real-valued functional $\|\cdot\|$  on $X$ is called a \textbf{norm} provided for each $f$ and $g$ in $X$ and each real number $\alpha$,\\
		(The Triangle Inequality)
		\[
			\|f+g\|\le\|f\|+\|g\|,
		\]
		(Positive Homogeneity)
		\[
			\|\alpha f\|=|\alpha|\|f\|,	
		\]
		(Nonnegativity)
		\[
			\|f\|\ge 0\text{ and }\|f\|=0 \iff f=0.
		\]
	\end{namedthm*}
	A \textbf{normed linear space} is a linear space together with a norm.
	If $X$ is a linear space normed by $\|\cdot\|$ we say that a function $f$ in $X$ is a \textbf{unit function} provided $\|f\|=1$.
	For any $f\in X,f\neq 0$, the function $\frac{f}{\|f\|}$ is a unit function: it is a scalar multiple of $f$ which we call the \textbf{normalization} of $f$.

	\begin{namedthm*}{Example}[The Normed Linear Space $L^1(E)$]
		For a function $f$ in $L^1(E)$, define
		\[
		\|f\|_1=\int_E|f|.	
		\]
		Then $\|\cdot\|$ is a norm on $L^1(E)$.\\

	\end{namedthm*}

	\begin{namedthm*}{Example}[The Normed Linear Space $L^\infty(E)$]
		
	\end{namedthm*}

\end{flushleft}
\begin{center}
	\textbf{PROBLEMS}
\end{center}
\begin{enumerate}
	\setcounter{enumi}{0}
	\item For $f$ in $C[a,b]$, Define
	\[
	\| f \|_1 = \int_a^b |f|.	
	\]
	Show that this is a norm on $C[a,b]$.\\\\
	Let $f,g\in C[a,b]$. For each $x\in [a.b]$, we have the inequality $|f(x)+g(x)|\le|f(x)|+|g(x)|$, so by monotonicity and linearity of integration,
	\[
	\|f+g\|_1=\int_a^b|f(x)+g(x)| \le \int_a^b[|f(x)|+|g(x)|] = \int_a^b|f(x)| +\int_a^b|g(x)| = \|f\|_1+\|g\|_1.	
	\]
	Therefore subadditivity holds.\\
	Also, by linearity of integration, we have
	\[
	\|\alpha f\|_1 = \int_a^b|\alpha f| = \int_a^b|\alpha| |f|=|\alpha|\int_a^b| f| = |\alpha|\|f\|_1.	
	\]
	Therefore absolute homogeneity holds.\\
	Finally, by definition of absolute value, $0 \le |f(x)|$ for all $x\in [a,b]$, and by monotonicity of integration,
	\[
	0=\int_a^b 0 \le \int_a^b |f| = \|f\|_1.	
	\] 
	Clearly $\int_a^b |f| = 0$ iff $f\equiv 0$ on $[a,b]$.
	Therefore positive definiteness holds.\\
	Thus $\|\cdot\|_1$ is a norm on $C[a,b]$.\\
	Also show that there is no number $c \ge 0$ for which
	\[
	\| f \|_{\max}	\le c \| f \|_1 \text{ for all $f$ in $C[a,b]$},
	\]
	\\
	Consider the interval $[a,b]=[0,1]$.
	For any $c>0$ we choose, there exists an $n\in \mathbb{N}$ such that $n> c$, with the continuous function $f_n:[0,1]\to \mathbb{R}$ defined as
	\[ 
		f_n(x) =
		\begin{cases} 
			\frac{n-0}{1/n-0}x& \text{ if } x \in [0,\frac{1}{n}]\\
			\frac{0-n}{2/n-1/n}(x-\frac{1}{n})+n & \text{ if } x \in (\frac{1}{n},\frac{2}{n}]\\
			0& \text{ if } x \in (\frac{2}{n},1]
		\end{cases}
		=
		\begin{cases} 
			n^2x& \text{ if } x \in [0,\frac{1}{n}]\\
			-n^2(x-\frac{1}{n})+n & \text{ if } x \in (\frac{1}{n},\frac{2}{n}]\\
			0& \text{ if } x \in (\frac{2}{n},1]
		\end{cases}
	\]
	(This is a triangle-shaped function that reaches its peak $n$ at $x=\frac{1}{n}$.)\\
	Now, for any $n$, we have $\|f_n\|_1 = \int_0^1|f_n|=1$, and $\|f_n\|_{\max} = n$.\\
	Then $\| f \|_{\max}=n > c =  c \| f \|_1$.\\
	\\
	but there is a $c \ge 0$ for which 
	\[
	\| f \|_1	\le c \| f \|_{\max} \text{ for all $f$ in $C[a,b]$}.
	\]
	\\
	We can see that for any $f$ in $C[a,b]$, by monotonicity of the integral, 
	\begin{align*}
	\|f\|_1 &= \int_a^b|f(x)|\\ 
	&\le \int_a^b\max_{x\in [a,b]}|f(x)|\\
	&=\max_{x\in [a,b]}|f(x)| \int_a^b1\\
	&= \max_{x\in [a,b]}|f(x)| \cdot m([a,b]) \\
	&= \|f\|_{\max} \cdot m([a,b]).
	\end{align*}
	Therefore $\|f\|_1\le m([a,b])\|f\|_{\max}$ for all $f\in C[a,b]$.
	\item Let $X$ be the family of all polynomials with real coefficients defined on $\mathbb{R}$.
	Show that this is a linear space. For a polynomial $p$, define $\| p\|$ to be the sum of the absolute values of the coefficients of $p$.
	Is this a norm?\\
	For any two polynomials $p,q\in X$, there exists natural numbers $n,m$ (suppose without loss of generality that $n\le m$) such that
	\begin{align*}
	p(x) &= a_0+a_1x+a_2x^2+\cdots+a_{n-1}x^{n-1}+a_nx^n+\cdots+0x^m	\\
	q(x) &= b_0+b_1x+b_2x^2+\cdots+b_{n-1}x^{n-1}+b_nx^n+\cdots+b_mx^m	
	\end{align*}
	Now, considering any scalars $\alpha,\beta \in \mathbb{R}$, we have
	\begin{align*}
		\alpha p(x) + \beta q(x) &= \alpha (a_0+a_1x+a_2x^2+\cdots+a_{n-1}x^{n-1}+a_nx^n)\\
		&+ \beta (b_0+b_1x+b_2x^2+\cdots+b_{m-1}x^{m-1}+b_mx^m)\\
		&=(\alpha a_0)+(\alpha a_1)x+(\alpha a_2)x^2+\cdots+(\alpha a_{n-1})x^{n-1}+(\alpha a_n)x^n\\
		&+ (\beta b_0)+(\beta b_1)x+(\beta b_2)x^2+\cdots+(\beta b_{n-1})x^{n-1}+(\beta b_n)x^n+\cdots+(\beta b_m)x^m\\
		&=(\alpha a_0+\beta b_0)+(\alpha a_1+\beta b_1)x+\cdots+(\alpha a_n+\beta b_n)x^n+\cdots+(\beta b_m)x^m
	\end{align*}
	This is also a polynomial, as for each $i$, we have $(\alpha a_i+\beta b_i) \in \mathbb{R}$, so $X$ is a linear space.\\
	Now, for any polynomial
	\[
		p(x) = a_0+a_1x+a_2x^2+\cdots+a_nx^n,	
	\]
	we can define $\|p\| = |a_0|+|a_1|+|a_2|+\cdots+|a_n| = \sum_{i=0}^n|a_i|$.\\
	The triangle inequality is clear because 
	\[
		\|p+q\| = \sum_{i=0}^{m}|a_i+b_i|\le\sum_{i=0}^{m}[|a_i|+|b_i|]=\sum_{i=0}^{m}|a_i|+\sum_{i=0}^{m}|b_i|=\|p\|+\|q\|.
	\]
	Absolute homogeneity is clear because
	\[
		\|\alpha p\| = \sum_{i=0}^n|\alpha a_i|= \sum_{i=0}^n|\alpha|| a_i|=|\alpha|\sum_{i=0}^n| a_i|=|\alpha|\|p\|.
	\]
	Finally, positive definiteness is clear because
	\[
		0 \le |a_i| \implies 0\le \sum_{i=0}^n| a_i|=\|p\|,
	\]
	And $\|p\|=0$ if and only if $p(x)= 0+0x+0x^2+\cdots0x^n=0$.
	\item For $f$ in $L^1[a,b]$, define $\|f\| = \smallint_a^b x^2 |f(x)|dx$.
	Show that this is a norm on $L^1[a,b]$.
	\item For $f$ in $L^\infty[a,b]$, show that 
	\[
	\| f\|_\infty = \min \biggl \{ M \ \biggl |\ m \{x \in [a,b]\ |\ |f(x)| > M \} =0 \biggr \},
	\] 
	and if, furthermore, $f$ is continuous on $[a,b]$, that
	\[
	\| f \|_{\infty} = \| f \|_{\max}.	
	\]
	\item Show that $\ell^\infty$ and $\ell^1$ are normed linear spaces.
\end{enumerate}

% 7.2
\section{The Inequalities of Young, H\"older, and Minkowski}

% 7.3
\section{$L^p$ is Complete: The Riesz-Fischer Theorem}

% 7.4
\section{Approximation and Separability}
