% Chapter 7
\authoredby{finished}
\chapter{The $L^p$ Spaces: Completeness and Approximation}

% 7.1
%\authoredby{finished}
\section{Normed Linear Spaces}
\begin{flushleft}
	
	Throughout this chapter $E$ denotes a measurable set of real numbers.
	Define $\mathcal{F}$ to be the collection of all measurable extended real-valued functions on $E$ that are finite a.e. on $E$.
	We can say that two functions $f,g\in\mathcal{F}$ are equivalent, denoted by $f\cong g$, provided
	\[
	f(x)=g(x)\text{ for almost all }x\in E.	
	\]
	This is an equivalence relation and induces a partition of $\mathcal{F}$ into a disjoint collection of equivalence classes, denoted by $\mathcal{F}/\cong$, which is a linear space.
	There is a natural family $\{L^p(E)\}_{1\le p\le\infty}$ of subspaces of $\mathcal{F}/\cong$.
	\\
	For $1\le p<\infty$, define $L^p(E)$ to be the collection of equivalence class $[f]$ for which 
	\[
	\int_E|f|^p<\infty.	
	\]
	Then if $f\cong g$, then $\int_E|f|^p=\int_E|g|^p$.
	Showing that $L^p(E)$ is closed under linear combinations will prove that $L^p(E)$ is a linear subspace.
	To do this, let $c=\max\{|a|,|b|\}$ so that
	\[
	|a+b|\le|a|+|b|\le 2c,
	\]
	which implies
	\[
	|a+b|^p\le 2^pc^p\le2^p(|a|^p+|b|^p).	
	\]
	This inequality, together with the linearity and monotonicity of integration tells us that 
	\[
		\int_E|\alpha f+\beta g|^p \le 2^p(|\alpha|^p\int_E|f|^p+|\beta|^p\int_E| g|^p)<\infty.
	\]
	That is, for $[f],[g]\in L^p(E)$, then $\alpha[f]+\beta[g]\in L^p(E)$.\\
	We call a function $f\in\mathcal{F}$ \textbf{essentially bounded} provided there is some $M\ge 0$, called an \textbf{essential upper bound} for $f$, for which
	\[
	|f(x)|\le M\text{ for almost all }x\in E.	
	\]
	Then we can define $L^\infty(E)$ to be the collection of equivalence classes $[f]$ for which $f$ is essentially bounded. 
	Clearly $L^\infty(E)$ is a linear subspace because
	\[
	|\alpha f(x)+\beta g(x)|\le|\alpha| |f(x)|+|\beta|| g(x)|\le |\alpha| M + |\beta|M' = M'' \text{ a.e. on $E$}
	\]
	To state that a function $f$ in $L^p[a,b]$ is continuous means that there is a continuous function that agrees with $f$ a.e. on $[a,b]$.
	There is only one such continuous function and it is often convenient to consider this unique continuous function as the representative of $[f]$.\\
	It is useful to consider real-valued functions that have as their domain linear spaces of functions: such functions are called \textbf{functionals}.

	\begin{namedthm*}{Definition}
		Let $X$ be a linear space.
		A real-valued functional $\|\cdot\|$  on $X$ is called a \textbf{norm} provided for each $f$ and $g$ in $X$ and each real number $\alpha$,\\
		(The Triangle Inequality)
		\[
			\|f+g\|\le\|f\|+\|g\|,
		\]
		(Positive Homogeneity)
		\[
			\|\alpha f\|=|\alpha|\|f\|,	
		\]
		(Nonnegativity)
		\[
			\|f\|\ge 0\text{ and }\|f\|=0 \iff f=0.
		\]
	\end{namedthm*}
	A \textbf{normed linear space} is a linear space together with a norm.
	If $X$ is a linear space normed by $\|\cdot\|$ we say that a function $f$ in $X$ is a \textbf{unit function} provided $\|f\|=1$.
	For any $f\in X,f\neq 0$, the function $\frac{f}{\|f\|}$ is a unit function: it is a scalar multiple of $f$ which we call the \textbf{normalization} of $f$.

	\begin{namedthm*}{Example}[The Normed Linear Space $L^1(E)$]
		For a function $f$ in $L^1(E)$, define
		\[
		\|f\|_1=\int_E|f|.	
		\]
		Then $\|\cdot\|$ is a norm on $L^1(E)$.\\
		For $f,g\in L^1(E)\subseteq \mathcal{F}$, since $f$ and $g$ are finite a.e. on $E$, the triangle inequality for real numbers tells us that
		\[
		|f+g|\le|f|+|g|\text{ a.e. on }E.	
		\]
		Then by the monotonicity and linearity of integration, we have subadditivity:
		\[
			\|f+g\|_1=\int_E|f+g| \le \int_E[|f|+|g|] = \int_E|f| +\int_E|g| = \|f\|_1+\|g\|_1.	
		\]
		By the linearity of integration, clearly we have absolute homogeneity:
		\[
			\|\alpha f\|_1 = \int_E|\alpha f| = \int_E|\alpha| |f|=|\alpha|\int_E| f| = |\alpha|\|f\|_1.	
		\]
		Clearly $\|f\|$ is nonnegative. Finally, if $f\in L^1(E)$ and $\|f\|_1=0$, then $f=0$ a.e. on $E$. 
		Therefore $[f]$ is the zero element of the linear space $L^1(E)\subseteq \mathcal{F}/\cong$, that is $f=0$.
	\end{namedthm*}

	\begin{namedthm*}{Example}[The Normed Linear Space $L^\infty(E)$]
		For a function $f$ in $L^\infty(E)$, define $\|f\|_\infty$ to be the infimum of the essential upper bounds for $f$.
		\[
			\|f\|_\infty = \inf\{M\ :\ |f(x)|\le M\text{ a.e. on }E\}.
		\]
		We call $\|f\|_\infty$ the \textbf{essential supremum} of $f$ and claim that $\|\cdot\|_\infty$ is a norm on $L^\infty(E)$.
		\\Nonnegativity and positive homogeneity are clear. 
		\\To show that the triangle inequality holds, we see that for each natural number $n$, there is a subset $E_n$ of $E$ for which
		\[
		|f|\le \|f\|_\infty + \frac{1}{n}\text{ on }E\setminus E_n\text{ and }m(E_n)=0.	
		\]
		This is true because $\|f\|_\infty$ is the infimum, the greatest lower bound, so $\|f\|_\infty + \frac{1}{n}$ is not a lower bound and thus there exists a real number $M$ in the set of upper bounds a.e. of $f$ for which $\|f\|_\infty\le M < \|f\|_\infty + \frac{1}{n}$ a.e. on $E$, and so $|f|\le M < \|f\|_\infty + \frac{1}{n}$ a.e. on $E$.
		\\Accepting that the union of sets of measure zero is also measure zero, we can let $E_\infty = \bigcup_{n=1}^\infty E_n$, and so 
		\[
		|f|\le \|f\|_\infty \text{ on }E\setminus E_\infty\text{ and }m(E_n\infty)=0.	
		\]
		Thus we have that $|f|\le \|f\|_\infty$ a.e. on $E$; i.e., ess. sup$f$ is the smallest essential upper bound for $f$.
		\\Now, for $f,g\in L^\infty(E)$, 
		\[
			|f+g|\le|f|+|g|\le\|f\|_\infty+\|g\|_\infty\text{ a.e. on }E.
		\]
		Therefore $\|f\|_\infty+\|g\|_\infty$ is an essential bound for $f+g$ and thus the smallest essential upper bound, $\|f+g\|_\infty$, is such that
		\[
		\|f+g\|_\infty \le \|f\|_\infty+\|g\|_\infty.	
		\]
	\end{namedthm*}

	\begin{namedthm*}{Example}[The Normed Linear Spaces $\ell^1$ and $\ell^\infty$]
		For $1\le p<\infty$, define $\ell^p$ to be the collection of real sequences $a=(a_1,a_2,\cdots)$ for which 
		\[
		\sum_{k=1}^\infty|a_k|^p<\infty .	
		\]
		Let $a,b \in \ell^p$, and let $\alpha , \beta$ be real numbers.
		Then we have that $\sum_{k=1}^\infty|a_k|^p<\infty$ and $\sum_{k=1}^\infty|b_k|^p<\infty$.
		Using the inequality $|a+b|^p\le2^p(|a|^p+|b|^p)$, we have
		\begin{align*}
			\sum_{k=1}^\infty|\alpha a_k + \beta b_k|^p &\le \sum_{k=1}^\infty[2^p(|\alpha a_k|^p + |\beta b_k|^p)]\\ 
			&=\sum_{k=1}^\infty2^p|\alpha|^p |a_k|^p+\sum_{k=1}^\infty2^p|\beta|^p| b_k|^p\\
			&=2^p|\alpha|^p\sum_{k=1}^\infty |a_k|^p+2^p|\beta|^p\sum_{k=1}^\infty| b_k|^p\\
			&< 2^p|\alpha|^p\infty+2^p|\beta|^p\infty\\
			&=\infty.
		\end{align*}
		Thus $\ell^p$ is a linear space.\\
		We define $\ell^\infty$ to be the linear space of real bounded sequences: that is, for any $\{a_k\}$ in $\ell^\infty$, there exists a real number $M$ for which $|a_k|\le M$ for all $k$.
		We can define the following norms:
		\\For $\{a_k\}\in\ell^1$:
		\[
			\|\{a_k\}\|_1 = \sum_{k=1}^\infty|a_k|
		\]
		For $\{a_k\}\in\ell^\infty$:
		\[
			\|\{a_k\}\|_\infty = \sup_{1\le k<\infty}|a_k|
		\]
	\end{namedthm*}

	\begin{namedthm*}{Example}[The Normed Linear Space $C\lbrack a,b\rbrack$]
		Let $[a,b]$ be a closed, bounded interval. 
		The the linear space of continuous real-valued functions on $[a,b]$ is denoted by $C[a,b]$. 
		Since a continuous function on a compact set takes on a maximum value (ch1 problem 52), we can define
		\[
		\|f\|_{\max}=\max_{x\in [a,b]}|f(x)|.	
		\]		
	\end{namedthm*}

\end{flushleft}
\begin{center}
	\textbf{PROBLEMS}
\end{center}
\begin{enumerate}
	\setcounter{enumi}{0}
	\item For $f$ in $C[a,b]$, Define
	\[
	\| f \|_1 = \int_a^b |f|.	
	\]
	Show that this is a norm on $C[a,b]$.
	Also show that there is no number $c \ge 0$ for which
	\[
	\| f \|_{\max}	\le c \| f \|_1 \text{ for all $f$ in $C[a,b]$},
	\]
	but there is a $c \ge 0$ for which 
	\[
	\| f \|_1	\le c \| f \|_{\max} \text{ for all $f$ in $C[a,b]$}.
	\]
	\\
	Let $f,g\in C[a,b]$. For each $x\in [a.b]$, we have the inequality $|f(x)+g(x)|\le|f(x)|+|g(x)|$, so by monotonicity and linearity of integration,
	\[
	\|f+g\|_1=\int_a^b|f(x)+g(x)| \le \int_a^b[|f(x)|+|g(x)|] = \int_a^b|f(x)| +\int_a^b|g(x)| = \|f\|_1+\|g\|_1.	
	\]
	Therefore subadditivity holds.\\
	Also, by linearity of integration, we have
	\[
	\|\alpha f\|_1 = \int_a^b|\alpha f| = \int_a^b|\alpha| |f|=|\alpha|\int_a^b| f| = |\alpha|\|f\|_1.	
	\]
	Therefore absolute homogeneity holds.\\
	Finally, by definition of absolute value, $0 \le |f(x)|$ for all $x\in [a,b]$, and by monotonicity of integration,
	\[
	0=\int_a^b 0 \le \int_a^b |f| = \|f\|_1.	
	\] 
	Clearly $\int_a^b |f| = 0$ iff $f\equiv 0$ on $[a,b]$.
	Therefore positive definiteness holds.\\
	Thus $\|\cdot\|_1$ is a norm on $C[a,b]$.\\
	\\Consider the interval $[a,b]=[0,1]$.
	For any $c>0$ we choose, there exists an $n\in \mathbb{N}$ such that $n> c$, with the continuous function $f_n:[0,1]\to \mathbb{R}$ defined as
	\[ 
		f_n(x) =
		\begin{cases} 
			\frac{n-0}{1/n-0}x& \text{ if } x \in [0,\frac{1}{n}]\\
			\frac{0-n}{2/n-1/n}(x-\frac{1}{n})+n & \text{ if } x \in (\frac{1}{n},\frac{2}{n}]\\
			0& \text{ if } x \in (\frac{2}{n},1]
		\end{cases}
		=
		\begin{cases} 
			n^2x& \text{ if } x \in [0,\frac{1}{n}]\\
			-n^2(x-\frac{1}{n})+n & \text{ if } x \in (\frac{1}{n},\frac{2}{n}]\\
			0& \text{ if } x \in (\frac{2}{n},1]
		\end{cases}
	\]
	(This is a triangle-shaped function that reaches its peak $n$ at $x=\frac{1}{n}$.)\\
	Now, for any $n$, we have $\|f_n\|_1 = \int_0^1|f_n|=1$, and $\|f_n\|_{\max} = n$.\\
	Then $\| f \|_{\max}=n > c =  c \| f \|_1$.\\
	\\
	Finally, we can see that for any $f$ in $C[a,b]$, by monotonicity of the integral, 
	\begin{align*}
	\|f\|_1 &= \int_a^b|f(x)|\\ 
	&\le \int_a^b\max_{x\in [a,b]}|f(x)|\\
	&=\max_{x\in [a,b]}|f(x)| \int_a^b1\\
	&= \max_{x\in [a,b]}|f(x)| \cdot m([a,b]) \\
	&= \|f\|_{\max} \cdot m([a,b]).
	\end{align*}
	Therefore $\|f\|_1\le m([a,b])\|f\|_{\max}$ for all $f\in C[a,b]$.
	\item Let $X$ be the family of all polynomials with real coefficients defined on $\mathbb{R}$.
	Show that this is a linear space. For a polynomial $p$, define $\| p\|$ to be the sum of the absolute values of the coefficients of $p$.
	Is this a norm?\\
	For any two polynomials $p,q\in X$, there exists natural numbers $n,m$ (suppose without loss of generality that $n\le m$) such that
	\begin{align*}
	p(x) &= a_0+a_1x+a_2x^2+\cdots+a_{n-1}x^{n-1}+a_nx^n+\cdots+0x^m	\\
	q(x) &= b_0+b_1x+b_2x^2+\cdots+b_{n-1}x^{n-1}+b_nx^n+\cdots+b_mx^m	
	\end{align*}
	Now, considering any scalars $\alpha,\beta \in \mathbb{R}$, we have
	\begin{align*}
		\alpha p(x) + \beta q(x) &= \alpha (a_0+a_1x+a_2x^2+\cdots+a_{n-1}x^{n-1}+a_nx^n)\\
		&+ \beta (b_0+b_1x+b_2x^2+\cdots+b_{m-1}x^{m-1}+b_mx^m)\\
		&=(\alpha a_0)+(\alpha a_1)x+(\alpha a_2)x^2+\cdots+(\alpha a_{n-1})x^{n-1}+(\alpha a_n)x^n\\
		&+ (\beta b_0)+(\beta b_1)x+(\beta b_2)x^2+\cdots+(\beta b_{n-1})x^{n-1}+(\beta b_n)x^n+\cdots+(\beta b_m)x^m\\
		&=(\alpha a_0+\beta b_0)+(\alpha a_1+\beta b_1)x+\cdots+(\alpha a_n+\beta b_n)x^n+\cdots+(\beta b_m)x^m
	\end{align*}
	This is also a polynomial, as for each $i$, we have $(\alpha a_i+\beta b_i) \in \mathbb{R}$, so $X$ is a linear space.\\
	Now, for any polynomial
	\[
		p(x) = a_0+a_1x+a_2x^2+\cdots+a_nx^n,	
	\]
	we can define $\|p\| = |a_0|+|a_1|+|a_2|+\cdots+|a_n| = \sum_{i=0}^n|a_i|$.\\
	The triangle inequality is clear because 
	\[
		\|p+q\| = \sum_{i=0}^{m}|a_i+b_i|\le\sum_{i=0}^{m}[|a_i|+|b_i|]=\sum_{i=0}^{m}|a_i|+\sum_{i=0}^{m}|b_i|=\|p\|+\|q\|.
	\]
	Absolute homogeneity is clear because
	\[
		\|\alpha p\| = \sum_{i=0}^n|\alpha a_i|= \sum_{i=0}^n|\alpha|| a_i|=|\alpha|\sum_{i=0}^n| a_i|=|\alpha|\|p\|.
	\]
	Finally, positive definiteness is clear because
	\[
		0 \le |a_i| \implies 0\le \sum_{i=0}^n| a_i|=\|p\|,
	\]
	And $\|p\|=0$ if and only if $p(x)= 0+0x+0x^2+\cdots0x^n=0$.
	\item For $f$ in $L^1[a,b]$, define $\|f\| = \smallint_a^b x^2 |f(x)|dx$.
	Show that this is a norm on $L^1[a,b]$.\\
	For $f\in L^1[a,b]$, then $f$ is measurable and finite a.e. on $[a,b]$, and $\int_a^b|f(x)|dx<\infty$.\\
	Let $f,g\in L^1[a,b]$, and let $\alpha$ be a real number.\\
	Because the triangle inequality holds a.e. on $[a,b]$, by monotonicity and linearity of the integral, we have
	\begin{align*}
	\|f+g\|&=\int_a^b x^2 |f(x)+g(x)|dx\\
	&\le\int_a^b x^2 [|f(x)|+|g(x)|]dx\\
	&=\int_a^b [x^2 |f(x)|+x^2|g(x)|]dx\\
	&=\int_a^b x^2 |f(x)|dx+\int_a^b x^2 |g(x)|dx\\
	&= \|f\|+\|g\|.	
	\end{align*}
	Therefore $\|\cdot\|$ is subadditive.\\
	By linearity of the integral, we have
	\[
	\|\alpha f\| = \int_a^bx^2|\alpha f(x)|dx = \int_a^bx^2|\alpha| |f(x)|dx=|\alpha |\int_a^bx^2 |f(x)|dx	=|\alpha |\|f\|.
	\]
	Therefore $\|\cdot\|$ satisfies absolute homogeneity.\\
	We can use the fact that $0\le x^2$ and $0\le |f(x)|$ implies $0\le x^2|f(x)|$.
	By monotonicity of the integral, we have
	\[
	0=\int_a^b0dx \le \int_a^bx^2|f(x)|dx = \|f\|.
	\]
	Clearly $\|f\|=0$ if and only if $f=0$ a.e. on $[a,b]$ because $x^2\cdot 0 = 0$.\\
	Therefore $\|\cdot \|$ satisfies positive definiteness.
	\item For $f$ in $L^\infty[a,b]$, show that 
	\[
	\| f\|_\infty = \min \biggl \{ M \ \biggl |\ m \{x \in [a,b]\ |\ |f(x)| > M \} =0 \biggr \},
	\] 
	\\
	That is, the sup norm is the smallest real number $M$ such that $|f(x)|>M$ only on a set of measure zero.
	In an above example, we showed that $\|f\|_\infty$ is the smallest essential upper bound for $f$.
	That is, $|f|\le\|f\|_\infty$ a.e. on $E$ (That is, the inequality is true for $E\setminus E_0$, where $m(E_0)=0$.)
	\\
	and if, furthermore, $f$ is continuous on $[a,b]$, that
	\[
	\| f \|_{\infty} = \| f \|_{\max}.	
	\]
	If $f$ is continuous, then there are no jump discontinuities ($f$ is continuous at $x_0$ iff $f(x_0^-)=f(x_0)=f(x_0^+)$).
	Then $|f|\le\|f\|_\infty$ everywhere on $E$.
	\item Show that $\ell^\infty$ and $\ell^1$ are normed linear spaces.\\
	$\ell^\infty$:\\
	Let $a,b \in \ell^\infty$, and let $\alpha , \beta$ be real numbers.\\
	Then for some real numbers $M,N$, we have that $|a_k|\le M$ and $|b_k|\le N$ for all $k$.
	\begin{align*}
		\alpha a + \beta b &= \alpha (a_1,a_2,\cdots)+ \beta (b_1,b_2,\cdots)\\		
		&= (\alpha a_1,\alpha a_2,\cdots)+ (\beta b_1,\beta b_2,\cdots)\\
		&= (\alpha a_1+\beta b_1,\alpha a_2+\beta b_2,\cdots)
	\end{align*}
	Then $|\alpha a_k+\beta b_k|\le \alpha M + \beta N$ for all $k$, and $\ell^\infty$ is a linear space.\\
	To show that $\|a\|_\infty = \sup_{1\le k<\infty}|a_k|$ is a norm:
	\[
		\|a+b\|_\infty = \sup_{1\le k<\infty}|a_k+b_k|\le \sup_{1\le i<\infty}|a_i| + \sup_{1\le j<\infty}|b_j| = \|a\|_\infty + \|b\|_\infty,
	\]
	\[
		\|\alpha a\|_\infty = \sup_{1\le k<\infty}|\alpha a_k| = \sup_{1\le k<\infty}|\alpha|| a_k|= |\alpha|\sup_{1\le k<\infty}| a_k|=|\alpha|\|a\|_\infty, 
	\]
	\[
		0 \le \sup_{1\le k<\infty}| a_k| = \|a\|_\infty,\text{ and }\sup_{1\le k<\infty}| a_k|=0\text{ iff }a_k=0\text{ for all }k.	
	\]
	\\
	$\ell^1$:\\
	Let $a,b \in \ell^1$, and let $\alpha , \beta$ be real numbers.\\
	Then we have that $\sum_{k=1}^\infty|a_k|<\infty$ and $\sum_{k=1}^\infty|b_k|<\infty$.\\
	By the triangle inequality for real numbers, we have
	\[
		\sum_{k=1}^\infty |\alpha a_k + \beta b_k| \le\sum_{k=1}^\infty[ |\alpha|| a_k| + | \beta ||b_k|]= |\alpha|\sum_{k=1}^\infty | a_k| + |\beta |\sum_{k=1}^\infty |b_k| <|\alpha|\infty+|\beta|\infty = \infty.
	\]
	Therefore $\ell^1$ is a linear space.\\
	To show that $\|a\|_1 = \sum_{k=1}^\infty|a_k|$ is a norm:
	\[
		\|a+b\|_1 = \sum_{k=1}^\infty|a_k+b_k|\le\sum_{k=1}^\infty[|a_k|+|b_k|]=\sum_{k=1}^\infty|a_k|+\sum_{k=1}^\infty|b_k|<\infty +\infty = \infty,
	\]
	\[
		\|\alpha a\|_1 = \sum_{k=1}^\infty|\alpha a_k| = \sum_{k=1}^\infty|\alpha|| a_k|=|\alpha|\sum_{k=1}^\infty| a_k|=|\alpha|\|a\|_1, 
	\]
	\[
		0 \le | a_k| \implies 0 \le \sum_{k=1}^\infty |a_k| = \|a\|_1,\text{ and }\sum_{k=1}^\infty| a_k|=0\text{ iff }a_k=0\text{ for all }k.	
	\]
\end{enumerate}

% 7.2
\authoredby{inprogress}
\section{The Inequalities of Young, H\"older, and Minkowski}

\begin{center}
	\textbf{PROBLEMS}
\end{center}
\begin{enumerate}
	\setcounter{enumi}{5}
	\item Show that if H\"older's Inequality is true for normalized functions it is true in general.
	\item Verify the assertions in the above two examples regarding the membership of the function $f$ in $L^p(E)$. 
	\item Let $f$ and $g$ belong to $L^2(E)$. From the linearity of integration show that for any number $\lambda$,
	\[
		\lambda^2\int_Ef^2+2\lambda\int_Ef\cdot g+\int_Eg^2=\int_E(\lambda f+g)^2\ge0.	
	\] 
	From this and the quadratic formula directly derive the Cauchy-Schwarz Inequality.
	\item Show that in Young's Inequality there is equality iff $a^p=b^q$.
	\item Show that in H\"older's Inequality there is equality iff there are constants $\alpha,\beta$ not both zero, for which
	\[
		\alpha|f|^p=\beta|g|^q\text{ a.e. on }E.	
	\]
	For a point $x=(x_1,x_2,\cdots,x_n)$ in $\mathbb{R}^n$, define $T_x$ to be the step function on the interval ...
\end{enumerate}

% 7.3
\authoredby{inprogress}
\section{$L^p$ is Complete: The Riesz-Fischer Theorem}

\begin{center}
	\textbf{PROBLEMS}
\end{center}
\begin{enumerate}
	\setcounter{enumi}{22}
	\item Provide an example of a Cauchy sequence of real numbers that is not rapidly Cauchy.
	\item Let $X$ be a normed linear space.
	Assume that $\{f_n\}\to f$ in $X$, $\{g_n\}\to g$ in $X$, and $\alpha$ and $\beta$ are real numbers.
	Show that
	\[
		\{\alpha f_n+\beta g_n\}\to\alpha f+\beta g\text{ in }X.
	\]
	\item Assume that $E$ has finite measure and $1\le p_1< p_2\le\infty$.
	Show that if $\{f_n\}\to f$ in $L^{p_2}(E)$, then $\{f_n\}\to f$ in $L^{p_1}(E)$.
	\item (The $L^p$ Dominated Convergence Theorem) Let $\{f_n\}$ be a sequence of measurable functions that converges pointwise a.e. on $E$ to $f$.
	For $1\le p<\infty$, suppose there is a function $g$ in $L^p(E)$ such that for all $n$, $|f_n|\le g$ a.e. on $E$.
	Prove that $\{f_n\}\to f$ in $L^p(E)$.
	\item For $E$ a measurable set and $1\le p<\infty$, assume $\{f_n\}\to f$ in $L^p(E)$.
	Show that there is a subsequence $\{f_{n_k}\}$ and a function $g\in L^p(E)$ for which $|f_{n_k}|\le g$ a.e. on $E$ for all $k$.
	\item Assume $E$ has finite measure and $1\le p<\infty$.
	Suppose $\{f_n\}$ is a sequence of measurable functions that converges pointwise a.e. on $E$ to $f$.
	For $1\le p<\infty$, show that $\{f_n\}\to f$ in $L^p(E)$ if there is a $\theta>0$ such that $\{f_n\}$ belongs to and is bounded as a subset of $L^{p+\theta}(E)$.
	\item Consider the linear space of polynomials on $[a,b]$ normed by $\|\cdot\|_{\max}$ norm.
	Is this normed linear space a Banach space?
	\item Let $\{f_n\}$ be a sequence in $C[a,b]$ and $\sum_{k=1}^\infty a_k$ a convergent series of positive numbers such that
	\[
		\|f_{k+1}-f_k\|_{\max}\le a_k\text{ for all }k.
	\]
	Prove that
	\[
		|f_{n+k}(x)-f_n(x)|\le\|f_{n+k}-f_n\|_{\max}\le\sum_{j=n}^\infty a_j\text{ for all }k,n\text{ and all }x\in[a,b].
	\]
	Conclude that there is a function $f\in C[a,b]$ such that $\{f_n\}\to f$ uniformly on $[a,b]$.
	\item Use the preceding problem to show that $C[a,b]$, normed by the maximum norm, is a Banach space.\\
	\\Note: See Chapter 16 Problem 1 that $C[a,b]$ normed by the $L^2[a,b]$ norm is not a Banach space.
	In particular, we had a sequence of continuous functions in $[a,b]$ that was Cauchy (w.r.t. the $L^2$ norm ) but converged ($L^2$) to a discontinuous function in $[a,b]$.
	We again consider this sequence of functions $\{f_n\}$ and now show that it is no longer Cauchy (w.r.t. the maximum norm) and thus does not converge.
	Recall the definition of each function:
	\[
        f_n(x):=
        \begin{cases}
            0 &x\le t\\
            n(x-t)&t<x<t+\frac{1}{n}\\
            1 &x\ge t+\frac{1}{n}
        \end{cases}
    \]
	Fix $\epsilon=\frac{1}{2}$.
	For any $N\in\mathbb{N}$, consider $n\ge N$ and $m=3n>n\ge N$. 
	\\Then we have
	\[
        (f_m-f_n)(x)=
        \begin{cases}
            0-0 &x\le t\\
            m(x-t)-n(x-t)&x\in(t,t+\frac{1}{m})\\
            1-n(x-t)&x\in[t+\frac{1}{m},t+\frac{1}{n})\\
            1-1 &x\ge t+\frac{1}{n}
        \end{cases}
	\]
	then we can clearly see that the maximum occurs at $t+\frac{1}{m}$ so that 
	\[
		\|f_m-f_n\|_{\max}=\max_{x\in[a,b]}|(f_m-f_n)(x)|=1-n((t+\frac{1}{m})-t)=1-\frac{n}{3n}=\frac{2}{3}>\frac{1}{2},
	\]
	and the sequence is not Cauchy.\\
	\\Back to the proof of Problem 31:
	\\Consider the linear space $C[a,b]$ with the maximum norm. 
	Suppose that $\{f_n\}$ is a Cauchy sequence of functions in this space.
	By Proposition 5, there exists a rapidly Cauchy subsequence $\{f_{n_k}\}$.
	That is, there is a convergent series of positive numbers $\sum_{k=1}^\infty\epsilon_k$ for which
	\[
		\|f_{n_{k+1}}-f_{n_k}\|_{\max}\le\epsilon_k^2\text{ for all }k.
	\]
	For $\varepsilon=1>0$, there exists $N\in\mathbb{N}$ such that $\sum_{k=p}^\infty\epsilon_k<1$ for all $p\ge N$.
	Then because $0<\epsilon_k<1$ for each $n$, then $\epsilon_k^2<\epsilon_k$ so that  $\sum_{k=p}^\infty\epsilon_k^2<\sum_{k=p}^\infty\epsilon_k<\infty$, which implies that $\sum_{k=1}^\infty\epsilon_k^2$ is a convergent series of positive numbers.
	\\Therefore by the previous Problem 30, there is a function $f\in C[a,b]$ such that $\{f_{n_k}\}\to f$ uniformly on $[a,b]$.
	Then by Proposition 4, $\{f_n\}$ converges because it has a convergent subsequence $\{f_{n_k}\}$. 
	\item Let $\{f_n\}$ be a sequence in $L^\infty(E)$ and $\sum_{k=1}^\infty a_k$ a convergent series of positive numbers such that
	\[
		\|f_{k+1}-f_k\|_{\infty}\le a_k\text{ for all }k.
	\]
	Prove that there is a subset $E_0$ of $E$ which has measure zero and 
	\[
		|f_{n+k}(x)-f_n(x)|\le\|f_{n+k}-f_n\|_{\infty}\le\sum_{j=n}^\infty a_j\text{ for all }k,n\text{ and all }x\in E\setminus E_0.
	\]
	Conclude that there is a function $f\in L^\infty(E)$ such that $\{f_n\}\to f$ uniformly on $E\setminus E_0$.
	\item Use the preceding problem to show that $L^\infty(E)$ is a Banach space.\\
	\\Consider the linear space $L^\infty(E)$ with the supremum norm. 
	Suppose that $\{f_n\}$ is a Cauchy sequence of functions in this space.
	By Proposition 5, there exists a rapidly Cauchy subsequence $\{f_{n_k}\}$.
	That is, there is a convergent series of positive numbers $\sum_{k=1}^\infty\epsilon_k$ for which
	\[
		\|f_{n_{k+1}}-f_{n_k}\|_{\infty}\le\epsilon_k^2\text{ for all }k.
	\]
	Then $\sum_{k=1}^\infty\epsilon_k$ converges implies that $\sum_{k=1}^\infty\epsilon_k^2$ also converges.
	\\Therefore by the previous Problem 32, there is a function $f\in L^\infty(E)$ and a set $E_0\subseteq E$ of measure zero such that $\{f_{n_k}\}\to f$ uniformly on $E\setminus E_0$.
	Then by Proposition 4, $\{f_n\}$ converges because it has a convergent subsequence $\{f_{n_k}\}$. 
	\item Prove that for $1\le p\le\infty$, $\ell^p$ is a Banach space.
	\item Show that the space $c$ of all convergent sequences of real numbers and the space $c_0$ of all sequences that converge to zero are Banach spaces w.r.t. the $\ell^\infty$ norm. 
\end{enumerate}

% 7.4
\authoredby{untouched}
\section{Approximation and Separability}
