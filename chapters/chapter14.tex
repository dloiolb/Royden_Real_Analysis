% Chapter 14
\authoredby{inprogress}
\chapter{Duality for Normed Linear Spaces}

For a normed linear space $X$, we denoted the normed linear space of continuous linear real-valued functions of $X$ by $X^*$ and called it the \textbf{dual space} of $X$.
We aim to explore properties of the mapping from $X\times X^*$ to $\mathbb{R}$ defined by
\[
    (x,\psi)\mapsto\psi(x)\text{ for all }x\in X,\psi\in X^*.
\]

Hahn-Banach Theorem: extension of certain linear functionals on subspaces of an unnormed linear space to linear functionals on the whole space.
Hahn-Banach implies:
\begin{enumerate}
    \item for a normed linear space $X$, any bounded linear functional on a subspace of $X$ may be extended to a bounded linear functional on all of $X$, without increasing its norm
    \item for a locally convex Topological vector space $X$, any two disjoint closed convex sets of $X$ may be separated by a closed hyperplane
    \item for a reflexive Banach space $X$, any bounded sequence in $X$ has a weakly convergent subsequence.
\end{enumerate}

% 14.1
\authoredby{inprogress}
\section{Linear Functionals, Bounded Linear Functionals, and Weak Topologies}



\begin{center}
	\textbf{PROBLEMS}
\end{center}
\begin{enumerate}
	\setcounter{enumi}{0}
    \item hellos
\end{enumerate}

% 14.2
\authoredby{untouched}
\section{The Hahn-Banach Theorem}

% 14.3
\section{Reflexive Banach Spaces and Weak Sequential Convergence}

% 14.4
\section{Locally Convex Topological Vector Spaces}

% 14.5
\section{The Separation of Convex Sets and Mazur's Theorem}

% 14.6
\section{The Krein-Milman Theorem}
